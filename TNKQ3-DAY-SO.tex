% Câu 1
\begin{ex}
Cho cấp số cộng $(u_n)$ với $u_1=3$ và công sai $d=3$. Tính $u_3$ của cấp số cộng đã cho
\choice
{$6$}
{$7$}
{$8$}
{\True $9$}
\loigiai{
Ta có: $u_3=u_1+2d=3+2\cdot 3=9$.
}
\end{ex}
% Câu 2
\begin{ex}
Cấp số cộng $(u_n)$ có $u_1=1$ và $u_2=3$. Số hạng $u_5$ của cấp số cộng là
\choice
{$5$}
{$7$}
{\True $9$}
{$11$}
\loigiai{
Ta có $d=u_2-u_1=3-1=2$, nên $u_5=u_1+4d=1+4 \cdot 2=9$.
}
\end{ex}
% Câu 3
\begin{ex}
Cho cấp số cộng $(u_n)$ có $u_1=-2$ và công sai $d=3$. Tìm số hạng $u_{10}$.
\choice
{$u_{10}=-2 \cdot 3^9$}
{\True $u_{10}=25$}
{$u_{10}=-29$}
{$u_{10}=28$}
\loigiai{
Ta có $u_{10}=u_1+9d=25$.
}
\end{ex}
% Câu 4
\begin{ex}
Cho cấp số nhân $(u_n)$ với $u_1=6$ và $u_2=-12$. Công bội $q$ của cấp số nhân đã cho là
\choice
{$q=-\dfrac{1}{2}$}
{\True $q=-2$}
{$q=-18$}
{$q=-6$}
\loigiai{
Ta có: $q=\dfrac{u_2}{u_1}=\dfrac{-12}{6}=-2$.
}
\end{ex}
% Câu 5
\begin{ex}
Cho cấp số cộng $u_1=3$ có số hạng đầu, công sai $d=-4$. Số hạng thứ năm của cấp số cộng là
\choice
{$768$}
{\True $-13$}
{$-3072$}
{$-17$}
\loigiai{
$u_5=u_1+4d=3+4(-4)=-13$.
}
\end{ex}
% Câu 6
\begin{ex}
Cho cấp số cộng $(u_n)$, biết $u_1=1, d=2$. Giá trị của $u_{15}$ bằng
\choice
{$31$}
{\True $29$}
{$35$}
{$27$}
\loigiai{
Ta có $u_{15}=u_1+14d=1+14 \cdot 2=29$.
}
\end{ex}
% Câu 7
\begin{ex}
Cho cấp số cộng $u_1=3,u_2=6$. Xác định công sai $d$ của cấp số cộng
\choice
{$d=9$}
{$d=-3$}
{\True $d=2$}
{$d=3$}
\loigiai{
Ta có $d=u_2-u_1=6-3=3$.
}
\end{ex}
% Câu 8
\begin{ex}
Cấp số cộng có số hạng đầu $u_1=3$, công sai bằng $-5$. Tính $u_3$. 
\choice
{$75$}
{\True $-7$}
{$1$}
{$-45$}
\loigiai{
Ta có $u_3=u_1+2d=3+2\cdot (-5)=-7$.
}
\end{ex}
% Câu 9
\begin{ex}
Cho cấp số nhân $(u_n)$ có $u_1=2$ và công bội $q=3$. Số hạng $u_3$ của cấp số nhân đã cho là
\choice
{\True $18$}
{$5$}
{$6$}
{$8$}
\loigiai{
Ta có $u_3=u_1 \cdot q^2=2 \cdot 3^{2}=18$.
}
\end{ex}
% Câu 10
\begin{ex}
Cấp số nhân $(u_n)$ có $u_1=2$ và $u_2=-4$. Số hạng $u_6$ của cấp số nhân là
\choice
{\True $-64$}
{$128$}
{$-128$}
{$64$}
\loigiai{
Ta có $q=\dfrac{u_2}{u_1}=\dfrac{-4}{2}=-2$ suy ra $u_6=u_1 \cdot q^5=2 \cdot (-2)^{5}=-64$.
}
\end{ex}
% Câu 11
\begin{ex}
Cho $(u_n)$ với $u_n=2n-2$ thì $u_5$ bằng
\choice
{$10$}
{$9$}
{\True $8$}
{$7$}
\loigiai{
$u_5=2 \cdot 5-2=8$.
}
\end{ex}
% Câu 12
\begin{ex}
Cho cấp số nhân $(u_n)$ có $u_1=-2$ và $u_2=6$. Giá trị của $u_3$ bằng
\choice
{\True $-18$}
{$-12$}
{$18$}
{$12$}
\loigiai{
Ta có: $u_2=u_1 \cdot q\Rightarrow q=\dfrac{u_2}{u_1}=\dfrac{6}{-2}=-3$. Suy ra: $u_3=u_1 \cdot q^2=-18$.
}
\end{ex}
% Câu 13
\begin{ex}
Cho cấp số cộng $(u_n)$ biết $u_5=5, u_{10}=15$. Số hạng thứ bảy của cấp số cộng đã cho là
\choice
{$u_7=12$}
{$u_7=8$}
{$u_7=7$}
{\True $u_7=9$}
\loigiai{
Ta có: $u_{10}=u_5+5d \Rightarrow 15=5+5d \Rightarrow d=2$.\\
Vậy $u_7=u_5+2d=5+2\cdot 2=9$.
}
\end{ex}
% Câu 14
\begin{ex}
Cho cấp số nhân $(u_n)$ với $u_1=3$ và công bội $q=2$. Giá trị của $u_2$ bằng
\choice
{$8$}
{$9$}
{\True $6$}
{$\dfrac{3}{2}$}
\loigiai{
Ta có: $u_2=u_1 \cdot q=3 \cdot 2=6$.
}
\end{ex}
% Câu 15
\begin{ex}
Một cấp số nhân có số hạng đầu $u_1=3$, công bội $q=2$. Tìm $n$ biết $S_n=93$?
\choice
{$n=7$}
{$n=4$}
{$n=6$}
{\True $n=5$}
\loigiai{
$S_n=\dfrac{u_1\left(1-q^n\right)}{1-q}=\dfrac{3\left(1-2^n\right)}{1-2}=93 \Leftrightarrow 2^n=32 \Leftrightarrow n=5$.
}
\end{ex}
% Câu 16
\begin{ex}
Cho cấp số cộng $(u_n)$ có $u_1=-2$ và công sai $d=3$. Tìm số hạng $u_{10}$.
\choice
{$u_{10}=-2 \cdot 3^9$}
{\True $u_{10}=25$}
{$u_{10}=-29$}
{$u_{10}=28$}
\loigiai{
Ta có $u_{10}=u_1+9d=-2+9 \cdot 3=25$.
}
\end{ex}
% Câu 17
\begin{ex}
Cho cấp số nhân $(u_n)$ với $u_1=2$ và công bội $q=3$. Tìm số hạng thứ $4$ của cấp số nhân?
\choice
{\True $54$}
{$48$}
{$24$}
{$162$}
\loigiai{
Ta có: $u_4=u_1 \cdot q^3=2\cdot 3^3=54$.
}
\end{ex}
% Câu 18
\begin{ex}
Cho cấp số cộng $(u_n)$ có $u_1=5$, $u_{12}=38$ thì công sai là
\choice
{$d=1$}
{$d=2$}
{\True $d=3$}
{$d=4$}
\loigiai{
Ta có: $u_{12}=u_1+11d \Rightarrow 38=5+11d \Rightarrow d=\dfrac{38-5}{11}=3$.
}
\end{ex}
% Câu 19
\begin{ex}
Cho cấp số cộng $(u_n)$ với $u_1=-1$ và $u_2=4$. Giá trị $u_3$ bằng
\choice
{\True $9$}
{$-16$}
{$7$}
{$-8$}
\loigiai{
Ta có $u_2=u_1+d\Leftrightarrow 4=-1+d\Leftrightarrow d=5$.\\
Mà $u_3=u_1+2d=9$.\\
Vậy $u_3=9$.
}
\end{ex}
% Câu 20
\begin{ex}
Cấp số cộng $(u_n)$ có $u_1=2$, $u_2=5$. Tính $u_4$.
\choice
{$u_4=17$}
{$u_4=14$}
{\True $u_4=11$}
{$u_4=8$}
\loigiai{
$d=u_2-u_1=3$.\\
Vậy $u_4=u_1+3d=2+3\cdot 3=11$.
}
\end{ex}
% Câu 21
\begin{ex}
Cho một cấp số cộng $(u_n)$ có $u_1=1$ và tổng $100$ số hạng đầu bằng $10000$.
Tổng $S=\dfrac{1}{u_1u_2}+\dfrac{1}{u_2u_3}+ \ldots +\dfrac{1}{u_{99}u_{100}}$ bằng
\choice
{$\dfrac{200}{201}$}
{$\dfrac{198}{199}$}
{$\dfrac{100}{201}$}
{\True $\dfrac{99}{199}$}
\loigiai{
Ta có $S_{100}=10000 \Rightarrow 2+99d=200 \Rightarrow d=2$.\\
Khi đó $u_2=3,u_3=5,\ldots,u_{99}=197,u_{100}=199$.\\
$S=\dfrac{1}{1\cdot 3}+\dfrac{1}{3\cdot 5}+ \ldots +\dfrac{1}{197\cdot 199}=\dfrac{1}{2}\left(\dfrac{1}{1}-\dfrac{1}{3}+\dfrac{1}{3}-\dfrac{1}{5}+\cdots+\dfrac{1}{197}-\dfrac{1}{199}\right)=\dfrac{1}{2}\left(1-\dfrac{1}{199}\right)=\dfrac{99}{199}$.
}
\end{ex}
% Câu 22
\begin{ex}
Cho cấp số cộng $(u_n)$ có số hạng đầu $u_1=2$ và công sai $d=3$. Giá trị của $u_4$ bằng
\choice
{$54$}
{\True $11$}
{$12$}
{$8$}
\loigiai{
$u_4=u_1+3d=2+3 \cdot 3=11$.
}
\end{ex}
% Câu 23
\begin{ex}
Cho cấp số nhân $(u_n)$ có $u_1=2$ và $u_3=18$. Số hạng $u_5$ của cấp số nhân bằng
\choice
{$\dfrac{2}{81}$}
{$34$}
{$54$}
{\True $162$ }
\loigiai{
Ta có $u_3=u_1q^2\Rightarrow q^2=9\Rightarrow u_5=u_1q^4=2\cdot 9^2=162$.
}
\end{ex}
% Câu 24
\begin{ex}
Cho cấp số nhân $(u_n)$ với $u_1=-2$ và $q=-5$. Viết bốn số hạng đầu tiên của cấp số nhân $(u_n)$.
\choice
{$-2;-10;-50;-250$}
{\True $-2;10;-50;250$}
{$-2;10;50;250$}
{$-2;10;50;-250$}
\loigiai{
Ta có:
\begin{align*}
 & u_2=u_1 \cdot q=(-2) \cdot (-5)=10; \\
 & u_3=u_2 \cdot q=10 \cdot (-5)=-50; \\
 & u_4=u_3 \cdot q=(-50) \cdot (-5)=250;
\end{align*}
Vậy bốn số hạng đầu tiên của cấp số nhân $(u_n)$ là $-2;10;-50;250$.
}
\end{ex}
% Câu 25
\begin{ex}
Cho cấp số nhân $(u_n)$ có $u_1=8$ và công bội $q=3$. Số hạng $u_2$ của cấp số nhân là
\choice
{$11$}
{$-24$}
{$8$}
{\True $24$}
\loigiai{
Ta có $u_2=u_1 \cdot q=8 \cdot 3=24$.
}
\end{ex}
% Câu 26
\begin{ex}
Cho cấp số nhân $(u_n)$ có $u_1=2$, $q=5$. Số hạng $u_2$ của cấp số nhân là
\choice
{$50$}
{$12$}
{$7$}
{\True $10$}
\loigiai{
Ta có $u_2=u_1\cdot q=2\cdot 5=10$.
}
\end{ex}
% Câu 27
\begin{ex}
Cho cấp số nhân $(u_n)$ có $u_3=16$ và công bội $q=2$. Số hạng $u_1$ của cấp số nhân đẫ cho là
\choice
{$12$}
{\True $4$}
{$8$}
{$2$}
\loigiai{
Ta có $u_3=u_1\cdot q^2\Rightarrow u_1=\dfrac{u_3}{q^2}=\dfrac{16}{4}=4$.
}
\end{ex}
% Câu 28
\begin{ex}
Cho dãy số $(u_n)$ được cho bởi hệ thức truy hồi $\heva{& u_1=5,\\ &u_n=u_{n-1}+n\ (n\ge 2)}$. Giá trị của $u_3$ là
\choice
{\True $10$}
{$14$}
{$7$}
{$9$}
\loigiai{
Ta có $u_2=u_1+2=5+2=7$, $u_3=u_2+3=7+3=10$.
}
\end{ex}
% Câu 29
\begin{ex}
Cho cấp số cộng $(u_n)$ có số hạng đầu $u_1=2$ và công sai $d=3$. Tính $u_{10}$
\choice
{$u_{10}=32$}
{$u_{10}=35$}
{$u_{10}=26$}
{\True $u_{10}=29$}
\loigiai{
$u_{10}=u_1+9d=2+9 \cdot 3=29$.
}
\end{ex}
% Câu 30
\begin{ex}
Cho cấp số nhân $(u_n)$ có $u_1=2$ và $u_6=-64$. Số hạng $u_3$ của cấp số nhân đã cho là 
\choice
{$-2$}
{$16$}
{$-8$}
{\True $8$}
\loigiai{
Ta có: $u_6=-64\Rightarrow u_1 \cdot q^5=-64\Rightarrow 2 \cdot q^5=-64\Rightarrow q^5=-32\Rightarrow q=-2$.\\
$\Rightarrow u_3=u_1 \cdot q^2=2 \cdot (-2)^{2}=8$.
}
\end{ex}
% Câu 31
\begin{ex}
Cho cấp số cộng $(u_n)$ có các số hạng $u_2=2$ và $u_3=5$. Số hạng $u_5$ của cấp số cộng là
\choice
{$15$}
{\True $11$}
{$12$}
{$25$}
\loigiai{
Ta có $\heva{& u_2=2\\ & u_3=5} \Leftrightarrow \heva{& u_1+d=2\\ & u_1+2d=5} \Leftrightarrow \heva{& u_1=-1\\ & d=3}$.\\
Do đó $u_5=u_1+4d=11$.
}
\end{ex}
% Câu 32
\begin{ex}
Cho cấp số nhân $(u_n)$ có $u_{2024}=5,u_{2025}=25$. Tính công bội của cấp số nhân.
\choice
{\True $5$}
{$2$}
{$4$}
{$1$}
\loigiai{
Công bội của cấp số nhân $(u_n)$ là $q=\dfrac{u_{2025}}{u_{2024}}=\dfrac{25}{5}=5$.
}
\end{ex}
% Câu 33
\begin{ex}
Cho cấp số cộng $(u_n)$ có công sai $d=-2$ và tổng của $8$ số hạng đầu tiên $S_8=72$. Số hạng đầu tiên $u_1$ của cấp số cộng bằng
\choice
{\True $16$}
{$14$}
{$4$}
{$2$}
\loigiai{
Ta có $S_8=\dfrac{(u_1+u_8) \cdot 8}{2}=\dfrac{8}{2}(2u_1+7d) \Rightarrow 4(2u_1+7d)=72 \Rightarrow 2u_1+7(-2)=18 \Rightarrow u_1=16$.
}
\end{ex}
% Câu 34
\begin{ex}
Cho cấp số nhân $(u_n)$ với $u_1=3$ và công bội $q=2$. Giá trị của $u_3$ bằng 
\choice
{\True $12$}
{$18$}
{$7$}
{$6$}
\loigiai{
Ta có: $u_3=u_1 \cdot q^2=3 \cdot 2^{2}=12$.
}
\end{ex}
% Câu 35
\begin{ex}
Cho cấp số nhân $(u_n)$ có $u_1=1;u_2=4$. Khi đó, $u_3$ bằng
\choice
{$9$}
{$7$}
{\True $16$}
{$12$}
\loigiai{
Ta có $u_2=u_1 \cdot q$ nên $4=1 \cdot q\Leftrightarrow q=4$.\\
Khi đó $u_3=u_2 \cdot q=4 \cdot 4=16$.
}
\end{ex}
% Câu 36
\begin{ex}
Cho cấp số nhân $(u_n)$ có $u_1=2$ và $u_2=8$. Công bội của cấp số nhân đã cho bằng
\choice
{\True $4$}
{$-6$}
{$\dfrac{1}{2}$}
{$6$}
\loigiai{
Công bội của cấp số nhân $(u_n)$ là $q=\dfrac{u_2}{u_1}=\dfrac{8}{2}=4$.
}
\end{ex}
% Câu 37
\begin{ex}
Cho cấp số cộng $(u_n)$, có $u_1=-1$ và $u_2=3$. Số hạng $u_5$ của cấp số cộng là
\choice
{\True $15$}
{$5$}
{$9$}
{$13$}
\loigiai{
Ta có công sai $d=4$, suy ra $u_5=u_1+4d=-1+4 \cdot 4=15$.
}
\end{ex}
% Câu 38
\begin{ex}
Cho cấp số nhân $(u_n)$ với $u_1=2$ và $u_4=250$. Công bội $q$ của cấp số nhân đã cho bằng
\choice
{\True $q=5$}
{$q=\dfrac{2}{5}$}
{$q=3$}
{$q=\dfrac{5}{2}$ }
\loigiai{
Ta có $u_4=u_1\cdot q^3\Rightarrow 250=2\cdot q^3\Rightarrow q^3=125\Rightarrow q=5$\\
Vậy công bội của cấp số nhân là $q=5$.
}
\end{ex}
% Câu 39
\begin{ex}
Cho cấp số nhân $(u_n)$ có $u_1=3$ và $u_4=24$. Tính giá trị $u_3$.
\choice
{$21$}
{$18$}
{\True $12$}
{$48$}
\loigiai{
Ta có: $u_4=u_1 \cdot q^3\Leftrightarrow 3 \cdot q^3=24\Leftrightarrow q^3=8 \Rightarrow q=2$.\\
Vậy $u_3=u_1 \cdot q^2=3 \cdot 2^{2}=12$.
}
\end{ex}
% Câu 40
\begin{ex}
Cấp số nhân $(u_n)$ có $u_1=3,u_2=1$. Công bội của cấp số nhân đã cho bằng
\choice
{$-2$}
{\True $\dfrac{1}{3}$}
{$3$}
{$2$ }
\loigiai{
Ta có $u_2=u_1\cdot q\Rightarrow q=\dfrac{u_2}{u_1}=\dfrac{1}{3}$.
}
\end{ex}
% Câu 41
\begin{ex}
Cấp số cộng $(u_n)$ có $u_2=3$ và công sai $d=2$. Số hạng $u_3$ của cấp số cộng là
\choice
{$6$}
{$1$}
{$7$}
{\True $5$}
\loigiai{
Ta có $u_3=u_2+d=3+2=5$.
}
\end{ex}
% Câu 42
\begin{ex}
Cho cấp số cộng $(u_n)$ với $u_1=3$ và công sai $d=3$. Số hạng thứ ba $u_3$ của cấp số cộng bằng
\choice
{$8$}
{$6$}
{\True $9$}
{$7$}
\loigiai{
$u_3=u_1+2d=3+2 \cdot 3=9$.
}
\end{ex}
% Câu 43
\begin{ex}
Cấp số nhân $(u_n)$ có $u_1=2$ và $u_2=6$. Số hạng $u_4$ của cấp số nhân là
\choice
{\True $27$}
{$162$}
{$54$}
{$11$}
\loigiai{
Ta có $u_2=u_1 \cdot q\Leftrightarrow q=\dfrac{u_2}{u_1}=3$.\\
Nên $u_4=u_1 \cdot q^3=2 \cdot 3^{3}=54$.
}
\end{ex}
% Câu 44
\begin{ex}
Cho cấp số cộng $(u_n)$ có số hạng đầu $u_1=3$, công sai $d=2$. Số hạng thứ 5 của $(u_n)$ bằng
\choice
{$14$}
{$5$}
{$6$}
{\True $11$}
\loigiai{
Ta có: $u_5=u_1+4d=3+4 \cdot 2=11$.
}
\end{ex}
% Câu 45
\begin{ex}
Cho cấp số cộng $(u_n)$ có $u_1=1$ và công sai $d=2$. Tổng của $10$ số hạng đầu tiên của cấp số cộng đã cho là 
\choice
{$200$}
{$110$}
{$95$}
{\True $100$}
\loigiai{
Ta có tổng của $10$ số hạng là $S_{10}=\dfrac{10}{2}\left[2u_1+(10-1)d\right]=100$.
}
\end{ex}
% Câu 46
\begin{ex}
Cho dãy số $(u_n)$ là một cấp số nhân có số hạng đầu $u_1=2$ và công bội $q=3$. Xác định số hạng thứ hai của cấp số nhân $(u_n)$.
\choice
{\True $u_2=6$}
{$u_2=18$}
{$u_2=8$}
{$u_2=5$}
\loigiai{
Ta có $u_2=u_1\cdot q=2\cdot 3=6$.
}
\end{ex}
% Câu 47
\begin{ex}
Cho cấp số nhân $(u_n)$ biết $u_2 \cdot u_6=64$. Giá trị của $u_3 \cdot u_5$ bằng
\choice
{$-64$}
{\True $64$}
{$-8$}
{$8$}
\loigiai{
Gọi $q$ là công bội của cấp số nhân đã cho.\\
Ta có $u_2 \cdot u_6=(u_1 q)\cdot(u_1 q^5)=u_1^2 q^6=64$.\\
Vậy $u_3 \cdot u_5=(u_1 q^2)\cdot(u_1 q^4)=u_1^2 q^6=64$.
}
\end{ex}
% Câu 48
\begin{ex}
Cho cấp số nhân $(u_n)$ với $u_1=-2$ và $q=-5$. Viết bốn số hạng đầu tiên của cấp số nhân.
\choice
{$-2;10;50;-250$}
{$-2;-10;-50;-250$}
{$-2;10;50;250$}
{\True $-2;10;-50;250$}
\loigiai{
Ta có:\\
$u_2=u_1 \cdot q=(-2) \cdot (-5)=10$.\\
$u_3=u_1 \cdot q^2=(-2) \cdot (-5)^2=-50$.\\
$u_4=u_1 \cdot q^3=(-2) \cdot (-5)^3=250$.\\
Vậy bốn số hạng đầu tiên của cấp số nhân là $-2;10;-50;250$.
}
\end{ex}
% Câu 49
\begin{ex}
Cho cấp số nhân $(u_n)$ với $u_1=1$ và $u_2=2$. Công bội của cấp số nhân đã cho là
\choice
{$q = \dfrac{1}{2}$}
{\True $q = 2$}
{$q = -2$}
{$q = -\dfrac{1}{2}$}
\loigiai{
Ta có $q=\dfrac{u_2}{u_1}=2$.
}
\end{ex}
% Câu 50
\begin{ex}
Cho cấp số nhân $(u_n)$ với $u_3=8$ và công bội $q=-2$. Giá trị $u_2$ bằng
\choice
{$-16$}
{$10$}
{$6$}
{\True $-4$}
\loigiai{
Ta có $u_2=\dfrac{u_3}{q}=\dfrac{8}{-2}=-4$.
}
\end{ex}
% Câu 51
\begin{ex}
Cho cấp số nhân $(u_n)$ có $u_1=-2$ và $u_2=6$. Số hạng thứ ba của cấp số nhân $(u_n)$ bằng
\choice
{$u_3=-12$}
{$u_3=18$}
{\True $u_3=-18$}
{$u_3=10$}
\loigiai{
Ta có $u_2=u_1\cdot q\Rightarrow q=-3$. Suy ra $u_3=u_2\cdot q=-18$.
}
\end{ex}
% Câu 52
\begin{ex}
Cho cấp số cộng $(u_n)$ có $u_2=5$, $u_4=11$. Công sai của cấp số cộng đã cho bằng
\choice
{$6$}
{\True $3$}
{$9$}
{$2$}
\loigiai{
Ta có: $\heva{& u_2=5\\ & u_4=11} \Leftrightarrow \heva{& u_1+d=5\\ & u_1+3d=11} \Leftrightarrow \heva{& d=3\\ & u_1=2}$.
}
\end{ex}
% Câu 53
\begin{ex}
Cấp số nhân $(u_n)$ có $u_1=3$ và công bội $q=4$. Số hạng $u_2$ của cấp số nhân là
\choice
{$64$}
{\True $12$}
{$81$}
{$7$}
\loigiai{
Ta có $u_2=u_1\cdot q=3\cdot 4=12$.
}
\end{ex}
% Câu 54
\begin{ex}
Trong các dãy số sau, dãy nào là một cấp số cộng?
\choice
{$u_n=7-3^n$}
{\True $u_n=7-3n$}
{$u_n=\dfrac{7}{3^n}$}
{$u_n=7 \cdot 3^n$}
\loigiai{
Do $u_{n+1}-u_n=[7-3(n+1)]-(7-3n)=-3,\ \forall n\in \mathbb{N}^*$.
}
\end{ex}
% Câu 55
\begin{ex}
Cho cấp số nhân $(u_n)$ có $u_1=2$ và $u_2=6$. Công bội $q$ của cấp số nhân bằng
\choice
{$q=8$}
{$q=4$}
{$q=12$}
{\True $q=3$}
\loigiai{
Ta có: $u_2=u_1 \cdot q\Rightarrow q=\dfrac{u_2}{u_1}=\dfrac{6}{2}=3$.
}
\end{ex}
% Câu 56
\begin{ex}
Cho cấp số cộng $(u_n)$ có $u_1=3$ và công sai $d=-3$. Số hạng thứ ba $u_3$ của cấp số cộng bằng
\choice
{\True $u_3=-3$}
{$u_3=3$}
{$u_3=0$}
{$u_3=6$}
\loigiai{
Ta có $u_3=u_1+2d=-3$.
}
\end{ex}
% Câu 57
\begin{ex}
Cho cấp số nhân $(u_n)$ với số hạng đầu $u_1=6$ và công bội $q=-\dfrac{1}{2}$. Tính $u_5$.
\choice
{\True $\dfrac{3}{8}$}
{$-3$}
{$-\dfrac{3}{8}$}
{$-\dfrac{4}{3}$}
\loigiai{
Ta có: $u_5=u_1 \cdot q^4=6 \cdot \left(-\dfrac{1}{2}\right)^4=\dfrac{3}{8}$.
}
\end{ex}
% Câu 58
\begin{ex}
Cho cấp số cộng $(u_n)$ có $u_1=-1,u_2=5$. Công sai của cấp số cộng đó bằng
\choice
{$-5$}
{$-6$}
{\True $6$}
{$4$}
\loigiai{
Ta có: $d=u_2-u_1=5-(-1)=6$.
}
\end{ex}
% Câu 59
\begin{ex}
Cho cấp số nhân $(u_n)$ với $u_1=2$ và công bội $q=3$. Giá trị của $u_4$ bằng
\choice
{\True $54$}
{$162$}
{$24$}
{$48$}
\loigiai{
Vì $(u_n)$ là cấp số nhân với $u_1=2$ và công bội $q=3\Rightarrow u_4=u_1q^3=2 \cdot 3^{3}=54$.
}
\end{ex}
% Câu 60
\begin{ex}
Cho cấp số nhân $(u_n)$ có $u_1=2$ và công bội $q=5$. Số hạng $u_3$ của cấp số nhân đã cho là
\choice
{\True $u_3=50$}
{$u_3=12$}
{$u_3=10$}
{$u_3=7$}
\loigiai{
$u_3=u_1 \cdot q^2=2 \cdot 5^{2}=50$.
}
\end{ex}
% Câu 61
\begin{ex}
Cho cấp số cộng $(u_n)$ có $u_1=1$ và $u_2=-3$. Số hạng $u_4$ của cấp số cộng đã cho là
\choice
{$-7$}
{\True $-11$}
{$-27$}
{$-14$}
\loigiai{
Ta có: $u_1=1$ và $u_2=-3\Rightarrow d=-4$.\\
$u_n=u_1+(n-1)d\Rightarrow u_4=-11$.
}
\end{ex}
% Câu 62
\begin{ex}
Cấp số cộng $(u_n)$ có $u_1=1$ và $u_2=3$. Số hạng $u_5$ của cấp số cộng là
\choice
{$11$}
{$7$}
{$5$}
{\True $9$}
\loigiai{
Ta có $d=u_2-u_1=3-1=2$.\\
Vậy $u_5=u_1+4d=1+4 \cdot 2=9$.
}
\end{ex}
% Câu 63
\begin{ex}
Cho cấp số nhân $(u_n)$ có $u_1=2$ và $u_2=-8$. Công bội $q$ của cấp số nhân là
\choice
{$q=10$}
{$q=-\dfrac{1}{4}$}
{\True $q=-4$}
{$q=-10$}
\loigiai{
Ta có $q=\dfrac{u_2}{u_1}=\dfrac{-8}{2}=-4$.
}
\end{ex}
% Câu 64
\begin{ex}
Cho cấp số nhân $(u_n)$ có $u_1=1$ và $u_2=2$. Số hạng $u_3$ của cấp số nhân đó là 
\choice
{$8$}
{\True $4$}
{$2$}
{$3$}
\loigiai{
Cấp số nhân $(u_n)$ có công bội $q=\dfrac{u_2}{u_1}=2$. \\
Vậy số hạng $u_3$ của cấp số nhân đó là $u_3=u_2 \cdot q=2 \cdot 2=4$.
}
\end{ex}
% Câu 65
\begin{ex}
Cho cấp số nhân $(u_n)$ biết $u_1=5$, công bội $q=2$. Tính $u_{12}$
\choice
{\True $u_{12}=10240$}
{$u_{12}=4096$}
{$u_{12}=2048$}
{$u_{12}=20480$}
\loigiai{
Ta có $u_{12}=u_1 \cdot q^{11}=5 \cdot 2^{11}=10240$.
}
\end{ex}
% Câu 66
\begin{ex}
Cho cấp số nhân $(u_n)$ có $u_1=1$, $u_4=27$. Công bội $q$ của cấp số nhân đã cho bằng
\choice
{$\dfrac{26}{3}$}
{$\pm 3$}
{$-3$}
{\True $q=3$}
\loigiai{
Ta có $u_4=u_1 \cdot q^3\Rightarrow q^3=\dfrac{u_4}{u_1}=\dfrac{27}{1}=3^3$. Vậy $q=3$\\
Vậy công bội của cấp số nhân là $q=3$.
}
\end{ex}
% Câu 67
\begin{ex}
    Cho cấp số cộng $(u_n)$ có $u_5=9$, $u_6=15$. Công sai của cấp số cộng $(u_n)$ bằng
\choice
{$\dfrac{3}{5}$}
{$\dfrac{5}{3}$}
{\True $6$}
{$-6$}
\loigiai{
$d=u_6-u_5=15-9=6$.
}
\end{ex}
% Câu 68
\begin{ex}
Cấp số nhân $(u_n)$ có $u_1=-2$ và công bội $q=3$. Số hạng $u_2$ là
\choice
{$u_2=1$}
{\True $u_2=-6$}
{$u_2=6$}
{$u_2=-18$}
\loigiai{
Ta có: $u_2=u_1 \cdot q=-2 \cdot 3=-6$.
}
\end{ex}
% Câu 69
\begin{ex}
Cho cấp số nhân $(u_n)$ với $u_1=3$ và công bội $q=-2$. Giá trị của $u_4$ bằng 
\choice
{$24$}
{$-12$}
{\True $-24$}
{$12$}
\loigiai{
Chọn C. \\
$u_4=u_1 \cdot q^3=3 \cdot (-2)^{3}=-24$.
}
\end{ex}
% Câu 70
\begin{ex}
Cho cấp số nhân có $u_1=2$ và công bội $q=-3$. Số hạng $u_4$ bằng
\choice
{$-18$}
{$18$}
{$54$}
{\True $-54$}
\loigiai{
Vì $u_4=u_1 \cdot q^3=2 \cdot (-3)^{3}=-54$.
}
\end{ex}
% Câu 71
\begin{ex}
Cho cấp số cộng $(u_n)$ có $u_1=5,\ u_7=29$. Công sai của cấp số cộng đã cho là
\choice
{$d=24$}
{$d=\dfrac{29}{5}$}
{\True $d=4$}
{$d=145$}
\loigiai{
Vì $(u_n)$ là cấp số cộng nên công sai $d=\dfrac{u_7-u_1}{6}=\dfrac{29-5}{6}=4$.
}
\end{ex}
% Câu 72
\begin{ex}
Cho cấp số nhân $(u_n)$ với $u_1=2$ và $u_2=8$. Giá trị của $u_4$ bằng
\choice
{\True $128$}
{32}
{512}
{18}
\loigiai{
$q=\dfrac{u_2}{u_1}=\dfrac{8}{2}=4$.\\
$u_4=u_1 \cdot q^3=2 \cdot 4^{3}=128$.
}
\end{ex}
% Câu 73
\begin{ex}
Cho cấp số nhân $(u_n)$ có $u_3=12$ và công bội $q=2$. Số hạng đầu tiên $u_1$ bằng
\choice
{4}
{\True $3$}
{8}
{6}
\loigiai{
Ta có: $u_3=u_1 \cdot q^2 \Leftrightarrow 12=u_1 \cdot 2^2 \Leftrightarrow u_1=3$.
}
\end{ex}
% Câu 74
\begin{ex}
Cho cấp số nhân $(u_n)$, biết $u_1=1;u_4=64$. Tính công bội $q$ của cấp số nhân.
\choice
{$q=\pm 4$}
{\True $q=4$}
{$q=21$}
{$q=2\sqrt{2}$}
\loigiai{
Ta có $u_4=u_1 \cdot q^3\Rightarrow q^3=\dfrac{u_4}{u_1}=64\Rightarrow q=4$\\
Vậy công bội của cấp số nhân là $q=4$.
}
\end{ex}
% Câu 75
\begin{ex}
Cho cấp số cộng có hai số hạng liên tiếp là $70$ và $65$. Tính số hạng tiếp theo.
\choice
{\True $60$}
{$62$}
{$59$}
{$57$}
\loigiai{
Vì cấp số cộng có hai số hạng liên tiếp là $70$ và $65$ nên công sai của cấp số cộng là $d=65-70=-5$. Vậy số hạng tiếp theo là $65-5=60$.
}
\end{ex}
% Câu 76
\begin{ex}
Cho cấp số cộng $(u_n)$ có công sai $d=2$ và tổng của 25 số hạng đầu là 625. Tính $u_5$
\choice
{$u_5=16$}
{\True $u_5=9$}
{$u_5=12$}
{$u_5=22$}
\loigiai{
Ta có $S_{25}=\dfrac{25}{2}(2u_1+24d)=625 \Rightarrow 25(u_1+12d)=625 \Rightarrow u_1+12d=25$.\\
Vì $d=2$ nên $u_1+24=25 \Rightarrow u_1=1$.\\
Vậy $u_5=u_1+4d=1+4\cdot 2=9$.
}
\end{ex}
% Câu 77
\begin{ex}
    Cho cấp số cộng $(a_n)$ có số hạng đầu $a_1=5$ và công sai $d=3$. Số hạng thứ bảy của cấp số cộng đó là
\choice
{$20$}
{$28$}
{\True $23$}
{$22$}
\loigiai{
    $a_7=a_1+(7-1) \cdot d=5+(7-1) \cdot 3=23$.
}
\end{ex}
% Câu 78
\begin{ex}
Cho cấp số nhân $(u_n)$ có $u_3=12$ và công bội $q=2$. Số hạng $u_1$ của cấp số nhân đã cho là?
\choice
{$7$}
{\True $3$}
{$9$}
{$6$}
\loigiai{
$u_3=u_1 \cdot q^2\Leftrightarrow u_1=\dfrac{u_3}{q^2}=\dfrac{12}{2^2}=3$.
}
\end{ex}
% Câu 79
\begin{ex}
Cho cấp số cộng $(u_n)$ biết $u_1=\dfrac{1}{3}$, $u_8=26$. Công sai $d$ của cấp số cộng đó là 
\choice
{\True $\dfrac{11}{3}$}
{$\dfrac{10}{3}$}
{$\dfrac{3}{10}$}
{$\dfrac{3}{11}$}
\loigiai{
Ta có $u_8=u_1+7d\Leftrightarrow 26=\dfrac{1}{3}+7d\Leftrightarrow d=\dfrac{11}{3}$.
}
\end{ex}
% Câu 80
\begin{ex}
Cho cấp số cộng $(u_n)$ biết $u_1=2$, công sai $d=-5$.Tổng $10$ số hạng đầu của cấp số cộng đó là
\choice
{$-410$}
{\True $-205$}
{$245$}
{$-230$}
\loigiai{
Tổng $10$ số hạng đầu của cấp số cộng đó là $S_{10}=\dfrac{10}{2}\left[2 \cdot 2+9 \cdot (-5)\right]=-205$.
}
\end{ex}
% Câu 81
\begin{ex}
    Cho một cấp số nhân $(u_n)$ có $u_1=2$ và $u_2=8$. Công bội của cấp số nhân đã cho bằng
\choice
{\True $4$}
{$6$}
{$\dfrac{1}{2}$}
{$-6$}
\loigiai{
    Công bội của cấp số nhân $(u_n)$ là $q=\dfrac{u_2}{u_1}=\dfrac{8}{2}=4$.
}
\end{ex}
% Câu 82
\begin{ex}
Cho cấp số cộng $(u_n)$ có $u_1=-2$ và công sai $d=3$. Số hạng tổng quát $u_n$ là
\choice
{$u_n=3n-5$}
{$u_n=-2n+3$}
{$u_n=-3n+2$}
{$u_n=3n-2$}
\loigiai{
Số hạng tổng quát của cấp số là $u_n=u_1+(n-1)d=-2+3(n-1)=3n-5$.
}
\end{ex}
% Câu 83
\begin{ex}
Dãy số nào sau đây không phải là cấp số nhân?
\choice
{$1;-2;4;-8;16$}
{$1;2;4;8;16$}
{$1;-1;1;-1;1$}
{\True $1;-3;9;-27;54$}
\loigiai{.
}
\end{ex}
% Câu 84
\begin{ex}
Cho cấp số cộng $(u_n)$ có $u_4=-12;u_{14}=18$. Tìm $u_1,d$ của cấp số cộng?
\choice
{$u_1=-21,d=-3$}
{$u_1=-22,d=3$}
{\True $u_1=-21,d=3$}
{$u_1=20,d=-3$}
\loigiai{
Ta có $\begin{cases} u_4=-12 \\ u_{14}=18 \end{cases} \Leftrightarrow \begin{cases} u_1+3d=-12 \\ u_1+13d=18 \end{cases} \Leftrightarrow \begin{cases} u_1=-21 \\ d=3 \end{cases}$.
}
\end{ex}
% Câu 85
\begin{ex}
Cho cấp số nhân $(u_n)$ với $u_1=-8$ và $u_4=1$. Công bội của cấp số nhân đã cho bằng
\choice
{$\dfrac{1}{2}$}
{$-2$}
{$2$}
{\True $-\dfrac{1}{2}$}
\loigiai{
Ta có: $u_4=u_1 \cdot q^3\Leftrightarrow -8=1 \cdot q^3\Leftrightarrow q^3=-\dfrac{1}{8}\Leftrightarrow q=-\dfrac{1}{2}$.
}
\end{ex}
% Câu 86
\begin{ex}
Cho cấp số cộng $(u_n)$ thỏa mãn $u_{n+1}=u_n-2$ với mọi $n\in \mathbb{N}^*$. Công sai $d$ của cấp số cộng này bằng
\choice
{2}
{-1}
{1}
{-2}
\loigiai{
Do $u_{n+1}-u_n=-2$ với mọi $n\in \mathbb{N}^*$ nên công sai $d=-2$.
}
\end{ex}
% Câu 87
\begin{ex}
Cho cấp số nhân lùi vô hạn $u_1, u_1q, \ldots, u_1q^{n-1},\ldots$. Tổng của cấp số nhân lùi vô hạn đã cho được xác định bởi công thức nào sau đây ? 
\choice
{$S=u_1+u_1q+u_1q^2+ \cdots=\dfrac{q}{1-u_1}$}
{\True $S=u_1+u_1q+u_1q^2+ \cdots=\dfrac{u_1}{1-q}$}
{$S=u_1+u_1q+u_1q^2+ \cdots=\dfrac{u_1}{1+q}$}
{$S=u_1+u_1q+u_1q^2+ \cdots=\dfrac{u_1}{q-1}$}
\loigiai{
Tổng của cấp số nhân lùi vô hạn được xác định bởi công thức : $S=u_1+u_1q+u_1q^2+ \cdots=\dfrac{u_1}{1-q}$.
}
\end{ex}
% Câu 88
\begin{ex}
Cấp số cộng $(u_n)$ có $u_n=3n+4$. Công sai của cấp số cộng này là
\choice
{4}
{\True 3}
{7}
{$-3$}
\loigiai{
Ta có: $u_1=3 \cdot 1+4=7;u_2=3 \cdot 2+4=10\Rightarrow d=u_2-u_1=10-7=3$.
}
\end{ex}
% Câu 89
\begin{ex}
    Cấp số nhân $(u_n)$ có $u_1=2$ và $u_2=-8$. Công bội $q$ của cấp số nhân là
\choice
{$q=10$}
{$q=-\dfrac{1}{4}$}
{\True $q=-4$}
{$q=-10$}
\loigiai{
    Ta có $q=\dfrac{u_2}{u_1}=\dfrac{-8}{2}=-4$.
}
\end{ex}
% Câu 90
\begin{ex}
Cho cấp số cộng $(u_n)$ với $u_1=3$ và $u_2=9$. Công sai của cấp số cộng đã cho bằng
\choice
{$12$}
{$3$}
{$-6$}
{\True $6$}
\loigiai{
Ta có $d=u_2-u_1=9-3=6\Rightarrow d=6$.
}
\end{ex}
% Câu 91
\begin{ex}
Cho cấp số cộng $(u_n)$, biết $u_2=3$ và $u_4=7$. Giá trị của $u_{15}$ bằng
\choice
{$27$}
{$31$}
{$35$}
{\True $29$}
\loigiai{
Ta có $\begin{cases} u_2=u_1+d \\ u_4=u_1+3d \end{cases} \Leftrightarrow \begin{cases} u_1+d=3 \\ u_1+3d=7 \end{cases} \Leftrightarrow \begin{cases} d=2 \\ u_1=1 \end{cases}$.\\
Vậy $u_{15}=u_1+14d=1+14 \cdot 2=29$.
}
\end{ex}
% Câu 92
\begin{ex}
Cho cấp số nhân $(u_n)$ có $u_2=4$ và công bội $q=2$. Số hạng $u_5$ của cấp số nhân là
\choice
{$16$}
{$64$}
{$128$}
{\True $32$}
\loigiai{
Ta có $u_5=u_1 \cdot q^4=u_2 \cdot q^3=4 \cdot 2^{3}=32$.
}
\end{ex}
% Câu 93
\begin{ex}
Cho cấp số cộng $(u_n)$ có $u_1=-3$, $u_6=27$. Tính công sai $d$.
\choice
{$d=7$}
{$d=5$}
{$d=8$}
{\True $d=6$}
\loigiai{
Ta có $u_6=u_1+5d\Rightarrow 27=-3+5d\Rightarrow d=6$.
}
\end{ex}
% Câu 94
\begin{ex}
Cho cấp số cộng $(u_n)$ với $u_1=9$ và công sai $d=2$. Giá trị của $u_2$ bằng
\choice
{\True $11$}
{$\dfrac{9}{2}$}
{$18$}
{$7$}
\loigiai{
Chọn A. Ta có: $u_2=u_1+d=9+2=11$.
}
\end{ex}
% Câu 95
\begin{ex}
Cho cấp số nhân $(u_n)$ có $u_3=5$ và $u_6=40$. Số hạng $u_4$ của cấp số nhân là
\choice
{$u_4=-15$}
{$u_4=-10$}
{$u_4=15$}
{\True $u_4=10$}
\loigiai{
Ta có $\begin{cases} u_3=u_1q^2 \\ u_6=u_1q^5 \end{cases} \Leftrightarrow \begin{cases} u_1=\dfrac{5}{4} \\ q=2 \end{cases} \Rightarrow u_4=10$.
}
\end{ex}
% Câu 96
\begin{ex}
Cho cấp số nhân $(u_n)$ với $u_1=3$ và công bội $q=2$. Giá trị của $u_2$ bằng
\choice
{\True $6$}
{$\dfrac{3}{2}$}
{$9$}
{$8$}
\loigiai{
$u_2=u_1 \cdot q=3 \cdot 2=6$.
}
\end{ex}
% Câu 97
\begin{ex}
Cho cấp số nhân $(u_n)$ có số hạng đầu $u_1=2$ và $u_6=486$. Công bội q bằng
\choice
{$q=5$}
{$q=\dfrac{3}{2}$}
{$q=\dfrac{2}{3}$}
{\True $q=3$}
\loigiai{
Ta có: $\begin{cases} u_1=2 \\ u_6=486 \end{cases} \Leftrightarrow \begin{cases} u_1=2 \\ u_1 \cdot q^5=486 \end{cases} \Rightarrow q^5=243 \Rightarrow q=3$.
}
\end{ex}
% Câu 98
\begin{ex}
Cấp số cộng $(u_n)$ có $u_1=-1$ và $u_9=23$. Số hạng $u_5$ của cấp số cộng là
\choice
{$10$}
{$14$}
{\True $11$}
{$8$}
\loigiai{
Ta có $u_9=u_1+8d\Leftrightarrow 23=-1+8d\Leftrightarrow d=3$ suy ra $u_5=u_1+4d=-1+4 \cdot 3=11$.
}
\end{ex}
% Câu 99
\begin{ex}
Cấp số cộng $(u_n)$ có số hạng đầu $u_1=3$, công sai $d=-2$ thì số hạng thứ $5$ là
\choice
{\True $u_5=-5$}
{$u_5=1$}
{$u_5=8$}
{$u_5=-7$}
\loigiai{
Ta có $u_5=u_1+(5-1)d=3+4(-2)=-5$.
}
\end{ex}
% Câu 100
\begin{ex}
Cho cấp số cộng $(u_n)$ có $u_1=-2$ và công sai $d=3$. Tìm số hạng thứ 10 của cấp số cộng.
\choice
{$u_{10}=25$}
{$u_{10}=26$}
{\True $u_{10}=25$}
{$u_{10}=27$}
\loigiai{
Ta có $u_{10}=u_1+9d=-2+9 \cdot 3=25$.
}
\end{ex}
% Câu 101
\begin{ex}
Cho cấp số nhân $(u_n)$ với $u_1=2$ và công bội $q=3$. Tìm số hạng thứ 4 của cấp số nhân.
\choice
{$48$}
{\True $54$}
{$24$}
{$162$}
\loigiai{
Số hạng thứ 4 của cấp số nhân $u_4=u_1 \cdot q^3=2 \cdot 3^{3}=54$.
}
\end{ex}
% Câu 102
\begin{ex}
Giá tiền khoan giếng được tính như sau: Giá của mét đầu tiên là $60000$ đồng, từ mét khoan thứ hai trở đi, giá của mỗi mét khoan sau tăng $7\%$ so với mét khoan trước đó. Nếu khoan giếng sâu $50\text{m}$ thì cần số tiền là
\choice
{$24492000$}
{$24399000$}
{\True $24392000$}
{$24382000$}
\loigiai{
Gọi $u_1,u_2,\ldots,u_{50}$ lần lượt là giá mỗi mét khoan giếng, ta có\\
$u_1=60000$\\
$u_2=60000+\dfrac{7}{100} \cdot 60000=60000\left(1+\dfrac{7}{100}\right)$\\
$u_3=60000\left(1+\dfrac{7}{100}\right)^2$\\
$\cdots$\\
Vậy $u_1,u_2,u_3, \ldots ,u_{50}$ theo thứ tự là các số hạng của cấp số nhân với số hạng đầu $u_1=60000$ và công bội $q=1+\dfrac{7}{100}$.\\
Khi đó tổng số tiền phải trả là $S=u_1+u_2+ \ldots +u_{50}=u_1 \cdot \dfrac{1-q^{50}}{1-q}=24392000$..
}
\end{ex}
% Câu 103
\begin{ex}
Một trường học tổ chức trải nghiệm cho học sinh bằng cách tổ chức các trò chơi, trong đó có trò chơi sử dụng đồng xu để xếp thành một kim tự tháp. Yêu cầu mỗi nhóm học sinh sử dụng $253$ đồng tiền xu để xếp một mô hình kim tự tháp. Biết rằng tầng dưới cùng có $58$ đồng xu và cứ lên thêm một tầng thì số đồng xu giảm đi $7$ đồng. Tập hợp số xu ở mỗi tầng tạo thành
\choice
{một cấp số nhân với số hạng đầu và công bội lần lượt là $u_1=58;q=7$}
{\True một cấp số cộng với số hạng đầu và công sai lần lượt là $u_1=58;d=-7$}
{một cấp số cộng với số hạng đầu và công sai lần lượt là $u_1=58;d=7$}
{một cấp số nhân với số hạng đầu và công bội lần lượt là $u_1=58;q=-7$}
\loigiai{
Gọi số đồng xu ở tầng dưới cùng là $u_1=58$.\\
Số đồng xu ở tầng thứ hai là $u_2=u_1+(-7)=58-7=51$.\\
Số đồng xu ở tầng thứ ba là $u_3=u_2+(-7)=51-7=44$.\\
\dots\\
Số đồng xu ở tầng thứ $n$ bằng $u_n=-7n+65$.\\
Tổng số đồng xu bằng $253$ nên ta có $S_n=253 \Leftrightarrow \dfrac{n}{2}\left[2 \cdot u_1+(n-1)d\right]=253$.\\
Ta có $n=11$ thỏa mãn.\\
Vậy tập hợp số xu ở mỗi tầng tạo thành một cấp số cộng với số hạng đầu và công sai lần lượt là $u_1=58;d=-7$..
}
\end{ex}
