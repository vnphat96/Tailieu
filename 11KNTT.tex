\documentclass[10pt,a4paper,onecolumn,titlepage,twoside,openany]{book}
% \usepackage[utf8]{vietnam}
%\usepackage{fouriernc}
\usepackage{tasks}
\usepackage{xcolor} 
%%%%%%%%%%%%% KIỂU MÀU VÀ ĐỘ RỘNG NOTE
\def\kieumau{N} %Y: Màu; N: đen-trắng
% \def\kieumau{Y} %Y: Màu; N: đen-trắng
\def\leftnote{5} %Độ rộng cột Note
%%%%%%%%%%%%% ĐN CƠ BẢN
\input{cautrucDT/color\kieumau} %MÀU
%=====================================
% Khai báo nhóm Tex (cơ bản)
%=====================================
\usepackage{amsmath,amssymb,mathrsfs,maybemath,xlop,polynom,slashbox}
\usepackage{yhmath} %\let\widering\relax %cần khi sd với font fouriernc

\usepackage{enumerate}
\usepackage{tikz} 
\usepackage{tkz-euclide}
%\usepackage{ex_tkz-euclide}
%\usetkzobj{all}
\usepackage{tikz-3dplot}
\usepackage{tkz-tab}
\usepackage{pifont} %kí hiệu đặc biệt
% \usepackage{xcolor}
%\usepackage{bbding}
%\usepackage{array}
\usepackage{tasks}
% \usepackage{casiovn}
%==========
\usetikzlibrary{math,through,calc,intersections,angles,quotes,shapes,shapes.geometric,arrows,patterns,snakes,matrix,chains,arrows.meta,decorations.shapes,decorations.fractals,decorations.markings,shadows}
\usetikzlibrary{positioning,decorations.text,decorations.pathmorphing}% Để uốn cong văn bản 
\usetikzlibrary{shadings,fadings} %ĐỔ BÓNG
\usepackage{pgfplots}
\usepackage{pgfornament}
\usepgfplotslibrary{fillbetween}
\pgfplotsset{compat=1.9}
\usepackage[hidelinks,unicode]{hyperref}
\usepackage{currfile}
\usepackage[outline]{contour} %viền
\usepackage{fontawesome} % Gói kí hiệu
\usepackage{lipsum} %Lấy text
\usepackage{tabularx}
%%---------
%\usepackage{setspace}
%\usepackage{scrextend}
\usepackage{varwidth}
%===========Bảng
\usepackage{longtable,multirow,makecell}
\usepackage{diagbox}
\renewcommand{\tabcolsep}{3mm}
\newcolumntype{C}[1]{>{\centering\arraybackslash}p{#1}}
\newcolumntype{L}[1]{>{\raggedright\arraybackslash}p{#1}}
%-----------Trang vb


%%%%%%%%%%%%% Các thông số trang tài liệu
\def\tren{1.5}\def\duoi{1.5}\def\trai{1.25}\def\phai{0.75} %cách lề
\def\topset{0.75} %kc giữa đáy header và vùng vb
\def\botset{0.75} %kc giữa đỉnh footer và vùng vb
%\usepackage{ifoddpage}
\pgfmathsetmacro{\mepphai}{\phai+\leftnote} 
%\usepackage[top=\tren cm, bottom=\duoi cm, left=\trai cm, right=\mepphai cm] {geometry}
%%%%%%%%%%%%%
%---------------------------------Các thông số trang tài liệu
\pgfmathsetmacro{\so}{\leftnote - 0.5} 
\usepackage[top=\tren cm, bottom=\duoi cm, left=\trai cm, right=\mepphai cm,
marginparwidth=\so cm, marginparsep=5mm,
%,headsep=6mm
%,footskip=10mm
] {geometry}
%-------------------------------------
\usepackage{marginnote}
\setlength{\marginparwidth}{\so cm}
\renewcommand*{\marginfont}{\small}
%--------------Gói trắc nghiệm EX-TEST
% \usepackage[dethi]{ex_test}
\usepackage[loigiai]{ex_test} 
% \usepackage[solcolor]{ex_test}
%----Lời giải, Hiền thị tên EX; Dấu kết thúc
\font\damEX=ugqb8v at 11pt
\def\loigiaiEX{\color{\mauLG}\damEX\strut\faCommenting\ Lời giải.}
%lời giải EXS
\def\loigiaiEXS{\loigiaiEX{\fontsize{8}{16}\selectfont\color{\maucham}\dotfill}}
%--
\renewcommand{\nameex}{\damEX\color{\mauEX} CÂU}
\newtheorem{EX}{\nameex} %MÔI TRƯỜNG PHỤ CHO TÁCH CÂU
\def\mauVuong{cyan}
\def\qedEX{\color{\mauVuong}\ensuremath{\square}}
%--------------Cài đặt lại dòng kẻ \dotline
\renewcommand{\dotlineEX}[1]{
	\def\numlinedot{#1}
	\par
	\foreach \dotline in{1,...,\numlinedot}
	{
		\noindent
		\fontsize{8}{16}\selectfont
		\color{\maucham}\dotfill
		\par
	}
}
% sd cho \dongcham
\newcommand{\dotlineEXS}[1]{
	\def\numlinedot{#1}
	\foreach \dotline in{1,...,\numlinedot}
	{
		\noindent
		\fontsize{8}{16}\selectfont
		\color{\maucham}\dotfill
		\par
	}
}
%---------- Khai báo viết tắt, in đáp án
\newcommand{\hoac}[1]{ %hệ hoặc
	\left[\begin{aligned}#1\end{aligned}\right.}
\newcommand{\heva}[1]{ %hệ và
	\left\{\begin{aligned}#1\end{aligned}\right.}
%--In đáp án
\newcommand{\indapan}[2]{
	\addcontentsline{toc}{subsection}{\sf Bảng đáp án} % đưa MT vào mục lục
	\begin{center}
		\begin{tikzpicture}%
			\node[thick,scale=1,fill=\mauEX!2,draw=\maufoot,minimum width=3.5cm,minimum height=0.1cm,rounded corners=2mm]{\fontfamily{qag}\fontsize{11}{11}\selectfont\bfseries\color{\mauEX} BẢNG ĐÁP ÁN};
		\end{tikzpicture}%
	\end{center}
	\inputansbox{#1}{#2}
}
%----------
\usepackage{esvect}
\def\vec{\vv} %vecto
\def\overrightarrow{\vv}
%Lệnh song song
\DeclareSymbolFont{symbolsC}{U}{txsyc}{m}{n}
\DeclareMathSymbol{\varparallel}{\mathrel}{symbolsC}{9}
\DeclareMathSymbol{\parallel}{\mathrel}{symbolsC}{9}
%--------------------------
% HEADER AND FOOTER STYLING
%--------------------------
%--------------------------
\newcommand{\myfancyhead}{% trên và chấm trái
		\boldmath
\begin{tikzpicture}[remember picture,overlay,>=stealth]
		\path ([yshift=-\tren cm+0.5*\topset cm]current page.north west) coordinate (AA)
		++(\paperwidth,0)coordinate (BB); 
\checkoddpage\ifoddpage %nếu trang lẻ
		%-----đường kẻ
		\draw[\maufoot, line width=2pt] 
		([xshift=\trai cm]AA) --([xshift=-\phai cm]BB);
		%-----bên phải
		\node[text=\maufoot, anchor=south east,inner sep=0pt] at ([xshift=-\phai cm,yshift=4pt]BB){
			\fontfamily{qag}\fontsize{8.5pt}{12pt}\selectfont 
			{\color{\mauSO}\faMapMarker}\,\, \diachi\,\,{\color{\mauSO}\faMapMarker}
		};
		%-----bên trái
		\node[text=\maufoot, anchor=south west,inner sep=0pt] at ([xshift=\trai cm,yshift=4pt]AA){
			\fontfamily{qag}\fontsize{10pt}{10pt}\selectfont\bfseries\faEdit\, \tenchuyende
		};
		%----- Kẻ đứng
		\draw[\maufoot] ([xshift=-\mepphai cm+2.5mm]BB)--([yshift=\duoi cm-0.75*\botset cm,xshift=-\mepphai cm+2.5mm]current page.south east);
		%--		
		\path ([yshift=-\tren cm+0.5*\topset cm-0.5cm,xshift=-\phai cm-0.5*\leftnote cm+2.5mm]current page.north east) coordinate (DDD); 
		\begin{scope}
			\clip ([yshift=-\tren cm+0.5*\topset cm-1pt,xshift=-\phai cm]current page.north east) rectangle ([yshift=\duoi cm-0.5*\botset cm,xshift=-\mepphai cm+5mm]current page.south east);% cắt chấm
			\node[inner sep =0pt,scale=1,anchor=north] at ([yshift=0cm,xshift=0pt]DDD) {
				\parbox{\leftnote cm}{\centering
					\def\maucham{\maufoot}\dotlineEX{60}
				}
			};
			%--note dưới
			\node[inner sep =6pt, text=white,scale=1,anchor=north,fill=\maufoot] (noteduoi) at ([yshift=2.5mm]DDD) {
				\parbox{\leftnote cm-5mm-12pt}{ \fontsize{11}{1}\fontfamily{qag}\selectfont\bfseries\centering
					QUICK NOTE
				}
			};
			\draw[\maufoot, line width=0.4pt] ([yshift=-2pt]noteduoi.south west)--([yshift=-2pt]noteduoi.south east);
		\end{scope}
\else %chẵn
		%-----đường kẻ
		\draw[\maufoot, line width=2pt] 
		([xshift=\phai cm]AA) --([xshift=-\trai cm]BB);
		%-----bên trái
		\node[text=\maufoot, anchor=south west,inner sep=0pt] at ([xshift=\phai cm,yshift=4pt]AA){
			\fontfamily{qag}\fontsize{8.5pt}{12pt}\selectfont 
			{\color{\mauSO}\faMapMarker}\,\, \diachi\,\,{\color{\mauSO}\faMapMarker}
		};
		%-----bên phải
		\node[text=\maufoot, anchor=south east,inner sep=0pt] at ([xshift=-\trai cm,yshift=4pt]BB){
			\fontfamily{qag}\fontsize{10pt}{10pt}\selectfont\bfseries\faEdit\, \tenchuyende
		};
		%----- Kẻ đứng
		\draw[\maufoot] ([xshift=\mepphai cm-2.5mm]AA)--([yshift=\duoi cm-0.75*\botset cm,xshift=\mepphai cm-2.5mm]current page.south west);
		%--		
		\path ([yshift=-\tren cm+0.5*\topset cm-0.5cm,xshift=\phai cm+0.5*\leftnote cm-2.5mm]current page.north west) coordinate (DDD); 
		\begin{scope}
			\clip ([yshift=-\tren cm+0.5*\topset cm-1pt,xshift=\phai cm]current page.north west) rectangle ([yshift=\duoi cm-0.5*\botset cm,xshift=\mepphai cm-5mm]current page.south west);% cắt chấm
			\node[inner sep =0pt,scale=1,anchor=north] at ([yshift=0cm,xshift=0pt]DDD) {
				\parbox{\leftnote cm}{\centering
					\def\maucham{\maufoot}\dotlineEX{60}
				}
			};
			%--note dưới
			\node[inner sep =6pt, text=white,scale=1,anchor=north,fill=\maufoot] (noteduoi) at ([yshift=2.5mm]DDD) {
				\parbox{\leftnote cm-5mm-12pt}{ \fontsize{11}{1}\fontfamily{qag}\selectfont\bfseries\centering
					QUICK NOTE
				}
			};
			\draw[\maufoot, line width=0.4pt] ([yshift=-2pt]noteduoi.south west)--([yshift=-2pt]noteduoi.south east);
		\end{scope}
\fi
\end{tikzpicture}%
}
% trên mục lục
\newcommand{\headmucluc}{%
	\boldmath
\begin{tikzpicture}[remember picture,overlay,>=stealth]
	\path ([yshift=-\tren cm+0.5*\topset cm]current page.north west) coordinate (AA)
	++(\paperwidth,0)coordinate (BB); 
\checkoddpage\ifoddpage %nếu trang lẻ
	%-----đường kẻ
	\draw[\maufoot, line width=2pt] 
	([xshift=\trai cm]AA) --([xshift=-\phai cm]BB);
	%-----bên phải
	\node[text=\maufoot, anchor=south east,inner sep=0pt] at ([xshift=-\phai cm,yshift=4pt]BB){
		\fontfamily{qag}\fontsize{8.5pt}{12pt}\selectfont 
		{\color{\mauSO}\faMapMarker}\,\, \diachi\,\,{\color{\mauSO}\faMapMarker}
	};
	%-----bên trái
	\node[text=\maufoot, anchor=south west,inner sep=0pt] at ([xshift=\trai cm,yshift=4pt]AA){
		\fontfamily{qag}\fontsize{12pt}{12pt}\selectfont\bfseries\faEdit\, \tenchuyende
	};
\else %chẵn
	%-----đường kẻ
	\draw[\maufoot, line width=2pt] 
	([xshift=\phai cm]AA) --([xshift=-\trai cm]BB);
	%-----bên trái
	\node[text=\maufoot, anchor=south west,inner sep=0pt] at ([xshift=\phai cm,yshift=4pt]AA){
		\fontfamily{qag}\fontsize{8.5pt}{12pt}\selectfont 
		{\color{\mauSO}\faMapMarker}\,\, \diachi\,\,{\color{\mauSO}\faMapMarker}
	};
	%-----bên phải
	\node[text=\maufoot, anchor=south east,inner sep=0pt] at ([xshift=-\trai cm,yshift=4pt]BB){
		\fontfamily{qag}\fontsize{12pt}{12pt}\selectfont\bfseries\faEdit\, \tenchuyende
	};
\fi
\end{tikzpicture}%
}
%===========================
\newcommand{\myfancyfoot}{% dưới
	\begin{tikzpicture}[remember picture,overlay]
	\path ([yshift=\duoi cm-0.75*\botset cm]current page.south west) coordinate (AA)
	++(\paperwidth,0)coordinate (BB); 
	\checkoddpage\ifoddpage %nếu trang lẻ
		%---kẻ
		\draw[\maufoot, line width=2pt] ([xshift=2*\trai cm+4pt]AA)--([xshift=-\phai cm+3pt]BB);
		%-----bên trái
		\fill[fill=\maufoot, rounded corners=2mm] ([xshift=2*\trai cm,yshift=0.25 cm]AA) rectangle +(-3*\trai cm,-0.5cm);
		%-----trang
		\node[anchor=west,text=white,inner sep=0pt,xshift=-0.75cm] at ([xshift=2*\trai cm]AA) {\fontfamily{put}\bfseries\thepage};
		%-----tên tg
		\node[anchor=west,text=\maufoot,inner sep=0pt,fill=white] at ([xshift=2*\trai cm]AA){\fontfamily{qag}\fontsize{9pt}{1pt}\selectfont\bfseries \,\,\, \tentacgia \,\,\, };
	\else %chẵn
		%---kẻ
		\draw[\maufoot, line width=2pt] ([xshift=-2*\trai cm+4pt]BB)--([xshift=\phai cm-3pt]AA);
		%-----bên trái
		\fill[fill=\maufoot, rounded corners=2mm] ([xshift=-2*\trai cm,yshift=0.25 cm]BB) rectangle +(3*\trai cm,-0.5cm);
		%-----trang
		\node[anchor=east,text=white,inner sep=0pt,xshift=0.75cm] at ([xshift=-2*\trai cm]BB) {\fontfamily{put}\bfseries\thepage};
		%-----tên tg
		\node[anchor=east,text=\maufoot,inner sep=0pt,fill=white] at ([xshift=-2*\trai cm]BB){\fontfamily{qag}\fontsize{9pt}{1pt}\selectfont\bfseries \,\,\, \tentacgia \,\,\,};
	\fi
	\end{tikzpicture}%
}
%======================Head chapter theo note, fullwidth
%----------------------
\usepackage{changepage}
\strictpagecheck
\usepackage{lastpage}
\usepackage{fancyhdr,lastpage}
\pagestyle{fancy}
\fancyhf{}
\fancypagestyle{plain}{
	\fancyhead[LO,RE]{\headmucluc}
	\fancyfoot[LO,RE]{\myfancyfoot}
}
\fancyhead[LO,RE]{\myfancyhead}
\fancyfoot[LO,RE]{\myfancyfoot}
\renewcommand{\footrulewidth}{0pt}
\renewcommand{\headrulewidth}{0pt}
%--------------4.2
\usepackage[most]{tcolorbox}
\colorlet{tcbcol@back}{tcbcolback}
\colorlet{tcbcol@frame}{tcbcolframe}
%---------------------------------------------------------------
% ĐỊNH NGHĨA SECTION. SUBSECTION, SUBSUBSECTION ... THEO Ý RIÊNG
%---------------------------------------------------------------
\usepackage[explicit]{titlesec} % để gọi #1
\usepackage{titledot} % gói lệnh chứa cả titlesec và titletoc
%=====================================
\setcounter{secnumdepth}{4} %độ sâu
\renewcommand{\thechapter}{\Roman{chapter}}
\renewcommand{\thesection}{\arabic{section}}
\renewcommand{\thesubsection}{\Alph{subsection}}
\renewcommand{\thesubsubsection}{\arabic{subsubsection}}
%--------------Tròn
\newcommand{\tron}[1]{
	\begin{tikzpicture}[baseline=(A.base)]%
		\node[circle,draw=\mauSUBSEC,line width=0.5pt,fill=white,inner sep=2pt,outer sep=1pt] (A) {\color{white} #1};
		\node[circle,draw=none,fill=\mauSUBSEC,inner sep=1pt,outer sep=1pt] (A) {\color{white} #1};
	\end{tikzpicture}%
}
%================= Đn chương
\font\fontchap=ugqb8v at 21pt
\titlespacing{\chapter}{0cm}{0cm}{0.5cm}[0cm] %1: , 2: Trên, 3: dưới
\titleformat{\chapter}[display]
{\fontsize{20pt}{20pt}\fontfamily{qag}\selectfont\bfseries\color{\mauCHUONG}} %định dạng chung
{\fontsize{16pt}{20pt}\selectfont\chaptername\, \thechapter.} %đánh số
{1mm}
{\fontchap\centering\MakeUppercase{#1}}
[\vspace{0cm}]
%============================Mục lục - Chapter*
\titleformat{name=\chapter,numberless}[display]
{\fontsize{14pt}{16pt}\fontfamily{qag}\selectfont\bfseries\color{\mauCHUONG}} %định dạng chung
{}
{-1em}
{%
	\begin{tikzpicture}
		%-----Nội dung
		\node[inner sep=0pt,right] (ndchuong) at (0,0){\fontchap \MakeUppercase{#1}};
%		%-----Đường kẻ ngang
%		\begin{scope}
%			\clip (0,-0.75) rectangle +(\textwidth,1.5);
%			\draw[\mauchuong,line width=2pt] (ndchuong.south east)++(10pt,8pt) --++(\linewidth,0);
%		\end{scope}
	\end{tikzpicture}
}
[
\vspace{-3mm}
%\thispagestyle{empty}
]
%--------Đn Section---------------------------
%\titlespacing*{\section}{0cm}{0cm}{0cm}[0cm]
\titleformat
{\section}
{\color{\mauSEC}\fontfamily{qag}\fontsize{16pt}{1pt}\selectfont\bfseries\centering}
{Bài\,\thesection.}
{3mm}
{\MakeUppercase{#1}}
[]
%-------Đn subsection---------------------------
\titlespacing{\subsection}{0cm}{0cm}{0cm}[0cm]
\titleformat{\subsection}
{\normalfont\fontsize{15pt}{20pt}\fontfamily{put}\selectfont\bfseries\color{\mausubsec}}
{\thesubsection.}
{3mm}
{\MakeUppercase{#1}}
[]
%----------ĐN subsubsection-----------------------
\titlespacing{\subsubsection}{0pt}{0mm}{0mm}[0cm]
\titleformat{\subsubsection}
{\fontsize{13pt}{18pt}\fontfamily{put}\selectfont\bfseries\color{\mausubsubsec}}
{\thesubsubsection.}
{3mm}
{#1}
[]
%----------ĐN paragraph-----------------------
\titlespacing{\paragraph}{0pt}{0mm}{0mm}[0cm]
\titleformat{\paragraph}
{\fontsize{11.5pt}{17pt}\fontfamily{put}\selectfont\bfseries\color{\mausubsubsec}}
{\theparagraph.}
{3mm}
{#1}
[]
%============================
\def\itemKN{\color{\mauitemKN}\faCheckSquareO}
\def\itemCI{\color{\mauitemCI}\faCheckCircleO}
%%======= Thiết lập labelitem, labelenumerate
%\renewcommand{\labelitemi}{\color{red}\faCheckSquareO}
\renewcommand{\labelitemi}{\color{\mauitem}\faCheckCircleO}
\renewcommand{\labelitemii}{\color{\mauitem}\bf ---}
\renewcommand{\labelitemiii}{\color{\mauitem}\bf +}
\renewcommand{\labelenumi}{\alph{enumi})}
%\renewcommand{\labelenumii}{\color{blue}\bf\arabic{enumi}.\arabic{enumii}}
%============================
%============================
% Canh chỉnh mục lục chính
\setcounter{secnumdepth}{4} %Độ sâu đánh số
\setcounter{tocdepth}{2} %Độ sâu mục lục
\contentsmargin{0cm}
%~~~~~~~~~~~~~~~~~~~~~
\renewcommand*\l@part[2]{%
	\ifnum \c@tocdepth >-2\relax
	\addpenalty{-\@highpenalty}%
	\addvspace{10pt \@plus\p@}%
	\setlength\@tempdima{3em}%
	\begingroup
	\hypersetup{linkcolor=violet}
	\tikz[remember picture, overlay]{
		\fill[\mauPHAN] (0,0) rectangle +(\textwidth,1);
		\draw (0,0.5) node[right=5pt]
		{\color{white}\fontsize{16pt}{1pt}\fontfamily{qag}\selectfont\bfseries  {\scshape Phần} #1};
	}
	\par\smallskip
	\penalty\@highpenalty
	\endgroup
	\fi
}
%------------------------
%\titlecontents{part}[0pc]
%{\addvspace{10pt}%
%	\color{red!70!black}\fontsize{18pt}{1pt}\fontfamily{qag}\selectfont\bfseries 
%}%
%{}
%{}
%{}%
%~~~~~~~~~~~~~~~~~~~~~
\titlecontents{chapter}[6.5pc] % nd cách trái
{\addvspace{5pt}%
	\color{\mauCHUONG}\fontsize{13pt}{16pt}\fontfamily{put}\selectfont\bfseries
}%
{\contentslabel[\chaptertitlename\,\thecontentslabel.]{6.5pc}} %nhãn
{}
{\hfill\bfseries\thecontentspage
}%
[\vspace*{5pt}]
%~~~~~~~~~~~~~~~~~~~~~
\titlecontents{section}[10pc]
{\addvspace{0pt}\bfseries\color{\mauSEC}}
{\fontsize{12.5pt}{15pt}\selectfont\sffamily\contentslabel[{Bài\,\thecontentslabel.}]{3.5pc}}
{}
{\hfill
	\thecontentspage
}
[]
%~~~~~~~~~~~~~~~~~~~~~
\titlecontents{subsection}[10pc]
{\addvspace{0pt}\color{\mauSUBSEC}}
{\fontsize{12pt}{15pt}\selectfont\sffamily\contentslabel[\tron{\thecontentslabel}]{2.8pc}}
{}
{{\tiny\dotfill}\thecontentspage}
[]
%~~~~~~~~~~~~~~~~~~~~~
%--------------------------
% ĐỊNH NGHĨA CÁC MÔI TRƯỜNG 
%--------------------------
\listenumerate{dn,dl,tc,nx,ex}%xuống dòng khi liệt kê
\theoremstyle{plain} %
\theoremheaderfont{\scshape} %đầu
\theorembodyfont{\normalfont} % thân
\theoremseparator {.} % Ngăn cách
\newtheorem{dn}{\color{\maudn}\faBolt\, Định nghĩa}[section]
%===================================ĐNghĩa
\theoremstyle{plain} %
\theoremheaderfont{\fontfamily{put}\bfseries} %đầu
\theorembodyfont{\normalfont} % thân
\theoremseparator {.} % Ngăn cách
\newtheorem{vd}{\color{\mauVD}\damEX
	%\faToggleOn\ 
	%\faUnlink\ 
	VÍ DỤ}%[section]
\newtheorem{bt}{\color{\mauBT}\damEX BÀI}
%===================================
\theoremstyle{plain} %
\theoremheaderfont{\scshape} %đầu
\theorembodyfont{\slshape} % thân 
\theoremseparator {.} % Ngăn cách 
\newtheorem{dl}{\color{\maudl}\faBolt\, Định lí}[section]
\newtheorem{tc}{\color{\mauhq!70!black}\faBolt\, Tính chất}[section]
\newtheorem{hq}{\color{\mauhq!70!black}\faBolt\, Hệ quả}[section]
%====================================
\theoremstyle{nonumberplain} %ko đánh số, ko xuống dòng
\theoremheaderfont{\scshape} %đầu
\theorembodyfont{\normalfont} %phần thân
\theoremseparator {.} %ngăn cách
\newtheorem{nx}{\color{\mauhq!70!black}\faBolt\, Nhận xét}
\newtheorem{tomtat}{\!\!\!\!\!\!\!\!}
%====================================Hộp
%--------------------Chú ý
\newenvironment{note}
{\begin{tcolorbox}
		[enhanced jigsaw,breakable,pad at break*=1mm,
		opacityback=0,boxrule=0pt,frame hidden,
		left=8mm, right=0pt, bottom=0pt, top=0pt,
		before skip=1mm,
		after skip=1mm,
		underlay unbroken and first={
			\draw ([xshift=0.3cm,yshift=-0.32cm]interior.north west) node[\mauly]{\large\bfseries \faExclamationTriangle};
		},
		fontupper=\it,
		]}
	{\end{tcolorbox}}
%\let\mynote\note
%\renewcommand{\note}{\mynote{\bfseries\color{\mauly}Lưu ý:}} 
%---------------Dạng toán
\newcounter{dang}\setcounter{dang}{0}
\renewcommand{\thedang}{\arabic{dang}}
%---Dạng 1
\newtcolorbox{dang}[1]{
	fonttitle=\fontfamily{qag}\bfseries,%fontupper=\itshape,
	colframe=\maudang,colback=yellow!20,coltitle=white,
	sharp corners, breakable, halign title=center,%adjusted title=center, %canh giữa DẠNG
	before skip=2mm,after skip=3mm,
	left=2mm,right=2mm,top=2mm,bottom=2mm,
	boxrule=1pt,
	title={\faFolderOpen\ Dạng~\stepcounter{dang}\thedang.\ #1}
		\addcontentsline{toc}{subsection}{\it\sffamily \faFolderOpen\ Dạng~\thedang.#1}
		\setcounter{subsubsection}{0}
		\setcounter{vd}{0}
		\setcounter{ex}{0}
		\setcounter{bt}{0}
}
%====================
\setlength{\parindent}{0pt} %không thụt đầu dòng
%--Name
\newcounter{deso}
\font\dam=ugqb8v at 13pt
\font\damTT=ugqb8v at 18pt
%================đn notenam
\def\notename{
		\begin{tikzpicture}[remember picture,overlay,>=stealth]
		\checkoddpage\ifoddpage %nếu trang lẻ
		%--tiêu đề phải
		\path ([yshift=-\tren cm+0.5*\topset cm-0.5cm,xshift=-\phai cm-0.5*\leftnote cm+2.5mm]current page.north east) coordinate (DD); 
		%--
		\fill[white] ([yshift=-\tren cm+0.5*\topset cm-5pt,xshift=\trai cm-2pt]current page.north east) rectangle ([yshift=\duoi cm-0.5*\botset cm-12cm,xshift=-\mepphai cm+3mm]current page.north east);
		\node[inner sep =0pt,anchor=north] (thanhta) at ([yshift=-2mm]DD) {
			\includegraphics[width=4.5cm]{logo/logo.jpg}		
		};	
		\else
		%--tiêu đề phải
		\path ([yshift=-\tren cm+0.5*\topset cm-0.5cm,xshift=\phai cm+0.5*\leftnote cm-2.5mm]current page.north west) coordinate (DD); 
		%--
		\fill[white] ([yshift=-\tren cm+0.5*\topset cm-5pt,xshift=\phai cm-2pt]current page.north west) rectangle ([yshift=\duoi cm-0.5*\botset cm-12cm,xshift=\mepphai cm-3mm]current page.north west);
		\node[inner sep =0pt,anchor=north] (thanhta) at ([yshift=-2mm]DD) {
			\includegraphics[width=4.5cm]{logo/logo.jpg}		
		};
		\fi
		%\draw (thanhta.south) node[below=0pt,xscale=0.8]{\small\normalfont\color{\mauname} Sưu tầm \& Biên tập};
		%---note
		\node[inner sep =6pt, text=black,scale=1,anchor=north,fill=\maufoot!3,draw=\maufoot] (bon) at ([yshift=-1cm]thanhta.south) {
			\parbox{\leftnote cm-5mm-12pt}{ \fontsize{10}{15}\selectfont\normalfont
				\vspace*{2pt}
				\chamngon%
			}
		};
		\draw[\maufoot, line width=5pt] (bon.north west)--(bon.north east);
		\draw[\maufoot!50] ([yshift=9pt,line width=0.4pt]bon.north east)--([yshift=9pt]bon.north west)
		node[fill=white,inner sep=2pt,anchor=south west,yshift=-2pt,xshift=-2pt]{\bfseries\color{\maufoot}ĐIỂM:}
		;
		%--note dưới
		\node[inner sep =6pt, text=white,scale=1,anchor=north,fill=\maufoot] (noteduoi) at ([yshift=-0.25cm]bon.south) {
			\parbox{\leftnote cm-5mm-12pt}{ \fontsize{11}{1}\selectfont\bfseries\centering
				QUICK NOTE
			}
		};
		\draw[\maufoot, line width=0.4pt] ([yshift=-2pt]noteduoi.south west)--([yshift=-2pt]noteduoi.south east);
	\end{tikzpicture}
}
%===================note và nonote
%FULL WIDTH
\def\FULLWIDTH{
	\newpage
	\fancyhead[LO,RE]{\headmucluc}
	\def\notename{}
	\newgeometry{top=\tren cm, bottom=\duoi cm, left=\trai cm, right=\phai cm}
}
\def\NOTE{
	\newpage
	\fancyhead[LO,RE]{\myfancyhead}
	\def\notename{
		\begin{tikzpicture}[remember picture,overlay,>=stealth]
			\checkoddpage\ifoddpage %nếu trang lẻ
			%--tiêu đề phải
			\path ([yshift=-\tren cm+0.5*\topset cm-0.5cm,xshift=-\phai cm-0.5*\leftnote cm+2.5mm]current page.north east) coordinate (DD); 
			%--
			\fill[white] ([yshift=-\tren cm+0.5*\topset cm-5pt,xshift=\trai cm-2pt]current page.north east) rectangle ([yshift=\duoi cm-0.5*\botset cm-12cm,xshift=-\mepphai cm+3mm]current page.north east);
			\node[inner sep =0pt,anchor=north] (thanhta) at ([yshift=-2mm]DD) {
				\includegraphics[width=4.5cm]{logo/logo.jpg}		
			};	
			\else
			%--tiêu đề phải
			\path ([yshift=-\tren cm+0.5*\topset cm-0.5cm,xshift=\phai cm+0.5*\leftnote cm-2.5mm]current page.north west) coordinate (DD); 
			%--
			\fill[white] ([yshift=-\tren cm+0.5*\topset cm-5pt,xshift=\phai cm-2pt]current page.north west) rectangle ([yshift=\duoi cm-0.5*\botset cm-12cm,xshift=\mepphai cm-3mm]current page.north west);
			\node[inner sep =0pt,anchor=north] (thanhta) at ([yshift=-2mm]DD) {
				\includegraphics[width=4.5cm]{logo/logo.jpg}		
			};
			\fi
			%\draw (thanhta.south) node[below=0pt,xscale=0.8]{\small\normalfont\color{\mauname} Sưu tầm \& Biên tập};
			%---note
			\node[inner sep =6pt, text=black,scale=1,anchor=north,fill=\maufoot!3,draw=\maufoot] (bon) at ([yshift=-1cm]thanhta.south) {
				\parbox{\leftnote cm-5mm-12pt}{ \fontsize{10}{15}\selectfont\normalfont
					\vspace*{2pt}
					\chamngon%
				}
			};
			\draw[\maufoot, line width=5pt] (bon.north west)--(bon.north east);
			\draw[\maufoot!50] ([yshift=9pt,line width=0.4pt]bon.north east)--([yshift=9pt]bon.north west)
			node[fill=white,inner sep=2pt,anchor=south west,yshift=-2pt,xshift=-2pt]{\bfseries\color{\maufoot}ĐIỂM:}
			;
			%--note dưới
			\node[inner sep =6pt, text=white,scale=1,anchor=north,fill=\maufoot] (noteduoi) at ([yshift=-0.25cm]bon.south) {
				\parbox{\leftnote cm-5mm-12pt}{ \fontsize{11}{1}\selectfont\bfseries\centering
					QUICK NOTE
				}
			};
			\draw[\maufoot, line width=0.4pt] ([yshift=-2pt]noteduoi.south west)--([yshift=-2pt]noteduoi.south east);
		\end{tikzpicture}
	}
	\newgeometry{top=\tren cm, bottom=\duoi cm, left=\trai cm, right=\mepphai cm}
}
%===================đn name
\newcommand{\name}[4]{
%	\NOTE
%	\newpage
	\setcounter{ex}{0}\setcounter{bt}{0}%\setcounter{EX}{0}
	\boldmath\fontfamily{qag}\selectfont\color{\mauname}
\hoten \dotfill {\fontsize{10}{11}\selectfont \ngaylamde}
	\begin{tcolorbox}[boxrule=0.7pt,arc=0mm,breakable,colframe=\mauSO,colback=\mauname!2,before skip=2mm,after skip=2mm]\color{\mauname}
	\begin{center}
		%%---
		{\damTT \MakeUppercase{#1}}\\[1pt]
		{\dam \MakeUppercase{#2 --- Đề} \stepcounter{deso}\thedeso}\\[1pt]
		% {\dam \MakeUppercase{#2}}\\[1pt]
		{\dam\color{\mauSO} \MakeUppercase{#3}}\\[1pt]
		{\fontsize{10}{10}\selectfont \textit{#4}}%\\[-1mm]
	\end{center}
	\end{tcolorbox}
	%%--- Phần note đầu đề
	\notename
\vspace*{0.5cm}
	\addcontentsline{toc}{section}{\hspace*{-4.2cm}\sf Đề \thedeso: #2 --- #3} % đưa MT vào mục lục
}
%--Sang trang 
\BeforeBeginEnvironment{name}{
	\ifnum\the\value{deso}>0
	\newpage
	\fi
}
%%---Đánh số trang
%\AtEndEnvironment{name}{
%	\ifnum\the\value{deso}=1
%	\pagenumbering{arabic}%đánh số trang dạng 1,2,...
%	\fi
%}
%---------
\def\chap#1{
	\begin{center}
		\fontchap\color{\mauCHUONG} #1
	\end{center}
	\addcontentsline{toc}{chapter}{\hspace*{-2.75cm}#1}
}
%---Hiện bảng ĐA
\newcommand{\hienDA}{
	\renewcommand{\indapan}[2]{
		\addcontentsline{toc}{subsection}{\hspace*{-4.2cm}\sf Bảng đáp án} % đưa MT vào mục lục
		%		\begin{center}
		\par\vspace*{5mm}
		\begin{tikzpicture}%
			\draw (0,0)++(0.5*\textwidth,0) node[thick,scale=1,fill=\mauEX!2,draw=\maufoot,minimum width=3.5cm,minimum height=0.1cm,rounded corners=2mm] {\damEX\color{\mauname} BẢNG ĐÁP ÁN};
		\end{tikzpicture}%
		%		\end{center}
		\vspace*{-2mm}
		\inputansbox{##1}{##2}
	}
}
%---Ẩn bảng ĐA
\newcommand{\anDA}{
	\renewcommand{\indapan}[2]{}
}
%---Dòng chấm từng câu
\newcommand{\dongchamEX}[1]{
%	\hideansEX{ex}
	\anLG
	\AfterEndEnvironment{ex}{%
		\foreach \cauEX/\dongEX in {#1}{
			\ifnum\dongEX=0
			\else
			\ifnum\the\value{ex}=\cauEX
			\par\noindent\loigiaiEXS\par
			\dotlineEX{\dongEX}
			\fi
			\fi
		}
	}
}
%---Dòng chấm nhiều câu
\newcommand{\dongchamEXS}[2]{
%	\hideansEX{ex}
	\anLG
	\AfterEndEnvironment{ex}{%
		\foreach \cauEX in {#1}{
			\ifnum#2=0
			\else
			\ifnum\the\value{ex}=\cauEX
			\par\noindent\loigiaiEXS\par
			\dotlineEXS{#2}
			\fi
			\fi
		}
	}
}
%---Dòng chấm từng câu theo đề
\newcommand{\DEdongchamEX}[2]{
%	\hideansEX{ex}
	\anLG
	\AfterEndEnvironment{ex}{%
		\foreach \cauEX/\dongEX in {#2}{
			\ifnum\dongEX=0
			\else
			\ifnum\the\value{deso}=#1
			\ifnum\the\value{ex}=\cauEX
			\par\noindent\loigiaiEXS\par
			\dotlineEX{\dongEX}
			\fi
			\fi
			\fi
		}
	}
}
%---Dòng chấm nhiều câu theo đề
\newcommand{\DEdongchamEXS}[3]{
%	\hideansEX{ex}
	\anLG
	\AfterEndEnvironment{ex}{%
		\foreach \cauEX in {#2}{
			\ifnum#3=0
			\else
			\ifnum\the\value{deso}=#1
			\ifnum\the\value{ex}=\cauEX
			\par\noindent\loigiaiEXS\par
			\dotlineEX{#3}
			\fi
			\fi
			\fi
		}
	}
}
%---Ẩn LG
\newcommand{\anLG}{
	\renewcommand{\loigiai}[1]{	}%
	% \chooseNSA
	\renewcommand{\TrueTF}{\FalseTF}
	\renewcommand{\TrueEX}{\FalseEX}
	\renewcommand{\writekeyTFone}{\gdef\TrueX{}\gdef\FalseX{}}
	\renewcommand{\writekeyTF}{&&}
}
%---Hiện LG
\newcommand{\hienLG}{
	%Xuất hiện chữ Lời giải trong môi trường onlysolution
	\renewcommand{\loigiai}[1]{%
		\begin{onlysolution}%
			##1
		\end{onlysolution}%
	}%
	%---
	\def\loigiaiEXS{}
	%\choiceTF
	\renewcommand{\writekeyTFone}{\gdef\TrueX{}\gdef\FalseX{\tickF}}
	\renewcommand{\writekeyTF}{%
		&\centering\leavevmode\TrueX%
		&\parbox[t]{\linewidth}{\centering\leavevmode\FalseX}%
			\gdef\TrueX{}\gdef\FalseX{\tickF}%
	}
	\def\kindSA{ShowSAKeyColor}
	\showanswers
	% \SAOPTN{kindSA=oly}
	\renewcommand{\dotlineEXS}[1]{}	
}
%=======Đn các phương án
\def\khoanhtrondapan{
	\renewcommand*\circled[1]{\tikz[baseline=(char.base)]{
			\node[shape=circle,draw=\mauDA,inner sep=1pt] (char) {##1};}}
	\renewcommand{\TrueEX}{\stepcounter{dapan}
		{\squareEX{\textbf{\damEX\color{\mauDA}\Alph{dapan}}}} \ignorespaces}
	\renewcommand{\FalseEX}{\stepcounter{dapan}
		{\circled{\textbf{\damEX\color{\mauDA}\Alph{dapan}}}} \ignorespaces}
	%---Chọn đáp án
	\renewcommand{\circEX}[2][fill=\mauTrue!3,draw=\mauTrue]{%
	\tikz[baseline=(char.base)]{\node[shape=circle,inner sep=1pt,##1] (char) {\color{red}##2};}}
}
%----
\def\khongkhoanhtrondapan{
	\renewcommand{\TrueEX}{\stepcounter{dapan}
		{\squareEX{\textbf{\damEX\color{\mauDA}\Alph{dapan}}}} \ignorespaces}
	\renewcommand{\FalseEX}{\stepcounter{dapan}
		{\textbf{\damEX\color{\mauDA}\Alph{dapan}.}} \ignorespaces}
	%---Chọn đáp án
	\renewcommand{\circEX}[2][fill=\mauTrue!3,draw=\mauTrue]{%
	\tikz[baseline=(char.base)]{\node[shape=circle,inner sep=1pt,##1] (char) {\color{red}##2};}}
}
%=============ĐN HIỆN CÂU EX CẦN THIẾT if
\newcommand{\hienEXS}[2]{
	%\foreach \bdem in {#1,...,#2}{%11-30
	\def\biendau{#1}\def\biencuoi{#2}%
	\pgfmathsetmacro{\sodau}{\fpeval{round(\biendau-1,0)}}
	\pgfmathsetmacro{\socuoi}{\fpeval{round(\biencuoi+1,0)}}
	\setcounter{EX}{#1-1}
	\RenewEnviron{ex}{
		\stepcounter{ex}%
		\ifnum\value{ex}<\socuoi
		\ifnum\value{ex}>\sodau
		\par%
		\begin{EX}
			\BODY% 
		\end{EX}
		\fi\fi
	}
	%
	\AtEndEnvironment{name}{\setcounter{EX}{#1-1}}
	%
	\AtEndEnvironment{EX}{
		\ifnum\the\value{numTrue}=1
		\scantokens{\begin{EXsol}A\end{EXsol}}
		\fi
		\ifnum\the\value{numTrue}=2
		\scantokens{\begin{EXsol}B\end{EXsol}}
		\fi
		\ifnum\the\value{numTrue}=3
		\scantokens{\begin{EXsol}C\end{EXsol}}
		\fi
		\ifnum\the\value{numTrue}=4
		\scantokens{\begin{EXsol}D\end{EXsol}}
		\fi
		\setcounter{numTrue}{0}
	}
}
%%%%%%%%%%%%%%%%%%%%%%%

%============================ Khung
\newenvironment{khung}
{\begin{tcolorbox}[
		enhanced,breakable,
		colback=yellow!10,
		colframe=blue,
		boxrule=0.5pt,
		%		drop fuzzy shadow=gray,
		left=5pt,right=5pt,top=5pt,bottom=5pt,
		arc=0mm
		]}
	{\end{tcolorbox}}
%-----------------------------Mục con = subsub
\newcounter{muccon}
\newcommand{\muccon}[1]{%
	\stepcounter{muccon}
	{%\setcounter{bt}{0}\setcounter{vd}{0}\setcounter{ex}{0}
		%\fontsize{13pt}{15pt}\selectfont
		%		\color{violet!70!black}\sffamily
		\bfseries\sffamily\bfseries\hspace*{0mm}\themuccon.\  
		#1}
}
%----------------------------------------------------

% Hộp định nghĩa
\newenvironment{boxdn}
{\begin{tcolorbox}
		[enhanced jigsaw,breakable,pad at break*=1mm,
		colback=cyan!2,
		%standard jigsaw, 
		opacityback=0, %ko nền
		boxrule=0pt,frame hidden, left=0.7cm, right=0pt, bottom=2pt, top=0pt,
		borderline west={1mm}{0.5cm}{cyan},
		overlay={
			\fill[fill=cyan!20,draw=none] ([xshift=0.6cm]interior.north west) rectangle (interior.south east)
			;
		}
		\setcounter{muccon}{0}
		]}%0mm lề trái
	{\end{tcolorbox}}
%===============================================
\theoremstyle{nonumberbreak} % ko đánh số
\theoremheaderfont{\sffamily\bfseries} %tên
\theorembodyfont{\normalfont} %thân
\theoremsymbol{\ensuremath{_\blacksquare}} %Dấu kết thúc là ô vuông đen.
\theoremseparator {:} % Dấu ngăn cách
\newtheorem{myphantich}{\color{violet}%\faServer\ 
	\faFileText\ PHÂN TÍCH}
%===============================================
\newenvironment{phantich}{\begin{boxdn}\begin{myphantich}}{\end{myphantich}\end{boxdn}}
%-------------- Khung (Trong main này ko sd)
\newtcolorbox[auto counter]{khung4}[1]{enhanced, breakable,
	before skip=1mm,after skip=1mm,
	left=1mm,right=1mm,top=2mm,bottom=1mm,
	colframe=myblue,colback=cyan!0,colbacktitle=cyan!6,coltitle=myblue,colupper=black,sharp corners,
	,boxrule=0.4mm,
	coltext=mauE,
	attach boxed title to top center=
	{yshift=-0.1mm-\tcboxedtitleheight/2,yshifttext=2mm-\tcboxedtitleheight/2},
	varwidth boxed title*=-3cm,
	boxed title style={boxrule=0.3mm,
		frame code={ \path[tcb fill frame] ([xshift=-4mm]frame.west)
			-- (frame.north west) -- (frame.north east) -- ([xshift=4mm]frame.east)
			-- (frame.south east) -- (frame.south west) -- cycle; },
		interior code={ \path[tcb fill interior] ([xshift=-2mm]interior.west)
			-- (interior.north west) -- (interior.north east)
			-- ([xshift=2mm]interior.east) -- (interior.south east) -- (interior.south west)
			-- cycle;} 
	},
	fonttitle=\fontsize{10}{0}
	\bfseries,
	fontupper=\fontsize{10}{0},
	title={#1}
}

\newcommand{\boxmini}[1]{
	\vspace*{-2mm}
	\begin{center}
		\begin{tikzpicture}[outline/.style={draw=##1,thick,fill=##1!3},outline/.default=myblue]
			\node [outline,
			sharp corners] at (0,0) {\fontfamily{qag} \selectfont\bfseries\color{\mauEX} #1};
		\end{tikzpicture}
	\end{center}
	\vspace*{0mm}
}

%-----------------------
\newcommand{\inden}[1]{
	{\fontsize{11pt}{9pt}\sffamily \selectfont\bfseries\color{\maudn} #1}
}
\newcommand{\indam}[1]{
	{\fontsize{11.5pt}{9pt}\sffamily \selectfont\bfseries\color{\maudn} #1}
}
\newcommand{\indamm}[1]{
	{\fontsize{11.5pt}{9pt}\sffamily \selectfont\bfseries\color{\maudl} #1}
}
\newcommand{\ind}[1]{
	{\fontsize{11.5pt}{9pt}\sffamily \selectfont\bfseries\color{\maucham} #1}
}
%%%===Các biểu tượng===
\def\iconGN{{\color{magenta}\faPencilSquareO}}
\def\iconNS{{\color{gray}\faStar}}
\def\iconQS{{\color{magenta}\faFolderOpen}}
\def\iconMT{{\color{magenta!80!black}\faSunO}}
\def\iconX{{\color{red}\faClose}}
\def\iconCH{{\color{myblue}\faCheckCircle}}
\def\iconVD{\faCubes}
\def\iconCV{{\color{myblue}\faCubes}}

\newcommand{\dongcham}[1]{
	\def\sod{#1}
	\pgfmathsetmacro{\sodong}{2*\sod -1} 
	\columnsep=10pt
	\vspace*{-3.5mm}
	\begin{multicols}{2}
		\foreach \dotline in{1,...,\sodong}
		{\noindent\color{gray}{\dotfill}\\[1mm]
		}\noindent\color{gray}{\dotfill}\\[-4mm]
	\end{multicols}
}


\def\TNTF{
    {\bfseries Phần II. Trong mỗi ý a), b), c) và d) ở mỗi câu, học sinh chọn đúng hoặc sai.}
}
\def\TN{
    {\bfseries Phần I. Mỗi câu hỏi học sinh chọn một trong bốn phương án A, B, C, D.}
}
\def\TNSA{
    {\bfseries Phần III. Học sinh điền kết quả vào ô trống.}
}
\def\BTTL{
    \begin{center}
        \fcolorbox{black}{white}{{\bfseries BÀI TẬP TỰ LUẬN TRẢ LỜI NGẮN}}
    \end{center}
}
\def\BTTF{
    \begin{center}
        \fcolorbox{black}{white}{{\bfseries BÀI TẬP TRẮC NGHIỆM ĐÚNG SAI}}
    \end{center}
%    \TNTF
}
\def\BTTN{
    \begin{center}
        \fcolorbox{black}{white}{{\bfseries BÀI TẬP TRẮC NGHIỆM 4 PHƯƠNG ÁN}}
    \end{center}
}
\def\TL{
    {\bfseries Phần II. Câu hỏi tự luận.}
}
 %Khai báo cơ bản
\usepackage{tkz-euclide,circuitikz}
%%%%%%%%%%%%% ĐIỀU KHIỂN LỜI GIẢI ,DÒNG CHẤM, ĐÁP SỐ
%------------Dòng chấm bằng chiều dài LG (bật)
% \dotlinefull{ex}
%------------Thay Loi giải bằng n dòng kẻ (bật)
% \dotlineans{2}{ex}
%------------Ẩn lời giải
%\hideansEX{ex}
%------------Dòng chấm tùy ý (ko cần \loigiai{})
%---Nhiều câu cùng dòng chấm (tách dụng lên mọi đề)
%\dongchamEXS{1,...,20}{2}
%\dongchamEXS{21,...,40}{5}
%\dongchamEXS{41,...,50}{10}
%---Nhiều câu cùng dòng chấm (tách dụng lên MỘT ĐỀ đc chọn)
%\DEdongchamEXS{3}{1,...,20}{2} %{3} là đề thứ 3
%---Dòng chấm từng câu, tác dụng lên mọi đề
%\dongchamEX{1/3,2/5,3/7} % câu / số dòng chấm của câu đó
%---Dòng chấm từng câu, tác dụng lên 1 đê
%\DEdongchamEX{3}{1/3,2/5,3/7} % câu / số dòng chấm của câu đó, {3} là đề số 3 
\renewcommand{\dongcham}[1]{}
%------------Ẩn đáp số (bật), đáp án
\exitdapso %ẩn đs
%\renewcommand{\indapan}[2]{} %ẩn đáp án
%%%%%%%%%%%%% khung NAME
\def\ngaylamde{Ngày làm đề: ...../...../........} %để {} nếu ko muốn
\def\tenchude{HÀM SỐ LƯỢNG GIÁC}
% \def\tendethi{ }
\def\tentruong{PHedu}
\def\thoigian{Thời gian làm bài: 90 phút, không kể thời gian phát đề}
%%%%%%%%%%%%% Nội dung head & foot
% \def\diachi{ }
\def\diachi{VNPmath - 0962940819}
\def\tenchuyende{\tenchude}
\def\tentacgia{GV.VŨ NGỌC PHÁT}
\def\chamngon{\lq\lq It's not how much time you have, it's how you use it.\rq\rq
}
%%%%%%%%%%%%% Đn lại A.B.C.D
\khoanhtrondapan
% \khongkhoanhtrondapan
%%%%%%%%%%%%%
\renewcommand{\arraystretch}{1}
\newcommand{\viduminhhoa}{\subsubsection{Ví dụ minh hoạ}}
\newcommand{\baitaptl}{\subsubsection{Bài tập tự luận}}

\newenvironment{boxdl}
{\begin{tcolorbox}
		[enhanced jigsaw,breakable,pad at break*=1mm,
		boxrule=0pt,frame hidden, left=2mm, right=0pt, bottom=1.5pt, top=1.5pt,
		before skip=2mm,
		after skip=2mm,
		%		borderline west={1mm}{0cm}{green!70!black}
		overlay={\draw[double,line width=1.5pt,\maudl] ([xshift=3pt]interior.north west)--([xshift=3pt]interior.south west);}
		]}%0mm lề trái
	{\end{tcolorbox}}
%%===================================================
%=================BẮT ĐẦU TÀI LIỆU===================

\begin{document}
\renewcommand{\chaptername}{Chương}
\pagenumbering{arabic}%đánh số trang dạng 1,2,...
%====================================================
%==================BẮT ĐẦU TÀI LIỆU==================
%\hienEXS{41}{50} %chỉ hiện câu từ 41 đến 50 của đề
%--------Đề bài
\NOTE \anLG \anDA 
\notename

%%Bài 1
% 
\section{Giá trị lượng giác của một góc lượng giác}
\subsection{Tóm tắt lý thuyết}
\begin{tomtat}
	\subsubsection{Khái niệm góc lượng giác và số đo của góc lượng giác}
	Trong mặt phẳng, cho hai tia $Ou$, $Ov$. Xét tia $Om$ cùng nằm trong mặt phẳng này. Nếu tia $Om$ quay quanh điểm $O$, theo một chiều nhất định từ $Ou$ đến $Ov$, thì ta nói nó quét một góc lượng giác với tia đầu  $Ou$, tia cuối $Ov$ và kí hiệu là ($Ou$, $Ov$).\\
	Mỗi góc lượng giác gốc $O$ được xác định bởi tia đầu $Ou$, tia cuối $Ov$ và số đo của nó.
	\begin{center}
		\begin{minipage}[H]{0.3\textwidth}
			\begin{tikzpicture}[scale=.7]	
				\draw (0,0) -- (4,0)node[below] {$u$};
				\draw[red] (0,0) -- (45:4)node[below right] {$v$};
				\draw[dashed,green!50!black] (0,0) -- (20:4)node[below right] {$m$};
				\draw[-stealth,red] (0:1) arc (0:45:1);
				\draw[-stealth] (2.75,1) arc (0:45:1);
				\path (30:3.5) node[below=-2pt]{$+$};
				\path (0:0) node[below left]{$O$};
			\end{tikzpicture}
		\end{minipage}
		\begin{minipage}[H]{0.3\textwidth}
			\begin{tikzpicture} [scale=.7]	
				\draw (0,0) -- (4,0)node[below] {$u$};
				\draw[red] (0,0) -- (45:4)node[below right] {$v$};
				\draw[dashed,green!50!black] (0,0) -- (75:3.5)node[below right] {$m$};
				\draw[red,-stealth,smooth,samples=100] plot[domain =0:2.25*pi]({.5*(1.1)^(\x) *cos(\x r)},{.5*(1.1)^(\x) *sin(\x r)});
				\draw[-stealth] (0.5,2) arc (75:110:2);
				\path (90:2.5) node[below=-2pt]{$+$};
				\path (0:0) node[below left]{$O$};
			\end{tikzpicture}
		\end{minipage}
		\begin{minipage}[H]{0.3\textwidth}
			\begin{tikzpicture}[scale=.7]		
				\draw (0,0) -- (4,0)node[below] {$u$};
				\draw[red] (0,0) -- (45:4)node[below right] {$v$};
				\draw[dashed,green!50!black] (0,0) -- (120:3)node[above right] {$m$};
				\draw[-stealth,red] (0:.8) arc (0:-315:.8);
				\draw[-stealth] (-1,1.7) arc (120:60:1);
				\path (105:2.4) node[below=-2pt]{$-$};
				\path (0:0) node[below left]{$O$};
			\end{tikzpicture}
		\end{minipage}
	\end{center}
	\subsubsection{Hệ thức Chasles}
	\immini{Hệ thức Chasles: Với ba tia $Ou$, $Ov$, $Ow$ bất kì, ta có 	
	$$
		\text{sđ}(Ou, Ov)+\text{sđ}(Ov,Ow)=\text{sđ}(Ou,Ow)+k 360^{\circ}(k \in \mathbb{Z}). 
		$$}
	{\begin{tikzpicture}[scale=0.77, font=\footnotesize, line join=round, line cap=round, >=stealth]		
			\draw (0,0) -- (4,0)node[below] {$u$};
			\draw[red] (0,0) -- (75:3.5)node[below right] {$w$};
			\draw[green!50!black] (0,0) -- (30:4)node[below right] {$v$};
			\draw[-stealth,red] (0:1) arc (0:-285:1);
			\draw[-stealth] (0:.9) arc (0:30:.9);
			\draw[-stealth] (.86,.5) arc (30:75:1);
			\path (0:0) node[below left]{$O$};
	\end{tikzpicture}}
	% Nhận xét. Từ hệ thức Chasles, ta suy ra:
	% Với ba tia tuỳ ý $Ox$, $Ou$, $Ov$ ta có
	% $$
	% \text{sđ}(Ou, Ov)=\text{sđ}(Ox, Ov)-\text{sđ}(Ox,Ou)+k360^{\circ}(k \in \mathbb{Z}). 
	% $$
	% Hệ thức này đóng vai trò quan trọng trong việc tính toán số đo của góc lượng giác.
	\subsubsection{Đơn vị đo góc và cung tròn}
	\textbf{Đơn vị độ}: Góc $1^{\circ}$ bằng $\dfrac{1}{180}$ góc bẹt.\\
	Đơn vị độ được chia thành những đơn vị nhỏ hơn: $1^{\circ}=60'; 1'=60"$.\\
	% Đối với các góc lượng giác, khi mà số vòng quay trong chuyển động tương ứng từ tia đầu đến tia cuối là khá lớn thì số đo của chúng tính bằng độ sẽ trở nên cồng kềnh. Do đó, trong khoa học và kĩ thuật, bên cạnh việc đo bằng độ, người ta còn sử dụng đơn vị đo góc bằng rađian.\\
	\immini{\textbf{Đơn vị rađian}: Cho đường tròn $(O)$ tâm $O$, bán kính $R$ và một cung $AB$ trên $(O)$.
		Ta nói cung tròn $AB$ có số đo bằng 1 rađian nếu độ dài của nó đúng bằng bán kính $R$.
		Khi đó ta cũng nói rằng góc $AOB$ có số đo bằng 1 rađian và viết: $\overset\frown{AOB}=1$ rad.}
	{\begin{tikzpicture}[scale=0.77, font=\footnotesize, line join=round, line cap=round, >=stealth]		
			\draw[green!50!black] (30:2) arc (30:-270:2);
			\draw[red] (30:2) arc (30:90:2);
			\draw (90:2)node[above]{$B$}--(0:0)--(30:2)node[above right]{$A$};
			\path (0:0) node[below left]{$O$};
			\draw(60:2) node[above right]{1 rad};
	\end{tikzpicture}}
	\textbf{Quan hệ giữa độ và rađian:}
	$$
	1 \text{ góc bẹt }=180^\circ = 1 \mathrm{rad} \Leftrightarrow  1^\circ=\dfrac{\pi}{180} \mathrm{rad} \quad \text { và }\quad 1\,  \mathrm{rad}=\left(\dfrac{180}{\pi}\right)^\circ.
	$$
	\begin{note}
		Khi viết số đo của một góc theo đơn vị rađian, người ta thường không viết chữ rad sau số đo. Chẳng hạn góc $\dfrac{\pi}{2}$ được hiểu là góc $\dfrac{\pi}{2}$ rad.
	\end{note}
	\begin{note}
		Dưới đây là bảng tương ứng giữa số đo bằng độ và số đo bằng rađian của các góc đặc biệt trong phạm vi từ $0^{\circ}$ đến $180^{\circ}$.
	\end{note}
	\begin{center}
		\renewcommand{\arraystretch}{2}
		\begin{tabular}{|l|c|c|c|c|c|c|c|c|c|}
			\hline Độ & $0^{\circ}$ & $30^{\circ}$ & $45^{\circ}$ & $60^{\circ}$ & $90^{\circ}$ & $120^{\circ}$ & $135^{\circ}$ & $150^{\circ}$ & $180^{\circ}$ \\
			\hline Rađian & 0 & $\dfrac{\pi}{6}$ & $\dfrac{\pi}{4}$ & $\dfrac{\pi}{3}$ & $\dfrac{\pi}{2}$ & $\dfrac{2 \pi}{3}$ & $\dfrac{3 \pi}{4}$ & $\dfrac{5 \pi}{6}$ & $\pi$ \\
			\hline
		\end{tabular}
	\end{center}
	\subsubsection{Độ dài cung tròn}
	Một cung của đường tròn bán kính $R$ và có số đo $\alpha$ rad thì có độ dài $l=R \alpha$.
	\subsubsection{Đường tròn lượng giác}
	\immini{\begin{itemize}
			\item Đường tròn lượng giác là đường tròn có tâm tại gốc toạ độ, bán kính bằng $1$, được định hướng và lấy điểm $A(1 ; 0)$ làm điểm gốc của đường tròn.
			\item Điểm trên đường tròn lượng giác biểu diễn góc lượng giác có số đo $\alpha$ là điểm $M$ trên đường tròn lượng giác sao cho sđ$(OA, OM)=\alpha$.
	\end{itemize}
	\begin{note}
		Góc $\alpha$ và $\beta$ có chung điểm biểu diễn khi \fbox{$\alpha - \beta = k2\pi$} (chẵn lần $\pi$)
		\end{note}}
	{
		\begin{tikzpicture}[line join = round, line cap = round, >=stealth, font=\footnotesize, scale=0.6]
			\tikzset{label style/.style={font=\footnotesize}}
			\path (0,0) coordinate (O)
			(3,0) coordinate (A)
			(0,3) coordinate (B)
			(0,-3) coordinate (B')
			(-3,0) coordinate (A')
			(0:0)++(150:3) coordinate (M)
			($(O)!(M)!(A')$) coordinate (H)
			($(O)!(M)!(B)$) coordinate (K)
			;
			\draw[->] (-4,0) -- (4,0) node[above,blue]{$x$};
			\draw[->] (0,-4) -- (0.,4) node[left,blue]{$y$};
			\draw[orange] (O) circle (3cm);
			\draw[rotate=0,->,green!50!black] (0.5,0) arc (0:150:0.5cm);
			\draw (0.35,0.25) node[above,blue] {$\alpha$};
			\draw[dashed] (H)--(M)--(K);
			\draw[green!50!black] (M)--(O);
			\foreach \p/\r in {A/-45,M/150,H/-90,O/-150,A'/-135,B'/-45,B/45,K/0}
			\fill (\p) circle (1pt) node[shift={(\r:3mm)},blue]{$\p$};
		\end{tikzpicture}
	}
	\subsubsection{Các giá trị lượng giác của góc lượng giác}
	\immini{Gọi $M(x;y)$ là điểm biểu diễn của góc lượng giác $\alpha$ trên đường tròn lượng giác. Khi đó, ta có:
		\begin{itemize}
			\item $\cos\alpha=x.$
			\item $\sin\alpha=y.$
			\item $\tan\alpha=\dfrac{\sin\alpha}{\cos\alpha}=\dfrac{y}{x} ~(x\neq0).$
		\item $\cot\alpha=\dfrac{\cos\alpha}{\sin\alpha}=\dfrac{x}{y} ~(y\neq0).$
	\end{itemize}}
	{\begin{tikzpicture}[line join = round, line cap = round, >=stealth, font=\footnotesize, scale=0.6]
			\tikzset{label style/.style={font=\footnotesize}}
			\path (0,0) coordinate (O)
			(3,0) coordinate (A)
			(0,3) coordinate (B)
			(0,-3) coordinate (B')
			(-3,0) coordinate (A')
			(0:0)++(40:3) coordinate (M)
			($(O)!(M)!(A')$) coordinate (H)
			($(O)!(M)!(B)$) coordinate (K)
			;
			\draw[->] (-4,0) -- (4,0) node[above,blue]{$x$};
			\draw[->] (0,-4) -- (0.,4) node[left,blue]{$y$};
			\draw[orange] (O) circle (3cm);
			\draw[rotate=0,->,green!50!black] (0.7,0) arc (0:40:0.7cm);
			\draw (1,0) node[above,blue] {$\alpha$};
			\draw[dashed] (H)--(M)--(K);
			\draw[green!50!black] (M)--(O);
			\draw[blue,fill=black] (0,2) node[left]{$\sin\alpha$}(2,0) circle(1pt) node[below]{$\cos\alpha$}(3,2.3) node{$M(x;y)$};
			\foreach \p/\r in {A/-45,O/-135,A'/-135,B'/-45,B/45}
			\fill (\p) circle (1pt) node[shift={(\r:3mm)},blue]{$\p$};
	\end{tikzpicture}}
	\begin{note}
		a) Ta còn gọi trục tung là trục sin, trục hoành là trục côsin.\\
		b) Từ định nghĩa ta suy ra:
		\begin{itemize}
			\item $\sin\alpha$, $\cos\alpha$ xác định với mọi giá trị của $\alpha$ và ta có:
			$$-1\leq \sin\alpha\leq 1; \quad -1\leq \cos\alpha\leq 1; \quad \sin(\alpha+k2\pi)=\sin\alpha;\quad \cos(\alpha+k2\pi)=\cos\alpha\,\, (k\in\mathbb{Z}).$$
			\item $\tan\alpha$ xác định khi $\alpha\neq\dfrac{\pi}{2}+k\pi\,\,  (k\in\mathbb{Z})$.
			\item $\cot\alpha$ xác định khi $\alpha\neq k\pi\,\,  (k\in\mathbb{Z})$.
			\item Dấu của các giá trị lượng giác của một góc lượng giác phụ thuộc vào vị trí điểm biểu diễn $M$ trên đường tròn lượng giác.
		\end{itemize}
	\end{note}
	\begin{minipage}[h]{0.6\textwidth}
		\begin{tabular}{c|c|c|c|c|}
			\cline{2-5}
			& \multicolumn{4}{c|}{Góc phần tư} \\ \hline
			\multicolumn{1}{|c|}{Giá trị lượng giác} & I     & II     & III     & IV    \\ \hline
			\multicolumn{1}{|c|}{$\sin \alpha$}     &   $+$    &  $ +$      &    $-$    &   $-$  \\ \hline
			\multicolumn{1}{|c|}{$\cos \alpha$}     &   $+$    &  $ -$      &    $-$    &   $+$  \\ \hline
			\multicolumn{1}{|c|}{$\tan \alpha$}     &   $+$    &  $ -$      &    $+$    &   $-$  \\ \hline
			\multicolumn{1}{|c|}{$\cot \alpha$}     &   $+$    &  $ -$      &    $+$    &   $-$  \\ \hline
		\end{tabular}
	\end{minipage}
	\begin{minipage}[h]{0.6\textwidth}
		\begin{tikzpicture}[line join = round, line cap = round, >=stealth, font=\footnotesize, scale=0.6]
			\tikzset{label style/.style={font=\footnotesize}}
			\path (0,0) coordinate (O)
			(3,0) coordinate (A)
			(0,3) coordinate (B)
			(0,-3) coordinate (B')
			(-3,0) coordinate (A')
			(0:0)++(-60:3) coordinate (M)
			($(O)!(M)!(A')$) coordinate (H)
			($(O)!(M)!(B)$) coordinate (K)
			;
			\draw[->] (-4,0) -- (4,0) node[above,blue]{$x$};
			\draw[->] (0,-4) -- (0.,4) node[left,blue]{$y$};
			\draw[orange] (O) circle (3cm);
			\draw[rotate=0,->,green!50!black] (0.5,0) arc (0:-60:0.5cm);
			\draw (0.75,-0.35) node[blue] {$\alpha$};
			\draw[dashed] (H)--(M)--(K);
			\draw[green!50!black] (M)--(O);
			\draw[blue] (2.5,2.5) node{$I$}(-2.5,2.5) node{$II$}(-2.5,-2.5) node{$III$}(2.5,-2.5) node{$IV$};
			\foreach \p/\r in {A/-45,M/-60,H/90,O/-150,A'/-135,B'/-45,B/45,K/180}
			\fill (\p) circle (1pt) node[shift={(\r:3mm)},blue]{$\p$};
		\end{tikzpicture}
	\end{minipage}
	% \subsubsection{Giá trị lượng giác của các góc đặc biệt}
	% \begin{center}
	% 	\renewcommand{\arraystretch}{2}
	% 	\begin{tabular}{|c|c|c|c|c|c|}
	% 		\hline
	% 		\multirow{2}{*}{Góc $\alpha$} & $0$              & $\dfrac{\pi}{6}$  & $\dfrac{\pi}{4}$  & $\dfrac{\pi}{3}$  & $\dfrac{\pi}{2}$              \\ \cline{2-6} 
	% 		& $0^\circ$              & $30^\circ$  & $45^\circ$  & $60^\circ$  & $90^\circ$             \\ \hline
	% 		$\sin\alpha$                  & $0$             & $\dfrac{1}{2}$ & $\dfrac{\sqrt{2}}{2}$ & $\dfrac{\sqrt{3}}{2}$ & 1              \\ \hline
	% 		$\cos\alpha$                  & $1$             & $\dfrac{\sqrt{3}}{2}$ & $\dfrac{\sqrt{2}}{2}$ & $\dfrac{1}{2}$ & 0              \\ \hline
	% 		$\tan\alpha$                 & $0$             & $\dfrac{1}{\sqrt{3}}$ & 1 & $\sqrt{3}$ & Không xác định \\ \hline
	% 		$\cot\alpha$                  & Không xác định & $\sqrt{3}$ & 1 & $\dfrac{1}{\sqrt{3}}$ & 0              \\ \hline
	% 	\end{tabular}
	% \end{center}
	\subsubsection{Các công thức lượng giác cơ bản}
	Đối với các giá trị lượng giác, ta có các hệ thức cơ bản sau
	\begin{enumEX}[$\bullet$]{2}
		\item $\sin^2 \alpha  + \cos^2 \alpha =1$
		\item $ 1+ \tan^2 \alpha= \dfrac{1}{\cos^2 \alpha}$ $\left(\alpha \neq \dfrac{\pi}{2}+k\pi , k\in \mathbb{Z}\right)$
		\item $ 1+ \cot^2 \alpha= \dfrac{1}{\sin^2 \alpha}$ $\left(\alpha \neq k\pi , k\in \mathbb{Z}\right)$
		\item $\tan \alpha \cdot \cot \alpha =1 $ $\left(\alpha \neq \dfrac{k\pi}{2}, k\in \mathbb{Z}\right)$
	\end{enumEX}
	\newpage
	\subsubsection{Giá trị lượng giác của các góc có liên quan đặc biệt}
	\begin{enumerate}
		\item Góc đối nhau ($\alpha$ và $-\alpha$)
		\immini{\begin{itemize}
				\item $\cos (-\alpha)=\cos \alpha$
				\item $\sin (-\alpha) =-\sin \alpha$
				\item $\tan (-\alpha) =-\tan \alpha$
				\item $\cot (-\alpha) =-\cot \alpha$
		\end{itemize}}
		{\vspace*{-1cm}\begin{tikzpicture}[line join = round, line cap = round, >=stealth, font=\footnotesize, scale=0.5]
				\tikzset{label style/.style={font=\footnotesize}}
				\path (0,0) coordinate (O)
				(3,0) coordinate (A)
				(0:0)++(120:3) coordinate (M)
				(0:0)++(-120:3) coordinate (N)
				(0,4) coordinate (C)
				(0,-4) coordinate (D)
				($(O)!(M)!(C)$) coordinate (E)
				($(O)!(N)!(D)$) coordinate (F)
				;
				\draw[->] (-4,0) -- (4,0) node[above,blue]{$x$};
				\draw[->] (0,-4) -- (0.,4) node[left,blue]{$y$};
				\draw[orange] (O) circle (3cm);
				\draw[rotate=0,->,red] (0.5,0) arc (0:120:0.5cm);
				\draw[rotate=0,->,green!50!black] (0.6,0) arc (0:-120:0.6cm);
				\draw (0,0) node[above right=2pt,blue] {$\alpha$} (0,-0) node[below right=2pt,blue]{$-\alpha$};
				\draw[dashed] (E)--(M)--(N)--(F);
				\draw[green!50!black] (M)--(O);
				\draw[red] (N)--(O);
				\foreach \p/\r in {A/-45,M/120,N/-120,O/-150}
				\fill (\p) circle (1pt) node[shift={(\r:3mm)},blue]{$\p$};
		\end{tikzpicture}}
		\item Góc bù nhau ($\alpha$ và $\pi-\alpha$)
		\immini{\begin{itemize}
				\item $\sin (\pi -\alpha)=\sin \alpha$
				\item $\cos (\pi -\alpha) =-\cos \alpha$
				\item $\tan (\pi -\alpha) =-\tan \alpha$
				\item $\cot (\pi -\alpha) =-\cot \alpha$
		\end{itemize}}
		{\vspace*{-0.5cm}\begin{tikzpicture}[line join = round, line cap = round, >=stealth, font=\footnotesize, scale=0.5]
				\tikzset{label style/.style={font=\footnotesize}}
				\path (0,0) coordinate (O)
				(3,0) coordinate (A)
				(0:0)++(30:3) coordinate (M)
				(0:0)++(150:3) coordinate (N)
				(4,0) coordinate (C)
				(-4,0) coordinate (D)
				($(O)!(M)!(C)$) coordinate (E)
				($(O)!(N)!(D)$) coordinate (F)
				;
				\draw[->] (-4,0) -- (4,0) node[above,blue]{$x$};
				\draw[->] (0,-4) -- (0.,4) node[left,blue]{$y$};
				\draw[orange] (O) circle (3cm);
				\draw[rotate=0,->,red] (0.5,0) arc (0:150:0.5cm);
				\draw[rotate=0,->,green!50!black] (1.6,0) arc (0:30:1.6cm);
				\draw (2,0) node[above,blue] {$\alpha$} (0.3,1.5) node[below,blue]{$\pi-\alpha$};
				\draw[dashed] (E)--(M)--(N)--(F);
				\draw[red] (O)--(N);
				\draw[green!50!black] (O)--(M);
				\foreach \p/\r in {A/-45,M/30,N/150,O/-130}
				\fill (\p) circle (1pt) node[shift={(\r:3mm)},blue]{$\p$};
		\end{tikzpicture}}
		\item Góc phụ nhau ($\alpha$ và $\dfrac{\pi}{2}-\alpha$)
		\immini{\begin{itemize}
				\item $\sin \left( \dfrac{\pi}{2}-\alpha\right)=\cos \alpha$
				\item $\cos \left( \dfrac{\pi}{2}-\alpha\right)=\sin \alpha$
				\item $\tan \left( \dfrac{\pi}{2}-\alpha\right)=\cot \alpha$
				\item $\cot \left( \dfrac{\pi}{2}-\alpha\right)=\tan \alpha$
		\end{itemize}}
		{\vspace*{-0.5cm}\begin{tikzpicture}[line join = round, line cap = round, >=stealth, font=\footnotesize, scale=0.5]
				\tikzset{label style/.style={font=\footnotesize}}
				\path (0,0) coordinate (O)
				(3,0) coordinate (A)
				(0:0)++(20:3) coordinate (M)
				(0:0)++(70:3) coordinate (N)
				(0,4) coordinate (C)
				(4,0) coordinate (D)
				($(O)!(M)!(C)$) coordinate (E)
				($(O)!(N)!(D)$) coordinate (F)
				(2.82,0) coordinate (G)
				(0,2.82) coordinate (H)
				;
				\draw[->] (-4,0) -- (4,0) node[above,blue]{$x$};
				\draw[->] (0,-4) -- (0.,4) node[left,blue]{$y$};
				\draw[orange] (O) circle (3cm);
				\draw[rotate=0,->,red] (0.7,0) arc (0:70:0.7cm);
				\draw[rotate=0,->,green!50!black] (1.6,0) arc (0:20:1.6cm);
				\draw (2,0) node[above,blue] {$\alpha$} (1,-.2) node[below,blue]{$\frac{\pi}{2}-\alpha$};
				\draw[dashed] (E)--(M) (F)--(N) (G)--(M) (H)--(N);
				\draw[dashed] (-3,-3)--(3,3);
				\draw[->] (0.8,-.5)--(0.5,0.45);
				\draw[red] (O)--(N);
				\draw[green!50!black] (O)--(M);
				\foreach \p/\r in {A/-45,M/20,N/70,O/-220}
				\fill (\p) circle (1pt) node[shift={(\r:3mm)},blue]{$\p$};
		\end{tikzpicture}}
		\item Góc hơn kém $\pi$ ($\alpha$ và $\pi+\alpha$)
		\immini{\begin{itemize}
				\item $\sin (\pi +\alpha)=-\sin \alpha$
				\item $\cos (\pi +\alpha)=-\cos \alpha$
				\item $\tan (\pi +\alpha)=\tan \alpha$
				\item $\cot (\pi +\alpha)=\cot \alpha$
		\end{itemize}}
		{\vspace*{-0.5cm}\begin{tikzpicture}[line join = round, line cap = round, >=stealth, font=\footnotesize, scale=0.5]
				\tikzset{label style/.style={font=\footnotesize}}
				\path (0,0) coordinate (O)
				(3,0) coordinate (A)
				(0:0)++(60:3) coordinate (M)
				(0:0)++(240:3) coordinate (N)
				;
				\draw[->] (-4,0) -- (4,0) node[above,blue]{$x$};
				\draw[->] (0,-4) -- (0.,4) node[left,blue]{$y$};
				\draw[orange] (O) circle (3cm);
				\draw[rotate=0,->,red] (1.7,0) arc (0:240:1.7cm);
				\draw[rotate=0,->,green!50!black] (0.6,0) arc (0:60:0.6cm);
				\draw (1,0) node[above,blue] {$\alpha$};
				\draw (-1.2,1) node[below,blue,rotate=60]{$\pi+\alpha$};
				\draw[red] (O)--(N);
				\draw[green!50!black] (O)--(M);
				\foreach \p/\r in {A/-45,M/60,N/240,O/-150}
				\fill (\p) circle (1pt) node[shift={(\r:3mm)},blue]{$\p$};
		\end{tikzpicture}}
	\end{enumerate}
\end{tomtat}
 \foreach \i in {1,2,...,7} {\input{data/11KNTT/data1/1K1-2-\i.tex}}

%%Bài 2
% %\chapter{Hàm số  lượng giác và phương trình lượng giác}
\setcounter{section}{1}
\section{Công thức lượng giác}
\subsection{Tóm tắt lý thuyết}
\begin{tomtat}
% 	\begin{center}
% 		\begin{tikzpicture}[scale = 2.5]
% 			\path (0,0) coordinate (O) (1.5,0) coordinate (x) (0,1.5) coordinate (y);
% 			\draw[thick,->] (-1.5,0)--(x);
% 			\draw[thick,->] (0,-1.5)--(y);
% 			\draw (O) circle (1);
% 			\path ($(O)+(55:1)$) coordinate (M) 
% 			($(O)+(30:1)$) coordinate (N);
% 			\path ($(O)!(M)!(x)$) coordinate (x_M)
% 			($(O)!(M)!(y)$) coordinate (y_M)
% 			($(O)!(N)!(x)$) coordinate (x_N)
% 			($(O)!(N)!(y)$) coordinate (y_N);
% 			\draw[dashed] (x_M)--(M)--(y_M) (x_N)--(N)--(y_N);
% 			\foreach \x/\g in {O/-135,x/-90,y/180,x_M/-90,x_N/-90,y_M/180,y_N/180}
% 			\fill ($(\g:1mm)+(\x)$) node {$\x$};
% 			\fill 	(M) circle (0.5pt)
% 			($(15:4mm)+(M)$) node {$M\left(x_M,y_M\right)$};
% 			\fill (N) circle (0.5pt)
% 			($(15:4mm)+(N)$) node {$N\left(x_N,y_N\right)$};
% 	\draw (M)--(O)--(N);		
% 	\draw pic[draw,,angle radius=6mm,->,red]{angle=x--O--M};
% 	\fill[red] (45:3mm) node {$\alpha$};
% 	\draw pic[red,draw,,angle radius=10mm,->]{angle=x--O--N};
% 	\fill[red] (15:5mm) node {$\beta$};
% 		\end{tikzpicture}
% 	\end{center}
% Trong mặt phẳng $Oxy$ cho hai điểm $M,N$ trên đường tròn lượng giác.\\ Đặt $\alpha = \text{sđ} (Ox,OM), \beta = \text{sđ} (Ox,ON)$, ta có $M(\cos \alpha,\sin \alpha)$ và $N(\cos \beta, \sin \beta)$. Khi đó ta tính được $\overrightarrow{OM}.\overrightarrow{ON}$ bằng hai cách
% \begin{align*}
% 	\overrightarrow{OM}.\overrightarrow{ON}&=\left|\overrightarrow{OM}\right|.\left|\overrightarrow{ON}\right|.\cos \left(\overrightarrow{OM},\overrightarrow{ON}\right) = \cos (\alpha-\beta),\\
% 	\overrightarrow{OM}.\overrightarrow{ON} &= x_Mx_N+y_My_N= \cos \alpha \cos\beta +\sin\alpha\sin\beta.
% \end{align*}
% Từ đó dẫn tới công thức
% \begin{align*}
% 	\cos (\alpha-\beta) = \cos \alpha \cos \beta + \sin \alpha\sin \beta \tag{$\star$}
% \end{align*}
% Tất cả các công thức trong bài học được xây dựng dựa trên công thức $(\star)$.\\
% Trong suốt bài học, khi không nói gì thêm, chỉ xét các góc lượng giác mà trong đó giá trị lượng giác được để cập có nghĩa. 
	\subsubsection{Công thức cộng}
	\begin{khung4}{Công thức cộng}
	\begin{tasks}[style=itemize](2)
		\task $\cos (a-b) = \cos a \cos b + \sin a\sin b$.
		\task $\cos (a+b) = \cos a \cos b - \sin a\sin b$.
		\task $\sin (a-b) = \sin a \cos b - \sin b \cos a$.
		\task $\sin (a+b) = \sin a \cos b + \sin b \cos a$.
		\task $\tan (a-b) = \dfrac{\tan a - \tan b}{1+\tan a \tan b}$.
		\task $\tan (a+b) = \dfrac{\tan a + \tan b}{1-\tan a \tan b}$.
	\end{tasks}
	\end{khung4}
	\subsubsection{Công thức nhân đôi}
	Công thức nhân đôi được xây dựng bằng cách thay $b=a$ trong công thức cộng.
	\begin{khung4}{Công thức nhân đôi}
		\begin{tasks}[style=itemize]
			\task $\sin 2a = 2\sin a \cos a$.
			\task $\cos 2a = \cos^2a-\sin^2a = 2\cos^2a-1 = 1-2\sin^2a$.
			\task $\tan 2a = \dfrac{2\tan a}{1-\tan^2a}$.
		\end{tasks}
		\end{khung4}
\begin{note}
	Từ công thức nhân đôi, ta có công thức hạ bậc:
\end{note}
	\begin{khung4}{Công thức hạ bậc}
		\begin{tasks}[style=itemize](3)
			\task $\sin^2a= \dfrac{1-\cos 2a}{2}$.
		\task $\cos^2a = \dfrac{1+\cos 2a}{2}$.
		\task $\tan^2a=\dfrac{1-\cos2a}{1+\cos 2a}$.
		\end{tasks}
		\end{khung4}

\begin{note}
	Áp dụng công thức cộng cho $3a = a +2a$, ta có công thức nhân ba:
\end{note}
	\begin{khung4}{Công thức nhân ba}
		\begin{tasks}[style=itemize](2)
			\task $\sin3a= 3\sin a -4\sin^3a$.
		\task $\cos3a= 4\cos^3a-3\cos a$.
		\task $\tan3a = \dfrac{3\tan a - \tan^3 a}{1-3\tan^2a}$.
		\end{tasks}
		\end{khung4}

	\subsubsection{Công thức biến đổi tích thành tổng}
	\begin{khung4}{Công thức tích thành tổng}
		\begin{tasks}[style=itemize]
			\task $\cos a \cos b = \dfrac{1}{2}\left[\cos (a-b) + \cos (a+b)\right]$.
		\task $\sin a \sin b = \dfrac{1}{2}\left[\cos (a-b)-\cos(a+b)\right]$.
		\task $\sin a \cos b = \dfrac{1}{2}\left[\sin (a-b)+\sin (a+b)\right]$.
		\end{tasks}
		\end{khung4}
	\subsubsection{Công thức biến đổi tổng thành tích}
	Công thức biến đổi tổng thành tích được xây dựng bằng cách $a=\dfrac{a+b}{2}, b = \dfrac{a-b}{2}$ trong công thức biến đổi tích thành tổng.
	\begin{khung4}{Công thức tổng thành tích}
		\begin{tasks}[style=itemize](2)
			\task $\cos a+ \cos b = 2\cos\dfrac{a+b}{2}\cos \dfrac{a-b}{2}$.
		\task $\cos a- \cos b = -2\sin\dfrac{a+b}{2}\sin \dfrac{a-b}{2}$.
		\task $\sin a+ \sin b = 2\sin\dfrac{a+b}{2}\cos \dfrac{a-b}{2}$.
		\task $\sin a -\sin b = 2\cos\dfrac{a+b}{2}\sin \dfrac{a-b}{2}$.
		\end{tasks}
		\end{khung4}
\end{tomtat}

 \foreach \i in {1,2,...,5} {\input{data/11KNTT/data1/1K1-2-\i.tex}}

%%Bài 3
% \setcounter{section}{2}
\section{Hàm số lượng giác}
\subsection{Tóm tắt lý thuyết}
\begin{tomtat}
	% \subsubsection{Định nghĩa hàm số lượng giác}
	% \begin{dn}
	% 	\begin{itemize}
	% 		\item Hàm số sin $y=\sin x$ có tập xác định là $\mathbb{R}$.
	% 		\item Hàm số cos $y=\cos x$ có tập xác định là $\mathbb{R}$.
	% 		\item Hàm số tan  $y=\tan x$ có tập xác định là $\mathbb{R} \setminus\left\{\dfrac{\pi}{2}+k \pi \Big| k \in \mathbb{Z} \right\}$.
	% 		\item  Hàm số cot $y=\cot x$ có tập xác định là $\mathbb{R} \setminus \left\{k \pi \Big| k \in \mathbb{Z} \right\}$.
	% 	\end{itemize}
	% \end{dn}
	\subsubsection{Hàm số chẵn, hàm số lẻ}
	\begin{dn}
	Cho hàm số $y=f(x)$ có tập xác định là $\mathscr{D}$.
	\begin{itemize}
		\item Hàm số $f(x)$ được gọi là \textbf{hàm số chẵn} nếu $\forall x \in \mathscr{D}$ thì $-x \in \mathscr{D}$ và $f(-x)=f(x)$. Đồ thị của một hàm số chẵn nhận trục tung là trục đối xứng.
		\item Hàm số $f(x)$ được gọi là \textbf{hàm số lẻ} nếu $\forall x \in \mathscr{D}$ thì $-x \in \mathscr{D}$ và $f(-x)=-f(x)$. Đồ thị của một hàm số lẻ nhận gốc toạ độ là tâm đối xứng.
	\end{itemize}
	\end{dn}
	\subsubsection{Hàm số tuần hoàn}
	\begin{dn}
		Hàm số $y=f(x)$ có tập xác định $\mathscr{D}$ được gọi là \textbf{hàm số tuần hoàn} nếu tồn tại số $T \neq 0$ sao cho với mọi $x \in \mathscr{D}$ ta có:
		\begin{enumerate}[i)]
			\item $x+T \in \mathscr{D}$ và $x-T \in \mathscr{D}$;
			\item $f(x+T)=f(x)$.
		\end{enumerate}
		Số $T$ dương nhỏ nhất thỏa mãn các điều kiện trên (nếu có) được gọi là \textbf{chu kì} của hàm số tuần hoàn đó.
	\end{dn}
	\begin{nx}
		\
		\begin{itemize}
			\item  Các hàm số $y=\sin x$ và $y=\cos x$ tuần hoàn với chu kì $2 \pi$. Các hàm số $y=\tan x$ và $y=\cot x$ tuần hoàn với chu kì $\pi$.
		\end{itemize}
	\end{nx}
	\begin{note} 
		Tổng quát, người ta chứng minh được các hàm số $y=A \sin \omega x$ và $y=A \cos \omega x$ $(\omega>0)$ là những hàm số tuần hoàn với chu kì \fbox{$T=\dfrac{2 \pi}{\omega}$}.
	\end{note}
	\subsubsection{Đồ thị và tính chất của hàm số $y=\sin x$}
	\begin{tc}
		Hàm số $y=\sin x$:
		\begin{itemize}
			\item   Có tập xác định là $\mathbb{R}$ và tập giá trị là $[-1 ; 1]$;
			\item   Là hàm số lẻ và tuần hoàn với chu kì $2 \pi$;
			\item    Đồng biến trên mỗi khoảng $\left(-\dfrac{\pi}{2}+k 2 \pi ; \dfrac{\pi}{2}+k 2 \pi\right)$ và nghịch biến trên mỗi khoảng \\
			$\left(\dfrac{\pi}{2}+k 2 \pi ; \dfrac{3 \pi}{2}+k 2 \pi\right)$, $k \in \mathbb{Z}$;
			\item    Có đồ thị đối xứng qua gốc toạ độ và gọi là một \textbf{đường hình sin}.
		\end{itemize}
	\begin{center}
		\begin{tikzpicture}[>=stealth,scale=0.7,transform shape] 
			\path
			({-2.5*pi},0) coordinate (X1)
			({-2*pi},0) coordinate (X2)
			({-1.5*pi},0) coordinate (X3)
			({-pi},0) coordinate (X4)
			({-0.5*pi},0) coordinate (X5)
			(0,0) coordinate (O)
			({0.5*pi},0) coordinate (X6)
			({pi},0) coordinate (X7)
			({1.5*pi},0) coordinate (X8)
			({2*pi},0) coordinate (X9)
			({2.5*pi},0) coordinate (X10)
			({-pi},-2) coordinate (A)
			({pi},-2) coordinate (B)
			;
			\draw[->] (-9.5,0) -- (9.5,0) node[below] {\small $x$};
			\draw[->] (0,-1.5) -- (0,1.8) node[right] {\small $y$};
			\draw [dotted] (X3)--({-1.5*pi},1)--({2.5*pi},1)--({2.5*pi},0)  ({0.5*pi},1)--({0.5*pi},0)
			(X1)--({-2.5*pi},-1)--({1.5*pi},-1)--({1.5*pi},0)  ({-0.5*pi},-1)--({-0.5*pi},0)
			({-pi},0) -- (A) ({pi},0) -- (B);
			\foreach \x/\g/\z in {X1/90/-\tfrac{5\pi}{2},X2/140/-2\pi,X3/-90/-\tfrac{3\pi}{2},X4/-135/-\pi,X5/90/-\tfrac{\pi}{2},X6/-90/\tfrac{\pi}{2},X7/60/\pi,X8/90/\tfrac{3\pi}{2},X9/-40/2\pi,X10/-90/\tfrac{5\pi}{2}} 
			\fill[black] (\x) circle(1pt) +(\g:5mm) node {$\z$};
			\draw [<->] ({-pi},-1.7)--({pi},-1.7) ; 
			\draw (0,0) node[below right]{$O$} (0,-1.7) node[below]{$T=2\pi$}
			(0,1) node[above right]{$1$} (0,-1) node[below right]{$-1$};
			\clip (-9.5,-1.4) rectangle (9.5,1.6) ;
			\draw[thick,samples=100,domain=-9.3:9.3] plot(\x,{sin((\x)*180/pi)});
			
		\end{tikzpicture}
	\end{center}
	\end{tc}
	\subsubsection{Đồ thị và tính chất của hàm số $y=\cos x$}
	\begin{tc}
		Hàm số $y=\cos x$:
		\begin{itemize}
			\item    Có tập xác định là $\mathbb{R}$ và tập giá trị là $[-1 ; 1]$;
			\item    Là hàm số chẵn và tuần hoàn với chu kì $2 \pi$;
			\item    Đồng biến trên mỗi khoảng $(-\pi+k 2 \pi ; k 2 \pi)$ và nghịch biến trên mỗi khoảng $(k 2 \pi ; \pi+k 2 \pi), k \in \mathbb{Z}$;
			\item    Có đồ thị là một đường hình sin đối xứng qua trục tung.
		\end{itemize}
	\begin{center}
		\begin{tikzpicture}[>=stealth,scale=0.77,transform shape] 
			\path
			({-2.5*pi},0) coordinate (X1)
			({-2*pi},0) coordinate (X2)
			({-1.5*pi},0) coordinate (X3)
			({-pi},0) coordinate (X4)
			({-0.5*pi},0) coordinate (X5)
			(0,0) coordinate (O)
			({0.5*pi},0) coordinate (X6)
			({pi},0) coordinate (X7)
			({1.5*pi},0) coordinate (X8)
			({2*pi},0) coordinate (X9)
			({2.5*pi},0) coordinate (X10)
			({-pi},-2) coordinate (A)
			({pi},-2) coordinate (B)
			;
			\draw[->] (-9.5,0) -- (9.5,0) node[below] {\small $x$};
			\draw[->] (0,-1.5) -- (0,1.8) node[right] {\small $y$};
			\draw [dotted] (X2)--({-2*pi},1)--({2*pi},1)--({2*pi},0) (X4)--({-pi},-1)--({pi},-1)--({pi},0) 
			({-pi},0) -- (A) ({pi},0) -- (B);
			\foreach \x/\g/\z in {X1/125/-\tfrac{5\pi}{2},X2/-90/-2\pi,X3/-120/-\tfrac{3\pi}{2},X4/90/-\pi,X5/110/-\tfrac{\pi}{2},X6/-120/\tfrac{\pi}{2},X7/90/\pi,X8/120/\tfrac{3\pi}{2},X9/-90/2\pi,X10/-120/\tfrac{5\pi}{2}} 
			\fill[black] (\x) circle(1pt) +(\g:5mm) node {$\z$};
			\draw [<->] ({-pi},-1.7)--({pi},-1.7) ; 
			\draw (0,0) node[below right]{$O$} (0,-1.7) node[below]{$T=2\pi$}
			(0,1) node[above right]{$1$} (0,-1) node[below right]{$-1$}
			;
			\clip (-9.5,-1.4) rectangle (9.5,1.6) ;
			\draw[thick,samples=100,domain=-9.3:9.3] plot(\x,{cos((\x)*180/pi)});
			
		\end{tikzpicture}
	\end{center}
	\end{tc}
	\subsubsection{Đồ thị và tính chất của hàm số $y=\tan x$}
	\begin{tc}
		Hàm số $y=\tan x$:
		\begin{itemize}
			\item    Có tập xác định là $\mathbb{R} \setminus\left\{\dfrac{\pi}{2}+k \pi \Big| k \in \mathbb{Z} \right\}$ và tập giá trị là $\mathbb{R}$;
			\item    Là hàm số lẻ và tuần hoàn với chu kì $\pi$;
			\item    Đồng biến trên mỗi khoảng $\left(-\dfrac{\pi}{2}+k \pi ; \dfrac{\pi}{2}+k \pi\right)$, $k \in \mathbb{Z}$;
			\item    Có đồ thị đối xứng qua gốc toạ độ.
		\end{itemize}
	\begin{center}
		\begin{tikzpicture}[>=stealth,scale=0.77,transform shape] 
			\path
			({-1.5*pi},0) coordinate (X1)
			({-pi},0) coordinate (X2)
			({-0.5*pi},0) coordinate (X3)
			({0.5*pi},0) coordinate (X4)
			({pi},0) coordinate (X5)
			({1.5*pi},0) coordinate (X6)
			;
			\draw[->] (-6.5,0) -- (6.5,0) node[below] {\small $x$};
			\draw[->] (0,-3.5) -- (0,3.5) node[right] {\small $y$};
			\draw [dashed] ({-3*pi/2},3.5)--({-3*pi/2},-3.5) ({-pi/2},3.5)--({-pi/2},-3.5) ({pi/2},3.5)--({pi/2},-3.5) ({3*pi/2},3.5)--({3*pi/2},-3.5) ;
			\foreach \x/\g/\z in {X1/-150/-\tfrac{3\pi}{2},X2/-40/-\pi,X3/-40/-\tfrac{\pi}{2},X4/-40/\tfrac{\pi}{2},X5/-400/\pi,X6/-40/\tfrac{3\pi}{2}} 
			\fill[black] (\x) circle(1pt) +(\g:5mm) node {$\z$};
			\draw (0,0) node[below right]{$O$};
			\clip (-6.5,-3.5) rectangle (6.5,3.5) ;
			\draw[thick,samples=100,domain={-pi/2+0.2}:{pi/2-0.2}] plot(\x,{tan((\x)*180/pi)});
			\draw[thick,samples=100,domain={pi/2+0.2}:{3*pi/2-0.2}] plot(\x,{tan((\x)*180/pi)});
			\draw[thick,samples=100,domain={-3*pi/2+0.2}:{-pi/2-0.2}] plot(\x,{tan((\x)*180/pi)});
		\end{tikzpicture}
	\end{center}
	\end{tc}
	\subsubsection{Đồ thị và tính chất của hàm số $y=\cot x$}
	\begin{tc}
		Hàm số $y=\cot x$:
		\begin{itemize}
			\item    Có tập xác định là $\mathbb{R} \setminus\{k \pi \mid k \in \mathbb{Z} \}$ và tập giá trị là $\mathbb{R}$; 
			\item    Là hàm số lẻ và tuần hoàn với chu kì $\pi$;
			\item    Nghịch biến trên mỗi khoảng $(k \pi ; \pi+k \pi), k \in \mathbb{Z}$;
			\item    Có đồ thị đối xứng qua gốc toạ độ.
		\end{itemize}
	\begin{center}
		\begin{tikzpicture}[>=stealth,scale=0.77,transform shape] 
			\path
			({-2*pi},0) coordinate (X1)
			({-1.5*pi},0) coordinate (X2)
			({-pi},0) coordinate (X3)
			({-0.5*pi},0) coordinate (X4)
			({0.5*pi},0) coordinate (X5)
			({pi},0) coordinate (X6)
			({1.5*pi},0) coordinate (X7)
			({2*pi},0) coordinate (X8)
			;
			\draw[->] (-7.5,0) -- (7.5,0) node[below] {\small $x$};
			\draw[->] (0,-3.5) -- (0,3.5) node[left] {\small $y$};
			\draw [dashed] ({-2*pi},3.5)--({-2*pi},-3.5) ({-pi},3.5)--({-pi},-3.5) ({pi},3.5)--({pi},-3.5) ({2*pi},3.5)--({2*pi},-3.5) ;
			\foreach \x/\g/\z in {X1/-150/-2\pi,X2/-130/-\tfrac{3\pi}{2},X3/-140/-\pi,X4/-100/-\tfrac{\pi}{2},X5/-100/\tfrac{\pi}{2},X6/-120/\pi,X7/-100/\tfrac{3\pi}{2}, X8/-120/2\pi}
			\fill[black] (\x) circle(1pt) +(\g:5mm) node {$\z$};
			\draw (0,0) node[below left]{$O$};
			\clip (-6.5,-3.5) rectangle (6.5,3.5) ;
			\draw[thick,samples=100,domain={0.2}:{pi-0.2}] plot(\x,{cot((\x)*180/pi)});
			\draw[thick,samples=100,domain={pi+0.2}:{2*pi-0.2}] plot(\x,{cot((\x)*180/pi)});
			\draw[thick,samples=100,domain={-0.2}:{-pi+0.2}] plot(\x,{cot((\x)*180/pi)});
			\draw[thick,samples=100,domain={-pi-0.2}:{-2*pi+0.2}] plot(\x,{cot((\x)*180/pi)});
		\end{tikzpicture}
	\end{center}
	\end{tc}
\end{tomtat}
\subsection{Các dạng toán thường gặp}
 \foreach \i in {1,2,...,5} {\input{data/11KNTT/data1/1K1-3-\i.tex}}

%%Bài 4

\setcounter{section}{3}
\section{Phương trình lượng giác cơ bản}
\subsection{Tóm tắt lý thuyết}
\begin{tomtat}
\subsubsection{Phương trình $\sin x=m$}
\begin{itemize}
	\item Với $|m|>1$ thì phương trình $\sin x=m$ vô nghiệm.
	\item Với $|m|\leq 1$, sẽ tồn tại duy nhất $\alpha \in \left[-\dfrac{\pi}{2}; \dfrac{\pi}{2}\right]$ thỏa mãn $\sin\alpha=m$. Khi đó
	\begin{center}
		$\sin x=m\Leftrightarrow\sin x=\sin\alpha\Leftrightarrow\hoac{&x=\alpha+k2\pi\\&x=\pi-\alpha+k2\pi}$ ($k\in \mathbb{Z}$).
	\end{center}
\item Nếu số đo của góc $\alpha$ được đo bằng đơn vị độ thì
\begin{center}
	$\sin x=\sin\alpha^\circ\Leftrightarrow\hoac{&x=\alpha^\circ+k360^\circ\\&x=180^\circ-\alpha^\circ+k360^\circ}$ ($k\in\mathbb{Z}$).
\end{center}
\item Tổng quát,
\begin{center}
	$\sin f(x)=\sin g(x)\Leftrightarrow\hoac{&f(x)=g(x)+k2\pi\\&f(x)=\pi - g(x)+k2\pi}$ ($k\in\mathbb{Z}$).
\end{center}
\item Một số trường hợp đặt biệt:
\begin{enumEX}[\faCheckCircleO]{1}
	\item $\sin x=0\Leftrightarrow x=k\pi$, $k\in\mathbb{Z}$.
	\item $\sin x=1\Leftrightarrow x=\dfrac{\pi}{2}+k2\pi$, $k\in\mathbb{Z}$.
	\item $\sin x=-1\Leftrightarrow x=-\dfrac{\pi}{2}+k2\pi$, $k\in \mathbb{Z}$.
\end{enumEX}
\end{itemize}
	\subsubsection{Phương trình $\cos x=m$}
\begin{itemize}
	\item Với $|m|>1$ thì phương trình $\cos x=m$ vô nghiệm.
	\item Với $|m|\leq 1$, sẽ tồn tại duy nhất $\alpha\in\left[0; \pi\right]$ thỏa mãn $\cos x=m$. Khi đó
	\begin{center}
		$\cos x=m\Leftrightarrow\cos x=\cos \alpha\Leftrightarrow\hoac{&x=\alpha+k2\pi\\&x=-\alpha+k2\pi}$ ($k\in\mathbb{Z}$).
	\end{center}
\item Nếu số đo của góc $\alpha$ được đo bằng đơn vị độ thì
\begin{center}
	$\cos x=\cos\alpha\Leftrightarrow\hoac{&x=\alpha^\circ+k360^\circ\\&x=-\alpha^\circ+k360^\circ}$ ($k\in\mathbb{Z}$).
\end{center}
\item Tổng quát,
\begin{center}
	$\cos f(x)=\cos g(x)\Leftrightarrow\hoac{&f(x)=g(x)+k2\pi\\&f(x)=-g(x)+k2\pi}$ ($k\in \mathbb{Z}$)
\end{center}
\item Một số trường hợp đặc biệt:
\begin{enumEX}[\faCheckCircleO]{1}
\item $\cos x=0\Leftrightarrow x=\dfrac{\pi}{2}+k\pi$, $k\in\mathbb{Z}$.
\item $\cos x=1\Leftrightarrow x=k2\pi$, $k\in\mathbb{Z}$.
\item $\cos x=-1\Leftrightarrow x=\pi+k2\pi$, $k\in\mathbb{Z}$.
\end{enumEX}
\end{itemize}
\subsubsection{Phương trình $\tan x=m$}
\begin{itemize}
	\item Với mọi $m\in\mathbb{R}$, tồn tại duy nhất $\alpha\in\left(-\dfrac{\pi}{2};\dfrac{\pi}{2}\right)$ thỏa mãn $\tan \alpha=m$. Khi đó
	\begin{center}
		$\tan x=m\Leftrightarrow\tan x=\tan \alpha\Leftrightarrow x=\alpha+k\pi$ ($k\in\mathbb{Z}$).
	\end{center}
\item Nếu số đo của góc $\alpha$ được đo bằng đơn vị độ thì
\begin{center}
	$\tan x=\tan\alpha^\circ\Leftrightarrow x=\alpha^\circ+k180^\circ$, $k\in \mathbb{Z}$
\end{center}
\item Tổng quát,
\begin{center}
	$\tan f(x)=\tan g(x)\Leftrightarrow f(x)=g(x)+k\pi$, $k\in\mathbb{Z}$.
\end{center}
\end{itemize}
\subsubsection{Phương trình $\cot x=m$}
\begin{itemize}
	\item Với mọi $m\in\mathbb{R}$, tồn tại duy nhất $\alpha\in\left(0;\pi\right)$ thỏa mãn $\cot\alpha=m$. Khi đó
	\begin{center}
		$\cot x=m\Leftrightarrow\cot x=\cot\alpha\Leftrightarrow x=\alpha+k\pi$ $k\in\mathbb{Z}$.
	\end{center}
\item Nếu số đo của góc $\alpha$ được đo bằng đơn vị độ thì
\begin{center}
	$\cot x=\cot\alpha^\circ\Leftrightarrow x=\alpha^\circ+k180^\circ$, $k\in\mathbb{Z}$.
\end{center}
\item Tổng quát,
\begin{center}
	$\cot f(x)=\cot g(x)\Leftrightarrow f(x)=g(x)+k\pi$, $k\in\mathbb{Z}$.
\end{center}
\end{itemize}
\end{tomtat}

 \foreach \i in {2,3,4,5,6,7,11} {\input{data/11KNTT/data1/1K1-4-\i.tex}}


% %--------Lời giải chi tiết
% \FULLWIDTH \hienLG \anDA 
% % % --
% \setcounter{deso}{0}
% \chap{LỜI GIẢI CHI TIẾT}
% \setcounter{section}{0} 
% \setcounter{ex}{0} 
% \setcounter{dang}{0}
%%Bài 1
% 
\section{Giá trị lượng giác của một góc lượng giác}
\subsection{Tóm tắt lý thuyết}
\begin{tomtat}
	\subsubsection{Khái niệm góc lượng giác và số đo của góc lượng giác}
	Trong mặt phẳng, cho hai tia $Ou$, $Ov$. Xét tia $Om$ cùng nằm trong mặt phẳng này. Nếu tia $Om$ quay quanh điểm $O$, theo một chiều nhất định từ $Ou$ đến $Ov$, thì ta nói nó quét một góc lượng giác với tia đầu  $Ou$, tia cuối $Ov$ và kí hiệu là ($Ou$, $Ov$).\\
	Mỗi góc lượng giác gốc $O$ được xác định bởi tia đầu $Ou$, tia cuối $Ov$ và số đo của nó.
	\begin{center}
		\begin{minipage}[H]{0.3\textwidth}
			\begin{tikzpicture}[scale=.7]	
				\draw (0,0) -- (4,0)node[below] {$u$};
				\draw[red] (0,0) -- (45:4)node[below right] {$v$};
				\draw[dashed,green!50!black] (0,0) -- (20:4)node[below right] {$m$};
				\draw[-stealth,red] (0:1) arc (0:45:1);
				\draw[-stealth] (2.75,1) arc (0:45:1);
				\path (30:3.5) node[below=-2pt]{$+$};
				\path (0:0) node[below left]{$O$};
			\end{tikzpicture}
		\end{minipage}
		\begin{minipage}[H]{0.3\textwidth}
			\begin{tikzpicture} [scale=.7]	
				\draw (0,0) -- (4,0)node[below] {$u$};
				\draw[red] (0,0) -- (45:4)node[below right] {$v$};
				\draw[dashed,green!50!black] (0,0) -- (75:3.5)node[below right] {$m$};
				\draw[red,-stealth,smooth,samples=100] plot[domain =0:2.25*pi]({.5*(1.1)^(\x) *cos(\x r)},{.5*(1.1)^(\x) *sin(\x r)});
				\draw[-stealth] (0.5,2) arc (75:110:2);
				\path (90:2.5) node[below=-2pt]{$+$};
				\path (0:0) node[below left]{$O$};
			\end{tikzpicture}
		\end{minipage}
		\begin{minipage}[H]{0.3\textwidth}
			\begin{tikzpicture}[scale=.7]		
				\draw (0,0) -- (4,0)node[below] {$u$};
				\draw[red] (0,0) -- (45:4)node[below right] {$v$};
				\draw[dashed,green!50!black] (0,0) -- (120:3)node[above right] {$m$};
				\draw[-stealth,red] (0:.8) arc (0:-315:.8);
				\draw[-stealth] (-1,1.7) arc (120:60:1);
				\path (105:2.4) node[below=-2pt]{$-$};
				\path (0:0) node[below left]{$O$};
			\end{tikzpicture}
		\end{minipage}
	\end{center}
	\subsubsection{Hệ thức Chasles}
	\immini{Hệ thức Chasles: Với ba tia $Ou$, $Ov$, $Ow$ bất kì, ta có 	
	$$
		\text{sđ}(Ou, Ov)+\text{sđ}(Ov,Ow)=\text{sđ}(Ou,Ow)+k 360^{\circ}(k \in \mathbb{Z}). 
		$$}
	{\begin{tikzpicture}[scale=0.77, font=\footnotesize, line join=round, line cap=round, >=stealth]		
			\draw (0,0) -- (4,0)node[below] {$u$};
			\draw[red] (0,0) -- (75:3.5)node[below right] {$w$};
			\draw[green!50!black] (0,0) -- (30:4)node[below right] {$v$};
			\draw[-stealth,red] (0:1) arc (0:-285:1);
			\draw[-stealth] (0:.9) arc (0:30:.9);
			\draw[-stealth] (.86,.5) arc (30:75:1);
			\path (0:0) node[below left]{$O$};
	\end{tikzpicture}}
	% Nhận xét. Từ hệ thức Chasles, ta suy ra:
	% Với ba tia tuỳ ý $Ox$, $Ou$, $Ov$ ta có
	% $$
	% \text{sđ}(Ou, Ov)=\text{sđ}(Ox, Ov)-\text{sđ}(Ox,Ou)+k360^{\circ}(k \in \mathbb{Z}). 
	% $$
	% Hệ thức này đóng vai trò quan trọng trong việc tính toán số đo của góc lượng giác.
	\subsubsection{Đơn vị đo góc và cung tròn}
	\textbf{Đơn vị độ}: Góc $1^{\circ}$ bằng $\dfrac{1}{180}$ góc bẹt.\\
	Đơn vị độ được chia thành những đơn vị nhỏ hơn: $1^{\circ}=60'; 1'=60"$.\\
	% Đối với các góc lượng giác, khi mà số vòng quay trong chuyển động tương ứng từ tia đầu đến tia cuối là khá lớn thì số đo của chúng tính bằng độ sẽ trở nên cồng kềnh. Do đó, trong khoa học và kĩ thuật, bên cạnh việc đo bằng độ, người ta còn sử dụng đơn vị đo góc bằng rađian.\\
	\immini{\textbf{Đơn vị rađian}: Cho đường tròn $(O)$ tâm $O$, bán kính $R$ và một cung $AB$ trên $(O)$.
		Ta nói cung tròn $AB$ có số đo bằng 1 rađian nếu độ dài của nó đúng bằng bán kính $R$.
		Khi đó ta cũng nói rằng góc $AOB$ có số đo bằng 1 rađian và viết: $\overset\frown{AOB}=1$ rad.}
	{\begin{tikzpicture}[scale=0.77, font=\footnotesize, line join=round, line cap=round, >=stealth]		
			\draw[green!50!black] (30:2) arc (30:-270:2);
			\draw[red] (30:2) arc (30:90:2);
			\draw (90:2)node[above]{$B$}--(0:0)--(30:2)node[above right]{$A$};
			\path (0:0) node[below left]{$O$};
			\draw(60:2) node[above right]{1 rad};
	\end{tikzpicture}}
	\textbf{Quan hệ giữa độ và rađian:}
	$$
	1 \text{ góc bẹt }=180^\circ = 1 \mathrm{rad} \Leftrightarrow  1^\circ=\dfrac{\pi}{180} \mathrm{rad} \quad \text { và }\quad 1\,  \mathrm{rad}=\left(\dfrac{180}{\pi}\right)^\circ.
	$$
	\begin{note}
		Khi viết số đo của một góc theo đơn vị rađian, người ta thường không viết chữ rad sau số đo. Chẳng hạn góc $\dfrac{\pi}{2}$ được hiểu là góc $\dfrac{\pi}{2}$ rad.
	\end{note}
	\begin{note}
		Dưới đây là bảng tương ứng giữa số đo bằng độ và số đo bằng rađian của các góc đặc biệt trong phạm vi từ $0^{\circ}$ đến $180^{\circ}$.
	\end{note}
	\begin{center}
		\renewcommand{\arraystretch}{2}
		\begin{tabular}{|l|c|c|c|c|c|c|c|c|c|}
			\hline Độ & $0^{\circ}$ & $30^{\circ}$ & $45^{\circ}$ & $60^{\circ}$ & $90^{\circ}$ & $120^{\circ}$ & $135^{\circ}$ & $150^{\circ}$ & $180^{\circ}$ \\
			\hline Rađian & 0 & $\dfrac{\pi}{6}$ & $\dfrac{\pi}{4}$ & $\dfrac{\pi}{3}$ & $\dfrac{\pi}{2}$ & $\dfrac{2 \pi}{3}$ & $\dfrac{3 \pi}{4}$ & $\dfrac{5 \pi}{6}$ & $\pi$ \\
			\hline
		\end{tabular}
	\end{center}
	\subsubsection{Độ dài cung tròn}
	Một cung của đường tròn bán kính $R$ và có số đo $\alpha$ rad thì có độ dài $l=R \alpha$.
	\subsubsection{Đường tròn lượng giác}
	\immini{\begin{itemize}
			\item Đường tròn lượng giác là đường tròn có tâm tại gốc toạ độ, bán kính bằng $1$, được định hướng và lấy điểm $A(1 ; 0)$ làm điểm gốc của đường tròn.
			\item Điểm trên đường tròn lượng giác biểu diễn góc lượng giác có số đo $\alpha$ là điểm $M$ trên đường tròn lượng giác sao cho sđ$(OA, OM)=\alpha$.
	\end{itemize}
	\begin{note}
		Góc $\alpha$ và $\beta$ có chung điểm biểu diễn khi \fbox{$\alpha - \beta = k2\pi$} (chẵn lần $\pi$)
		\end{note}}
	{
		\begin{tikzpicture}[line join = round, line cap = round, >=stealth, font=\footnotesize, scale=0.6]
			\tikzset{label style/.style={font=\footnotesize}}
			\path (0,0) coordinate (O)
			(3,0) coordinate (A)
			(0,3) coordinate (B)
			(0,-3) coordinate (B')
			(-3,0) coordinate (A')
			(0:0)++(150:3) coordinate (M)
			($(O)!(M)!(A')$) coordinate (H)
			($(O)!(M)!(B)$) coordinate (K)
			;
			\draw[->] (-4,0) -- (4,0) node[above,blue]{$x$};
			\draw[->] (0,-4) -- (0.,4) node[left,blue]{$y$};
			\draw[orange] (O) circle (3cm);
			\draw[rotate=0,->,green!50!black] (0.5,0) arc (0:150:0.5cm);
			\draw (0.35,0.25) node[above,blue] {$\alpha$};
			\draw[dashed] (H)--(M)--(K);
			\draw[green!50!black] (M)--(O);
			\foreach \p/\r in {A/-45,M/150,H/-90,O/-150,A'/-135,B'/-45,B/45,K/0}
			\fill (\p) circle (1pt) node[shift={(\r:3mm)},blue]{$\p$};
		\end{tikzpicture}
	}
	\subsubsection{Các giá trị lượng giác của góc lượng giác}
	\immini{Gọi $M(x;y)$ là điểm biểu diễn của góc lượng giác $\alpha$ trên đường tròn lượng giác. Khi đó, ta có:
		\begin{itemize}
			\item $\cos\alpha=x.$
			\item $\sin\alpha=y.$
			\item $\tan\alpha=\dfrac{\sin\alpha}{\cos\alpha}=\dfrac{y}{x} ~(x\neq0).$
		\item $\cot\alpha=\dfrac{\cos\alpha}{\sin\alpha}=\dfrac{x}{y} ~(y\neq0).$
	\end{itemize}}
	{\begin{tikzpicture}[line join = round, line cap = round, >=stealth, font=\footnotesize, scale=0.6]
			\tikzset{label style/.style={font=\footnotesize}}
			\path (0,0) coordinate (O)
			(3,0) coordinate (A)
			(0,3) coordinate (B)
			(0,-3) coordinate (B')
			(-3,0) coordinate (A')
			(0:0)++(40:3) coordinate (M)
			($(O)!(M)!(A')$) coordinate (H)
			($(O)!(M)!(B)$) coordinate (K)
			;
			\draw[->] (-4,0) -- (4,0) node[above,blue]{$x$};
			\draw[->] (0,-4) -- (0.,4) node[left,blue]{$y$};
			\draw[orange] (O) circle (3cm);
			\draw[rotate=0,->,green!50!black] (0.7,0) arc (0:40:0.7cm);
			\draw (1,0) node[above,blue] {$\alpha$};
			\draw[dashed] (H)--(M)--(K);
			\draw[green!50!black] (M)--(O);
			\draw[blue,fill=black] (0,2) node[left]{$\sin\alpha$}(2,0) circle(1pt) node[below]{$\cos\alpha$}(3,2.3) node{$M(x;y)$};
			\foreach \p/\r in {A/-45,O/-135,A'/-135,B'/-45,B/45}
			\fill (\p) circle (1pt) node[shift={(\r:3mm)},blue]{$\p$};
	\end{tikzpicture}}
	\begin{note}
		a) Ta còn gọi trục tung là trục sin, trục hoành là trục côsin.\\
		b) Từ định nghĩa ta suy ra:
		\begin{itemize}
			\item $\sin\alpha$, $\cos\alpha$ xác định với mọi giá trị của $\alpha$ và ta có:
			$$-1\leq \sin\alpha\leq 1; \quad -1\leq \cos\alpha\leq 1; \quad \sin(\alpha+k2\pi)=\sin\alpha;\quad \cos(\alpha+k2\pi)=\cos\alpha\,\, (k\in\mathbb{Z}).$$
			\item $\tan\alpha$ xác định khi $\alpha\neq\dfrac{\pi}{2}+k\pi\,\,  (k\in\mathbb{Z})$.
			\item $\cot\alpha$ xác định khi $\alpha\neq k\pi\,\,  (k\in\mathbb{Z})$.
			\item Dấu của các giá trị lượng giác của một góc lượng giác phụ thuộc vào vị trí điểm biểu diễn $M$ trên đường tròn lượng giác.
		\end{itemize}
	\end{note}
	\begin{minipage}[h]{0.6\textwidth}
		\begin{tabular}{c|c|c|c|c|}
			\cline{2-5}
			& \multicolumn{4}{c|}{Góc phần tư} \\ \hline
			\multicolumn{1}{|c|}{Giá trị lượng giác} & I     & II     & III     & IV    \\ \hline
			\multicolumn{1}{|c|}{$\sin \alpha$}     &   $+$    &  $ +$      &    $-$    &   $-$  \\ \hline
			\multicolumn{1}{|c|}{$\cos \alpha$}     &   $+$    &  $ -$      &    $-$    &   $+$  \\ \hline
			\multicolumn{1}{|c|}{$\tan \alpha$}     &   $+$    &  $ -$      &    $+$    &   $-$  \\ \hline
			\multicolumn{1}{|c|}{$\cot \alpha$}     &   $+$    &  $ -$      &    $+$    &   $-$  \\ \hline
		\end{tabular}
	\end{minipage}
	\begin{minipage}[h]{0.6\textwidth}
		\begin{tikzpicture}[line join = round, line cap = round, >=stealth, font=\footnotesize, scale=0.6]
			\tikzset{label style/.style={font=\footnotesize}}
			\path (0,0) coordinate (O)
			(3,0) coordinate (A)
			(0,3) coordinate (B)
			(0,-3) coordinate (B')
			(-3,0) coordinate (A')
			(0:0)++(-60:3) coordinate (M)
			($(O)!(M)!(A')$) coordinate (H)
			($(O)!(M)!(B)$) coordinate (K)
			;
			\draw[->] (-4,0) -- (4,0) node[above,blue]{$x$};
			\draw[->] (0,-4) -- (0.,4) node[left,blue]{$y$};
			\draw[orange] (O) circle (3cm);
			\draw[rotate=0,->,green!50!black] (0.5,0) arc (0:-60:0.5cm);
			\draw (0.75,-0.35) node[blue] {$\alpha$};
			\draw[dashed] (H)--(M)--(K);
			\draw[green!50!black] (M)--(O);
			\draw[blue] (2.5,2.5) node{$I$}(-2.5,2.5) node{$II$}(-2.5,-2.5) node{$III$}(2.5,-2.5) node{$IV$};
			\foreach \p/\r in {A/-45,M/-60,H/90,O/-150,A'/-135,B'/-45,B/45,K/180}
			\fill (\p) circle (1pt) node[shift={(\r:3mm)},blue]{$\p$};
		\end{tikzpicture}
	\end{minipage}
	% \subsubsection{Giá trị lượng giác của các góc đặc biệt}
	% \begin{center}
	% 	\renewcommand{\arraystretch}{2}
	% 	\begin{tabular}{|c|c|c|c|c|c|}
	% 		\hline
	% 		\multirow{2}{*}{Góc $\alpha$} & $0$              & $\dfrac{\pi}{6}$  & $\dfrac{\pi}{4}$  & $\dfrac{\pi}{3}$  & $\dfrac{\pi}{2}$              \\ \cline{2-6} 
	% 		& $0^\circ$              & $30^\circ$  & $45^\circ$  & $60^\circ$  & $90^\circ$             \\ \hline
	% 		$\sin\alpha$                  & $0$             & $\dfrac{1}{2}$ & $\dfrac{\sqrt{2}}{2}$ & $\dfrac{\sqrt{3}}{2}$ & 1              \\ \hline
	% 		$\cos\alpha$                  & $1$             & $\dfrac{\sqrt{3}}{2}$ & $\dfrac{\sqrt{2}}{2}$ & $\dfrac{1}{2}$ & 0              \\ \hline
	% 		$\tan\alpha$                 & $0$             & $\dfrac{1}{\sqrt{3}}$ & 1 & $\sqrt{3}$ & Không xác định \\ \hline
	% 		$\cot\alpha$                  & Không xác định & $\sqrt{3}$ & 1 & $\dfrac{1}{\sqrt{3}}$ & 0              \\ \hline
	% 	\end{tabular}
	% \end{center}
	\subsubsection{Các công thức lượng giác cơ bản}
	Đối với các giá trị lượng giác, ta có các hệ thức cơ bản sau
	\begin{enumEX}[$\bullet$]{2}
		\item $\sin^2 \alpha  + \cos^2 \alpha =1$
		\item $ 1+ \tan^2 \alpha= \dfrac{1}{\cos^2 \alpha}$ $\left(\alpha \neq \dfrac{\pi}{2}+k\pi , k\in \mathbb{Z}\right)$
		\item $ 1+ \cot^2 \alpha= \dfrac{1}{\sin^2 \alpha}$ $\left(\alpha \neq k\pi , k\in \mathbb{Z}\right)$
		\item $\tan \alpha \cdot \cot \alpha =1 $ $\left(\alpha \neq \dfrac{k\pi}{2}, k\in \mathbb{Z}\right)$
	\end{enumEX}
	\newpage
	\subsubsection{Giá trị lượng giác của các góc có liên quan đặc biệt}
	\begin{enumerate}
		\item Góc đối nhau ($\alpha$ và $-\alpha$)
		\immini{\begin{itemize}
				\item $\cos (-\alpha)=\cos \alpha$
				\item $\sin (-\alpha) =-\sin \alpha$
				\item $\tan (-\alpha) =-\tan \alpha$
				\item $\cot (-\alpha) =-\cot \alpha$
		\end{itemize}}
		{\vspace*{-1cm}\begin{tikzpicture}[line join = round, line cap = round, >=stealth, font=\footnotesize, scale=0.5]
				\tikzset{label style/.style={font=\footnotesize}}
				\path (0,0) coordinate (O)
				(3,0) coordinate (A)
				(0:0)++(120:3) coordinate (M)
				(0:0)++(-120:3) coordinate (N)
				(0,4) coordinate (C)
				(0,-4) coordinate (D)
				($(O)!(M)!(C)$) coordinate (E)
				($(O)!(N)!(D)$) coordinate (F)
				;
				\draw[->] (-4,0) -- (4,0) node[above,blue]{$x$};
				\draw[->] (0,-4) -- (0.,4) node[left,blue]{$y$};
				\draw[orange] (O) circle (3cm);
				\draw[rotate=0,->,red] (0.5,0) arc (0:120:0.5cm);
				\draw[rotate=0,->,green!50!black] (0.6,0) arc (0:-120:0.6cm);
				\draw (0,0) node[above right=2pt,blue] {$\alpha$} (0,-0) node[below right=2pt,blue]{$-\alpha$};
				\draw[dashed] (E)--(M)--(N)--(F);
				\draw[green!50!black] (M)--(O);
				\draw[red] (N)--(O);
				\foreach \p/\r in {A/-45,M/120,N/-120,O/-150}
				\fill (\p) circle (1pt) node[shift={(\r:3mm)},blue]{$\p$};
		\end{tikzpicture}}
		\item Góc bù nhau ($\alpha$ và $\pi-\alpha$)
		\immini{\begin{itemize}
				\item $\sin (\pi -\alpha)=\sin \alpha$
				\item $\cos (\pi -\alpha) =-\cos \alpha$
				\item $\tan (\pi -\alpha) =-\tan \alpha$
				\item $\cot (\pi -\alpha) =-\cot \alpha$
		\end{itemize}}
		{\vspace*{-0.5cm}\begin{tikzpicture}[line join = round, line cap = round, >=stealth, font=\footnotesize, scale=0.5]
				\tikzset{label style/.style={font=\footnotesize}}
				\path (0,0) coordinate (O)
				(3,0) coordinate (A)
				(0:0)++(30:3) coordinate (M)
				(0:0)++(150:3) coordinate (N)
				(4,0) coordinate (C)
				(-4,0) coordinate (D)
				($(O)!(M)!(C)$) coordinate (E)
				($(O)!(N)!(D)$) coordinate (F)
				;
				\draw[->] (-4,0) -- (4,0) node[above,blue]{$x$};
				\draw[->] (0,-4) -- (0.,4) node[left,blue]{$y$};
				\draw[orange] (O) circle (3cm);
				\draw[rotate=0,->,red] (0.5,0) arc (0:150:0.5cm);
				\draw[rotate=0,->,green!50!black] (1.6,0) arc (0:30:1.6cm);
				\draw (2,0) node[above,blue] {$\alpha$} (0.3,1.5) node[below,blue]{$\pi-\alpha$};
				\draw[dashed] (E)--(M)--(N)--(F);
				\draw[red] (O)--(N);
				\draw[green!50!black] (O)--(M);
				\foreach \p/\r in {A/-45,M/30,N/150,O/-130}
				\fill (\p) circle (1pt) node[shift={(\r:3mm)},blue]{$\p$};
		\end{tikzpicture}}
		\item Góc phụ nhau ($\alpha$ và $\dfrac{\pi}{2}-\alpha$)
		\immini{\begin{itemize}
				\item $\sin \left( \dfrac{\pi}{2}-\alpha\right)=\cos \alpha$
				\item $\cos \left( \dfrac{\pi}{2}-\alpha\right)=\sin \alpha$
				\item $\tan \left( \dfrac{\pi}{2}-\alpha\right)=\cot \alpha$
				\item $\cot \left( \dfrac{\pi}{2}-\alpha\right)=\tan \alpha$
		\end{itemize}}
		{\vspace*{-0.5cm}\begin{tikzpicture}[line join = round, line cap = round, >=stealth, font=\footnotesize, scale=0.5]
				\tikzset{label style/.style={font=\footnotesize}}
				\path (0,0) coordinate (O)
				(3,0) coordinate (A)
				(0:0)++(20:3) coordinate (M)
				(0:0)++(70:3) coordinate (N)
				(0,4) coordinate (C)
				(4,0) coordinate (D)
				($(O)!(M)!(C)$) coordinate (E)
				($(O)!(N)!(D)$) coordinate (F)
				(2.82,0) coordinate (G)
				(0,2.82) coordinate (H)
				;
				\draw[->] (-4,0) -- (4,0) node[above,blue]{$x$};
				\draw[->] (0,-4) -- (0.,4) node[left,blue]{$y$};
				\draw[orange] (O) circle (3cm);
				\draw[rotate=0,->,red] (0.7,0) arc (0:70:0.7cm);
				\draw[rotate=0,->,green!50!black] (1.6,0) arc (0:20:1.6cm);
				\draw (2,0) node[above,blue] {$\alpha$} (1,-.2) node[below,blue]{$\frac{\pi}{2}-\alpha$};
				\draw[dashed] (E)--(M) (F)--(N) (G)--(M) (H)--(N);
				\draw[dashed] (-3,-3)--(3,3);
				\draw[->] (0.8,-.5)--(0.5,0.45);
				\draw[red] (O)--(N);
				\draw[green!50!black] (O)--(M);
				\foreach \p/\r in {A/-45,M/20,N/70,O/-220}
				\fill (\p) circle (1pt) node[shift={(\r:3mm)},blue]{$\p$};
		\end{tikzpicture}}
		\item Góc hơn kém $\pi$ ($\alpha$ và $\pi+\alpha$)
		\immini{\begin{itemize}
				\item $\sin (\pi +\alpha)=-\sin \alpha$
				\item $\cos (\pi +\alpha)=-\cos \alpha$
				\item $\tan (\pi +\alpha)=\tan \alpha$
				\item $\cot (\pi +\alpha)=\cot \alpha$
		\end{itemize}}
		{\vspace*{-0.5cm}\begin{tikzpicture}[line join = round, line cap = round, >=stealth, font=\footnotesize, scale=0.5]
				\tikzset{label style/.style={font=\footnotesize}}
				\path (0,0) coordinate (O)
				(3,0) coordinate (A)
				(0:0)++(60:3) coordinate (M)
				(0:0)++(240:3) coordinate (N)
				;
				\draw[->] (-4,0) -- (4,0) node[above,blue]{$x$};
				\draw[->] (0,-4) -- (0.,4) node[left,blue]{$y$};
				\draw[orange] (O) circle (3cm);
				\draw[rotate=0,->,red] (1.7,0) arc (0:240:1.7cm);
				\draw[rotate=0,->,green!50!black] (0.6,0) arc (0:60:0.6cm);
				\draw (1,0) node[above,blue] {$\alpha$};
				\draw (-1.2,1) node[below,blue,rotate=60]{$\pi+\alpha$};
				\draw[red] (O)--(N);
				\draw[green!50!black] (O)--(M);
				\foreach \p/\r in {A/-45,M/60,N/240,O/-150}
				\fill (\p) circle (1pt) node[shift={(\r:3mm)},blue]{$\p$};
		\end{tikzpicture}}
	\end{enumerate}
\end{tomtat}
 \foreach \i in {1,2,...,7} {\input{data/11KNTT/data1/1K1-2-\i.tex}}

%%Bài 2
% %\chapter{Hàm số  lượng giác và phương trình lượng giác}
\setcounter{section}{1}
\section{Công thức lượng giác}
\subsection{Tóm tắt lý thuyết}
\begin{tomtat}
% 	\begin{center}
% 		\begin{tikzpicture}[scale = 2.5]
% 			\path (0,0) coordinate (O) (1.5,0) coordinate (x) (0,1.5) coordinate (y);
% 			\draw[thick,->] (-1.5,0)--(x);
% 			\draw[thick,->] (0,-1.5)--(y);
% 			\draw (O) circle (1);
% 			\path ($(O)+(55:1)$) coordinate (M) 
% 			($(O)+(30:1)$) coordinate (N);
% 			\path ($(O)!(M)!(x)$) coordinate (x_M)
% 			($(O)!(M)!(y)$) coordinate (y_M)
% 			($(O)!(N)!(x)$) coordinate (x_N)
% 			($(O)!(N)!(y)$) coordinate (y_N);
% 			\draw[dashed] (x_M)--(M)--(y_M) (x_N)--(N)--(y_N);
% 			\foreach \x/\g in {O/-135,x/-90,y/180,x_M/-90,x_N/-90,y_M/180,y_N/180}
% 			\fill ($(\g:1mm)+(\x)$) node {$\x$};
% 			\fill 	(M) circle (0.5pt)
% 			($(15:4mm)+(M)$) node {$M\left(x_M,y_M\right)$};
% 			\fill (N) circle (0.5pt)
% 			($(15:4mm)+(N)$) node {$N\left(x_N,y_N\right)$};
% 	\draw (M)--(O)--(N);		
% 	\draw pic[draw,,angle radius=6mm,->,red]{angle=x--O--M};
% 	\fill[red] (45:3mm) node {$\alpha$};
% 	\draw pic[red,draw,,angle radius=10mm,->]{angle=x--O--N};
% 	\fill[red] (15:5mm) node {$\beta$};
% 		\end{tikzpicture}
% 	\end{center}
% Trong mặt phẳng $Oxy$ cho hai điểm $M,N$ trên đường tròn lượng giác.\\ Đặt $\alpha = \text{sđ} (Ox,OM), \beta = \text{sđ} (Ox,ON)$, ta có $M(\cos \alpha,\sin \alpha)$ và $N(\cos \beta, \sin \beta)$. Khi đó ta tính được $\overrightarrow{OM}.\overrightarrow{ON}$ bằng hai cách
% \begin{align*}
% 	\overrightarrow{OM}.\overrightarrow{ON}&=\left|\overrightarrow{OM}\right|.\left|\overrightarrow{ON}\right|.\cos \left(\overrightarrow{OM},\overrightarrow{ON}\right) = \cos (\alpha-\beta),\\
% 	\overrightarrow{OM}.\overrightarrow{ON} &= x_Mx_N+y_My_N= \cos \alpha \cos\beta +\sin\alpha\sin\beta.
% \end{align*}
% Từ đó dẫn tới công thức
% \begin{align*}
% 	\cos (\alpha-\beta) = \cos \alpha \cos \beta + \sin \alpha\sin \beta \tag{$\star$}
% \end{align*}
% Tất cả các công thức trong bài học được xây dựng dựa trên công thức $(\star)$.\\
% Trong suốt bài học, khi không nói gì thêm, chỉ xét các góc lượng giác mà trong đó giá trị lượng giác được để cập có nghĩa. 
	\subsubsection{Công thức cộng}
	\begin{khung4}{Công thức cộng}
	\begin{tasks}[style=itemize](2)
		\task $\cos (a-b) = \cos a \cos b + \sin a\sin b$.
		\task $\cos (a+b) = \cos a \cos b - \sin a\sin b$.
		\task $\sin (a-b) = \sin a \cos b - \sin b \cos a$.
		\task $\sin (a+b) = \sin a \cos b + \sin b \cos a$.
		\task $\tan (a-b) = \dfrac{\tan a - \tan b}{1+\tan a \tan b}$.
		\task $\tan (a+b) = \dfrac{\tan a + \tan b}{1-\tan a \tan b}$.
	\end{tasks}
	\end{khung4}
	\subsubsection{Công thức nhân đôi}
	Công thức nhân đôi được xây dựng bằng cách thay $b=a$ trong công thức cộng.
	\begin{khung4}{Công thức nhân đôi}
		\begin{tasks}[style=itemize]
			\task $\sin 2a = 2\sin a \cos a$.
			\task $\cos 2a = \cos^2a-\sin^2a = 2\cos^2a-1 = 1-2\sin^2a$.
			\task $\tan 2a = \dfrac{2\tan a}{1-\tan^2a}$.
		\end{tasks}
		\end{khung4}
\begin{note}
	Từ công thức nhân đôi, ta có công thức hạ bậc:
\end{note}
	\begin{khung4}{Công thức hạ bậc}
		\begin{tasks}[style=itemize](3)
			\task $\sin^2a= \dfrac{1-\cos 2a}{2}$.
		\task $\cos^2a = \dfrac{1+\cos 2a}{2}$.
		\task $\tan^2a=\dfrac{1-\cos2a}{1+\cos 2a}$.
		\end{tasks}
		\end{khung4}

\begin{note}
	Áp dụng công thức cộng cho $3a = a +2a$, ta có công thức nhân ba:
\end{note}
	\begin{khung4}{Công thức nhân ba}
		\begin{tasks}[style=itemize](2)
			\task $\sin3a= 3\sin a -4\sin^3a$.
		\task $\cos3a= 4\cos^3a-3\cos a$.
		\task $\tan3a = \dfrac{3\tan a - \tan^3 a}{1-3\tan^2a}$.
		\end{tasks}
		\end{khung4}

	\subsubsection{Công thức biến đổi tích thành tổng}
	\begin{khung4}{Công thức tích thành tổng}
		\begin{tasks}[style=itemize]
			\task $\cos a \cos b = \dfrac{1}{2}\left[\cos (a-b) + \cos (a+b)\right]$.
		\task $\sin a \sin b = \dfrac{1}{2}\left[\cos (a-b)-\cos(a+b)\right]$.
		\task $\sin a \cos b = \dfrac{1}{2}\left[\sin (a-b)+\sin (a+b)\right]$.
		\end{tasks}
		\end{khung4}
	\subsubsection{Công thức biến đổi tổng thành tích}
	Công thức biến đổi tổng thành tích được xây dựng bằng cách $a=\dfrac{a+b}{2}, b = \dfrac{a-b}{2}$ trong công thức biến đổi tích thành tổng.
	\begin{khung4}{Công thức tổng thành tích}
		\begin{tasks}[style=itemize](2)
			\task $\cos a+ \cos b = 2\cos\dfrac{a+b}{2}\cos \dfrac{a-b}{2}$.
		\task $\cos a- \cos b = -2\sin\dfrac{a+b}{2}\sin \dfrac{a-b}{2}$.
		\task $\sin a+ \sin b = 2\sin\dfrac{a+b}{2}\cos \dfrac{a-b}{2}$.
		\task $\sin a -\sin b = 2\cos\dfrac{a+b}{2}\sin \dfrac{a-b}{2}$.
		\end{tasks}
		\end{khung4}
\end{tomtat}

 \foreach \i in {1,2,...,5} {\input{data/11KNTT/data1/1K1-2-\i.tex}}

%%Bài 3
% \setcounter{section}{2}
\section{Hàm số lượng giác}
\subsection{Tóm tắt lý thuyết}
\begin{tomtat}
	% \subsubsection{Định nghĩa hàm số lượng giác}
	% \begin{dn}
	% 	\begin{itemize}
	% 		\item Hàm số sin $y=\sin x$ có tập xác định là $\mathbb{R}$.
	% 		\item Hàm số cos $y=\cos x$ có tập xác định là $\mathbb{R}$.
	% 		\item Hàm số tan  $y=\tan x$ có tập xác định là $\mathbb{R} \setminus\left\{\dfrac{\pi}{2}+k \pi \Big| k \in \mathbb{Z} \right\}$.
	% 		\item  Hàm số cot $y=\cot x$ có tập xác định là $\mathbb{R} \setminus \left\{k \pi \Big| k \in \mathbb{Z} \right\}$.
	% 	\end{itemize}
	% \end{dn}
	\subsubsection{Hàm số chẵn, hàm số lẻ}
	\begin{dn}
	Cho hàm số $y=f(x)$ có tập xác định là $\mathscr{D}$.
	\begin{itemize}
		\item Hàm số $f(x)$ được gọi là \textbf{hàm số chẵn} nếu $\forall x \in \mathscr{D}$ thì $-x \in \mathscr{D}$ và $f(-x)=f(x)$. Đồ thị của một hàm số chẵn nhận trục tung là trục đối xứng.
		\item Hàm số $f(x)$ được gọi là \textbf{hàm số lẻ} nếu $\forall x \in \mathscr{D}$ thì $-x \in \mathscr{D}$ và $f(-x)=-f(x)$. Đồ thị của một hàm số lẻ nhận gốc toạ độ là tâm đối xứng.
	\end{itemize}
	\end{dn}
	\subsubsection{Hàm số tuần hoàn}
	\begin{dn}
		Hàm số $y=f(x)$ có tập xác định $\mathscr{D}$ được gọi là \textbf{hàm số tuần hoàn} nếu tồn tại số $T \neq 0$ sao cho với mọi $x \in \mathscr{D}$ ta có:
		\begin{enumerate}[i)]
			\item $x+T \in \mathscr{D}$ và $x-T \in \mathscr{D}$;
			\item $f(x+T)=f(x)$.
		\end{enumerate}
		Số $T$ dương nhỏ nhất thỏa mãn các điều kiện trên (nếu có) được gọi là \textbf{chu kì} của hàm số tuần hoàn đó.
	\end{dn}
	\begin{nx}
		\
		\begin{itemize}
			\item  Các hàm số $y=\sin x$ và $y=\cos x$ tuần hoàn với chu kì $2 \pi$. Các hàm số $y=\tan x$ và $y=\cot x$ tuần hoàn với chu kì $\pi$.
		\end{itemize}
	\end{nx}
	\begin{note} 
		Tổng quát, người ta chứng minh được các hàm số $y=A \sin \omega x$ và $y=A \cos \omega x$ $(\omega>0)$ là những hàm số tuần hoàn với chu kì \fbox{$T=\dfrac{2 \pi}{\omega}$}.
	\end{note}
	\subsubsection{Đồ thị và tính chất của hàm số $y=\sin x$}
	\begin{tc}
		Hàm số $y=\sin x$:
		\begin{itemize}
			\item   Có tập xác định là $\mathbb{R}$ và tập giá trị là $[-1 ; 1]$;
			\item   Là hàm số lẻ và tuần hoàn với chu kì $2 \pi$;
			\item    Đồng biến trên mỗi khoảng $\left(-\dfrac{\pi}{2}+k 2 \pi ; \dfrac{\pi}{2}+k 2 \pi\right)$ và nghịch biến trên mỗi khoảng \\
			$\left(\dfrac{\pi}{2}+k 2 \pi ; \dfrac{3 \pi}{2}+k 2 \pi\right)$, $k \in \mathbb{Z}$;
			\item    Có đồ thị đối xứng qua gốc toạ độ và gọi là một \textbf{đường hình sin}.
		\end{itemize}
	\begin{center}
		\begin{tikzpicture}[>=stealth,scale=0.7,transform shape] 
			\path
			({-2.5*pi},0) coordinate (X1)
			({-2*pi},0) coordinate (X2)
			({-1.5*pi},0) coordinate (X3)
			({-pi},0) coordinate (X4)
			({-0.5*pi},0) coordinate (X5)
			(0,0) coordinate (O)
			({0.5*pi},0) coordinate (X6)
			({pi},0) coordinate (X7)
			({1.5*pi},0) coordinate (X8)
			({2*pi},0) coordinate (X9)
			({2.5*pi},0) coordinate (X10)
			({-pi},-2) coordinate (A)
			({pi},-2) coordinate (B)
			;
			\draw[->] (-9.5,0) -- (9.5,0) node[below] {\small $x$};
			\draw[->] (0,-1.5) -- (0,1.8) node[right] {\small $y$};
			\draw [dotted] (X3)--({-1.5*pi},1)--({2.5*pi},1)--({2.5*pi},0)  ({0.5*pi},1)--({0.5*pi},0)
			(X1)--({-2.5*pi},-1)--({1.5*pi},-1)--({1.5*pi},0)  ({-0.5*pi},-1)--({-0.5*pi},0)
			({-pi},0) -- (A) ({pi},0) -- (B);
			\foreach \x/\g/\z in {X1/90/-\tfrac{5\pi}{2},X2/140/-2\pi,X3/-90/-\tfrac{3\pi}{2},X4/-135/-\pi,X5/90/-\tfrac{\pi}{2},X6/-90/\tfrac{\pi}{2},X7/60/\pi,X8/90/\tfrac{3\pi}{2},X9/-40/2\pi,X10/-90/\tfrac{5\pi}{2}} 
			\fill[black] (\x) circle(1pt) +(\g:5mm) node {$\z$};
			\draw [<->] ({-pi},-1.7)--({pi},-1.7) ; 
			\draw (0,0) node[below right]{$O$} (0,-1.7) node[below]{$T=2\pi$}
			(0,1) node[above right]{$1$} (0,-1) node[below right]{$-1$};
			\clip (-9.5,-1.4) rectangle (9.5,1.6) ;
			\draw[thick,samples=100,domain=-9.3:9.3] plot(\x,{sin((\x)*180/pi)});
			
		\end{tikzpicture}
	\end{center}
	\end{tc}
	\subsubsection{Đồ thị và tính chất của hàm số $y=\cos x$}
	\begin{tc}
		Hàm số $y=\cos x$:
		\begin{itemize}
			\item    Có tập xác định là $\mathbb{R}$ và tập giá trị là $[-1 ; 1]$;
			\item    Là hàm số chẵn và tuần hoàn với chu kì $2 \pi$;
			\item    Đồng biến trên mỗi khoảng $(-\pi+k 2 \pi ; k 2 \pi)$ và nghịch biến trên mỗi khoảng $(k 2 \pi ; \pi+k 2 \pi), k \in \mathbb{Z}$;
			\item    Có đồ thị là một đường hình sin đối xứng qua trục tung.
		\end{itemize}
	\begin{center}
		\begin{tikzpicture}[>=stealth,scale=0.77,transform shape] 
			\path
			({-2.5*pi},0) coordinate (X1)
			({-2*pi},0) coordinate (X2)
			({-1.5*pi},0) coordinate (X3)
			({-pi},0) coordinate (X4)
			({-0.5*pi},0) coordinate (X5)
			(0,0) coordinate (O)
			({0.5*pi},0) coordinate (X6)
			({pi},0) coordinate (X7)
			({1.5*pi},0) coordinate (X8)
			({2*pi},0) coordinate (X9)
			({2.5*pi},0) coordinate (X10)
			({-pi},-2) coordinate (A)
			({pi},-2) coordinate (B)
			;
			\draw[->] (-9.5,0) -- (9.5,0) node[below] {\small $x$};
			\draw[->] (0,-1.5) -- (0,1.8) node[right] {\small $y$};
			\draw [dotted] (X2)--({-2*pi},1)--({2*pi},1)--({2*pi},0) (X4)--({-pi},-1)--({pi},-1)--({pi},0) 
			({-pi},0) -- (A) ({pi},0) -- (B);
			\foreach \x/\g/\z in {X1/125/-\tfrac{5\pi}{2},X2/-90/-2\pi,X3/-120/-\tfrac{3\pi}{2},X4/90/-\pi,X5/110/-\tfrac{\pi}{2},X6/-120/\tfrac{\pi}{2},X7/90/\pi,X8/120/\tfrac{3\pi}{2},X9/-90/2\pi,X10/-120/\tfrac{5\pi}{2}} 
			\fill[black] (\x) circle(1pt) +(\g:5mm) node {$\z$};
			\draw [<->] ({-pi},-1.7)--({pi},-1.7) ; 
			\draw (0,0) node[below right]{$O$} (0,-1.7) node[below]{$T=2\pi$}
			(0,1) node[above right]{$1$} (0,-1) node[below right]{$-1$}
			;
			\clip (-9.5,-1.4) rectangle (9.5,1.6) ;
			\draw[thick,samples=100,domain=-9.3:9.3] plot(\x,{cos((\x)*180/pi)});
			
		\end{tikzpicture}
	\end{center}
	\end{tc}
	\subsubsection{Đồ thị và tính chất của hàm số $y=\tan x$}
	\begin{tc}
		Hàm số $y=\tan x$:
		\begin{itemize}
			\item    Có tập xác định là $\mathbb{R} \setminus\left\{\dfrac{\pi}{2}+k \pi \Big| k \in \mathbb{Z} \right\}$ và tập giá trị là $\mathbb{R}$;
			\item    Là hàm số lẻ và tuần hoàn với chu kì $\pi$;
			\item    Đồng biến trên mỗi khoảng $\left(-\dfrac{\pi}{2}+k \pi ; \dfrac{\pi}{2}+k \pi\right)$, $k \in \mathbb{Z}$;
			\item    Có đồ thị đối xứng qua gốc toạ độ.
		\end{itemize}
	\begin{center}
		\begin{tikzpicture}[>=stealth,scale=0.77,transform shape] 
			\path
			({-1.5*pi},0) coordinate (X1)
			({-pi},0) coordinate (X2)
			({-0.5*pi},0) coordinate (X3)
			({0.5*pi},0) coordinate (X4)
			({pi},0) coordinate (X5)
			({1.5*pi},0) coordinate (X6)
			;
			\draw[->] (-6.5,0) -- (6.5,0) node[below] {\small $x$};
			\draw[->] (0,-3.5) -- (0,3.5) node[right] {\small $y$};
			\draw [dashed] ({-3*pi/2},3.5)--({-3*pi/2},-3.5) ({-pi/2},3.5)--({-pi/2},-3.5) ({pi/2},3.5)--({pi/2},-3.5) ({3*pi/2},3.5)--({3*pi/2},-3.5) ;
			\foreach \x/\g/\z in {X1/-150/-\tfrac{3\pi}{2},X2/-40/-\pi,X3/-40/-\tfrac{\pi}{2},X4/-40/\tfrac{\pi}{2},X5/-400/\pi,X6/-40/\tfrac{3\pi}{2}} 
			\fill[black] (\x) circle(1pt) +(\g:5mm) node {$\z$};
			\draw (0,0) node[below right]{$O$};
			\clip (-6.5,-3.5) rectangle (6.5,3.5) ;
			\draw[thick,samples=100,domain={-pi/2+0.2}:{pi/2-0.2}] plot(\x,{tan((\x)*180/pi)});
			\draw[thick,samples=100,domain={pi/2+0.2}:{3*pi/2-0.2}] plot(\x,{tan((\x)*180/pi)});
			\draw[thick,samples=100,domain={-3*pi/2+0.2}:{-pi/2-0.2}] plot(\x,{tan((\x)*180/pi)});
		\end{tikzpicture}
	\end{center}
	\end{tc}
	\subsubsection{Đồ thị và tính chất của hàm số $y=\cot x$}
	\begin{tc}
		Hàm số $y=\cot x$:
		\begin{itemize}
			\item    Có tập xác định là $\mathbb{R} \setminus\{k \pi \mid k \in \mathbb{Z} \}$ và tập giá trị là $\mathbb{R}$; 
			\item    Là hàm số lẻ và tuần hoàn với chu kì $\pi$;
			\item    Nghịch biến trên mỗi khoảng $(k \pi ; \pi+k \pi), k \in \mathbb{Z}$;
			\item    Có đồ thị đối xứng qua gốc toạ độ.
		\end{itemize}
	\begin{center}
		\begin{tikzpicture}[>=stealth,scale=0.77,transform shape] 
			\path
			({-2*pi},0) coordinate (X1)
			({-1.5*pi},0) coordinate (X2)
			({-pi},0) coordinate (X3)
			({-0.5*pi},0) coordinate (X4)
			({0.5*pi},0) coordinate (X5)
			({pi},0) coordinate (X6)
			({1.5*pi},0) coordinate (X7)
			({2*pi},0) coordinate (X8)
			;
			\draw[->] (-7.5,0) -- (7.5,0) node[below] {\small $x$};
			\draw[->] (0,-3.5) -- (0,3.5) node[left] {\small $y$};
			\draw [dashed] ({-2*pi},3.5)--({-2*pi},-3.5) ({-pi},3.5)--({-pi},-3.5) ({pi},3.5)--({pi},-3.5) ({2*pi},3.5)--({2*pi},-3.5) ;
			\foreach \x/\g/\z in {X1/-150/-2\pi,X2/-130/-\tfrac{3\pi}{2},X3/-140/-\pi,X4/-100/-\tfrac{\pi}{2},X5/-100/\tfrac{\pi}{2},X6/-120/\pi,X7/-100/\tfrac{3\pi}{2}, X8/-120/2\pi}
			\fill[black] (\x) circle(1pt) +(\g:5mm) node {$\z$};
			\draw (0,0) node[below left]{$O$};
			\clip (-6.5,-3.5) rectangle (6.5,3.5) ;
			\draw[thick,samples=100,domain={0.2}:{pi-0.2}] plot(\x,{cot((\x)*180/pi)});
			\draw[thick,samples=100,domain={pi+0.2}:{2*pi-0.2}] plot(\x,{cot((\x)*180/pi)});
			\draw[thick,samples=100,domain={-0.2}:{-pi+0.2}] plot(\x,{cot((\x)*180/pi)});
			\draw[thick,samples=100,domain={-pi-0.2}:{-2*pi+0.2}] plot(\x,{cot((\x)*180/pi)});
		\end{tikzpicture}
	\end{center}
	\end{tc}
\end{tomtat}
\subsection{Các dạng toán thường gặp}
 \foreach \i in {1,2,...,5} {\input{data/11KNTT/data1/1K1-3-\i.tex}}

%%Bài 4
% 
\setcounter{section}{3}
\section{Phương trình lượng giác cơ bản}
\subsection{Tóm tắt lý thuyết}
\begin{tomtat}
\subsubsection{Phương trình $\sin x=m$}
\begin{itemize}
	\item Với $|m|>1$ thì phương trình $\sin x=m$ vô nghiệm.
	\item Với $|m|\leq 1$, sẽ tồn tại duy nhất $\alpha \in \left[-\dfrac{\pi}{2}; \dfrac{\pi}{2}\right]$ thỏa mãn $\sin\alpha=m$. Khi đó
	\begin{center}
		$\sin x=m\Leftrightarrow\sin x=\sin\alpha\Leftrightarrow\hoac{&x=\alpha+k2\pi\\&x=\pi-\alpha+k2\pi}$ ($k\in \mathbb{Z}$).
	\end{center}
\item Nếu số đo của góc $\alpha$ được đo bằng đơn vị độ thì
\begin{center}
	$\sin x=\sin\alpha^\circ\Leftrightarrow\hoac{&x=\alpha^\circ+k360^\circ\\&x=180^\circ-\alpha^\circ+k360^\circ}$ ($k\in\mathbb{Z}$).
\end{center}
\item Tổng quát,
\begin{center}
	$\sin f(x)=\sin g(x)\Leftrightarrow\hoac{&f(x)=g(x)+k2\pi\\&f(x)=\pi - g(x)+k2\pi}$ ($k\in\mathbb{Z}$).
\end{center}
\item Một số trường hợp đặt biệt:
\begin{enumEX}[\faCheckCircleO]{1}
	\item $\sin x=0\Leftrightarrow x=k\pi$, $k\in\mathbb{Z}$.
	\item $\sin x=1\Leftrightarrow x=\dfrac{\pi}{2}+k2\pi$, $k\in\mathbb{Z}$.
	\item $\sin x=-1\Leftrightarrow x=-\dfrac{\pi}{2}+k2\pi$, $k\in \mathbb{Z}$.
\end{enumEX}
\end{itemize}
	\subsubsection{Phương trình $\cos x=m$}
\begin{itemize}
	\item Với $|m|>1$ thì phương trình $\cos x=m$ vô nghiệm.
	\item Với $|m|\leq 1$, sẽ tồn tại duy nhất $\alpha\in\left[0; \pi\right]$ thỏa mãn $\cos x=m$. Khi đó
	\begin{center}
		$\cos x=m\Leftrightarrow\cos x=\cos \alpha\Leftrightarrow\hoac{&x=\alpha+k2\pi\\&x=-\alpha+k2\pi}$ ($k\in\mathbb{Z}$).
	\end{center}
\item Nếu số đo của góc $\alpha$ được đo bằng đơn vị độ thì
\begin{center}
	$\cos x=\cos\alpha\Leftrightarrow\hoac{&x=\alpha^\circ+k360^\circ\\&x=-\alpha^\circ+k360^\circ}$ ($k\in\mathbb{Z}$).
\end{center}
\item Tổng quát,
\begin{center}
	$\cos f(x)=\cos g(x)\Leftrightarrow\hoac{&f(x)=g(x)+k2\pi\\&f(x)=-g(x)+k2\pi}$ ($k\in \mathbb{Z}$)
\end{center}
\item Một số trường hợp đặc biệt:
\begin{enumEX}[\faCheckCircleO]{1}
\item $\cos x=0\Leftrightarrow x=\dfrac{\pi}{2}+k\pi$, $k\in\mathbb{Z}$.
\item $\cos x=1\Leftrightarrow x=k2\pi$, $k\in\mathbb{Z}$.
\item $\cos x=-1\Leftrightarrow x=\pi+k2\pi$, $k\in\mathbb{Z}$.
\end{enumEX}
\end{itemize}
\subsubsection{Phương trình $\tan x=m$}
\begin{itemize}
	\item Với mọi $m\in\mathbb{R}$, tồn tại duy nhất $\alpha\in\left(-\dfrac{\pi}{2};\dfrac{\pi}{2}\right)$ thỏa mãn $\tan \alpha=m$. Khi đó
	\begin{center}
		$\tan x=m\Leftrightarrow\tan x=\tan \alpha\Leftrightarrow x=\alpha+k\pi$ ($k\in\mathbb{Z}$).
	\end{center}
\item Nếu số đo của góc $\alpha$ được đo bằng đơn vị độ thì
\begin{center}
	$\tan x=\tan\alpha^\circ\Leftrightarrow x=\alpha^\circ+k180^\circ$, $k\in \mathbb{Z}$
\end{center}
\item Tổng quát,
\begin{center}
	$\tan f(x)=\tan g(x)\Leftrightarrow f(x)=g(x)+k\pi$, $k\in\mathbb{Z}$.
\end{center}
\end{itemize}
\subsubsection{Phương trình $\cot x=m$}
\begin{itemize}
	\item Với mọi $m\in\mathbb{R}$, tồn tại duy nhất $\alpha\in\left(0;\pi\right)$ thỏa mãn $\cot\alpha=m$. Khi đó
	\begin{center}
		$\cot x=m\Leftrightarrow\cot x=\cot\alpha\Leftrightarrow x=\alpha+k\pi$ $k\in\mathbb{Z}$.
	\end{center}
\item Nếu số đo của góc $\alpha$ được đo bằng đơn vị độ thì
\begin{center}
	$\cot x=\cot\alpha^\circ\Leftrightarrow x=\alpha^\circ+k180^\circ$, $k\in\mathbb{Z}$.
\end{center}
\item Tổng quát,
\begin{center}
	$\cot f(x)=\cot g(x)\Leftrightarrow f(x)=g(x)+k\pi$, $k\in\mathbb{Z}$.
\end{center}
\end{itemize}
\end{tomtat}

 \foreach \i in {2,3,4,5,6,7,11} {\input{data/11KNTT/data1/1K1-4-\i.tex}}


% ---------Mục lục chính
\FULLWIDTH
\tableofcontents %lệnh in mục lục chính
\begin{center}
\includegraphics[width=5cm]{QRcode/11D1X4.png}
\end{center}
\end{document}