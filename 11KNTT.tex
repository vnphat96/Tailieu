\documentclass[10pt,a4paper,onecolumn,titlepage,twoside,openany]{book}
% \usepackage[utf8]{vietnam}
%\usepackage{fouriernc}
\usepackage{tasks}
\usepackage{xcolor} 
%%%%%%%%%%%%% KIỂU MÀU VÀ ĐỘ RỘNG NOTE
% \def\kieumau{N} %Y: Màu; N: đen-trắng
\def\kieumau{Y} %Y: Màu; N: đen-trắng
\def\leftnote{5} %Độ rộng cột Note
%%%%%%%%%%%%% ĐN CƠ BẢN
\input{cautrucDT/color\kieumau} %MÀU
%=====================================
% Khai báo nhóm Tex (cơ bản)
%=====================================
\usepackage{amsmath,amssymb,mathrsfs,maybemath,xlop,polynom,slashbox}
\usepackage{yhmath} %\let\widering\relax %cần khi sd với font fouriernc

\usepackage{enumerate}
\usepackage{tikz} 
\usepackage{tkz-euclide}
%\usepackage{ex_tkz-euclide}
%\usetkzobj{all}
\usepackage{tikz-3dplot}
\usepackage{tkz-tab}
\usepackage{pifont} %kí hiệu đặc biệt
% \usepackage{xcolor}
%\usepackage{bbding}
%\usepackage{array}
\usepackage{tasks}
% \usepackage{casiovn}
%==========
\usetikzlibrary{math,through,calc,intersections,angles,quotes,shapes,shapes.geometric,arrows,patterns,snakes,matrix,chains,arrows.meta,decorations.shapes,decorations.fractals,decorations.markings,shadows}
\usetikzlibrary{positioning,decorations.text,decorations.pathmorphing}% Để uốn cong văn bản 
\usetikzlibrary{shadings,fadings} %ĐỔ BÓNG
\usepackage{pgfplots}
\usepackage{pgfornament}
\usepgfplotslibrary{fillbetween}
\pgfplotsset{compat=1.9}
\usepackage[hidelinks,unicode]{hyperref}
\usepackage{currfile}
\usepackage[outline]{contour} %viền
\usepackage{fontawesome} % Gói kí hiệu
\usepackage{lipsum} %Lấy text
\usepackage{tabularx}
%%---------
%\usepackage{setspace}
%\usepackage{scrextend}
\usepackage{varwidth}
%===========Bảng
\usepackage{longtable,multirow,makecell}
\usepackage{diagbox}
\renewcommand{\tabcolsep}{3mm}
\newcolumntype{C}[1]{>{\centering\arraybackslash}p{#1}}
\newcolumntype{L}[1]{>{\raggedright\arraybackslash}p{#1}}
%-----------Trang vb


%%%%%%%%%%%%% Các thông số trang tài liệu
\def\tren{1.5}\def\duoi{1.5}\def\trai{1.25}\def\phai{0.75} %cách lề
\def\topset{0.75} %kc giữa đáy header và vùng vb
\def\botset{0.75} %kc giữa đỉnh footer và vùng vb
%\usepackage{ifoddpage}
\pgfmathsetmacro{\mepphai}{\phai+\leftnote} 
%\usepackage[top=\tren cm, bottom=\duoi cm, left=\trai cm, right=\mepphai cm] {geometry}
%%%%%%%%%%%%%
%---------------------------------Các thông số trang tài liệu
\pgfmathsetmacro{\so}{\leftnote - 0.5} 
\usepackage[top=\tren cm, bottom=\duoi cm, left=\trai cm, right=\mepphai cm,
marginparwidth=\so cm, marginparsep=5mm,
%,headsep=6mm
%,footskip=10mm
] {geometry}
%-------------------------------------
\usepackage{marginnote}
\setlength{\marginparwidth}{\so cm}
\renewcommand*{\marginfont}{\small}
%--------------Gói trắc nghiệm EX-TEST
% \usepackage[dethi]{ex_test}
\usepackage[loigiai]{ex_test} 
% \usepackage[solcolor]{ex_test}
%----Lời giải, Hiền thị tên EX; Dấu kết thúc
\font\damEX=ugqb8v at 11pt
\def\loigiaiEX{\color{\mauLG}\damEX\strut\faCommenting\ Lời giải.}
%lời giải EXS
\def\loigiaiEXS{\loigiaiEX{\fontsize{8}{16}\selectfont\color{\maucham}\dotfill}}
%--
\renewcommand{\nameex}{\damEX\color{\mauEX} CÂU}
\newtheorem{EX}{\nameex} %MÔI TRƯỜNG PHỤ CHO TÁCH CÂU
\def\mauVuong{cyan}
\def\qedEX{\color{\mauVuong}\ensuremath{\square}}
%--------------Cài đặt lại dòng kẻ \dotline
\renewcommand{\dotlineEX}[1]{
	\def\numlinedot{#1}
	\par
	\foreach \dotline in{1,...,\numlinedot}
	{
		\noindent
		\fontsize{8}{16}\selectfont
		\color{\maucham}\dotfill
		\par
	}
}
% sd cho \dongcham
\newcommand{\dotlineEXS}[1]{
	\def\numlinedot{#1}
	\foreach \dotline in{1,...,\numlinedot}
	{
		\noindent
		\fontsize{8}{16}\selectfont
		\color{\maucham}\dotfill
		\par
	}
}
%---------- Khai báo viết tắt, in đáp án
\newcommand{\hoac}[1]{ %hệ hoặc
	\left[\begin{aligned}#1\end{aligned}\right.}
\newcommand{\heva}[1]{ %hệ và
	\left\{\begin{aligned}#1\end{aligned}\right.}
%--In đáp án
\newcommand{\indapan}[2]{
	\addcontentsline{toc}{subsection}{\sf Bảng đáp án} % đưa MT vào mục lục
	\begin{center}
		\begin{tikzpicture}%
			\node[thick,scale=1,fill=\mauEX!2,draw=\maufoot,minimum width=3.5cm,minimum height=0.1cm,rounded corners=2mm]{\fontfamily{qag}\fontsize{11}{11}\selectfont\bfseries\color{\mauEX} BẢNG ĐÁP ÁN};
		\end{tikzpicture}%
	\end{center}
	\inputansbox{#1}{#2}
}
%----------
\usepackage{esvect}
\def\vec{\vv} %vecto
\def\overrightarrow{\vv}
%Lệnh song song
\DeclareSymbolFont{symbolsC}{U}{txsyc}{m}{n}
\DeclareMathSymbol{\varparallel}{\mathrel}{symbolsC}{9}
\DeclareMathSymbol{\parallel}{\mathrel}{symbolsC}{9}
%--------------------------
% HEADER AND FOOTER STYLING
%--------------------------
%--------------------------
\newcommand{\myfancyhead}{% trên và chấm trái
		\boldmath
\begin{tikzpicture}[remember picture,overlay,>=stealth]
		\path ([yshift=-\tren cm+0.5*\topset cm]current page.north west) coordinate (AA)
		++(\paperwidth,0)coordinate (BB); 
\checkoddpage\ifoddpage %nếu trang lẻ
		%-----đường kẻ
		\draw[\maufoot, line width=2pt] 
		([xshift=\trai cm]AA) --([xshift=-\phai cm]BB);
		%-----bên phải
		\node[text=\maufoot, anchor=south east,inner sep=0pt] at ([xshift=-\phai cm,yshift=4pt]BB){
			\fontfamily{qag}\fontsize{8.5pt}{12pt}\selectfont 
			{\color{\mauSO}\faMapMarker}\,\, \diachi\,\,{\color{\mauSO}\faMapMarker}
		};
		%-----bên trái
		\node[text=\maufoot, anchor=south west,inner sep=0pt] at ([xshift=\trai cm,yshift=4pt]AA){
			\fontfamily{qag}\fontsize{10pt}{10pt}\selectfont\bfseries\faEdit\, \tenchuyende
		};
		%----- Kẻ đứng
		\draw[\maufoot] ([xshift=-\mepphai cm+2.5mm]BB)--([yshift=\duoi cm-0.75*\botset cm,xshift=-\mepphai cm+2.5mm]current page.south east);
		%--		
		\path ([yshift=-\tren cm+0.5*\topset cm-0.5cm,xshift=-\phai cm-0.5*\leftnote cm+2.5mm]current page.north east) coordinate (DDD); 
		\begin{scope}
			\clip ([yshift=-\tren cm+0.5*\topset cm-1pt,xshift=-\phai cm]current page.north east) rectangle ([yshift=\duoi cm-0.5*\botset cm,xshift=-\mepphai cm+5mm]current page.south east);% cắt chấm
			\node[inner sep =0pt,scale=1,anchor=north] at ([yshift=0cm,xshift=0pt]DDD) {
				\parbox{\leftnote cm}{\centering
					\def\maucham{\maufoot}\dotlineEX{60}
				}
			};
			%--note dưới
			\node[inner sep =6pt, text=white,scale=1,anchor=north,fill=\maufoot] (noteduoi) at ([yshift=2.5mm]DDD) {
				\parbox{\leftnote cm-5mm-12pt}{ \fontsize{11}{1}\fontfamily{qag}\selectfont\bfseries\centering
					QUICK NOTE
				}
			};
			\draw[\maufoot, line width=0.4pt] ([yshift=-2pt]noteduoi.south west)--([yshift=-2pt]noteduoi.south east);
		\end{scope}
\else %chẵn
		%-----đường kẻ
		\draw[\maufoot, line width=2pt] 
		([xshift=\phai cm]AA) --([xshift=-\trai cm]BB);
		%-----bên trái
		\node[text=\maufoot, anchor=south west,inner sep=0pt] at ([xshift=\phai cm,yshift=4pt]AA){
			\fontfamily{qag}\fontsize{8.5pt}{12pt}\selectfont 
			{\color{\mauSO}\faMapMarker}\,\, \diachi\,\,{\color{\mauSO}\faMapMarker}
		};
		%-----bên phải
		\node[text=\maufoot, anchor=south east,inner sep=0pt] at ([xshift=-\trai cm,yshift=4pt]BB){
			\fontfamily{qag}\fontsize{10pt}{10pt}\selectfont\bfseries\faEdit\, \tenchuyende
		};
		%----- Kẻ đứng
		\draw[\maufoot] ([xshift=\mepphai cm-2.5mm]AA)--([yshift=\duoi cm-0.75*\botset cm,xshift=\mepphai cm-2.5mm]current page.south west);
		%--		
		\path ([yshift=-\tren cm+0.5*\topset cm-0.5cm,xshift=\phai cm+0.5*\leftnote cm-2.5mm]current page.north west) coordinate (DDD); 
		\begin{scope}
			\clip ([yshift=-\tren cm+0.5*\topset cm-1pt,xshift=\phai cm]current page.north west) rectangle ([yshift=\duoi cm-0.5*\botset cm,xshift=\mepphai cm-5mm]current page.south west);% cắt chấm
			\node[inner sep =0pt,scale=1,anchor=north] at ([yshift=0cm,xshift=0pt]DDD) {
				\parbox{\leftnote cm}{\centering
					\def\maucham{\maufoot}\dotlineEX{60}
				}
			};
			%--note dưới
			\node[inner sep =6pt, text=white,scale=1,anchor=north,fill=\maufoot] (noteduoi) at ([yshift=2.5mm]DDD) {
				\parbox{\leftnote cm-5mm-12pt}{ \fontsize{11}{1}\fontfamily{qag}\selectfont\bfseries\centering
					QUICK NOTE
				}
			};
			\draw[\maufoot, line width=0.4pt] ([yshift=-2pt]noteduoi.south west)--([yshift=-2pt]noteduoi.south east);
		\end{scope}
\fi
\end{tikzpicture}%
}
% trên mục lục
\newcommand{\headmucluc}{%
	\boldmath
\begin{tikzpicture}[remember picture,overlay,>=stealth]
	\path ([yshift=-\tren cm+0.5*\topset cm]current page.north west) coordinate (AA)
	++(\paperwidth,0)coordinate (BB); 
\checkoddpage\ifoddpage %nếu trang lẻ
	%-----đường kẻ
	\draw[\maufoot, line width=2pt] 
	([xshift=\trai cm]AA) --([xshift=-\phai cm]BB);
	%-----bên phải
	\node[text=\maufoot, anchor=south east,inner sep=0pt] at ([xshift=-\phai cm,yshift=4pt]BB){
		\fontfamily{qag}\fontsize{8.5pt}{12pt}\selectfont 
		{\color{\mauSO}\faMapMarker}\,\, \diachi\,\,{\color{\mauSO}\faMapMarker}
	};
	%-----bên trái
	\node[text=\maufoot, anchor=south west,inner sep=0pt] at ([xshift=\trai cm,yshift=4pt]AA){
		\fontfamily{qag}\fontsize{12pt}{12pt}\selectfont\bfseries\faEdit\, \tenchuyende
	};
\else %chẵn
	%-----đường kẻ
	\draw[\maufoot, line width=2pt] 
	([xshift=\phai cm]AA) --([xshift=-\trai cm]BB);
	%-----bên trái
	\node[text=\maufoot, anchor=south west,inner sep=0pt] at ([xshift=\phai cm,yshift=4pt]AA){
		\fontfamily{qag}\fontsize{8.5pt}{12pt}\selectfont 
		{\color{\mauSO}\faMapMarker}\,\, \diachi\,\,{\color{\mauSO}\faMapMarker}
	};
	%-----bên phải
	\node[text=\maufoot, anchor=south east,inner sep=0pt] at ([xshift=-\trai cm,yshift=4pt]BB){
		\fontfamily{qag}\fontsize{12pt}{12pt}\selectfont\bfseries\faEdit\, \tenchuyende
	};
\fi
\end{tikzpicture}%
}
%===========================
\newcommand{\myfancyfoot}{% dưới
	\begin{tikzpicture}[remember picture,overlay]
	\path ([yshift=\duoi cm-0.75*\botset cm]current page.south west) coordinate (AA)
	++(\paperwidth,0)coordinate (BB); 
	\checkoddpage\ifoddpage %nếu trang lẻ
		%---kẻ
		\draw[\maufoot, line width=2pt] ([xshift=2*\trai cm+4pt]AA)--([xshift=-\phai cm+3pt]BB);
		%-----bên trái
		\fill[fill=\maufoot, rounded corners=2mm] ([xshift=2*\trai cm,yshift=0.25 cm]AA) rectangle +(-3*\trai cm,-0.5cm);
		%-----trang
		\node[anchor=west,text=white,inner sep=0pt,xshift=-0.75cm] at ([xshift=2*\trai cm]AA) {\fontfamily{put}\bfseries\thepage};
		%-----tên tg
		\node[anchor=west,text=\maufoot,inner sep=0pt,fill=white] at ([xshift=2*\trai cm]AA){\fontfamily{qag}\fontsize{9pt}{1pt}\selectfont\bfseries \,\,\, \tentacgia \,\,\, };
	\else %chẵn
		%---kẻ
		\draw[\maufoot, line width=2pt] ([xshift=-2*\trai cm+4pt]BB)--([xshift=\phai cm-3pt]AA);
		%-----bên trái
		\fill[fill=\maufoot, rounded corners=2mm] ([xshift=-2*\trai cm,yshift=0.25 cm]BB) rectangle +(3*\trai cm,-0.5cm);
		%-----trang
		\node[anchor=east,text=white,inner sep=0pt,xshift=0.75cm] at ([xshift=-2*\trai cm]BB) {\fontfamily{put}\bfseries\thepage};
		%-----tên tg
		\node[anchor=east,text=\maufoot,inner sep=0pt,fill=white] at ([xshift=-2*\trai cm]BB){\fontfamily{qag}\fontsize{9pt}{1pt}\selectfont\bfseries \,\,\, \tentacgia \,\,\,};
	\fi
	\end{tikzpicture}%
}
%======================Head chapter theo note, fullwidth
%----------------------
\usepackage{changepage}
\strictpagecheck
\usepackage{lastpage}
\usepackage{fancyhdr,lastpage}
\pagestyle{fancy}
\fancyhf{}
\fancypagestyle{plain}{
	\fancyhead[LO,RE]{\headmucluc}
	\fancyfoot[LO,RE]{\myfancyfoot}
}
\fancyhead[LO,RE]{\myfancyhead}
\fancyfoot[LO,RE]{\myfancyfoot}
\renewcommand{\footrulewidth}{0pt}
\renewcommand{\headrulewidth}{0pt}
%--------------4.2
\usepackage[most]{tcolorbox}
\colorlet{tcbcol@back}{tcbcolback}
\colorlet{tcbcol@frame}{tcbcolframe}
%---------------------------------------------------------------
% ĐỊNH NGHĨA SECTION. SUBSECTION, SUBSUBSECTION ... THEO Ý RIÊNG
%---------------------------------------------------------------
\usepackage[explicit]{titlesec} % để gọi #1
\usepackage{titledot} % gói lệnh chứa cả titlesec và titletoc
%=====================================
\setcounter{secnumdepth}{4} %độ sâu
\renewcommand{\thechapter}{\Roman{chapter}}
\renewcommand{\thesection}{\arabic{section}}
\renewcommand{\thesubsection}{\Alph{subsection}}
\renewcommand{\thesubsubsection}{\arabic{subsubsection}}
%--------------Tròn
\newcommand{\tron}[1]{
	\begin{tikzpicture}[baseline=(A.base)]%
		\node[circle,draw=\mauSUBSEC,line width=0.5pt,fill=white,inner sep=2pt,outer sep=1pt] (A) {\color{white} #1};
		\node[circle,draw=none,fill=\mauSUBSEC,inner sep=1pt,outer sep=1pt] (A) {\color{white} #1};
	\end{tikzpicture}%
}
%================= Đn chương
\font\fontchap=ugqb8v at 21pt
\titlespacing{\chapter}{0cm}{0cm}{0.5cm}[0cm] %1: , 2: Trên, 3: dưới
\titleformat{\chapter}[display]
{\fontsize{20pt}{20pt}\fontfamily{qag}\selectfont\bfseries\color{\mauCHUONG}} %định dạng chung
{\fontsize{16pt}{20pt}\selectfont\chaptername\, \thechapter.} %đánh số
{1mm}
{\fontchap\centering\MakeUppercase{#1}}
[\vspace{0cm}]
%============================Mục lục - Chapter*
\titleformat{name=\chapter,numberless}[display]
{\fontsize{14pt}{16pt}\fontfamily{qag}\selectfont\bfseries\color{\mauCHUONG}} %định dạng chung
{}
{-1em}
{%
	\begin{tikzpicture}
		%-----Nội dung
		\node[inner sep=0pt,right] (ndchuong) at (0,0){\fontchap \MakeUppercase{#1}};
%		%-----Đường kẻ ngang
%		\begin{scope}
%			\clip (0,-0.75) rectangle +(\textwidth,1.5);
%			\draw[\mauchuong,line width=2pt] (ndchuong.south east)++(10pt,8pt) --++(\linewidth,0);
%		\end{scope}
	\end{tikzpicture}
}
[
\vspace{-3mm}
%\thispagestyle{empty}
]
%--------Đn Section---------------------------
%\titlespacing*{\section}{0cm}{0cm}{0cm}[0cm]
\titleformat
{\section}
{\color{\mauSEC}\fontfamily{qag}\fontsize{16pt}{1pt}\selectfont\bfseries\centering}
{Bài\,\thesection.}
{3mm}
{\MakeUppercase{#1}}
[]
%-------Đn subsection---------------------------
\titlespacing{\subsection}{0cm}{0cm}{0cm}[0cm]
\titleformat{\subsection}
{\normalfont\fontsize{15pt}{20pt}\fontfamily{put}\selectfont\bfseries\color{\mausubsec}}
{\thesubsection.}
{3mm}
{\MakeUppercase{#1}}
[]
%----------ĐN subsubsection-----------------------
\titlespacing{\subsubsection}{0pt}{0mm}{0mm}[0cm]
\titleformat{\subsubsection}
{\fontsize{13pt}{18pt}\fontfamily{put}\selectfont\bfseries\color{\mausubsubsec}}
{\thesubsubsection.}
{3mm}
{#1}
[]
%----------ĐN paragraph-----------------------
\titlespacing{\paragraph}{0pt}{0mm}{0mm}[0cm]
\titleformat{\paragraph}
{\fontsize{11.5pt}{17pt}\fontfamily{put}\selectfont\bfseries\color{\mausubsubsec}}
{\theparagraph.}
{3mm}
{#1}
[]
%============================
\def\itemKN{\color{\mauitemKN}\faCheckSquareO}
\def\itemCI{\color{\mauitemCI}\faCheckCircleO}
%%======= Thiết lập labelitem, labelenumerate
%\renewcommand{\labelitemi}{\color{red}\faCheckSquareO}
\renewcommand{\labelitemi}{\color{\mauitem}\faCheckCircleO}
\renewcommand{\labelitemii}{\color{\mauitem}\bf ---}
\renewcommand{\labelitemiii}{\color{\mauitem}\bf +}
\renewcommand{\labelenumi}{\alph{enumi})}
%\renewcommand{\labelenumii}{\color{blue}\bf\arabic{enumi}.\arabic{enumii}}
%============================
%============================
% Canh chỉnh mục lục chính
\setcounter{secnumdepth}{4} %Độ sâu đánh số
\setcounter{tocdepth}{2} %Độ sâu mục lục
\contentsmargin{0cm}
%~~~~~~~~~~~~~~~~~~~~~
\renewcommand*\l@part[2]{%
	\ifnum \c@tocdepth >-2\relax
	\addpenalty{-\@highpenalty}%
	\addvspace{10pt \@plus\p@}%
	\setlength\@tempdima{3em}%
	\begingroup
	\hypersetup{linkcolor=violet}
	\tikz[remember picture, overlay]{
		\fill[\mauPHAN] (0,0) rectangle +(\textwidth,1);
		\draw (0,0.5) node[right=5pt]
		{\color{white}\fontsize{16pt}{1pt}\fontfamily{qag}\selectfont\bfseries  {\scshape Phần} #1};
	}
	\par\smallskip
	\penalty\@highpenalty
	\endgroup
	\fi
}
%------------------------
%\titlecontents{part}[0pc]
%{\addvspace{10pt}%
%	\color{red!70!black}\fontsize{18pt}{1pt}\fontfamily{qag}\selectfont\bfseries 
%}%
%{}
%{}
%{}%
%~~~~~~~~~~~~~~~~~~~~~
\titlecontents{chapter}[6.5pc] % nd cách trái
{\addvspace{5pt}%
	\color{\mauCHUONG}\fontsize{13pt}{16pt}\fontfamily{put}\selectfont\bfseries
}%
{\contentslabel[\chaptertitlename\,\thecontentslabel.]{6.5pc}} %nhãn
{}
{\hfill\bfseries\thecontentspage
}%
[\vspace*{5pt}]
%~~~~~~~~~~~~~~~~~~~~~
\titlecontents{section}[10pc]
{\addvspace{0pt}\bfseries\color{\mauSEC}}
{\fontsize{12.5pt}{15pt}\selectfont\sffamily\contentslabel[{Bài\,\thecontentslabel.}]{3.5pc}}
{}
{\hfill
	\thecontentspage
}
[]
%~~~~~~~~~~~~~~~~~~~~~
\titlecontents{subsection}[10pc]
{\addvspace{0pt}\color{\mauSUBSEC}}
{\fontsize{12pt}{15pt}\selectfont\sffamily\contentslabel[\tron{\thecontentslabel}]{2.8pc}}
{}
{{\tiny\dotfill}\thecontentspage}
[]
%~~~~~~~~~~~~~~~~~~~~~
%--------------------------
% ĐỊNH NGHĨA CÁC MÔI TRƯỜNG 
%--------------------------
\listenumerate{dn,dl,tc,nx,ex}%xuống dòng khi liệt kê
\theoremstyle{plain} %
\theoremheaderfont{\scshape} %đầu
\theorembodyfont{\normalfont} % thân
\theoremseparator {.} % Ngăn cách
\newtheorem{dn}{\color{\maudn}\faBolt\, Định nghĩa}[section]
%===================================ĐNghĩa
\theoremstyle{plain} %
\theoremheaderfont{\fontfamily{put}\bfseries} %đầu
\theorembodyfont{\normalfont} % thân
\theoremseparator {.} % Ngăn cách
\newtheorem{vd}{\color{\mauVD}\damEX
	%\faToggleOn\ 
	%\faUnlink\ 
	VÍ DỤ}%[section]
\newtheorem{bt}{\color{\mauBT}\damEX BÀI}
%===================================
\theoremstyle{plain} %
\theoremheaderfont{\scshape} %đầu
\theorembodyfont{\slshape} % thân 
\theoremseparator {.} % Ngăn cách 
\newtheorem{dl}{\color{\maudl}\faBolt\, Định lí}[section]
\newtheorem{tc}{\color{\mauhq!70!black}\faBolt\, Tính chất}[section]
\newtheorem{hq}{\color{\mauhq!70!black}\faBolt\, Hệ quả}[section]
%====================================
\theoremstyle{nonumberplain} %ko đánh số, ko xuống dòng
\theoremheaderfont{\scshape} %đầu
\theorembodyfont{\normalfont} %phần thân
\theoremseparator {.} %ngăn cách
\newtheorem{nx}{\color{\mauhq!70!black}\faBolt\, Nhận xét}
\newtheorem{tomtat}{\!\!\!\!\!\!\!\!}
%====================================Hộp
%--------------------Chú ý
\newenvironment{note}
{\begin{tcolorbox}
		[enhanced jigsaw,breakable,pad at break*=1mm,
		opacityback=0,boxrule=0pt,frame hidden,
		left=8mm, right=0pt, bottom=0pt, top=0pt,
		before skip=1mm,
		after skip=1mm,
		underlay unbroken and first={
			\draw ([xshift=0.3cm,yshift=-0.32cm]interior.north west) node[\mauly]{\large\bfseries \faExclamationTriangle};
		},
		fontupper=\it,
		]}
	{\end{tcolorbox}}
%\let\mynote\note
%\renewcommand{\note}{\mynote{\bfseries\color{\mauly}Lưu ý:}} 
%---------------Dạng toán
\newcounter{dang}\setcounter{dang}{0}
\renewcommand{\thedang}{\arabic{dang}}
%---Dạng 1
\newtcolorbox{dang}[1]{
	fonttitle=\fontfamily{qag}\bfseries,%fontupper=\itshape,
	colframe=\maudang,colback=yellow!20,coltitle=white,
	sharp corners, breakable, halign title=center,%adjusted title=center, %canh giữa DẠNG
	before skip=2mm,after skip=3mm,
	left=2mm,right=2mm,top=2mm,bottom=2mm,
	boxrule=1pt,
	title={\faFolderOpen\ Dạng~\stepcounter{dang}\thedang.\ #1}
		\addcontentsline{toc}{subsection}{\it\sffamily \faFolderOpen\ Dạng~\thedang.#1}
		\setcounter{subsubsection}{0}
		\setcounter{vd}{0}
		\setcounter{ex}{0}
		\setcounter{bt}{0}
}
%====================
\setlength{\parindent}{0pt} %không thụt đầu dòng
%--Name
\newcounter{deso}
\font\dam=ugqb8v at 13pt
\font\damTT=ugqb8v at 18pt
%================đn notenam
\def\notename{
		\begin{tikzpicture}[remember picture,overlay,>=stealth]
		\checkoddpage\ifoddpage %nếu trang lẻ
		%--tiêu đề phải
		\path ([yshift=-\tren cm+0.5*\topset cm-0.5cm,xshift=-\phai cm-0.5*\leftnote cm+2.5mm]current page.north east) coordinate (DD); 
		%--
		\fill[white] ([yshift=-\tren cm+0.5*\topset cm-5pt,xshift=\trai cm-2pt]current page.north east) rectangle ([yshift=\duoi cm-0.5*\botset cm-12cm,xshift=-\mepphai cm+3mm]current page.north east);
		\node[inner sep =0pt,anchor=north] (thanhta) at ([yshift=-2mm]DD) {
			\includegraphics[width=4.5cm]{logo/logo.jpg}		
		};	
		\else
		%--tiêu đề phải
		\path ([yshift=-\tren cm+0.5*\topset cm-0.5cm,xshift=\phai cm+0.5*\leftnote cm-2.5mm]current page.north west) coordinate (DD); 
		%--
		\fill[white] ([yshift=-\tren cm+0.5*\topset cm-5pt,xshift=\phai cm-2pt]current page.north west) rectangle ([yshift=\duoi cm-0.5*\botset cm-12cm,xshift=\mepphai cm-3mm]current page.north west);
		\node[inner sep =0pt,anchor=north] (thanhta) at ([yshift=-2mm]DD) {
			\includegraphics[width=4.5cm]{logo/logo.jpg}		
		};
		\fi
		%\draw (thanhta.south) node[below=0pt,xscale=0.8]{\small\normalfont\color{\mauname} Sưu tầm \& Biên tập};
		%---note
		\node[inner sep =6pt, text=black,scale=1,anchor=north,fill=\maufoot!3,draw=\maufoot] (bon) at ([yshift=-1cm]thanhta.south) {
			\parbox{\leftnote cm-5mm-12pt}{ \fontsize{10}{15}\selectfont\normalfont
				\vspace*{2pt}
				\chamngon%
			}
		};
		\draw[\maufoot, line width=5pt] (bon.north west)--(bon.north east);
		\draw[\maufoot!50] ([yshift=9pt,line width=0.4pt]bon.north east)--([yshift=9pt]bon.north west)
		node[fill=white,inner sep=2pt,anchor=south west,yshift=-2pt,xshift=-2pt]{\bfseries\color{\maufoot}ĐIỂM:}
		;
		%--note dưới
		\node[inner sep =6pt, text=white,scale=1,anchor=north,fill=\maufoot] (noteduoi) at ([yshift=-0.25cm]bon.south) {
			\parbox{\leftnote cm-5mm-12pt}{ \fontsize{11}{1}\selectfont\bfseries\centering
				QUICK NOTE
			}
		};
		\draw[\maufoot, line width=0.4pt] ([yshift=-2pt]noteduoi.south west)--([yshift=-2pt]noteduoi.south east);
	\end{tikzpicture}
}
%===================note và nonote
%FULL WIDTH
\def\FULLWIDTH{
	\newpage
	\fancyhead[LO,RE]{\headmucluc}
	\def\notename{}
	\newgeometry{top=\tren cm, bottom=\duoi cm, left=\trai cm, right=\phai cm}
}
\def\NOTE{
	\newpage
	\fancyhead[LO,RE]{\myfancyhead}
	\def\notename{
		\begin{tikzpicture}[remember picture,overlay,>=stealth]
			\checkoddpage\ifoddpage %nếu trang lẻ
			%--tiêu đề phải
			\path ([yshift=-\tren cm+0.5*\topset cm-0.5cm,xshift=-\phai cm-0.5*\leftnote cm+2.5mm]current page.north east) coordinate (DD); 
			%--
			\fill[white] ([yshift=-\tren cm+0.5*\topset cm-5pt,xshift=\trai cm-2pt]current page.north east) rectangle ([yshift=\duoi cm-0.5*\botset cm-12cm,xshift=-\mepphai cm+3mm]current page.north east);
			\node[inner sep =0pt,anchor=north] (thanhta) at ([yshift=-2mm]DD) {
				\includegraphics[width=4.5cm]{logo/logo.jpg}		
			};	
			\else
			%--tiêu đề phải
			\path ([yshift=-\tren cm+0.5*\topset cm-0.5cm,xshift=\phai cm+0.5*\leftnote cm-2.5mm]current page.north west) coordinate (DD); 
			%--
			\fill[white] ([yshift=-\tren cm+0.5*\topset cm-5pt,xshift=\phai cm-2pt]current page.north west) rectangle ([yshift=\duoi cm-0.5*\botset cm-12cm,xshift=\mepphai cm-3mm]current page.north west);
			\node[inner sep =0pt,anchor=north] (thanhta) at ([yshift=-2mm]DD) {
				\includegraphics[width=4.5cm]{logo/logo.jpg}		
			};
			\fi
			%\draw (thanhta.south) node[below=0pt,xscale=0.8]{\small\normalfont\color{\mauname} Sưu tầm \& Biên tập};
			%---note
			\node[inner sep =6pt, text=black,scale=1,anchor=north,fill=\maufoot!3,draw=\maufoot] (bon) at ([yshift=-1cm]thanhta.south) {
				\parbox{\leftnote cm-5mm-12pt}{ \fontsize{10}{15}\selectfont\normalfont
					\vspace*{2pt}
					\chamngon%
				}
			};
			\draw[\maufoot, line width=5pt] (bon.north west)--(bon.north east);
			\draw[\maufoot!50] ([yshift=9pt,line width=0.4pt]bon.north east)--([yshift=9pt]bon.north west)
			node[fill=white,inner sep=2pt,anchor=south west,yshift=-2pt,xshift=-2pt]{\bfseries\color{\maufoot}ĐIỂM:}
			;
			%--note dưới
			\node[inner sep =6pt, text=white,scale=1,anchor=north,fill=\maufoot] (noteduoi) at ([yshift=-0.25cm]bon.south) {
				\parbox{\leftnote cm-5mm-12pt}{ \fontsize{11}{1}\selectfont\bfseries\centering
					QUICK NOTE
				}
			};
			\draw[\maufoot, line width=0.4pt] ([yshift=-2pt]noteduoi.south west)--([yshift=-2pt]noteduoi.south east);
		\end{tikzpicture}
	}
	\newgeometry{top=\tren cm, bottom=\duoi cm, left=\trai cm, right=\mepphai cm}
}
%===================đn name
\newcommand{\name}[4]{
%	\NOTE
%	\newpage
	\setcounter{ex}{0}\setcounter{bt}{0}%\setcounter{EX}{0}
	\boldmath\fontfamily{qag}\selectfont\color{\mauname}
\hoten \dotfill {\fontsize{10}{11}\selectfont \ngaylamde}
	\begin{tcolorbox}[boxrule=0.7pt,arc=0mm,breakable,colframe=\mauSO,colback=\mauname!2,before skip=2mm,after skip=2mm]\color{\mauname}
	\begin{center}
		%%---
		{\damTT \MakeUppercase{#1}}\\[1pt]
		{\dam \MakeUppercase{#2 --- Đề} \stepcounter{deso}\thedeso}\\[1pt]
		% {\dam \MakeUppercase{#2}}\\[1pt]
		{\dam\color{\mauSO} \MakeUppercase{#3}}\\[1pt]
		{\fontsize{10}{10}\selectfont \textit{#4}}%\\[-1mm]
	\end{center}
	\end{tcolorbox}
	%%--- Phần note đầu đề
	\notename
\vspace*{0.5cm}
	\addcontentsline{toc}{section}{\hspace*{-4.2cm}\sf Đề \thedeso: #2 --- #3} % đưa MT vào mục lục
}
%--Sang trang 
\BeforeBeginEnvironment{name}{
	\ifnum\the\value{deso}>0
	\newpage
	\fi
}
%%---Đánh số trang
%\AtEndEnvironment{name}{
%	\ifnum\the\value{deso}=1
%	\pagenumbering{arabic}%đánh số trang dạng 1,2,...
%	\fi
%}
%---------
\def\chap#1{
	\begin{center}
		\fontchap\color{\mauCHUONG} #1
	\end{center}
	\addcontentsline{toc}{chapter}{\hspace*{-2.75cm}#1}
}
%---Hiện bảng ĐA
\newcommand{\hienDA}{
	\renewcommand{\indapan}[2]{
		\addcontentsline{toc}{subsection}{\hspace*{-4.2cm}\sf Bảng đáp án} % đưa MT vào mục lục
		%		\begin{center}
		\par\vspace*{5mm}
		\begin{tikzpicture}%
			\draw (0,0)++(0.5*\textwidth,0) node[thick,scale=1,fill=\mauEX!2,draw=\maufoot,minimum width=3.5cm,minimum height=0.1cm,rounded corners=2mm] {\damEX\color{\mauname} BẢNG ĐÁP ÁN};
		\end{tikzpicture}%
		%		\end{center}
		\vspace*{-2mm}
		\inputansbox{##1}{##2}
	}
}
%---Ẩn bảng ĐA
\newcommand{\anDA}{
	\renewcommand{\indapan}[2]{}
}
%---Dòng chấm từng câu
\newcommand{\dongchamEX}[1]{
%	\hideansEX{ex}
	\anLG
	\AfterEndEnvironment{ex}{%
		\foreach \cauEX/\dongEX in {#1}{
			\ifnum\dongEX=0
			\else
			\ifnum\the\value{ex}=\cauEX
			\par\noindent\loigiaiEXS\par
			\dotlineEX{\dongEX}
			\fi
			\fi
		}
	}
}
%---Dòng chấm nhiều câu
\newcommand{\dongchamEXS}[2]{
%	\hideansEX{ex}
	\anLG
	\AfterEndEnvironment{ex}{%
		\foreach \cauEX in {#1}{
			\ifnum#2=0
			\else
			\ifnum\the\value{ex}=\cauEX
			\par\noindent\loigiaiEXS\par
			\dotlineEXS{#2}
			\fi
			\fi
		}
	}
}
%---Dòng chấm từng câu theo đề
\newcommand{\DEdongchamEX}[2]{
%	\hideansEX{ex}
	\anLG
	\AfterEndEnvironment{ex}{%
		\foreach \cauEX/\dongEX in {#2}{
			\ifnum\dongEX=0
			\else
			\ifnum\the\value{deso}=#1
			\ifnum\the\value{ex}=\cauEX
			\par\noindent\loigiaiEXS\par
			\dotlineEX{\dongEX}
			\fi
			\fi
			\fi
		}
	}
}
%---Dòng chấm nhiều câu theo đề
\newcommand{\DEdongchamEXS}[3]{
%	\hideansEX{ex}
	\anLG
	\AfterEndEnvironment{ex}{%
		\foreach \cauEX in {#2}{
			\ifnum#3=0
			\else
			\ifnum\the\value{deso}=#1
			\ifnum\the\value{ex}=\cauEX
			\par\noindent\loigiaiEXS\par
			\dotlineEX{#3}
			\fi
			\fi
			\fi
		}
	}
}
%---Ẩn LG
\newcommand{\anLG}{
	\renewcommand{\loigiai}[1]{	}%
	% \chooseNSA
	\renewcommand{\TrueTF}{\FalseTF}
	\renewcommand{\TrueEX}{\FalseEX}
	\renewcommand{\writekeyTFone}{\gdef\TrueX{}\gdef\FalseX{}}
	\renewcommand{\writekeyTF}{&&}
}
%---Hiện LG
\newcommand{\hienLG}{
	%Xuất hiện chữ Lời giải trong môi trường onlysolution
	\renewcommand{\loigiai}[1]{%
		\begin{onlysolution}%
			##1
		\end{onlysolution}%
	}%
	%---
	\def\loigiaiEXS{}
	%\choiceTF
	\renewcommand{\writekeyTFone}{\gdef\TrueX{}\gdef\FalseX{\tickF}}
	\renewcommand{\writekeyTF}{%
		&\centering\leavevmode\TrueX%
		&\parbox[t]{\linewidth}{\centering\leavevmode\FalseX}%
			\gdef\TrueX{}\gdef\FalseX{\tickF}%
	}
	\def\kindSA{ShowSAKeyColor}
	\showanswers
	% \SAOPTN{kindSA=oly}
	\renewcommand{\dotlineEXS}[1]{}	
}
%=======Đn các phương án
\def\khoanhtrondapan{
	\renewcommand*\circled[1]{\tikz[baseline=(char.base)]{
			\node[shape=circle,draw=\mauDA,inner sep=1pt] (char) {##1};}}
	\renewcommand{\TrueEX}{\stepcounter{dapan}
		{\squareEX{\textbf{\damEX\color{\mauDA}\Alph{dapan}}}} \ignorespaces}
	\renewcommand{\FalseEX}{\stepcounter{dapan}
		{\circled{\textbf{\damEX\color{\mauDA}\Alph{dapan}}}} \ignorespaces}
	%---Chọn đáp án
	\renewcommand{\circEX}[2][fill=\mauTrue!3,draw=\mauTrue]{%
	\tikz[baseline=(char.base)]{\node[shape=circle,inner sep=1pt,##1] (char) {\color{red}##2};}}
}
%----
\def\khongkhoanhtrondapan{
	\renewcommand{\TrueEX}{\stepcounter{dapan}
		{\squareEX{\textbf{\damEX\color{\mauDA}\Alph{dapan}}}} \ignorespaces}
	\renewcommand{\FalseEX}{\stepcounter{dapan}
		{\textbf{\damEX\color{\mauDA}\Alph{dapan}.}} \ignorespaces}
	%---Chọn đáp án
	\renewcommand{\circEX}[2][fill=\mauTrue!3,draw=\mauTrue]{%
	\tikz[baseline=(char.base)]{\node[shape=circle,inner sep=1pt,##1] (char) {\color{red}##2};}}
}
%=============ĐN HIỆN CÂU EX CẦN THIẾT if
\newcommand{\hienEXS}[2]{
	%\foreach \bdem in {#1,...,#2}{%11-30
	\def\biendau{#1}\def\biencuoi{#2}%
	\pgfmathsetmacro{\sodau}{\fpeval{round(\biendau-1,0)}}
	\pgfmathsetmacro{\socuoi}{\fpeval{round(\biencuoi+1,0)}}
	\setcounter{EX}{#1-1}
	\RenewEnviron{ex}{
		\stepcounter{ex}%
		\ifnum\value{ex}<\socuoi
		\ifnum\value{ex}>\sodau
		\par%
		\begin{EX}
			\BODY% 
		\end{EX}
		\fi\fi
	}
	%
	\AtEndEnvironment{name}{\setcounter{EX}{#1-1}}
	%
	\AtEndEnvironment{EX}{
		\ifnum\the\value{numTrue}=1
		\scantokens{\begin{EXsol}A\end{EXsol}}
		\fi
		\ifnum\the\value{numTrue}=2
		\scantokens{\begin{EXsol}B\end{EXsol}}
		\fi
		\ifnum\the\value{numTrue}=3
		\scantokens{\begin{EXsol}C\end{EXsol}}
		\fi
		\ifnum\the\value{numTrue}=4
		\scantokens{\begin{EXsol}D\end{EXsol}}
		\fi
		\setcounter{numTrue}{0}
	}
}
%%%%%%%%%%%%%%%%%%%%%%%

%============================ Khung
\newenvironment{khung}
{\begin{tcolorbox}[
		enhanced,breakable,
		colback=yellow!10,
		colframe=blue,
		boxrule=0.5pt,
		%		drop fuzzy shadow=gray,
		left=5pt,right=5pt,top=5pt,bottom=5pt,
		arc=0mm
		]}
	{\end{tcolorbox}}
%-----------------------------Mục con = subsub
\newcounter{muccon}
\newcommand{\muccon}[1]{%
	\stepcounter{muccon}
	{%\setcounter{bt}{0}\setcounter{vd}{0}\setcounter{ex}{0}
		%\fontsize{13pt}{15pt}\selectfont
		%		\color{violet!70!black}\sffamily
		\bfseries\sffamily\bfseries\hspace*{0mm}\themuccon.\  
		#1}
}
%----------------------------------------------------

% Hộp định nghĩa
\newenvironment{boxdn}
{\begin{tcolorbox}
		[enhanced jigsaw,breakable,pad at break*=1mm,
		colback=cyan!2,
		%standard jigsaw, 
		opacityback=0, %ko nền
		boxrule=0pt,frame hidden, left=0.7cm, right=0pt, bottom=2pt, top=0pt,
		borderline west={1mm}{0.5cm}{cyan},
		overlay={
			\fill[fill=cyan!20,draw=none] ([xshift=0.6cm]interior.north west) rectangle (interior.south east)
			;
		}
		\setcounter{muccon}{0}
		]}%0mm lề trái
	{\end{tcolorbox}}
%===============================================
\theoremstyle{nonumberbreak} % ko đánh số
\theoremheaderfont{\sffamily\bfseries} %tên
\theorembodyfont{\normalfont} %thân
\theoremsymbol{\ensuremath{_\blacksquare}} %Dấu kết thúc là ô vuông đen.
\theoremseparator {:} % Dấu ngăn cách
\newtheorem{myphantich}{\color{violet}%\faServer\ 
	\faFileText\ PHÂN TÍCH}
%===============================================
\newenvironment{phantich}{\begin{boxdn}\begin{myphantich}}{\end{myphantich}\end{boxdn}}
%-------------- Khung (Trong main này ko sd)
\newtcolorbox[auto counter]{khung4}[1]{enhanced, breakable,
	before skip=1mm,after skip=1mm,
	left=1mm,right=1mm,top=2mm,bottom=1mm,
	colframe=myblue,colback=cyan!0,colbacktitle=cyan!6,coltitle=myblue,colupper=black,sharp corners,
	,boxrule=0.4mm,
	coltext=mauE,
	attach boxed title to top center=
	{yshift=-0.1mm-\tcboxedtitleheight/2,yshifttext=2mm-\tcboxedtitleheight/2},
	varwidth boxed title*=-3cm,
	boxed title style={boxrule=0.3mm,
		frame code={ \path[tcb fill frame] ([xshift=-4mm]frame.west)
			-- (frame.north west) -- (frame.north east) -- ([xshift=4mm]frame.east)
			-- (frame.south east) -- (frame.south west) -- cycle; },
		interior code={ \path[tcb fill interior] ([xshift=-2mm]interior.west)
			-- (interior.north west) -- (interior.north east)
			-- ([xshift=2mm]interior.east) -- (interior.south east) -- (interior.south west)
			-- cycle;} 
	},
	fonttitle=\fontsize{10}{0}
	\bfseries,
	fontupper=\fontsize{10}{0},
	title={#1}
}

\newcommand{\boxmini}[1]{
	\vspace*{-2mm}
	\begin{center}
		\begin{tikzpicture}[outline/.style={draw=##1,thick,fill=##1!3},outline/.default=myblue]
			\node [outline,
			sharp corners] at (0,0) {\fontfamily{qag} \selectfont\bfseries\color{\mauEX} #1};
		\end{tikzpicture}
	\end{center}
	\vspace*{0mm}
}

%-----------------------
\newcommand{\inden}[1]{
	{\fontsize{11pt}{9pt}\sffamily \selectfont\bfseries\color{\maudn} #1}
}
\newcommand{\indam}[1]{
	{\fontsize{11.5pt}{9pt}\sffamily \selectfont\bfseries\color{\maudn} #1}
}
\newcommand{\indamm}[1]{
	{\fontsize{11.5pt}{9pt}\sffamily \selectfont\bfseries\color{\maudl} #1}
}
\newcommand{\ind}[1]{
	{\fontsize{11.5pt}{9pt}\sffamily \selectfont\bfseries\color{\maucham} #1}
}
%%%===Các biểu tượng===
\def\iconGN{{\color{magenta}\faPencilSquareO}}
\def\iconNS{{\color{gray}\faStar}}
\def\iconQS{{\color{magenta}\faFolderOpen}}
\def\iconMT{{\color{magenta!80!black}\faSunO}}
\def\iconX{{\color{red}\faClose}}
\def\iconCH{{\color{myblue}\faCheckCircle}}
\def\iconVD{\faCubes}
\def\iconCV{{\color{myblue}\faCubes}}

\newcommand{\dongcham}[1]{
	\def\sod{#1}
	\pgfmathsetmacro{\sodong}{2*\sod -1} 
	\columnsep=10pt
	\vspace*{-3.5mm}
	\begin{multicols}{2}
		\foreach \dotline in{1,...,\sodong}
		{\noindent\color{gray}{\dotfill}\\[1mm]
		}\noindent\color{gray}{\dotfill}\\[-4mm]
	\end{multicols}
}


\def\TNTF{
    {\bfseries Phần II. Trong mỗi ý a), b), c) và d) ở mỗi câu, học sinh chọn đúng hoặc sai.}
}
\def\TN{
    {\bfseries Phần I. Mỗi câu hỏi học sinh chọn một trong bốn phương án A, B, C, D.}
}
\def\TNSA{
    {\bfseries Phần III. Học sinh điền kết quả vào ô trống.}
}
\def\BTTL{
    \begin{center}
        \fcolorbox{black}{white}{{\bfseries BÀI TẬP TỰ LUẬN TRẢ LỜI NGẮN}}
    \end{center}
}
\def\BTTF{
    \begin{center}
        \fcolorbox{black}{white}{{\bfseries BÀI TẬP TRẮC NGHIỆM ĐÚNG SAI}}
    \end{center}
%    \TNTF
}
\def\BTTN{
    \begin{center}
        \fcolorbox{black}{white}{{\bfseries BÀI TẬP TRẮC NGHIỆM 4 PHƯƠNG ÁN}}
    \end{center}
}
\def\TL{
    {\bfseries Phần II. Câu hỏi tự luận.}
}
 %Khai báo cơ bản
\usepackage{tkz-euclide,circuitikz}
%%%%%%%%%%%%% ĐIỀU KHIỂN LỜI GIẢI ,DÒNG CHẤM, ĐÁP SỐ
%------------Dòng chấm bằng chiều dài LG (bật)
% \dotlinefull{ex}\dotlinefull{vd}\dotlinefull{bt}
%------------Thay Loi giải bằng n dòng kẻ (bật)
% \dotlineans{2}{ex}
%------------Ẩn lời giải
%\hideansEX{ex}
%------------Dòng chấm tùy ý (ko cần \loigiai{})
%---Nhiều câu cùng dòng chấm (tách dụng lên mọi đề)
%\dongchamEXS{1,...,20}{2}
%\dongchamEXS{21,...,40}{5}
%\dongchamEXS{41,...,50}{10}
%---Nhiều câu cùng dòng chấm (tách dụng lên MỘT ĐỀ đc chọn)
%\DEdongchamEXS{3}{1,...,20}{2} %{3} là đề thứ 3
%---Dòng chấm từng câu, tác dụng lên mọi đề
%\dongchamEX{1/3,2/5,3/7} % câu / số dòng chấm của câu đó
%---Dòng chấm từng câu, tác dụng lên 1 đê
%\DEdongchamEX{3}{1/3,2/5,3/7} % câu / số dòng chấm của câu đó, {3} là đề số 3 
\renewcommand{\dongcham}[1]{}
%------------Ẩn đáp số (bật), đáp án
\exitdapso %ẩn đs
%\renewcommand{\indapan}[2]{} %ẩn đáp án
%%%%%%%%%%%%% khung NAME
\def\ngaylamde{Ngày làm đề: ...../...../........} %để {} nếu ko muốn
\def\tenchude{DÃY SỐ - CSC - CSN}
% \def\tendethi{ }
\def\tentruong{PHedu}
\def\thoigian{ }
%%%%%%%%%%%%% Nội dung head & foot
% \def\diachi{ }
\def\diachi{VNPmath - 0962940819}
\def\tenchuyende{\tenchude}
\def\tentacgia{GV.VŨ NGỌC PHÁT}
\def\chamngon{\lq\lq It's not how much time you have, it's how you use it.\rq\rq
}
%%%%%%%%%%%%% Đn lại A.B.C.D
\khoanhtrondapan
% \khongkhoanhtrondapan
%%%%%%%%%%%%%
\renewcommand{\arraystretch}{1}
\newcommand{\viduminhhoa}{\subsubsection{Ví dụ minh hoạ}}
\newcommand{\baitaptl}{\subsubsection{Bài tập tự luận}}

\newenvironment{boxdl}
{\begin{tcolorbox}
		[enhanced jigsaw,breakable,pad at break*=1mm,
		boxrule=0pt,frame hidden, left=2mm, right=0pt, bottom=1.5pt, top=1.5pt,
		before skip=2mm,
		after skip=2mm,
		%		borderline west={1mm}{0cm}{green!70!black}
		overlay={\draw[double,line width=1.5pt,\maudl] ([xshift=3pt]interior.north west)--([xshift=3pt]interior.south west);}
		]}%0mm lề trái
	{\end{tcolorbox}}
%%===================================================

\TFOPTN{kindTF=t,dapanTF=a,boldTF=1,phatbieu=Mệnh đề,viettat=1}
\SAOPTN{kindSA=oly,ketquaSA=KQ:,widthSA=4,heightSA=0.9,dapanSA=a}
%=================BẮT ĐẦU TÀI LIỆU===================

\begin{document}
\renewcommand{\chaptername}{Chương}
\pagenumbering{arabic}%đánh số trang dạng 1,2,...
%====================================================
%==================BẮT ĐẦU TÀI LIỆU==================
%\hienEXS{41}{50} %chỉ hiện câu từ 41 đến 50 của đề
%--------Đề bài
\NOTE \anLG \anDA 
% \FULLWIDTH
\notename

%Chương I
%%Bài 1. GTLG
% 
\section{Giá trị lượng giác của một góc lượng giác}
\subsection{Tóm tắt lý thuyết}
\begin{tomtat}
	\subsubsection{Khái niệm góc lượng giác và số đo của góc lượng giác}
	Trong mặt phẳng, cho hai tia $Ou$, $Ov$. Xét tia $Om$ cùng nằm trong mặt phẳng này. Nếu tia $Om$ quay quanh điểm $O$, theo một chiều nhất định từ $Ou$ đến $Ov$, thì ta nói nó quét một góc lượng giác với tia đầu  $Ou$, tia cuối $Ov$ và kí hiệu là ($Ou$, $Ov$).\\
	Mỗi góc lượng giác gốc $O$ được xác định bởi tia đầu $Ou$, tia cuối $Ov$ và số đo của nó.
	\begin{center}
		\begin{minipage}[H]{0.3\textwidth}
			\begin{tikzpicture}[scale=.7]	
				\draw (0,0) -- (4,0)node[below] {$u$};
				\draw[red] (0,0) -- (45:4)node[below right] {$v$};
				\draw[dashed,green!50!black] (0,0) -- (20:4)node[below right] {$m$};
				\draw[-stealth,red] (0:1) arc (0:45:1);
				\draw[-stealth] (2.75,1) arc (0:45:1);
				\path (30:3.5) node[below=-2pt]{$+$};
				\path (0:0) node[below left]{$O$};
			\end{tikzpicture}
		\end{minipage}
		\begin{minipage}[H]{0.3\textwidth}
			\begin{tikzpicture} [scale=.7]	
				\draw (0,0) -- (4,0)node[below] {$u$};
				\draw[red] (0,0) -- (45:4)node[below right] {$v$};
				\draw[dashed,green!50!black] (0,0) -- (75:3.5)node[below right] {$m$};
				\draw[red,-stealth,smooth,samples=100] plot[domain =0:2.25*pi]({.5*(1.1)^(\x) *cos(\x r)},{.5*(1.1)^(\x) *sin(\x r)});
				\draw[-stealth] (0.5,2) arc (75:110:2);
				\path (90:2.5) node[below=-2pt]{$+$};
				\path (0:0) node[below left]{$O$};
			\end{tikzpicture}
		\end{minipage}
		\begin{minipage}[H]{0.3\textwidth}
			\begin{tikzpicture}[scale=.7]		
				\draw (0,0) -- (4,0)node[below] {$u$};
				\draw[red] (0,0) -- (45:4)node[below right] {$v$};
				\draw[dashed,green!50!black] (0,0) -- (120:3)node[above right] {$m$};
				\draw[-stealth,red] (0:.8) arc (0:-315:.8);
				\draw[-stealth] (-1,1.7) arc (120:60:1);
				\path (105:2.4) node[below=-2pt]{$-$};
				\path (0:0) node[below left]{$O$};
			\end{tikzpicture}
		\end{minipage}
	\end{center}
	\subsubsection{Hệ thức Chasles}
	\immini{Hệ thức Chasles: Với ba tia $Ou$, $Ov$, $Ow$ bất kì, ta có 	
	$$
		\text{sđ}(Ou, Ov)+\text{sđ}(Ov,Ow)=\text{sđ}(Ou,Ow)+k 360^{\circ}(k \in \mathbb{Z}). 
		$$}
	{\begin{tikzpicture}[scale=0.77, font=\footnotesize, line join=round, line cap=round, >=stealth]		
			\draw (0,0) -- (4,0)node[below] {$u$};
			\draw[red] (0,0) -- (75:3.5)node[below right] {$w$};
			\draw[green!50!black] (0,0) -- (30:4)node[below right] {$v$};
			\draw[-stealth,red] (0:1) arc (0:-285:1);
			\draw[-stealth] (0:.9) arc (0:30:.9);
			\draw[-stealth] (.86,.5) arc (30:75:1);
			\path (0:0) node[below left]{$O$};
	\end{tikzpicture}}
	% Nhận xét. Từ hệ thức Chasles, ta suy ra:
	% Với ba tia tuỳ ý $Ox$, $Ou$, $Ov$ ta có
	% $$
	% \text{sđ}(Ou, Ov)=\text{sđ}(Ox, Ov)-\text{sđ}(Ox,Ou)+k360^{\circ}(k \in \mathbb{Z}). 
	% $$
	% Hệ thức này đóng vai trò quan trọng trong việc tính toán số đo của góc lượng giác.
	\subsubsection{Đơn vị đo góc và cung tròn}
	\textbf{Đơn vị độ}: Góc $1^{\circ}$ bằng $\dfrac{1}{180}$ góc bẹt.\\
	Đơn vị độ được chia thành những đơn vị nhỏ hơn: $1^{\circ}=60'; 1'=60"$.\\
	% Đối với các góc lượng giác, khi mà số vòng quay trong chuyển động tương ứng từ tia đầu đến tia cuối là khá lớn thì số đo của chúng tính bằng độ sẽ trở nên cồng kềnh. Do đó, trong khoa học và kĩ thuật, bên cạnh việc đo bằng độ, người ta còn sử dụng đơn vị đo góc bằng rađian.\\
	\immini{\textbf{Đơn vị rađian}: Cho đường tròn $(O)$ tâm $O$, bán kính $R$ và một cung $AB$ trên $(O)$.
		Ta nói cung tròn $AB$ có số đo bằng 1 rađian nếu độ dài của nó đúng bằng bán kính $R$.
		Khi đó ta cũng nói rằng góc $AOB$ có số đo bằng 1 rađian và viết: $\overset\frown{AOB}=1$ rad.}
	{\begin{tikzpicture}[scale=0.77, font=\footnotesize, line join=round, line cap=round, >=stealth]		
			\draw[green!50!black] (30:2) arc (30:-270:2);
			\draw[red] (30:2) arc (30:90:2);
			\draw (90:2)node[above]{$B$}--(0:0)--(30:2)node[above right]{$A$};
			\path (0:0) node[below left]{$O$};
			\draw(60:2) node[above right]{1 rad};
	\end{tikzpicture}}
	\textbf{Quan hệ giữa độ và rađian:}
	$$
	1 \text{ góc bẹt }=180^\circ = 1 \mathrm{rad} \Leftrightarrow  1^\circ=\dfrac{\pi}{180} \mathrm{rad} \quad \text { và }\quad 1\,  \mathrm{rad}=\left(\dfrac{180}{\pi}\right)^\circ.
	$$
	\begin{note}
		Khi viết số đo của một góc theo đơn vị rađian, người ta thường không viết chữ rad sau số đo. Chẳng hạn góc $\dfrac{\pi}{2}$ được hiểu là góc $\dfrac{\pi}{2}$ rad.
	\end{note}
	\begin{note}
		Dưới đây là bảng tương ứng giữa số đo bằng độ và số đo bằng rađian của các góc đặc biệt trong phạm vi từ $0^{\circ}$ đến $180^{\circ}$.
	\end{note}
	\begin{center}
		\renewcommand{\arraystretch}{2}
		\begin{tabular}{|l|c|c|c|c|c|c|c|c|c|}
			\hline Độ & $0^{\circ}$ & $30^{\circ}$ & $45^{\circ}$ & $60^{\circ}$ & $90^{\circ}$ & $120^{\circ}$ & $135^{\circ}$ & $150^{\circ}$ & $180^{\circ}$ \\
			\hline Rađian & 0 & $\dfrac{\pi}{6}$ & $\dfrac{\pi}{4}$ & $\dfrac{\pi}{3}$ & $\dfrac{\pi}{2}$ & $\dfrac{2 \pi}{3}$ & $\dfrac{3 \pi}{4}$ & $\dfrac{5 \pi}{6}$ & $\pi$ \\
			\hline
		\end{tabular}
	\end{center}
	\subsubsection{Độ dài cung tròn}
	Một cung của đường tròn bán kính $R$ và có số đo $\alpha$ rad thì có độ dài $l=R \alpha$.
	\subsubsection{Đường tròn lượng giác}
	\immini{\begin{itemize}
			\item Đường tròn lượng giác là đường tròn có tâm tại gốc toạ độ, bán kính bằng $1$, được định hướng và lấy điểm $A(1 ; 0)$ làm điểm gốc của đường tròn.
			\item Điểm trên đường tròn lượng giác biểu diễn góc lượng giác có số đo $\alpha$ là điểm $M$ trên đường tròn lượng giác sao cho sđ$(OA, OM)=\alpha$.
	\end{itemize}
	\begin{note}
		Góc $\alpha$ và $\beta$ có chung điểm biểu diễn khi \fbox{$\alpha - \beta = k2\pi$} (chẵn lần $\pi$)
		\end{note}}
	{
		\begin{tikzpicture}[line join = round, line cap = round, >=stealth, font=\footnotesize, scale=0.6]
			\tikzset{label style/.style={font=\footnotesize}}
			\path (0,0) coordinate (O)
			(3,0) coordinate (A)
			(0,3) coordinate (B)
			(0,-3) coordinate (B')
			(-3,0) coordinate (A')
			(0:0)++(150:3) coordinate (M)
			($(O)!(M)!(A')$) coordinate (H)
			($(O)!(M)!(B)$) coordinate (K)
			;
			\draw[->] (-4,0) -- (4,0) node[above,blue]{$x$};
			\draw[->] (0,-4) -- (0.,4) node[left,blue]{$y$};
			\draw[orange] (O) circle (3cm);
			\draw[rotate=0,->,green!50!black] (0.5,0) arc (0:150:0.5cm);
			\draw (0.35,0.25) node[above,blue] {$\alpha$};
			\draw[dashed] (H)--(M)--(K);
			\draw[green!50!black] (M)--(O);
			\foreach \p/\r in {A/-45,M/150,H/-90,O/-150,A'/-135,B'/-45,B/45,K/0}
			\fill (\p) circle (1pt) node[shift={(\r:3mm)},blue]{$\p$};
		\end{tikzpicture}
	}
	\subsubsection{Các giá trị lượng giác của góc lượng giác}
	\immini{Gọi $M(x;y)$ là điểm biểu diễn của góc lượng giác $\alpha$ trên đường tròn lượng giác. Khi đó, ta có:
		\begin{itemize}
			\item $\cos\alpha=x.$
			\item $\sin\alpha=y.$
			\item $\tan\alpha=\dfrac{\sin\alpha}{\cos\alpha}=\dfrac{y}{x} ~(x\neq0).$
		\item $\cot\alpha=\dfrac{\cos\alpha}{\sin\alpha}=\dfrac{x}{y} ~(y\neq0).$
	\end{itemize}}
	{\begin{tikzpicture}[line join = round, line cap = round, >=stealth, font=\footnotesize, scale=0.6]
			\tikzset{label style/.style={font=\footnotesize}}
			\path (0,0) coordinate (O)
			(3,0) coordinate (A)
			(0,3) coordinate (B)
			(0,-3) coordinate (B')
			(-3,0) coordinate (A')
			(0:0)++(40:3) coordinate (M)
			($(O)!(M)!(A')$) coordinate (H)
			($(O)!(M)!(B)$) coordinate (K)
			;
			\draw[->] (-4,0) -- (4,0) node[above,blue]{$x$};
			\draw[->] (0,-4) -- (0.,4) node[left,blue]{$y$};
			\draw[orange] (O) circle (3cm);
			\draw[rotate=0,->,green!50!black] (0.7,0) arc (0:40:0.7cm);
			\draw (1,0) node[above,blue] {$\alpha$};
			\draw[dashed] (H)--(M)--(K);
			\draw[green!50!black] (M)--(O);
			\draw[blue,fill=black] (0,2) node[left]{$\sin\alpha$}(2,0) circle(1pt) node[below]{$\cos\alpha$}(3,2.3) node{$M(x;y)$};
			\foreach \p/\r in {A/-45,O/-135,A'/-135,B'/-45,B/45}
			\fill (\p) circle (1pt) node[shift={(\r:3mm)},blue]{$\p$};
	\end{tikzpicture}}
	\begin{note}
		a) Ta còn gọi trục tung là trục sin, trục hoành là trục côsin.\\
		b) Từ định nghĩa ta suy ra:
		\begin{itemize}
			\item $\sin\alpha$, $\cos\alpha$ xác định với mọi giá trị của $\alpha$ và ta có:
			$$-1\leq \sin\alpha\leq 1; \quad -1\leq \cos\alpha\leq 1; \quad \sin(\alpha+k2\pi)=\sin\alpha;\quad \cos(\alpha+k2\pi)=\cos\alpha\,\, (k\in\mathbb{Z}).$$
			\item $\tan\alpha$ xác định khi $\alpha\neq\dfrac{\pi}{2}+k\pi\,\,  (k\in\mathbb{Z})$.
			\item $\cot\alpha$ xác định khi $\alpha\neq k\pi\,\,  (k\in\mathbb{Z})$.
			\item Dấu của các giá trị lượng giác của một góc lượng giác phụ thuộc vào vị trí điểm biểu diễn $M$ trên đường tròn lượng giác.
		\end{itemize}
	\end{note}
	\begin{minipage}[h]{0.6\textwidth}
		\begin{tabular}{c|c|c|c|c|}
			\cline{2-5}
			& \multicolumn{4}{c|}{Góc phần tư} \\ \hline
			\multicolumn{1}{|c|}{Giá trị lượng giác} & I     & II     & III     & IV    \\ \hline
			\multicolumn{1}{|c|}{$\sin \alpha$}     &   $+$    &  $ +$      &    $-$    &   $-$  \\ \hline
			\multicolumn{1}{|c|}{$\cos \alpha$}     &   $+$    &  $ -$      &    $-$    &   $+$  \\ \hline
			\multicolumn{1}{|c|}{$\tan \alpha$}     &   $+$    &  $ -$      &    $+$    &   $-$  \\ \hline
			\multicolumn{1}{|c|}{$\cot \alpha$}     &   $+$    &  $ -$      &    $+$    &   $-$  \\ \hline
		\end{tabular}
	\end{minipage}
	\begin{minipage}[h]{0.6\textwidth}
		\begin{tikzpicture}[line join = round, line cap = round, >=stealth, font=\footnotesize, scale=0.6]
			\tikzset{label style/.style={font=\footnotesize}}
			\path (0,0) coordinate (O)
			(3,0) coordinate (A)
			(0,3) coordinate (B)
			(0,-3) coordinate (B')
			(-3,0) coordinate (A')
			(0:0)++(-60:3) coordinate (M)
			($(O)!(M)!(A')$) coordinate (H)
			($(O)!(M)!(B)$) coordinate (K)
			;
			\draw[->] (-4,0) -- (4,0) node[above,blue]{$x$};
			\draw[->] (0,-4) -- (0.,4) node[left,blue]{$y$};
			\draw[orange] (O) circle (3cm);
			\draw[rotate=0,->,green!50!black] (0.5,0) arc (0:-60:0.5cm);
			\draw (0.75,-0.35) node[blue] {$\alpha$};
			\draw[dashed] (H)--(M)--(K);
			\draw[green!50!black] (M)--(O);
			\draw[blue] (2.5,2.5) node{$I$}(-2.5,2.5) node{$II$}(-2.5,-2.5) node{$III$}(2.5,-2.5) node{$IV$};
			\foreach \p/\r in {A/-45,M/-60,H/90,O/-150,A'/-135,B'/-45,B/45,K/180}
			\fill (\p) circle (1pt) node[shift={(\r:3mm)},blue]{$\p$};
		\end{tikzpicture}
	\end{minipage}
	% \subsubsection{Giá trị lượng giác của các góc đặc biệt}
	% \begin{center}
	% 	\renewcommand{\arraystretch}{2}
	% 	\begin{tabular}{|c|c|c|c|c|c|}
	% 		\hline
	% 		\multirow{2}{*}{Góc $\alpha$} & $0$              & $\dfrac{\pi}{6}$  & $\dfrac{\pi}{4}$  & $\dfrac{\pi}{3}$  & $\dfrac{\pi}{2}$              \\ \cline{2-6} 
	% 		& $0^\circ$              & $30^\circ$  & $45^\circ$  & $60^\circ$  & $90^\circ$             \\ \hline
	% 		$\sin\alpha$                  & $0$             & $\dfrac{1}{2}$ & $\dfrac{\sqrt{2}}{2}$ & $\dfrac{\sqrt{3}}{2}$ & 1              \\ \hline
	% 		$\cos\alpha$                  & $1$             & $\dfrac{\sqrt{3}}{2}$ & $\dfrac{\sqrt{2}}{2}$ & $\dfrac{1}{2}$ & 0              \\ \hline
	% 		$\tan\alpha$                 & $0$             & $\dfrac{1}{\sqrt{3}}$ & 1 & $\sqrt{3}$ & Không xác định \\ \hline
	% 		$\cot\alpha$                  & Không xác định & $\sqrt{3}$ & 1 & $\dfrac{1}{\sqrt{3}}$ & 0              \\ \hline
	% 	\end{tabular}
	% \end{center}
	\subsubsection{Các công thức lượng giác cơ bản}
	Đối với các giá trị lượng giác, ta có các hệ thức cơ bản sau
	\begin{enumEX}[$\bullet$]{2}
		\item $\sin^2 \alpha  + \cos^2 \alpha =1$
		\item $ 1+ \tan^2 \alpha= \dfrac{1}{\cos^2 \alpha}$ $\left(\alpha \neq \dfrac{\pi}{2}+k\pi , k\in \mathbb{Z}\right)$
		\item $ 1+ \cot^2 \alpha= \dfrac{1}{\sin^2 \alpha}$ $\left(\alpha \neq k\pi , k\in \mathbb{Z}\right)$
		\item $\tan \alpha \cdot \cot \alpha =1 $ $\left(\alpha \neq \dfrac{k\pi}{2}, k\in \mathbb{Z}\right)$
	\end{enumEX}
	\newpage
	\subsubsection{Giá trị lượng giác của các góc có liên quan đặc biệt}
	\begin{enumerate}
		\item Góc đối nhau ($\alpha$ và $-\alpha$)
		\immini{\begin{itemize}
				\item $\cos (-\alpha)=\cos \alpha$
				\item $\sin (-\alpha) =-\sin \alpha$
				\item $\tan (-\alpha) =-\tan \alpha$
				\item $\cot (-\alpha) =-\cot \alpha$
		\end{itemize}}
		{\vspace*{-1cm}\begin{tikzpicture}[line join = round, line cap = round, >=stealth, font=\footnotesize, scale=0.5]
				\tikzset{label style/.style={font=\footnotesize}}
				\path (0,0) coordinate (O)
				(3,0) coordinate (A)
				(0:0)++(120:3) coordinate (M)
				(0:0)++(-120:3) coordinate (N)
				(0,4) coordinate (C)
				(0,-4) coordinate (D)
				($(O)!(M)!(C)$) coordinate (E)
				($(O)!(N)!(D)$) coordinate (F)
				;
				\draw[->] (-4,0) -- (4,0) node[above,blue]{$x$};
				\draw[->] (0,-4) -- (0.,4) node[left,blue]{$y$};
				\draw[orange] (O) circle (3cm);
				\draw[rotate=0,->,red] (0.5,0) arc (0:120:0.5cm);
				\draw[rotate=0,->,green!50!black] (0.6,0) arc (0:-120:0.6cm);
				\draw (0,0) node[above right=2pt,blue] {$\alpha$} (0,-0) node[below right=2pt,blue]{$-\alpha$};
				\draw[dashed] (E)--(M)--(N)--(F);
				\draw[green!50!black] (M)--(O);
				\draw[red] (N)--(O);
				\foreach \p/\r in {A/-45,M/120,N/-120,O/-150}
				\fill (\p) circle (1pt) node[shift={(\r:3mm)},blue]{$\p$};
		\end{tikzpicture}}
		\item Góc bù nhau ($\alpha$ và $\pi-\alpha$)
		\immini{\begin{itemize}
				\item $\sin (\pi -\alpha)=\sin \alpha$
				\item $\cos (\pi -\alpha) =-\cos \alpha$
				\item $\tan (\pi -\alpha) =-\tan \alpha$
				\item $\cot (\pi -\alpha) =-\cot \alpha$
		\end{itemize}}
		{\vspace*{-0.5cm}\begin{tikzpicture}[line join = round, line cap = round, >=stealth, font=\footnotesize, scale=0.5]
				\tikzset{label style/.style={font=\footnotesize}}
				\path (0,0) coordinate (O)
				(3,0) coordinate (A)
				(0:0)++(30:3) coordinate (M)
				(0:0)++(150:3) coordinate (N)
				(4,0) coordinate (C)
				(-4,0) coordinate (D)
				($(O)!(M)!(C)$) coordinate (E)
				($(O)!(N)!(D)$) coordinate (F)
				;
				\draw[->] (-4,0) -- (4,0) node[above,blue]{$x$};
				\draw[->] (0,-4) -- (0.,4) node[left,blue]{$y$};
				\draw[orange] (O) circle (3cm);
				\draw[rotate=0,->,red] (0.5,0) arc (0:150:0.5cm);
				\draw[rotate=0,->,green!50!black] (1.6,0) arc (0:30:1.6cm);
				\draw (2,0) node[above,blue] {$\alpha$} (0.3,1.5) node[below,blue]{$\pi-\alpha$};
				\draw[dashed] (E)--(M)--(N)--(F);
				\draw[red] (O)--(N);
				\draw[green!50!black] (O)--(M);
				\foreach \p/\r in {A/-45,M/30,N/150,O/-130}
				\fill (\p) circle (1pt) node[shift={(\r:3mm)},blue]{$\p$};
		\end{tikzpicture}}
		\item Góc phụ nhau ($\alpha$ và $\dfrac{\pi}{2}-\alpha$)
		\immini{\begin{itemize}
				\item $\sin \left( \dfrac{\pi}{2}-\alpha\right)=\cos \alpha$
				\item $\cos \left( \dfrac{\pi}{2}-\alpha\right)=\sin \alpha$
				\item $\tan \left( \dfrac{\pi}{2}-\alpha\right)=\cot \alpha$
				\item $\cot \left( \dfrac{\pi}{2}-\alpha\right)=\tan \alpha$
		\end{itemize}}
		{\vspace*{-0.5cm}\begin{tikzpicture}[line join = round, line cap = round, >=stealth, font=\footnotesize, scale=0.5]
				\tikzset{label style/.style={font=\footnotesize}}
				\path (0,0) coordinate (O)
				(3,0) coordinate (A)
				(0:0)++(20:3) coordinate (M)
				(0:0)++(70:3) coordinate (N)
				(0,4) coordinate (C)
				(4,0) coordinate (D)
				($(O)!(M)!(C)$) coordinate (E)
				($(O)!(N)!(D)$) coordinate (F)
				(2.82,0) coordinate (G)
				(0,2.82) coordinate (H)
				;
				\draw[->] (-4,0) -- (4,0) node[above,blue]{$x$};
				\draw[->] (0,-4) -- (0.,4) node[left,blue]{$y$};
				\draw[orange] (O) circle (3cm);
				\draw[rotate=0,->,red] (0.7,0) arc (0:70:0.7cm);
				\draw[rotate=0,->,green!50!black] (1.6,0) arc (0:20:1.6cm);
				\draw (2,0) node[above,blue] {$\alpha$} (1,-.2) node[below,blue]{$\frac{\pi}{2}-\alpha$};
				\draw[dashed] (E)--(M) (F)--(N) (G)--(M) (H)--(N);
				\draw[dashed] (-3,-3)--(3,3);
				\draw[->] (0.8,-.5)--(0.5,0.45);
				\draw[red] (O)--(N);
				\draw[green!50!black] (O)--(M);
				\foreach \p/\r in {A/-45,M/20,N/70,O/-220}
				\fill (\p) circle (1pt) node[shift={(\r:3mm)},blue]{$\p$};
		\end{tikzpicture}}
		\item Góc hơn kém $\pi$ ($\alpha$ và $\pi+\alpha$)
		\immini{\begin{itemize}
				\item $\sin (\pi +\alpha)=-\sin \alpha$
				\item $\cos (\pi +\alpha)=-\cos \alpha$
				\item $\tan (\pi +\alpha)=\tan \alpha$
				\item $\cot (\pi +\alpha)=\cot \alpha$
		\end{itemize}}
		{\vspace*{-0.5cm}\begin{tikzpicture}[line join = round, line cap = round, >=stealth, font=\footnotesize, scale=0.5]
				\tikzset{label style/.style={font=\footnotesize}}
				\path (0,0) coordinate (O)
				(3,0) coordinate (A)
				(0:0)++(60:3) coordinate (M)
				(0:0)++(240:3) coordinate (N)
				;
				\draw[->] (-4,0) -- (4,0) node[above,blue]{$x$};
				\draw[->] (0,-4) -- (0.,4) node[left,blue]{$y$};
				\draw[orange] (O) circle (3cm);
				\draw[rotate=0,->,red] (1.7,0) arc (0:240:1.7cm);
				\draw[rotate=0,->,green!50!black] (0.6,0) arc (0:60:0.6cm);
				\draw (1,0) node[above,blue] {$\alpha$};
				\draw (-1.2,1) node[below,blue,rotate=60]{$\pi+\alpha$};
				\draw[red] (O)--(N);
				\draw[green!50!black] (O)--(M);
				\foreach \p/\r in {A/-45,M/60,N/240,O/-150}
				\fill (\p) circle (1pt) node[shift={(\r:3mm)},blue]{$\p$};
		\end{tikzpicture}}
	\end{enumerate}
\end{tomtat}

% \foreach \i in {1,2,...,7} {\input{data/11KNTT/data1/1K1-2-\i.tex}}

%%Bài 2. CTLG
% %\chapter{Hàm số  lượng giác và phương trình lượng giác}
\setcounter{section}{1}
\section{Công thức lượng giác}
\subsection{Tóm tắt lý thuyết}
\begin{tomtat}
% 	\begin{center}
% 		\begin{tikzpicture}[scale = 2.5]
% 			\path (0,0) coordinate (O) (1.5,0) coordinate (x) (0,1.5) coordinate (y);
% 			\draw[thick,->] (-1.5,0)--(x);
% 			\draw[thick,->] (0,-1.5)--(y);
% 			\draw (O) circle (1);
% 			\path ($(O)+(55:1)$) coordinate (M) 
% 			($(O)+(30:1)$) coordinate (N);
% 			\path ($(O)!(M)!(x)$) coordinate (x_M)
% 			($(O)!(M)!(y)$) coordinate (y_M)
% 			($(O)!(N)!(x)$) coordinate (x_N)
% 			($(O)!(N)!(y)$) coordinate (y_N);
% 			\draw[dashed] (x_M)--(M)--(y_M) (x_N)--(N)--(y_N);
% 			\foreach \x/\g in {O/-135,x/-90,y/180,x_M/-90,x_N/-90,y_M/180,y_N/180}
% 			\fill ($(\g:1mm)+(\x)$) node {$\x$};
% 			\fill 	(M) circle (0.5pt)
% 			($(15:4mm)+(M)$) node {$M\left(x_M,y_M\right)$};
% 			\fill (N) circle (0.5pt)
% 			($(15:4mm)+(N)$) node {$N\left(x_N,y_N\right)$};
% 	\draw (M)--(O)--(N);		
% 	\draw pic[draw,,angle radius=6mm,->,red]{angle=x--O--M};
% 	\fill[red] (45:3mm) node {$\alpha$};
% 	\draw pic[red,draw,,angle radius=10mm,->]{angle=x--O--N};
% 	\fill[red] (15:5mm) node {$\beta$};
% 		\end{tikzpicture}
% 	\end{center}
% Trong mặt phẳng $Oxy$ cho hai điểm $M,N$ trên đường tròn lượng giác.\\ Đặt $\alpha = \text{sđ} (Ox,OM), \beta = \text{sđ} (Ox,ON)$, ta có $M(\cos \alpha,\sin \alpha)$ và $N(\cos \beta, \sin \beta)$. Khi đó ta tính được $\overrightarrow{OM}.\overrightarrow{ON}$ bằng hai cách
% \begin{align*}
% 	\overrightarrow{OM}.\overrightarrow{ON}&=\left|\overrightarrow{OM}\right|.\left|\overrightarrow{ON}\right|.\cos \left(\overrightarrow{OM},\overrightarrow{ON}\right) = \cos (\alpha-\beta),\\
% 	\overrightarrow{OM}.\overrightarrow{ON} &= x_Mx_N+y_My_N= \cos \alpha \cos\beta +\sin\alpha\sin\beta.
% \end{align*}
% Từ đó dẫn tới công thức
% \begin{align*}
% 	\cos (\alpha-\beta) = \cos \alpha \cos \beta + \sin \alpha\sin \beta \tag{$\star$}
% \end{align*}
% Tất cả các công thức trong bài học được xây dựng dựa trên công thức $(\star)$.\\
% Trong suốt bài học, khi không nói gì thêm, chỉ xét các góc lượng giác mà trong đó giá trị lượng giác được để cập có nghĩa. 
	\subsubsection{Công thức cộng}
	\begin{khung4}{Công thức cộng}
	\begin{tasks}[style=itemize](2)
		\task $\cos (a-b) = \cos a \cos b + \sin a\sin b$.
		\task $\cos (a+b) = \cos a \cos b - \sin a\sin b$.
		\task $\sin (a-b) = \sin a \cos b - \sin b \cos a$.
		\task $\sin (a+b) = \sin a \cos b + \sin b \cos a$.
		\task $\tan (a-b) = \dfrac{\tan a - \tan b}{1+\tan a \tan b}$.
		\task $\tan (a+b) = \dfrac{\tan a + \tan b}{1-\tan a \tan b}$.
	\end{tasks}
	\end{khung4}
	\subsubsection{Công thức nhân đôi}
	Công thức nhân đôi được xây dựng bằng cách thay $b=a$ trong công thức cộng.
	\begin{khung4}{Công thức nhân đôi}
		\begin{tasks}[style=itemize]
			\task $\sin 2a = 2\sin a \cos a$.
			\task $\cos 2a = \cos^2a-\sin^2a = 2\cos^2a-1 = 1-2\sin^2a$.
			\task $\tan 2a = \dfrac{2\tan a}{1-\tan^2a}$.
		\end{tasks}
		\end{khung4}
\begin{note}
	Từ công thức nhân đôi, ta có công thức hạ bậc:
\end{note}
	\begin{khung4}{Công thức hạ bậc}
		\begin{tasks}[style=itemize](3)
			\task $\sin^2a= \dfrac{1-\cos 2a}{2}$.
		\task $\cos^2a = \dfrac{1+\cos 2a}{2}$.
		\task $\tan^2a=\dfrac{1-\cos2a}{1+\cos 2a}$.
		\end{tasks}
		\end{khung4}

\begin{note}
	Áp dụng công thức cộng cho $3a = a +2a$, ta có công thức nhân ba:
\end{note}
	\begin{khung4}{Công thức nhân ba}
		\begin{tasks}[style=itemize](2)
			\task $\sin3a= 3\sin a -4\sin^3a$.
		\task $\cos3a= 4\cos^3a-3\cos a$.
		\task $\tan3a = \dfrac{3\tan a - \tan^3 a}{1-3\tan^2a}$.
		\end{tasks}
		\end{khung4}

	\subsubsection{Công thức biến đổi tích thành tổng}
	\begin{khung4}{Công thức tích thành tổng}
		\begin{tasks}[style=itemize]
			\task $\cos a \cos b = \dfrac{1}{2}\left[\cos (a-b) + \cos (a+b)\right]$.
		\task $\sin a \sin b = \dfrac{1}{2}\left[\cos (a-b)-\cos(a+b)\right]$.
		\task $\sin a \cos b = \dfrac{1}{2}\left[\sin (a-b)+\sin (a+b)\right]$.
		\end{tasks}
		\end{khung4}
	\subsubsection{Công thức biến đổi tổng thành tích}
	Công thức biến đổi tổng thành tích được xây dựng bằng cách $a=\dfrac{a+b}{2}, b = \dfrac{a-b}{2}$ trong công thức biến đổi tích thành tổng.
	\begin{khung4}{Công thức tổng thành tích}
		\begin{tasks}[style=itemize](2)
			\task $\cos a+ \cos b = 2\cos\dfrac{a+b}{2}\cos \dfrac{a-b}{2}$.
		\task $\cos a- \cos b = -2\sin\dfrac{a+b}{2}\sin \dfrac{a-b}{2}$.
		\task $\sin a+ \sin b = 2\sin\dfrac{a+b}{2}\cos \dfrac{a-b}{2}$.
		\task $\sin a -\sin b = 2\cos\dfrac{a+b}{2}\sin \dfrac{a-b}{2}$.
		\end{tasks}
		\end{khung4}
\end{tomtat}


% \foreach \i in {1,2,...,5} {\input{data/11KNTT/data1/1K1-2-\i.tex}}

%%Bài 3. HSLG
% \setcounter{section}{2}
\section{Hàm số lượng giác}
\subsection{Tóm tắt lý thuyết}
\begin{tomtat}
	% \subsubsection{Định nghĩa hàm số lượng giác}
	% \begin{dn}
	% 	\begin{itemize}
	% 		\item Hàm số sin $y=\sin x$ có tập xác định là $\mathbb{R}$.
	% 		\item Hàm số cos $y=\cos x$ có tập xác định là $\mathbb{R}$.
	% 		\item Hàm số tan  $y=\tan x$ có tập xác định là $\mathbb{R} \setminus\left\{\dfrac{\pi}{2}+k \pi \Big| k \in \mathbb{Z} \right\}$.
	% 		\item  Hàm số cot $y=\cot x$ có tập xác định là $\mathbb{R} \setminus \left\{k \pi \Big| k \in \mathbb{Z} \right\}$.
	% 	\end{itemize}
	% \end{dn}
	\subsubsection{Hàm số chẵn, hàm số lẻ}
	\begin{dn}
	Cho hàm số $y=f(x)$ có tập xác định là $\mathscr{D}$.
	\begin{itemize}
		\item Hàm số $f(x)$ được gọi là \textbf{hàm số chẵn} nếu $\forall x \in \mathscr{D}$ thì $-x \in \mathscr{D}$ và $f(-x)=f(x)$. Đồ thị của một hàm số chẵn nhận trục tung là trục đối xứng.
		\item Hàm số $f(x)$ được gọi là \textbf{hàm số lẻ} nếu $\forall x \in \mathscr{D}$ thì $-x \in \mathscr{D}$ và $f(-x)=-f(x)$. Đồ thị của một hàm số lẻ nhận gốc toạ độ là tâm đối xứng.
	\end{itemize}
	\end{dn}
	\subsubsection{Hàm số tuần hoàn}
	\begin{dn}
		Hàm số $y=f(x)$ có tập xác định $\mathscr{D}$ được gọi là \textbf{hàm số tuần hoàn} nếu tồn tại số $T \neq 0$ sao cho với mọi $x \in \mathscr{D}$ ta có:
		\begin{enumerate}[i)]
			\item $x+T \in \mathscr{D}$ và $x-T \in \mathscr{D}$;
			\item $f(x+T)=f(x)$.
		\end{enumerate}
		Số $T$ dương nhỏ nhất thỏa mãn các điều kiện trên (nếu có) được gọi là \textbf{chu kì} của hàm số tuần hoàn đó.
	\end{dn}
	\begin{nx}
		\
		\begin{itemize}
			\item  Các hàm số $y=\sin x$ và $y=\cos x$ tuần hoàn với chu kì $2 \pi$. Các hàm số $y=\tan x$ và $y=\cot x$ tuần hoàn với chu kì $\pi$.
		\end{itemize}
	\end{nx}
	\begin{note} 
		Tổng quát, người ta chứng minh được các hàm số $y=A \sin \omega x$ và $y=A \cos \omega x$ $(\omega>0)$ là những hàm số tuần hoàn với chu kì \fbox{$T=\dfrac{2 \pi}{\omega}$}.
	\end{note}
	\subsubsection{Đồ thị và tính chất của hàm số $y=\sin x$}
	\begin{tc}
		Hàm số $y=\sin x$:
		\begin{itemize}
			\item   Có tập xác định là $\mathbb{R}$ và tập giá trị là $[-1 ; 1]$;
			\item   Là hàm số lẻ và tuần hoàn với chu kì $2 \pi$;
			\item    Đồng biến trên mỗi khoảng $\left(-\dfrac{\pi}{2}+k 2 \pi ; \dfrac{\pi}{2}+k 2 \pi\right)$ và nghịch biến trên mỗi khoảng \\
			$\left(\dfrac{\pi}{2}+k 2 \pi ; \dfrac{3 \pi}{2}+k 2 \pi\right)$, $k \in \mathbb{Z}$;
			\item    Có đồ thị đối xứng qua gốc toạ độ và gọi là một \textbf{đường hình sin}.
		\end{itemize}
	\begin{center}
		\begin{tikzpicture}[>=stealth,scale=0.7,transform shape] 
			\path
			({-2.5*pi},0) coordinate (X1)
			({-2*pi},0) coordinate (X2)
			({-1.5*pi},0) coordinate (X3)
			({-pi},0) coordinate (X4)
			({-0.5*pi},0) coordinate (X5)
			(0,0) coordinate (O)
			({0.5*pi},0) coordinate (X6)
			({pi},0) coordinate (X7)
			({1.5*pi},0) coordinate (X8)
			({2*pi},0) coordinate (X9)
			({2.5*pi},0) coordinate (X10)
			({-pi},-2) coordinate (A)
			({pi},-2) coordinate (B)
			;
			\draw[->] (-9.5,0) -- (9.5,0) node[below] {\small $x$};
			\draw[->] (0,-1.5) -- (0,1.8) node[right] {\small $y$};
			\draw [dotted] (X3)--({-1.5*pi},1)--({2.5*pi},1)--({2.5*pi},0)  ({0.5*pi},1)--({0.5*pi},0)
			(X1)--({-2.5*pi},-1)--({1.5*pi},-1)--({1.5*pi},0)  ({-0.5*pi},-1)--({-0.5*pi},0)
			({-pi},0) -- (A) ({pi},0) -- (B);
			\foreach \x/\g/\z in {X1/90/-\tfrac{5\pi}{2},X2/140/-2\pi,X3/-90/-\tfrac{3\pi}{2},X4/-135/-\pi,X5/90/-\tfrac{\pi}{2},X6/-90/\tfrac{\pi}{2},X7/60/\pi,X8/90/\tfrac{3\pi}{2},X9/-40/2\pi,X10/-90/\tfrac{5\pi}{2}} 
			\fill[black] (\x) circle(1pt) +(\g:5mm) node {$\z$};
			\draw [<->] ({-pi},-1.7)--({pi},-1.7) ; 
			\draw (0,0) node[below right]{$O$} (0,-1.7) node[below]{$T=2\pi$}
			(0,1) node[above right]{$1$} (0,-1) node[below right]{$-1$};
			\clip (-9.5,-1.4) rectangle (9.5,1.6) ;
			\draw[thick,samples=100,domain=-9.3:9.3] plot(\x,{sin((\x)*180/pi)});
			
		\end{tikzpicture}
	\end{center}
	\end{tc}
	\subsubsection{Đồ thị và tính chất của hàm số $y=\cos x$}
	\begin{tc}
		Hàm số $y=\cos x$:
		\begin{itemize}
			\item    Có tập xác định là $\mathbb{R}$ và tập giá trị là $[-1 ; 1]$;
			\item    Là hàm số chẵn và tuần hoàn với chu kì $2 \pi$;
			\item    Đồng biến trên mỗi khoảng $(-\pi+k 2 \pi ; k 2 \pi)$ và nghịch biến trên mỗi khoảng $(k 2 \pi ; \pi+k 2 \pi), k \in \mathbb{Z}$;
			\item    Có đồ thị là một đường hình sin đối xứng qua trục tung.
		\end{itemize}
	\begin{center}
		\begin{tikzpicture}[>=stealth,scale=0.77,transform shape] 
			\path
			({-2.5*pi},0) coordinate (X1)
			({-2*pi},0) coordinate (X2)
			({-1.5*pi},0) coordinate (X3)
			({-pi},0) coordinate (X4)
			({-0.5*pi},0) coordinate (X5)
			(0,0) coordinate (O)
			({0.5*pi},0) coordinate (X6)
			({pi},0) coordinate (X7)
			({1.5*pi},0) coordinate (X8)
			({2*pi},0) coordinate (X9)
			({2.5*pi},0) coordinate (X10)
			({-pi},-2) coordinate (A)
			({pi},-2) coordinate (B)
			;
			\draw[->] (-9.5,0) -- (9.5,0) node[below] {\small $x$};
			\draw[->] (0,-1.5) -- (0,1.8) node[right] {\small $y$};
			\draw [dotted] (X2)--({-2*pi},1)--({2*pi},1)--({2*pi},0) (X4)--({-pi},-1)--({pi},-1)--({pi},0) 
			({-pi},0) -- (A) ({pi},0) -- (B);
			\foreach \x/\g/\z in {X1/125/-\tfrac{5\pi}{2},X2/-90/-2\pi,X3/-120/-\tfrac{3\pi}{2},X4/90/-\pi,X5/110/-\tfrac{\pi}{2},X6/-120/\tfrac{\pi}{2},X7/90/\pi,X8/120/\tfrac{3\pi}{2},X9/-90/2\pi,X10/-120/\tfrac{5\pi}{2}} 
			\fill[black] (\x) circle(1pt) +(\g:5mm) node {$\z$};
			\draw [<->] ({-pi},-1.7)--({pi},-1.7) ; 
			\draw (0,0) node[below right]{$O$} (0,-1.7) node[below]{$T=2\pi$}
			(0,1) node[above right]{$1$} (0,-1) node[below right]{$-1$}
			;
			\clip (-9.5,-1.4) rectangle (9.5,1.6) ;
			\draw[thick,samples=100,domain=-9.3:9.3] plot(\x,{cos((\x)*180/pi)});
			
		\end{tikzpicture}
	\end{center}
	\end{tc}
	\subsubsection{Đồ thị và tính chất của hàm số $y=\tan x$}
	\begin{tc}
		Hàm số $y=\tan x$:
		\begin{itemize}
			\item    Có tập xác định là $\mathbb{R} \setminus\left\{\dfrac{\pi}{2}+k \pi \Big| k \in \mathbb{Z} \right\}$ và tập giá trị là $\mathbb{R}$;
			\item    Là hàm số lẻ và tuần hoàn với chu kì $\pi$;
			\item    Đồng biến trên mỗi khoảng $\left(-\dfrac{\pi}{2}+k \pi ; \dfrac{\pi}{2}+k \pi\right)$, $k \in \mathbb{Z}$;
			\item    Có đồ thị đối xứng qua gốc toạ độ.
		\end{itemize}
	\begin{center}
		\begin{tikzpicture}[>=stealth,scale=0.77,transform shape] 
			\path
			({-1.5*pi},0) coordinate (X1)
			({-pi},0) coordinate (X2)
			({-0.5*pi},0) coordinate (X3)
			({0.5*pi},0) coordinate (X4)
			({pi},0) coordinate (X5)
			({1.5*pi},0) coordinate (X6)
			;
			\draw[->] (-6.5,0) -- (6.5,0) node[below] {\small $x$};
			\draw[->] (0,-3.5) -- (0,3.5) node[right] {\small $y$};
			\draw [dashed] ({-3*pi/2},3.5)--({-3*pi/2},-3.5) ({-pi/2},3.5)--({-pi/2},-3.5) ({pi/2},3.5)--({pi/2},-3.5) ({3*pi/2},3.5)--({3*pi/2},-3.5) ;
			\foreach \x/\g/\z in {X1/-150/-\tfrac{3\pi}{2},X2/-40/-\pi,X3/-40/-\tfrac{\pi}{2},X4/-40/\tfrac{\pi}{2},X5/-400/\pi,X6/-40/\tfrac{3\pi}{2}} 
			\fill[black] (\x) circle(1pt) +(\g:5mm) node {$\z$};
			\draw (0,0) node[below right]{$O$};
			\clip (-6.5,-3.5) rectangle (6.5,3.5) ;
			\draw[thick,samples=100,domain={-pi/2+0.2}:{pi/2-0.2}] plot(\x,{tan((\x)*180/pi)});
			\draw[thick,samples=100,domain={pi/2+0.2}:{3*pi/2-0.2}] plot(\x,{tan((\x)*180/pi)});
			\draw[thick,samples=100,domain={-3*pi/2+0.2}:{-pi/2-0.2}] plot(\x,{tan((\x)*180/pi)});
		\end{tikzpicture}
	\end{center}
	\end{tc}
	\subsubsection{Đồ thị và tính chất của hàm số $y=\cot x$}
	\begin{tc}
		Hàm số $y=\cot x$:
		\begin{itemize}
			\item    Có tập xác định là $\mathbb{R} \setminus\{k \pi \mid k \in \mathbb{Z} \}$ và tập giá trị là $\mathbb{R}$; 
			\item    Là hàm số lẻ và tuần hoàn với chu kì $\pi$;
			\item    Nghịch biến trên mỗi khoảng $(k \pi ; \pi+k \pi), k \in \mathbb{Z}$;
			\item    Có đồ thị đối xứng qua gốc toạ độ.
		\end{itemize}
	\begin{center}
		\begin{tikzpicture}[>=stealth,scale=0.77,transform shape] 
			\path
			({-2*pi},0) coordinate (X1)
			({-1.5*pi},0) coordinate (X2)
			({-pi},0) coordinate (X3)
			({-0.5*pi},0) coordinate (X4)
			({0.5*pi},0) coordinate (X5)
			({pi},0) coordinate (X6)
			({1.5*pi},0) coordinate (X7)
			({2*pi},0) coordinate (X8)
			;
			\draw[->] (-7.5,0) -- (7.5,0) node[below] {\small $x$};
			\draw[->] (0,-3.5) -- (0,3.5) node[left] {\small $y$};
			\draw [dashed] ({-2*pi},3.5)--({-2*pi},-3.5) ({-pi},3.5)--({-pi},-3.5) ({pi},3.5)--({pi},-3.5) ({2*pi},3.5)--({2*pi},-3.5) ;
			\foreach \x/\g/\z in {X1/-150/-2\pi,X2/-130/-\tfrac{3\pi}{2},X3/-140/-\pi,X4/-100/-\tfrac{\pi}{2},X5/-100/\tfrac{\pi}{2},X6/-120/\pi,X7/-100/\tfrac{3\pi}{2}, X8/-120/2\pi}
			\fill[black] (\x) circle(1pt) +(\g:5mm) node {$\z$};
			\draw (0,0) node[below left]{$O$};
			\clip (-6.5,-3.5) rectangle (6.5,3.5) ;
			\draw[thick,samples=100,domain={0.2}:{pi-0.2}] plot(\x,{cot((\x)*180/pi)});
			\draw[thick,samples=100,domain={pi+0.2}:{2*pi-0.2}] plot(\x,{cot((\x)*180/pi)});
			\draw[thick,samples=100,domain={-0.2}:{-pi+0.2}] plot(\x,{cot((\x)*180/pi)});
			\draw[thick,samples=100,domain={-pi-0.2}:{-2*pi+0.2}] plot(\x,{cot((\x)*180/pi)});
		\end{tikzpicture}
	\end{center}
	\end{tc}
\end{tomtat}
\subsection{Các dạng toán thường gặp}

% \foreach \i in {1,2,...,5} {\input{data/11KNTT/data1/1K1-3-\i.tex}}

%%Bài 4. PTLG
% 
\setcounter{section}{3}
\section{Phương trình lượng giác cơ bản}
\subsection{Tóm tắt lý thuyết}
\begin{tomtat}
\subsubsection{Phương trình $\sin x=m$}
\begin{itemize}
	\item Với $|m|>1$ thì phương trình $\sin x=m$ vô nghiệm.
	\item Với $|m|\leq 1$, sẽ tồn tại duy nhất $\alpha \in \left[-\dfrac{\pi}{2}; \dfrac{\pi}{2}\right]$ thỏa mãn $\sin\alpha=m$. Khi đó
	\begin{center}
		$\sin x=m\Leftrightarrow\sin x=\sin\alpha\Leftrightarrow\hoac{&x=\alpha+k2\pi\\&x=\pi-\alpha+k2\pi}$ ($k\in \mathbb{Z}$).
	\end{center}
\item Nếu số đo của góc $\alpha$ được đo bằng đơn vị độ thì
\begin{center}
	$\sin x=\sin\alpha^\circ\Leftrightarrow\hoac{&x=\alpha^\circ+k360^\circ\\&x=180^\circ-\alpha^\circ+k360^\circ}$ ($k\in\mathbb{Z}$).
\end{center}
\item Tổng quát,
\begin{center}
	$\sin f(x)=\sin g(x)\Leftrightarrow\hoac{&f(x)=g(x)+k2\pi\\&f(x)=\pi - g(x)+k2\pi}$ ($k\in\mathbb{Z}$).
\end{center}
\item Một số trường hợp đặt biệt:
\begin{enumEX}[\faCheckCircleO]{1}
	\item $\sin x=0\Leftrightarrow x=k\pi$, $k\in\mathbb{Z}$.
	\item $\sin x=1\Leftrightarrow x=\dfrac{\pi}{2}+k2\pi$, $k\in\mathbb{Z}$.
	\item $\sin x=-1\Leftrightarrow x=-\dfrac{\pi}{2}+k2\pi$, $k\in \mathbb{Z}$.
\end{enumEX}
\end{itemize}
	\subsubsection{Phương trình $\cos x=m$}
\begin{itemize}
	\item Với $|m|>1$ thì phương trình $\cos x=m$ vô nghiệm.
	\item Với $|m|\leq 1$, sẽ tồn tại duy nhất $\alpha\in\left[0; \pi\right]$ thỏa mãn $\cos x=m$. Khi đó
	\begin{center}
		$\cos x=m\Leftrightarrow\cos x=\cos \alpha\Leftrightarrow\hoac{&x=\alpha+k2\pi\\&x=-\alpha+k2\pi}$ ($k\in\mathbb{Z}$).
	\end{center}
\item Nếu số đo của góc $\alpha$ được đo bằng đơn vị độ thì
\begin{center}
	$\cos x=\cos\alpha\Leftrightarrow\hoac{&x=\alpha^\circ+k360^\circ\\&x=-\alpha^\circ+k360^\circ}$ ($k\in\mathbb{Z}$).
\end{center}
\item Tổng quát,
\begin{center}
	$\cos f(x)=\cos g(x)\Leftrightarrow\hoac{&f(x)=g(x)+k2\pi\\&f(x)=-g(x)+k2\pi}$ ($k\in \mathbb{Z}$)
\end{center}
\item Một số trường hợp đặc biệt:
\begin{enumEX}[\faCheckCircleO]{1}
\item $\cos x=0\Leftrightarrow x=\dfrac{\pi}{2}+k\pi$, $k\in\mathbb{Z}$.
\item $\cos x=1\Leftrightarrow x=k2\pi$, $k\in\mathbb{Z}$.
\item $\cos x=-1\Leftrightarrow x=\pi+k2\pi$, $k\in\mathbb{Z}$.
\end{enumEX}
\end{itemize}
\subsubsection{Phương trình $\tan x=m$}
\begin{itemize}
	\item Với mọi $m\in\mathbb{R}$, tồn tại duy nhất $\alpha\in\left(-\dfrac{\pi}{2};\dfrac{\pi}{2}\right)$ thỏa mãn $\tan \alpha=m$. Khi đó
	\begin{center}
		$\tan x=m\Leftrightarrow\tan x=\tan \alpha\Leftrightarrow x=\alpha+k\pi$ ($k\in\mathbb{Z}$).
	\end{center}
\item Nếu số đo của góc $\alpha$ được đo bằng đơn vị độ thì
\begin{center}
	$\tan x=\tan\alpha^\circ\Leftrightarrow x=\alpha^\circ+k180^\circ$, $k\in \mathbb{Z}$
\end{center}
\item Tổng quát,
\begin{center}
	$\tan f(x)=\tan g(x)\Leftrightarrow f(x)=g(x)+k\pi$, $k\in\mathbb{Z}$.
\end{center}
\end{itemize}
\subsubsection{Phương trình $\cot x=m$}
\begin{itemize}
	\item Với mọi $m\in\mathbb{R}$, tồn tại duy nhất $\alpha\in\left(0;\pi\right)$ thỏa mãn $\cot\alpha=m$. Khi đó
	\begin{center}
		$\cot x=m\Leftrightarrow\cot x=\cot\alpha\Leftrightarrow x=\alpha+k\pi$ $k\in\mathbb{Z}$.
	\end{center}
\item Nếu số đo của góc $\alpha$ được đo bằng đơn vị độ thì
\begin{center}
	$\cot x=\cot\alpha^\circ\Leftrightarrow x=\alpha^\circ+k180^\circ$, $k\in\mathbb{Z}$.
\end{center}
\item Tổng quát,
\begin{center}
	$\cot f(x)=\cot g(x)\Leftrightarrow f(x)=g(x)+k\pi$, $k\in\mathbb{Z}$.
\end{center}
\end{itemize}
\end{tomtat}


% \foreach \i in {2,3,4,5,6,7,11} {\input{data/11KNTT/data1/1K1-4-\i.tex}}

%%Ôn tập chương I
% \newpage
% \section*{LÝ THUYẾT GÓC LƯỢNG GIÁC - GIÁ TRỊ - HÀM SỐ LƯỢNG GIÁC}
\boldmath
\subsection{GTLG Góc lượng giác}
\begin{itemize}
	\item \textbf{Đổi đơn vị đo}: \fbox{1 vòng = $360^\circ = 2\pi\ rad$}, \fbox{$180^\circ = \pi rad$}
	      \begin{center}
		      \renewcommand{\arraystretch}{2}
		      \begin{tabular}{|l|c|c|c|c|c|c|c|c|c|}
			      \hline Độ     & $0^{\circ}$ & $30^{\circ}$     & $45^{\circ}$     & $60^{\circ}$     & $90^{\circ}$     & $120^{\circ}$      & $135^{\circ}$      & $150^{\circ}$      & $180^{\circ}$ \\
			      \hline Rađian & 0           & $\dfrac{\pi}{6}$ & $\dfrac{\pi}{4}$ & $\dfrac{\pi}{3}$ & $\dfrac{\pi}{2}$ & $\dfrac{2 \pi}{3}$ & $\dfrac{3 \pi}{4}$ & $\dfrac{5 \pi}{6}$ & $\pi$         \\
			      \hline
		      \end{tabular}
	      \end{center}
	      \immini{
	\item \textbf{Độ dài cung tròn} bán kính $R$ số đo $\alpha$ rad là \fbox{$l=R\alpha$}.
	\item \textbf{Điểm biểu diễn góc lượng giác} $\alpha$ lên đường tròn lượng giác là $M$. Khi đó $M$ cũng biểu diễn các góc lượng giác $\alpha+k2\pi$.\\
	      Góc $\alpha$ và $\beta$ có chung điểm biểu diễn khi \fbox{$\alpha - \beta = k2\pi$} (chẵn lần $\pi$)}
	      {
	      \begin{tikzpicture}[line join = round, line cap = round, >=stealth, font=\footnotesize, scale=0.4]
		      \tikzset{label style/.style={font=\footnotesize}}
		      \path (0,0) coordinate (O)
		      (3,0) coordinate (A)
		      (0,3) coordinate (B)
		      (0,-3) coordinate (B')
		      (-3,0) coordinate (A')
		      (0:0)++(150:3) coordinate (M)
		      ($(O)!(M)!(A')$) coordinate (H)
		      ($(O)!(M)!(B)$) coordinate (K)
		      ;
		      \draw[->] (-4,0) -- (4,0) node[above,blue]{$x$};
		      \draw[->] (0,-4) -- (0.,4) node[left,blue]{$y$};
		      \draw[orange] (O) circle (3cm);
		      \draw[rotate=0,->,green!50!black] (0.5,0) arc (0:150:0.5cm);
		      \draw (0.35,0.25) node[above,blue] {$\alpha$};
		      \draw[dashed] (H)--(M)--(K);
		      \draw[green!50!black] (M)--(O);
		      \foreach \p/\r in {A/-45,M/150,O/-150,A'/-135,B'/-45,B/45}
		      \fill (\p) circle (1pt) node[shift={(\r:3mm)},blue]{$\p$};
	      \end{tikzpicture}
	      }
\end{itemize}
\begin{multicols}{2}
	\begin{khung4}{Định nghĩa GTLG}
		\immini{\begin{itemize}
				\item $\cos \alpha =x$
				\item $\sin \alpha = y$
				\item $\tan \alpha =\dfrac{\sin\alpha}{\cos\alpha}=\dfrac{y}{x}$
				\item $\cot\alpha=\dfrac{\cos\alpha}{\sin\alpha}=\dfrac{x}{y}$
			\end{itemize}}
		{\begin{tikzpicture}[line join = round, line cap = round, >=stealth, font=\footnotesize, scale=0.4]
				\tikzset{label style/.style={font=\footnotesize}}
				\path (0,0) coordinate (O)
				(3,0) coordinate (A)
				(0,3) coordinate (B)
				(0,-3) coordinate (B')
				(-3,0) coordinate (A')
				(0:0)++(40:3) coordinate (M)
				($(O)!(M)!(A')$) coordinate (H)
				($(O)!(M)!(B)$) coordinate (K)
				;
				\draw[->] (-4,0) -- (4,0) node[above,blue]{$x$};
				\draw[->] (0,-4) -- (0.,4) node[left,blue]{$y$};
				\draw[orange] (O) circle (3cm);
				\draw[rotate=0,->,green!50!black] (0.7,0) arc (0:40:0.7cm);
				\draw (1,0) node[above,blue] {$\alpha$};
				\draw[dashed] (H)--(M)--(K);
				\draw[green!50!black] (M)--(O);
				\draw[blue,fill=black] (0,2) node[left]{$\sin\alpha$}(2,0) circle(1pt) node[below]{$\cos\alpha$}(3,2.3) node{$M(x;y)$};
				\foreach \p/\r in {A/-45,O/-135,A'/-135,B'/-45,B/45}
				\fill (\p) circle (1pt) node[shift={(\r:3mm)},blue]{$\p$};
			\end{tikzpicture}}
	\end{khung4}
	\begin{khung4}{Các công thức lượng giác cơ bản}
		\begin{itemize}
			\item $\sin^2 \alpha  + \cos^2 \alpha =1$
			\item $ 1+ \tan^2 \alpha= \dfrac{1}{\cos^2 \alpha}$ $\left(\alpha \neq \dfrac{\pi}{2}+k\pi , k\in \mathbb{Z}\right)$
			\item $ 1+ \cot^2 \alpha= \dfrac{1}{\sin^2 \alpha}$ $\left(\alpha \neq k\pi , k\in \mathbb{Z}\right)$
			\item $\tan \alpha \cdot \cot \alpha =1 $ $\left(\alpha \neq \dfrac{k\pi}{2}, k\in \mathbb{Z}\right)$
		\end{itemize}
	\end{khung4}
\end{multicols}
\textit{\underline{Chú ý}: $\tan\alpha$ xác định khi $\alpha\neq\dfrac{\pi}{2}+k\pi\,\,  (k\in\mathbb{Z})$ và $\cot\alpha$ xác định khi $\alpha\neq k\pi\,\,  (k\in\mathbb{Z})$.}

\begin{multicols}{2}
	\begin{khung4}{$\cos$ đối}
		\immini{
			\begin{itemize}
				\item $\cos (-\alpha)=\cos \alpha$
				\item $\sin (-\alpha) =-\sin \alpha$
				\item $\tan (-\alpha) =-\tan \alpha$
				\item $\cot (-\alpha) =-\cot \alpha$
			\end{itemize}
		}{\begin{tikzpicture}[line join = round, line cap = round, >=stealth, font=\footnotesize, scale=0.4]
				\tikzset{label style/.style={font=\footnotesize}}
				\path (0,0) coordinate (O)
				(3,0) coordinate (A)
				(0:0)++(120:3) coordinate (M)
				(0:0)++(-120:3) coordinate (N)
				(0,4) coordinate (C)
				(0,-4) coordinate (D)
				($(O)!(M)!(C)$) coordinate (E)
				($(O)!(N)!(D)$) coordinate (F)
				;
				\draw[->] (-4,0) -- (4,0) node[above,blue]{$x$};
				\draw[->] (0,-4) -- (0.,4) node[left,blue]{$y$};
				\draw[orange] (O) circle (3cm);
				\draw[rotate=0,->,red] (0.5,0) arc (0:120:0.5cm);
				\draw[rotate=0,->,green!50!black] (0.6,0) arc (0:-120:0.6cm);
				\draw (0,0) node[above right=2pt,blue] {$\alpha$} (0,-0) node[below right=2pt,blue]{$-\alpha$};
				\draw[dashed] (E)--(M)--(N)--(F);
				\draw[green!50!black] (M)--(O);
				\draw[red] (N)--(O);
				\foreach \p/\r in {A/-45,M/120,N/-120,O/-150}
				\fill (\p) circle (1pt) node[shift={(\r:3mm)},blue]{$\p$};
			\end{tikzpicture}}
	\end{khung4}
	\begin{khung4}{phụ chéo}
		\immini{\begin{itemize}
				\item $\sin \left( \dfrac{\pi}{2}-\alpha\right)=\cos \alpha$
				\item $\cos \left( \dfrac{\pi}{2}-\alpha\right)=\sin \alpha$
				\item $\tan \left( \dfrac{\pi}{2}-\alpha\right)=\cot \alpha$
				\item $\cot \left( \dfrac{\pi}{2}-\alpha\right)=\tan \alpha$
			\end{itemize}}
		{\begin{tikzpicture}[line join = round, line cap = round, >=stealth, font=\footnotesize, scale=0.4]
				\tikzset{label style/.style={font=\footnotesize}}
				\path (0,0) coordinate (O)
				(3,0) coordinate (A)
				(0:0)++(20:3) coordinate (M)
				(0:0)++(70:3) coordinate (N)
				(0,4) coordinate (C)
				(4,0) coordinate (D)
				($(O)!(M)!(C)$) coordinate (E)
				($(O)!(N)!(D)$) coordinate (F)
				(2.82,0) coordinate (G)
				(0,2.82) coordinate (H)
				;
				\draw[->] (-4,0) -- (4,0) node[above,blue]{$x$};
				\draw[->] (0,-4) -- (0.,4) node[left,blue]{$y$};
				\draw[orange] (O) circle (3cm);
				\draw[rotate=0,->,red] (0.7,0) arc (0:70:0.7cm);
				\draw[rotate=0,->,green!50!black] (1.6,0) arc (0:20:1.6cm);
				\draw (2,0) node[above,blue] {$\alpha$} (-2,3.5) node[blue]{$\frac{\pi}{2}-\alpha$};
				\draw[dashed] (E)--(M) (F)--(N) (G)--(M) (H)--(N);
				\draw[dashed] (-3,-3)--(3,3);
				\draw[->] (-2,3)--(0.5,2);
				\draw[red] (O)--(N);
				\draw[green!50!black] (O)--(M);
				\foreach \p/\r in {A/-45,M/20,N/70,O/-220}
				\fill (\p) circle (1pt) node[shift={(\r:3mm)},blue]{$\p$};
			\end{tikzpicture}}
	\end{khung4}
	\begin{khung4}{$\sin$ bù}
		\immini{\begin{itemize}
				\item $\sin (\pi -\alpha)=\sin \alpha$
				\item $\cos (\pi -\alpha) =-\cos \alpha$
				\item $\tan (\pi -\alpha) =-\tan \alpha$
				\item $\cot (\pi -\alpha) =-\cot \alpha$
			\end{itemize}}
		{\begin{tikzpicture}[line join = round, line cap = round, >=stealth, font=\footnotesize, scale=0.4]
				\tikzset{label style/.style={font=\footnotesize}}
				\path (0,0) coordinate (O)
				(3,0) coordinate (A)
				(0:0)++(30:3) coordinate (M)
				(0:0)++(150:3) coordinate (N)
				(4,0) coordinate (C)
				(-4,0) coordinate (D)
				($(O)!(M)!(C)$) coordinate (E)
				($(O)!(N)!(D)$) coordinate (F)
				;
				\draw[->] (-4,0) -- (4,0) node[above,blue]{$x$};
				\draw[->] (0,-4) -- (0.,4) node[left,blue]{$y$};
				\draw[orange] (O) circle (3cm);
				\draw[rotate=0,->,red] (0.5,0) arc (0:150:0.5cm);
				\draw[rotate=0,->,green!50!black] (1.6,0) arc (0:30:1.6cm);
				\draw (2,0) node[above,blue] {$\alpha$} (0.3,1.5) node[below,blue]{$\pi-\alpha$};
				\draw[dashed] (E)--(M)--(N)--(F);
				\draw[red] (O)--(N);
				\draw[green!50!black] (O)--(M);
				\foreach \p/\r in {A/-45,M/30,N/150,O/-130}
				\fill (\p) circle (1pt) node[shift={(\r:3mm)},blue]{$\p$};
			\end{tikzpicture}}
	\end{khung4}
	\begin{khung4}{$\pm \pi\ \tan, \cot$}
		\immini{\begin{itemize}
				\item $\sin (\pi +\alpha)=-\sin \alpha$
				\item $\cos (\pi +\alpha)=-\cos \alpha$
				\item $\tan (\pi +\alpha)=\tan \alpha$
				\item $\cot (\pi +\alpha)=\cot \alpha$
			\end{itemize}}
		{\begin{tikzpicture}[line join = round, line cap = round, >=stealth, font=\footnotesize, scale=0.4]
				\tikzset{label style/.style={font=\footnotesize}}
				\path (0,0) coordinate (O)
				(3,0) coordinate (A)
				(0:0)++(60:3) coordinate (M)
				(0:0)++(240:3) coordinate (N)
				;
				\draw[->] (-4,0) -- (4,0) node[above,blue]{$x$};
				\draw[->] (0,-4) -- (0.,4) node[left,blue]{$y$};
				\draw[orange] (O) circle (3cm);
				\draw[rotate=0,->,red] (1.7,0) arc (0:240:1.7cm);
				\draw[rotate=0,->,green!50!black] (0.6,0) arc (0:60:0.6cm);
				\draw (1,0) node[above,blue] {$\alpha$};
				\draw (-1.2,1) node[below,blue,rotate=60]{$\pi+\alpha$};
				\draw[red] (O)--(N);
				\draw[green!50!black] (O)--(M);
				\foreach \p/\r in {A/-45,M/60,N/240,O/-150}
				\fill (\p) circle (1pt) node[shift={(\r:3mm)},blue]{$\p$};
			\end{tikzpicture}}
	\end{khung4}
\end{multicols}
\newpage

\subsection{Công thức lượng giác}
\subsubsection{Công thức cộng}
\begin{khung4}{Công thức cộng}
	\begin{multicols}{2}
	\begin{itemize}
		\item $\cos (a-b) = \cos a \cos b + \sin a\sin b$.
		\item $\cos (a+b) = \cos a \cos b - \sin a\sin b$.
		\item $\sin (a-b) = \sin a \cos b - \sin b \cos a$.
		\item $\sin (a+b) = \sin a \cos b + \sin b \cos a$.
		\item $\tan (a-b) = \dfrac{\tan a - \tan b}{1+\tan a \tan b}$.
		\item $\tan (a+b) = \dfrac{\tan a + \tan b}{1-\tan a \tan b}$.
	\end{itemize}
\end{multicols}
\end{khung4}
\begin{khung4}{Trường hợp đặc biệt}
	\begin{itemize}
		\item $\sin x + \cos x = \sqrt{2}\sin\left(x+\dfrac{\pi}{4}\right) = \sqrt{2}\cos \left(x-\dfrac{\pi}{4}\right)$.
		\item $\sqrt{3}\sin x + \cos x = 2\sin \left(x+\dfrac{\pi}{6}\right) = 2\cos \left(x-\dfrac{\pi}{3}\right)$.
		\item $\sin x + \sqrt{3}\cos x = 2\sin \left(x+ \dfrac{\pi}{3}\right) = 2\cos \left(x-\dfrac{\pi}{6}\right)$.
	\end{itemize}
\end{khung4}
\subsubsection{Công thức nhân đôi}
\begin{multicols}{2}
	\begin{khung4}{Công thức nhân đôi}
		\begin{itemize}
			\item $\sin 2a = 2\sin a \cos a$.
			\item $\cos 2a = \cos^2a-\sin^2a = 2\cos^2a-1 = 1-2\sin^2a$.
			\item $\tan 2a = \dfrac{2\tan a}{1-\tan^2a}$.
		\end{itemize}
	\end{khung4}
	\begin{khung4}{Công thức hạ bậc}
		\begin{itemize}
			\item $\sin^2a= \dfrac{1-\cos 2a}{2}$.
			\item $\cos^2a = \dfrac{1+\cos 2a}{2}$.
			\item $\tan^2a=\dfrac{1-\cos2a}{1+\cos 2a}$.
		\end{itemize}
	\end{khung4}
\end{multicols}

\begin{note}
	Áp dụng công thức cộng cho $3a = a +2a$, ta có công thức nhân ba:
\end{note}
\begin{khung4}{Công thức nhân ba}
	\begin{multicols}{2}
	\begin{itemize}
		\item $\sin3a= 3\sin a -4\sin^3a$.		
		\item $\cos3a= 4\cos^3a-3\cos a$.
		\item $\tan3a = \dfrac{3\tan a - \tan^3 a}{1-3\tan^2a}$.
	\end{itemize}
\end{multicols}
\end{khung4}

\subsubsection{Công thức biến đổi tích thành tổng}
\begin{khung4}{Công thức tích thành tổng}
	\begin{multicols}{2}
	\begin{itemize}
		\item $\cos a \cos b = \dfrac{1}{2}\left[\cos (a-b) + \cos (a+b)\right]$.
		\item $\sin a \sin b = \dfrac{1}{2}\left[\cos (a-b)-\cos(a+b)\right]$.
		\item $\sin a \cos b = \dfrac{1}{2}\left[\sin (a-b)+\sin (a+b)\right]$.
	\end{itemize}
	\end{multicols}
\end{khung4}
\subsubsection{Công thức biến đổi tổng thành tích}
Công thức biến đổi tổng thành tích được xây dựng bằng cách $a=\dfrac{a+b}{2}, b = \dfrac{a-b}{2}$ trong công thức biến đổi tích thành tổng.
\begin{khung4}{Công thức tổng thành tích}
	\begin{multicols}{2}
	\begin{itemize}
		\item $\cos a+ \cos b = 2\cos\dfrac{a+b}{2}\cos \dfrac{a-b}{2}$.
		\item $\cos a- \cos b = -2\sin\dfrac{a+b}{2}\sin \dfrac{a-b}{2}$.
		\item $\sin a+ \sin b = 2\sin\dfrac{a+b}{2}\cos \dfrac{a-b}{2}$.
		\item $\sin a -\sin b = 2\cos\dfrac{a+b}{2}\sin \dfrac{a-b}{2}$.
	\end{itemize}
\end{multicols}
\end{khung4}
\newpage

\subsection{Hàm số lượng giác}
\begin{khung4}{Hàm số chẵn, hàm số lẻ}
	\begin{multicols}{2}
		\begin{itemize}
			\item Hàm số $f(x)$ được gọi là \textbf{hàm số chẵn} nếu $\forall x \in \mathscr{D}$ thì $-x \in \mathscr{D}$ và $f(-x)=f(x)$. Đồ thị của một \textbf{hàm số chẵn} nhận \textbf{trục tung} là trục đối xứng.
			\item Hàm số $f(x)$ được gọi là \textbf{hàm số lẻ} nếu $\forall x \in \mathscr{D}$ thì $-x \in \mathscr{D}$ và $f(-x)=-f(x)$. Đồ thị của một \textbf{hàm số lẻ} nhận \textbf{gốc toạ độ} là tâm đối xứng.
		\end{itemize}
	\end{multicols}
	Các hàm số $y=\sin x$, $y=\tan x$, $y=\cot x$ là hàm số \textit{lẻ}, hàm số $y=\cos x$ là hàm số \textit{chẵn}.
\end{khung4}
\begin{khung4}{Hàm số tuần hoàn}
	\begin{dn}
		Hàm số $y=f(x)$ có tập xác định $\mathscr{D}$ được gọi là \textbf{hàm số tuần hoàn} nếu tồn tại số $T \neq 0$ sao cho với mọi $x \in \mathscr{D}$ ta có:
		\begin{itemize}
			\item $x+T \in \mathscr{D}$ và $x-T \in \mathscr{D}$;
			\item $f(x+T)=f(x)$.
		\end{itemize}
		Số $T$ dương nhỏ nhất thỏa mãn các điều kiện trên (nếu có) được gọi là \textbf{chu kì} của hàm số tuần hoàn đó.
	\end{dn}
	Các hàm số $y=A \sin \omega x$ và $y=A \cos \omega x$ $(\omega>0)$ là những hàm số tuần hoàn với chu kì $T=\dfrac{2 \pi}{\omega}$.\\
	Các hàm số $y=A \tan \omega x$ và $y=A \cot \omega x$ $(\omega>0)$ là những hàm số tuần hoàn với chu kì $T=\dfrac{\pi}{\omega}$.
\end{khung4}
\begin{center}
	\begin{tikzpicture}[line join = round, line cap = round, >=stealth, font=\small, thick, scale=.7]
		\path (0,0) coordinate (O)
		(3,0) coordinate (A)
		(0,3) coordinate (B)
		(0,-3) coordinate (B')
		(-3,0) coordinate (A');
		\draw[->] (-4,0) -- (4,0) node[above,blue]{$\cos$};
		\draw[->] (0,-4) -- (0.,3.5) node[left,blue]{$\sin$};
		\draw[orange] (O) circle (3cm);
		\draw[->,green!50!black] (40:3.5) arc (40:50:3.5cm) node[midway,above right] {(+)};
		\draw[->,green!50!black](5,0)node[align=center,rotate=90,below]{$\sin x$ đồng biến\\ trên $\left(-\dfrac{\pi}{2}+k2\pi;\dfrac{\pi}{2}+k2\pi\right)$}
		(-30:5) arc (-30:30:5cm) ;
		\draw[->,red!50!black](0,5)node[align=center,above]{$\cos x$ nghịch biến\\ trên $\left(k2\pi;\pi+k2\pi\right)$}
		(60:5) arc (60:120:5cm) ;
		\draw[->,red!50!black](-5,0)node[align=center,rotate=-90,below]{$\sin x$ nghịch biến\\ trên $\left(\dfrac{\pi}{2}+k2\pi;\dfrac{3\pi}{2}+k2\pi\right)$}
		(150:5) arc (150:210:5cm) ;

		\draw[->,green!50!black](0,-5)node[align=center,below]{$\cos x$ đồng biến\\ trên $\left(-\pi+k2\pi;k2\pi\right)$}
		(-120:5) arc (-120:-60:5cm) ;

		\foreach \p/\r in {A/-45,O/-150,A'/-135,B'/-45,B/45}
		\fill (\p) circle (1pt) node[shift={(\r:3mm)},blue]{$\p$};
	\end{tikzpicture}
\end{center}

\subsection{Phương trình lượng giác}
\begin{khung4}{Phương trình $\sin x = a$.}
	\begin{enumerate}[\faCheckSquareO]
		\item Trường hợp $a>1$ hoặc $a<-1$ phương trình vô nghiệm.
		\item Trường hợp $a \in \{-1;0;1\}$.
		      \begin{multicols}{3}
			      \begin{tikzpicture}[smooth,samples=300,scale=0.8,>=stealth]
				      \draw[->] (-1.5,0)--(1.5,0) node[below]{\footnotesize$cos$};
				      \draw[->] (0,-1.5)--(0,1.8) node[right]{\footnotesize$sin$};
				      \draw (0,0) node[below left]{\footnotesize$O$};
				      \tkzDefPoints{0/0/I}
				      \draw[orange] (I) circle(1cm);
				      \draw[fill,blue] (0,1) circle(2.5pt) node[above left] {$B$};
				      \node[below] at (0,-1.5) {\fbox{$\sin x=1 \Leftrightarrow x=\tfrac{\pi}{2}+k2\pi$}};
			      \end{tikzpicture}
			      \hspace{-0.4cm}
			      \begin{tikzpicture}[smooth,samples=300,scale=0.8,>=stealth]
				      \draw[->] (-1.5,0)--(1.5,0) node[below]{\footnotesize$cos$};
				      \draw[->] (0,-1.5)--(0,1.8) node[right]{\footnotesize$sin$};
				      \draw (0,0) node[below left]{\footnotesize$O$};
				      \tkzDefPoints{0/0/I}
				      \draw[orange] (I) circle(1cm);
				      \draw[fill,blue] (0,-1) circle(2.5pt)node[below left] {$B'$};
				      \node[below] at (0,-1.5) {\fbox{$\sin x=-1 \Leftrightarrow x=-\tfrac{\pi}{2}+k2\pi$}};
			      \end{tikzpicture}
			      \hspace{-0.4cm}
			      \begin{tikzpicture}[smooth,samples=300,scale=0.8,>=stealth]
				      \draw[->] (-1.5,0)--(1.5,0) node[below]{\footnotesize$cos$};
				      \draw[->] (0,-1.5)--(0,1.8) node[right]{\footnotesize$sin$};
				      \draw (0,0) node[below left]{\footnotesize$O$};
				      \tkzDefPoints{0/0/I}
				      \draw[orange] (I) circle(1cm);
				      \draw[fill,blue] (1,0) circle(2.5pt) (-1,0) circle(2.5pt) node[above right] at (1,0) {$A$};
				      \node[above left,blue] at (-1,0) {$A'$};
				      \node[below] at (0,-1.5) {\fbox{$\sin x=0 \Leftrightarrow x=k\pi$}};
			      \end{tikzpicture}
		      \end{multicols}
		\item Trường hợp $a \in \left\{\pm \dfrac{1}{2};\pm \dfrac{\sqrt{2}}{2};\pm \dfrac{\sqrt{3}}{2}\right\}$ hoặc $a \in (-1;1)$. Ta bấm máy \shiftk \sink để đổi tìm góc $\alpha$ hoặc $\beta^\circ$.
			      \immini{
				      \begin{listEX}[1]
					      \item [\ding{172}] Công thức theo đơn vị rad:
					      $\sin x = \sin \alpha \Leftrightarrow \hoac{&x=\alpha+k2\pi\\&x=\pi-\alpha+k2\pi}, \,k \in \mathbb{Z}$
					      \item [\ding{173}] Công thức theo đơn vị độ: $\sin x = \sin \beta^\circ \Leftrightarrow \hoac{&x=\beta^\circ+k360^\circ\\&x=180^\circ-\beta^\circ+k360^\circ}, \,k \in \mathbb{Z}$
				      \end{listEX}
			      }{\begin{tikzpicture}[smooth,samples=300,scale=1,>=stealth]
					      \draw[->] (-1.5,0)--(1.5,0);
					      \draw[->] (0,-1.3)--(0,1.5) node[right]{\footnotesize$sin$};
					      \draw (0,0) node[below left]{\footnotesize$O$};
					      \tkzDefPoints{0/0/I}
					      \draw[orange] (I) circle(1cm);
					      \coordinate (M) at ($(I)+(50:1cm)$);
					      \coordinate (N) at ($(I)+(130:1cm)$);
					      \tkzDrawPoints[size=3,fill=blue](M,N)
					      \tkzDrawSegments(I,M I,N)
					      \tkzDrawSegments[dashed](M,N)
					      \tkzLabelPoints[right,blue](M)
					      \tkzLabelPoints[left,blue](N)
					      \node[below right] at (0,0.7) {$a$};
				      \end{tikzpicture}
			      }
	\end{enumerate}
\end{khung4}
\begin{khung4}{Phương trình $\cos x = a$.}
\begin{enumerate}[\faCheckSquareO]
	\item Trường hợp $a>1$ hoặc $a<-1$ phương trình vô nghiệm.
	\item Trường hợp $a \in \{-1;0;1\}$.
	      \begin{multicols}{3}
		      \begin{tikzpicture}[smooth,samples=300,scale=0.8,>=stealth]
			      \draw[->] (-1.5,0)--(1.8,0) node[below]{\footnotesize$cos$};
			      \draw[->] (0,-1.5)--(0,1.8) node[right]{\footnotesize$sin$};
			      \draw (0,0) node[below left]{\footnotesize$O$};
			      \tkzDefPoints{0/0/I}
			      \draw[orange] (I) circle(1cm);
			      \draw[fill,blue] (1,0) circle(2.5pt)node[above right]  {$A$};
			      \node[below] at (0,-1.5) {\fbox{$\cos x=1 \Leftrightarrow x=k2\pi$}};
		      \end{tikzpicture}
		      \begin{tikzpicture}[smooth,samples=300,scale=0.8,>=stealth]
			      \draw[->] (-1.5,0)--(1.5,0) node[below]{\footnotesize$cos$};
			      \draw[->] (0,-1.8)--(0,1.5) node[right]{\footnotesize$sin$};
			      \draw (0,0) node[below left]{\footnotesize$O$};
			      \tkzDefPoints{0/0/I}
			      \draw[orange] (I) circle(1cm);
			      \draw[fill,blue] (-1,0) circle(2.5pt)node[below left] {$A'$};
			      \node[below] at (0,-1.5) {\fbox{$\cos x=-1 \Leftrightarrow x=\pi+k2\pi$}};
		      \end{tikzpicture}
		      \begin{tikzpicture}[smooth,samples=300,scale=0.8,>=stealth]
			      \draw[->] (-1.5,0)--(1.8,0) node[below]{\footnotesize$cos$};
			      \draw[->] (0,-1.5)--(0,1.5);
			      \draw (0,0) node[below left]{\footnotesize$O$};
			      \tkzDefPoints{0/0/I}
			      \draw[orange] (I) circle(1cm);
			      \draw[fill,blue] (0,1) circle(2.5pt) (0,-1) circle(2.5pt) node[above right] at (0,1) {$B$};
			      \node[below left,blue] at (0,-1) {$B'$};
			      \node[below] at (0,-1.5) {\fbox{$\cos x=0 \Leftrightarrow x=\frac{\pi}{2}+k\pi$}};
		      \end{tikzpicture}
	      \end{multicols}
	\item Trường hợp $a \in \left\{\pm \dfrac{1}{2};\pm \dfrac{\sqrt{2}}{2};\pm \dfrac{\sqrt{3}}{2}\right\}$ hoặc $a \in (-1;1)$. Ta bấm máy \shiftk \cosk để tìm góc $\alpha$ hoặc $\beta^\circ$ tương ứng.
	      \immini{
		      \begin{listEX}[1]
			      \item [\ding{172}] Công thức theo đơn vị rad:
			      $\cos x = \cos \alpha \Leftrightarrow \hoac{&x=\alpha+k2\pi\\&x=-\alpha+k2\pi}, \,k \in \mathbb{Z}$
			      \item [\ding{173}] Công thức theo đơn vị độ: $\cos x = \cos \beta^\circ \Leftrightarrow \hoac{&x=\beta^\circ+k360^\circ\\&x=-\beta^\circ+k360^\circ}, \,k \in \mathbb{Z}$
		      \end{listEX}
	      }{\begin{tikzpicture}[smooth,samples=300,scale=1,>=stealth]
			      \draw[->] (-1.5,0)--(1.7,0)node[above]{\footnotesize$cos$};
			      \draw[->] (0,-1.3)--(0,1.5);
			      \draw (0,0) node[below left]{\footnotesize$O$};
			      \tkzDefPoints{0/0/I}
			      \draw[orange] (I) circle(1cm);
			      \coordinate (M) at ($(I)+(50:1cm)$);
			      \coordinate (N) at ($(I)+(-50:1cm)$);
			      \tkzDrawPoints[size=3,fill=blue](M,N)
			      \tkzDrawSegments(I,M I,N)
			      \tkzDrawSegments[dashed](M,N)
			      \tkzLabelPoints[above,blue](M)
			      \tkzLabelPoints[below,blue](N)
			      \node[below right] at (0.7,0) {$a$};
		      \end{tikzpicture}}
\end{enumerate}
\end{khung4}
\begin{khung4}{Phương trình $\tan x = a$ và $\cot x = b$.}
\begin{enumerate}[\faCheckSquareO]
	\item Trường hợp $a \in \left\{0;\pm \dfrac{\sqrt{3}}{3};\pm 1; \pm \sqrt{3}\right\}$ hoặc $a$ bất kì. Ta bấm máy \shiftk \tank để tìm góc $\alpha$ hoặc $\beta^\circ$ tương ứng.
		      \immini{
			      \begin{listEX}[1]
				      \item [\ding{172}] Công thức theo đơn vị rad:
				      $$\tan x = \tan \alpha \Leftrightarrow x=\alpha+k\pi, \,k \in \mathbb{Z}$$
				      \item [\ding{173}] Công thức theo đơn vị độ:
				      $$\tan x = \tan \beta^\circ \Leftrightarrow x=\beta^\circ +k180^\circ, \,k \in \mathbb{Z}$$
			      \end{listEX}
		      }{\begin{tikzpicture}[smooth,samples=300,scale=1,>=stealth]
				      \draw[->] (-1.5,0)--(1.5,0);
				      \draw[->] (0,-1.3)--(0,1.5);
				      \draw (0,0) node[below right]{\footnotesize$O$};
				      \tkzDefPoints{0/0/I, 1/0.9/A}
				      \draw[orange] (I) circle(1cm);
				      \draw[->] (1,-1.3)--(1,1.5)node[right]{\footnotesize$tan$};
				      \tkzInterLC[R](I,A)(I,1cm)\tkzGetPoints{M}{N}
				      \tkzDrawPoints[size=3,fill=blue](I,M,N,A)
				      \tkzDrawSegments(A,N)
				      \tkzLabelPoints[below,font=\footnotesize,blue](M)
				      \tkzLabelPoints[above,font=\footnotesize,blue](N)
				      \node[right] at (1,0.9) {$a$};
			      \end{tikzpicture}}
\end{enumerate}

\textbf{$\bigstar$ Phương trình $\cot x = b$.}
	$b \in \left\{\pm \dfrac{\sqrt{3}}{3};\pm 1; \pm \sqrt{3}\right\}$ hoặc $b$ bất kì. Ta bấm máy \shiftk \tank \fbox{$\tfrac{1}{b}$} để tìm góc $\alpha$ hoặc $\beta^\circ$ tương ứng. Riêng $b=0$ thì $\alpha=\dfrac{\pi}{2}$. Công thức nghiệm tương tự phương trình $\tan x =a$
\end{khung4}
% 
% \section{BÀI TẬP ÔN TẬP CHƯƠNG I}
\Opensolutionfile{ans}[ans/ans-1K1-4-OTC]
\begin{ex}%[Câu 1]%[1K1Y1-1]
	Đổi $225^\circ$ sang rađian.
	\choice
	{$\dfrac{4\pi}{5}$}
	{$\dfrac{6\pi}{5}$}
	{$\dfrac{3\pi}{7}$}
	{\True $\dfrac{5\pi}{4}$}
	%<MyLT2>
	\loigiai{
		Ta có $225^\circ = \dfrac{225}{180}\pi=\dfrac{5\pi}{4}$ (rađian).
	}
\end{ex}
\begin{ex}%[Câu 2]%[1K1Y1-3]
	Một đường tròn có bán kính $R=10$ cm. Độ dài cung $40^\circ$ trên đường tròn gần bằng
	\choice
	{$11$ cm}
	{$13$ cm}
	{\True $7$ cm}
	{$9$ cm}
	\loigiai{
		Ta có $40^\circ = 40\cdot \dfrac{\pi}{180} =\dfrac{2\pi}{9}$ rađian.\\
		Độ dài cung $l=\dfrac{2\pi}{9}\cdot 10=\dfrac{20\pi}{9}\approx 7$ cm.
	}
\end{ex}
\begin{ex}%[Câu 3]%[1K1B1-4]
	Bánh xe của người đi xe đạp quay được $2$ vòng trong $6$ giây. Hỏi trong $1$ giây, bánh xe quay được bao nhiêu độ?
	\choice
	{$60^\circ$}
	{$72^\circ$}
	{$240^\circ$}
	{\True $120^\circ$}
	\loigiai{
		Trong $6$ giây, bánh xe quay được $2\cdot 360^\circ=720^\circ$.\\
		Trong $1$ giây, bánh xe quay được $720^\circ\colon 6=120^\circ$.
	}
\end{ex}
\begin{ex}%[Câu 4]%[1K1Y1-6]
	Cho góc $\alpha$ thỏa mãn $90^\circ <\alpha <180^\circ$. Khẳng định nào sau đây đúng?
	\choice
	{$\cos\alpha>0$}
	{\True $\sin\alpha>0$}
	{$\tan\alpha>0$}
	{$\cot\alpha>0$}
	\loigiai
	{Vì $90^\circ <\alpha <180^\circ$ nên $\sin\alpha>0$, $\cos\alpha<0$, $\tan\alpha<0$ và $\cot\alpha<0$.}
\end{ex}
\begin{ex}%[Câu 5]%[1K1B1-6]
	Cho $\sin \alpha =\dfrac{1}{3}$ và $\dfrac{\pi}{2}<\alpha<\pi$. Khi đó $\cos \alpha$ có giá trị là
	\choice
	{$\cos \alpha =-\dfrac{2}{3}$}
	{$\cos \alpha =\dfrac{2\sqrt{2}}{3}$}
	{$\cos \alpha =\dfrac{8}{9}$}
	{\True $\cos \alpha =-\dfrac{2\sqrt{2}}{3}$}
	\loigiai{
		Ta có $\cos^2 \alpha =1-\sin^2 \alpha =1-\left(\dfrac{1}{3}\right)^2=\dfrac{8}{9}$.\\
		Vì $\dfrac{\pi}{2}<\alpha<\pi$ nên $\cos \alpha <0$.\\
		Do đó $\cos \alpha =-\dfrac{2\sqrt{2}}{3}$.
	}
\end{ex}
\begin{ex}%[Câu 6]%[1K1B1-7]
	Cho $A$, $B$, $C$ là ba góc của tam giác $ABC$. Trong các khẳng định sau, khẳng định nào \textbf{sai}?
	\choice
	{$\sin (B+C)=\sin A$}
	{$\cos (B+C)=-\cos A$}
	{\True $\tan (B+C)=\tan A$}
	{$\cot (B+C)=-\cot A$}
	\loigiai{
		Ta có $B+C=180^\circ-A$.\\Suy ra $\tan(B+C)=\tan (180^\circ-A)=-\tan A$.
	}
\end{ex}
\begin{ex}%[Câu 7]%[1K1B1-8]
	Tính giá trị biểu thức $P=\cos ^2\dfrac{\pi}{8}+\cos ^2\dfrac{{3\pi}}{8}+\cos ^2\dfrac{{5\pi}}{8}+\cos ^2\dfrac{{7\pi}}{8}.$
	\choice
	{$P=-1$}
	{$P=0$}
	{$P=1$}
	{\True $P=2$}
	\loigiai{Ta có $\cos ^2\dfrac{7\pi}{8}=\cos ^2\dfrac{\pi}{8}$ và $\cos ^2\dfrac{5\pi}{8}=\cos ^2\dfrac{3\pi}{8}$
		$\Rightarrow P=2\left({{\cos}^2\dfrac{\pi}{8}+{\cos}^2\dfrac{{3\pi}}{8}}\right)$.\\
		Vì $\dfrac{\pi}{8}+\dfrac{{3\pi}}{8}=\dfrac{\pi}{2}\Rightarrow \cos \dfrac{\pi}{8}=\sin \dfrac{{3\pi}}{8}\Rightarrow \cos ^2\dfrac{\pi}{8}=\sin ^2\dfrac{{3\pi}}{8}.$\\
		Do đó $ P=2 \left({{\sin}^2\dfrac{{3\pi}}{8}+{\cos}^2\dfrac{{3\pi}}{8}}\right)=2\cdot1=2.$
	}
\end{ex}
\begin{ex}%[Câu 8]%[1K1B1-8]
	Cho $\sin a + \cos a = - \dfrac 54$, khi đó giá trị của $\sin a \cos a$ bằng
	\choice
	{$1$}
	{$\dfrac{5}{4}$}
	{$\dfrac{3}{16}$}
	{\True $\dfrac{9}{32}$}
	\loigiai{
		$\sin a\cos a = \dfrac{(\sin a + \cos a)^2 -1}{2} = \dfrac{9}{32}$.
	}
\end{ex}
\begin{ex}%[Câu 9]%[1K1Y1-8]
	Cho $\tan x=\dfrac{1}{2}$. Tính $\tan \left(x+\dfrac{\pi}{4}\right)$.
	\choice
	{$2$}
	{$\dfrac{3}{2}$}
	{$6$}
	{\True $3$}
	\loigiai
	{
		Ta có $\tan \left(x+\dfrac{\pi}{4}\right)=\dfrac{\tan x+\tan \dfrac{\pi}{4}}{1-\tan x\cdot \tan \dfrac{\pi}{4}}=\dfrac{\dfrac{1}{2}+1}{1-\dfrac{1}{2}} = 3$.
	}
\end{ex}
\begin{ex}%[Câu 10]%[1K1Y1-3]
	Biểu diễn các góc lượng giác $\alpha=-\dfrac{5\pi}{6}$, $\beta=\dfrac{\pi}{3}$, $\gamma=\dfrac{25\pi}{3}$, $\delta=\dfrac{17\pi}{6}$ trên đường tròn lượng giác. Các góc nào có điểm biểu diễn trùng nhau?
	\choice
	{\True $\beta$ và $\gamma$}
	{$\alpha$, $\beta$, $\gamma$}
	{$\beta$, $\gamma$, $\delta$}
	{$\alpha$ và $\beta$}
	\loigiai{
		Ta có $\beta+8\pi=\dfrac{\pi}{3}+8\pi=\dfrac{25\pi}{3}=\gamma$.\\
		Do đó, $\beta$ và $\gamma$ có điểm biểu diễn trùng nhau trên đường tròn lượng giác.
	}
\end{ex}
\begin{ex}%[Câu 11]%[1K1Y1-7]
	Trong các khẳng định sau, khẳng định  nào là \textbf{sai}?
	\choice
	{$\sin(\pi-\alpha)=\sin\alpha$}
	{\True $\cos(\pi-\alpha)=\cos \alpha$}
	{$\sin(\pi+\alpha)=-\sin\alpha$}
	{$\cos(\pi+\alpha)=-\cos \alpha$}
	\loigiai{
		Ta có $\cos(\pi-\alpha)=-\cos \alpha$ nên $\cos(\pi-\alpha)=\cos \alpha$ là khẳng định \textbf{sai}.
	}
\end{ex}
\begin{ex}%[Câu 12]%[1K1Y1-2]
	Góc lượng giác nào tương ứng với chuyển động quay $3\dfrac{1}{5}$ vòng ngược chiều kim đồng hồ?
	\choice
	{$\dfrac{16 \pi}{5}$}
	{$\left(\dfrac{16}{5}\right)^\circ$}
	{\True $1152^\circ$}
	{$1152 \pi$}
	\loigiai{
		Chuyển động quay ngược chiều kim đồng hồ là quay theo chiều dương; góc tương ứng là
		$$3\dfrac{1}{5}\cdot 2\pi=\dfrac{32\pi}{5}, \text{ tương ứng với } 1152^\circ.$$
	}
\end{ex}

\begin{ex}%[Câu 13]%[1K1Y2-1]
	Trong các khẳng định sau, khẳng định nào \textbf{sai}?
	\choice
	{\True $\cos (a-b)=\cos a\cos b-\sin a\sin b$}
	{$\sin (a-b)=\sin a\cos b-\cos a\sin b$}
	{$\cos (a+b)=\cos a\cos b-\sin a\sin b$}
	{$\sin (a+b)=\sin a\cos b+\cos a\sin b$}
	\loigiai{
		Ta có $\cos (a-b)=\cos a\cos b+\sin a\sin b$ nên $\cos (a-b)=\cos a\cos b-\sin a\sin b$ là khẳng định \textbf{sai}.
	}
\end{ex}
\begin{ex}%[Câu 14]%[1K1Y1-7]
	Trong trường hợp nào dưới đây $\cos \alpha=\cos \beta$ và $\sin \alpha=-\sin \beta$?
	\choice
	{\True $\beta=-\alpha$}
	{$\beta=\pi-\alpha$}
	{$\beta=\pi+\alpha$}
	{$\beta=\dfrac{\pi}{2}+\alpha$}
	\loigiai{
		Trong trường hợp hai cung đối nhau thì các giá trị $\cos$ của chúng bằng nhau, các giá trị $\sin$ của chúng đối nhau.
	}
\end{ex}

\begin{ex}%[Câu 15]%[1K1B2-2]
	Nếu $\cos a=\dfrac{1}{4}$ thì $\cos 2 a$ bằng
	\choice
	{$\dfrac{7}{8}$}
	{\True $-\dfrac{7}{8}$}
	{$\dfrac{15}{16}$}
	{$-\dfrac{15}{16}$}
	\loigiai{
		Ta có  $\cos 2 a =2 \cos^2 a-1=2 \cdot \left( \dfrac{1}{4} \right)^2-1=-\dfrac{7}{8}$.
	}
\end{ex}
\begin{ex}%[Câu 16]%[1K1K2-2]
	Nếu $\tan (a+b)=3, \tan (a-b)=-3$ thì $\tan 2 a$ bằng
	\choice
	{\True $0$}
	{$\dfrac{3}{5}$}
	{$1$}
	{$-\dfrac{3}{4}$}
	\loigiai{
		Ta có $\tan (a+b)=3 \Leftrightarrow \tan a+ \tan b= 3- 3 \tan a \tan b$ \quad (1)\\
		và 
		$\tan (a-b)=-3 \Leftrightarrow \tan a- \tan b= -3- 3 \tan a \tan b$. \quad (2) \\
		Lấy vế trừ vế của (1) và (2) ta được $2\tan b=6\Leftrightarrow \tan b =3$.\\
		Thay $\tan b =3$ vào (1) ta được $\tan a= 0$.\\
		Khi đó $\tan 2 a = \dfrac{2 \tan a}{1- \tan^2 a}=0$.
	}
\end{ex}

\begin{ex}%[Câu 17]%[1K1K2-3]
	Nếu $\cos a=\dfrac{3}{5}$ và $\cos b=-\dfrac{4}{5}$ thì $\cos (a+b) \cos (a-b)$ bằng
	\choice
	{\True $0$}
	{$2$}
	{$4$}
	{$5$}
	\loigiai{
		Do  $\cos a=\dfrac{3}{5}$ và $\cos b=-\dfrac{4}{5}$ nên  $\cos 2a=-\dfrac{7}{25}$ và $\cos 2b=\dfrac{7}{25}$.\\
		Ta có $2 \cos (a+b) \cos (a-b)= \cos 2a +\cos 2b=-\dfrac{7}{25}+ \dfrac{7}{25}=0$.\\
		Do đó $\cos (a+b) \cos (a-b)=0$.
	}
\end{ex}




\begin{ex}%[Câu 18]%[1K1B2-3]
	Rút gọn biểu thức $M=\cos(a+b)\cos(a-b)-\sin (a+b)\sin(a-b)$, ta được
	\choice
	{$M=\sin 4a$}
	{$M=1-2\cos^2a$}
	{\True $M=1-2\sin^2a$}
	{$M=\cos 4a$}
	\loigiai{
		Ta có
		\allowdisplaybreaks
		\begin{eqnarray*}
			M&=&\cos(a+b)\cos(a-b)-\sin (a+b)\sin(a-b)\\
			&=&\dfrac{1}{2}\left(\cos2a+\cos 2b\right)+\dfrac{1}{2}\left(\cos2a-\cos 2b\right)\\
			&=&\cos 2a\\
			&=&1-2\sin^2a.
		\end{eqnarray*}
	}
\end{ex}
\begin{ex}%[Câu 19]%[1K1B2-2]
	Nếu $\sin x +\cos x = \dfrac{1}{2}$ thì $\sin 2x$ bằng
	\choice
	{$\dfrac{3}{4}$}
	{$\dfrac{3}{8}$}
	{$\dfrac{\sqrt{2}}{2}$}
	{\True $-\dfrac{3}{4}$}
	\loigiai{
		Ta có $\sin 2x=\left( {\sin x+\cos x} \right)^2-\left( {\sin^2 x+\cos ^2x} \right)=\left( {\dfrac{1}{2}} \right)^2-1=-\dfrac{3}{4}$.}
\end{ex}
\begin{ex}%[Câu 20]%[1K1Y2-3]
	Mệnh đề nào dưới đây đúng?
	\choice
	{\True $\cos3x\cdot\cos5x=\dfrac{1}{2}(\cos8x+\cos2x)$}
	{$\cos3x\cdot\cos5x=\dfrac{1}{2}(\cos8x-\cos2x)$}
	{$\cos3x\cdot\cos5x=\dfrac{1}{2}(\cos2x-\cos8x)$}
	{$\cos3x\cdot\cos5x=\dfrac{1}{2}(\sin8x+\sin2x)$}
	\loigiai{Ta có $\cos3x\cdot\cos5x=\dfrac{1}{2}[\cos(3x+5x)+\cos(3x-5x)]=\dfrac{1}{2}(\cos8x+\cos2x)$.}
\end{ex}
\begin{ex}%[Câu 21]%[1K1B2-4]
	Giả sử $3\sin ^4x-\cos ^4x=\dfrac{1}{2}$ thì $\sin ^4x+3\cos ^4x$ có giá trị bằng
	\choice
	{$2$}
	{\True $1$}
	{$4$}
	{$3$}
	\loigiai{
		\begin{eqnarray*}
			3\sin ^4x-\cos ^4x=\dfrac{1}{2}\Leftrightarrow 6\sin ^4x-2\cos ^4x=1&\Leftrightarrow& 6\sin ^4x-2\left(1-\sin^2\alpha\right)^2=1\\
			&\Leftrightarrow& 4\sin ^4x-4\sin ^2\alpha-3=0\\
			&\Leftrightarrow& \left(2\sin^2\alpha+3\right)\left(2\sin^2\alpha-1\right)=0\\
			&\Rightarrow&{\sin ^2}\alpha=\dfrac{1}{2}.
		\end{eqnarray*}
		Ta có $\sin ^4x+3\cos ^4x$ $=\sin ^4\alpha+3\left(1-\sin^2\alpha\right)^2$ $=\dfrac{1}{4}+3\left(1-\dfrac{1}{2}\right)^2=1$.}
\end{ex}
\begin{ex}%[Câu 22]%[1K1B3-2]
	Hàm số $y=\sin x$ đồng biến trên khoảng
	\choice
	{$(0 ; \pi)$}
	{$\left(-\dfrac{3 \pi}{2} ;-\dfrac{\pi}{2}\right)$}
	{\True $\left(-\dfrac{ \pi}{2} ;\dfrac{\pi}{2}\right)$}
	{$(-\pi ; 0)$}
	\loigiai{ 
		Do hàm số $y=\sin x$ đồng biến trên mỗi khoảng $\left( -\dfrac{\pi}{2}+k2 \pi;\dfrac{\pi}{2}+k2 \pi \right) $ nên ứng với $k=0$, ta có hàm số $y=\sin x$ đồng biến trên khoảng $\left(-\dfrac{ \pi}{2} ;\dfrac{\pi}{2}\right)$.
	}
\end{ex}

\begin{ex}%[Câu 23]%[1K1B3-2]
	Hàm số nghịch biến trên khoảng $(\pi ; 2 \pi)$ là
	\choice
	{$y=\sin x$}
	{$y=\cos x$}
	{$y=\tan x$}
	{\True $y=\cot x$}
	\loigiai{
		Do hàm số $y=\cot x$ nghịch biến trên mỗi khoảng $\left( k \pi; \pi +k \pi \right) $ nên  ứng với $k=1$, ta có hàm số $y=\cot x$ nghịch biến trên khoảng $(\pi ; 2 \pi)$.
	}
\end{ex}
\begin{ex}%[Câu 24]%[1K1B3-1]
	Tập xác định của hàm số $y=\dfrac{\cos x}{\sin x-1}$ là
	\choice
	{$\mathbb{R}\setminus \left\{k2\pi| k\in\mathbb{Z}\right\}$}
	{\True $\mathbb{R}\setminus \left\{\dfrac{\pi}{2}+k2\pi| k\in\mathbb{Z}\right\}$}
	{$\mathbb{R}\setminus \left\{\dfrac{\pi}{2}+k\pi| k\in\mathbb{Z}\right\}$}
	{$\mathbb{R}\setminus \left\{k\pi| k\in\mathbb{Z}\right\}$}
	\loigiai{
		Hàm số xác định khi và chỉ khi $\sin x-1\ne 0\Leftrightarrow\sin x\ne 1\Leftrightarrow x\ne \dfrac{\pi}{2}+k2\pi$ với $k\in \mathbb{Z}$.\\
		Vậy tập xác định của hàm số là $\mathbb{R}\setminus \left\{\dfrac{\pi}{2}+k2\pi| k\in\mathbb{Z}\right\}$.
	}
\end{ex}
\begin{ex}%[Câu 25]%[1K1Y3-1]
	Khẳng định nào sau đây là \textbf{sai}?
	\choice
	{Hàm số $y=\cos x$ có tập xác định là $\mathbb{R}$}
	{Hàm số $y=\cos x$ có tập giá trị là $[-1;1]$}
	{\True Hàm số $y=\cos x$ là hàm số lẻ}
	{Hàm số $y=\cos x$ tuần hoàn với chu kì $2\pi$}
	\loigiai{
		Hàm số $y=\cos x$ là hàm số chẵn.
	}
\end{ex}
\begin{ex}%[Câu 26]%[1K1Y3-4]
	Trong các hàm số sau đây, hàm số nào là hàm tuần hoàn?
	\choice
	{$y=\tan x+x$}
	{$y=x^2+1$}
	{\True $y=\cot x$}
	{$y=\dfrac{\sin x}{x}$}
	\loigiai{
		Hàm số $y=\cot x$ là hàm số tuần hoàn với chu kỳ $T=\pi$.
	}
\end{ex}
\begin{ex}%[Câu 27]%[1K1Y3-3]
	Khẳng định nào sau đây đúng?
	\choice
	{Hàm số $y=\sin x$ là hàm số chẵn}
	{\True Hàm số $y=\cos x$ là hàm số chẵn}
	{Hàm số $y=\tan x$ là hàm số chẵn}
	{Hàm số $y=\cot x$ là hàm số chẵn}
	\loigiai{
		Hàm số $y=\cos x$ là hàm số chẵn;  các hàm số còn lại là hàm số lẻ.
	}
\end{ex}
\begin{ex}%[Câu 28]%[1K1B3-3]
	Khẳng định nào sau đây là đúng?
	\choice
	{Hàm số $y=\cos x$ là hàm số lẻ}
	{\True Hàm số $y=\tan 2x- \sin x$ là hàm số lẻ}
	{Hàm số $y=\sin x$ là hàm số chẵn}
	{Hàm số $y=\tan x \cdot \sin x$ là hàm số lẻ}
	\loigiai{
		Xét hàm số $y=f\left( {x} \right)=\tan 2x-\sin x$.
		\\Hàm số xác định khi $\cos 2x \ne 0 \Leftrightarrow x \ne \dfrac{\pi}{4}+k\dfrac{\pi}{2}$, $\left( {k \in \mathbb{Z}} \right)$.
		\\Tập xác định $\mathscr{D}=\mathbb{R} \setminus \left\{ {\dfrac{\pi}{4}+k\dfrac{\pi}{2},k \in \mathbb{Z}} \right\}$.
		\\ Với mọi $x \in \mathscr{D}$ thì $-x \in \mathscr{D}$ và $f\left( {-x} \right)=\tan \left( {-2x} \right)- \sin \left( {-x} \right)=-\tan 2x + \sin x=-f\left( {x} \right)$.\\ Do đó hàm số $y=\tan 2x-\sin x$ là hàm số lẻ.}
\end{ex}
\begin{ex}%[Câu 29]%[1K1B3-1]
	Tập xác định của hàm số $y=\dfrac{\cot x}{\cos x-1}$ là
	\choice
	{$\mathbb{R}\setminus\left\{\dfrac{k\pi}{2}, k\in\mathbb{Z}\right\}$}
	{$\mathbb{R}\setminus\left\{\dfrac{k}{2}+k\pi, k\in\mathbb{Z}\right\}$}
	{\True $\mathbb{R}\setminus\left\{k\pi, k\in\mathbb{Z}\right\}$}
	{$\mathbb{R}\setminus\left\{k2\pi, k\in\mathbb{Z}\right\}$}
	\loigiai{
		Hàm số xác định khi và chỉ khi $\heva{&\sin x\ne 0\\ &\cos x\ne 1}\Leftrightarrow\heva{&x\ne k\pi\\ &x\ne l2\pi}\ (k,l\in\mathbb{Z})\Leftrightarrow x\ne k\pi, k\in\mathbb{Z}$.\\
		Vậy, tập xác định của hàm số $y=\dfrac{\cot x}{\cos x-1}$ là $\mathbb{R}\setminus\left\{k\pi, k\in\mathbb{Z}\right\}$.
	}
\end{ex}
\begin{ex}%[Câu 30]%[1K1K3-2]
	Cho đồ thị hàm số $y=\sin x$ như hình vẽ sau
	\begin{center}
		
		\begin{tikzpicture}[>=stealth,scale=0.7]
			\draw [->] (-11,0)--(0,0)
			node[below right]{$O$}--(11,0)node[below]{$x$}; % Hệ trục tọa độ
			\draw[->] (0,-1.5) --(0,2) node[left]{$y$};
			\draw[dashed] (-5*pi/2,0)node[above]{$-\tfrac{5\pi}{2}$}--(-5*pi/2,-1)--(3*pi/2,-1)--(3*pi/2,0)node[above]{$\tfrac{3\pi}{2}$};
			\draw[dashed] (-3*pi/2,0)node[below]{$-\tfrac{3\pi}{2}$} --(-3*pi/2,1)--(5*pi/2,1)--(5*pi/2,0)node[below]{$\tfrac{5\pi}{2}$};
			\draw[dashed] (-pi/2,0)node[above]{$-\tfrac{\pi}{2}$}--(-pi/2,-1);
			\draw[dashed] (pi/2,0)node[below]{$\tfrac{\pi}{2}$}--(pi/2,1);
			\draw(-2.9*pi,0) node[below left]{$-3\pi$}(-1.9*pi,0) node[below]{$-2\pi$}(-1.1*pi,0) node[below]{$-\pi$}(3*pi,0) node[below]{$3\pi$}(2*pi,0) node[below]{$2\pi$}(pi,0) node[above]{$\pi$}(0,-1.6)node[below]{$2\pi$}(0,1)node[above right]{$1$};
			\draw[dashed] (-pi,0)--(-pi,-1.6)(pi,0)--(pi,-1.6);
			\draw[<->](-pi,-1.6)--(pi,-1.6);
			\draw [domain=-3.5*pi:3.4*pi,samples=100] plot (\x, {sin(\x r)});
		\end{tikzpicture}
	\end{center}
	Mệnh đề nào dưới đây \textbf{sai}?
	\choice
	{Hàm số $y=\sin x$ tăng trên khoảng $\left(-\dfrac{\pi}{2};\dfrac{\pi}{2}\right)$}
	{Hàm số $y=\sin x$ giảm trên khoảng $\left(\dfrac{\pi}{2};\dfrac{3\pi}{2}\right)$}
	{Hàm số $y=\sin x$ giảm trên khoảng $\left(-\dfrac{3\pi}{2};-\pi \right)$}
	{\True Hàm số $y=\sin x$ tăng trên khoảng $\left(0;\pi \right)$}
	\loigiai{
		\begin{itemize}
			\item Hàm số $y=\sin x$ tăng trên $\left(0;\dfrac{\pi}{2}\right)$ và giảm trên $\left(\dfrac{\pi}{2};\pi \right)$.
			\item Vậy trên khoảng $\left(0;\pi \right)$, hàm số $y=\sin x$ vừa tăng vừa giảm nên khẳng định hàm số $y=\sin x$ tăng trên khoảng $\left(0;\pi \right)$ là khẳng định \textbf{sai}.
	\end{itemize}}
\end{ex}
\begin{ex}%[Câu 31]%[1K1B3-2]
	Chọn khẳng định đúng trong các khẳng định sau
	\choice
	{Hàm số $y = \tan x$ tuần hoàn với chu kì $2\pi$}
	{Hàm số $y = \cos x$ tuần hoàn với chu kì $\pi$}
	{\True Hàm số $y = \sin x$ đồng biến trên khoảng $\left(0; \dfrac{\pi}{2}\right)$}
	{Hàm số $y = \cot x$ nghịch biến trên $\mathbb{R}$}
	\loigiai{Ta xét $y = \sin x$ suy ra  $y'  = \cos x$. Dễ thấy $\cos x > 0\ ,\  \forall x\in \left(0; \dfrac{\pi}{2}\right)$. Do đó hàm số $y = \sin x$ đồng biến trên khoảng $\left(0; \dfrac{\pi}{2}\right)$.
	}
\end{ex}
\begin{ex}%[Câu 32]%[1K1K3-6]
	Đồ thị của hàm số $y=\sin x$ và $y=\cos x$ cắt nhau tại bao nhiêu điểm có hoành độ thuộc đoạn $\left[-2\pi;\dfrac{5\pi}{2}\right]$?
	\choice
	{\True $5$}
	{$6$}
	{$4$}
	{$7$}
	\loigiai{
		Xét phương trình hoành độ giao điểm của hai đồ thị hàm số $\sin x=\cos x$.\\
		Nếu $\cos x=0$ thì $\sin x=0$ nên vô lý.\\
		Do đó, $\cos x\ne 0$. Ta có
		\allowdisplaybreaks
		\begin{eqnarray*}
			\sin x=\cos x&\Leftrightarrow&\tan x=1\\
			&\Leftrightarrow&x=\dfrac{\pi}{4}+k\pi ,\quad \left(k\in\mathbb{Z}\right).
		\end{eqnarray*}
		Ta lại có
		\allowdisplaybreaks
		\begin{eqnarray*}
			-2\pi \le x\le \dfrac{5\pi}{2}&\Leftrightarrow& -2\pi \le \dfrac{\pi}{4}+k\pi\le \dfrac{5\pi}{2}\\
			&\Leftrightarrow& -2 \le \dfrac{1}{4}+k\le \dfrac{5}{2}\\
			&\Leftrightarrow& \dfrac{-9}{4} \le k\le \dfrac{9}{4}.
		\end{eqnarray*}
		Do $k\in\mathbb{Z}$ nên $k\in\left\{-2;-1;0;1;2\right\}$.\\
		Vậy hai đồ thị hàm số cắt nhau tại $5$ điểm có hoành độ thuộc đoạn $\left[-2\pi;\dfrac{5\pi}{2}\right]$.
	}
\end{ex}
\begin{ex}%[Câu 33]%[1K1B3-5]
	Tìm tập giá trị của hàm số $y=2\cos3x +1$.
	\choice
	{$[-3;1]$}
	{$[-3;-1]$}
	{\True $[-1;3]$}
	{$[1;3]$}
	\loigiai{
		$\forall x\in \mathbb{R}$ ta có
		\begin{eqnarray*}
			&& -1\leq\cos3x\leq1 \\
			&\Leftrightarrow& -2\leq2\cos3x\leq2 \\
			&\Leftrightarrow& -1\leq2\cos3x+1\leq3.
		\end{eqnarray*}
	}
\end{ex}
\begin{ex}%[Câu 34]%[1K1B3-6]
	Đường cong trong hình bên là đồ thị trên đoạn $\left[-\pi ;\pi\right]$ của một hàm số trong bốn hàm số được liệt kê ở bốn phương án $\textbf{A, B, C, D}$ dưới đây. Hỏi đó là hàm số nào?
	\begin{center}
		\definecolor{x}{rgb}{0.75,0.75,0.75}
		\begin{tikzpicture}[scale=1, line join=round, line cap=round,>=stealth]
			\draw[->] (-4,0.) -- (4,0.)node [above] { $x$};
			\draw[shift={(-3.14,0)}] node[below left] {\footnotesize $-\pi$};
			\draw[shift={(-1.57,0)}] node[above left] {\footnotesize $-\dfrac{\pi}{2}$};
			\draw[shift={(1.57,0)}] node[below left] {\footnotesize $\dfrac{\pi}{2}$};
			\draw[shift={(3.14,0)}] node[below left] {\footnotesize $\pi$};
			\draw[->] (0.,-1.3) -- (0.,1.5)node [right] { $y$};
			\draw (0,1) node[above left] {\footnotesize $1$};
			\draw (0,-1) node[above left] {\footnotesize $-1$};
			\draw (0pt,-10pt) node[right] {\footnotesize $O$};
			\clip(-4.2,-1.3) rectangle (4.2,1.5);
			\draw[line width=1.2pt,smooth,samples=100,domain=-3.14:3.14] plot(\x,{sin(((\x))*180/pi)});
			\draw [dashed] (-1.57,0)--(-1.57,-1)--(0,-1)(1.57,0)--(1.57,1)--(0,1);
		\end{tikzpicture}
	\end{center}
	\choice
	{\True $y=\sin x$}
	{$y=\cos x$}
	{$y=\tan x$}
	{$y=\cot x$}
	\loigiai{
		Đồ thị hàm số đi qua các điểm $(0;0),(\pi;0), \left(\dfrac{\pi}{2};1\right)$ và nhận $O$ làm tâm đối xứng.
	}
\end{ex}
\begin{ex}%[Câu 35]%[1K1Y4-3]
	Phương trình $\cot x=-1$ có nghiệm là
	\choice
	{\True $-\dfrac{\pi}{4}+k \pi(k \in \mathbb{Z})$}
	{$\dfrac{\pi}{4}+k \pi(k \in \mathbb{Z})$}
	{$\dfrac{\pi}{4}+k 2 \pi(k \in \mathbb{Z})$}
	{$-\dfrac{\pi}{4}+k 2 \pi(k \in \mathbb{Z})$}
	\loigiai{
		Ta có $\cot x=-1 \Leftrightarrow \cot x=\cot \left(-\dfrac{\pi}{4}\right) \Leftrightarrow x =-\dfrac{\pi}{4}+k \pi(k \in \mathbb{Z}) $.
	}
\end{ex}
\begin{ex}%[Câu 36]%[1K1Y4-3]
	Trong các phép biến đổi sau, phép biến đổi nào \textbf{sai}?
	\choice
	{$\sin x=1\Leftrightarrow x=\dfrac{\pi}{2}+k2\pi,(k\in \mathbb{Z})$}
	{$\tan x=1\Leftrightarrow x=\dfrac{\pi}{4}+k\pi,(k\in \mathbb{Z})$}
	{$\cos x=\dfrac{1}{2}\Leftrightarrow \hoac{
			& x=\dfrac{\pi}{3}+k2\pi,(k\in \mathbb{Z}) \\
			& x=-\dfrac{\pi}{3}+k2\pi,(k\in \mathbb{Z})}$}
	{\True $\sin x=0\Leftrightarrow x=k2\pi,(k\in \mathbb{Z})$}
	\loigiai{
		Ta có $\sin x=0\Leftrightarrow x=k\pi,(k\in \mathbb{Z})$, nên đáp án $\sin x=0\Leftrightarrow x=k2\pi,(k\in \mathbb{Z})$ sai.}
\end{ex}
\begin{ex}%[Câu 37]%[1K1B4-5]
	Nghiệm của phương trình $\sin x\cdot \cos x=\dfrac{1}{2}$ là
	\choice
	{$x=k2\pi$; $k\in \mathbb{Z}$}
	{$x=\dfrac{k\pi}{4}$; $k\in \mathbb{Z}$}
	{\True $x=\dfrac{\pi}{4}+k\pi$; $k\in \mathbb{Z}$}
	{$x=k\pi$; $k\in \mathbb{Z}$}
	\loigiai{
		Ta có $\sin x\cdot \cos x=\dfrac{1}{2}\Leftrightarrow \sin 2x=1\Leftrightarrow 2x=\dfrac{\pi}{2}+k2\pi\Leftrightarrow x=\dfrac{\pi}{4}+k\pi$ với $k\in\mathbb{Z}$.
	}
\end{ex}
\begin{ex}%[Câu 38]%[1K1Y4-3]
	Họ nghiệm của phương trình $\sin2x=1$ là
	\choice
	{$x=\dfrac{\pi}{2}+k\pi,\,k\in\mathbb{Z}$}
	{$x=\dfrac{\pi}{2}+k2\pi,\,k\in\mathbb{Z}$}
	{\True $x=\dfrac{\pi}{4}+k\pi,\,k\in\mathbb{Z}$}
	{$x=\dfrac{\pi}{4}+\dfrac{k\pi}{2},\,k\in\mathbb{Z}$}
	\loigiai{
		Ta có $\sin2x=1\Leftrightarrow 2x=\dfrac{\pi}{2}+k2\pi\Leftrightarrow x=\dfrac{\pi}{4}+k\pi,\, k\in\mathbb{Z}$.
	}
\end{ex}
\begin{ex}%[Câu 39]%[1K1B4-5]
	Phương trình $\sin 2x \cos x = \sin 7x \cos 4x$ có các họ nghiệm là
	\choice
	{$x=\dfrac{k2\pi}{5};x=\dfrac{\pi}{12}+\dfrac{k\pi}{6} (k \in \Bbb{Z})$}
	{$x=\dfrac{k\pi}{5};x=\dfrac{\pi}{12}+\dfrac{k\pi}{3} (k \in \Bbb{Z})$}
	{\True $x=\dfrac{k\pi}{5};x=\dfrac{\pi}{12}+\dfrac{k\pi}{6} (k \in \Bbb{Z})$}
	{$x=\dfrac{k2\pi}{5};x=\dfrac{\pi}{12}+\dfrac{k\pi}{3} (k \in \Bbb{Z})$}
	\loigiai{
		Ta có \begin{eqnarray*}
			\sin 2x \cos x = \sin 7x \cos 4x &\Leftrightarrow & \dfrac{1}{2}(\sin 3x+\sin x)=\dfrac{1}{2}(\sin 11x+\sin 3x)\\
			&\Leftrightarrow & \sin 11x = \sin x\\
			&\Leftrightarrow & \hoac{&x=\dfrac{k\pi}{5}\\&x=\dfrac{\pi}{12}+\dfrac{k\pi}{3} }(k \in \Bbb{Z}).
		\end{eqnarray*}
	}
\end{ex}
\begin{ex}%[Câu 40]%[1K1K4-3]
	Số nghiệm của phương trình $\cos x=0$ trên đoạn $[0 ; 10 \pi]$ là
	\choice
	{$5$}
	{$9$}
	{\True $10$}
	{$11$}
	\loigiai{
		Ta có $\cos x=0 \Leftrightarrow x =\dfrac{\pi}{2}+ k \pi (k \in \mathbb{Z})$.\\
		Do $0 \leq x \leq 10 \pi \Leftrightarrow 0 \leq \dfrac{\pi}{2}+ k \pi \leq 10 \Leftrightarrow -\dfrac{1}{2} \leq k \leq \dfrac{19}{2}\Leftrightarrow 0\leq k \leq 9( k \in  \mathbb{Z} )$.\\
		Do đó phương trình $\cos x=0$ có $10$ nghiệm.
	}
\end{ex}

\begin{ex}%[Câu 41]%[1K1B4-3]
	Số nghiệm của phương trình $\sin x=0$ trên đoạn $[0 ; 10 \pi]$ là
	\choice
	{$10$}
	{$6$}
	{$5$}
	{\True $11$}
	\loigiai{
		Ta có $\sin x=0 \Leftrightarrow x = k \pi (k \in \mathbb{Z})$.\\
		Do $0 \leq x \leq 10 \pi \Leftrightarrow 0 \leq k\leq 10$.\\
		Do đó phương trình $\sin x=0$ có $11$ nghiệm.
	}
\end{ex}

\begin{ex}%[Câu 42]%[1K1B4-3]
	Số nghiệm của phương trình $\sin \left(x+\dfrac{\pi}{4}\right)=\dfrac{\sqrt{2}}{2}$ trên đoạn $[0; \pi]$ là
	\choice
	{$4$}
	{$1$}
	{\True $2$}
	{$3$}
	\loigiai{
		Ta có $\sin \left(x+\dfrac{\pi}{4}\right)=\dfrac{\sqrt{2}}{2} \Leftrightarrow \sin \left(x+\dfrac{\pi}{4}\right)=\sin \left( \dfrac{\pi}{4}\right) \Leftrightarrow  \hoac{&x= k 2 \pi\\&x=\dfrac{\pi}{2}+ k 2 \pi} (k \in \mathbb{Z})$.\\
		Do $x \in [0 ; \pi]$ nên $x=0$ hoặc $x=\dfrac{\pi}{2}$.
	}
\end{ex}
\begin{ex}%[Câu 43]%[1K1B4-5]
	Phương trình $ \sin{2x}+3\cos x=0 $ có bao nhiêu nghiệm trong khoảng $ (0;\pi)$?
	\choice
	{$ 0 $}
	{\True $ 1 $}
	{$ 2 $}
	{$ 3 $}
	\loigiai
	{
		Ta có $ \sin{2x}+3\cos x=0 \Leftrightarrow \hoac{& \cos x=0 \\ &\sin x=-\dfrac{3}{2}}\Leftrightarrow \cos x=0 \Leftrightarrow x= \dfrac{\pi}{2}+k\pi$. Do $ x \in (0;\pi) $ nên có một nghiệm là $ x=\dfrac{\pi}{2}$.
	}
\end{ex}

\begin{ex}%[Câu 44]%[1K1B1-9]
	Một bánh xe có $72$ răng. Số đo góc mà bánh xe đã quay được khi di chuyển $10$ răng là
	\choice
	{$40^\circ	$}
	{\True $50^\circ$}
	{$60^\circ$}
	{$30^\circ$}
	\loigiai{
		1 bánh răng tương ứng với $\dfrac{360^\circ}{72}=5^\circ$$\Rightarrow 10$ bánh răng là $50^\circ$.}
\end{ex}

\begin{ex}%[Câu 45]%[1K1K1-9]
	\immini{Người ta muốn làm một cánh diều hình quạt có bán kính là $a$, độ dài cung tròn là $b$ và có chu vi là $80$ cm (như hình vẽ). Khi diện tích cánh diều đạt giá trị lớn nhất, tổng $a+b$ bằng
		\choice
		{$50$ cm}
		{$40$ cm}
		{$70$ cm}
		{\True $60$ cm}}{
		\begin{tikzpicture}
			\draw (0,3) arc (150:210:3);
			\coordinate [label=below:$A$] (A) at (0,0);
			\coordinate [label=right:$O$] (O) at (30:3);
			\coordinate [label=above:$C$] (C) at (0,3);
			\foreach \point in {O,A,C} \fill[black] (\point) circle (1pt);
			\draw (O)--(C) (O)--(A);
		\end{tikzpicture}}
	\loigiai{
		Gọi $\varphi$ (rad) là số đo cung của hình quạt. Khi đó $\varphi =\dfrac{b}{a}$.\\
		Chu vi cánh diều bằng $b+2a=80$.\\
		Diện tích cánh diều bằng $S=\dfrac{\varphi a^2}{2}=\dfrac{ab}{2}=\dfrac{1}{4}(b \cdot 2a) \le \dfrac{1}{4} \cdot \left(\dfrac{b+2a}{2}\right)^2=400$.\\
		Dấu bằng xảy ra khi và chỉ khi $\heva{&b=2a \\& b+2a=80}\Leftrightarrow \heva{&b=40 \\& a=20.}$\\
		Do vậy $a+b=60$ cm.}
\end{ex}
\begin{ex}%[Câu 46]%[1K1K1-9]
	\immini{
		Khi một tia sáng truyền từ không khí vào mặt nước thì một phần tia sáng bị phản xạ trên bề mặt, phần còn lại bị khúc xạ như hình bên. Góc tới $i$ liên hệ với góc khúc xạ $r$ bởi Định luật khúc xạ ánh sáng
		$$\dfrac{\sin i}{\sin r}=\dfrac{n_2}{n_1}.$$
		Ở đây, $n_1$ và $n_2$ tương ứng với chiết suất của môi trường $1$ (không khí) và môi trường $2$ (nước). Cho biết góc tới $i=50^\circ$ và  chiết suất của không khí bằng $1$ còn chiết suất của nước là $1{,}33$. Khi đó  góc khúc xạ gần với kết quả nào sau đây.
		\choice
		{\True$35{,}17^\circ$}
		{$55{,}47^\circ$}
		{$31{,}42^\circ$}
		{$12{,}35^\circ$}
	}
	{
		\begin{tikzpicture}[>=stealth,line join=round,line cap=round,font=\footnotesize,scale=.7]
			\path
			(0,0)coordinate(I)++(90:3)coordinate(N)++(-90:6)coordinate(N')
			(I)++(0:3)coordinate(B)++(180:6)coordinate(A)
			(I)++(30:3)coordinate(S')
			(I)++(150:3)coordinate(S)
			(I)++(-55:3)coordinate(R)
			;
			\fill[cyan!20!](-3,-3)rectangle(3,0)
			;
			\draw (A)--(B)
			;
			\draw[dashed](N)--(N')
			;
			\draw[->,midway](S)--(I)
			;
			\draw[->](I)--(S')
			;
			\draw[->](I)--(R)
			;
			\foreach \p/\r in {N/180,N'/180,S/160,S'/90,R/0,I/-135}
			\fill (\p) node[shift={(\r:3mm)}]{$\p$}
			;
			\draw pic[angle radius=3mm,draw=red,fill=green!50,angle eccentricity=1.5] {angle = N--I--S}
			;
			\draw pic[angle radius=4mm,draw=orange,fill=orange!50,angle eccentricity=1.5] {angle = S'--I--N}
			;
			\draw pic[angle radius=4mm,draw=blue,fill=blue!50,angle eccentricity=1.5] {angle = N'--I--R}
			;
			\draw (-2.5,.5)circle(7pt)node{$1$}
			(-2.5,-.5)circle(8pt)node{$2$}
			;	
		\end{tikzpicture}
	}
	\loigiai{
		Ta có $\dfrac{\sin i}{\sin r}=\dfrac{n_2}{n_1}\Leftrightarrow \dfrac{\sin 50^\circ}{\sin r}=\dfrac{1{,}33}{1}\Leftrightarrow \sin r=\dfrac{\sin 50^\circ}{1{,}33}\Rightarrow r\approx 35{,}17^\circ$.
	}
\end{ex}
\begin{ex}%[Câu 47]%[1K1G2-4]
	Giả sử $a, b, c$ lần lượt là ba cạnh đối diện với ba góc $A, B, C$ của tam giác $ABC$ thỏa điều kiện $2\cos\dfrac{B}{2}\cos\dfrac{C}{2}=\dfrac{1}{2}+\dfrac{b+c}{a}\sin\dfrac{A}{2}$. Tính góc $A$ của tam giác $ABC$.
	\choice
	{$30^\circ$}
	{$45^\circ$}
	{\True $60^\circ$}
	{$90^\circ$}
	\loigiai
	{\noindent Đặt $2\cos\dfrac{B}{2}\cos\dfrac{C}{2}=\dfrac{1}{2}+\dfrac{b+c}{a}\sin\dfrac{A}{2}\;(\star)$. Ta có
		\begin{align*}
			(\star)&\Leftrightarrow  2\cos\dfrac{B}{2}\cos\dfrac{C}{2}=\dfrac{1}{2}+\dfrac{\sin B+\sin C}{\sin A}\sin\dfrac{A}{2}\\
			&\Leftrightarrow  \cos\dfrac{B+C}{2}+\cos\dfrac{B-C}{2}=\dfrac{1}{2}+\dfrac{2\sin\dfrac{B+C}{2}\cos\dfrac{B-C}{2}}{2\sin\dfrac{A}{2}\cos\dfrac{A}{2}}\sin\dfrac{A}{2}\\
			&\Leftrightarrow  \sin\dfrac{A}{2}+\cos\dfrac{B-C}{2}=\dfrac{1}{2}+\cos\dfrac{B-C}{2}\;\left(\text{vì}\;\sin\dfrac{A}{2}>0, \cos\dfrac{A}{2}=\sin\dfrac{B+C}{2}\right)\\
			&\Leftrightarrow  \sin\dfrac{A}{2}=\dfrac{1}{2}\Leftrightarrow  A=\dfrac{\pi}{3}.
		\end{align*}
	}
\end{ex}
\begin{ex}%[Câu 48]%[1K1G4-5]
	Phương trình $2\sqrt{3}\sin\left(x-\dfrac{\pi}{8}\right)\cos\left(x-\dfrac{\pi}{8}\right)+2\cos^2\left(x-\dfrac{\pi}{8}\right) = \sqrt{3}+1$ có nghiệm là
	\choice
	{\True $x=\dfrac{5\pi}{24}+k\pi$, $x=\dfrac{3\pi}{8}+k\pi$ với $k\in\mathbb{Z}$}
	{$x=\dfrac{5\pi}{12}+k\pi$, $x=\dfrac{3\pi}{4}+k\pi$ với $k\in\mathbb{Z}$}
	{$x=\dfrac{5\pi}{4}+k\pi$, $x=\dfrac{5\pi}{16}+k\pi$ với $k\in\mathbb{Z}$}
	{$x=\dfrac{5\pi}{8}+k\pi$, $x=\dfrac{7\pi}{24}+k\pi$ với $k\in\mathbb{Z}$}
	\loigiai
	{
		Ta có
		\allowdisplaybreaks
		\begin{eqnarray*}
			&& 2\sqrt{3}\sin\left(x-\dfrac{\pi}{8}\right)\cos\left(x-\dfrac{\pi}{8}\right)+2\cos^2\left(x-\dfrac{\pi}{8}\right) = \sqrt{3}+1\\
			&\Leftrightarrow & \sqrt{3}\sin\left(2x-\dfrac{\pi}{4}\right)+1+\cos\left(2x-\dfrac{\pi}{4}\right) = \sqrt{3}+1\\
			&\Leftrightarrow & \sqrt{3}\sin\left(2x-\dfrac{\pi}{4}\right)+\cos\left(2x-\dfrac{\pi}{4}\right) = \sqrt{3}\\
			&\Leftrightarrow & \dfrac{\sqrt{3}}{2}\sin\left(2x-\dfrac{\pi}{4}\right)+\dfrac{1}{2}\cos\left(2x-\dfrac{\pi}{4}\right) = \dfrac{\sqrt{3}}{2}\\
			&\Leftrightarrow & \sin\left(2x-\dfrac{\pi}{12}\right) = \dfrac{\sqrt{3}}{2}\\
			&\Leftrightarrow & \left[\begin{aligned}&2x-\dfrac{\pi}{12}=\dfrac{\pi}{3}+k2\pi,k\in\mathbb{Z} \\&2x-\dfrac{\pi}{12}=\dfrac{2\pi}{3}+k2\pi,k\in\mathbb{Z}\end{aligned}\right.\\
			&\Leftrightarrow & \left[\begin{aligned}&x=\dfrac{5\pi}{24}+k\pi,k\in\mathbb{Z} \\&x=\dfrac{3\pi}{8}+k\pi,k\in\mathbb{Z}.\end{aligned}\right.
		\end{eqnarray*}
		Vậy phương trình đã cho có nghiệm $x=\dfrac{5\pi}{24}+k\pi$, $x=\dfrac{3\pi}{8}+k\pi$ với $k\in\mathbb{Z}$.
	}
\end{ex}
\begin{ex}%[Câu 49]%[1K1G4-5]
	Nghiệm dương nhỏ nhất của phương trình $\sin x+\sin 2x=\cos x+2\cos^2 x$ là
	\choice{$\dfrac{\pi}{6}$}{$\dfrac{\pi}{3}$}{$2\dfrac{\pi}{3}$}{\True$\dfrac{\pi}{4}$}
	\loigiai{\begin{eqnarray*}
			& &\sin x+\sin 2x=\cos x+2\cos^2 x\\
			&\Leftrightarrow & \sin x + 2\sin x\cos x = \cos x\left(2\cos x + 1\right)\\
			&\Leftrightarrow & \sin x\left(2\cos x + 1\right) = \cos x\left(2\cos x + 1\right)\\
			&\Leftrightarrow & \hoac{&\cos x = - \dfrac{1}{2}\\&\sin x = \cos x}\\
			&\Leftrightarrow & \hoac{&x = \pm \dfrac{2\pi}{3} + k2\pi\\&x = \dfrac{\pi}{4} + k\pi} \quad \left(k \in \mathbb{Z}\right).
		\end{eqnarray*}
		Khi đó nghiệm dương nhỏ nhất của phương trình là $x = \dfrac{\pi}{4}$.}
\end{ex}
\begin{ex}%[Câu 50]%[1K1G4-5]
	Số nghiệm của phương trình $ \dfrac{2\sin x-1}{2\sin^2x+\sin x-1}=2 $ trong khoảng $ \left(\dfrac{\pi}{2}; \dfrac{7\pi}{2}\right) $ là
	\choice
	{$ 5 $}
	{$ 2 $}
	{$ 4 $}
	{\True $ 3 $}
	\loigiai{
		Điều kiện $ 2\sin^2 x+\sin x-1\neq 0\Leftrightarrow\heva{& \sin x\neq -1\\ & \sin x\neq \dfrac{1}{2}} $.\\
		Khi đó phương trình đã cho tương đương với \begin{eqnarray*}
			& &2\sin x-1=4\sin^2 x+2\sin x-2\\
			& \Leftrightarrow & 4\sin^2 x=1\\
			&\Leftrightarrow &\hoac{& \sin x=\dfrac{1}{2}\ (\text{không thỏa mãn điều kiện})\\
				&\sin x=-\dfrac{1}{2}\ (\text{thỏa mãn điều kiện})}\\
			&\Leftrightarrow & \hoac{& x=-\dfrac{\pi}{6}+k2\pi\\ & x=\dfrac{7\pi}{6}+k2\pi},\ k\in\mathbb{Z}.
		\end{eqnarray*}
		\begin{itemize}
			\item Trường hợp $ x=-\dfrac{\pi}{6}+k2\pi $. Khi đó, $\begin{aligned}[t]
				x\in \left(\dfrac{\pi}{2}; \dfrac{7\pi}{2}\right)&\Leftrightarrow \dfrac{\pi}{2}< -\dfrac{\pi}{6}+k2\pi<\dfrac{7\pi}{2}\\
				&\Leftrightarrow \dfrac{2\pi}{3}<k2\pi<\dfrac{11\pi}{3}\\
				&\Leftrightarrow \dfrac{1}{3}<k<\dfrac{11}{6}\\
				&\Leftrightarrow k=1 \ (\text{vì}\; k\in\mathbb{Z}).
			\end{aligned} $
			\item Trường hợp $ x=\dfrac{7\pi}{6}+k2\pi $. Khi đó, $\begin{aligned}[t]
				x\in \left(\dfrac{\pi}{2}; \dfrac{7\pi}{2}\right)&\Leftrightarrow \dfrac{\pi}{2}< \dfrac{7\pi}{6}+k2\pi<\dfrac{7\pi}{2}\\
				&\Leftrightarrow -\dfrac{\pi}{3}<k2\pi<\dfrac{7\pi}{3}\\
				&\Leftrightarrow -\dfrac{1}{6}<k<\dfrac{7}{6}\\
				&\Leftrightarrow k\in\{0; 1\} \ (\text{vì}\; k\in\mathbb{Z}).
			\end{aligned} $
		\end{itemize}
		Vậy phương trình đã cho có tất cả 3 nghiệm thuộc khoảng $ \left(\dfrac{\pi}{2}; \dfrac{7\pi}{2}\right) $.
	}
\end{ex}
\Closesolutionfile{ans}
% \setcounter{deso}{0}
\begin{name}
	{\tenchude}
	{ĐỀ ÔN TẬP CHƯƠNG I}
	{LỚP TOÁN THẦY PHÁT}
	{\thoigian}
\end{name}
\TN
%Câu 1
\begin{ex}
	Cho góc lượng giác $\alpha $. Mệnh đề nào sau đây đúng?
	\choice
	{$\sin \left(-\alpha\right)=\sin \alpha $}
	{$\cos \left(-\alpha\right)=-\cos \alpha $}
	{$\tan \left(-\alpha\right)=\tan \alpha $}
	{\True $\cot \left(-\alpha\right)=-\cot \alpha $}
	\loigiai{
		Dựa vào tính chất của hai góc đối nhau nên $\cot \left(-\alpha\right)=-\cot \alpha $
	}
\end{ex}
%Câu 2
\begin{ex}
	Giá trị $\cos 75^\circ $ là :
	\choice
	{$\dfrac{\sqrt{6}+\sqrt{2}}{4}$}
	{$\dfrac{\sqrt{6}-\sqrt{2}}{2}$}
	{\True $\dfrac{\sqrt{6}-\sqrt{2}}{4}$}
	{$\dfrac{\sqrt{6}+\sqrt{2}}{2}$}
	\loigiai{
		Ta có $\cos 75^\circ =\cos \left(30^\circ+45^\circ \right)=\cos 30^\circ \cos 45^\circ -\sin 30^\circ \sin 45^\circ=\dfrac{\sqrt{6}-\sqrt{2}}{4}$
	}
\end{ex}
%Câu 3
\begin{ex}
	Cho $\sin \alpha =\dfrac{5}{13}$ với $\dfrac{\pi }{2}<\alpha <\pi $. Mệnh đề nào sau đây đúng?
	\choice
	{$\cos \alpha =\dfrac{12}{13}$}
	{$\cos \alpha =\dfrac{8}{13}$}
	{$\cos \alpha =-\dfrac{8}{13}$}
	{\True $\cos \alpha =-\dfrac{12}{13}$}
	\loigiai{
	Ta có $\cos \alpha =\pm \sqrt{1-{{\sin }^2}\alpha }=\pm \dfrac{12}{13}$. Do $\dfrac{\pi }{2}<\alpha <\pi $ nên $\cos \alpha =-\dfrac{12}{13}$
	}
\end{ex}
%Câu 4
\begin{ex}
	Cho các góc $\alpha $, $\beta $ thỏa mãn $\alpha ,\beta \in \left(\dfrac{\pi }{2};\pi\right)$ và $\sin \alpha =\dfrac{1}{3}$, $\cos \beta =-\dfrac{2}{3}$. Tính $\sin \left(\alpha +\beta\right)$.
	\choice
	{\True $\sin \left(\alpha +\beta\right)=-\dfrac{2+2\sqrt{10}}{9}$}
	{$\sin \left(\alpha +\beta\right)=\dfrac{2\sqrt{10}-2}{9}$}
	{$\sin \left(\alpha +\beta\right)=\dfrac{\sqrt{5}-4\sqrt{2}}{9}$}
	{$\sin \left(\alpha +\beta\right)=\dfrac{\sqrt{5}+4\sqrt{2}}{9}$}
	\loigiai{
	Do $\alpha ,\beta \in \left(\dfrac{\pi }{2};\pi\right)$ nên có: $\heva{& \cos \alpha <0 \\& \sin \beta >0}$.\\
	Ta có $\cos \alpha =-\ \sqrt{1-{{\sin }^2}\alpha }=-\ \sqrt{1-\dfrac{1}{9}}=-\ \dfrac{2\sqrt{2}}{3}$ và $\sin \beta =\sqrt{1-{{\cos }^2}\beta }=\sqrt{1-\dfrac{4}{9}}=\dfrac{\sqrt{5}}{3}$.\\
	Suy ra $\sin \left(\alpha +\beta\right)=\sin \alpha \cdot \cos \beta +\cos \alpha \cdot \sin \beta =\dfrac{1}{3} \cdot \left(-\dfrac{2}{3}\right)+\left(-\dfrac{2\sqrt{2}}{3}\right) \cdot \dfrac{\sqrt{5}}{3}=-\ \dfrac{2+2\sqrt{10}}{9}$.\\
	Vậy $\sin \left(\alpha +\beta\right)=-\ \dfrac{2+2\sqrt{10}}{9}$
	}
\end{ex}
%Câu 5
\begin{ex}
	Biết $\sin \alpha +\text{cos}\alpha =m$. Tính $P=\text{cos}\left(\alpha -\dfrac{\pi }{4}\right)$ theo $m$.
	\choice
	{$P=2m$}
	{$P=\dfrac{m}{2}$}
	{\True $P=\dfrac{m}{\sqrt{2}}$}
	{$P=m\sqrt{2}$}
	\loigiai{
	Ta có $P=\text{cos}\left(\alpha -\dfrac{\pi }{4}\right)=\text{cos}\alpha \cdot \cos \dfrac{\pi }{4}+\sin \alpha \sin \dfrac{\pi }{4}=\dfrac{1}{\sqrt{2}}\text{cos}\alpha +\dfrac{1}{\sqrt{2}}\sin \alpha $\\
	$\Rightarrow P=\dfrac{1}{\sqrt{2}}\left(\sin \alpha +\text{cos}\alpha\right)=\dfrac{m}{\sqrt{2}}$
	}
\end{ex}
%Câu 6
\begin{ex}
	Cho $x=\tan \alpha $. Tính $\sin 2\alpha $ theo $x$.
	\choice
	{$2x\sqrt{1+x^2}$}
	{$\dfrac{1-x^2}{1+x^2}$}
	{$\dfrac{2x}{1-x^2}$}
	{\True $\dfrac{2x}{1+x^2}$}
	\loigiai{
	Ta có $\sin 2\alpha =2\sin \alpha \cdot \cos\alpha =2\dfrac{\sin \alpha }{\cos\alpha }\cdot \cos^2\alpha =2\tan \alpha \cdot \dfrac{1}{1+{{\tan }^2}\alpha }=\dfrac{2x}{1+x^2}$
	}
\end{ex}
%Câu 7
\begin{ex}
	Tập xác định của hàm số $y=\cot x$ là
	\choice
	{$D=\mathbb{R}$}
	{$D=\mathbb{R}\backslash \left\{ k\dfrac{\pi }{2}\left| k\in \mathbb{Z} \right. \right\}$}
	{$D=\mathbb{R}\backslash \left\{ \pi +k\dfrac{\pi }{2}\left| k\in \mathbb{Z} \right. \right\}$}
	{\True $D=\mathbb{R}\backslash \left\{ k\pi \left| k\in \mathbb{Z} \right. \right\}$}
	\loigiai{
		Điều kiện: $\sin x\ne 0\Leftrightarrow x\ne k\pi \left(k\in \mathbb{Z}\right)$.\\
		Do đó, tập xác định của hàm số $y=\cot x$ là $D=\mathbb{R}\backslash \left\{ k\pi \left| k\in \mathbb{Z} \right. \right\}$
	}
\end{ex}
%Câu 8
\begin{ex}
	Trên khoảng $\left(-\pi ;\pi\right)$, hàm số $y=\sin x$ nghịch biến trên khoảng nào sau đây?
	\choice
	{$\left(-\pi ;0\right)$}
	{$\left(-\dfrac{\pi }{2};\dfrac{\pi }{2}\right)$}
	{$\left(0;\pi\right)$}
	{\True $\left(\dfrac{\pi }{2};\pi\right)$}
	\loigiai{
		Hàm số $y=\sin x$ nghịch biến trong khoảng $\left(\dfrac{\pi }{2};\pi\right)$
	}
\end{ex}
%Câu 9
\begin{ex}
	Hàm số $y={{\sin }^2}2x-{{\cos }^2}2x$ tuần hoàn với chu kỳ bằng
	\choice
	{$2\pi $}
	{$\pi $}
	{\True $\dfrac{\pi }{2}$}
	{$\dfrac{\pi }{4}$}
	\loigiai{
	Ta có $y={{\sin }^2}2x-{{\cos }^2}2x=-\cos 4x$. Vậy hàm số đã cho tuần hoàn với chu kỳ $\dfrac{2\pi }{4}=\dfrac{\pi }{2}$
	}
\end{ex}
%Câu 10
\begin{ex}
	Nghiệm của phương trình $2\sin x+1=0$ là
	\choice
	{$x=\dfrac{\pm \pi }{6}+k2\pi ,k\in \mathbb{Z}$}
	{$x=\dfrac{\pi }{6}+k2\pi ,k\in \mathbb{Z}$}
	{$x=\dfrac{7\pi }{6}+k2\pi ,k\in \mathbb{Z}$}
	{\True $\hoac{& x=\dfrac{-\pi }{6}+k2\pi \\& x=\dfrac{7\pi }{6}+k2\pi},k\in \mathbb{Z}$}
	\loigiai{
		Ta có: $2\sin x+1=0\Leftrightarrow \sin x=\dfrac{-1}{2}\Leftrightarrow \hoac{& x=\dfrac{-\pi }{6}+k2\pi \\& x=\dfrac{7\pi }{6}+k2\pi},k\in \mathbb{Z}$
	}
\end{ex}
%Câu 11
\begin{ex}
	Phương trình nào dưới đây vô nghiệm.
	\choice
	{$\cos x=\dfrac{1}{2}$}
	{\True $\sin x-\cos x=2$}
	{$\sin (5x+1)=1$}
	{$\sin x+\sqrt{3}\cos x=1$}
	\loigiai{
		Chú ý\\
		- $\left| \sin \alpha \right|\le 1,\forall \alpha \in \mathbb{R}$ và $\left| \cos \alpha \right|\le 1,\forall \alpha \in \mathbb{R}$ nên các phương trình ở đáp án A, C có nghiệm.\\
		- Phương trình $a\sin x+b\cos x=c$ có nghiệm khi $a^2+b^2\ge c^2$, ta kiểm tra được phương trình đáp án B vô nghiệm, đáp án D có nghiệm
	}
\end{ex}
%Câu 12
\begin{ex}
	Cho phương trình $2\tan x-3=\dfrac{-2}{\tan x+1}$. Gọi $S$ là tập hợp các nghiệm của phương trình thuộc khoảng $\left(0;\dfrac{\pi }{2}\right)$. Tổng các phần tử của $S$ là
	\choice
	{$0$}
	{$\dfrac{\pi }{3}$}
	{\True $\dfrac{\pi }{4}$}
	{$1$}
	\loigiai{
		Điều kiện : $\cos x\ne 0,\tan x\ne -1$.\\
		Vì $x\in \left(0;\dfrac{\pi }{2}\right)\Rightarrow \tan x>0$.\\
		Phương trình ban đầu tương đương\\
		$\begin{aligned}
				& \Leftrightarrow \left(2\tan x-3\right)\left(\tan x+1\right)=-2\Leftrightarrow 2{{\tan }^2}x-\tan x-3=-2 \\& \Leftrightarrow 2{{\tan }^2}x-\tan x-1=0 \end{aligned}$\\
		$\Leftrightarrow \hoac{& \tan x=1\begin{matrix}\\
					{} & (TM) \\\\
				\end{matrix} \\& \tan x=\dfrac{-1}{2}(L)}$\\
		+ Với $\tan x=1\Leftrightarrow x=\dfrac{\pi }{4}+k\pi ,k\in \mathbb{Z}$. Vì $x\in \left(0;\dfrac{\pi }{2}\right)$ nên $x=\dfrac{\pi }{4}$.\\
		Vậy $S=\left\{ \dfrac{\pi }{4} \right\}$ và tổng các phần tử của $S$ là $\dfrac{\pi }{4}$
	}
\end{ex}
\TNTF
%Câu 13
\begin{ex}
	Xét tính đúng sai của các mệnh đề sau:
	\choiceTF
	{${{\sin }^2}x=\dfrac{1+\sin 2x}{2}$}
	{\True Nếu $\cos \alpha =\dfrac{1}{3}$ thì $\cos 2\alpha =-\dfrac{7}{9}$}
	{\True Nếu $\sin x=\dfrac{3}{4}$ với $x\in \left(0;\dfrac{\pi }{2}\right)$ thì $\sin 2x=\dfrac{3\sqrt{7}}{8}$}
	{\True Cho $\cos \alpha =\dfrac{2}{3}$ với $\alpha \in \left(-\dfrac{\pi }{2};0\right)$ biết $\tan \left(\alpha +\dfrac{\pi }{4}\right)=a+b\sqrt{c}$, $c$ là số nguyên tố $\left(a,b,c\in \mathbb{Z},c\ge 0\right)$ Khi đó $a+b+c=0$}
	\loigiai{
	a) ${{\sin }^2}x=\dfrac{1-\cos 2x}{2}$\\
	b) $\cos 2\alpha =2{{\cos }^2}\alpha -1=2{{\left(\dfrac{1}{3}\right)}^2}-1=\dfrac{-7}{9}$\\
	c) Ta có ${{\cos }^2}x=1-{{\sin }^2}x=1-{{\left(\dfrac{3}{4}\right)}^2}=\dfrac{7}{16}$.\\
	Vì $x\in \left(0;\dfrac{\pi }{2}\right)$ nên $\cos x>0\Rightarrow \cos x=\dfrac{\sqrt{7}}{4}$ suy ra $\sin 2x=2\sin x \cdot \cos x=2\cdot \dfrac{\sqrt{7}}{4}\cdot \dfrac{3}{4}=\dfrac{3\sqrt{7}}{8}$\\
	d) Ta có ${{\tan }^2}\alpha =\dfrac{1}{{{\cos }^2}\alpha }-1=\dfrac{1}{{{\left(\dfrac{2}{3}\right)}^2}}-1=\dfrac{5}{4}$\\
	Vì $\alpha \in \left(-\dfrac{\pi }{2};0\right)$ nên $\tan \alpha <0\Rightarrow \tan \alpha =\dfrac{-\sqrt{5}}{2}$\\
	$\tan \left(\alpha +\dfrac{\pi }{4}\right)=\dfrac{\tan \alpha +\tan \dfrac{\pi }{4}}{1-\tan \alpha \cdot \tan \dfrac{\pi }{4}}=\dfrac{\dfrac{-\sqrt{5}}{2}+1}{1-\left(\dfrac{-\sqrt{5}}{2}\right) \cdot 1}=-9+4\sqrt{5}$\\
	Vậy $a=-9,b=4,c=5$ nên mệnh đề đúng
	}
\end{ex}
%Câu 14
\begin{ex}
	Biết $\cos x=\dfrac{1}{3}$ và $-\dfrac{\pi }{2}<x<0$. Khi đó: Các mệnh đề sau đúng hay sai?
	\choiceTF
	{\True $\sin \left(\dfrac{\pi }{2}-x\right)>0$}
    {$\sin 2x=\dfrac{4\sqrt{2}}{9}$}
	{\True $\cos \left(x+\dfrac{4\pi }{3}\right)=-\dfrac{1+3\sqrt{6}}{6}$}
	{\True $\sin x+\sin 3x=-\dfrac{8\sqrt{2}}{27}$}
	\loigiai{
	a) Ta có $\sin \left(\dfrac{\pi }{2}-x\right)=\cos x=\dfrac{1}{3}>0$\\
	b) Ta có ${{\sin }^2}x=1-{{\cos }^2}x=1-{{\left(\dfrac{1}{3}\right)}^2}=\dfrac{8}{9}\Rightarrow \sin x=\pm \dfrac{2\sqrt{2}}{3}$.\\
	Vì $-\dfrac{\pi }{2}<x<0$ nên $\sin x=-\dfrac{2\sqrt{2}}{3}$.\\
	Áp dụng công thức nhân đôi ta có: $\sin 2x=2\sin x\cos x=2 \cdot \left(-\dfrac{2\sqrt{2}}{3}\right) \cdot \dfrac{1}{3}=-\dfrac{4\sqrt{2}}{9}$\\
	c) $\cos \left(x+\dfrac{4\pi }{3}\right)=\cos x \cdot \cos \dfrac{4\pi }{3}-\sin x \cdot \sin \dfrac{4\pi }{3}=\dfrac{1}{3} \cdot \left(-\dfrac{1}{2}\right)-\left(-\dfrac{2\sqrt{2}}{2}\right) \cdot \left(-\dfrac{\sqrt{3}}{2}\right)=-\dfrac{1+3\sqrt{6}}{6}$\\
	d) Áp dụng công thức ta có:\\
	$\sin x+\sin 3x=2\sin 2x \cdot \cos x=2 \cdot \left(-\dfrac{4\sqrt{2}}{9}\right) \cdot \dfrac{1}{3}=-\dfrac{8\sqrt{2}}{27}$
	}
\end{ex}
%Câu 15
\begin{ex}
	Cho hàm số $f(x)=-2\sin \left(2x-\dfrac{\pi }{2}\right)+2025$. Các mệnh đề sau đúng hay sai?
	\choiceTF
	{\True Hàm số $f(x)$ có tập xác định là $\mathbb{R}$}
	{Hàm số $f(x)$ tuần hoàn với chu kì $T=2\pi $}
	{Hàm số $f(x)$ không chẵn, không lẻ}
	{\True Hàm số $f(x)$ đạt giá trị lớn nhất tại $x=k\pi ,k\in \mathbb{Z}$}
	\loigiai{
		a). Vì tập xác định của hàm $\sin $ là $\mathbb{R}$ nên hàm số $f(x)$ có tập xác định là $\mathbb{R}$.\\
		b). Ta có $-2\sin \left(2x-\dfrac{\pi }{2}\right)+2025=2\sin \left(\dfrac{\pi }{2}-2x\right)+2025=2\cos 2x+2025$.\\
		Do đó $f(x)=2\cos 2x+2025$ nên hàm số $f(x)$ tuần hoàn với chu kì $T=\dfrac{2\pi }{2}=\pi $.\\
		c) Ta có $\forall x\in \mathbb{R},-x\in \mathbb{R}$ và $f(-x)=2\cos (-2x)+2025=2\cos 2x+2025=f(x)$ nên hàm số $f(x)$ là hàm số chẵn.\\
		d) Ta có $-2\le 2\cos 2x\le 2,\forall x\in \mathbb{R}$ hay $2023\le 2\cos 2x+2025\le 2027,\forall x\in \mathbb{R}$.\\
		Do đó $f(x)=2027\Leftrightarrow \cos 2x=1\Leftrightarrow x=k\pi ,k\in \mathbb{Z}$.\\
		Vậy hàm số $f(x)$ đạt giá trị lớn nhất tại $x=k\pi ,k\in \mathbb{Z}$
	}
\end{ex}
%Câu 16
\begin{ex}
	Cho hàm số $f(x)=\dfrac{1}{{{\cos }^2}x}+\dfrac{1}{{{\sin }^2}x}$. Xét tính đúng sai của các mệnh đề sau
	\choiceTF
	{\True Hàm số đã cho là hàm số tuần hoàn}
	{\True Hàm số đã cho là hàm số chẵn}
	{Tập xác định của hàm số là $D=\mathbb{R}\backslash \left\{ \dfrac{\pi }{2}+k\pi ,k\in \mathbb{Z} \right\}$}
	{\True Giá trị nhỏ nhất của hàm số là 4}
	\loigiai{
		a) Hàm số tuần hoàn do hai hàm $y=\operatorname{sinx}$ và $y=\cos x$ cùng tuần hoàn với chu kì $2\pi $.\\
		b) Ta có $f(-x)=\dfrac{1}{{{\cos }^2}(-x)}+\dfrac{1}{{{\sin }^2}(-x)}=\dfrac{1}{{{\left(\operatorname{cosx}\right)}^2}}+\dfrac{1}{{{\left(-\sin x\right)}^2}}=\dfrac{1}{{{\cos }^2}x}+\dfrac{1}{{{\sin }^2}x}=f(x)$.\\
		Do đó hàm số đã cho là hàm số chẵn\\
		c) Hàm số xác định khi $\heva{& \operatorname{sinx}\ne 0 \\& \operatorname{cosx}\ne 0}\Leftrightarrow \sin 2x\ne 0\Leftrightarrow 2x\ne k\pi \Leftrightarrow x\ne \dfrac{k\pi }{2},k\in \mathbb{Z}$\\
		Tập xác định của hàm số là $D=\mathbb{R}\backslash \left\{ \dfrac{k\pi }{2},k\in \mathbb{Z} \right\}$.\\
		d) Khi $x\ne \dfrac{k\pi }{2},k\in \mathbb{Z}$ ta có\\
		$f(x)=\dfrac{1}{{{\cos }^2}x}+\dfrac{1}{{{\sin }^2}x}\ge 2\sqrt{\dfrac{1}{{{\cos }^2}x} \cdot \dfrac{1}{{{\sin }^2}x}}=2\sqrt{\dfrac{4}{{{\sin }^2}2x}}=\dfrac{4}{\left| \sin 2x \right|}\ge \dfrac{4}{1}=4$.\\
		Nên giá trị nhỏ nhất của hàm số là 4
	}
\end{ex}
\TNSA
%Câu 19
\begin{ex}
    Tìm tập giá trị của các hàm số $y=\sqrt{2+\cos x}-5$ là đoạn $[a;b]$. Giá trị $a+b$ (làm tròn đến hàng phần chục) là
    \shortans{-7,3}
    \loigiai{
    Vì $\cos x\ge -1\Leftrightarrow 2+\cos x\ge 1>0,\forall x\in \mathbb{R}$ nên tập xác định của hàm số là $D=\mathbb{R}$.\\
    $\forall x\in \mathbb{R}$, ta có:
    \begin{eqnarray*}
        & & -1\le \cos x\le 1 \\
        & \Leftrightarrow & 1\le 2+\cos x\le 3 \\
        & \Leftrightarrow & 1\le \sqrt{2+\cos x}\le \sqrt{3} \\
        & \Leftrightarrow & -4\le \sqrt{2+\cos x}\,-5\le \sqrt{3}-5
    \end{eqnarray*}
    Vậy tập giá trị của hàm số là $T=\left[-4;\sqrt{3}-5\right]$. Suy ra $a+b \approx -7,3$.
    }    
\end{ex}
%Câu 18
\begin{ex}
	Tổng số giờ ban ngày của ngày thứ $x$ trong một năm không nhuận được tính bởi công thức
	$g(x)=3\sin (0,0172x-1,376)+12$.
	Trong đó $x$ đại diện cho ngày trong năm, $1\le x\le 365$. Ngày $\overline{ab}$ tháng $\overline{cd}$ có số giờ ban ngày dài nhất. Số $\overline{abcd}$ bằng
	\shortans{2006}
	\loigiai{
		Ta có $-1\le \sin (0,0172x-1,376)\le 1$\\
		$-3\le 3\sin (0,0172x-1,376)\le 3$\\
		$9\le 3\sin (0,0172x-1,376)+12\le 15$\\
		Suy ra $9\le g(x)\le 15$\\
		Do đó, số giờ ban ngày dài nhất trong một ngày là 15 giờ.\\
		Ta có phương trình $3\sin (0,0172x-1,376)+12=15$\\
		$\sin (0,0172x-1,376)=1$\\
		$x\approx 171{,}3$\\
		Vậy vào khoảng ngày thứ 171 trong năm (ngày 20 tháng 6) thì số giờ ban ngày dài nhất
	}
\end{ex}
%Câu 19
\begin{ex}
	Hai thành phố có cùng kinh độ. Vĩ tuyến của thành phố A là $10^\circ $ Bắc và vĩ tuyến của thành phố B là $40^\circ $ Bắc. Giả sử bán kính trái đất là 3960 dặm. Tìm khoảng cách giữa hai thành phố (làm tròn đến chữ số hàng đơn vị)
	\shortans{2073}
	\loigiai{
		Khoảng cách từ điểm trên đường xích đạo đến thành phố B ở cùng kinh độ là $3960 \cdot \dfrac{40}{180} \cdot \pi =880\pi $ (dặm)\\
		Khoảng cách từ điểm trên đường xích đạo đến thành phố A ở cùng kinh độ là $3960 \cdot \dfrac{10}{180}\pi =220\pi $ (dặm)\\
		Khoảng cách giữa hai thành phố A và B là $880\pi -220\pi =660\pi \approx 2073$ (dặm)
	}
\end{ex}
%Câu 20
\begin{ex}
	Giả sử vận tốc $v$ (tính bằng lít/ giây) của luồng khí trong một chu kì hô hấp (tức là thời gian từ lúc bắt đầu của một nhịp thở đến khi bắt đầu của nhịp thở tiếp theo) của một người nào đó ở trạng thái nghỉ ngơi được cho bởi công thức $v=0{,}85\sin \dfrac{\pi t}{3}$, trong đó $t$ là thời gian (tính bằng giây). Biết rằng quá trình hít vào xảy ra khi $v>0$ và quá trình thở ra xảy ra khi $v<0$. Trong khoảng thời gian từ 5 đến 10 giây, khoảng thời điểm sau $a$ giây đến trước $b$ giây thì người đó hít vào. Tính $\sqrt{a+b}$ (làm tròn đến hàng phần trăm).
	\shortans{3,87}
	\loigiai{
		+) Vì quá trình hít vào xảy ra khi $v>0$ nên ta có\\
		$0{,}85\sin \dfrac{\pi t}{3}>0\Leftrightarrow \sin \dfrac{\pi t}{3}>0\Leftrightarrow \dfrac{\pi t}{3}\in \left(k2\pi ;\pi +k2\pi\right)(k\in \mathbb{Z})$\\
		$\Leftrightarrow t\in \left(6k;3+6k\right)\,\left(k\in \mathbb{Z}\right)$\\
		+) Vì $t\in [5;10]$ nên $k=1$ suy ra $t\in \left(6;9\right)$.\\
		Trong khoảng thời gian từ 5 đến 10 giây, khoảng thời điểm sau $6$ giây đến trước $9$ giây thì người đó hít vào nên $\sqrt{a+b}=\sqrt{15}\approx 3,87$.
	}
\end{ex}
%Câu 21
\begin{ex}
	Nghiệm phương trình lượng giác $\sqrt{3}\sin x-\cos x=0$ có dạng $x=\dfrac{\pi }{a}+k \cdot b\pi $ ($a$, $b$, $k\in \mathbb{Z}$, $a\ne 0$). Tính $(a+b)^4$.
	\shortans{2401}
	\loigiai{
		Phương trình tương đương\\
		$\dfrac{\sqrt{3}}{2}\sin x-\dfrac{1}{2}\cos x=0$\\
		$\Leftrightarrow \sin x\cos \dfrac{\pi }{6}-\cos x\sin \dfrac{\pi }{6}=0$\\
		$\Leftrightarrow \sin \left(x-\dfrac{\pi }{6}\right)=0$\\
		$\Leftrightarrow x-\dfrac{\pi }{6}=k\pi $ ($k\in \mathbb{Z}$)\\
		$\Leftrightarrow x=\dfrac{\pi }{6}+k\pi $.\\
		Phương trình có nghiệm là: $x=\dfrac{\pi }{6}+k\pi $ ($k\in \mathbb{Z}$).\\
		Suy ra $a=6$; $b=1$. Vậy $(a+b)^4=7^4=2401$.
	}
\end{ex}
%Câu 22
\begin{ex}
	Một vật $M$ được gắn vào đầu lò xo và dao động quanh vị trí cân bằng, toạ độ $x$ (đơn vị: cm) tại thời điểm $t$ (giây) được tính bởi công thức $x=8{,}6\sin \left(8t+\dfrac{\pi }{2}\right)$. Có $n$ thời điểm trong khoảng 2 giây đầu tiên thì $s=4{,}3$ cm. Giá trị $\sqrt[3]{n}$ (làm tròn đến hàng phần trăm)
	\shortans{1,71}
	\loigiai{
	Khi $x=4{,}3$ thì $8{,}6\sin \left(8t+\dfrac{\pi }{2}\right)=4{,}3\Rightarrow \sin \left(8t+\dfrac{\pi }{2}\right)=\dfrac{1}{2}$\\
	$\Leftrightarrow \hoac{&8t+\dfrac{\pi }{2}=\dfrac{\pi }{6}+k2\pi  \\
			&8t+\dfrac{\pi }{2}=\dfrac{5\pi }{6}+l2\pi}(k,l\in \mathbb{Z})
            \Leftrightarrow \hoac{& t=-\dfrac{\pi }{24}+k\dfrac{\pi }{4} \\
            & t=\dfrac{\pi }{24}+l\dfrac{\pi }{4} }(k,l\in \mathbb{Z})$.\\
	Vì $t\in (0;2)$ nên $\heva{& 0<-\dfrac{\pi }{24}+k\dfrac{\pi }{4}<2 \\ & 0<\dfrac{\pi }{24}+l\dfrac{\pi }{4}<2} \Leftrightarrow \heva{& \dfrac{1}{6}<k<\dfrac{8}{\pi }+\dfrac{1}{6}  \\ & -\dfrac{1}{6}<l<\dfrac{8}{\pi }-\dfrac{1}{6}}$\\
	Mà $k,l\in \mathbb{Z}$ nên $k\in \left\{ 1;2 \right\}$; $l\in \left\{ 0;1;2 \right\}$.\\
	Vậy có $5$ thời điểm thỏa mãn đề bài nên $\sqrt[3]{n}\approx 1,71$
	}
\end{ex}

% \begin{name}
	{\tenchude}
	{ĐỀ ÔN TẬP CHƯƠNG I}
	{LỚP TOÁN THẦY PHÁT}
	{\thoigian}
\end{name}
\TN
\setcounter{ex}{0}
\Opensolutionfile{ans}[ans/ans-TN-C1-De1]
\TN
%Câu 1
\begin{ex}
	Rút gọn biểu thức $M=\cos 2x \cdot \cos x+\sin 2x \cdot \sin x$ ta được kết quả là:
	\choice
	{\True $M=\cos x$}
	{$M=\cos 3x$}
	{$M=\sin x$}
	{$M=\sin 3x$}
	\loigiai{
		Ta có: $M=\cos 2x \cdot \cos x+\sin 2x \cdot \sin x=\cos (2x-x)=\cos x$
	}
\end{ex}
%Câu 2
\begin{ex}
	Đẳng thức nào không đúng với mọi $x$?
	\choice
	{$\cos^2 3x=\dfrac{1+\cos 6x}{2}$}
	{$\cos 2x=1-2\sin^2x$}
	{$\sin 2x=2\sin x\cos x$}
	{\True $\sin^2 2x=\dfrac{1+\cos 4x}{2}$}
	\loigiai{
		Ta có $\sin^2 2x=\dfrac{1-\cos 4x}{2}$
	}
\end{ex}
%Câu 3
\begin{ex}
	Góc có số đo $\dfrac{\pi }{24}$ đổi sang độ bằng
	\choice
	{$7^\circ $}
	{\True $7^\circ 3{0}'$}
	{$8^\circ $}
	{$8^\circ 3{0}'$}
	\loigiai{
		Ta có: $\dfrac{\pi }{24}=\dfrac{180^\circ }{24}=7^\circ 30'$
	}
\end{ex}
%Câu 4
\begin{ex}
	Một đường tròn có đường kính là $50$ (cm). Độ dài của cung tròn trên đường tròn có số đo là $\dfrac{\pi }{4}$ bằng (làm tròn đến hàng đơn vị)
	\choice
	{$40$ (cm)}
	{$39$ (cm)}
	{$19$ (cm)}
	{\True $20$ (cm)}
	\loigiai{
		Độ dài của cung tròn $l=\alpha \cdot R=\dfrac{\pi }{4} \cdot 25=\dfrac{25}{4}\pi \approx 20$ (cm)
	}
\end{ex}
%Câu 5
\begin{ex}
	Chọn phát biểu đúng:
	\choice
	{Các hàm số $y=\sin x$, $y=\cos x$, $y=\cot x$ đều là hàm số chẵn}
	{Các hàm số $y=\sin x$, $y=\cos x$, $y=\cot x$ đều là hàm số lẻ}
	{Các hàm số $y=\sin x$, $y=\cot x$, $y=\tan x$ đều là hàm số chẵn}
	{\True Các hàm số $y=\sin x$, $y=\cot x$, $y=\tan x$ đều là hàm số lẻ}
	\loigiai{
		Hàm số $y=\cos x$ là hàm số chẵn, hàm số $y=\sin x$, $y=\cot x$, $y=\tan x$ là các hàm số lẻ
	}
\end{ex}
%Câu 6
\begin{ex}
	Nếu $\sin x+\cos x=\dfrac{1}{2}$ thì $\sin 2x$ bằng
	\choice
	{$\dfrac{3}{4}$}
	{$\dfrac{3}{8}$}
	{$\dfrac{\sqrt{2}}{2}$}
	{\True $\dfrac{-3}{4}$}
	\loigiai{
	Do $\sin x+\cos x=\dfrac{1}{2}\Rightarrow \dfrac{1}{4}={{\left(\sin x+\cos x\right)}^2}={{\left(\sin x\right)}^2}+{{\left(\text{cosx}\right)}^2}+2\sin x \cdot \cos x$\\
	$\Rightarrow \dfrac{1}{4}=1+\sin 2x\Rightarrow \sin 2x =\dfrac{-3}{4}$
	}
\end{ex}
%Câu 7
\begin{ex}
	Một con lắc lò xo sau khi được kéo xuống dưới vị trí cân bằng $4$ cm và thả ra thì nó dao động điều hòa với phương trình: $y=-4\cos 8t$ (cm). Biên độ $A$ cm và chu kỳ $T$ của dao động là
	\choice
	{\True $A=4$ cm, $T=\dfrac{\pi }{4}$}
	{$A=4$ cm, $T=\dfrac{\pi }{2}$}
	{$A=8$ cm, $T=\dfrac{\pi }{4}$}
	{$A=4$ cm, $T=2\pi $}
	\loigiai{
		Biên độ của dao động là: $A=|-4|=4$ (cm).\\
		Chu kỳ của dao động là:$T=\dfrac{2\pi }{|8|}=\dfrac{\pi }{4}$
	}
\end{ex}
%Câu 8
\begin{ex}
	Hãy tìm tập tất cả các giá trị của $m$ để phương trình $\left| \sin x \right|=m$ có nghiệm?
	\choice
	{$-1\le m\le 1$}
	{$-1\le m\le 0$}
	{$-1<m<0$}
	{\True $0\le m\le 1$}
	\loigiai{
		Vì $0<= |\sin x|<=1, \forall x \in \mathbb{R}$ nên phương trình $\left| \sin x \right|=m$ có nghiệm khi và chỉ khi $0\le m\le 1$.
	}
\end{ex}
%Câu 9
\begin{ex}
	Nghiệm của phương trình $2\sin \left(4x-\dfrac{\pi }{3}\right)-1=0$ là:
	\choice
	{$x=\pi +k2\pi ;x=k\dfrac{\pi }{2}\ (k \in \mathbb{Z})$}
	{$x=\dfrac{\pi }{8}+k\dfrac{\pi }{2};x=\dfrac{7\pi }{24}+k\dfrac{\pi }{2}\ (k \in \mathbb{Z})$}
	{$x=k2\pi ;x=\dfrac{\pi }{2}+k2\pi\ (k \in \mathbb{Z})$}
	{$x=k\pi ;x=\pi +k2\pi\ (k \in \mathbb{Z})$}
	\loigiai{
		$2\sin \left(4x-\dfrac{\pi }{3}\right)-1=0\Leftrightarrow \sin \left(4x-\dfrac{\pi }{3}\right)=\dfrac{1}{2}\Leftrightarrow \hoac{& 4x-\dfrac{\pi }{3}=\dfrac{\pi }{6}+k2\pi \\& 4x-\dfrac{\pi }{3}=\pi -\dfrac{\pi }{6}+k2\pi}\Leftrightarrow \hoac{& x=\dfrac{\pi }{8}+k\dfrac{\pi }{2} \\& x=\dfrac{7\pi }{24}+k\dfrac{\pi }{2}}\left(k\in \mathbb{Z}\right)$
	}
\end{ex}
%Câu 10
\begin{ex}
	Biết $\sin \left(\alpha +\dfrac{3\pi }{2}\right)+\cos \left(\alpha +\dfrac{3\pi }{2}\right)=\sqrt{2}$. Tính $\sin \left(\alpha +\pi\right)-2\cos \left(\alpha -\pi\right)$.
	\choice
	{$\dfrac{3}{\sqrt{2}}$}
	{\True $-\dfrac{3}{\sqrt{2}}$}
	{$-\dfrac{1}{\sqrt{2}}$}
	{$\dfrac{1}{\sqrt{2}}$}
	\loigiai{
	Ta có $\sin \left(\alpha +\dfrac{3\pi }{2}\right)=\sin \left(\alpha +2\pi -\dfrac{\pi }{2}\right)=\sin \left(\alpha -\dfrac{\pi }{2}\right)=-\sin \left(\dfrac{\pi }{2}-\alpha\right)=-\cos \alpha $.\\
	$\cos \left(\alpha +\dfrac{3\pi }{2}\right)=\cos \left(\alpha +2\pi -\dfrac{\pi }{2}\right)=\cos \left(\alpha -\dfrac{\pi }{2}\right)=\cos \left(\dfrac{\pi }{2}-\alpha\right)=\sin \alpha $.\\
	Suy ra $\sin \alpha -\cos \alpha =\sqrt{2}\Rightarrow \sin \alpha =\cos \alpha +\sqrt{2}$.\\
	Vì ${{\sin }^2}\alpha +{{\cos }^2}\alpha =1\Rightarrow 2{{\cos }^2}\alpha +2\sqrt{2}\cos \alpha +2=1$\\
	$\Leftrightarrow 2{{\cos }^2}\alpha +2\sqrt{2}\cos \alpha +1=0\Leftrightarrow \cos \alpha =-\dfrac{1}{\sqrt{2}}\Rightarrow \sin \alpha =\dfrac{1}{\sqrt{2}}$.\\
	Do đó $\sin \left(\alpha +\pi\right)-2\cos \left(\alpha -\pi\right)=-\sin \alpha +2\cos \alpha =-\dfrac{3}{\sqrt{2}}$
	}
\end{ex}
%Câu 11
\begin{ex}
	Hằng ngày mực nước của con kênh lên xuống theo thủy triều. Độ sâu $h$(mét) của mực nước trong kênh được tính tại thời điểm $t$ (giờ) trong một ngày bởi công thức $h=3\cos \left(\dfrac{\pi t}{7=8}+\dfrac{\pi }{4}\right)+12$. Mực nước của kênh cao nhất khi:
	\choice
	{$t=13$(giờ)}
	{\True $t=14$(giờ)}
	{$t=15$(giờ)}
	{$t=16$(giờ)}
	\loigiai{
		Mực nước của kênh cao nhất khi $h$ lớn nhất\\
		$\Leftrightarrow \cos \left(\dfrac{\pi t}{8}+\dfrac{\pi }{4}\right)=1\Leftrightarrow \dfrac{\pi t}{8}+\dfrac{\pi }{4}=k2\pi $ với $0<t\le 24$ và $k\in \mathbb{Z}$.\\
		Lần lượt thay các đáp án, ta được đáp án B thỏa mãn.\\
		Vì với $t=14$ thì $\dfrac{\pi t}{8}+\dfrac{\pi }{4}=2\pi $ (đúng với $k=1\in \mathbb{Z}$)
	}
\end{ex}
%Câu 12
\begin{ex}
	Số giờ có ánh sáng mặt trời của một thành phố A ở vĩ độ ${{40}^{\text{o}}}$ bắc trong ngày thứ t của một năm không nhuận được cho bởi hàm số $d(t)=3\sin \left[\dfrac{\pi }{180}(t-80)\right]+12$ với $t\in \mathbb{Z}$ và $0<t\le 365$. Vào ngày nào trong năm thì thành phố A có nhiều giờ có ánh sáng mặt trời nhất?
	\choice
	{\True 170}
	{171}
	{172}
	{173}
	\loigiai{
		Ta có $d(t)=3\sin \left[\dfrac{\pi }{180}(t-80)\right]+12\le 3 \cdot 1+12=15$.\\
		Vậy thành phố A có nhiều giờ có ánh sáng mặt trời nhất khi $\sin \left[\dfrac{\pi }{180}(t-80)\right]=1\Leftrightarrow \dfrac{\pi }{180}(t-80)=\dfrac{\pi }{2}+k2\pi \Leftrightarrow t=170+360k (k\in \mathbb{Z})$.\\
		Vì $0<t\le 365$ nên $0<170+360k\le 365\Leftrightarrow -\dfrac{17}{36}<k\le \dfrac{39}{72}\Rightarrow k=0\Rightarrow t=170$.
	}
\end{ex}

\TNTF
%Câu 13
\begin{ex}
	Cho phương trình $\sin x=a$ (1).
	\choiceTF
	{\True Nếu $a>1$ thì phương trình (1) vô nghiệm}
	{Nếu $a=1$ thì phương trình (1) có nghiệm $\alpha =\dfrac{\pi }{2}+k\pi ,\left(k\in \mathbb{Z}\right)$}
	{\True Nếu $-1\le a\le 1$ thì phương trình (1) có nghiệm $\hoac{& x=\alpha +k2\pi \\ & x=\pi -\alpha +k2\pi} \left(k\in \mathbb{Z}\right)$}
	{Phương trình (1) luôn có hai điểm biểu diễn nghiệm trên đường tròn lượng giác}
	\loigiai{
		Nếu $a=1\Rightarrow \sin \alpha =1\Leftrightarrow \alpha =\dfrac{\pi }{2}+k2\pi ,\left(k\in \mathbb{Z}\right)$
	}
\end{ex}
%Câu 14
\begin{ex}
	Các mệnh đề sau đúng hay sai?
	\choiceTF
	{\True Hàm số $y=\sin \sqrt{x+4}$ có tập xác định là $D=\left[-4;+\infty\right)$}
	{Hàm số $y=\cot \left(\dfrac{\pi }{2}+x\right)$ có tập xác định là $D=\mathbb{R}$}
	{\True Hàm số $y=\sqrt{3-2\cos x}$ có tập xác định là $D=\mathbb{R}$}
	{Hàm số $y=\dfrac{1-3\cos x}{\sin x}$ có tập xác định là $D=\mathbb{R}\backslash \left\{ k\dfrac{\pi }{2},k\in \mathbb{Z} \right\}$}
	\loigiai{
	a) Hàm số xác định khi và chỉ khi $x+4\ge 0\Leftrightarrow x\ge -4$.\\
	Vậy tập xác định của hàm số là $D=\left[-4;+\infty\right)$.\\
	b) Hàm số xác định khi và chỉ khi $\sin \left(x+\dfrac{\pi }{2}\right)\ne 0\Leftrightarrow x+\dfrac{\pi }{2}\ne k\pi \Leftrightarrow x\ne -\dfrac{\pi }{2}+k\pi ;k\in \mathbb{Z}$.\\
	Vậy tập xác định của hàm số là $D=\mathbb{R}\backslash \left\{ -\dfrac{\pi }{2}+k\pi ;k\in \mathbb{Z} \right\}$.\\
	c) Hàm số xác định khi $3-2\cos x\ge 0\Leftrightarrow \cos x\le \dfrac{3}{2}$ (đúng $\forall x\in \mathbb{R}$), vì $-1\le \cos x\le 1,\forall x\in \mathbb{R}$.\\
	Vậy tập xác định của hàm là $D=\mathbb{R}$.\\
	d) Hàm số xác định khi và chỉ khi $\sin x\ne 0\Leftrightarrow x\ne k\pi \left(k\in \mathbb{Z}\right)$.\\
	Vậy tập xác định của hàm số là $D=\mathbb{R}\backslash \left\{ k\pi ,k\in \mathbb{Z} \right\}$
	}
\end{ex}
%Câu 15
\begin{ex}
	Hằng ngày mực nước của con kênh lên xuống theo thủy triều. Độ sâu $h$ (mét) của mực nước trong kênh tính theo thời gian $t$ (giờ) được cho bởi công thức $h(t)=3\cos \left(\dfrac{\pi t}{6}+\dfrac{\pi }{4}\right)+14$.
	\choiceTF
	{Công thức tuần hoàn với chu kì $T=2\pi $}
	{\True Chiều sâu của mực nước thấp nhất là $11 \text{m}$}
	{Chiều sâu của mực nước cao nhất là $14 \text{m}$}
	{\True Thời gian để mực nước cao nhất là $t=9$}
	\loigiai{
		a) Công thức có dạng $y=\cos (ax+b)$ tuần hoàn với chu kì $T=\dfrac{2\pi }{|a|}$ nên chu kì cần tìm là $T=\dfrac{2\pi }{\left| \dfrac{\pi }{6} \right|}=12$.\\
		b) Ta có $\forall t\colon -1\le \cos \left(\dfrac{\pi t}{6}+\dfrac{\pi }{4}\right)\le 1\Leftrightarrow -3\le 3\cos \left(\dfrac{\pi t}{6}+\dfrac{\pi }{4}\right)\le 3\Leftrightarrow 11\le 3\cos \left(\dfrac{\pi t}{6}+\dfrac{\pi }{4}\right)+14\le 17\Leftrightarrow 11\le h\le 17$. Vậy chiều sâu của mực nước thấp nhất là $11 \text{m}$.\\
		c) Ta có $\forall t\colon -1\le \cos \left(\dfrac{\pi t}{6}+\dfrac{\pi }{4}\right)\le 1\Leftrightarrow -3\le 3\cos \left(\dfrac{\pi t}{6}+\dfrac{\pi }{4}\right)\le 3\Leftrightarrow 11\le 3\cos \left(\dfrac{\pi t}{6}+\dfrac{\pi }{4}\right)+14\le 17\Leftrightarrow 11\le h\le 17$. Chiều sâu của mực nước cao nhất là $17 \text{m}$.\\
		d) Ta có $\forall t\colon -1\le \cos \left(\dfrac{\pi t}{6}+\dfrac{\pi }{4}\right)\le 1\Leftrightarrow -3\le 3\cos \left(\dfrac{\pi t}{6}+\dfrac{\pi }{4}\right)\le 3\Leftrightarrow 11\le 3\cos \left(\dfrac{\pi t}{6}+\dfrac{\pi }{4}\right)+14\le 17\Leftrightarrow 11\le h\le 17$. Chiều sâu của mực nước cao nhất là $17 \text{m}$.\\
		Max $h=17\Leftrightarrow \cos \left(\dfrac{\pi t}{6}+\dfrac{\pi }{4}\right)=1\Leftrightarrow \dfrac{\pi t}{6}+\dfrac{\pi }{4}=k2\pi \Leftrightarrow t=-3+12k,k\in \mathbb{Z}$.\\
		Vì thời gian không âm và $k\in \mathbb{Z}$ nên ta chọn $t=1$. Vậy thời gian ngắn nhất $t=-3+12=9$
	}
\end{ex}
%Câu 16
\begin{ex}
	Cho phương trình $\left(2\cos x-1\right)\left(\sin 2x-m\right)=0$ (1).
	\choiceTF
	{\True $x=\dfrac{7\pi }{3}$ là một nghiệm của phương trình $(1)$}
	{Khi $m=2$ thì phương trình $(1)\Leftrightarrow \hoac{& x=\pm\dfrac{\pi }{3}+k2\pi \\& x=\dfrac{\pi }{2}+l2\pi} (k,l \in \mathbb{Z})$}
	{\True Khi $m=1$ thì tập nghiệm của phương trình $(1)$ có tất cả 4 điểm biểu diễn trên đường tròn lượng giác}
	{Chỉ tìm được một giá trị của $m$ để phương trình $(1)$ có đúng hai nghiệm thuộc $\left(-\dfrac{\pi }{4};\dfrac{3\pi }{4}\right]$}
			\loigiai{
			Ta có $\left(2\cos x-1\right)\left(\sin 2x-m\right)=0\Leftrightarrow \hoac{& \cos x=\dfrac{1}{2} \\& \sin 2x=m}\Leftrightarrow \hoac{& x=\dfrac{\pi }{3}+k2\pi \\& x=-\dfrac{\pi }{3}+k2\pi \\& \sin 2x=m}$\\
			a) Thay $x=\dfrac{7\pi }{3}$ phương trình $(1)$ ta thấy thỏa mãn nên $x=\dfrac{7\pi }{3}$ là một nghiệm của phương trình $(1)$.\\
			b) Khi $m=2$ thì phương trình $(1)\Leftrightarrow \hoac{& x=\dfrac{\pi }{3}+k2\pi \\& x=-\dfrac{\pi }{3}+k2\pi} (k \in \mathbb{Z})$\\
			c) Khi $m=1$ phương trình $(1)\Leftrightarrow \hoac{& x=\dfrac{\pi }{3}+k2\pi \\& x=-\dfrac{\pi }{3}+k2\pi \\& \sin 2x=1}\Leftrightarrow \hoac{& x=\dfrac{\pi }{3}+k2\pi \\& x=-\dfrac{\pi }{3}+k2\pi \\& x=\dfrac{\pi }{4}+l\pi}$.\\
			Do đó tập nghiệm của phương trình $(1)$ có tất cả $4$ điểm biểu diễn trên đường tròn lượng giác.\\
			d) Do phương trình $(2)$ có một nghiệm $x=\dfrac{\pi }{3}$ thuộc $\left(-\dfrac{\pi }{4};\dfrac{3\pi }{4}\right]$.\\
			Do đó để phương trình $(1)$ có đúng hai nghiệm thuộc $\left(-\dfrac{\pi }{4};\dfrac{3\pi }{4}\right]$ thì phương trình $\sin 2x=m$ có 1 nghiệm thuộc $\left(-\dfrac{\pi }{4};\dfrac{3\pi }{4}\right]$ khác $\dfrac{\pi }{3}$ (*)\\
			Ta có $x\in \left(-\dfrac{\pi }{4};\dfrac{3\pi }{4}\right]\Rightarrow 2x\in \left(-\dfrac{\pi }{2};\dfrac{3\pi }{2}\right]$ hay $2x\in \left[0;2\pi\right]$\\
			Từ (*) suy ra $m=1$ hoặc $m=-1$\\
	}
\end{ex}

\TNSA
%Câu 17
\begin{ex}
	Cho góc $\alpha $ thỏa mãn $\sin \alpha =\dfrac{1}{5}$. Khi đó giá trị biểu thức $P={{\cos }^2}2x+{{\cos }^2}x$ bằng $\dfrac{a}{b}$. Tính $a+b$. Biết rằng phân số $\dfrac{a}{b}$ là phân số tối giản
	\shortans{1754}
	\loigiai{
		Biến đổi biểu thức $P$ rồi thay giá trị $\sin \alpha =\dfrac{1}{5}$ vào $P$, ta được:\\
		$\begin{aligned}
				& P={{\cos }^2}2x+{{\cos }^2}x \\& \text{ }={{\left(1-2{{\sin }^2}\alpha\right)}^2}+\left(1-{{\sin }^2}\alpha\right)={{\left(1-2 \cdot {{\left(\dfrac{1}{5}\right)}^2}\right)}^2}+\left(1-{{\left(\dfrac{1}{5}\right)}^2}\right)=\dfrac{1129}{625} \end{aligned}$\\
		$\Rightarrow \heva{& a=1129 \\& b=625}\Rightarrow a+b=1754$
	}
\end{ex}
%Câu 18
\begin{ex}
	Số điểm chung của đồ thị hàm số $y=\sin x$ và $y=\cos x$ trên $\left[ -\dfrac{\pi }{2};\dfrac{3\pi }{2} \right]$ là $n$. Giá trị $\sqrt{n}$ (làm tròn đến hàng phần trăm) bằng
	\shortans{1,41}
	\loigiai{
		Số điểm chung của đồ thị hàm số $y=\sin x$ và $y=\cos x$ trên $\left[ -\dfrac{\pi }{2};\dfrac{3\pi }{2} \right]$ bằng số nghiệm phương trình $\sin x = \cos x$ trên $\left[ -\dfrac{\pi }{2};\dfrac{3\pi }{2} \right]$.\\
		Ta có $\sin x = \cos x \Leftrightarrow \sin x - \cos x =0 \Leftrightarrow \sin \left(x-\dfrac{\pi}{4} \right)=0 \Leftrightarrow x-\dfrac{\pi}{4}=k \pi \Leftrightarrow x= \dfrac{\pi}{4} +k\pi \ (k \in \mathbb{Z})$.\\
		$x \in \left[ -\dfrac{\pi }{2};\dfrac{3\pi }{2} \right]$ nên $x \in \left\{ \dfrac{\pi}{4}; \dfrac{5\pi}{4}\right\}$.\\
		Vậy $n=2$ nên $\sqrt{n} \approx 1,41$.
	}
\end{ex}
%Câu 19
\begin{ex}
	Biết có $n$ giá trị nguyên của tham số $m$ để phương trình $\cos x=m$ có nghiệm. Giá trị $\sqrt{n}$ (làm tròn đến hàng phần trăm) bằng
	\shortans{1,73}
	\loigiai{
		$\cos x=m$ có nghiệm $\Leftrightarrow -1\le m\le 1$. Mà $m\in \mathbb{Z}\Rightarrow m\in \left\{ -1;0;1 \right\}$. Vậy $\sqrt{n}\approx 1{,}73$
	}
\end{ex}
%Câu 20
\begin{ex}
	Biết $x=x_0$ là nghiệm duy nhất của phương trình $2\sin \left(x-\dfrac{\pi }{6}\right)+2=0$ trên khoảng $\left(0;2\pi\right)$. Giá trị $x_0$ (làm tròn đến hàng phần trăm) bằng
	\shortans{5,24}
	\loigiai{
		Ta có: $2\sin \left(x-\dfrac{\pi }{6}\right)+2=0\Leftrightarrow \sin \left(x-\dfrac{\pi }{6}\right)=-1\Leftrightarrow x=-\dfrac{\pi }{3}+k2\pi ,k\in \mathbb{Z}$\\
		Do $x\in \left(0;2\pi\right)$ nên $0<-\dfrac{\pi }{3}+k2\pi <2\pi \Leftrightarrow \dfrac{1}{6}<k<\dfrac{7}{6}\Leftrightarrow k=1$.\\
		Vậy phương trình có một nghiệm $x=\dfrac{5\pi }{3}\approx 5{,}24$
	}
\end{ex}
%Câu 21
\begin{ex}
	Gọi $M$ và $m$ lần lượt là giá trị lớn nhất và giá trị nhỏ nhất của hàm số $y=\sin x+\sqrt{3}\cos x+\sqrt{2}$. Tính $M^2m$ (làm tròn đến hàng phần trăm)
	\shortans{6,83}
	\loigiai{
		Ta có $y=\sin x+\sqrt{3}\cos x+\sqrt{2}=2\left(\dfrac{1}{2}\sin x+\dfrac{\sqrt{3}}{2}\cos x\right)+\sqrt{2}=2\sin \left(x+\dfrac{\pi }{3}\right)+\sqrt{2}$.\\
		Suy ra $M=2+\sqrt{2}$, $m=-2+\sqrt{2}$. Nên $M^2m\approx 6{,}83$
	}
\end{ex}
%Câu 22
\begin{ex}
	Mùa xuân ở Hội Lim (tỉnh Bắc Ninh) thường có trò chơi đu. Khi người chơi đu nhún đều, cây đu sẽ đưa người chơi đu dao động qua lại vị trí cân bằng. Nghiên cứu trò chơi này, người ta thấy khoảng cách $h$ (mét) được tính từ vị trí chân người chơi đu đến vị trí cân bằng được biểu diễn bởi hệ thức $h=|d|$ với $d=3\cos \left[\dfrac{\pi }{3}(2t-1)\right]$ ($t\ge 0$ và được tính bằng giây), trong đó ta quy ước $d>0$ khi vị trí cân bằng ở về phía sau lưng người chơi đu và $d<0$ trong trường hợp ngược lại.
	Biết $t_1$, $t_2$ lần lượt là thời điểm đầu tiên người đu ở vị trí phía sau lưng và vị trí phía trước vị trí cân bằng $1{,}5$ mét. Giá trị $t_1+t_2^2$ (làm tròn đến hàng phần trăm) bằng
	\shortans{3,25}
	\loigiai{
	Người chơi cách vị trí cân bằng 1 mét khi $3\cos \left[\dfrac{\pi }{3}(2t-1)\right]=\pm 1{,}5$\\
	$\Leftrightarrow \cos^2\left[\dfrac{\pi }{3}(2t-1)\right]=\dfrac{1}{4}\Leftrightarrow \cos \left[\dfrac{2\pi }{3}(2t-1)\right]=-\dfrac{1}{2}$ $\Leftrightarrow \hoac{& \dfrac{2\pi }{3}(2t-1)=\dfrac{2\pi }{3}+k2\pi \\& \dfrac{2\pi }{3}(2t-1)=-\dfrac{2\pi }{3}+k2\pi} \left(k\in \mathbb{Z}\right)
		\Leftrightarrow \hoac{& t=1+\dfrac{3k}{2} \\& t=\dfrac{3k}{2}}\left(k\in \mathbb{Z}\right)$.\\
	Vì $t>0$ nên $t_1=1$ và $t_2=1{,}5$. Vậy $t_1+t_2^2=3{,}25$
	}
\end{ex}

\Closesolutionfile{ans}

\indapan{10}{ans/ans-SA-C1-De1}
% \section*{BT ÔN TẬP CHƯƠNG 1}
\setcounter{ex}{0}\setcounter{bt}{0}
\Opensolutionfile{ans}[ans/ans1C3-CD-1]
\noindent\textbf{I. PHẦN TRẮC NGHIỆM:}
\begin{ex}%[1T1B2-3]
        Tính tổng $S=\sin^25^{\circ}+\sin^210^{\circ}+\sin^215^{\circ}+ \cdots +\sin^285^{\circ}$.
        \choice
        {$S=\dfrac{19}{2}$}
        {\True $S=\dfrac{17}{2}$}
        {$S=8$}
        {$S=9$}
        \loigiai{
            \begin{align*}
                S&=\sin^25^{\circ}+\sin^210^{\circ}+\sin^215^{\circ}+ \cdots +\sin^285^{\circ}\\
                &=\left(\sin^25^{\circ}+\sin^285^{\circ}\right)+\left(\sin^210^{\circ}+\sin^280^{\circ}\right)+ \cdots +\left(\sin^240^{\circ}+\sin^250^{\circ}\right)+\sin^245^{\circ} \\
                &=8+\dfrac{1}{2}=\dfrac{17}{2}.
        \end{align*}}
    \end{ex}

\begin{ex}%[1C1Y1-2]
Cho góc lượng giác với tia đầu và tia cuối như trong hình. Tên của góc lượng giác là
    \begin{center}
        \begin{tikzpicture}[scale=1, font=\footnotesize, line join=round, line cap=round, >=stealth]
            \begin{axis}[
                axis line style={draw=none},
                axis lines=middle,
                axis equal image,
                enlargelimits,
                xtick=\empty,
                ytick=\empty,
                data cs=polar,
                samples=200,
                thick,
                line cap=round,
                line join=round,
                >=stealth
                ]
                \addplot [smooth, domain=0:390,->] {1+x/5000};
                \addplot [smooth, domain=420:450,->] {1.5+x/5000} node[above,midway]{$+$};
                \addplot [mark=none] (0.475,0) node [below left] {$O$};
                \addplot [mark=none] coordinates {(0,0) (390,3+390/5000)} node[below]{$y$};
                \addplot [mark=none] coordinates {(0,0) (0,3+0/5000)} node[below]{$x$};
                \addplot [mark=none,dashed] coordinates {(0,0) (420,3+420/5000)} node[right]{$m$};
            \end{axis}
        \end{tikzpicture}
    \end{center}
\choice
{\True $(Ox,Oy)$}
{$(Oy,Ox)$}
{$(Om,Oy)$}
{$(Om,Ox)$}
    \loigiai{
        Trong hình, góc lượng giác là $(Ox,Oy)$ với tia đầu $Ox$ và tia cuối $Oy$.
    }
    \end{ex}

\begin{ex}%[1T1B2-2]
        Cho $\tan a=\dfrac{2}{3}$, $5\pi <a<\dfrac{11\pi}{2}$. Khi đó $\cos \left(a+\dfrac{\pi}{3}\right)$ bằng
        \choice
        {$\dfrac{2\sqrt{3}+3}{2\sqrt{13}}$}
        {\True $\dfrac{2\sqrt{3}-3}{2\sqrt{13}}$}
        {$\dfrac{-2\sqrt{3}+3}{2\sqrt{13}}$}
        {$\dfrac{-2\sqrt{3}-3}{2\sqrt{13}}$}
        \loigiai{
            Ta có $\cos^2a=\dfrac{1}{1+\tan^2 a}=\dfrac{1}{1+\dfrac{4}{9}}=\dfrac{9}{13}$.\\
            Vì $5\pi <a<\dfrac{11\pi}{2}$ nên $\cos a<0$ và $\sin a<0$.\\
             Do đó, $\cos a=-\dfrac{3\sqrt{13}}{13}$ và $\sin a=-\dfrac{2\sqrt{13}}{13}$.\\
            Vậy $\cos \left(a+\dfrac{\pi}{3}\right)=\dfrac{1}{2}\cos a-\dfrac{\sqrt{3}}{2}\sin a=\dfrac{-3+2\sqrt{3}}{2\sqrt{13}}$.}
    \end{ex}

\begin{ex}%[Tex hóa SGK CD-CT,T12/22, TVN-006]%[1K1Y1-8]
    Trong các khẳng định sau, khẳng định  nào là \textbf{sai}?
    \choice
    {$\sin(\pi-\alpha)=\sin\alpha$}
    {\True $\cos(\pi-\alpha)=\cos \alpha$}
    {$\sin(\pi+\alpha)=-\sin\alpha$}
    {$\cos(\pi+\alpha)=-\cos \alpha$}
    \loigiai{
        Ta có $\cos(\pi-\alpha)=-\cos \alpha$ nên $\cos(\pi-\alpha)=\cos \alpha$ là khẳng định \textbf{sai}.
    }
\end{ex}

\begin{ex}%[1C1B1-2]
    Cho góc lượng giác gốc $O$ có tia đầu $Ou$, tia cuối $Ov$ và có số đo $\dfrac{2\pi}{3}$. Cho góc lượng giác $(O'u',O'v')$ có tia đầu $O'u'\equiv Ou$, tia cuối $O'v'\equiv Ov$. Viết công thức biểu thị số đo góc lượng giác $(O'u',O'v')$.
\choice
{$(O'u',Ov')=\dfrac{\pi}{3}+k2\pi\ (k\in \mathbb{Z})$}
{$(O'u',Ov')=\dfrac{4\pi}{3}+k2\pi\ (k\in \mathbb{Z})$}
{\True $(O'u',Ov')=\dfrac{2\pi}{3}+k2\pi\ (k\in \mathbb{Z})$}
{$(O'u',Ov')=-\dfrac{\pi}{3}+k2\pi\ (k\in \mathbb{Z})$}
    \loigiai{
        Ta có $(O'u',Ov')=(Ou,Ov)+k2\pi=\dfrac{2\pi}{3}+k2\pi\ (k\in \mathbb{Z})$.
    }
\end{ex}

\begin{ex}%[Tex hóa SGK CD-CT,T12/22, TVN-006]%[1K1B2-3]
    Rút gọn biểu thức $M=\cos(a+b)\cos(a-b)-\sin (a+b)\sin(a-b)$, ta được
    \choice
    {$M=\sin 4a$}
    {$M=1-2\cos^2a$}
    {\True $M=1-2\sin^2a$}
    {$M=\cos 4a$}
    \loigiai{
        Ta có
        \allowdisplaybreaks
        \begin{eqnarray*}
            M&=&\cos(a+b)\cos(a-b)-\sin (a+b)\sin(a-b)\\
            &=&\dfrac{1}{2}\left(\cos2a+\cos 2b\right)+\dfrac{1}{2}\left(\cos2a-\cos 2b\right)\\
            &=&\cos 2a\\
            &=&1-2\sin^2a.
        \end{eqnarray*}
    }
\end{ex}

\begin{ex}%[1T1B5-3]
        Tập nghiệm của phương trình $3\cos\left(3x-\dfrac{\pi}{3}\right)=0$ là
        \choice
        {$\left\{\dfrac{\pi}{2}+k\pi, k \in \mathbb{Z}\right\}$}
        {$\left\{\dfrac{5\pi}{6}+k 2\pi, k \in \mathbb{Z}\right\}$}
        {$\left\{\dfrac{5\pi}{18}+\dfrac{k 2\pi}{3}, k \in \mathbb{Z}\right\}$}
        {\True $\left\{\dfrac{5\pi}{18}+\dfrac{k\pi}{3}, k \in \mathbb{Z}\right\}$}
        \loigiai{
            $3\cos\left(3x-\dfrac{\pi}{3}\right)=0\Leftrightarrow 3x-\dfrac{\pi}{3}=\dfrac{\pi}{2}+k\pi\Leftrightarrow x=\dfrac{5\pi}{18}+\dfrac{k\pi}{3}, k\in\mathbb{Z}$.
            Tập nghiệm phương trình $S=\left\{\dfrac{5\pi}{18}+\dfrac{k\pi}{3}, k\in\mathbb{Z}\right\}$.
        }
    \end{ex}

\begin{ex}%[1T1B6-3]
        Phương trình $\sqrt{3}\sin x+\cos x=1$ tương đương với phương trình nào sau đây?
        \choice
        {$\cos \left( x+\dfrac{\pi}{6}\right) =\dfrac{1}{2}$}
        {$\sin \left( x+\dfrac{\pi}{3}\right) =\dfrac{1}{2}$}
        {\True $\cos \left( x-\dfrac{\pi}{3}\right) =\dfrac{1}{2}$}
        {$\sin \left( x-\dfrac{\pi}{6}\right) =\dfrac{1}{2}$}
        \loigiai{
            Chia hai vế của phương trình cho $2$, ta được
            \begin{eqnarray*}
                &\sqrt{3}\sin x+\cos x=1&\Leftrightarrow\dfrac{\sqrt{3}}{2}\sin x+\dfrac{1}{2}\cos x=\dfrac{1}{2}\\
                &&\Leftrightarrow\sin\dfrac{\pi}{3}\sin x+\cos\dfrac{\pi}{3}\cos x=\dfrac{1}{2}\\
                &&\Leftrightarrow\cos\left( x-\dfrac{\pi}{3}\right) =\dfrac{1}{2}.
            \end{eqnarray*}
        }
    \end{ex}

\begin{ex}%[1T1Y4-1]
        Tìm điều kiện xác định của hàm số  $y=\cot x$.
        \choice
        {$x \neq \dfrac{\pi}{4}+k \pi, k \in \mathbb{Z}$}
        { $x \neq k 2 \pi, k \in \mathbb{Z}$}
        {\True $x \neq k \pi, k \in \mathbb{Z}$}
        {$x\neq \dfrac{\pi}{2}+k \pi, k \in \mathbb{Z}$}
        \loigiai{
            Hàm số $y=\cot x$ xác định khi và chỉ khi $\sin x \ne 0 \Leftrightarrow x\neq k \pi, k \in \mathbb{Z} .$}
    \end{ex}

\begin{ex}%[1T1B4-2]
        Hàm số nào sau đây đồng biến trên khoảng $(0;\pi)$?
        \choice
        {\True $y=x^2$}
        {$y=\cos x$}
        {$y=\sin x$}
        {$y=\tan x$}
        \loigiai{
            Hàm số $y = x^2$ đồng biến khi $x > 0 \Rightarrow$ hàm số đồng biên trên khoảng $\left(0;\pi\right)$.}
    \end{ex}

\begin{ex}%[1C1B1-2]
    Cho góc lượng giác gốc $O$ có tia đầu $Ou$, tia cuối $Ov$ và có số đo $-\dfrac{5\pi}{6}$. Cho góc lượng giác $(O'u',O'v')$ có tia đầu $O'u'\equiv Ou$, tia cuối $O'v'\equiv Ov$. Viết công thức biểu thị số đo góc lượng giác $(O'u',O'v')$.
    \choice
    {$(O'u',Ov')=\dfrac{\pi}{6}+k2\pi\ (k\in \mathbb{Z})$}
    {$(O'u',Ov')=\dfrac{4\pi}{3}+k2\pi\ (k\in \mathbb{Z})$}
    {$(O'u',Ov')=-\dfrac{\pi}{6}+k2\pi\ (k\in \mathbb{Z})$}
    {\True $(O'u',Ov')=-\dfrac{5\pi}{6}+k2\pi\ (k\in \mathbb{Z})$}
    \loigiai{
        Ta có $(O'u',Ov')=(Ou,Ov)+k2\pi=-\dfrac{5\pi}{6}+k2\pi\ (k\in \mathbb{Z})$.
    }
\end{ex}

\begin{ex}%[1T1B4-6]
        Hình bên dưới là đồ thị của hàm số nào dưới đây?
        \begin{center}
            \begin{tikzpicture}[>=stealth,line join=round,line cap=round,font=\footnotesize,scale=0.7]
                \def\a{3.141592654}
                \draw[color=gray,dash pattern=on 1pt off 1pt,xstep=3.14cm,ystep=1.0cm] (-9.424,-3) grid (9.424,3);
                \draw[->] (-9.424,0) -- (9.7,0)node[below]{\scriptsize $x$};
                \draw[->] (0,-3.5) -- (0,3.5) node[left] {\scriptsize $y$};
                \draw (0,0)node[below left]{\scriptsize $O$};
                \clip (-9.58,-3.5)rectangle(9.58,3.5);
                \draw[samples=300,smooth,domain=-9.424:9.424] plot(\x,{-3*cos(\x*180/pi)});
                \path(-2*\a,0)node[shift={(-150:12pt)}]{$-2\pi$}
                (-\a,0)node[shift={(-150:12pt)}]{$-\pi$}
                (\a,0)node[shift={(-135:10pt)}]{$\pi$}
                (2*\a,0)node[shift={(-140:10pt)}]{$2\pi$}
                (0,3)node[shift={(-140:10pt)}]{$3$}
                (0,-3)node[shift={(-140:10pt)}]{$-3$};
                \foreach \x/\y in{-2*\a/0,-\a/0,\a/0,2*\a/0,-3*\a/3,3*\a/3,-2*\a/-3,2*\a/-3,\a/3,-\a/3,0/3,0/-3,0/0}\draw(\x,\y) circle (1pt);
            \end{tikzpicture}
        \end{center}
        \choice
        {\True $y=-3\cos x$}
        {$y=-2-\cos x$}
        {$y=2+|\cos x|$}
        {$y=\cos x-4$}
        \loigiai{
            \begin{itemize}
                \item $y(0)=-3\Rightarrow $ loại $y=\cos x-4$ và $y=2+|\cos x|$.
                \item $y(\pi)=3\Rightarrow $ loại $y=-2-\cos x$.
            \end{itemize}
        }
    \end{ex}

\begin{ex}%[1T1Y4-1]
        Điều kiện xác định của hàm số $y=\cot x$ là
        \choice
        { $x \ne \dfrac{\pi}{8}+k\dfrac{\pi}{2}$}
        {$x \ne \dfrac{\pi}{2}+k\pi$}
        {\True $x \ne k\pi$}
        {$x \ne \dfrac{\pi}{4}+k\pi$}
        \loigiai{
            Hàm số xác định khi và chỉ khi $\sin x \ne 0 \Leftrightarrow x \ne k\pi$, $k \in \mathbb{Z}$.
        }
    \end{ex}

\begin{ex}%[1T1B4-5]
        Cho hàm số $y=\sin^2x-\sin x+2$. Gọi $M,N$ lần lượt là GTLN và GTNN của hàm số đã cho. Khi đó $M+N$ bằng
        \choice
        {$k=-\dfrac{1}{2}$}
        {\True $\dfrac{23}{4}$}
        {$\dfrac{15}{4}$}
        {$6$}
        \loigiai{
            Ta có $y=\sin^2x-\sin x+2=\left(\sin x-\dfrac{1}{2}\right)^2+\dfrac{7}{4}$. \\
            Vì $-1 \leq \sin x \leq 1,\,\forall x \in \mathbb{R}$ nên $-\dfrac{3}{2} \leq \sin x-\dfrac{1}{2} \leq \dfrac{1}{2},\,\forall x \in \mathbb{R}$.\\
            Suy ra $0 \leq \left(\sin x-\dfrac{1}{2}\right)^2 \leq \dfrac{9}{4},\,\forall x \in \mathbb{R}$.\\
            Suy ra $\dfrac{7}{4} \leq \left(\sin x-\dfrac{1}{2}\right)^2+\dfrac{7}{4} \leq 4,\,\forall x \in \mathbb{R}$.\\
            Suy ra $\dfrac{7}{4} \leq y \leq 4,\,\forall x \in \mathbb{R}$.\\
            Vậy $M+N=\dfrac{7}{4}+4=\dfrac{23}{4}$.}
    \end{ex}

\begin{ex}%[Tex hóa SGK CD-CT,T12/22, TVN-006]%[1K1Y3-4]
    Trong các hàm số sau đây, hàm số nào là hàm tuần hoàn?
    \choice
    {$y=\tan x+x$}
    {$y=x^2+1$}
    {\True $y=\cot x$}
    {$y=\dfrac{\sin x}{x}$}
    \loigiai{
        Hàm số $y=\cot x$ là hàm số tuần hoàn với chu kỳ $T=\pi$.
    }
\end{ex}

\begin{ex}%[1T1Y1-1]
        Góc $18^\circ$ có số đo bằng rađian là bao nhiêu?
        \choice
        {$\pi$}
        {$\dfrac{\pi}{360}$}
        {\True $\dfrac{\pi}{10}$}
        {$\dfrac{\pi}{18}$}
        \loigiai{
            Ta có $18^\circ=\dfrac{\pi}{10}$ rad.
        }
    \end{ex}

\begin{ex}%[Tex hóa SGK CD-CT,T12/22, TVN-006]%[1K1Y1-4]
    Biểu diễn các góc lượng giác $\alpha=-\dfrac{5\pi}{6}$, $\beta=\dfrac{\pi}{3}$, $\gamma=\dfrac{25\pi}{3}$, $\delta=\dfrac{17\pi}{6}$ trên đường tròn lượng giác. Các góc nào có điểm biểu diễn trùng nhau?
    \choice
    {\True $\beta$ và $\gamma$}
    {$\alpha$, $\beta$, $\gamma$}
    {$\beta$, $\gamma$, $\delta$}
    {$\alpha$ và $\beta$}
    \loigiai{
        Ta có $\beta+8\pi=\dfrac{\pi}{3}+8\pi=\dfrac{25\pi}{3}=\gamma$.\\
        Do đó, $\beta$ và $\gamma$ có điểm biểu diễn trùng nhau trên đường tròn lượng giác.
    }
\end{ex}

\begin{ex}%[1C1B1-2]
    Cho góc lượng giác $(Ou,Ov)$ có số đo là $\dfrac{3\pi}{4}$, góc lượng giác $(Ou,Ow)$ có số đo là $\dfrac{5\pi}{4}$. Số đo của góc lượng giác $(Ov,Ow)$ là
\choice
{\True $(Ov,Ow)=\dfrac{\pi}{2}+k2\pi\ (k\in \mathbb{Z})$}
{$(Ov,Ow)=2\pi+k2\pi\ (k\in \mathbb{Z})$}
{$(Ov,Ow)=-\dfrac{\pi}{2}+k2\pi\ (k\in \mathbb{Z})$}
{$(Ov,Ow)=-\dfrac{\pi}{6}+k2\pi\ (k\in \mathbb{Z})$}
    \loigiai{
        Theo hệ thức Chasles, ta có
        \begin{eqnarray*}
            (Ov,Ow)&=&(Ou,Ow)-(Ou,Ov)+k2\pi\\
            &=&\dfrac{5\pi}{4}-\dfrac{3\pi}{4}+k2\pi\\
            &=&\dfrac{\pi}{2}+k2\pi\ (k\in \mathbb{Z}).
        \end{eqnarray*}
    }
\end{ex}

\begin{ex}%[1C1B1-2]
    Cho góc lượng giác gốc $O$ có tia đầu $Ou$, tia cuối $Ov$ và có số đo $45^\circ$. Cho góc lượng giác $(O'u',O'v')$ có tia đầu $O'u'\equiv Ou$, tia cuối $O'v'\equiv Ov$. Công thức biểu thị số đo góc lượng giác $(O'u',O'v')$ là
\choice
{$(O'u',Ov')=-45^\circ+k360^\circ\ (k\in \mathbb{Z})$}
{\True $(O'u',Ov')=45^\circ+k360^\circ\ (k\in \mathbb{Z})$}
{$(O'u',Ov')=135^\circ+k360^\circ\ (k\in \mathbb{Z})$}
{$(O'u',Ov')=-135^\circ+k360^\circ\ (k\in \mathbb{Z})$}
    \loigiai{
        Ta có $(O'u',Ov')=(Ou,Ov)+k360^\circ=45^\circ+k360^\circ\ (k\in \mathbb{Z})$.
    }
\end{ex}

\begin{ex}%[1T1B4-5]
        Hàm số $y=3-5\sin x$ có giá trị lớn nhất bằng
        \choice
        {$6$}
        {$2$}
        {\True $8$}
        {$4$}
        \loigiai{
            Ta có
            $$-1\le \sin x\le 1 \Leftrightarrow 5\ge-5\sin x\ge-5\Leftrightarrow  8\ge 3-5\sin x\ge -2\Rightarrow -2\le y\le 8.$$
            Suy ra giá trị lớn nhất của hàm số là $8$, đạt được khi $x=\dfrac{\pi}{2}+k2\pi,k\in\mathbb{Z}$.
        }
    \end{ex}

\begin{ex}%[1T1B2-4]
        Rút gọn biểu thức $M=\sin(\pi-a)+\tan\left(\dfrac{\pi}{2}-a\right)+\sin(-a)+\cot(\pi+a)$ được
        \choice
        {$M=2\cos a$}
        {$M=2\tan a$}
        {\True $M=2\cot a$}
        {$M=0$}
        \loigiai{
            Ta có $M=\sin a+\cot a-\sin a+\cot a=2\cot a$.
        }
    \end{ex}

\begin{ex}%[1T1Y4-6]
        Đồ thị hàm số $y=\cos x$ đi qua điểm nào sau đây?
        \choice
        {$P(-1;\pi)$}
        {$M(\pi;1)$}
        {$Q(3\pi; 1)$}
        {\True $N(0;1)$}
        \loigiai{
            Điểm $N(0;1)$ thuộc đồ thị hàm số.
        }
    \end{ex}

\begin{ex}%[1T1B4-1]
        Tập xác định của hàm số $y=2017\tan^{2018} \left( 2x+\dfrac{\pi}{3}\right)$ là
        \choice
        {\True $\mathscr{D}=\mathbb{R}\setminus\left\lbrace\dfrac{\pi}{12}+k\dfrac{\pi}{2}, k\in\mathbb{Z} \right\rbrace $}
        {$\mathscr{D}=\mathbb{R}\setminus\left\lbrace\dfrac{\pi}{2}+k\dfrac{\pi}{2}, k\in\mathbb{Z} \right\rbrace $}
        {$\mathscr{D}=\mathbb{R}\setminus\left\lbrace\dfrac{\pi}{2}+k\dfrac{\pi}{2}, k\in\mathbb{Z} \right\rbrace $}
        {$\mathscr{D}=\mathbb{R}\setminus\left\lbrace\dfrac{\pi}{2}+k\dfrac{\pi}{2}, k\in\mathbb{Z} \right\rbrace $}
        \loigiai{
            Hàm số xác định khi $2x+\dfrac{\pi}{3}\ne \dfrac{\pi}{2}+k\pi\Leftrightarrow x\ne\dfrac{\pi}{12}+k\dfrac{\pi}{2}, k\in\mathbb{Z}.$
        }
        \end{ex}

\begin{ex}%[1K1Y1-7]
        Tìm khẳng định đúng (với điều kiện các hệ thức đã xác định).
        \choice
        {$\cos \left(\pi -\alpha \right)=\cos \alpha$}
        {\True $\cos \left(-\alpha \right)=\cos \alpha$}
        {$\sin \left(\pi -\alpha \right)=-\sin \alpha$}
        {$\sin \left(-\alpha \right)=\sin \alpha$}
        \loigiai{
            Ta có
            \begin{itemize}
                \item $\sin \left(-\alpha \right)=-\sin \alpha$.
                \item $\cos \left(\pi -\alpha \right)=-\cos \alpha$.
                \item $\cos \left(-\alpha \right)=\cos \alpha$.
                \item $\sin \left(\pi -\alpha \right)=\sin \alpha$.
            \end{itemize}
        }
    \end{ex}


\noindent\textbf{II. PHẦN TỰ LUẬN:}
\begin{ex}%[Cánh Diều]%[1C1B4-3]
    Giải các phương trình
    \begin{multicols}{3}
        \begin{enumerate}[a)]
            \item $\sin x=-\dfrac{1}{2}$;
            \item $\sin x=\dfrac{\sqrt{2}}{2}$;
            \item $\sin3x=\sin2x$;
            \item $\sin x=\cos3x$;
            \item $\cos x=\dfrac{\sqrt{3}}{2}$;
            \item $\cos x=-\dfrac{\sqrt{2}}{2}$;
            \item $\cos x=-\dfrac{1}{2}$;
            \item $\cos3x=\cos\left(x+\dfrac{\pi}{3}\right)$;
            \item $\tan x=\dfrac{1}{\sqrt{3}}$;
            \item $\tan x=-1$;
            \item $\cot2x=-\sqrt{3}$.
        \end{enumerate}
    \end{multicols}
\loigiai{
\begin{enumerate}[a)]
    \item Do $\sin\left(-\dfrac{\pi}{6}\right)=-\dfrac{1}{2}$ nên $$\sin x=\sin\left(-\dfrac{\pi}{6}\right)\Leftrightarrow\hoac{&x=-\dfrac{\pi}{6}+k2\pi\\&x=\pi-\left(-\dfrac{\pi}{6}\right)+k2\pi}\Leftrightarrow\hoac{&x=-\dfrac{\pi}{6}+k2\pi\\&x=\dfrac{7\pi}{6}+k2\pi}\,(k\in\mathbb{Z}).$$
    \item Do $\sin\dfrac{\pi}{4}=\dfrac{\sqrt{2}}{2}$ nên $$\sin x=\sin\dfrac{\pi}{4}\Leftrightarrow\hoac{&x=\dfrac{\pi}{4}+k2\pi\\&x=\pi-\dfrac{\pi}{4}+k2\pi}\Leftrightarrow\hoac{&x=\dfrac{\pi}{4}+k2\pi\\&x=\dfrac{3\pi}{4}+k2\pi}\,(k\in\mathbb{Z}).$$
    \item $\sin3x=\sin2x\Leftrightarrow\hoac{&3x=2x+k2\pi\\&3x=\pi-2x+k2\pi}\Leftrightarrow\hoac{&x=k2\pi\\&x=\dfrac{\pi}{5}+k\dfrac{2\pi}{5}}\,(k\in\mathbb{Z})$.
    \item $\sin x=\cos3x\Leftrightarrow\sin x=\sin\left(\dfrac{\pi}{2}-3x\right)\Leftrightarrow\hoac{&x=\dfrac{\pi}{2}-3x+k2\pi\\&x=\pi-\left(\dfrac{\pi}{2}-3x\right)+k2\pi}\Leftrightarrow\hoac{&x=\dfrac{\pi}{8}+k\dfrac{\pi}{2}\\&x=-\dfrac{\pi}{4}+k\pi}\,(k\in\mathbb{Z})$.
    \item $\cos x=\dfrac{\sqrt{3}}{2}\Leftrightarrow\cos x=\cos\dfrac{\pi}{6}\Leftrightarrow x=\pm\dfrac{\pi}{6}+k2\pi,\,(k\in\mathbb{Z})$;
    \item $\cos x=-\dfrac{\sqrt{2}}{2}\Leftrightarrow \cos x=\cos\dfrac{3\pi}{4}\Leftrightarrow x=\pm\dfrac{3\pi}{4}+k2\pi,\,(k\in\mathbb{Z})$;
    \item $\cos x=-\dfrac{1}{2}\Leftrightarrow\cos x=\cos\dfrac{2\pi}{3}\Leftrightarrow x=\pm\dfrac{2\pi}{3}+k2\pi,\,(k\in\mathbb{Z})$;
    \item $\cos3x=\cos\left(x+\dfrac{\pi}{3}\right)\Leftrightarrow\hoac{&3x=x+\dfrac{\pi}{3}+k2\pi\\&3x=-x-\dfrac{\pi}{3}+k2\pi}\Leftrightarrow\hoac{&x=\dfrac{\pi}{6}+k\pi\\&x=-\dfrac{\pi}{12}+\dfrac{k\pi}{2}},\,(k\in\mathbb{Z})$;
    \item $\tan x=\dfrac{1}{\sqrt{3}}\Leftrightarrow\tan x=\tan\dfrac{\pi}{6}\Leftrightarrow x=\dfrac{\pi}{6}+k\pi,\,(k\in\mathbb{Z})$;
    \item $\tan x=-1\Leftrightarrow\tan x=\tan\left(-\dfrac{\pi}{4}\right)\Leftrightarrow x=-\dfrac{\pi}{4}+k\pi,\,(k\in\mathbb{Z})$;
    \item $\cot2x=-\sqrt{3}\Leftrightarrow\cot2x=\cot\left(-\dfrac{\pi}{6}\right)\Leftrightarrow x=-\dfrac{\pi}{6}+k\pi,\,(k\in\mathbb{Z})$.
\end{enumerate}}
\end{ex}

\begin{ex}%[1C1B4-3]
    Giải phương trình:
    \begin{listEX}[3]
        \item $\sin \left(2x-\dfrac{\pi}{3}\right)=-\dfrac{\sqrt{3}}{2}$;
        \item $\sin \left(3x+\dfrac{\pi}{4}\right)=-\dfrac{1}{2}$;
        \item $\cos \left(\dfrac{x}{2}+\dfrac{\pi}{4}\right) =\dfrac{\sqrt{3}}{2}$;
        \item $2\cos 3x+5=3$;
        \item $3\tan x=-\sqrt{3}$;
        \item $\cot x-3=\sqrt{3}\left(1-\cot x\right)$.
    \end{listEX}
    \loigiai{
        \begin{enumerate}[a)]
            \item Ta có 
            \begin{eqnarray*}
                &&\sin \left(2x-\dfrac{\pi}{3}\right)=-\dfrac{\sqrt{3}}{2}\\
                &\Leftrightarrow& \sin \left(2x-\dfrac{\pi}{3}\right) =\sin \left(-\dfrac{\pi}{3}\right)\\
                &\Leftrightarrow& \hoac{
                    &2x-\dfrac{\pi}{3} =-\dfrac{\pi}{3}+k2\pi\\
                    &2x-\dfrac{\pi}{3} = \pi+\dfrac{\pi}{3}+k2\pi}\\
                &\Leftrightarrow&
                \hoac{&2x=k2\pi\\&2x=\dfrac{5\pi}{3} +k2\pi}\\
                &\Leftrightarrow&
                \hoac{&x=k\pi\\ &x=\dfrac{5\pi}{6}+k\pi} (k\in \mathbb{Z}).
            \end{eqnarray*}         
            \item Ta có 
            \begin{eqnarray*}
                &&\sin \left(3x+\dfrac{\pi}{4}\right)=-\dfrac{1}{2} \\
                &\Leftrightarrow& \sin \left(3x+\dfrac{\pi}{4}\right) =\sin \left(-\dfrac{\pi}{6}\right)\\  
                &\Leftrightarrow& \hoac{
                    &3x+\dfrac{\pi}{4} = -\dfrac{\pi}{6}+k2\pi\\
                    &3x+\dfrac{\pi}{4}=\pi -\left(-\dfrac{\pi}{6}\right)+k2\pi} \\
                &\Leftrightarrow&
                \hoac{
                    &3x= -\dfrac{5\pi}{12}+k2\pi\\
                    &3x=\dfrac{11\pi}{12}+k2\pi} \\
                &\Leftrightarrow&
                \hoac{&x=-\dfrac{5}{36}+\dfrac{k2\pi}{3}\\
                    &x=\dfrac{11\pi}{36}+\dfrac{k2\pi}{3}} (k\in \mathbb{Z}).
            \end{eqnarray*}                     
            \item Ta có 
            \begin{eqnarray*}
                &&\cos \left(\dfrac{x}{2}+\dfrac{\pi}{4}\right) =\dfrac{\sqrt{3}}{2} \\ &\Leftrightarrow& \cos \left(\dfrac{x}{2}+\dfrac{\pi}{4}\right)=\cos \dfrac{\pi}{6}\\
                &\Leftrightarrow& \hoac{
                    &\dfrac{x}{2}+\dfrac{\pi}{4} = \dfrac{\pi}{6}+k2\pi\\
                    &\dfrac{x}{2}+\dfrac{\pi}{4}= -\dfrac{\pi}{6}+k2\pi}\\
                &\Leftrightarrow&
                \hoac{
                    &\dfrac{x}{2}=-\dfrac{\pi}{12}+k2\pi\\
                    &\dfrac{x}{2}=-\dfrac{5\pi}{12}+k2\pi}\\
                &\Leftrightarrow&
                \hoac{
                    &x=-\dfrac{\pi}{6}+k4\pi\\
                    &x=-\dfrac{5\pi}{6}+k4\pi} (k\in \mathbb{Z}).
            \end{eqnarray*} 
            \item Ta có $2\cos 3x+5=3 \Leftrightarrow \cos 3x =-1 \Leftrightarrow 3x=\pi+k2\pi \Leftrightarrow x = \dfrac{\pi}{3}+\dfrac{k2\pi}{3}\,\,(k\in \mathbb{Z})$.
            \item Ta có $3\tan x=-\sqrt{3} \Leftrightarrow \tan x =-\dfrac{\sqrt{3}}{3} \Leftrightarrow 
            \tan x=\tan \left(-\dfrac{\pi}{6}\right) \Leftrightarrow x=-\dfrac{\pi}{6}+k\pi\,\, (k\in \mathbb{Z}).$
            \item Ta có 
            \begin{eqnarray*}
                &&\cot x-3=\sqrt{3}\left(1-\cot x\right)\\
                &\Leftrightarrow& \cot x-3 =\sqrt{3}-\sqrt{3}\cot x\\
                &\Leftrightarrow& (1+\sqrt{3})\cot x=\sqrt{3}(1+\sqrt{3})\\
                &\Leftrightarrow& \cot x=\sqrt{3}\\
                &\Leftrightarrow& \cot x=\cot \dfrac{\pi}{6}\\
                &\Leftrightarrow& x=\dfrac{\pi}{6}+k\pi\,\, (k\in \mathbb{Z}).
            \end{eqnarray*} 
        \end{enumerate}
    }
\end{ex}

\begin{ex}%[1C1B4-3]
    Giải phương trình:
    \begin{listEX}[3]
        \item $\sin \left(2x+\dfrac{\pi}{4}\right)=\sin x$;
        \item $\sin 2x=\cos 3x$;
        \item $\cos^2 2x =\cos^2 \left(x+\dfrac{\pi}{6}\right)$.
    \end{listEX}
    \loigiai{
        \begin{enumerate}[a)]
            \item Ta có 
            \[\sin \left(2x+\dfrac{\pi}{4}\right)=\sin x
            \Leftrightarrow 
            \hoac{&2x+\dfrac{\pi}{4}=x+k2\pi\\&2x+\dfrac{\pi}{4}=\pi-x+k2\pi} 
            \Leftrightarrow \hoac{&x=-\dfrac{\pi}{4}+k2\pi\\&3x=-\dfrac{\pi}{4}+k2\pi} \Leftrightarrow \hoac{&x=-\dfrac{\pi}{4}+k2\pi\\&x=-\dfrac{\pi}{12}+\dfrac{k2\pi}{3}
            },\, (k\in \mathbb{Z}).\]
            \item Ta có 
            \begin{eqnarray*}
                \sin 2x=\cos 3x &\Leftrightarrow& \cos 3x =\cos \left(\dfrac{\pi}{2}-2x\right)\\
                &\Leftrightarrow& \hoac{
                    &3x=\dfrac{\pi}{2}-2x+k2\pi\\
                    &3x=\pi-\left(\dfrac{\pi}{2}-2x\right) +k2\pi}\\
                &\Leftrightarrow&
                \hoac{&5x=\dfrac{\pi}{2}+k2\pi\\&x=\dfrac{\pi}{2}+k2\pi}\\
                &\Leftrightarrow&
                \hoac{&x=\dfrac{\pi}{12}+\dfrac{k2\pi}{5}\\
                    &x=\dfrac{\pi}{2}+k2\pi}\, (k\in \mathbb{Z}).
            \end{eqnarray*} 
            \item Ta có $\cos^2 2x =\cos^2 \left(x+\dfrac{\pi}{6}\right) \Leftrightarrow 
            \hoac{
                &\cos 2x=\cos \left(x+\dfrac{\pi}{6}\right) &(1)\\
                &\cos 2x=-\cos \left(x+\dfrac{\pi}{6}\right). &(2)
            }$\\
            +) $(1) \Leftrightarrow \hoac{
                &2x=x+\dfrac{\pi}{6}+k2\pi\\
                &2x=-\left(x+\dfrac{\pi}{6}\right)+k2\pi
            } \Leftrightarrow
            \hoac{
                &x=\dfrac{\pi}{6}+k2\pi\\
                &3x=-\dfrac{\pi}{6}+k2\pi
            }
            \Leftrightarrow
            \hoac{
                &x=\dfrac{\pi}{6}+k2\pi\\
                &x=-\dfrac{\pi}{18}+\dfrac{k2\pi}{3}
            }(k\in \mathbb{Z})$.\\
            +) $(2) \Leftrightarrow 
            \cos 2x=\cos\left[\pi- \left(x+\dfrac{\pi}{6}\right) \right]
            \Leftrightarrow
            \hoac{
                &2x=\pi- \left(x+\dfrac{\pi}{6}\right)+k2\pi\\
                &2x=-\left[\pi- \left(x+\dfrac{\pi}{6}\right)\right]+k2\pi
            } $
            \[\Leftrightarrow
            \hoac{
                &3x=\dfrac{5\pi}{6}+k2\pi\\
                &x=-\dfrac{5\pi}{6}+k2\pi
            } \Leftrightarrow
            \hoac{
                &x=\dfrac{5\pi}{18}+\dfrac{k2\pi}{3}\\
                &x=-\dfrac{5\pi}{6}+k2\pi
            } \,(k\in\mathbb{Z}).\]
        \end{enumerate}
    }
\end{ex}

\begin{ex}Giải các phương trình sau
    \begin{listEX}[2]
        \item $2\sin x+\sqrt{2}=0$;
        \item $\sin2x-\cos x+2\sin x=1$;
        \item $3\sin ^2 x-5\sin x+2=0$;
        \item $\sqrt{3}\tan^2 x-2\tan x+\sqrt{3}=0$;
        \item $2\cos^2 2x-5\cos 2x+2=0$;
        \item $\sin^2\dfrac{x}{2}+\sin\dfrac{x}{2}-2=0$.
    \end{listEX}
    \loigiai{
        \begin{enumerate}[a)]
            \item $2\sin x+\sqrt{2}=0\Leftrightarrow\sin x=-\dfrac{\sqrt{2}}{2}\Leftrightarrow\sin x=\sin\left(-\dfrac{\pi}{4}\right)\Leftrightarrow\hoac{&x=-\dfrac{\pi}{4}+k2\pi\\&x=\dfrac{5\pi}{4}+k2\pi}\,(k\in\mathbb{Z})$;
            \item $\sin2x-\cos x+2\sin x=1\Leftrightarrow2\sin x\cos x-\cos x+2\sin x-1=0\Leftrightarrow(2\sin x-1)(\cos x+1)=0\\\Leftrightarrow\hoac{&\sin x=\dfrac{1}{2}\\&\cos x=-1}\Leftrightarrow\hoac{&\sin x=\sin\dfrac{\pi}{6}\\&x=(2k+1)\pi}\Leftrightarrow\hoac{&x=\dfrac{\pi}{6}+k2\pi\\&x=\dfrac{5\pi}{6}+k2\pi\\&x=(2k+1)\pi}\,(k\in\mathbb{Z})$;
            \item Đặt $t=\cos x$, $-1\le t\le 1$, phương trình đã cho trở thành $3t^2-5t+2=0$, ta được $t=1$ hoặc $t=\dfrac{2}{3}$.\\
            Với $t=1$ ta có $\cos x=1 \Leftrightarrow x= k2\pi,\, (k\in\mathbb{Z})$.\\
            Với $t=\dfrac{2}{3}$ ta có $\cos x=\dfrac{2}{3}=\cos\alpha\Leftrightarrow x=\pm \alpha +k2\pi,\, (k\in\mathbb{Z})$.\\
            Vậy tập nghiệm của phương trình đã cho là $S=\left\{ k2\pi, \pm \alpha+k2\pi, k\in\mathbb{Z} \right\}$.
            \item Đặt $t=\tan x$, phương trình đã cho trở thành $\sqrt{3} t^2-2t+\sqrt{3}=0$. Phương trình này vô nghiệm. Vậy phương trình đã cho vô nghiệm.
            \item $2\cos^2 2x-5\cos 2x+2=0\Leftrightarrow\hoac{&\cos2x=2\\&\cos2x=\dfrac{1}{2}}\Leftrightarrow\cos2x=\cos\dfrac{\pi}{3}\Leftrightarrow x=\pm\dfrac{\pi}{6}+k\pi,\, (k\in\mathbb{Z})$;
            \item $\sin^2\dfrac{x}{2}+\sin\dfrac{x}{2}-2=0\Leftrightarrow\hoac{&\sin\dfrac{x}{2}=1\\&\sin\dfrac{x}{2}=-2}\Leftrightarrow\dfrac{x}{2}=\dfrac{\pi}{2}+k2\pi\Leftrightarrow x=\pi+k4\pi,\, (k\in\mathbb{Z})$.
        \end{enumerate}
    }
\end{ex}

\begin{ex}%[1D1G1-5]
    Tìm giá trị lớn nhất và giá trị nhỏ nhất của hàm số  $y=2(\sin x+\cos x)+\sin 2 x+3$.
    \loigiai
    {Tập xác định $\mathscr{D}=\mathbb{R}$.\\
        Đặt $t=\sin x+ \cos x=\sqrt{2}\sin \left(x+\dfrac{\pi}{4}\right)$, $t\in \left[-\sqrt{2};\sqrt{2}\right]$.\\
        Ta có $t^2=\left(\sin x+ \cos x\right)^2=1+2\sin x\cos x=1+\sin 2x\Rightarrow \sin 2x =t^2-1$.\\
        Hàm số trở thành $y=g(t)=t^2+2t+2$. \\
        Bảng biến thiên của hàm số $y=g(t)$ trên đoạn $ \left[-\sqrt{2};\sqrt{2}\right]$
        \begin{center}
            \begin{tikzpicture}
                \tkzTabInit[lgt=1,espcl=3,deltacl=1]%nocadre,
                {$t$/1, $g(t)$ /2}
                {$-\sqrt{2}$ , $-1$ ,  $\sqrt{2}$}
                \tkzTabVar {+/$4-2\sqrt{2}$,-/$1$ ,+/ $4+2\sqrt{2}$}
            \end{tikzpicture}
        \end{center}
        Vậy $\max\limits_{x \in \mathbb{R}} y=4+2\sqrt{2}$ và $\min\limits_{x \in \mathbb{R}} y=1$.
    }
\end{ex}

\begin{ex}%[1D1K1-5]
    Tìm giá trị lớn nhất và giá trị nhỏ nhất của hàm số $y=\sqrt{3} \sin x-\cos x+5$.
    \loigiai
    {
        Tập xác định $\mathscr{D}=\mathbb{R}$.\\
        Biến đổi $y=\sqrt{3} \sin x-\cos x+5=2\left(\dfrac{\sqrt{3}}{2}\cdot\sin x-\dfrac{1}{2}\cdot\cos x\right)+5=2\sin\left(x-\dfrac{\pi}{6}\right)+5$.\\
        Với mọi $x\in \mathbb{R}$ ta có
        \allowdisplaybreaks
        \begin{eqnarray*}
            & & -1\leq \sin\left(x-\dfrac{\pi}{6}\right)\leq 1\\
            &\Leftrightarrow& -2\leq 2\sin\left(x-\dfrac{\pi}{6}\right)\leq 2\\
            &\Leftrightarrow&3\leq  2\sin\left(x-\dfrac{\pi}{6}\right)+5\leq 7.
        \end{eqnarray*}
        Vậy $\max\limits_{x \in \mathbb{R}} y=7$ khi $x=\dfrac{2\pi}{3}$ và $\min\limits_{x \in \mathbb{R}} y=3$ khi $x=-\dfrac{\pi}{3}$.
    }
\end{ex}
\Closesolutionfile{ans}
%Chương II
%%Bài 5. DS

\setcounter{section}{4}
\setcounter{dang}{0}
\setcounter{ex}{0}
\setcounter{bt}{0}
\setcounter{vd}{0}
\section{Dãy số}
\subsection{Tóm tắt lý thuyết}
\begin{tomtat}
	\subsubsection{Định nghĩa dãy số} 
	\begin{itemize}
		\item Mỗi hàm $u$ xác định trên tập các số nguyên dương $\mathbb{N^{*}}$ được gọi là một dãy vô hạn (gọi tắt là dãy số), kí hiệu $u=u(n)$.
		\item 	Ta thường viết $u_n$ thay cho $u(n)$ và kí hiệu dãy số $u=u(n)$ bởi $(u_n)$, do đó dãy số $(u_n)$ được viết dưới dạng khai triển $u_1, u_2, u_3, \ldots, u_n, \ldots$\\
		Số $u_1$ gọi là số hạng đầu, $u_n$ gọi là số hạng thứ $n$ và gọi là số hạng tổng quát của dãy số.
		\item Nếu $\forall n \in \mathrm{N^*}, u_n=c$ thì $(u_n)$ được gọi là dãy số không đổi.
		\item Mỗi hàm $u$ xác định trên tập $\mathrm{M}=\left\{1;2;3;\ldots;m\right\}, \forall m \in \mathrm{N^*}$ được gọi là một dãy số hữu hạn.
		\item Dạng khai triển của dãy hữu hạn là $u_1, u_2, u_3, \ldots, u_m$.\\
		Số $u_1$ gọi là số hạng đầu, số $u_m$ gọi là số hạng cuối.
		
	\end{itemize}
	\subsubsection{Các cách cho một dãy số}
	Một dãy số có thể cho bằng:
	\begin{itemize}
		\item Liệt kê các số hạng (chỉ dùng cho các dãy hữu hạn và có ít số hạng);
		\item Công thức của số hạng tổng quát;
		\item Phương pháp mô tả;
		\item Phương pháp truy hồi.
	\end{itemize}
	\subsubsection{Dãy số tăng, dãy số giảm, dãy số bị chặn}
	\begin{itemize}
		\item Dãy số $(u_n)$ được gọi là dãy số tăng nếu ta có $u_{n+1}>u_n, \forall n \in \mathrm{N^*}$.
		\item Dãy số $(u_n)$ được gọi là dãy số giảm nếu ta có $u_{n+1}<u_n, \forall n \in \mathrm{N^*}$.
		\item Dãy số $(u_n)$ được gọi là bị chặn trên nếu tồn tại số $M$ sao cho $u_n \le M, \forall n \in \mathrm{N^*}$.
		\item Dãy số $(u_n)$ được gọi là bị chặn dưới nếu tồn tại số $m$ sao cho $u_n \ge m, \forall n \in \mathrm{N^*}$.
		\item Dãy số $(u_n)$ được gọi là bị chặn nếu nó vừa bị chặn trên vừa bị chặn dưới, tức là  tồn tại các số $m, M$ sao cho $m \le u_n \le M, \forall n \in \mathrm{N^*}$.
	\end{itemize}
\end{tomtat}

\subsection{Các dạng toán thường gặp}
\begin{dang}{Số hạng tổng quát, biểu diễn dãy số}
	Để tìm số hạng tổng quát của một dãy bất kỳ khi biết một vài số hạng đầu của dãy số ta làm như sau
	\begin{itemize}
		\item Phân tích các số hạng sau theo các số hạng đã biết theo một quy luật nào đó.
		\item Dự đoán số hạng tổng quát 
		\item Kiểm tra bằng cách thay lần lượt các giá trị $n\in \mathrm{N^*}$ vào công thức tổng quát (Chứng minh bằng phương pháp quy nạp).
	\end{itemize}
	Để biểu diễn một dãy số khi biết công thức tổng quát ta lần lượt thay $n\in \mathrm{N^*}$ vào công thức tổng quát để tìm các số hạng thứ nhất, thứ hai, $\ldots$
\end{dang}
\subsubsection{Ví dụ minh hoạ}
\begin{vd}[NB]%[DCHT Toán 11 - KNTT -Tên GV]%[1K2Y1-1]
	Xác định số hạng đầu và số hạng tổng quát của dãy số $(u_n)$ các số tự nhiên lẻ $1, 3, 5, 7, \ldots $
	\dapso{$u_n=2n-1$}	
	\loigiai{Dãy $(u_n)$ có số hạng đầu $u_1=1$ và số hạng tổng quát $u_n=2n-1$.}
\end{vd}
\begin{vd}[NB]%[DCHT Toán 11 - KNTT -Tên GV]%[1K2Y1-1]
	Xác định số hạng đầu và số hạng tổng quát của dãy số $(v_n)$ các số nguyên dương chia hết cho $5$: $5,10,15,20,\ldots$
	\dapso{$v_n=5n$}
	\loigiai{Dãy $(v_n)$ có số hạng đầu $v_1=5$ và số hạng tổng quát $v_n=5n$.}
\end{vd}
% \begin{vd}[NB]%[DCHT Toán 11 - KNTT -Tên GV]%[1K2Y1-1]
% 	Viết năm số hạng đầu và số hạng thứ $100$ của dãy số $(u_n)$ có số hạng tổng quát $u_n=3n-2$.
% 	\dapso{$u_{100}=298$}
% 	\loigiai{Năm số hạng đầu của dãy số là $1,4,7,10,13$.\\
% 		Số hạng thứ $100$ của dãy là $u_{100}=3\cdot100-2=298$.}
% \end{vd}
% \begin{vd}[NB]%[DCHT Toán 11 - KNTT -Tên GV]%[1K2Y1-1]
% 	Cho dãy số xác định bằng hệ thức truy hồi: $u_1=1, u_n=3u_{n-1}+2$ với $n\ge 2$. Viết ba số hạng đầu của dãy số này.
% 	\dapso{$u_1=1, u_2=5, u_3=17$}
% 	\loigiai{Ta có $u_1=1, u_2=3u_1+2=5, u_3=3u_2+2=17$.}
% \end{vd}
% \begin{vd}[NB]%[DCHT Toán 11 - KNTT -Tên GV]%[1K2Y1-1]
% 	Dãy số $(u_n)$ cho bởi hệ thức truy hồi: $u_1=1, u_n=n \cdot u_{n-1}$ với $n \ge 2$. Viết năm số hạng đầu của dãy số và dự đoán công thức tổng quát $u_n$.
% 	\dapso{$u_n=n!$}
% 	\loigiai{Năm số hạng đầu của dãy là
% 		$u_1=1, u_2=2\cdot u_1=2, u_3=3\cdot u_2=6, u_4=4 \cdot u_3= 24, u_5=5 \cdot u_4=124$.\\
% 		Số hạng tổng quát\\
% 		Ta có $ u_2=2\cdot 1, u_3=6=3\cdot 2\cdot 1, u_4=24=4\cdot 3\cdot 2\cdot 1, u_5=124= 5\cdot4\cdot3\cdot2\cdot1  $.\\
% 		Vậy số hạng tổng quát $u_n=n!$.}
% \end{vd}
\subsubsection{Bài tập tự luận}
 
\begin{bt}[NB]%[DCHT Toán 11 - KNTT -Tên GV]%[1K2Y1-1]
	Xét dãy số hữu hạn gồm các số tự nhiên lẻ nhỏ hơn 20, sắp xếp theo thứ tự từ bé đến lớn. Liệt kê tất cả các số hạng của dãy số này, tìm số hạng đầu và số hạng cuối của dãy. 
	\dapso{$u_1=1$, $u_{11}=19$}
	\loigiai{Các số hạng của dãy là $1,3,5,7,9,10,11,13,15,17,19$.\\
		Số hạng đầu của dãy là $u_1=1$.\\
		Số hạng cuối của dãy là $u_{11}=19$.}
\end{bt}
\begin{bt}[TH]%[DCHT Toán 11 - KNTT -Tên GV]%[1K2Y1-1]
	Xét dãy số gồm tất cả các số tự nhiên chia cho $5$ dư $1$. Xác định số hạng tổng quát của dãy số.
	\dapso{$u_n=5n+1$}
	\loigiai{Các số tự nhiên chia cho $5$ dư $1$ gồm các số sau:
		$6,11,16,21, \ldots $\\
		Số hạng tổng quát $u_n=5n+1$.}
\end{bt}
% \begin{bt}[NB]%[DCHT Toán 11 - KNTT -Tên GV] %[1K2Y1-1]
% 	Tìm năm số hạng đầu và số hạng thứ $100$ của dãy $(u_n)$ có số hạng tổng quát $u_n= \dfrac{(-1)^n}{n}$.
% 	\dapso{$u_1=-1, \dfrac{1}{2},-\dfrac{1}{3}, \dfrac{1}{4}, -\dfrac{1}{5} $, $u_{100}=\dfrac{1}{100}$}
% 	\loigiai{
% 		Năm số hạng đầu của dãy là $u_1=-1, \dfrac{1}{2},-\dfrac{1}{3}, \dfrac{1}{4}, -\dfrac{1}{5} $.\\
% 		Số hạng thứ $100$ là $u_{100}=\dfrac{1}{100}$.}
% \end{bt}
\begin{bt}[NB]%[DCHT Toán 11 - KNTT -Tên GV]%[1K2Y1-1]
	Viết năm số hạng đầu của dãy số gồm các số nguyên tố theo thứ tự tăng dần.
	\dapso{$2,3,5,7,11$}
	\loigiai {Năm số hạng đầu của dãy số trên là $2,3,5,7,11$.} 
\end{bt}
% \begin{bt}[NB]%[DCHT Toán 11 - KNTT-Tên GV]%[1K2Y1-1]
% 	Viết năm số hạng đầu của dãy $(u_n)$ với số hạng tổng quát là $u_n=n!$.
% 	\dapso{$1,2,6,24,120$}
% 	\loigiai{Năm số hạng đầu của dãy trên là $1,2,6,24,120$.}
% \end{bt}
\subsubsection{Câu hỏi trắc nghiệm}
\Opensolutionfile{ans}[ans/ans-1K2-1-Dang1]
%Cau1
\begin{ex}%[DCHT Toán 11 - KNTT -Tên GV]%[1K2B1-1]
	Cho dãy số có các số hạng đầu là $5,10,15,20,25, \ldots$ Số hạng tổng quát của dãy số này là
	\choice
	{$ u_n=5(n-1) $}
	{\True$ u_n=5n $}
	{$ u_n=5+n $}
	{$ u_n=5n+1 $}
	\loigiai
	{Ta có $5=5\cdot 1, 10=5 \cdot 2, 15 = 5\cdot 3, 20=5 \cdot 4, 25 = 5\cdot 5, \ldots$\\
		Vậy dãy trên có số hạng tổng quát là $u_n=5n$.
	}
\end{ex}
%Cau2
\begin{ex}%[DCHT Toán 11 - KNTT -Tên GV]%[1K2B1-1]
	Cho dãy số $(u_n)$ với $u_n=\dfrac{an^2}{n+1}$, $a$ là hằng số. $u_{n+1}$ là số hạng nào trong các số hạng sau
	\choice
	{\True $u_{n+1}=\dfrac{a(n+1)^2}{n+2} $}
	{$u_{n+1}=\dfrac{a(n+1)^2}{n+1}$}
	{$u_{n+1}=\dfrac{an^2+1}{n+1}$}
	{$u_{n+1}=\dfrac{an^2}{n+2} $}
	\loigiai
	{Ta có $u_{n+1}=\dfrac{a(n+1)^2}{n+1+1}=\dfrac{a(n+1)^2}{n+2}$.
	}
\end{ex}
%cau3
\begin{ex}%[DCHT Toán 11 - KNTT -Tên GV]%[1K2B1-1]
	Cho dãy số có các số hạng đầu là $8,15,22,29,36, \ldots$ Số hạng tổng quát của dãy số này là
	\choice
	{$ u_n=7n+7 $}
	{$ u_n=7n $}
	{\True $ u_n=7n+1 $}
	{$ u_n$ không viết được dưới dạng công thức }
	\loigiai
	{Ta có $8=7\cdot 1+1, 15=7 \cdot 2+1, 22 = 7\cdot 3+1, 29=7 \cdot 4+1, 36 = 7\cdot 5+1, \ldots$\\
		Vậy dãy trên có số hạng tổng quát là $u_n=7n+1$.
	}
\end{ex}
%Cau4
\begin{ex}%[DCHT Toán 11 - KNTT -Tên GV]%[1K2B1-1]
	Cho dãy số có các số hạng đầu là $0,\dfrac{1}{2},\dfrac{2}{3},\dfrac{3}{4},\dfrac{4}{5}, \ldots$ Số hạng tổng quát của dãy số này là
	\choice
	{$ u_n=\dfrac{n+1}{n}$}
	{\True $ u_n=\dfrac{n}{n+1} $}
	{$ u_n=\dfrac{n-1}{n}$}
	{$ u_n=\dfrac{n^2-n}{n+1}$  }
	\loigiai
	{Ta có $0=\dfrac{0}{0+1}, \dfrac{1}{2}=\dfrac{1}{1+1} ,\dfrac{2}{3} = \dfrac{2}{2+1}, \dfrac{3}{4}=\dfrac{3}{3+1}, \dfrac{4}{5} = \dfrac{4}{4+1}, \ldots$\\
		Vậy dãy trên có số hạng tổng quát là $u_n=\dfrac{n}{n+1}$.
	}
\end{ex}
%Cau5
\begin{ex}%[DCHT Toán 11 - KNTT -Tên GV]%[1K2B1-1]
	Cho dãy số $(u_n)$ với $u_1=1, u_{n+1}=u_n-1$. Số hạng tổng quát $u_n$ của dãy số là số hạng nào dưới đây?
	\choice
	{\True $ u_n=2-n$}
	{$ u_n$ không xác định}
	{$ u_n=1-n$}
	{$ u_n=-n$, với mọi $n$ }
	\loigiai
	{Ta có $u_1=1, u_2=0 ,u_3 = -1, u_4=-2,  \ldots$\\
		Dễ dàng dự đoán được số hạng tổng quát là $u_n=2-n$.
	}
\end{ex}
% %cau6
% \begin{ex}%[DCHT Toán 11 - KNTT -Tên GV]%[1K2B1-1]
% 	Cho dãy số $(u_n)$ với $u_n=\dfrac{2n^2-1}{n^2+3}, \forall n \in \mathrm{N*}$. Số hạng đầu tiên của dãy số là 
% 	\choice
% 	{$ u_1=-\dfrac{1}{3}$}
% 	{$ u_1=\dfrac{2}{3}$}
% 	{$ u_1=\dfrac{1}{3}$}
% 	{\True $ u_1=\dfrac{1}{4}$ }
% 	\loigiai
% 	{Ta có $u_1=\dfrac{2\cdot 1^2-1}{1^2+3}=\dfrac{1}{4}$.
% 	}
% \end{ex}
% %cau7
% \begin{ex}%[DCHT Toán 11 - KNTT -Tên GV]%[1K2B1-1]
% 	Cho dãy số $(u_n)$ với $u_1=-1, u_{n+1}=u_n+3$ với $n \ge 1$. Ba số hạng đầu tiên của dãy số lần lượt là 
% 	\choice
% 	{\True $-1, 2, 5$}
% 	{$ 1, 4, 7$}
% 	{$ 4,7,10$}
% 	{$-1,3,7$ }
% 	\loigiai
% 	{Ta có $u_1=-1, u_2=-1+3=2 ,u_3 = 2+3=5$.
% 	}
% \end{ex}
\Closesolutionfile{ans}
\begin{indapan}{10}
	{ans/ans-1K2-1-Dang1}
\end{indapan}

\begin{dang}{Tìm số hạng cụ thể của dãy số}
	Để tìm số hạng cụ thể của dãy số ta làm như sau
	\begin{itemize} 
		\item Với trường hợp dãy số đã cho biết công thức tổng quát của dãy số thì ta chỉ cần thay giá trị tương ứng của số hạng đó vào công thức tổng quát.
		\item  Với trường hợp dãy số cho bởi công thức truy hồi hoặc dưới dạng thì ta phải tìm lần lượt từ những số hạng đầu tiên cho đến số đứng trước số cần tìm trong dãy.
	\end{itemize}
\end{dang}
\subsubsection{Ví dụ minh hoạ}
\begin{vd}[NB]%[1K2Y5-2]
	Cho dãy số $(u_n),$ biết $u_n=(-1 )^n\cdot \dfrac{2^n}{n}$. Tìm số hạng $u_3$.
	\dapso{$u_3=-\dfrac{8}{3}$}
	\choice
	{\True $u_3=-\dfrac{8}{3}$}
	{$u_3=2$}
	{$u_3=-2$}
	{$u_3=\dfrac{8}{3}$}
	\loigiai{
		Ta có
		$$u_3=(-1)^3\cdot \dfrac{2^3}{3}=-\dfrac{8}{3}.$$}
\end{vd}
\begin{vd}[NB]%[DCHT Toán 11 - KNTT -Nguyễn Long]%[1K2Y5-2]
	Cho dãy số $(u_n)$, biết $u_n=\dfrac{2n^2-1}{n^2+3}$. Tìm số hạng $u_5$.
	\dapso{$u_5=\dfrac{7}{4}$}
	\choice
	{$u_5=\dfrac{1}{4}$}
	{\True $u_5=\dfrac{7}{4}$}
	{$u_5=\dfrac{17}{12}$}
	{$u_5=\dfrac{71}{39}$}
	\loigiai{
		Ta có $u_5=\dfrac{2\cdot 5^2-1}{5^2+3}=\dfrac{49}{28}=\dfrac{7}{4}$.}
	
\end{vd}
\begin{vd}[NB]%[1K2Y5-2]
	Cho dãy số $u_n$ bao gồm các số nguyên tố. Tìm số hạng thứ $5$ của dãy số.
	\dapso{$u_5=11$}
	\loigiai{ 
		Ta có
		$u_1=2,u_2=3,u_3=5,u_4=7,u_5=11$. \\
		Vậy số hạng thứ $5$ của dãy số là $11$.
	}
\end{vd}
\begin{vd}[NB]%[1K2Y5-2]
	Cho dãy số $(u_n) $ thỏa mãn $ \heva{& u_1 = 5 \\& u_{n+1} = u_n+n}$. Tìm số hạng thứ $5$ của dãy số.
	\dapso{$u_5=15$}
	\choice
	{$ 11 $}
	{\True $ 15 $}
	{$ 16 $}
	{$ 12 $}
	\loigiai{
		Ta có $ u_2=u_1+1=6$, $ u_3=u_2+2=8$, $ u_4=u_3+3=11$,  $ u_5=u_4+4=15$.
	}
\end{vd}

\begin{vd}[TH]%[VD 5 SGK KNTT]%[1K2B5-2]
	Cho dãy số xác định bằng hệ thức truy hồi
	$$
	u_1=1, u_n=3 u_{n-1}+2 \text { với } n \geq 2
	$$
	Viết ba số hạng đầu của dãy số này.
	\dapso{$u_5=17$}
	\loigiai{
		Ta có: $u_1=1, u_2=3 u_1+2=3 \cdot 1+2=5, u_3=3 u_2+2=3 \cdot 5+2=17$.
	}
\end{vd}

\begin{vd}[VD]%[1K2B5-2]
	Cho dãy số $\left(u_n\right)\colon\heva{&u_1=5 \\ &u_{n+1}=u_n+n}$. Số $20$ là số hạng thứ mấy trong dãy?
	\dapso{số hạng thứ $6$}
	\loigiai{
		Ta có $u_1=5, u_2=6, u_3=8, u_4=11, u_5=16, u_6=20$.\\
		Vậy số $20$ là số hạng thứ $6$.}
\end{vd}

\subsubsection{Bài tập tự luận}
 
\begin{bt}[NB]%[1K2Y5-2]
	Cho dãy số $u_n=\dfrac{1}{\sqrt{n}+1}$. Tìm số hạng $u_4$.	
	\dapso{$u_4=\dfrac{1}{3}$}
	\loigiai{ Ta có
		$u_4=\dfrac{1}{\sqrt{4}}+1=\dfrac{1}{3}.$		
	}
\end{bt}
%%%
\begin{bt}[NB]%[1K2Y5-2]
	Cho dãy số $(u_n)$ có số hạng tổng quát: $u_n=2 n+\sqrt{n^2+4}$. Tìm số hạng thứ $6$ của dãy số.
	\dapso{$u_6=12+2\sqrt{10}$}
	\loigiai{
		Ta có $u_6=12+2 \sqrt{10}$.
	}
\end{bt}
%%%
\begin{bt}[NB]%[1K2Y5-2]
	Cho dãy số $(u_n)$ xác định bởi: $\heva{&u_1=-1 ; u_2=3 \\&u_{n+1}=5 u_n-6 u_{n-1} \forall n \geq 2}.$ Tìm số hạng thứ $7$ của dãy.
	\dapso{$3261$}
	\loigiai{
		Ta có
		$$
		u_3=5 u_2-6 u_1=21 ;~ u_4=5 u_3-6 u_2=87 ;~ u_5=309 ;~ u_6=1023 ;~ u_7=3261
		$$
		Vậy số hạng thứ $7$ của dãy là $3261$.
	}
\end{bt}
%%%%%%%
\begin{bt}[NB]%[1K2Y5-2]
	Viết năm số hạng đầu của dãy số Fibonacci $\left(F_n\right)$ cho bởi hệ thức truy hồi
	$$
	\heva{
		&F_1=1, F_2=1 \\
		&F_n=F_{n-1}+F_{n-2}~(n \geq 3) .
	}
	$$
	\dapso{$F_3=2,~F_4=3,~F_5=5$}
	\loigiai{
		Ta có $F_3=2,~F_4=3,~F_5=5.$
	}
\end{bt}
%%%
\begin{bt}[NB]%[1K2T5-2]
	Người ta nuôi cấy $5$ con vi khuẩn E-coli trong môi trường nhân tạo. Cứ $30$ phút thì vi khuẩn E-coli sẽ nhân đôi 1 lần. Tính số lượng vi khuẩn thu được sau $1,2,3$ lần nhân đôi.
	\dapso{$u_2=10, u_3=20, u_4=40$}
	\loigiai{
		Đặt $u_1=5$, gọi số vi khuẩn sau $n$ lần phân chia là $u_{n+1}$, khi đó ta có dãy số $(u_n)$ thỏa mãn $$u_1=5, \; u_{n+1}=2u_n$$
		Ta có $u_2=10, u_3=20, u_4=40$.
	}	
\end{bt}
%%%%%
\begin{bt}[TH]%[1K2B5-2]
	Cho dãy số $(u_n)$ được xác định bởi $u_n=\dfrac{n^2+3n+7}{n+1}$.
	\begin{listEX}
		\item Viết năm số hạng đầu của dãy.
		\item Dãy số có bao nhiêu số hạng nhận giá trị nguyên.
	\end{listEX}
	\dapso{$u_1=\dfrac{11}{2}$; $u_2=\dfrac{17}{3}$; $u_3=\dfrac{25}{4}$; $u_4=7$; $u_5=\dfrac{47}{6}$. $u_4=7 $}
	\loigiai{		
		\begin{listEX}
			\item Ta có năm số hạng đầu của dãy
			$u_1=\dfrac{1^2+3.1+7}{1+1}=\dfrac{11}{2}$; $u_2=\dfrac{17}{3}$; $u_3=\dfrac{25}{4}$; $u_4=7$; $u_5=\dfrac{47}{6}$.
			\item Ta có: $u_n=n+2+\dfrac{5}{n+1}$, do đó $u_n$ nguyên khi và chỉ khi $ \dfrac{5}{n+1}$ nguyên hay $ n+1 $ là ước của 5. Điều đó xảy ra khi $ n+1=5\Leftrightarrow n=4 $. Vậy dãy số có duy nhất một số hạng nguyên là $u_4=7 $.
		\end{listEX}		
	}
\end{bt}
\begin{bt}[VD]%[1K2K5-2]
	Cho dãy số $\left(x_n\right)$ thỏa mãn điều kiện $x_1=1, x_{n+1}-x_n=\dfrac{1}{n(n+1)}, n=1,2,3, \ldots$. Số hạng $x_{2023}$ bằng
	\dapso{$x_{2023}=\dfrac{4045}{2023}$}
	\loigiai{
		Ta có
		$$
		\begin{aligned}
			x_{n+1}-x_n=\dfrac{1}{n(n+1)}=\dfrac{1}{n}-\dfrac{1}{n+1} & \Leftrightarrow \sum_{k=1}^{n-1}\left(x_{k+1}-x_k\right)=\sum_{k=1}^{n-1}\left(\dfrac{1}{k}-\dfrac{1}{k+1}\right) \\
			& \Leftrightarrow x_n-x_1=1-\dfrac{1}{n} \\
			& \Leftrightarrow x_n=\dfrac{2n-1}{n} .
		\end{aligned}
		$$
	}
\end{bt}
\begin{bt}[VDC]%[1K2G5-2]
	Cho dãy số $\left(u_n\right)$ biết $\heva{&u_1=99 \\&u_{n+1}=u_n-2 n-1, n \geq 1}$. Hỏi số $-861$ là số hạng thứ mấy?
	\dapso{$-861$ là số hạng thứ $31$}
	\loigiai{
		Ta có
		$$
		\begin{aligned}
			&u_n &=& &u_{n-1}-2 n+1 \\
			&u_{n-1} & = & &u_{n-2}-2 n+3 \\
			&\vdots &\vdots&  &\vdots \\
			&u_3 & = & &u_2-2 n+2 n-5 \\
			&u_2 & = & &u_1-2 n+2 n-3
		\end{aligned}
		$$
		Suy ra
		$$
		\begin{aligned}
			& u_n=u_1-2 n \cdot(n-1)+1+3+5+\cdots+(2 n-5)+(2 n-3) \\
			& u_n=99-2 n^2+2 n+\dfrac{n-1}{2}\cdot[2 \cdot 1+(n-2) \cdot 2]=100-n^2
		\end{aligned}
		$$
		Giả sử $u_n=-861 \Rightarrow n^2=961 \Rightarrow n=31$ (vì $n \in \mathbb{N}$).
		Vậy số $-861$ là số hạng thứ $31$ .}
\end{bt}
\subsubsection{Câu hỏi trắc nghiệm}
\Opensolutionfile{ans}[ans/ans-1K2-1-Dang2]
%Câu 1
\begin{ex}%[1K2Y5-2]
	Cho dãy số $({{u}_{n}} )$, biết ${{u}_{n}}=\dfrac{n}{{{3}^{n}}-1}$. Ba số hạng đầu tiên của dãy số đó lần lượt là những số nào dưới đây?
	\choice
	{$\dfrac{1}{2};\dfrac{1}{4};\dfrac{1}{16}$}
	{$\dfrac{1}{2};\dfrac{2}{3};\dfrac{3}{4}$}
	{ \True $\dfrac{1}{2};\dfrac{1}{4};\dfrac{3}{26}$}
	{$\dfrac{1}{2};\dfrac{1}{4};\dfrac{1}{8}$}
	\loigiai {
		Ta có
		${{u}_{1}}=\dfrac{1}{2};\,\,{{u}_{2}}=\dfrac{2}{{{3}^2}-1}=\dfrac{2}{8}=\dfrac{1}{4};\,\,{{u}_{3}}=\dfrac{3}{{{3}^3}-1}=\dfrac{3}{26}$.}
	
\end{ex}
%%%%%%%%%%
%Câu 2
\begin{ex}%[1K2Y5-2]
	Cho dãy số $({{u}_{n}} ),$ biết ${{u}_{n}}={{(-1 )}^{n}}\cdot 2n$. Mệnh đề nào sau đây {\bf sai}?
	\choice
	{${{u}_{3}}=-6$}
	{${{u}_{2}}=4$}
	{ \True ${{u}_{4}}=-8$}
	{${{u}_{1}}=-2$}
	\loigiai {
		Ta có\\
		${{u}_{1}}=-2\cdot 1=-2;\,\,{{u}_{2}}={{(-1 )}^2}\cdot 2\cdot 2=4,\,\,{{u}_{3}}={{(-1 )}^3}\cdot 2\cdot 3=-6;\,\,{{u}_{4}}={{(-1 )}^4}\cdot 2\cdot 4=8$.\\
		\textbf{Nhận xét:} Dễ thấy ${{u}_{n}}>0$ khi $n$ chẵn và ngược lại nên đáp án $u_4=-8$ sai.}
	
\end{ex}
%%%%%%%%%%
%Câu 3
\begin{ex}%[1K2Y5-2]
	Cho dãy số $({{u}_{n}} )$ xác định bởi $\heva{
		& {{u}_{1}}=2 \\
		& {{u}_{n+1}}=\dfrac{1}{3}({{u}_{n}}+1 ) \\
	}.$ Tìm số hạng ${{u}_{4}}$.
	\choice
	{${{u}_{4}}=\dfrac{2}{3}$}
	{${{u}_{4}}=1$}
	{${{u}_{4}}=\dfrac{14}{27}$}
	{ \True ${{u}_{4}}=\dfrac{5}{9}$}
	\loigiai {
		Ta có
		${{u}_{2}}=\dfrac{1}{3}({{u}_{1}}+1 )=\dfrac{1}{3}(2+1 )=1;\,\,{{u}_{3}}=\dfrac{1}{3}({{u}_{2}}+1 )=\dfrac{2}{3};\,\,{{u}_{4}}=\dfrac{1}{3}({{u}_{3}}+1 )=\dfrac{1}{3}\cdot\left(\dfrac{2}{3}+1\right)=\dfrac{5}{9}$. \\
	}
\end{ex}
%%%%%%%%%%
%Câu 4
\begin{ex}%[1K2Y5-2]
	Cho dãy số $({{u}_{n}} )$, biết $\heva{
		& {{u}_{1}}=-1 \\
		& {{u}_{n+1}}={{u}_{n}}+3 \\
	}$ với $n\ge 0$. Ba số hạng đầu tiên của dãy số đó là lần lượt là những số nào dưới đây?
	\choice
	{\True $-1;\,2;\,5$}
	{$-1;3;7$}
	{$1;\,4;\,7$}
	{$4;\,7;\,10$}
	\loigiai {
		Ta có ${{u}_{1}}=-1;\,\,{{u}_{2}}={{u}_{1}}+3=2;\,\,{{u}_{3}}={{u}_{2}}+3=5$. \\
		\textbf{Nhận xét.} (i) Dùng chức năng “lặp” của MTCT để tính:\\
		Nhập vào màn hình: $X=X+3$ \\
		Bấm CALC và cho $X=-1$ (ứng với ${{u}_{1}}=-1)$ \\
		Để tính ${{u}_{n}}$ cần bấm “=” ra kết quả liên tiếp $n-1$ lần. Ví dụ để tính ${{u}_{2}}$ ta bấm “=” ra kết quả lần đầu tiên, bấm “=” ra kết quả thứ hai chính là ${{u}_{3}},\ldots$\\
		(ii) Vì ${{u}_{1}}=-1$ nên loại các đáp án $u_1=1$, $u_1=4$.\\
		Còn lại các đáp án có $u_1=-1$; để biết đáp án nào ta chỉ cần kiểm tra ${{u}_{2}}$ (vì ${{u}_{2}}$ ở hai đáp án là khác nhau): ${{u}_{2}}={{u}_{1}}+3=2$.
	}
	
\end{ex}
%%%%%%%%%%
%Câu 5
\begin{ex}%[1K2B5-2]
	Cho dãy số $({{u}_{n}} ),$ biết ${{u}_{n}}=\dfrac{2n+5}{5n-4}$. Số $\dfrac{7}{12}$ là số hạng thứ mấy của dãy số?
	\choice
	{$9$}
	{$6$}
	{$10$}
	{\True $8$}
	\loigiai {
		Ta có
		$${{u}_{n}}=\dfrac{2n+5}{5n-4}=\dfrac{7}{12}\Leftrightarrow 24n+60=35n-28\Leftrightarrow 11n=88\Leftrightarrow n=8.$$}
	
\end{ex}
%%%%%%%%%%
%Câu 6
\begin{ex}%[1K2B5-2]
	Cho dãy $(u_n)$ xác định bởi $\heva{& u_1=3 \\& u_{n+1}=\dfrac{u_n}{2}+2}$. Mệnh đề nào sau đây {\bf sai}?
	\choice
	{\True ${{u}_{2}}=\dfrac{5}{2}$}
	{${{u}_{4}}=\dfrac{31}{8}$}
	{${{u}_{3}}=\dfrac{15}{4}$}
	{${{u}_{5}}=\dfrac{63}{16}$}
	\loigiai {
		Ta có $\heva{
			& {{u}_{2}}=\dfrac{{{u}_{1}}}{2}+2=\dfrac{3}{2}+2=\dfrac{7}{2};\,\,{{u}_{3}}=\dfrac{{{u}_{2}}}{2}+2=\dfrac{7}{4}+2=\dfrac{15}{4}. \\
			& {{u}_{4}}=\dfrac{{{u}_{3}}}{2}+2=\dfrac{15}{8}+2=\dfrac{31}{8};\,\,{{u}_{5}}=\dfrac{{{u}_{4}}}{2}+2=\dfrac{31}{16}+2=\dfrac{63}{16}. \\
		}$}
\end{ex}
%%%%%%%%%%
%Câu 7
\begin{ex}%[1K2B5-2]
	Cho dãy số $({{u}_{n}} ),$ với ${{u}_{n}}={{\left(\dfrac{n-1}{n+1} \right)}^{2n+3}}$. Tìm số hạng ${{u}_{n+1}}$.
	\choice
	{${{u}_{n+1}}={{\left(\dfrac{n-1}{n+1} \right)}^{2(n-1 )+3}}$}
	{${{u}_{n+1}}={{\left(\dfrac{n-1}{n+1} \right)}^{2(n+1 )+3}}$ }
	{\True ${{u}_{n+1}}={{\left(\dfrac{n}{n+2} \right)}^{2n+5}}$}
	{${{u}_{n+1}}={{\left(\dfrac{n}{n+2} \right)}^{2n+3}}$}
	\loigiai {
		${{u}_{n}}={{\left(\dfrac{n-1}{n+1} \right)}^{2n+3}}\Rightarrow {{u}_{n+1}}={{\left(\dfrac{(n+1 )-1}{(n+1 )+1} \right)}^{2(n+1 )+3}}={{\left(\dfrac{n}{n+2} \right)}^{2n+5}}$.}
	
\end{ex}
%%%%%%%%%%
%Câu 8
\begin{ex}%[1K2K5-2]
	Cho dãy số $({{a}_{n}} ),$ được xác định $\heva{
		& {{a}_{1}}=3 \\
		& {{a}_{n+1}}=\dfrac{1}{2}{{a}_{n}},~n\ge 1 \\
	}$. Mệnh đề nào sau đây {\bf sai}?
	\choice
	{${{a}_{1}}+{{a}_{2}}+{{a}_{3}}+{{a}_{4}}+{{a}_{5}}=\dfrac{93}{16}$}
	{${{a}_{10}}=\dfrac{3}{512}$}
	{ \True ${{a}_{n}}=\dfrac{3}{{{2}^{n}}}$}
	{${{a}_{n+1}}+{{a}_{n}}=\dfrac{9}{{{2}^{n}}}$}
	\loigiai {
		Ta có ${{a}_{1}}=3;\,{{a}_{2}}=\dfrac{{{u}_{1}}}{2};\,\,{{a}_{3}}=\dfrac{{{u}_{2}}}{2}=\dfrac{{{u}_{1}}}{{{2}^2}};\,\,{{a}_{4}}=\dfrac{{{u}_{3}}}{2}=\dfrac{{{u}_{1}}}{{{2}^3}},\ldots \\
		\Rightarrow {{u}_{n}}=\dfrac{{{u}_{1}}}{{{2}^{n-1}}}=\dfrac{3}{{{2}^{n-1}}}$ nên suy ra đáp án ${{a}_{n}}=\dfrac{3}{{{2}^{n}}}$ sai. \\
		Xét đáp án\\
		${{a}_{1}}+{{a}_{2}}+{{a}_{3}}+{{a}_{4}}+{{a}_{5}}=3\left(1+\dfrac{1}{2}+\dfrac{1}{{{2}^2}}+\dfrac{1}{{{2}^3}}+\dfrac{1}{{{2}^4}}\right)=3.\dfrac{1-{{(\dfrac{1}{2} )}^5}}{1-\dfrac{1}{2}}=\dfrac{93}{16}\Rightarrow $  đúng.\\
		Xét đáp án ${{a}_{10}}=\dfrac{3}{{{2}^{9}}}=\dfrac{3}{512}\Rightarrow $  đúng.\\
		Xét đáp án ${{a}_{n+1}}+{{a}_{n}}=\dfrac{3}{{{2}^{n}}}+\dfrac{3}{{{2}^{n-1}}}=\dfrac{3+3\cdot 2}{{{2}^{n}}}=\dfrac{9}{{{2}^{n}}}\Rightarrow $ đúng.}
	
\end{ex}
%%%%%%%%%%
%Câu 9
\begin{ex}%[1K2K5-2]
	Cho dãy số $(u_n)$ biết $\heva{&u_1=1\\&u_2=4\\&u_{n+2}=3u_{n+1}-2u_n}$ với mọi $n \ge 1$. Giá trị $u_{101}-u_{100}$ là 
	\choice
	{$3\cdot 2^{102} $}
	{$3\cdot 2^{101} $}
	{$3\cdot 2^{100} $}
	{\True $ 3\cdot 2^{99}$}
	\loigiai{
		Theo bài  ta có 
		\begin{eqnarray*}
			&u_{n+2}=3u_{n+1}-2u_n\\
			\Leftrightarrow \,& u_{n+2}=u_{n+1}+2(u_{n+1}-u_n)\\
			\Leftrightarrow \,& u_{n+2}-u_{n+1}=2(u_{n+1}-u_n).
		\end{eqnarray*}
		Với $n=99$ ta có 
		\begin{align*}
			u_{101}-u_{100}&=2(u_{100}-u_{99})\\
			&=2\cdot 2 (u_{99}-u_{98})\\
			&= \ldots\\
			&=2^{99}\cdot(u_2-u_1)=3\cdot2^{99}.
		\end{align*}
	}
\end{ex}
%%%%%%%%%%
%Câu 10
\begin{ex}%[1K2G5-2]
	Cho dãy số $\left(u_n\right)$ thoả mãn $u_1=\sqrt{2}$ và $u_{n+1}=\sqrt{2+u_n}$ với mọi $n\geq 1$. Tìm $u_{2023}$.
	\choice
	{$u_{2023}=\sqrt{2}\cos\dfrac{\pi}{2^{2022}}$}
	{\True $u_{2023}=\sqrt{2}\cos\dfrac{\pi}{2^{2024}}$}
	{$u_{2023}=\sqrt{2}\cos\dfrac{\pi}{2^{2023}}$}
	{$u_{2023}=2$}
	\loigiai{Ta chứng minh bằng phương pháp quy nạp số hạng tổng quát của dãy là $u_n=2\cos\dfrac{\pi}{2^{n+1}}$.\\
		Dễ thấy, với $n=1$ ta có $u_1=\sqrt{2}$ (đúng).\\
		Giả sử mệnh đề đúng với $n=k, \forall k\in \mathbb{N}^\ast$ nghĩa là $u_k=2\cos\dfrac{\pi}{2^{k+1}}$ ta phải chứng minh mệnh đề đúng với $n=k+1$ nghĩa là $u_{k+1}=2\cos\dfrac{\pi}{2^{k+2}}$.\\
		Thật vậy, $u_{k+1}=\sqrt{2+u_k}=\sqrt{2+2\cos\dfrac{\pi}{2^{k+1}}}=\sqrt{4\cos^2\dfrac{\pi}{2^{k+2}}}=2\cos\dfrac{\pi}{2^{k+2}}$.\\
		Áp dụng công thức tổng quát trên ta có $u_{2023}=\sqrt{2}\cos\dfrac{\pi}{2^{2024}}$.
	}
\end{ex}
%%%%%%%%%%
\Closesolutionfile{ans}
\begin{indapan}{10}
	{ans/ans-1K2-1-Dang2}
\end{indapan}
\begin{dang}{Xét tính tăng giảm của dãy số}
	\begin{enumerate}
		\item Phương pháp 1. Xét dấu của hiệu số $u_{n+1}-u_n$.
		\begin{enumerate}
			\item Nếu $u_{n+1}-u_n>0, \forall n \in \mathbb{N}^\ast$ thì $(u_n)$ là dãy số tăng.
			\item Nếu $u_{n+1}-u_n<0, \forall n \in \mathbb{N}^\ast$ thì $(u_n)$ là dãy số giảm.
		\end{enumerate}
		\item Phương pháp 2. Nếu $u_n>0, \forall n\in \mathbb{N}^\ast$ thì ta có thể so sánh thương $\dfrac{u_{n+1}}{u_n}$ với $1$.
		\begin{enumerate}
			\item Nếu $\dfrac{u_{n+1}}{u_n}>1$ thì $(u_n)$ là dãy số tăng.
			\item Nếu $\dfrac{u_{n+1}}{u_n}<1$ thì $(u_n)$ là dãy số giảm.
		\end{enumerate}
		Nếu $u_n<0, \forall n\in \mathbb{N}^\ast$ thì ta có thể so sánh thương $\dfrac{u_{n+1}}{u_n}$ với $1$.
		\begin{enumerate}
			\item Nếu $\dfrac{u_{n+1}}{u_n}<1$ thì $(u_n)$ là dãy số tăng.
			\item Nếu $\dfrac{u_{n+1}}{u_n}>1$ thì $(u_n)$ là dãy số giảm.
		\end{enumerate}
		\item Phương pháp 3. Nếu dãy số $(u_n)$ cho bởi hệ thức truy hồi thì thường dùng phương pháp quy nạp để chứng minh $u_{n+1}>u_n, \forall n \in \mathbb{N}^\ast$ (hoặc $u_{n+1}<u_n \forall n \in \mathbb{N}^\ast$).
	\end{enumerate}
\end{dang}
\subsubsection{Ví dụ minh hoạ}
\begin{vd}[NB]%[1K2Y5-3]
	Xét sự tăng giảm của dãy số $(u_n)$ với $u_n=(-1)^n$.
	\dapso{dãy không tăng không giảm}
	\loigiai{
		Ta có:\\ $u_1=(-1)^1=-1,\,
		u_2=(-1)^2=1,\,
		u_3=(-1)^3=-1.$\\
		Vậy $(u_n)$ là dãy không tăng không giảm.
	}
\end{vd}
\begin{vd}[NB]%[1K2Y5-3]
	Xét tính tăng giảm của dãy số sau $(u_n)$ với $u_n=\dfrac{2n+1}{n+1}$.
	\dapso{dãy số tăng}
	\loigiai
	{
		Ta có: $u_n=\dfrac{2n+1}{n+1}=2-\dfrac{1}{n+1}$.\\
		$u_{n+1}-u_n=\left(2-\dfrac{1}{n+1+1}\right)-\left(2-\dfrac{1}{n+1}\right)=\dfrac{1}{n+1}-\dfrac{1}{n+2}>0, \forall n \in \mathbb{N}^\ast$.\\
		Vậy dãy số $(u_n)$ là dãy số tăng.
	}
\end{vd}
\begin{vd}[TH]%[1K2B5-3]
	Xét tính tăng giảm của dãy số $(u_n)$ với $u_n=\sqrt{n}-\sqrt{n+2}$.
	\dapso{dãy số tăng}
	\loigiai{
		Ta có $u_n=\sqrt{n}-\sqrt{n+2}=\dfrac{-2}{\sqrt{n}+\sqrt{n+2}}$.\\
		Xét hiệu\\ 
		$\begin{aligned}
			u_{n+1}-u_n&=\dfrac{-2}{\sqrt{n+1}+\sqrt{n+3}}-\dfrac{-2}{\sqrt{n}+\sqrt{n+2}}\\
			&=\dfrac{2}{\sqrt{n}+\sqrt{n+2}}-\dfrac{2}{\sqrt{n+1}+\sqrt{n+3}}>0, \forall n\in \mathbb{N}^\ast.
		\end{aligned}$\\
		Vậy $(u_n)$ là dãy số tăng.
	}
\end{vd}

\begin{vd}[TH]%[1K2B5-3]
	Xét tính tăng giảm của dãy số $(u_n)$ với $u_n=\dfrac{n}{3^n}$.
	\dapso{dãy số giảm}
	\loigiai{
		Ta có $u_n=\dfrac{n}{3^n}>0, \forall n \in \mathbb{N}^\ast$.\\
		Xét thương $\dfrac{u_{n+1}}{u_n}=\dfrac{n+1}{3^{n+1}}:\dfrac{n}{3^n}=\dfrac{n+1}{3.n}<1, \forall  n \in \mathbb{N}^\ast$.\\
		Vậy $(u_n)$ là dãy số giảm.
	}
\end{vd}
\begin{vd}[VD]%[1K2K5-3]
	Xét tính tăng giảm của dãy số $(u_n)$ với $ \heva{& u_1=2\\
		& u_{n+1}=\dfrac{3u_n+1}{u_n+1}, n\in \mathbb{N}^\ast.}$
	\dapso{dãy số tăng}
	\loigiai{
		Giả sử $u_{n+1}>u_n , \forall n \in \mathbb{N}^\ast. \qquad (*)$\\
		Ta chứng minh $(*)$ bằng phương pháp quy nạp.
		\begin{itemize}
			\item Với $n=1, u_2=\dfrac{3.2+1}{2+1}=\dfrac{6}{3}=\dfrac{7}{3}>u_1=2.$
			\item Giả sử $(*)$ đúng khi $n=k, k\in \mathbb{N}^\ast$, tức là $u_{k+1}>u_k$.\\
			Ta sẽ chứng minh $(*)$ đúng với $n=k+1$, tức là
			$u_{k+2}>u_{k+1}$.\\
			Thật vậy\\ $u_{k+2}-u_{k+1}=\left(3-\dfrac{2}{u_{k+1}+1}\right)-\left(3-\dfrac{2}{u_k+1}\right)=\dfrac{2}{u_k+1}-\dfrac{2}{u_{k+1}+1}.$\\
			Theo giả thiết quy nạp ta có:\\ $u_{k+1}>u_k \Rightarrow u_{k+1}+1>u_k+1 \Rightarrow \dfrac{2}{u_k+1}>\dfrac{2}{u_{k+1}+1}$.\\
			Vậy $u_{k+2}-u_{k+1}>0$.\\
			Do đó, $(*)$ đúng với mọi số nguyên dương $n$.
		\end{itemize}
		Vậy $(u_n)$ là dãy số tăng.}
\end{vd}
\subsubsection{Bài tập tự luận}
 
\begin{bt}[NB]%[1K2Y5-3]
	Xét tính tăng giảm của dãy số $(u_n)$ với $u_n=\dfrac{\sqrt{2}}{3^n}$.
	\dapso{dãy số giảm}
	\loigiai{
		Ta có $u_n>0, \forall n \in \mathbb{N}^\ast$.\\
		Xét thương $$\dfrac{u_{n+1}}{u_n}=\dfrac{\sqrt{2}}{3^{n+1}}: \dfrac{\sqrt{2}}{\sqrt{3^2}}=\dfrac{3^n}{3^{n+1}}=\dfrac{1}{3}<1.$$
		Vậy $\left(u_n\right)$ là dãy số giảm.
	}
\end{bt}
\begin{bt}[NB]%[1K2Y5-3]
	Xét tính tăng giảm của dãy số $\left(u_n\right)$ với $u_n=\dfrac{1}{n(n+1)}$.
	\dapso{dãy số tăng}
	\loigiai{
		Ta có $u_n=\dfrac{1}{n(n+1)}=\dfrac{1}{n}-\dfrac{1}{n+1}$.
		Xét hiệu:
		$$
		\begin{aligned}
			u_{n+1}-u_n & =\left(\dfrac{1}{n}-\dfrac{1}{n+1}\right)-\left(\dfrac{1}{n+1}-\dfrac{1}{n+2}\right) \\
			& =\dfrac{1}{n}-\dfrac{1}{n+2}>0, \forall n \in \mathbb{N}^\ast
		\end{aligned}
		$$
		Vậy  $\left(u_n\right)$ là dãy số tăng.
	}
\end{bt}
\begin{bt}[TH]%[1K2B5-3]
	Xét tính tăng giảm của dãy số $\left(u_n\right)$ với $u_n=n+\cos ^2 n$.
	\dapso{dãy số tăng}
	\loigiai{
		Xét hiệu
		$$
		\begin{aligned}
			u_{n+1}-u_n & =\left(n+1+\cos ^2(n+1)\right)-\left(n+\cos ^2 n\right) \\
			& =1+\cos ^2(n+1)-\cos ^2 n \\
			& =\cos ^2(n+1)+\sin ^2 n>0, \forall n \in \mathbb{N}^\ast .
		\end{aligned}
		$$
		Vậy $\left(u_n\right)$ là dãy số tăng.
	}
\end{bt}
\begin{bt}[TH]%[1K2B5-3]
	Xét tính tăng giảm của dãy số $(u_n)$ với $u_n=\dfrac{1}{n+1}+\dfrac{1}{n+2}+\ldots+\dfrac{1}{2n}$.
	\dapso{dãy số giảm}
	\loigiai{
		Xét hiệu\\
		$\begin{aligned}
			u_{n+1}-u_n&=\left(\dfrac{1}{n+2}+\dfrac{1}{n+3}+\ldots+\dfrac{1}{2(n+1)}\right)-\left(\dfrac{1}{n+1}+\dfrac{1}{n+2}+\ldots+\dfrac{1}{2n}\right)\\
			&=\dfrac{1}{n+2}-\dfrac{1}{2n+1}-\dfrac{1}{2n+2}\\
			&=\dfrac{1}{2n+2}-\dfrac{1}{2n+1}<0, \forall n\in \mathbb{N}^\ast.
		\end{aligned}$\\
		Vậy $(u_n)$ là dãy số giảm.
	}
	
\end{bt}

\begin{bt}[TH]%[1K2B5-3]
	Xét tính tăng giảm của dãy số $\left(u_n\right)$ với $u_n=\dfrac{1}{n+1}+\dfrac{1}{n+2}+\ldots+\dfrac{1}{2 n}$.
	\dapso{dãy số giảm}
	\loigiai{
		Xét hiệu
		$$
		\begin{aligned}
			u_{n+1}-u_n & =\left(\dfrac{1}{n+2}+\dfrac{1}{n+3}+\ldots+\dfrac{1}{2(n+1)}\right)-\left(\dfrac{1}{n+1}+\dfrac{1}{n+2}+\ldots+\dfrac{1}{2 n}\right) \\
			& =\dfrac{1}{n+2}-\dfrac{1}{2 n+1}-\dfrac{1}{2 n+2} \\
			& =\dfrac{1}{2 n+2}-\dfrac{1}{2 n+1}<0, \forall n \in \mathbb{N}^\ast
		\end{aligned}
		$$
		Vậy $\left(u_n\right)$ là dãy số giảm.
	}
\end{bt}
\begin{bt}[VD]%[1K2K5-3]
	Xét tính tăng giảm của dãy số $\left(u_n\right)$ cho bởi
	$$
	\left(u_n\right)\colon\heva{
		&u_1=1 ; u_2=2 \\
		&u_{n+1}=\sqrt{u_n}+\sqrt{u_{n-1}} \forall n \geq 2
	}
	$$
	\dapso{dãy số tăng}
	\loigiai{
		Ta chứng minh dãy $\left(u_n\right)$ là dãy tăng bằng phương pháp quy nạp.\\
		Dễ thấy $u_1<u_2<u_3$.\\
		Giả sử $u_{k-1}<u_k ~\forall k \geq 2$, ta chứng minh $u_{k+1}>u_k$.\\
		Thật vậy ta có $u_{k+1}=\sqrt{u_k}+\sqrt{u_{k-1}}>\sqrt{u_{k-1}}+\sqrt{u_{n-2}}=u_k$.\\ Vậy $\left(u_n\right)$ là dãy tăng.
	}
	
\end{bt}
\begin{bt}[VD]%[1K2K5-3]
	Cho dãy số $\left(u_n\right)$ biết $u_n=\dfrac{b \cdot2 n^2+1}{n^2+3}$ và $b \in \mathbb{R}$. Hãy xác định $b$ để
	\begin{listEX}[2]
		\item $\left(u_n\right)$ là dãy số giảm.
		\item $\left(u_n\right)$ là dãy số tăng.
	\end{listEX}
	\dapso{$b<\dfrac{1}{6}$ dãy số giảm; $b>\dfrac{1}{6}$ dãy số tăng}
	\loigiai{
		Ta có
		$$
		u_n=2 b+\dfrac{1-6 b}{n^2+3}
		$$
		Xét hiệu $$u_{n+1}-u_n=\dfrac{1-6 b}{(n+1)^2+3}-\dfrac{1-6 b}{n^2+3}=(1-6 b) \cdot\left(\dfrac{1}{(n+1)^2+3}-\dfrac{1}{n^2+3}\right)=A_n.$$
		\begin{listEX}
			\item Để $\left(u_n\right)$ là dãy sỗ giảm thì $A_n<0, \forall n \in \mathbb{N}^\ast$.
			$$
			A_n<0 \Leftrightarrow 1-6 b>0 \Leftrightarrow b<\dfrac{1}{6}
			$$
			\item Để $\left(u_n\right)$ là dãy số tăng thì $A_n>0, \forall n \in \mathbb{N}^\ast$.
			$$
			A_n>0 \Leftrightarrow 1-6 b<0 \Leftrightarrow b>\dfrac{1}{6}.
			$$
		\end{listEX}
		
	}
\end{bt}
\begin{bt}[VDC]%[1K2G5-3]
	Xét tính tăng giảm của dãy số $\left(u_n\right)$ với $u_n=\sin n+\cos n$.
	\dapso{dãy khōng tăng, không giảm}
	\loigiai{
		Ta có: $u_n=\sin n+\cos n=\sqrt{2} \sin \left(n+\dfrac{\pi}{4}\right)$.
		Xét hiệu
		$$
		\begin{aligned}
			u_{n+1}-u_n&=\sqrt{2} \sin \left(n+1+\dfrac{\pi}{4}\right)-\sqrt{2} \sin \left(n+\dfrac{\pi}{4}\right) \\
			&=2 \sqrt{2} \cdot \cos \left(2 n+\dfrac{1}{2}+\dfrac{\pi}{4}\right) \cdot \sin \dfrac{1}{2}=A_n . \\
		\end{aligned}
		$$
		Với  $n=1, A_1>0$. Với  $n=100, A_{100}<100$ . \\
		Vậy $\left(u_n\right)$ là dãy khōng tăng, không giảm.
	}
\end{bt}
\subsubsection{Bài tập trắc nghiệm}
\Opensolutionfile{ans}[ans/ans-1K2-1-Dang3]
%Câu 1
\begin{ex}%[1K2Y5-3]
	Cho các dãy số sau. Dãy số nào là dãy số tăng?
	\choice
	{$1;1;1;1;1;1;\ldots $}
	{$1;\dfrac{1}{2};\dfrac{1}{4};\dfrac{1}{8};\dfrac{1}{16};\ldots $}
	{$1;-\dfrac{1}{2};\dfrac{1}{4};-\dfrac{1}{8};\dfrac{1}{16};\ldots $}
	{ \True $1;3;5;7;9;\ldots $}
	\loigiai {
		Xét đáp án $1;1;1;1;1;1;\ldots$ đây là dãy hằng nên không tăng không giảm.\\
		Xét đáp án $1;-\dfrac{1}{2};\dfrac{1}{4};-\dfrac{1}{8};\dfrac{1}{16};\ldots \Rightarrow {{u}_{1}}>{{u}_{2}}<{{u}_{3}}\Rightarrow $ loại.\\
		Xét đáp án $1;3;5;7;9;\ldots \Rightarrow {{u}_{n}}<{{u}_{n+1}},\,\,n\in {{\mathbb{N}}^{*}}\Rightarrow $ chọn.\\
		Xét đáp án $1;\dfrac{1}{2};\dfrac{1}{4};\dfrac{1}{8};\dfrac{1}{16};\ldots \Rightarrow {{u}_{1}}>{{u}_{2}}>{{u}_{3}}\ldots >{{u}_{n}}>\ldots \Rightarrow $ loại.}
	
\end{ex}
%%%%%%%%%%
%Câu 2
\begin{ex}%[1K2Y5-3]
	Với giá trị nào của $a$ thì dãy số $\left(u_n\right)$ với $u_n=\dfrac{a n-1}{n+2}, \forall n \geq 1$ là dãy số tăng?
	\choice
	{$a>2$}
	{$a<-2$}
	{\True $a>-\dfrac{1}{2}$}
	{$a<-\dfrac{1}{2}$}
	\loigiai{
		Ta có $u_n=a-\dfrac{1+2 a}{n+2}$.\\
		$u_{n+1}-u_n=(1+2 a)\left(\dfrac{1}{n+2}-\dfrac{1}{n+3}\right)$.\\
		Suy ra dãy số đã cho tăng khi $a>-\dfrac{1}{2}$.
	}
\end{ex}
%%%%%%%%%%%%
%Câu 3
\begin{ex}%[1K2Y5-3]
	Trong các dãy $\left(u_n\right)$ sau đây dãy nào là dãy số giảm ?
	\choice
	{$u_n=(-1)^n$}
	{$u_n=2^n$}
	{$u_n=3 n+1$}
	{\True $u_n=\dfrac{1}{3^n}$}
	\loigiai{
		Xét dãy số $\left(u_n\right)$ có $u_n=\dfrac{1}{3^n}$, ta thấy $u_n>0, \forall n \in \mathbb{N}^\ast$ và $\dfrac{u_{n+1}}{u_n}=\dfrac{\dfrac{1}{3^{n+1}}}{\dfrac{1}{3^n}}=\dfrac{1}{3}<1$ nên dãy số $\left(u_n\right)$ này là dãy số giảm.
	}
\end{ex}
%%%%%%%%%%
%Câu 4
\begin{ex}%[1K2B5-3]
	Trong các dãy số $({{u}_{n}} )$ cho bởi số hạng tổng quát ${{u}_{n}}$ sau, dãy số nào là dãy số tăng?
	\choice
	{${{u}_{n}}=\dfrac{1}{n}$}
	{${{u}_{n}}=\dfrac{1}{{{2}^{n}}}$}
	{${{u}_{n}}=\dfrac{n+5}{3n+1}$}
	{ \True ${{u}_{n}}=\dfrac{2n-1}{n+1}$ }
	\loigiai {
		Vì ${{2}^{n}};\,n$ là các dãy dương và tăng nên $\dfrac{1}{{{2}^{n}}};\,\,\dfrac{1}{n}$ là các dãy giảm, do đó loại các đáp án ${{u}_{n}}=\dfrac{1}{{{2}^{n}}}$ và ${{u}_{n}}=\dfrac{1}{n}$.\\
		Xét đáp án ${{u}_{n}}=\dfrac{n+5}{3n+1}\Rightarrow \heva{
			& {{u}_{1}}=\dfrac{3}{2} \\
			& {{u}_{2}}=\dfrac{7}{6} \\
		}\Rightarrow {{u}_{1}}>{{u}_{2}}\Rightarrow $ loại.\\
		Xét đáp án ${{u}_{n}}=\dfrac{2n-1}{n+1}=2-\dfrac{3}{n+1}\Rightarrow  {{u}_{n+1}}-{{u}_{n}}=3\left(\dfrac{1}{n+1}-\dfrac{1}{n+2}\right)>0\Rightarrow$ nhận.}
	
\end{ex}
%%%%%%%%%%
%%%%%%%%%%
%Câu 5
\begin{ex}%[1K2B5-3]
	Trong các dãy số $({{u}_{n}} )$ cho bởi số hạng tổng quát ${{u}_{n}}$ sau, dãy số nào là dãy số giảm?
	\choice
	{${{u}_{n}}={{n}^2}$}
	{${{u}_{n}}=\dfrac{3n-1}{n+1}$}
	{${{u}_{n}}=\sqrt{n+2}$}
	{ \True ${{u}_{n}}=\dfrac{1}{{{2}^{n}}}$}
	\loigiai {
		Vì ${{2}^{n}}$ là dãy dương và tăng nên $\dfrac{1}{{{2}^{n}}}$ là dãy giảm. \\
		Xét ${{u}_{n}}=\dfrac{3n-1}{n+1}\Rightarrow \heva{
			& {{u}_{1}}=1 \\
			& {{u}_{2}}=\dfrac{5}{3} \\
		}\Rightarrow {{u}_{1}}<{{u}_{2}},$ loại.\\
		Hoặc
		${{u}_{n+1}}-{{u}_{n}}=\dfrac{3n+2}{n+2}-\dfrac{3n-1}{n+1}=\dfrac{4}{(n+1 )(n+2 )}>0$ nên $({{u}_{n}} )$ là dãy tăng.\\
		Xét ${{u}_{n}}={{n}^2}\Rightarrow {{u}_{n+1}}-{{u}_{n}}={{(n+1 )}^2}-{{n}^2}=2n+1>0,$ loại.\\
		Xét ${{u}_{n}}=\sqrt{n+2}\Rightarrow {{u}_{n+1}}-{{u}_{n}}=\sqrt{n+3}-\sqrt{n+2}=\dfrac{1}{\sqrt{n+3}+\sqrt{n+2}}>0,$ loại.}
	
\end{ex}

%Câu 6

\begin{ex}%[1K2B5-3]
	Trong các dãy số $(u_{n})$ sau, hãy chọn dãy số tăng.
	\choice
	{\True $u_{n}=(-1)^{2n}(5^{n}+1)$, $n\in \mathbb N^*$}
	{$u_{n}=\dfrac{n}{n^{2}+1}$, $n\in \mathbb N^*$}
	{$u_{n}=(-1)^{n+1}\sin \dfrac{\pi}{n}$, $n\in \mathbb N^*$}
	{$u_{n}=\dfrac{1}{\sqrt{n+1}+n}$, $n\in \mathbb N^*$}
	\loigiai
	{
		Xét dãy số $(u_n)$ với $u_{n}=(-1)^{2n}(5^{n}+1)$, ta có
		\[u_{n+1}-u_n = (-1)^{2n+2}(5^{n+1}+1)-(-1)^{2n}(5^{n}+1) = 5^{n+1}+1-5^n-1 = 4\cdot 5^n>0, \forall n\in\mathbb{N}^\ast.\]
		Vậy dãy trên là dãy số tăng.\\
		Xét các dãy số còn lại
		\begin{itemize}
			\item Với $u_{n}=(-1)^{n+1}\sin \dfrac{\pi}{n}$ ta có $u_1=0$, $u_2=-1$ hay $u_1>u_2$. Vậy dãy số này không là dãy số tăng.
			\item Với $u_{n}=\dfrac{1}{\sqrt{n+1}+n}$ ta có $u_1=\sqrt{2}-1$, $u_2=2-\sqrt{3}$ hay $u_1>u_2$. Vậy dãy số này không là dãy số tăng.
			\item Với $u_{n}=\dfrac{n}{n^{2}+1}$ ta có $u_1=\dfrac{1}{2}$, $u_2=\dfrac{2}{5}$ hay $u_1>u_2$. Vậy dãy số này không là dãy số tăng.
		\end{itemize}
	}
\end{ex}
%%%%%%%%%%
%Câu 7
\begin{ex}%[1K2K5-3]
	Trong các dãy số $({{u}_{n}} )$ cho bởi số hạng tổng quát ${{u}_{n}}$ sau, dãy số nào là dãy số giảm?
	\choice
	{${{u}_{n}}=\dfrac{{{n}^2}+1}{n}$}
	{${{u}_{n}}={{(-1 )}^{n}}\cdot ({{2}^{n}}+1 )$}
	{\True ${{u}_{n}}=\sqrt{n}-\sqrt{n-1}\,$}
	{${{u}_{n}}=\sin n$}
	\loigiai {
		Xét ${{u}_{n}}=\sin n\Rightarrow  {{u}_{n+1}}-{{u}_{n}}=2\cos \left(n+\dfrac{1}{2} \right)\sin \dfrac{1}{2}$ có thể dương hoặc âm phụ thuộc $n$ nên đáp án sai. Hoặc dễ thấy $\sin n$ có dấu thay đổi trên ${{\mathbb{N}}^{*}}$ nên dãy $\sin n$ không tăng, không giảm.\\
		Xét ${{u}_{n}}=\dfrac{{{n}^2}+1}{n}=n+\dfrac{1}{n}\Rightarrow  {{u}_{n+1}}-{{u}_{n}}=1+\dfrac{1}{n+1}-\dfrac{1}{n}=\dfrac{{{n}^2}+n-1}{n(n+1 )}>0$ nên dãy đã cho tăng nên đáp án sai.\\
		Xét ${{u}_{n}}=\sqrt{n}-\sqrt{n-1}=\dfrac{1}{\sqrt{n}+\sqrt{n+1}},$ dãy $\sqrt{n}+\sqrt{n-1}>0$ là dãy tăng nên suy ra ${{u}_{n}}$ giảm. \\
		Xét ${{u}_{n}}={{(-1 )}^{n}}({{2}^{n}}+1 )$ là dãy thay dấu nên không tăng không giảm, nên đáp án đúng.\\
		Cách trắc nghiệm\\
		Xét ${{u}_{n}}=\sin n$ có dấu thay đổi trên ${{\mathbb{N}}^{*}}$ nên dãy này không tăng không giảm.\\
		Xét ${{u}_{n}}=\dfrac{{{n}^2}+1}{n}$, ta có $\heva{
			& n=1\to {{u}_{1}}=2 \\
			& n=2\to {{u}_{2}}=\dfrac{5}{2} \\
		}\Rightarrow {{u}_{1}}<{{u}_{2}}\Rightarrow {{u}_{n}}=\dfrac{{{n}^2}+1}{n}$ không giảm.\\
		Xét ${{u}_{n}}=\sqrt{n}-\sqrt{n-1}$, ta có $\heva{
			& n=1\to {{u}_{1}}=1 \\
			& n=2\to {{u}_{2}}=\sqrt{2}-1 \\
		}\Rightarrow {{u}_{1}}>{{u}_{2}}$ nên dự đoán dãy này giảm.\\
		Xét ${{u}_{n}}={{(-1 )}^{n}}({{2}^{n}}+1 )$ là dãy thay dấu nên không tăng không giảm.\\
		Cách CASIO.\\
		Các dãy $\sin n;\,\,{{(-1 )}^{n}}({{2}^{n}}+1 )$ có dấu thay đổi trên ${{\mathbb{N}}^{*}}$ nên các dãy này không tăng không giảm nên loại các đáp án này.\\
		Xét hai đáp án còn lại, ta chỉ cần kiểm tra một đáp án bằng chức năng $TABLE$.\\
		Chẳng hạn kiểm tra đáp án ${{u}_{n}}=\dfrac{{{n}^2}+1}{n}$, ta vào chức năng $TABLE$ nhập $F(X )=\dfrac{X^2+1}{X}$ với thiết lập $\text{Start}=1,\text{ End}=10,\text{ Step}=1$.\\
		Nếu thấy cột $F(X )$ các giá trị tăng thì loại ${{u}_{n}}=\dfrac{{{n}^2}+1}{n}$ nếu ngược lại nếu thấy cột $F(X )$ các giá trị giảm dần thị chọn ${{u}_{n}}=\dfrac{{{n}^2}+1}{n}$.}
	
\end{ex}
%%%%%%%%%%
%Câu 8
\begin{ex}%[1K2K5-3]
	Mệnh đề nào sau đây đúng?
	\choice
	{Dãy số ${{u}_{n}}=\dfrac{1}{n}-2$ là dãy tăng}
	{\True Dãy số ${{u}_{n}}=2n+\cos \dfrac{1}{n}$ là dãy tăng}
	{Dãu số ${{u}_{n}}=\dfrac{n-1}{n+1}$ là dãy giảm}
	{Dãy số ${{u}_{n}}={{(-1 )}^{n}}({{2}^{n}}+1 )$ là dãy giảm}
	\loigiai {
		Xét đáp án ${{u}_{n}}=\dfrac{1}{n}-2\Rightarrow {{u}_{n+1}}-{{u}_{n}}=\dfrac{1}{n+1}-\dfrac{1}{n}<0\Rightarrow $loại.\\
		Xét đáp án ${{u}_{n}}={{(-1 )}^{n}}({{2}^{n}}+1 )$ là dãy có dấu thay đổi nên không giảm nên loại.\\
		Xét đáp án ${{u}_{n}}=\dfrac{n-1}{n+1}=1-\dfrac{2}{n+1}\Rightarrow {{u}_{n+1}}-{{u}_{n}}=2\left(\dfrac{1}{n+1}-\dfrac{1}{n+2}\right)>0\Rightarrow $ loại.\\
		Xét đáp án ${{u}_{n}}=2n+\cos \dfrac{1}{n}\Rightarrow {{u}_{n+1}}-{{u}_{n}}=\left(2-\cos \dfrac{1}{n+1}\right)+\cos \dfrac{1}{n+2}>0$ chọn.}
	
\end{ex}
%%%%%%%%%%
%Câu 9
\begin{ex}%[1K2K5-3]
	Mệnh đề nào sau đây {\bf sai}?
	\haicot
	{Dãy số ${{u}_{n}}=\dfrac{1-n}{\sqrt{n}}$ là dãy giảm}
	{Dãy số ${{u}_{n}}=n+\sin ^2n$ là dãy tăng}
	{\True Dãy số ${{u}_{n}}={{\left(1+\dfrac{1}{n}\right)}^{n}}$ là dãy giảm}
	{Dãy số ${{u}_{n}}=2{{n}^2}-5$ là dãy tăng}
	\loigiai {
		Xét đáp án \\ ${{u}_{n}}=\dfrac{1-n}{\sqrt{n}}=\dfrac{1}{\sqrt{n}}-\sqrt{n}\Rightarrow {{u}_{n+1}}-{{u}_{n}}=\dfrac{1}{\sqrt{n+1}}-\dfrac{1}{\sqrt{n}}+\sqrt{n}-\sqrt{n+1}<0$ nên dãy $({{u}_{n}} )$ là dãy giảm nên đúng.\\
		Xét đáp án ${{u}_{n}}=2{{n}^2}-5$ là dãy tăng vì ${{n}^2}$ là dãy tăng nên đúng. \\
		Hoặc
		${{u}_{n+1}}-{{u}_{n}}=2(2n+1 )>0$ nên $({{u}_{n}} )$ là dãy tăng.\\
		Xét đáp án ${{u}_{n}}={{\left(1+\dfrac{1}{n}\right)}^{n}}={{\left(\dfrac{n+1}{n} \right)}^{n}}>0\Rightarrow\dfrac{{{u}_{n+1}}}{{{u}_{n}}}=\dfrac{n+2}{n+1}\cdot {{\left(\dfrac{n+2}{n}\right)}^{n}}>1\Rightarrow ({{u}_{n}} )$ là dãy tăng nên sai.\\
		Xét đáp án ${{u}_{n}}=n+\sin ^2n\Rightarrow {{u}_{n+1}}-{{u}_{n}}=(1-\sin ^2(n+1 ) )+\sin ^2n>0$.}
	
\end{ex}
%%%%%%%%%%
%Câu 10%VDC 
\begin{ex}%[Nguyễn Long]%[1K2G5-3]
	Cho dãy $(u_n)\colon\heva{&u_1=1\\&u_{n+1}=\dfrac{n}{2(n+1)}u_n+\dfrac{3(n+2)}{2(n+1)}},n \in \mathbb{N^*}$. Nhận xét nào sau đây đúng
	\choice
	{\True Dãy số $(u_n)$ là dãy số tăng}
	{Dãy số $(u_n)$ là dãy số giảm}
	{Dãy số $(u_n)$ là dãy số không tăng, không giảm}
	{Tất cả các đáp án còn lại đều sai}
	\loigiai{ Ta chứng minh quy nạp $u_n<3, \forall n \in N^*$.\\
		Giả sử mđ đúng với $\mathrm{n}=\mathrm{k}$ khi đó có:
		$$
		u_{k+1}=\dfrac{k}{2(k+1)} u_k+\dfrac{3(k+2)}{2(k+1)}<\dfrac{3 k}{2(k+2)}+\dfrac{3(k+2)}{2(k+1)}=3 .
		$$
		Vậy mệnh đề đúng với $\mathrm{n}=\mathrm{k}+1$.
		Từ đó ta có $$u_{n+1}-u_n=\dfrac{\left(3-u_n\right)(n+2)}{n+1}>0.$$
		Vậy dãy $\left(u_n\right)$ tăng }
\end{ex}
\Closesolutionfile{ans}
\begin{indapan}{10}
	{ans/ans-1K2-1-Dang3}
\end{indapan}

\begin{dang}{Xét tính bị chặn của dãy số}
	\begin{itemize}
		\item Để chứng minh dãy số $(u_n)$ bị chặn trên bởi $M$, ta chứng minh $u_n\le M$, $\forall n\in\mathbb{N}^\ast$.
		\item Để chứng minh dãy số $(u_n)$ bị chặn dưới bởi $m$, ta chứng minh $u_n\ge m$, $\forall n\in\mathbb{N}^\ast$.
		\item Để chứng minh dãy số bị chặn ta chứng minh nó bị chặn trên và bị chặn dưới.
		\begin{itemize}
			\item Nếu dãy số $(u_n)$ tăng thì bị chặn dưới bởi $u_1$.
			\item Nếu dãy số $(u_n)$ giảm thì bị chặn trên bởi $u_1$.
		\end{itemize}
	\end{itemize}
\end{dang}
\subsubsection{Ví dụ minh hoạ}

%ví dụ 1
\begin{vd}[NB]%[1K2Y5-4]%[Trương Đăng Khoa]
	Chứng minh rằng dãy số $(u_n)$ với $u_n=\dfrac{3n}{n^2+9}$ bị chặn trên bởi $\dfrac{1}{2}$.
	\dapso{dãy số đã cho bị chặn trên bởi $\dfrac{1}{2}$}
	\loigiai{
		Với mọi $n\ge 1$, ta có $\dfrac{3n}{n^2+9}\le\dfrac{1}{2}\Leftrightarrow n^2+9\le 6n\Leftrightarrow(n-3)^2\le 0$ (đúng).\\
		Vậy dãy số đã cho bị chặn trên bởi $\dfrac{1}{2}$.
	}
\end{vd}

%ví dụ 2
\begin{vd}[NB]%[1K2Y5-4]%[Trương Đăng Khoa]
	Chứng minh rằng dãy số $(u_n)$ xác đinh bởi $u_n=\dfrac{8n+3}{3n+5}$ là một dãy số bị chặn.
	\dapso{dãy số bị chặn}
	\loigiai{
		Ta có $u_n>0$, $\forall n\ge 1$. Suy ra dãy số bị chặn dưới.\\
		Mặt khác $u_n=\dfrac{8n+3}{3n+5}<\dfrac{8n+3}{3n}=\dfrac{8}{3}+\dfrac{1}{n}<\dfrac{8}{3}+1=\dfrac{11}{3}$. Do đó dãy số bị chặn trên bởi $\dfrac{11}{3}$.\\
		Vậy dãy số đã cho bị chặn.
	}
\end{vd}
%ví dụ 4
\begin{vd}[TH]%[1K2B5-4]%[Trương Đăng Khoa]
	Xét tính bị chặn của dãy số $\left(u_n\right)$ với $u_n=\dfrac{3n+1}{n+3}$.
	\dapso{dãy số bị chặn}
	\loigiai{
		Với $n\in \mathbb{N}^\ast$ ta có $u_n=\dfrac{3n+1}{n+3}>0$.\\
		Nên dãy $\left(u_n\right)$ bị chặn dưới bởi $0$.\\
		Mặt khác $u_n=\dfrac{3n+1}{n+3}=\dfrac{3n+9-8}{n+3}=3-\dfrac{8}{n+3}<3$, $\forall n\in\mathbb{N}^\ast$.\\
		Nên dãy $\left(u_n\right)$ bị chặn trên bởi $3$.\\
		Vậy dãy số $\left(u_n\right)$ bị chặn.
	}
\end{vd}
%ví dụ 3
\begin{vd}[VD]%[1K2K5-4]%[Trương Đăng Khoa]
	Cho dãy số $(u_n)$ xác định bởi $u_1=1$ và $u_{n+1}=\dfrac{u_n+2}{u_n+1}$, $\forall n\ge 1$. Chứng minh rằng dãy $(u_n)$ bị chặn trên bởi sô $\dfrac{3}{2}$ và bị chặn dưới bởi số $1$.	
	\loigiai{
		Ta chứng minh $1\le u_n\le\dfrac{3}{2},\forall n\ge 1$ bằng phương pháp quy nạp.
		\begin{itemize}
			\item Với $n=1$ ta có $1\le u_1\le\dfrac{3}{2}$.
			\item Giả sử $1\le u_n\le\dfrac{3}{2}$ với mọi $n=k\ge 1$, tức là $1\le u_k\le\dfrac{3}{2}$. Ta cần chứng minh $1\le u_{k+1}\le\dfrac{3}{2}$.
		\end{itemize}
		Thật vậy 
		$u_{k+1}=1+\dfrac{1}{u_k+1}$.\\
		Vì $u_k+1>0$ nên $u_{k+1}=1+\dfrac{1}{u_k+1}>1$.\\
		Vì $u_k+1\ge 2$ nên $u_{k+1}=1+\dfrac{1}{u_k+1}\le 1+\dfrac{1}{2}=\dfrac{3}{2}$.\\
		Vậy $1\le u_n\le\dfrac{3}{2}$, $\forall n\ge 1$ hay dãy $(u_n)$ bị chặn trên bởi số $\dfrac{3}{2}$ và bị chặn dưới bởi số $1$.
	}
\end{vd}

%ví dụ 5
\begin{vd}[VD]%[1K2K5-4]%[Trương Đăng Khoa]
	Xét tính bị chặn của dãy số $\left(u_n\right)$ với $u_n=\sin n+ \cos n$.
	\dapso{dãy số bị chặn}
	\loigiai{
		Ta có $\begin{aligned}[t]
			&\ \sin n+\cos n \\
			=&\ \sqrt{2}\left(\dfrac{1}{\sqrt{2}}\sin n+\dfrac{1}{\sqrt{2}}\cos n\right)\\
			=&\ \sqrt{2}\left(\sin n\cdot\cos \dfrac{\pi}{4}+\cos n\cdot\sin \dfrac{\pi}{4}\right)\\
			=&\ \sqrt{2}\sin \left(n+\dfrac{\pi}{4}\right).
		\end{aligned}$\\
		Vì $\begin{aligned}[t]
			&\ -1\leq \sqrt{2}\sin \left(n+\dfrac{\pi}{4}\right) \leq 1\\
			\Rightarrow&\ -\sqrt{2}\leq \sqrt{2} \sin \left(n+\dfrac{\pi}{4}\right)\leq \sqrt{2}\\
			\Rightarrow&\ -\sqrt{2}\leq \sin n+\cos n \leq \sqrt{2},\ \forall n\in\mathbb{N}^\ast\\
			\Rightarrow&\ -\sqrt{2}\leq u_n \leq \sqrt{2},\ \forall n\in\mathbb{N}^\ast.
		\end{aligned}$\\
		Vậy dãy số $\left(u_n\right)$ là dãy số bị chặn.}
\end{vd}
\subsubsection{Bài tập tự luận}
 
%Bài 1 
\begin{bt}[TH]%[1K2B5-4]%[Trương Đăng Khoa]
	Xét tính bị chặn của các dãy số sau 
		\begin{listEX}[3]
			\item $u_n=\dfrac{1}{2n^2-1}$.
			\item 
			$u_n=3\cdot\cos\dfrac{n x}{3}$. 
			\item  $u_n=2n^3+1$.
			\item  $u_n=\dfrac{n^2+2n}{n^2+n+1}$.
			\item  $u_n=n+\dfrac{1}{n}$.
		\end{listEX}
	\loigiai{
		\begin{enumerate}
			\item  $u_n=\dfrac{1}{2n^2-1}$.\\
			Ta có $2n^2-1\ge 1\Rightarrow u_n=\dfrac{1}{2n^2-1}\le 1$, $\forall n\ge 1$.\\
			Vậy dãy số bị chặn trên bởi $1$.\\
			\item $u_n=3\cdot\cos\dfrac{n x}{3}$ có $-1\le\cos\dfrac{n x}{3}\le 1\Rightarrow-3\le 3\cdot\cos\dfrac{n x}{3}\le 3$.\\
			Vậy dãy số bị chặn dưới bởi $-3$ và chặn trên bởi $3$.
			\item  $u_n=2n^3+1$ có $2n^3+1\ge 3$, $\forall n\ge 1$.\\
			Vậy dãy số bị chặn dưới bởi $3$.
			\item $u_n=\dfrac{n^2+2n}{n^2+n+1}$ có $u_n=\dfrac{n^2+2n}{n^2+n+1}=1+\dfrac{n-1}{n^2+n+1}\ge 1$, $\forall n\ge 1$.\\
			Vậy dãy số bị chặn dưới bởi $1$.
			\item  $u_n=n+\dfrac{1}{n}$ có $u_n=n+\dfrac{1}{n}\ge 2\sqrt{n\cdot\dfrac{1}{n}}=2$, $\forall n>0$.\\
			Vậy dãy số bị chặn bởi $2$.
		\end{enumerate}
	}
\end{bt}
%Bài 2
\begin{bt}[VD]%[1K2K5-4]%[Trương Đăng Khoa]
	Xét tính bị chặn của dãy số $(u_n)$ với:
	\begin{listEX}[3]	 
		\item $u_{n}=\dfrac{4}{n}-5$.
		\item $u_{n}=\dfrac{n+4}{n+2}$.
		\item $u_{n}=\dfrac{5}{n^2+1}+\dfrac{n+2}{n+1}+\cos n$.
	\end{listEX}
	\loigiai{
		\begin{enumerate} 
			\item $u_{n}=\dfrac{4}{n}-5$.\\
			Ta có $u_n=\dfrac{4}{n}-5 \le \dfrac{4}{1}-5=-1$, $\forall n \in \mathbb{N}^{*}$ suy ra dãy $(u_n)$ bị chặn trên bởi $-1$.\\
			Mặt khác $u_n=\dfrac{4}{n}-5 \ge -5 \,\, \forall n \in \mathbb{N}^{*}$ suy ra dãy $(u_n)$ bị chặn dưới bởi $-5$.\\
			Vậy dãy $(u_n)$ bị chặn.	
			\item $u_{n}=\dfrac{n+4}{n+2}$.\\
			Ta có $u_n=	\dfrac{n+4}{n+2}=1+\dfrac{2}{n+2}> 1$, $\forall n \in \mathbb{N}^{*}$ suy ra dãy $(u_n)$ bị chặn dưới bởi $1$.\\
			Mặt khác $u_n=\dfrac{n+4}{n+2}=1+\dfrac{2}{n+2} \le 1+\dfrac{2}{1+2}=\dfrac{3}{5}$, $\forall n \in \mathbb{N}^{*}$ suy ra dãy $(u_n)$ bị chặn trên bởi $\dfrac{3}{5}$.\\
			Vậy dãy $(u_n)$ bị chặn.
			\item $u_{n}=\dfrac{5}{n^2+1}+\dfrac{n+2}{n+1}+\cos n$.\\
			Ta có $u_n=	\dfrac{5}{n^2+1}+\dfrac{n+2}{n+1}+\cos n=\dfrac{5}{n^2+1}+1+\dfrac{1}{n+1}+\cos n<5$, $\forall n \in \mathbb{N}^{*}$.\\
			Suy ra dãy $(u_n)$ bị chặn trên bởi $5$.\\
			Mặt khác $u_n=	\dfrac{5}{n^2+1}+\dfrac{n+2}{n+1}+\cos n=\dfrac{5}{n^2+1}+1+\dfrac{1}{n+1}+\cos n>0$, $\forall n \in \mathbb{N}^{*}$.\\
			Suy ra dãy $(u_n)$ bị chặn trên bởi $0$.\\
			Vậy dãy $(u_n)$ bị chặn.
		\end{enumerate}			
	}		
\end{bt}
%Bài 3
\begin{bt}[VDC]%[1K2G5-4]%[Trương Đăng Khoa]
	Xét tính bị chặn của dãy số $u_n=\left(1+\dfrac{1}{n}\right)^n$, $n\in N^\ast$.
	\loigiai{
		Ta có $u_n=\left(1+\dfrac{1}{n}\right)^n>0$, $\forall n\in N^\ast$ nên $(u_n)$ bị chặn dưới $(1)$.\\
		Lại có $\begin{aligned}[t]
			u_n&\ =\left(1+\dfrac{1}{n}\right)^n=\displaystyle\sum\limits_{k=0}^n C_n^k\left(\dfrac{1}{n}\right)^k\\
			&\ =\displaystyle\sum\limits_{k=0}^n\left[\dfrac{n!}{k!\cdot(n-k)!\cdot n^k}\right]\\
			&\ =\displaystyle\sum\limits_{k=0}^n\left[\dfrac{1}{k!}\cdot\dfrac{(n-k+1)}{n}\cdot\dfrac{(n-k+2)}{n}\ldots\dfrac{(n-k+k)}{n}\right]\le\displaystyle\sum\limits_{k=0}^n\dfrac{1}{k!},\, n\in \mathbb{N}^\ast
		\end{aligned}$\\
		Mà $\begin{aligned}[t]
			\displaystyle\sum\limits_{k=0}^n\dfrac{1}{k!}&\ \le 1+1+\dfrac{1}{1\cdot2}+\dfrac{1}{2\cdot3}+\dfrac{1}{3\cdot4}+\ldots+\dfrac{1}{(n-1)\cdot n}\\
			&\ =2+\left(1-\dfrac{1}{2}\right)+\left(\dfrac{1}{2}-\dfrac{1}{3}\right)+\ldots+\left(\dfrac{1}{n-1}-\dfrac{1}{n}\right)\\
			&\ 
			=3-\dfrac{1}{n}<3,\, \forall n\in \mathbb{N}^\ast.
		\end{aligned}$\\
		Suy ra $u_n<3$, $\forall n\in \mathbb{N}^\ast$ nên dãy số $(u_n)$ bị chặn trên $(2)$.\\
		Từ $(1)$ và $(2)$  suy ra dãy số $(u_n)$ bị chặn.}
\end{bt}
%Bài 4
\begin{bt}[VD]%[1K2K5-4]%[Trương Đăng Khoa]
	Cho dãy số $(u_n)$ xác định bởi $u_1=0$ và $u_{n+1}=\dfrac{1}{2}u_n+4$, $ \forall n\geq 1$.
	\begin{enumerate}
		\item Chứng minh dãy $(u_n)$ bị chặn trên bởi số $8$.
		\item Chứng minh dãy $(u_n)$ tăng, từ đó suy ra dãy $(u_n)$ bị chặn.
	\end{enumerate}
	\loigiai{
		\begin{enumerate}
			\item Ta chứng minh $u_n\leq 8$ với mọi $n\geq 1$.
			\begin{itemize}
				\item Khi $n=1$, ta có $u_1=0 <8$.
				\item Giả sử $u_n\leq 8$ với $n=k\geq 1$, tức là $u_k\leq 8$.\\
				Ta cần chứng minh $u_{k+1}\leq 8$.\\
				Thật vậy, $u_{k+1}=\dfrac{1}{2}u_k+4\leq \dfrac{1}{2}\cdot 8+4\leq 8$.
			\end{itemize}
			Vậy $u_n\leq 8$ với mọi $n\geq 1$, hay $(u_n)$ bị chặn trên bởi $8$.
			\item Với mọi $n\geq 1$, ta có $u_{n+1}-u_n=4-\dfrac{1}{2}u_n$. Mà $u_n\leq 8$ nên $u_{n+1}-u_n\geq 0$.\\
			Suy ra $u_n$ là dãy số tăng. Do đó $(u_n)$ bị chặn dưới bởi $u_1=0$.\\
			Kết hợp với câu a, ta được dãy số $(u_n)$ bị chặn.
		\end{enumerate}
	}
\end{bt}
%Bài 5
\begin{bt}[VD]%[1K2K5-4]%[Trương Đăng Khoa]
	Trong các dãy số $(u_n)$ sau, dãy số nào bị chặn trên, bị chặn dưới và bị chặn?
	\begin{listEX}[3]
		\item $u_n=n^2+5$.
		\item $u_n=\dfrac{3n+1}{2n+5}$.
		\item $u_n=(-1)^n\cos \dfrac{\pi}{2n}$.
		\item $u_n=\dfrac{n^2+2n}{n^2+n+1}$.
		\item $u_n=\dfrac{n}{\sqrt{n^2+2n}+n}$.
	\end{listEX}
	\loigiai{
		\begin{enumerate}
			\item Dãy số bị chặn dưới bởi $6$, không bị chặn trên.
			\item Dãy $(u_n)$ bị chặn dưới bởi $0$. Vì $u_n<\dfrac{3n+1}{2n}=\dfrac{3}{2}+\dfrac{1}{2n}<\dfrac{3}{2}+1=\dfrac{5}{2}$ nên dãy số bị chặn trên bởi $\dfrac{5}{2}$. Vậy dãy số bị chặn.
			\item Ta có $|u_n|\leq 1$ nên dãy số bị chặn trên bởi 1, bị chặn dưới bởi $-1$.
			\item Dãy số bị chặn dưới bởi $0$. Vì $u_n<\dfrac{n^2+2n}{n^2}=1+\dfrac{2}{n}\leq 3$ nên dãy số bị chặn trên. Vậy dãy số bị chặn.
			\item Ta có $0<u_n\leq 1$ vậy dãy số bị chặn.
		\end{enumerate}
	}
\end{bt}

\subsubsection{Câu hỏi trắc nghiệm}

\Opensolutionfile{ans}[ans/ans-1K2-1-Dang4]
%Câu 1
\begin{ex}%[1K2K5-4]%[Trương Đăng Khoa]
	Cho dãy số $(u_n)$ xác định bởi $u_1=3$ và $u_{n+1}=\dfrac{u_n+1}{2}$, $\forall n\geq 1$. Mệnh đề nào sau đây là đúng?
	\choice
	{\True Dãy số bị chặn}
	{Dãy số bị chặn trên}
	{Dãy số bị chặn dưới}
	{Dãy số không bị chặn}
	\loigiai{Ta chứng minh $u_n>1, \forall n\geq 1$ bằng phương pháp quy nạp.\\
		Suy ra dãy số bị chặn dưới bởi $1$.\\
		Ta có
		$u_{n+1}-u_n=\dfrac{1-u_n}{2}<0$, $\forall n\geq 1$.\\
		Do đó dãy số này là dãy số giảm nên nó bị chặn trên bởi $u_1=3$.\\
		Vậy dãy số đã cho là dãy số bị chặn.		
	}
\end{ex}
%Câu 2
\begin{ex}%[1K2K5-4]%[Trương Đăng Khoa]
	Cho dãy số $(u_n)$ xác định bởi $u_1=\sqrt{2}$ và $u_{n+1}=\sqrt{2+u_n}$, $\forall n\geq 1$. Mệnh đề nào sau đây là đúng?
	\choice
	{Dãy số bị chặn trên}
	{Dãy số bị chặn dưới}
	{\True Dãy số bị chặn}
	{Dãy số không bị chặn}
	\loigiai{Vì $u_n\geq 0$, $\forall n\geq 1$ nên dãy số bị chặn dưới bởi $0$.\\
		Ta chứng minh $u_n\geq 2, \forall n\geq 1$. Suy ra dãy số bị chặn trên bởi $2$.\\
		Vậy dãy số đã cho là dãy số bị chặn.		
	}
\end{ex}
%Câu 3
\begin{ex}%[1K2K5-4]%[Trương Đăng Khoa]
	Xét tính bị chặn của  dãy số $(u_n)$ với $u_n=\dfrac{1}{1\cdot2}+\dfrac{1}{2\cdot3}+\ldots+\dfrac{1}{n\cdot(n+1)}$.
	\choice{Không bị chặn}{Bị chặn trên}{Bị chặn dưới}{\True Bị chặn}
	\loigiai{
		Ta có $u_n=1-\dfrac{1}{2}+\dfrac{1}{2}-\dfrac{1}{3}+\ldots+\dfrac{1}{n}-\dfrac{1}{n+1}=1-\dfrac{1}{n+1}$.\\
		Do đó $0\leq u_n \leq 1$, $\forall n\geq 1$.\\
		Vậy dãy số đã cho bị chặn.
	}
\end{ex}
%Câu 4
\begin{ex}%[1K2G5-4]%[Trương Đăng Khoa]
	Cho dãy số $(u_n)$ với $u_n=\dfrac{1}{1\cdot4}+\dfrac{1}{2\cdot5}+\ldots+\dfrac{1}{n\cdot(n+3)}$. Dãy số $\left(u_n\right)$ bị chặn dưới và chặn trên lần lượt bởi các số $m$ và $M$ nào dưới đây?
	\choice
	{$m=0$, $M=1$}
	{$m=1$, $M=\dfrac{1}{2}$}
	{$m=1$, $M=\dfrac{10}{19}$}
	{\True $m=0$, $M=\dfrac{11}{18}$}
	\loigiai{
		Rõ ràng $u_n>0$, $\forall n\in\mathbb{N}^\ast$ nên $(u_n)$ bị chặn dưới.\\
		Mặt khác $\dfrac{1}{k(k+3)}=\dfrac{1}{3}\left(\dfrac{1}{k}-\dfrac{1}{k+3}\right)$.\\
		Suy ra $\begin{aligned}[t]
			u_n&\  =\dfrac{1}{3}\bigg[\left(1-\dfrac{1}{4}\right)+\left(\dfrac{1}{2}-\dfrac{1}{5}\right)+\left(\dfrac{1}{3}-\dfrac{1}{6}\right)+\left(\dfrac{1}{4}-\dfrac{1}{7}\right)+\\
			&\  \ldots+\left(\dfrac{1}{n-3}-\dfrac{1}{n}\right)+\left(\dfrac{1}{n-2}-\dfrac{1}{n+1}\right)+\left(\dfrac{1}{n-1}-\dfrac{1}{n+2}\right)+\left(\dfrac{1}{n}-\dfrac{1}{n+3}\right)\bigg]\\
			&\ = \dfrac{1}{3}\left(1+\dfrac{1}{2}+\dfrac{1}{3}-\dfrac{1}{n+1}-\dfrac{1}{n+2}-\dfrac{1}{n+3}\right)<\dfrac{11}{18}, \, \forall n\in\mathbb{N}^\ast.
		\end{aligned}$\\
		Do đó $(u_n)$ bị chặn trên.\\
		Vậy $m=0$, $M=\dfrac{11}{18}$.
	}
\end{ex}
%Câu 5
\begin{ex}%[1K2G5-4]%[Trương Đăng Khoa]
	Cho dãy số $(u_n)$ biết $u_n=\dfrac{1\cdot 3\cdot 5\ldots(2n-1)}{2\cdot 4\cdot 6\cdot 2n}$. Dãy số $\left(u_n\right)$ bị chặn dưới và chặn trên lần lượt bởi các số $m$ và $M$. Tính giá trị biểu thức $m+M$?
	\choice{$\dfrac{1}{\sqrt{2}}$}{\True $\dfrac{1}{\sqrt{3}}$}{$\dfrac{1}{\sqrt{5}}$}{$\dfrac{1}{\sqrt{7}}$}
	\loigiai{
		Xét $ \dfrac{2 k-1}{2 k}<\dfrac{2 k-1}{\sqrt{4 k^2-1}}
		=\dfrac{\sqrt{(2 k-1)^2}}{\sqrt{(2 k-1)(2 k+1)}} =\dfrac{\sqrt{2 k-1}}{\sqrt{2 k+1}}$,  $\forall k \ge 1$.\\
		$\Rightarrow u_n<\dfrac{\sqrt{1}}{\sqrt{3}} \cdot\dfrac{\sqrt{3}}{\sqrt{5}} \cdot\dfrac{\sqrt{5}}{\sqrt{7}}\cdot \ldots \cdot \dfrac{\sqrt{2 n-1}}{\sqrt{2 n+1}}=\dfrac{1}{\sqrt{2 n+1}} \le\dfrac{1}{\sqrt{3}}$,  $\forall n \in\mathbb{N}^\ast$.\\
		$\Rightarrow 0<u_n<\dfrac{1}{\sqrt{3}}$, $\forall n \in\mathbb{N}^\ast$.\\
		Vậy $m+M=0+\dfrac{1}{\sqrt{3}}$.
	}
\end{ex}
%Câu 6
\begin{ex}%[1K2G5-4]%[Trương Đăng Khoa]
	Cho dãy số $(u_n)$, với $u_n=\dfrac{1}{2^2}+\dfrac{1}{3^2}+\ldots+\dfrac{1}{n^2}$, $\forall n=2;3;4;\ldots$. Khẳng định nào sau đây là đúng?
	\choice
	{\True Dãy số bị chặn}
	{Dãy số bị chặn trên}
	{Dãy số bị chặn dưới}
	{Dãy số không bị chặn}
	\loigiai{
		Ta có $u_n>0\Rightarrow(u_n)$ bị chặn dưới bởi $0$.\\
		Mặt khác $\dfrac{1}{k^2}<\dfrac{1}{(k-1) k}=\dfrac{1}{k-1}-\dfrac{1}{k}$, ($k\in\mathbb{N}^\ast$, $k\ge 2$) nên suy ra
		\begin{eqnarray*}
			u_n&<&\dfrac{1}{1 \cdot 2}+\dfrac{1}{2 \cdot 3}+\dfrac{1}{3 \cdot 4}+\cdots+\dfrac{1}{n(n+1)}\\
			&=&1-\dfrac{1}{2}+\dfrac{1}{2}-\dfrac{1}{3}+\dfrac{1}{2}-\dfrac{1}{4}+\cdots+\dfrac{1}{n}-\dfrac{1}{n+1}=1-\dfrac{1}{n+1}<1.
		\end{eqnarray*}
		Nên dãy $(u_n)$ bị chặn trên, do đó dãy $(u_n)$ bị chặn.
	}
\end{ex}
%Câu 7
\begin{ex}%[1K2G5-4]%[Trương Đăng Khoa]
	Cho dãy số $\left(u_n\right)$ và đặt $u_n= \displaystyle \sum_{k=1}^{n} a_k$ với $a_k=\dfrac{1}{4k^2-1}$. Mệnh đề nào sau đây là đúng?
	\choice{$0<u_n <1$}
	{$0\leq u_n\leq \dfrac{1}{2}$}
	{\True $0<u_n<\dfrac{1}{2}$}
	{$0\leq u_n\leq 1$}
	\loigiai{
		\begin{itemize}
			\item
			Ta có $a_k=\dfrac{1}{4k^2-1}=\dfrac{1}{(2k+1)(2k-1)}=\dfrac{1}{2}\cdot\dfrac{(2k+1)-(2k-1)}{(2k+1)(2k-1)}=\dfrac{1}{2}\cdot\left(\dfrac{1}{2k-1}-\dfrac{1}{2k+1}\right)$.\\
			\item Mặt khác $u_n=\displaystyle \sum_{k=1}^{n} a_k$.
			Do đó
			\begin{eqnarray*}
				&u_n&=\dfrac{1}{2}\cdot\left(\dfrac{1}{1}-\dfrac{1}{3}\right)+\dfrac{1}{2}\cdot\left(\dfrac{1}{3}-\dfrac{1}{5}\right)+\ldots + \dfrac{1}{2}\cdot \left(\dfrac{1}{2n-1}-\dfrac{1}{2n+1}\right)\\
				&&=\dfrac{1}{2}\left(\dfrac{1}{1}-\dfrac{1}{2n+1}\right)\\
				&&=\dfrac{1}{2}\cdot\dfrac{2n}{2n+1}=\dfrac{n}{2n+1}.
			\end{eqnarray*}
			\item 
			
			Với mọi $n \in \mathbb{N}^\ast$ thì $u_n>0$ nên dãy số $\left(u_n\right)$ bị chặn dưới.\\
			Ta lại có $u_n=\dfrac{1}{2}\cdot\left(1-\dfrac{1}{2n+1}\right)<\dfrac{1}{2}$.\\
			Vậy dãy số bị chặn.
		\end{itemize}
	}
\end{ex}
%Câu 8
\begin{ex}%[1K2G5-4]%[Trương Đăng Khoa]
	Cho dãy số $\left(u_n\right)$ và đặt $u_n= \displaystyle \sum_{k=1}^{n} a_k$ với $a_k=\dfrac{1}{k(k+4)}$.  Dãy số $\left(u_n\right)$ bị chặn dưới và chặn trên lần lượt bởi các số $m$ và $M$ nào sau đây?
	\choice
	{\True $m=0$, $M=\dfrac{25}{48}$}
	{$m=0$, $M=\dfrac{25}{12}$}
	{$m=1$, $M=\dfrac{1}{4}$}
	{$m=1$, $M=\dfrac{1}{2}$}
	\loigiai{
		Ta có $a_k=\dfrac{1}{k(k+4)}=\dfrac{1}{4}\cdot\dfrac{4}{k(k+4)}=\dfrac{1}{4}\cdot\dfrac{k+4-k}{k(k+4)}=\dfrac{1}{4}\cdot\left(\dfrac{1}{k}-\dfrac{1}{k+4}\right)$.\\
		Mặt khác $u_n=\displaystyle \sum_{k=1}^{n} a_k$.
		Do đó
		\begin{eqnarray*}
			&u_n&=\dfrac{1}{4}\cdot\left(\dfrac{1}{1}-\dfrac{1}{5}\right)+\dfrac{1}{4}.\left(\dfrac{1}{2}-\dfrac{1}{6}\right)+\ldots + \dfrac{1}{4}\cdot\left(\dfrac{1}{n}-\dfrac{1}{n+4}\right)\\
			&&=\dfrac{1}{4}\left(\dfrac{1}{1}+\dfrac{1}{2}+\dfrac{1}{3}+\dfrac{1}{4}-\dfrac{1}{n+1}-\dfrac{1}{n+2}-\dfrac{1}{n+3}-\dfrac{1}{n+4}\right)\\
			&&=\dfrac{1}{4}\left(\dfrac{25}{12}-\dfrac{1}{n+1}-\dfrac{1}{n+2}-\dfrac{1}{n+3}-\dfrac{1}{n+4}\right).
		\end{eqnarray*}
		Với mọi $n \in \mathbb{N}^\ast$ thì $u_n>0$ nên dãy số $\left(u_n\right)$ bị chặn dưới.\\
		Ta lại có $u_n=\dfrac{1}{4}\cdot\left(\dfrac{25}{12}-\dfrac{1}{n+1}-\dfrac{1}{n+2}-\dfrac{1}{n+3}-\dfrac{1}{n+4}\right)<\dfrac{1}{4}\cdot\dfrac{25}{12}=\dfrac{25}{48}$.\\
		Vậy $m=0$, $M=\dfrac{25}{48}$.
	}
\end{ex}
%Câu 9
\begin{ex}%[1K2G5-4]%[Trương Đăng Khoa]
	Xét tính bị chặn của dãy số $\left(u_n\right)$ và đặt $u_n=\displaystyle \sum_{k=1}^{n} a_k$ với $a_k=\dfrac{1}{k(k+1)}$.
	\choice{\True Bị chặn}{Bị chặn dưới}{Bị chặn trên}{Không bị chặn.}
	\loigiai{
		Ta có $a_k=\dfrac{1}{k(k+1)}=\dfrac{1}{k}-\dfrac{1}{k+1}$. Do đó\\
		$u_n=\displaystyle \sum_{k=1}^{n}a_k=\left(1-\dfrac{1}{2}\right)+\left(\dfrac{1}{2}-\dfrac{1}{3}\right)+\ldots+\left(\dfrac{1}{n-1}-\dfrac{1}{n}\right)+\left(\dfrac{1}{n}-\dfrac{1}{n+1}\right)=1-\dfrac{1}{n+1}=\dfrac{n}{n+1}$.\\
		Với mọi $n \in \mathbb{N}^*$ thì $u_n>0$ nên dãy số $\left(u_n\right)$ bị chặn dưới.\\
		Ta lại có $u_n=1-\dfrac{n}{n+1}<1$, $\forall n \in \mathbb{N}^\ast$ nên dãy số $\left(u_n\right)$ bị chặn trên.\\
		Vậy dãy số bị chặn.
	}
\end{ex}
%Câu 10
\begin{ex}%[1K2G5-4]%[Trương Đăng Khoa]
	Cho dãy số $(u_n)$, xác định bởi $\heva{&u_1=6\\&u_{n+1}=\sqrt{6+u_n},\, \forall n\in\mathbb{N}^\ast}$. Mệnh đề nào sau đây là đúng?
	\choice{
		$\sqrt{6}<u_n<2\sqrt{3}$	
	}
	{\True $\sqrt{6}\leq u_n\leq 2\sqrt{3}$}
	{$\sqrt{6}<u_n\leq 2\sqrt{3}$	}
	{$\sqrt{6}\geq u_n<2\sqrt{3}$	}
	\loigiai{
		Ta có 
		$\heva{&u_1=6\\
			&u_{n+1}=\sqrt{6+u_n}} \Rightarrow
		\heva{
			&u_1=6\\
			&u_{n+1} \ge 0 } \Rightarrow u_n \ge 0 \Rightarrow\heva{&u_1=6\\
			&u_{n+1}=\sqrt{6+u_n}\ge\sqrt{6}}
		\Rightarrow u_n \ge\sqrt{6}$.\\
		Ta chứng minh quy nạp $\heva{&u_n\le 2\sqrt{3}\\ &u_1\le 2\sqrt{3}\\ &u_k\le 2\sqrt{3}.}$\\
		$\Rightarrow u_{k+1}=\sqrt{6+u_{k+1}} \le\sqrt{6+2 \sqrt{3}}<\sqrt{6+6}=2 \sqrt{3}$.\\
		Vậy $\sqrt{6} \leq  u_n \leq 2\sqrt{3}$.
	}
\end{ex}
\Closesolutionfile{ans}
\begin{indapan}{10}
	{ans/ans-1K2-1-Dang3}
\end{indapan}

\begin{dang}{Toán thực tế về dãy số}
\end{dang}
\subsubsection{Ví dụ minh hoạ}
% \begin{vd}%[1T2B1-5]%[Trương Đăng Khoa]%Ví dụ 1
% 	Một chồng cột gỗ được xếp thành các lớp, hai lớp liên tiếp hơn kém nhau một cột gỗ.
% 	\begin{center}
% 		\begin{tikzpicture}[font=\footnotesize, line join=round, line cap=round, >=stealth,scale=0.8]
% 			\def\r{0.2}
% 			\def\n{25}
% 			\def\g{110}
% 			\fill[teal!50!green](-6*\r,-3*\r)rectangle(3.5*\n*\r,0.5*\n*\r);
% 			\fill[teal!50!green,opacity=0.25](3.5*\n*\r,3*\r)rectangle(-6*\r,2*\n*\r);
% 			\foreach \j in {0,...,12}{
% 				\pgfmathsetmacro{\m}{\n-\j}
% 				\foreach \i in{0,...,\m}{
% 					\fill[left color=orange, right color=teal!30,draw=brown](2*\i*\r,0)++(60:2*\j*\r)++(\g:\r)--++(\g-90:6)arc(\g:-60:\r)--++(\g-270:6)--cycle;
% 					\fill[orange!20!brown!40,draw=teal](2*\i*\r,0)++(60:2*\j*\r)circle(\r);
% 				}
% 			}
% 		\end{tikzpicture}
% 	\end{center}
% 	\begin{enumerate}
% 		\item  Gọi $u_1=25$ là số cột gỗ có ở hàng dưới cùng của chồng cột gỗ, $u_n$ là số cột gỗ có ở hàng thứ $n$ tính từ dưới lên trên. Xét tính tăng, giảm của dãy số này.
% 		\item  Gọi $v_1=14$ là số cột gỗ có ở hàng trên cùng của chồng cột gỗ, $v_n$ là số cột gỗ có ở hàng thứ $n$ tính từ trên xuống dưới. Xét tinh tăng, giảm của dãy số này.
% 	\end{enumerate}
% 	\loigiai{
% 		\begin{enumerate}
% 			\item Ta có $u_n=26-n>u_{n+1}=26-n-1=25-n$.\\
% 			Vậy dãy số $(u_n)$ là dãy số giảm.
% 			\item Ta có $v_n=13+n<v_{n+1}=13+n+1=14+n$.\\
% 			Vậy dãy số $(u_n)$ là dãy số tăng
% 		\end{enumerate}
% 	}
% \end{vd}

\begin{vd}%[1T2B1-5]%[Trương Đăng Khoa]%Ví dụ 2
	Trên lưới ô vuông, mỗi ô cạnh $1$ đơn vị, người ta vẽ $8$ hình vuông và tô màu khác nhau như hình vẽ. Tìm dãy số biểu diễn độ dài cạnh của $8$ hình vuông đó từ nhỏ đến lớn. Có nhận xét gì về dãy số trên?
	\begin{center}
		\begin{tikzpicture}[scale=0.8]
			\def\r{21}
			\def\hv(#1){
				\ifnum #1= 1\else
				\pgfmathsetmacro{\R}{250*rnd}
				\pgfmathsetmacro{\G}{250*rnd}
				\pgfmathsetmacro{\B}{250*rnd}
				\definecolor{mau}{RGB}{\R,\G,\B}
				\fill[mau!30](0,0)rectangle(\r,\r);
				\draw[red,line width=1pt] (0,0) arc(180:90:\r)(0,0)rectangle(\r,\r);
				\pgfmathtruncatemacro{\k}{#1-1}
				\begin{scope}[shift={(45:\r*sqrt(2))},rotate=-90,scale={(sqrt(5)-1)/2}]
					\hv(\k)
					\pgfmathsetmacro{\n}{int((1/(sqrt(5))*(((1+sqrt(5))/2)^(\k)-(1-(sqrt(5))/2)^(\k)+1)}
					\ifnum \k>1
					\path (\r/2,\r/2)node[scale=1.75]{\color{red}$\n$};
					\else
					\fi
				\end{scope}
				\fi
			}
			\begin{scope}[scale=0.35]
				\hv(9)
				\draw[teal](0,0)grid(34,21);
				\path(21/2,21/2)node[scale=2]{21};
			\end{scope}
		\end{tikzpicture}
	\end{center}
	\loigiai{
		\begin{multicols}{4}
			\begin{itemize}
				\item $u_1=1$.
				\item $u_2=1$.
				\item $u_3=2$.
				\item $u_4=3$.
				\item $u_5=5$.
				\item $u_6=8$.
				\item $u_7=13$.
				\item $u_8=21$.
			\end{itemize}
		\end{multicols}
		Ta có dãy số $\left(u_n\right)\colon\heva{& u_1=1\\ &u_2=1\\ &u_n=u_{n-1}-u_{n-2}.}$
	}
\end{vd}

\begin{vd}%[1C2K1-5]%[Trương Đăng Khoa]% Ví dụ 3
	Chị Mai gửi tiền tiết kiệm vào ngân hàng theo thể thức lãi kép như sau. Lần đầu chị gửi $100$ triệu đồng. Sau đó, cứ hết $1$ tháng chị lại gửi thêm vào ngân hàng $6$ triệu đồng. Biết lãi suất của ngân hàng là $0{,}5\%$ một tháng. Gọi $P_n$ (triệu đồng) là số tiền chị có trong ngân hàng sau $n$ tháng.
	\begin{enumerate}
		\item Tính số tiền chị có trong ngân hàng sau $1$ tháng.
		\item Tính số tiền chị có trong ngân hàng sau $3$ tháng.
		\item Dự đoán công thức của $P_n$ tính theo $n$.
	\end{enumerate}
	\loigiai{
		\begin{enumerate}
			\item Số tiền chị có trong ngân hàng sau $1$ tháng là $P_1=+100+100\cdot0{,}5\%+6=100{,}5+6$ (triệu đồng).
			\item Số tiền chị có trong ngân hàng sau 2 tháng là 
			\begin{eqnarray*}
				P_2&=&100{,}5+6+(100{,}5+6)\cdot 0{,}5\%+6\\
				&=&(100{,}5+6)(1+0{,}5\%)+6\\
				&=& 100{,}5(1+0{,}5\%)+6\cdot(1+0{,}5\%)+6\, (\text{triệu đồng}).
			\end{eqnarray*}
			Số tiền chị có trong ngân hàng sau $3$ tháng là
			\begin{eqnarray*}
				P_3&=&(100{,}5+6)(1+0{,}5 \%)+6+[(100{,}5+6)(1+0{,}5 \%)+6] \cdot 0{,}5 \%+6\\
				&=& 100{,}5 \cdot(1+0{,}5 \%)^2+6(1+0{,}5 \%)^2+6 \cdot(1+0{,}5 \%)+6 \text{(triệu đồng)}.
			\end{eqnarray*}
			\item Số tiền chị có trong ngân hàng sau $4$ tháng là
			\begin{eqnarray*}
				P_4&=&(100{,}5+6)(1+0{,}5 \%)^2+6 \cdot(1+0{,}5 \%)+6+\left[(100{,}5+6)(1+0{,}5 \%)^2+6 \cdot(1+0{,}5 \%)+6\right]\cdot 0{,}5 \%+6\\ 
				&=&100{,}5 \cdot(1+0{,}5 \%)^3+6 \cdot(1+0{,}5 \%)^3+6\cdot(1+0{,}5 \%)^2+6 \cdot(1+0{,}5 \%)+6\, (\text{triệu đồng}).
			\end{eqnarray*}
			Số tiền chị có trong ngân hàng sau $n$ tháng là
			$$P_n=100{,}5 \cdot(1+0{,}5 \%)^{n-1}+6\cdot(1+0{,}5 \%)^{n-1}+6\cdot(1+0{,}5 \%)^{n-2}+6 \cdot(1+0{,}5 \%)^{n-3}+\ldots+6$$ với mọi $n \in\mathbb{N}^\ast$.
		\end{enumerate}
	}
\end{vd}

\begin{vd}%[1K2K5-5]%[Trương Đăng Khoa]% Ví dụ 4
	Anh Thanh vừa được tuyển dụng vào một công ty công nghệ, được cam kết lương năm đầu sẽ là $200$ triệu đồng và lương mỗi năm tiếp theo sẽ được tăng thêm $25$ triệu đồng. Gọi $s_n$ (triệu đồng) là lương vào năm thứ $n$ mà anh Thanh làm việc cho công ty đó. Khi đó ta có
	$$s_1=200,\, s_n=s_{n-1}+25\, \text{với}\, n \ge 2.$$
	\begin{enumerate}
		\item Tính lương của anh Thanh vào năm thứ $5$ làm việc cho công ty.
		\item Chứng minh $(s_n)$ là dãy số tăng. Giải thích ý nghĩa thực tế của kết quả này. 
	\end{enumerate}
	\loigiai{
		\begin{enumerate}
			\item Ta có \begin{eqnarray*}
				s_2&=&s_1+25=200+25=225\\
				s_3&=&s_2+25=225+25=250\\
				s_4&=&s_3+25=250+25=275\\
				s_5&=&s_4+25=275+25=300. 
			\end{eqnarray*}
			Vậy lương của anh Thanh vào năm thứ $5$ làm việc cho công ty là $300$ triệu đồng.
			\item  Ta có $s_n=s_{n-1}+25\Leftrightarrow s_n-s_{n-1}=25>0$ với mọi $n\ge 2$, $n\in\mathbb{N}^\ast$.\\
			Tức là $s_n>s_{n-1}$ với mọi $n\ge 2$, $n\in\mathbb{N}^\ast$.\\
			Vậy $(s_n)$ là dãy số tăng.\\
			Điều này có nghĩa là mức lương hàng năm của anh Thanh tăng dần theo thời gian làm việc.
		\end{enumerate}
	}
\end{vd}

\begin{vd}%[1K2K5-5]%[Trương Đăng Khoa]%Ví dụ 5
	Ông An gửi tiết kiệm $100$ triệu đồng kì hạn $1$ tháng với lãi suất $6\%$ một năm theo hình thức tính lãi kép. Số tiền (triệu đồng) của ông An thu được sau $n$ tháng được cho bởi công thứC 
	$$A_n=100\left(1+\dfrac{0{,}06}{12}\right)^n.$$
	\begin{enumerate}
		\item Tìm số tiền ông An nhận được sau tháng thứ nhất, sau tháng thứ hai.
		\item Tìm số tiền ông An nhận được sau $1$ năm.
	\end{enumerate}
	\loigiai{
		\begin{enumerate}
			\item Số tiền ông An nhận được sau tháng thứ nhất là 
			$$A_1=100\left(1+\dfrac{0{,}06}{12}\right)^1=100{,}5\, \text{(triệu đồng)}.$$
			Số tiền ông An nhận được sau tháng thứ hai là 
			$$A_2=100\left(1+\dfrac{0{,}06}{12}\right)^2=101{,}0025\, \text{(triệu đồng)}.$$
			\item  Số tiền ông An nhận được sau $1$ năm ($12$ tháng) là 
			$$A_{12}=100\left(1+\dfrac{0{,}06}{12}\right)^{12} \approx 106{,}17\, \text{(triệu đồng)}.$$
		\end{enumerate}
	}
\end{vd}

\begin{vd}%[1K2G5-5]%[Trương Đăng Khoa]%Ví dụ 6
	Chị Hương vay trả góp một khoản tiền $100$ triệu đồng và đồng ý trả dần $2$ triệu đồng mỗi tháng với lãi suất $0{,}8\%$ số tiền còn lại của mỗi tháng.
	Gọi $A_n$, ($n\in\mathbb{N}$) là số tiền còn nợ (triệu đồng) của chị Hương sau $n$ tháng.
	\begin{enumerate}
		\item Tìm lần lượt $A_0$, $A_1$, $A_2$, $A_3$, $A_4$, $A_5$, $A_6$ đễ tính số tiền còn nợ của chị Hương sau $6$ tháng.
		\item  Dự đoán hệ thức truy hồi đối với dãy số $(A_n)$.
	\end{enumerate}
	\loigiai{
		\begin{enumerate}
			\item  Ta có $A_0=100$ (triệu đồng).
			\begin{itemize}
				\item Tiền lãi chị Hương phải trả sau $1$ tháng là $100\cdot 0{,}8\%=0{,}8$ (triệu đồng).\\
				Do đó, số tiền gốc chị Hương trả được sau $1$ tháng là $2-0{,}8=1{,}2$ (triệu đồng).\\
				Khi đó, số tiền còn nợ của chị Hương sau $1$ tháng là 
				$A_1=100-1{,}2=98{,}8$ (triệu đồng).
				\item  Tiền lãi chị Hương phải trả sau $2$ tháng là $98{,}8\cdot 0{,}8\%=0{,}7904$ (triệu đồng).\\
				Do đó, số tiền gốc chị Hương trả được sau $2$ tháng là $2-0{,}7904=1{,}2096$ (triệu đồng).\\
				Khi đó, số tiền còn nợ của chị Hương sau $2$ tháng là 
				$A_2=98{,}8-1{,}2096=97{,}5904$ (triệu đồng).
				\item Tiền lãi chị Hương phải trả sau $3$ tháng là $97{,}5904\cdot 0{,}8\%=0{,}7807232$ (triệu đồng).\\
				Do đó, số tiền gốc chị Hương trả được sau $3$ tháng là $2-0{,}7807232=1{,}2192768$ (triệu đồng).\\
				Khi đó, số tiền còn nợ của chị Hương sau $3$ tháng là 
				$A_3=97{,}5904-1{,}2192768=96{,}3711232$ (triệu đồng).
				\item Tiền lãi chị Hương phải trả sau $4$ tháng là $96{,}3711232\cdot 0{,}8\%\approx 0{,}77097$ (triệu đồng).\\
				Do đó, số tiền gốc chị Hương trả được sau $4$ tháng là $2-0{,}77097=1{,}22903$ (triệu đồng).\\
				Khi đó, số tiền còn nợ của chị Hương sau $4$ tháng là 
				$A_4=96{,}3711232-1{,}22903=95{,}1420932$ (triệu đồng).
				\item Tiền lãi chị Hương phải trả sau $5$ tháng là $95{,}1420932\cdot 0{,}8\%\approx 0{,}76114$ (triệu đồng).\\
				Do đó, số tiền gốc chị Hương trả được sau $5$ tháng là $2-0{,}76114=1{,}23886$ (triệu đồng).\\
				Khi đó, số tiền còn nợ của chị Hương sau $5$ tháng là $A_5=95{,}1420932-1{,}23886=93{,}9032332$ (triệu đồng).
				\item Tiền lãi chị Hương phải trả sau $6$ tháng là $93{,}9032332\cdot 0{,}8\%\approx 0{,}75123$ (triệu đồng).\\
				Do đó, số tiền gốc chị Hương trả được sau $6$ tháng là $2-0{,}75123=1{,}24877$ (triệu đồng).\\
				Khi đó, số tiền còn nợ của chị Hương sau $6$ tháng là $A_6=93{,}9032332-1{,}24877=92{,}6544632$ (triệu đồng).
			\end{itemize}
			\item Dự đoán hệ thức truy hồi đối với dãy số $(A_n)$ là 
			\[A_0=100,\, A_n=A_{n-1}-\left(2-A_{n-1} \cdot 0{,}8 \%\right)=1{,}008 A_{n-1}-2\]
		\end{enumerate}
	}
\end{vd}

%%Bài 6. CSC
\def\tenchude{CẤP SỐ CỘNG}
\setcounter{section}{5}
\setcounter{dang}{0}
\setcounter{ex}{0}
\setcounter{bt}{0}
\setcounter{vd}{0}
\section{Cấp số cộng}
\subsection{Tóm tắt lý thuyết}
\begin{tomtat}
	\subsubsection{Định nghĩa}
	Dãy số là cấp số cộng nếu mỗi một số hạng (kể từ số hạng thứ hai) đều bằng tổng của số hạng đứng ngay trước nó với một số không đổi $ d $.\\
	Dãy số $ (u_n) $ là cấp số cộng $ \Leftrightarrow u_{n+1}=u_n+d $, $ \forall n \in \mathbb{N}^* $.\\
	$ d $ là số không đổi, gọi là \textbf{\textit{công sai}} của cấp số cộng.
	\subsubsection{Tính chất}
	Nếu $ (u_n) $ là cấp số cộng thì kể từ số hạng thứ hai (trừ số hạng cuối nếu là cấp số cộng hữu hạn) đều là trung bình cộng của hai số hạng đứng kề nó trong dãy. Tức là $$u_k=\dfrac{u_{k-1}+u_{k+1}}{2}, (\forall k\ge 2, k \in \mathbb{N}^*).$$
	\subsubsection{Số hạng tổng quát}
	Nếu cấp số cộng $ (u_n) $ có số hạng đầu $ u_1 $ và công sai $ d $ thì số hạng tổng quát $ u_n $ được xác định bởi công thức $$u_n=u_1+(n-1)d \text{ với $n\ge 2$}.$$
	\subsubsection{Tổng $ n $ số hạng đầu tiên}
	Cho cấp số cộng $ (u_n) $. Tổng $ n $ số hạng đầu tiên của cấp số cộng kí hiệu là $ S_n=u_1+u_2+\ldots+u_n $.\\
	Khi đó $ S_n $ được tính theo công thức $$ S_n=\dfrac{n(u_1+u_n)}{2}=\dfrac{n}{2}\left[ 2u_1+(n-1)d\right]. $$
\end{tomtat}
\subsection{Các dạng toán thường gặp}
\begin{dang}{Nhận diện cấp số cộng, công sai $ d $, số hạng tổng quát $ u_n $}
	% Dựa theo định nghĩa của cấp số cộng, để nhận diện $ (u_n) $ là cấp số cộng $ \Leftrightarrow u_{n+1}=u_n+d $.\\
	% Khi đó công sai $ d=u_{n+1}-u_{n} $, $ \forall n \in \mathbb{N}^* $.
\end{dang}
\subsubsection{Ví dụ minh hoạ}
\begin{vd}%[NB]%[DCHT Toán 11 - KNTT -Lê Hải Phụng] %[1K2Y6-1]
	Dãy số hữu hạn nào là một cấp số cộng? Vì sao?
	\begin{listEX}[2]
		\item  $-2$, $1$, $4$, $7$, $10$, $13$, $16$.
		\item  $ 1 $, $ -2 $, $ -4 $, $ -6 $, $ -8 $.
	\end{listEX}
	\dapso{ Dãy số 1 là một cấp số cộng, dãy số 2 không là một cấp số cộng.}
	\loigiai{
		\begin{enumerate}
			\item Ta thấy $ u_2=u_1+3 $  do $ 1=(-2)+3 $.\\
			Vì $ u_k=u_{k-1}+d,\ \forall k\geq2$ $\left(\ 1=\left(-2\right)+3;4=1+3;7=4+3;10=7+3;13=10+3;16=13+3\right) $ nên dãy số đã cho là cấp số cộng. 
			\item Ta thấy $ u_2=u_1+(-3) $  do $-2=1+(-3)$.\\
			Vì $ {u_3\neq u}_2+(-3) $ bởi $ \left(\ -4\neq-2+\left(-3\right)\right)\ $ nên dãy số đã cho không là cấp số cộng.
		\end{enumerate}
	}
\end{vd}
\begin{vd}%[TH]%[DCHT Toán 11 - KNTT -Lê Hải Phụng] %[1K2B6-1]
	Trong các dãy số dưới đây, dãy số nào là cấp số cộng? 
	\begin{listEX}[2]
		\item  Dãy số $\left({a_n}\right)$ với ${a_n}=4n-3$;
		\item  Dãy số $\left({c_n}\right)$ với ${c_n}={2018^n}$.
	\end{listEX}
	\dapso{Dãy số 1 là một cấp số cộng, dãy số 2 không là một cấp số cộng.}
	\loigiai{	
		\begin{enumerate}
			\item Ta có $a_{n+1}=4(n+1)-3=4n+1$ nên $a_{n+1}-a_n=(4n+1)-(4n-3)=4$,$\forall n\ge 1.$.\\
			Do đó $(a_n)$ là cấp số cộng.
			\item Ta có $c_{n+1}=2018^{n+1}$ nên $c_{n+1}-c_n=2018^{n+1}-2018^n=2017\cdot 2018^n$ (phụ thuộc vào giá trị của $n$).\\ 
			Suy ra $(c_n)$ không phải là một cấp số cộng.
		\end{enumerate}	
	}
\end{vd}
\begin{vd}%[NB]%[DCHT Toán 11 - KNTT -Lê Hải Phụng] %[1K2Y6-1]
	Cho cấp số cộng $(u_n)$  có công thức số hạng tổng quát $u_n=3n+1$, $n\in\mathbb{N}^\ast$ . Tìm số hạng đầu $u_1$ và công sai $d$?
	\dapso{$u_1=4 $, $d=3$.}
	\loigiai{
		Từ công thức số hạng tổng quát, ta có $ u_1=4 $, $u_2=7$ suy ra $d=u_2-u_1=3$.
	}
\end{vd}

\begin{vd}%[TH]%[DCHT Toán 11 - KNTT -Lê Hải Phụng] %[1K2B6-1]
	Cho cấp số cộng $(u_n)$ với $u_1=3$, $u_2=9$. Công sai của cấp số cộng đã cho bằng bao nhiêu?
	\dapso{$ d=6 $}
	\loigiai{
		Cấp số cộng $(u_n)$ có số hạng tổng quát là $u_n=u_1+(n-1)d$ với $n \ge 2$.\\
		Suy ra $u_2=u_1+d \Leftrightarrow 9=3+d \Leftrightarrow d=6$.\\
		Vậy công sai của cấp số cộng đã cho là $6$.
	}
\end{vd}
\begin{vd}%[VD]%[DCHT Toán 11 - KNTT -Lê Hải Phụng] %[1K2K6-1]
	Tính số hạng đầu $u_1$ và công sai $d$ của một cấp số cộng biết $u_4=10$ và $u_7=19$.
	\dapso{$ u_1=1 $, $ d=3 $.}
	\loigiai{Ta có $ \heva{& u_4=10 \\ & u_7=19} \Leftrightarrow \heva{& u_1+3d=10 \\ & u_1+6d=19} \Leftrightarrow \heva{& u_1=1 \\ & d=3.}$}
\end{vd}

\begin{vd}%[TH]%[Dự án DCHT-11-KNTT]%[Dao-V- Thuy]%[1K2B5-1]
	Xác định số hạng tổng quát của cấp số cộng $(u_n),$ biết $\heva{&u_7=8\\ &d=2.}$
	\dapso{$u_n=2n-6$}
	\loigiai{
		Ta có
		\begin{equation*}
			\heva{&u_7=8\\ &d=2} \Leftrightarrow \heva{&u_1+6d=8\\&d=2} \Leftrightarrow \heva{&u_1=-4\\ &d=2.}
		\end{equation*}
		Vậy công thức tổng quát của cấp số cộng
		\begin{center}
			$u_n=-4+(n-1)2 \Leftrightarrow u_n=2n-6 $ với $n \geq 2.$
		\end{center}	
	}
\end{vd}

% \begin{vd}%[TH]%[Dự án DCHT-11-KNTT]%[Dao-V- Thuy]%[1K2B5-1]
% 	Tìm số hạng đầu và công sai của cấp số cộng $(u_n)$, biết $\heva{&u_1+u_5-u_3=10\\ &u_1+u_6=17.}$
% 	\dapso{$u_1=16$, $d=-3$}
% 	\loigiai{
% 		Ta có
% 		\begin{align*}
% 			\heva{&u_1+u_5-u_3=10\\ &u_1+u_6=17} &\Leftrightarrow \heva{&u_1+u_1+4d-(u_1+2d)=10\\ &u_1+u_1+5d=17}\\ & \Leftrightarrow\heva{&u_1+2d=10 \\ &2u_1+5d=17} \Leftrightarrow \heva{&u_1=16 \\ &d=-3.}
% 		\end{align*}
% 		Vậy $u_1=16$, $d=-3$.
% 	}
% \end{vd}

\begin{vd}%[TH]%[Dự án DCHT-11-KNTT]%[Dao-V- Thuy]%[1K2B5-1]
	Cho cấp số cộng $(u_n)$ với $\heva{&u_1=-9\\ &u_{n-1}=u_n-5}$. Tìm số hạng tổng quát của cấp số cộng $(u_n)$.
	\dapso{$u_n= 5n-14$}
	\loigiai{
		Từ công thức $u_{n-1}=u_n-5 \Leftrightarrow u_n= u_{n-1}+5$, suy ra $d=5$.\\
		Vậy công thức tổng quát của cấp số cộng $(u_n)$ là $u_n=-9 + 5(n-1)= 5n-14$.
	}
\end{vd}

\begin{vd}%[TH]%[Dự án DCHT-11-KNTT]%[Dao-V- Thuy]%[1K2B5-1]
	Cho cấp số cộng $(u_n)$ có $u_{20}=-52$ và $u_{51}=-145$. Hãy tìm số hạng tổng quát của cấp số cộng đó.
	\dapso{$u_n= -3n+8$}
	\loigiai{
		Ta có
		\begin{eqnarray*}
			\heva{&u_{20}=-52 \\&u_{51}=-145} &\Leftrightarrow& \heva{&u_1+19d=-52\\ &u_1+50d=-145} \Leftrightarrow \heva{&u_1=5 \\ &d= -3.}
		\end{eqnarray*}
		Vậy số hạng tổng quát cần tìm là $u_n= u_1+ (n-1)d= 5+(n-1) \cdot (-3)= -3n+8$.
	}
\end{vd}

\begin{vd}%[VD]%[Dự án DCHT-11-KNTT]%[Dao-V- Thuy]%[1K2B5-1]
	Tìm số hạng đầu và công sai của cấp số cộng $(u_n)$, biết
		\begin{listEX}[2]
			\item $\heva{&u_9=5u_2\\ &u_{13}=2u_6+5.}$
			\item $\heva{&u_1-u_3+u_5=10\\ &u_1+u_6=7.}$
		\end{listEX}
	\dapso{$u_1=3$, $d=4$; $u_1=36$, $d=-13$}
	\loigiai
	{
		\begin{enumerate}
			\item Ta có
			\begin{eqnarray*}
				\heva{&u_9=5u_2\\ &u_{13}=2u_6+5} &\Leftrightarrow& \heva{&u_1+8d= 5 \left( u_1+d \right) \\ &u_1+12d = 2 \left( u_1+5d\right) + 5}\\
				&\Leftrightarrow& \heva{&-4u_1+3d=0\\ &-u_1+2d=5} \Leftrightarrow \heva{&u_1=3 \\ &d=4.}
			\end{eqnarray*}
			Vậy $u_1=3$, $d=4$.
			\item Ta có 
			\begin{eqnarray*}
				\heva{&u_1-u_3+u_5=10\\ &u_1+u_6=7} &\Leftrightarrow& \heva{&u_1-\left( u_1+2d\right) + \left( u_1+4d\right) = 10\\ &u_1+ \left( u_1+5d\right) = 7}\\
				&\Leftrightarrow& \heva{&u_1+2d=10\\ &2u_1+5d=7} \Leftrightarrow \heva{&u_1=36 \\ &d=-13.}
			\end{eqnarray*}
			Vậy $u_1=36$, $d=-13$.
		\end{enumerate}
	}
\end{vd}

\begin{vd}%[VD]%[Dự án DCHT-11-KNTT]%[Dao-V- Thuy]%[1K2K5-1]
	Tìm số hạng đầu và công sai của cấp số cộng $(u_n)$, biết
		\begin{listEX}[2]
			\item $\heva{&-u_3+u_7=8\\ &u_2u_7=75.}$
			\item $\heva{&u_5=4u_3\\ &u_2u_6=-11.}$
		\end{listEX}
	\dapso{$\heva{&u_1=3\\ &d=2} \text{ hoặc } \heva{&u_1=-17\\ &d=2}$; $\heva{&u_1=-4\\ &d=3}$ hoặc $\heva{&u_1=4\\ &d=-3}$}
	\loigiai{
		\begin{enumerate}
			\item Ta có
			\begin{eqnarray*}
				\heva{&-u_3+u_7=8\\ &u_2u_7=75} &\Leftrightarrow& \heva{&-\left( u_1+2d\right) + \left( u_1+6d\right) = 8\\ &\left( u_1+d\right) \left( u_1+6d\right) = 75}\\
				&\Leftrightarrow& \heva{&4d=8\\ &u_1^2+7u_1d+6d^2=75}\\
				&\Leftrightarrow& \heva{&d=2\\ &u_1^2+14u_1-51=0}\\
				&\Leftrightarrow& \heva{&u_1=3\\ &d=2} \text{ hoặc } \heva{&u_1=-17\\ &d=2.}
			\end{eqnarray*}
			Vậy $\heva{&u_1=3\\ &d=2} \text{ hoặc } \heva{&u_1=-17\\ &d=2.}$
			\item Ta có 
			\begin{eqnarray*}
				\heva{&u_5=4u_3\\ &u_2u_6=-11} &\Leftrightarrow& \heva{&u_1+4d=4 \left( u_1+2d\right)\\ &\left( u_1+d\right) \left( u_1+5d\right)= -11}\\
				&\Leftrightarrow& \heva{&3u_1+4d=0 &(1)\\ &u_1^2+6du_1+5d^2=-11 &(2)}
			\end{eqnarray*}
			Từ $(1)$ suy ra $3u_1=-4d$. Thay vào $(2)$ ta được
			\begin{eqnarray*}
				9u_1^2+54du_1+45d^2=-99 &\Leftrightarrow& 16d^2 -72d^2+45d^2=-99\\
				&\Leftrightarrow& -11d^2=-99 \Leftrightarrow \hoac{&d=3\\ &d=-3.}
			\end{eqnarray*}
			Với $d=3$, ta có $u_1=-4$.\\
			Với $d=-3$, ta có $u_1=4$.\\
			Vậy $\heva{&u_1=-4\\ &d=3}$ hoặc $\heva{&u_1=4\\ &d=-3.}$
		\end{enumerate}
	}
\end{vd}

\subsubsection{Bài tập tự luận}
 

\begin{bt}%[NB]%[DCHT Toán 11 - KNTT -Lê Hải Phụng] %[1K2Y6-1]
	Trong các dãy số sau, dãy số nào là một cấp số cộng?
	\begin{listEX}[1]
		\item $1$, $-3$, $-7$, $-11$, $-15$, $\ldots$;
		\item $1$, $-2$, $-4$, $-6$, $-8,$ $\ldots$.
		\item $ \dfrac{1}{2} $, $0$, $-\dfrac{1}{2}$, $-1$, $-\dfrac{3}{2}$, $\ldots$
	\end{listEX}
	\dapso{1) và 3) là cấp số cộng.}
	\loigiai{Ta lần lượt đi kiểm tra: $ u_2-u_1=u_3-u_2=u_4-u_3=\ldots $?\\
		Xét từng dãy số thì ta thấy 1) và 3) là cấp số cộng. 
	}
\end{bt}

\begin{bt}%[NB]%[DCHT Toán 11 - KNTT -Lê Hải Phụng] %[1K2Y6-1]
	Trong các dãy số sau, dãy nào là cấp số cộng. Tìm số hạng đầu và công sai của cấp số cộng đó.
	\begin{listEX}[2]
		\item Dãy số $ (u_n) $ với $ u_n=19n-5 $;
		\item Dãy số $ (u_n) $ với $ u_n=n^2+n+1 $. 
	\end{listEX}
	\dapso{Dãy số 1) $ (u_n) $ là một cấp số cộng với số hạng đầu là $ u_1=19\cdot1-5=14 $ và công sai $ d=19 $. Dãy số 2) không là một cấp số cộng.}
	\loigiai{
		\begin{enumerate}
			\item Dãy số $ (u_n) $ với $ u_n=19n-5 $.\\
			Ta có $ u_{n+1}-u_n=19(n+1)-5-(19n-5)=19 $. Vậy $ (u_n) $ là một cấp số cộng với số hạng đầu là $ u_1=19\cdot1-5=14 $ và công sai $ d=19 $.
			\item Dãy số $ (u_n) $ với $ u_n=n^2+n+1 $.\\
			Ta có $ u_{n+1}-u_n=(n+1)^2+(n+1)+1-(n^2+n+1)=2n+2$ phụ thuộc vào $ n $. Vậy $ (u_n) $ không là một cấp số cộng.
	\end{enumerate}}
\end{bt}

\begin{bt}%[TH]%[DCHT Toán 11 - KNTT -Lê Hải Phụng] %[1K2B6-1]
	Cho cấp số cộng $\left(u_n\right)$ với $u_1=3$, $u_2=9$. Công sai của cấp số cộng đã cho bằng bao nhiêu?
	\dapso{Công sai của cấp số cộng đã cho là 6.}
	\loigiai{Cấp số cộng $(u_n)$ có số hạng tổng quát là
		$u_n=u_1+\left(n-1\right)d$ với $n \ge 2$
		(số hạng đầu $u_1$ và công sai $d$)\\
		Suy ra $ u_2=u_1+d\Leftrightarrow9=3+d\Leftrightarrow d=6 $.\\
		Vậy công sai của cấp số cộng đã cho là 6.
	}
\end{bt}


\begin{bt}%[TH]%[Dự án DCHT-11-KNTT]%[Dao-V- Thuy]%[1K2B5-1]
	Xác định công thức tổng quát của cấp số cộng $(u_n)$, biết $\heva{&u_{11}=5\\ &d=-6.}$
	\loigiai{
		Ta có
		\begin{equation*}
			\heva{&u_{11}=5\\ &d=-6} \Leftrightarrow \heva{&u_1+10d=5\\ &d=-6} \Leftrightarrow \heva{&u_1=65\\ &d=-6.}
		\end{equation*}
		Vậy công thức tổng quát của cấp số cộng:
		\begin{center}
			$u_n=65+(n-1).(-6) \Leftrightarrow u_n=-6n+71$  với $n \geq 2.$
		\end{center}	
	}
\end{bt}

\begin{bt}%[TH]%[Dự án DCHT-11-KNTT]%[Dao-V- Thuy]%[1K2B5-1]
	Tìm số hạng đầu và công sai của cấp số cộng $(u_n),$ biết $\heva{&u_2+u_5-u_3=10\\ &u_4+u_6=26.}$
	\loigiai{
		Ta có
		\begin{align*}
			\heva{&u_2+u_5-u_3=10\\ &u_4+u_6=26} 
			&\Leftrightarrow \heva{&u_1+d+u_1+4d-(u_1+2d)=10\\ &u_1+3d+u_1+5d=26}\\ 
			& \Leftrightarrow\heva{&u_1+3d=10 \\ &2u_1+8d=26}
			\Leftrightarrow \heva{&u_1=1 \\&d=3.}
		\end{align*}
		Vậy $u_1=1$, $d=3$.
	}
\end{bt}

\begin{bt}%[TH]%[Dự án DCHT-11-KNTT]%[Dao-V- Thuy]%[1K2B5-1]
	Tìm số hạng đầu và công sai của cấp số cộng, biết
		\begin{listEX}[3]
			\item $\heva{&u_7 = 27\\&u_{15} = 59.}$
			\item $\heva{&u_9 = 5u_2\\&u_{13} = 2u_6 + 5.}$
			\item $\heva{&u_2 + u_4 - u_6 = -7\\&u_8 - u_7 = 2u_4.}$
			\item $\heva{&u_3 - u_7 = -8\\&u_2 \cdot u_7 = 75.}$
			\item $\heva{&u_6 + u_7 = 60\\&u_4^2 + u_{12}^2 = 1170.}$
		\end{listEX}
	\loigiai{
		\begin{enumerate}
			\item Ta có $\heva{&u_7 = 27\\&u_{15} = 59} \Leftrightarrow \heva{&u_1 + 6d = 27\\&u_1 + 14d = 59} \Leftrightarrow \heva{&u_1 = 3\\&d = 4.}$\\
			Vậy số hạng đầu của cấp số cộng là $u_1 = 3$, công sai là $d = 4$.
			\item Ta có $\heva{&u_9 = 5u_2\\&u_{13} = 2u_6 + 5} \Leftrightarrow \heva{&u_1 + 8d = 5u_1 + 5d\\&u_1 + 12d = 2u_1 + 10d + 5} \Leftrightarrow \heva{&4u_1 - 3d = 0\\&-u_1 + 2d = 5} \Leftrightarrow \heva{&u_1 = 3\\&d = 4.}$\\
			Vậy số hạng đầu của cấp số cộng là $u_1 = 3$, công sai là $d = 4$.
			\item Ta có $\heva{&u_2 + u_4 - u_6 = -7\\&u_8 - u_7 = 2u_4} \Leftrightarrow \heva{&u_1 + d + u_1 + 3d - u_1 - 5d = -7\\&u_1 + 7d - u_1 - 6d = 2u_1 + 6d} \Leftrightarrow \heva{&u_1 - d = -7\\&2u_1 + 5d = 0} \Leftrightarrow \heva{&u_1 = -5\\&d = 2.}$\\
			Vậy số hạng đầu của cấp số cộng là $u_1 = -5$, công sai là $d = 2$.
			\item Ta có $\heva{&u_3 - u_7 = -8\\&u_2 \cdot u_7 = 75} \Leftrightarrow \heva{&u_1 + 2d -u_1 - 6d = -8\\&(u_1 + d)(u_1 + 6d) = 75} \Leftrightarrow \heva{&d = 2\\&u_1^2 + 14u_1 - 51 = 0} \Leftrightarrow \heva{&d = 2\\&\hoac{&u_1 = 3\\&u_1 = -17.}}$\\
			Vậy số hạng đầu của cấp số cộng là $u_1 = 3$, công sai là $d = 2$ hoặc $u_1 = -17$, $d = 2$.
			\item Ta có $\heva{&u_6 + u_7 = 60\\&u_4^2 + u_{12}^2 = 1170} \Leftrightarrow \heva{&2u_6 + d = 60&(1)\\&(u_6 - 2d)^2 + (u_6 + 6d)^2 = 1170.&(2)}$\\
			Từ (1), suy ra $d = 60 - 2u_6$, thay vào (2), ta có
			$$(5u_6 - 120)^2 + (360 - 11u_6)^2 = 1170 \Leftrightarrow 146u_6^2 - 9120u_6 + 142830 = 0 \,\, (\text{vô nghiệm}).$$ 
			Vậy không tồn tại cấp số cộng thỏa yêu cầu bài toán.
		\end{enumerate}
	}
\end{bt}
% \begin{bt}%[TH]%[DCHT Toán 11 - KNTT -Lê Hải Phụng] %[1K2B6-1]
% 	Tìm số hạng đầu tiên, công sai của cấp số cộng sau $ \heva{&u_5=19\\&u_9=35.}$
	
% 	\dapso{Số hạng đầu tiên $ u_1=3 $, công sai $ d=4 $.}
% 	\loigiai{Áp dụng công thức $ u_n=u_1+(n-1)d $ ta có $\heva{&u_5=19\\&u_9=35} \Leftrightarrow \heva{&u_1+4d=19\\&u_1+8d=35} \Leftrightarrow \heva{&u_1=3\\&d=4.}$\\
% 	Vậy số hạng đầu tiên $ u_1=3 $, công sai $ d=4 $.}
% \end{bt}

\begin{bt}%[VD]%[1K2K6-1]
	Cho cấp số cộng $ (u_n) $ thỏa mãn $ \heva{&u_2+u_4-u_6=-7\\&u_8+u_7=2u_4} $. Xác định số hạng đầu $ u_1 $ và công sai $ d $ cấp số cộng.        
	
	% \dapso{$ u_1=5 $, $ d=2 $.}
	\loigiai{
		Ta có $ \heva{&u_2+u_4-u_6=-7\\&u_8+u_7=2u_4} \Leftrightarrow \heva{& u_1+d+(u_1+3d)-(u_1+5d)=-7 \\ & u_1+7d-(u_1+6d)=2(u_1+3d)} \Leftrightarrow \heva{& u_1-d=-7 \\ & 2u_1+5d=0} \Leftrightarrow \heva{&u_1=-5\\&d=2.}$}
\end{bt}

\begin{bt}%[VD]%[DCHT Toán 11 - KNTT -Lê Hải Phụng] %[1K2K6-1]
Cho cấp số cộng $ (u_n) $ thỏa mãn $ \heva{&u_2-u_3+u_5=10\\&u_4+u_6=26} $. Xác định số hạng đầu $ u_1 $ và công sai $ d $ cấp số cộng.         
\dapso{$ u_1=1 $, $ d=3 $.}
\loigiai{Ta có $ \heva{&u_2-u_3+u_5=10\\&u_4+u_6=26} \Leftrightarrow \heva{& u_1+d-(u_1+2d)+u_1+4d=10 \\ & u_1+3d+u_1+5d=26} \Leftrightarrow \heva{& u_1+3d=10 \\ & u_1+4d=13} \Leftrightarrow \heva{u_1=1\\d=3.}$}
\end{bt}

\begin{bt}%[VDC]%[DCHT Toán 11 - KNTT -Lê Hải Phụng] %[1K2G6-1]
Tính số hạng đầu $ u_1 $ và công sai $d$ của một cấp số cộng biết $ \heva{&u_1+u_2+u_3=27\\&u_1^2+u_2^2+u_3^2=275} $

\dapso{$ u_1=5 $, $ d=4 $ hoặc $ u_1=13 $, $ d=-4 $.}
\loigiai{Ta có $ \heva{&u_1+u_2+u_3=27\\&u_1^2+u_2^2+u_3^2=275} \Leftrightarrow \heva{&u_2-d+u_2+u_2+d=27\\&(u_2-d)^2+u_2^2+(u_2+d)^2=275}\Leftrightarrow \heva{&u_2=9\\&3u_2^2+2d^2=275.}$\\
Thay $ u_2=9 $ vào $ 3u_2^2+2d^2=275 $ ta được $ d=4 $ hay $ d=-4 $.\\
Vậy $ u_1=5 $, $ d=4 $ hoặc $ u_1=13 $, $ d=-4 $.}
\end{bt}
\subsubsection{Câu hỏi trắc nghiệm}
\Opensolutionfile{ans}[ans/ans-1K2-2-Dang1]

\begin{ex}%[DCHT Toán 11 - KNTT -Lê Hải Phụng] %[1K2Y6-1]
Trong các dãy số sau, dãy số nào là một cấp số cộng?
\choice
{\True $ 1 $; $ -3 $; $ -7 $; $ -11 $; $ -15 $; $ \ldots $}
{$ 1 $; $ -3 $; $ -6 $; $ -9 $; $ -12 $; $ \ldots $}
{$ 1 $; $ -2 $; $ -4 $; $ -6 $; $ -8 $; $ \ldots $}
{$ 1 $; $ -3 $; $ -5 $; $ -7 $; $ -9 $; $ \ldots $}
\loigiai
{
	Ta lần lượt tính khoảng cách $ d $ các phần tử, ta thấy dãy số đáp án A có $ d= -4$.
}
\end{ex}
%Cau2
\begin{ex}%[DCHT Toán 11 - KNTT -Lê Hải Phụng] %[1K2Y6-1]
Dãy số nào sau đây \textbf{không} phải là cấp số cộng?
\choice
{$ -\dfrac{2}{3} $; $ -\dfrac{1}{3} $; $ 0 $; $ \dfrac{1}{3} $; $ \dfrac{2}{3} $; $ 1 $; $ \dfrac{4}{3} $}
{$ 15\sqrt{2} $; $ 12\sqrt{2} $; $ 9\sqrt{2} $; $ 6\sqrt{2} $}
{\True $ \dfrac{4}{5} $; $ 1 $; $ \dfrac{7}{5} $; $ \dfrac{9}{5} $; $ \dfrac{11}{5} $}
{$ \dfrac{1}{\sqrt{3}} $; $ \dfrac{2\sqrt{3}}{3} $; $ \sqrt{3} $; $ \dfrac{4\sqrt{3}}{3} $; $ \dfrac{5}{\sqrt{3}} $}
\loigiai
{
	Ta lần lượt tính khoảng cách $ d $ các phần tử, ta thấy dãy số trừ đáp án C có khoảng cách các phần tử không bằng nhau.
}
\end{ex}
%Cau3
\begin{ex}%[DCHT Toán 11 - KNTT -Lê Hải Phụng] %[1K2Y6-1]
Cho cấp số cộng $ (u_n) $ với $ u_1=2 $ và $ u_2=6 $. Công sai của cấp số cộng đã cho là	
\choice
{\True $ 4 $}
{$ -4 $}
{$ 8 $}
{$ 3 $}
\loigiai
{
	Ta có $ u_2=6 \Leftrightarrow 6=u_1+d \Leftrightarrow d=4 $.
}
\end{ex}
%Cau4
\begin{ex}%[DCHT Toán 11 - KNTT -Lê Hải Phụng] %[1K2Y6-1]
Cho cấp số cộng $ (u_n) $ với $ u_1=-3 $ và $ u_6=27 $. Công sai $ d $ của cấp số cộng đã cho là	
\choice
{$ d=7 $}
{$ d=5 $}
{$ d=8 $}
{\True $ d=6 $}
\loigiai
{
	Ta có $ u_6=27 \Leftrightarrow 27=u_1+5d \Leftrightarrow d=6 $.
}
\end{ex}
%Cau5
\begin{ex}%[DCHT Toán 11 - KNTT -Lê Hải Phụng] %[1K2B6-1]
Cho cấp số cộng $ (u_n) $ với $ u_{17}=33 $ và $ u_{33}=65 $. Công sai của cấp số cộng đã cho là	
\choice
{$ 1 $}
{$ 3 $}
{$ -2 $}
{\True $ 2 $}
\loigiai
{
	Gọi $ u_1 $, $ d $ lần lượt là số hạng đầu và công sai của cấp số cộng $ (u_n) $.\\
	Khi đó, ta có $ u_{17}=u_1+16d $, $ u_{33}=u_1+32d $\\
	Suy ra $ u_{33}-u_{17}=65-33 \Leftrightarrow 16d=32 \Leftrightarrow d=2 $\\
	Vậy công sai bằng $ 2 $.
}
\end{ex}
%Cau6
\begin{ex}%[DCHT Toán 11 - KNTT -Tên GV] %[1K2B6-1]
Cho cấp số cộng có $ u_1=-3 $ và $ d=4 $. Chọn khẳng định đúng trong các khẳng định sau.
\choice
{$ u_5=15 $}
{$ u_4=8 $}
{\True $ u_3=5 $}
{$ u_2=2 $}
\loigiai
{
	Ta có $ u_3=u_1+2d=-3+2\cdot4=5 $.
}
\end{ex}
%Cau7
\begin{ex}%[DCHT Toán 11 - KNTT -Tên GV] %[1K2Y6-1]
Cho cấp số cộng có $ u_1=11 $ và công sai $ d=4 $. Hãy tính $ u_{99} $.
\choice
{$ 401 $}
{\True $ 403 $}
{$ 402 $}
{$ 404 $}
\loigiai
{
	Ta có $ u_{99}=u_1+98d=11+98\cdot4=403 $.
}
\end{ex}
%Cau8
\begin{ex}%[DCHT Toán 11 - KNTT -Tên GV] %[1K2B6-1]
Một cấp số cộng $ (u_n) $ có $ u_{13}=8 $ và $ d=-3 $. Tìm số hạng thứ ba của cấp số cộng $ (u_n) $.
\choice
{$ 50 $}
{$ 28 $}
{\True $ 38 $}
{$ 44 $}
\loigiai
{
	Ta có $ u_{13}=u_1+12d \Leftrightarrow 8=u_1+12\cdot(-3)\Rightarrow u_1=44 \Rightarrow u_{3}=u_1+2d=44-6=38$.
}
\end{ex}
%Cau9
\begin{ex}%[DCHT Toán 11 - KNTT -Tên GV] %[1K2Y6-1]
Cho cấp số cộng $(u_n) $ có số hạng đầu $ u_1=2 $ và công sai $ d=4 $. Hãy tính giá trị $ u_{2019} $ bằng
\choice
{\True $ 8074 $}
{$ 4074 $}
{$ 8078 $}
{$ 4078 $}
\loigiai
{
	Ta có $ u_{2019}=u_1+2018d=2+2018\cdot 4=8074 $.
}
\end{ex}
%Cau10
\begin{ex}%[DCHT Toán 11 - KNTT -Tên GV] %[1K2K6-1]
Cho cấp số cộng $ (u_n) $ có số hạng tổng quát là $ u_n=3n-2 $. Tìm công sai $ d $ của cấp số cộng.
\choice
{\True $ d=3 $}
{$ d=2 $}
{$ d=-2 $}
{$ d=-3 $}
\loigiai
{
	Ta có $ u_{n+1}-u_n=3(n+1)-2-3n+2=3 $. Suy ra công sai $ d=3 $.
}
\end{ex}

\begin{ex}%[Dự án DCHT-11-KNTT]%[Dao-V- Thuy]%[1K2Y5-1]
	Cho cấp số cộng $(u_n)$ có số hạng đầu $u_1$ và công sai $d$. Công thức tìm số hạng tổng quát $u_n$ là 
	\choice
	{\True $u_n=u_1+(n-1)d$}
	{$u_n=u_1+nd$}
	{$u_n=u_1+(n+1)d$}
	{$u_n=nu_1+d$}
	\loigiai{
		Ta có $u_n=u_1+(n-1)d$.
	}
\end{ex}

\begin{ex}%[Dự án DCHT-11-KNTT]%[Dao-V- Thuy]%[1K2Y5-1]
	Cho cấp số cộng $(u_n)$ có $u_1=-3$ và $d=\dfrac{1}{2}$. Khẳng định nào sau đây đúng?
	\choice
	{$u_n=-3+\dfrac{1}{2}(n+1 )$}
	{$u_n=-3+\dfrac{1}{2}n-1$}
	{\True $u_n=-3+\dfrac{1}{2}(n-1)$}
	{$u_n=-3+\dfrac{1}{4}(n-1 )$}
	\loigiai {
		Ta có $\heva{
			&u_1=-3 \\
			& d=\dfrac{1}{2} \\
		}\xrightarrow{CTTQ} u_n=u_1+(n-1 )d=-3+\dfrac{1}{2}(n-1 )$.}
\end{ex}

\begin{ex}%[Dự án DCHT-11-KNTT]%[Dao-V- Thuy]%[1K2Y5-1]
	Cho cấp số cộng $\left(u_n\right)$ xác định bởi $u_n=2n+1$. Xác định số hạng đầu $u_1$ và công sai $d$ của cấp số cộng.
	\choice
	{$u_1=3$, $d=1$}
	{$u_1=1$, $d=1$}
	{\True $u_1=3$, $d=2$}
	{$u_1=1$, $d=2$}
	\loigiai{
		Ta có $u_1=2\cdot 1+1=3$ và $u_2=2\cdot 2+1=5$, nên $d=u_2-u_1=2$.
	}
\end{ex}

\begin{ex}%[Dự án DCHT-11-KNTT]%[Dao-V- Thuy]%[1K2B5-1]
	Cho cấp số cộng $\left(u_n\right)$ có $u_4=-12$, $u_{14}=18$. Tìm số hạng đầu $u_1$ và công sai $d$ của cấp số cộng $\left(u_n\right)$. 
	\choice 
	{$u_1=-20$, $d=-3$}
	{$u_1=-22$, $d=3$ }
	{\True $u_1=-21$, $d=3$}
	{$u_1=-21$, $d=-3$}
	\loigiai{
		Ta có $$\heva{&u_4=u_1+(4-1)d\\&u_{14}=u_1+(14-1)d} \Leftrightarrow \heva{&-12=u_1+3d\\&18=u_1+13d}\Leftrightarrow \heva{&u_1=-12\\&d=3.}$$
	}
\end{ex}

\begin{ex}%[Dự án DCHT-11-KNTT]%[Dao-V- Thuy]%[1K2B5-1]
	Tìm số hạng đầu và công sai của cấp số cộng $(u_n)$ thỏa mãn $\heva{&u_1+u_9=12\\&u_4-3u_2=1.}$
	\choice
	{$u_1=\dfrac{1}{2}$; $d=\dfrac{13}{8}$}
	{$u_1=-1$; $d=\dfrac{13}{8}$}
	{\True $u_1=-\dfrac{1}{2}$; $d=\dfrac{13}{8}$}
	{$u_1=-1$; $d=2$}
	\loigiai{Ta có: $\heva{&u_1+u_9=12\\&u_4-3u_2=1}\Leftrightarrow\heva{&u_1+(u_1+8d)=12\\&(u_1+3d)-3(u_1+d)=1}\Leftrightarrow\heva{&2u_1+8d=12\\&-2u_1=1}\Leftrightarrow\heva{&d=\dfrac{13}{8}\\&u_1=-\dfrac{1}{2}}$}
\end{ex}

\begin{ex}%[Dự án DCHT-11-KNTT]%[Dao-V- Thuy]%[1K2B5-1]
	Cho cấp số cộng $(u_n)$ có $u_4=-12$ và $u_{14} =18$. Khi đó, số hạng đầu tiên $u_1$ và công sai $d$ của cấp số cộng $(u_n)$ lần lượt là
	\choice
	{$u_1=-20$, $d=-3$}
	{$u_1=-22$, $d=3$}
	{\True $u_1=-21$, $d=3$}
	{$u_1=-21$, $d=-3$}
	\loigiai{Ta có: $\heva{&u_4=-12\\&u_{14}=18}\Leftrightarrow\heva{&u_1+3d=-12\\&u_1+13d=18}\Leftrightarrow\heva{&u_1=-21\\&d=3.}$}
\end{ex}

\begin{ex}%[Dự án DCHT-11-KNTT]%[Dao-V- Thuy]%[1K2B5-1]
	Cho cấp số cộng $(u_n )$ có các số hạng đầu lần lượt là $5;\,9;\,13;\,17;\ldots $. Tìm số hạng tổng quát $u_n$ của cấp số cộng.
	\choice
	{$u_n=5n+1$}
	{$u_n=5n-1$}
	{\True $u_n=4n+1$}
	{$u_n=4n-1$}
	\loigiai{
		Cấp số cộng đã cho có $u_1=5$, $ d=u_2-u_1=4 $. Suy ra $u_n=u_1+(n-1 )d=5+4(n-1 )=4n+1$.
		}
\end{ex}

\begin{ex}%[Dự án DCHT-11-KNTT]%[Dao-V- Thuy]%[1K2B5-1]
	Cho cấp số cộng $(u_n)$ có $u_3=15$ và $d=-2$. Tìm $u_n$.
	\choice
	{\True $u_n=-2n+21$}
	{$u_n=-\dfrac{3}{2}n+12$}
	{$u_n=-3n-17$}
	{$u_n=\dfrac{3}{2}{{n}^2}-4$}
	\loigiai {
		Ta có $\heva{ & 15=u_3=u_1+2d \\& d=-2}
		\Leftrightarrow \heva{&u_1=19 \\& d=-2}
		\Rightarrow u_n=u_1+(n-1 )d=-2n+21$.
		}
\end{ex}

\begin{ex}%[Dự án DCHT-11-KNTT]%[Dao-V- Thuy]%[1K2B5-1]
	Trong các dãy số được cho dưới đây, dãy số nào {\bf không} phải là cấp số cộng?
	\choice
	{$u_n=-4n+9$}
	{$u_n=-2n+19$}
	{$u_n=-2n-21$}
	{\True $u_n=-2^n+15$}
	\loigiai {
		Dãy số $u_n=-2^n+15$ không có dạng $an+b$ nên có không phải là cấp số cộng.}
\end{ex}

\begin{ex}%[Dự án DCHT-11-KNTT]%[Dao-V- Thuy]%[1K2B5-1]
	Cho cấp số cộng $(u_n)$ có $u_4=-12$ và $u_{14}=18$. Tìm số hạng đầu tiên $u_1$ và công sai $d$ của cấp số cộng đã cho.
	\choice
	{\True $u_1=-21$; $d=3$}
	{$u_1=-20$; $d=-3$}
	{$u_1=-22$; $d=3$}
	{$u_1=-21$; $d=-3$}
	\loigiai {
		Ta có 
		$\heva{&u_4=-12\\ &u_{14}=18} \Leftrightarrow \heva{
			&u_1+3d=-12\\
			&u_1+13d=18 \\
		}\Leftrightarrow \heva{
			&u_1=-21 \\
			& d=3. \\
		}$}
\end{ex}

\begin{ex}%[Dự án DCHT-11-KNTT]%[Dao-V- Thuy]%[1K2K5-1]
	Cho cấp số cộng $(u_n)$ thoả mãn $\heva{&u_2-u_3+u_5=10\\ &u_3+u_4=17}$. Số hạng đầu tiên và công sai của cấp số cộng đó lần lượt là
	\choice
	{\True $1$ và $3$}
	{$-3$ và $4$}
	{$4$ và $-3$}
	{$-4$ và $-3$}
	\loigiai{
		$\heva{&u_2-u_3+u_5=10\\ &u_3+u_4=17}\Leftrightarrow\heva{&(u_1+d)-(u_1+2d)+(u_1+4d)=10\\&(u_1+2d)+(u_1+3d)=17}\Leftrightarrow\heva{&u_1+3d=10\\&2u_1+5d=17}\Leftrightarrow\heva{&u_1=1\\&d=3.}$}
\end{ex}

\begin{ex}%[Dự án DCHT-11-KNTT]%[Dao-V- Thuy]%[1K2K5-1]
	Cho cấp số cộng $(u_n)$ có công sai $d<0$, $u_{31}+u_{34}=11$ và $(u_{31})^2 + (u_{34})^2=101$. Số hạng tổng quát của $(u_n)$ là
	\choice
	{$u_{n}=86-3n$}
	{$u_{n}=92-3n$}
	{$u_{n}=95-3n$}
	{\True $u_{n}=103-3n$}
	\loigiai{Gọi cấp số cộng $(u_n)$ có công sai $d$.\\
		$(u_{31})^2 + (u_{34})^2=101 \Leftrightarrow \left( {u_{31}+u_{34}}\right)^2-2u_{31}.u_{34}=101$ $\Rightarrow u_{31}.u_{34}=10$.\\
		Do đó, ta có $\heva{&u_{31}+u_{34}=11\\ &u_{31}.u_{34}=10}$ $\Rightarrow \heva{&u_{31}=10 \\ &u_{34}=1}$(vì $d<0$)\\
		$u_{31}+u_{34}=11 \Rightarrow 2u_{31}+3d =11 \Rightarrow d=-3 \,\,\text{và}\,\, u_{1}=100$.\\
		Do đó: $u_{n}=103-3n$.}
\end{ex}
\Closesolutionfile{ans}
% \begin{indapan}{10}
% 	{ans/ans-1K2-2-Dang2}
% \end{indapan}
\begin{dang}{Tổng của $n$ số hạng đầu tiên của một cấp số cộng. Tính chất của cấp số cộng}
	Tổng của $n$ số hạng đầu tiên:	Đặt ${{S}_{n}}={{u}_{1}}+{{u}_{2}}+{{u}_{3}}+\cdots+{{u}_{n}}.$ Khi đó
	\begin{itemize}
		\item [$\bullet$] ${{S}_{n}}=\dfrac{n\left( {{u}_{1}}+{{u}_{n}} \right)}{2}=\dfrac{n\left( {{u}_{2}}+{{u}_{n-1}} \right)}{2}=\dfrac{n\left( {{u}_{3}}+{{u}_{n-2}} \right)}{2}=\cdots$
		\item [$\bullet$] Vì ${{u}_{n}}={{u}_{1}}+\left( n-1 \right)d$ nên công thức trên có thể viết lại là \fbox{${{S}_{n}}=\dfrac{n}{2}\left[2u_1 + \left(n-1\right)d \right]  .$}
	\end{itemize}
	Tính chất của cấp số cộng:
	\begin{itemize}
		\item [\ding{172}] Nếu $a$; $b$; $c$ theo thứ tự lập thành cấp số cộng thì $a+c=2b$.
		\item [\ding{173}] Lưu ý:
		\begin{itemize}
			\item [$\bullet$] Nếu cho ba số liên tiếp của một cấp số cộng, ta có thể xem ba số đó là $$a-d;\quad a; \quad a+d$$
			\item [$\bullet$] Nếu cho bốn số liên tiếp của một cấp số cộng, ta có thể xem ba số đó là $$a-3d;\quad a-d; \quad a+d; \quad a+3d.$$
		\end{itemize}
	\end{itemize}
\end{dang}
\viduminhhoa
\begin{vd}
	Cho một cấp số cộng $(u_n)$ có $u_3 + u_{28} = 100$. Hãy tính tổng của $30$ số hạng đầu tiên của cấp số cộng đó.\dapso{$1500$}
	\loigiai{Ta có $S_{30} = \dfrac{30(u_1 + u_{30})}{2} = \dfrac{30(u_1 + 2d + u_{30} - 2d)}{2} = \dfrac{30(u_3 + u_{28})}{2} = \dfrac{30 \cdot 100}{2} = 1500$.}
\end{vd}\dongcham{7}

\begin{vd}
	Cho một cấp số cộng $(u_n)$ có $S_6 = 18$ và $S_{10} = 110$. Tính $S_{20}$.	\dapso{$ 620 $.}
	\loigiai{
		Giả sử cấp số cộng $(u_n)$ có số hạng đầu là $u_1$ và công sai là $d$.\\
		Ta có $S_6 = 6u_1 + \dfrac{6 \cdot 5}{2}d \Leftrightarrow 6u_1 + 15d = 18$. \quad (1)\\
		$S_{10} = 10u_1 + \dfrac{10 \cdot 9}{2}d \Leftrightarrow 10u_1 + 45d = 110$. \quad (2)\\
		Từ (1) và (2), ta có hệ phương trình $\heva{&6u_1 + 15d = 18\\&10u_1 + 45d = 110} \Leftrightarrow \heva{&u_1 = -7\\&d = 4.}$\\
		Khi đó $S_{20} = 20u_1 + \dfrac{20 \cdot 19}{2}d = 20 \cdot (-7) + 190 \cdot 4 = 620$.
	}
\end{vd}\dongcham{8}


\begin{vd}
	Tìm số hạng đầu và công sai của cấp số cộng, biết
	\begin{tasks}(2)
		\task $\heva{&u_1^2 + u_2^2 + u_3^2 = 155\\&S_3 = 21.}$	\dapso{$u_1 = 9$, $d = -2$ hoặc $u_1 = 5$, $d = 2$.}
		\task $\heva{&S_3 = 12\\&S_5 = 35.}$	\dapso{$u_1 = 1$, $d = 3$.}
	\end{tasks}
	\loigiai{
		\begin{listEX}
			\item $\heva{&u_1^2 + u_2^2 + u_3^2 = 155\\&S_3 = 21} \Leftrightarrow \heva{&u_1^2 + (u_1 + d)^2 + (u_1 + 2d)^2 = 155 &(1)\\&3u_1 + 3d = 21.&(2)}$\\
			Từ (2), ta có $3u_1 + 3d = 21 \Rightarrow d = 7 - u_1$, thay vào (1)
			$$u_1^2 + 7^2 + (14 - u_1)^2 = 155 \Leftrightarrow 2u_1^2 - 28u_1 + 90 = 0 \Leftrightarrow \hoac{&u_1 = 9\\&u_1 = 5.}$$
			Với $u_1 = 9$ thì $d = -2$. Với $u_1 = 5$ thì $d = 2$.\\
			Vậy số hạng đầu của cấp số cộng là $u_1 = 9$, công sai là $d = -2$ hoặc $u_1 = 5$, $d = 2$.
			\item $\heva{&S_3 = 12\\&S_5 = 35} \Leftrightarrow \heva{&3u_1 + 3d = 12\\&5u_1 + 10d = 35} \Leftrightarrow \heva{&u_1 = 1\\&d = 3.}$\\
			Vậy số hạng đầu của cấp số cộng là $u_1 = 1$, công sai là $d = 3$.
	\end{listEX}}
\end{vd}\dongcham{12}

\begin{vd}
	Tìm số hạng tổng quát của cấp số cộng, biết 
	$\heva{&S_4 = 20\\&\dfrac{1}{u_1} + \dfrac{1}{u_2} + \dfrac{1}{u_3} + \dfrac{1}{u_4} = \dfrac{25}{24}}$ và cấp số cộng có công sai là một số nguyên âm.	\dapso{$ u_n=10-2n $.}
	\loigiai{
		$\heva{&S_4 = 20 &(1)\\&\dfrac{1}{u_1} + \dfrac{1}{u_2} + \dfrac{1}{u_3} + \dfrac{1}{u_4} = \dfrac{25}{24}&(2).}$\\
		Từ (1), suy ra $u_1 + u_4 = u_2 + u_3 = 10$ và $u_1 = 5 - \dfrac{3}{2}d$.\\
		Từ (2), ta có 
		\begin{eqnarray*}
			& &\dfrac{u_1 + u_4}{u_1 \cdot u_4} + \dfrac{u_2 + u_3}{u_2 \cdot u_3} = \dfrac{25}{24} \Leftrightarrow \dfrac{10}{u_1(u_1 + 3d)} + \dfrac{10}{(u_1 + d)(u_1 + 2d)} = \dfrac{25}{24}\\
			&\Leftrightarrow & \dfrac{10}{\left(5 - \dfrac{3}{2}d\right)\left(5 + \dfrac{3}{2}d\right)} + \dfrac{10}{\left(5 - \dfrac{1}{2}d\right)\left(5 + \dfrac{1}{2}d\right)} = \dfrac{25}{24} \Leftrightarrow \dfrac{10}{25 - \dfrac{9}{4}d^2} + \dfrac{10}{25 - \dfrac{1}{4}d^2} = \dfrac{25}{24}\\
			&\Leftrightarrow & 10\left(25 - \dfrac{9}{4}d^2 + 25 - \dfrac{1}{4}d^2\right) = \dfrac{25}{24}\left(25 - \dfrac{9}{4}d^2\right)\left(25 - \dfrac{1}{4}d^2\right)\\
			&\Leftrightarrow & \dfrac{75}{128}d^4 - \dfrac{1925}{48}d^2 + \dfrac{3625}{24} = 0 \Leftrightarrow \hoac{&d^2 = \dfrac{580}{9}\\&d^2 = 4} \Leftrightarrow \hoac{&d = \pm \dfrac{2\sqrt{145}}{3}\\&d = \pm 2.}
		\end{eqnarray*}
		Với $d = -2$ thì $u_1 = 8$. Suy ra $u_n=u_1+(n-1)d=10-2n$}
\end{vd}\dongcham{18}

\begin{vd}
	Tính các tổng sau
	\begin{tasks}(2)
		\task $S = 1 + 3 + 5 + \cdots + (2n - 1) + (2n + 1)$.\dapso{$S = (n + 1)^2$}
		\task $S = 100^2 - 99^2 + 98^2 - 97^2 + \cdots + 2^2 - 1^2$.\dapso{$S = 5050$}
	\end{tasks}
	\loigiai{
		\begin{enumEX}{1}
			\item $S = 1 + 3 + 5 + \cdots + (2n - 1) + (2n + 1)$.\\
			Xét cấp số cộng $(u_k)$, $k \in \mathbb{N}^*$ với số hạng đầu là $u_1 = 1$ và công sai là $d = 2$.\\
			Ta có $u_k = u_1 + (k - 1)d \Leftrightarrow 2n + 1 = 1 + 2(k - 1) \Leftrightarrow k = n + 1$.\\
			Vậy $S = \dfrac{k(u_1 + u_k)}{2} = \dfrac{(n + 1)(1 + 2n + 1)}{2} = (n + 1)^2$.
			\item $S = 100^2 - 99^2 + 98^2 - 97^2 + \cdots + 2^2 - 1^2 = 199 + 195 + \cdots + 3$.\\
			Xét cấp số cộng $(u_n)$ có số hạng đầu $u_1 = 199$ và công sai $d = u_2 - u_1 = 195 - 199 = -4$.\\
			Ta có $u_n = u_1 + (n - 1)d \Leftrightarrow 3 = 199 - 4(n - 1) \Leftrightarrow n = 50$.\\
			Khi đó $S = \dfrac{n(u_1 + u_{50})}{2} = \dfrac{50(199 + 3)}{2} = 5050$.
			
		\end{enumEX}
	}
\end{vd}\dongcham{18}

\begin{vd}
	Tìm ba số hạng liên tiếp của một cấp số cộng biết tổng của chúng bằng $27$ và tổng các bình phương của chúng là $293$.\dapso{$4$, $9$, $14$}
	\loigiai{
		Gọi ba số hạng liên tiếp của cấp số cộng là $x - d$, $x$, $x + d$ trong đó $d$ là công sai của cấp số cộng.\\
		Khi đó ta có $x - d + x + x + d = 27 \Leftrightarrow 3x = 27 \Leftrightarrow x = 9$.\\
		Mà $(x - d)^2 + x^2 + (x + d)^2 = 293 \Leftrightarrow (9 - d)^2 + 81 + (9 + d)^2 = 293 \Leftrightarrow 2d^2 -50 = 0 \Leftrightarrow \hoac{&d = 5\\&d = -5.}$\\	
		Với $d = 5$ thì ba số hạng của cấp số cộng là $4$, $9$, $14$.\\
		Với $d = -5$ thì ba số hạng của cấp số cộng là $14$, $9$, $4$.\\
		Vậy ba số hạng liên tiếp của cấp số cộng là $4$, $9$, $14$.
	}
\end{vd}\dongcham{14}

\begin{vd}
	Tìm bốn số hạng liên tiếp của một cấp số cộng, biết tổng của chúng bằng $10$ và tổng bình phương của chúng bằng $30$.\dapso{$1$, $2$, $3$, $4$}
	\loigiai{
		Gọi bốn số hạng liên tiếp của cấp số cộng là $x - 3d$, $x - d$, $x 
		+ d$, $x + 3d$ với $2d$ là công sai của cấp số cộng.\\
		Khi đó ta có $x - 3d + x - d + x + d + x + 3d = 10 \Leftrightarrow 4x = 10 \Leftrightarrow x = \dfrac{5}{2}$.\\
		Mặt khác $$(x - 3d)^2 + (x - d)^2 + (x + d)^2 + (x + 3d)^2 = 30 \Leftrightarrow 4x^2 + 20d^2 = 30 \Leftrightarrow d^2 = \dfrac{1}{4} \Leftrightarrow \hoac{&d = \dfrac{1}{2}\\&d = -\dfrac{1}{2}.}$$
		Với $x = \dfrac{5}{2}$ thì $d = \dfrac{1}{2}$, khi đó bốn số hạng liên tiếp của cấp số cộng là $1$, $2$, $3$, $4$.\\
		Với $x = \dfrac{5}{2}$ thì $d = -\dfrac{1}{2}$, khi đó bốn số hạng liên tiếp của cấp số cộng là $4$, $3$, $2$, $1$.\\
		Vậy bốn số hạng liên tiếp của cấp số cộng là $1$, $2$, $3$, $4$.
	}
\end{vd}\dongcham{14}

\begin{vd}
	Ba góc của một tam giác vuông lập thành một cấp số cộng. Tìm ba góc đó.
	\loigiai{Gọi ba góc của tam giác lần lượt là $A$, $B$, $C$.
		Khi đó ta có $A + B + C = 180^\circ$.\\
		Do ba góc $A$, $B$, $C$ của tam giác theo thứ tự lập thành một cấp số cộng nên $B-A=C-A \Leftrightarrow A + C = 2B$.\\
		Do đó $2B + B = 180^\circ \Rightarrow 3B = 180^\circ \Rightarrow B = 60^\circ$.\\
		Do tam giác $ABC$ vuông nên giả sử $C = 90^\circ$ khi đó công sai $d$ của cấp số cộng là $d = C - B = 30^\circ$.\\
		Vậy góc $A$ của tam giác là $A = 30^\circ$.}
\end{vd}\dongcham{10}

% \begin{vd}
% 	Cho $a$, $b$, $c$ là ba số hạng liên tiếp của một cấp số cộng. Chứng minh rằng
% 	\begin{tasks}(1)
% 		\task $a^2 + 2bc = c^2 + 2ab$.
% 		\task $2(a+b+c)^3 = 9 \left[ a^2(b+c) + b^2(a+c) + c^2(a+b) \right]$.
% 		\task  $b^2 + bc +c^2$, $a^2 + ac + c^2$, $a^2 + ab + b^2$ cũng là một cấp số cộng.
% 	\end{tasks}
% 	\loigiai{
% 		\begin{enumerate}[a)]
% 			\item Vì $a$, $b$, $c$ là ba số liên tiếp của một cấp số cộng nên $a + c = 2b \Rightarrow a = 2b -c$.\\
% 			Do đó
% 			$$a^2 +2bc = (2b-c)^2 + 2bc = 4b^2 - 2bc + c^2 = 2b(2b -c) + c^2 = 2ba + c^2 = c^2 + 2ab.$$
% 			Vậy $a^2 + 2bc = c^2 + 2ab$ (đpcm).
			
% 			\item Vì $a$, $b$, $c$ là ba số liên tiếp của một cấp số cộng nên $a + c = 2b \Rightarrow a = 2b -c$.\\
% 			Do đó
% 			\allowdisplaybreaks
% 			\begin{eqnarray*}
% 				& \mbox{VT}  & = 2(a+b+c)^3 = 2(3b)^3 = 54b^3\\
% 				& \mbox{VP}  & = 9\left[ a^2(b+c) + b^2(a+c) + c^2(a+b) \right] \\
% 				& & = 9\left[ (2b-c)^2(b+c) + b^2(2b-c+c) + c^2(2b-c+b) \right] \\ 
% 				& & = 9\left[ (4b^2 - 4bc+c^2)(b+c) + b^2(2b) + c^2(3b-c) \right] \\
% 				& & = 9\left[ 4b^3 - 4b^2c +bc^2 + 4b^2c - 4bc^2 + c^3 + 2b^3 + 3bc^2 - c^3 \right] \\
% 				& & = 9\cdot (6b^3) = 54b^3 = \mbox{ VT }.
% 			\end{eqnarray*}
% 			Vậy $2(a+b+c)^3 = 9 \left[ a^2(b+c) + b^2(a+c) + c^2(a+b) \right]$ (đpcm).
			
% 			\item Vì ba số $a$, $b$, $c$ theo thứ tự lập thành một cấp số cộng thì $a + c = 2b \Rightarrow a = 2b - c$.\\
% 			Xét
% 			\allowdisplaybreaks
% 			\begin{eqnarray*}
% 				& 2(a^2 + ac + c^2) - (a^2 + ab + b^2) & = a^2 + a(2c-b) + 2c^2 - b^2 \\
% 				& & = (2b-c)^2 + (2b-c)(2c-b) + 2c^2 - b^2 \\
% 				& & = b^2 + bc  +c^2\\
% 				&\Rightarrow (b^2 + bc  +c^2) + (a^2 + ab + b^2) &= 2(a^2 + ac + c^2).
% 			\end{eqnarray*}
% 			Vậy ba số: $b^2 + bc +c^2$, $a^2 + ac + c^2$, $a^2 + ab + b^2$ cũng là một cấp số cộng.
% 		\end{enumerate}
% 	}
% \end{vd}\dongcham{25}

\begin{vd}%[TH]%[Dự án DCHT-11-KNTT]%[Dao-V- Thuy]%[1K2B5-2]
	Xác định $4$ góc của một tứ giác lồi, biết rằng $4$ góc hợp thành cấp số cộng và góc lớn nhất bằng $5$ lần góc nhỏ nhất.
	\dapso{$36^\circ; \, 72^\circ; \, 108^\circ; \, 144^\circ$}
	\loigiai
	{
		Gọi số đo bốn góc cần tìm là $u_1$, $u_2$, $u_3$, $u_4$. Ta có
		\begin{eqnarray*}
			\heva{&u_1+u_2+u_3+u_4=360\\ &u_5=5u_1} \Leftrightarrow \heva{&4u_1+6d=360\\ &4d=4u_1} \Leftrightarrow \heva{&u_1=36\\ &d=36.}
		\end{eqnarray*}
		Vậy số đo bốn góc cần tìm là
		\[
		36^\circ; \, 72^\circ; \, 108^\circ; \, 144^\circ.
		\]
	}
\end{vd}

\subsubsection{Bài tập tự luận}
 

\begin{bt}%[TH]%[Dự án DCHT-11-KNTT]%[Dao-V- Thuy]%[1K2B5-2]
	Giữa các số $10$ và $64$ hãy đặt thêm $17$ số nữa để được một cấp số cộng.
	\dapso{$13; 16; 19; 22; 25; 28; 31; 34; 37; 40; 43; 46; 49; 52; 55; 58; 61$}
	\loigiai{
		Ta có
		\begin{equation*}
			\heva{&u_1=10\\ &u_{19}=64} \Leftrightarrow \heva{&u_1=10\\ &u_1+18d=64} \Leftrightarrow \heva{&u_1=10\\ &d=3.}
		\end{equation*}
		Vậy $17$ số đặt thêm giữa các số $10$ và $64$ để được một cấp số cộng là
		\begin{center}
			13; 16; 19; 22; 25; 28; 31; 34; 37; 40; 43; 46; 49; 52; 55; 58; 61.
		\end{center} 
	}
\end{bt}

\begin{bt}%[TH]%[Dự án DCHT-11-KNTT]%[Dao-V- Thuy]%[1K2B5-2]
	Tổng ba số hạng liên tiếp của một cấp số cộng bằng $2$ và tổng các bình phương của ba số đó bằng $\dfrac{14}{9}$. Xác định ba số đó và tính công sai của cấp số cộng.
	\dapso{$1;\dfrac{2}{3};\dfrac{1}{3}$ ứng với $d=-\dfrac{1}{3}$ hoặc $\dfrac{1}{3};\dfrac{2}{3};1$ ứng với $d=\dfrac{1}{3}$}
	\loigiai{
		Ta có hệ
		\begin{align*}
			&\quad \heva{&u_k+u_{k+1}+u_{k+2}=2\\ &u^2_k+u^2_{k+1}+u^2_{k+2}=\dfrac{14}{9}}
			\Leftrightarrow \heva{&u_k+u_k+d+u_k+2d=2\\ &u^2_k+\left(u_k+d\right)^2 +\left(u_k+2d\right)^2=\dfrac{14}{9}} \\
			&\Leftrightarrow \heva{&3u_k+3d=2\\ &3u^2_k+6u_kd+5d^2=\dfrac{14}{9}}
			\Leftrightarrow \heva{&u_k=1\\ &d=-\dfrac{1}{3}} \text{ hoặc } \heva{&u_k=\dfrac{1}{3}\\ &d=\dfrac{1}{3}.}
		\end{align*}
		Vậy ba số hạng liên tiếp của cấp số cộng thỏa yêu cầu bài toán $1;\dfrac{2}{3};\dfrac{1}{3}$ ứng với $d=-\dfrac{1}{3}$ hoặc $\dfrac{1}{3};\dfrac{2}{3};1$ ứng với $d=\dfrac{1}{3}.$
	}
\end{bt}

\begin{bt}%[TH]%[Dự án DCHT-11-KNTT]%[Dao-V- Thuy]%[1K2B5-2]
	Một cấp số cộng có $7$ số hạng với công sai $d$ dương và số hạng thứ tư bằng $11$. Hãy tìm các số hạng còn lại của cấp số cộng đó, biết hiệu của số hạng thứ ba và số hạng thứ năm bằng $6$.
	\dapso{$u_1=2$; $ u_2=5$; $u_4=11$; $u_6=17$; $u_7=20$}
	\loigiai{
		Gọi số hạng đầu của cấp số cộng là $u_1$, công sai $d$.
		Vì số hạng thứ tư của cấp số cộng bằng $11$ nên ta có $u_4=11$.\\
		Do $d$ dương nên $ u_5>u_3$.\\
		Vì hiệu của số hạng thứ ba và số hạng thứ năm bằng $6$ nên ta có $ u_5-u_3=6$.\\
		Ta có \begin{align*}
			\heva{&u_4=11\\&u_5-u_3=6 }
			\Leftrightarrow \heva{&u_1+3d=11\\&(u_1+4d)-(u_1+2d)=6 }
			\Leftrightarrow \heva{&u_1+3\cdot 3=11\\&d=3 }
			\Leftrightarrow \heva{&u_1=2\\&d=3.}
		\end{align*}
		Vậy các số  hạng còn lại của cấp số cộng là $u_1=2$; $ u_2=5$; $u_4=11$; $u_6=17$; $u_7=20$.
	}
\end{bt}

\begin{bt}%[VD]%[Dự án DCHT-11-KNTT]%[Dao-V- Thuy]%[1K2K5-2]
	Tìm bốn số hạng liên tiếp của một cấp số cộng, biết rằng:
	\begin{enumerate}
		\item Tổng của chúng bằng $10$ và tổng bình phương bằng $70$.
		\item Tổng của chúng bằng $22$ và tổng bình phương bằng $66$.
		\item  Tổng của chúng bằng $36$ và tổng bình phương bằng $504$.
		\item  Chúng có tổng bằng $20$ và tích của chúng bằng $384$.
		\item  Tổng của chúng bằng $ 20$, tổng nghịch đảo của chúng bằng $ \dfrac{25}{24}$ và các số này là những số nguyên.
		\item  Nó là số đo của một tứ giác lồi và góc lớn nhất gấp $5$ lần góc nhỏ nhất.
	\end{enumerate}
	\dapso{$-2$; $ 1$; $ 4$; $7$.} 
	\dapso{không tồn tại bốn số hạng liên tiếp của cấp số cộng thỏa mãn yêu cầu đề bài. $0$; $ 6$; $ 12$; $18$.}
	\dapso{$2$; $ 4$; $ 6$; $8$ hoặc $5-\sqrt{241}$; $ \dfrac{15-\sqrt{241}}{3}$; $ \dfrac{15+\sqrt{241}}{3}$; $5+\sqrt{241}$.} 
	\dapso{$30^\circ$; $70^\circ$; $ 110^\circ$; $150^\circ$.}
	\loigiai{
		\begin{enumerate}
			\item Gọi bốn số hạng liên tiếp của cấp số cộng là $x-3d$; $x-d$; $x+d$, $x+3d$ trong đó $2d$ là công sai.\\
			Theo đề bài ta có 
			\begin{align*}
				&\quad \heva{& (x-3d)+(x-d)+(x+d)+(x+3d)=10\\& (x-3d)^2+(x-d)^2+(x+d)^2+(x+3d)^2=70}
				\Leftrightarrow \heva{& 4x=10 \\& 4x^2+20d^2=70}\\
				&\Leftrightarrow  \heva{&x=\dfrac{5}{2}\\& 4\cdot \left( \dfrac{5}{2}\right) ^2+20d^2=70}
				\Leftrightarrow  \heva{&x=\dfrac{5}{2}\\& d^2=\dfrac{9}{4}}
				\Leftrightarrow  \heva{&x=\dfrac{5}{2}\\& d=\pm \dfrac{3}{2}.}
			\end{align*}
			Vậy bốn số hạng liên tiếp của cấp số cộng là $-2$; $ 1$; $ 4$; $7$.
			\item Gọi bốn số hạng liên tiếp của cấp số cộng là $x-3d$; $x-d$; $x+d$; $x+3d$ trong đó $2d$ là công sai.\\
			Theo đề bài ta có 
			\begin{align*}
				&\quad \heva{& (x-3d)+(x-d)+(x+d)+(x+3d)=22\\& (x-3d)^2+(x-d)^2+(x+d)^2+(x+3d)^2=66}
				\Leftrightarrow \heva{& 4x=22 \\& 4x^2+20d^2=66}\\
				&\Leftrightarrow  \heva{&x=\dfrac{11}{2}\\& 4\cdot \left( \dfrac{11}{2}\right) ^2+20d^2=66}
				\Leftrightarrow  \heva{&x=\dfrac{11}{2}\\& d^2=\dfrac{-11}{4} 
					\ (\text{loại}).}\\
			\end{align*}
			Vậy không tồn tại bốn số hạng liên tiếp của cấp số cộng thỏa mãn yêu cầu đề bài.
			\item Gọi bốn số hạng liên tiếp của cấp số cộng là $x-3d$; $x-d$; $x+d$; $x+3d$ trong đó $2d$ là công sai.\\
			Theo đề bài ta có 
			\begin{align*}
				&\quad \heva{& (x-3d)+(x-d)+(x+d)+(x+3d)=36\\& (x-3d)^2+(x-d)^2+(x+d)^2+(x+3d)^2=504}
				\Leftrightarrow \heva{& 4x=36 \\& 4x^2+20d^2=504}\\
				&\Leftrightarrow  \heva{&x=9\\& 4\cdot 9^2+20d^2=504}
				\Leftrightarrow  \heva{&x=9\\& d^2=9}
				\Leftrightarrow  \heva{&x=9\\& d=\pm 3.}
			\end{align*}
			Vậy  bốn  số hạng liên tiếp của cấp số cộng là $0$; $ 6$; $ 12$; $18$.
			\item Gọi bốn số hạng liên tiếp của cấp số cộng là $x-3d$; $x-d$; $x+d$; $x+3d$ trong đó $2d$ là công sai.\\
			Theo đề bài ta có 
			\begin{align*}
				&\quad \heva{& (x-3d)+(x-d)+(x+d)+(x+3d)=20\\& (x-3d)(x-d)(x+d)(x+3d)=384}
				\Leftrightarrow  \heva{&x=5\\& (x^2-d^2)(x^2-9d^2)=384}\\
				&\Leftrightarrow  \heva{&x=5\\& (25-d^2)(25-9d^2)=384}
				\Leftrightarrow  \heva{&x=5\\& 9d^4-250d^2+241=0}
				\Leftrightarrow  \heva{&x=5\\& \hoac{&d^2=1\\& d^2=\dfrac{241}{9}}}
				\Leftrightarrow  \heva{&x=5\\& \hoac{&d=\pm 1\\& d=\pm \dfrac{\sqrt{241}}{3}.}}
			\end{align*}
			Vậy bốn số hạng liên tiếp của cấp số cộng là $2$; $ 4$; $ 6$; $8$ hoặc $5-\sqrt{241}$; $ \dfrac{15-\sqrt{241}}{3}$; $ \dfrac{15+\sqrt{241}}{3}$; $5+\sqrt{241}$.
			\item Gọi bốn số hạng liên tiếp của cấp số cộng là $x-3d$; $x-d$; $x+d$; $x+3d$ trong đó $2d$ là công sai trong đó $ 2d \in \mathbb{Z}$.\\
			Theo đề bài ta có 
			\begin{align*}
				&\quad \heva{& (x-3d)+(x-d)+(x+d)+(x+3d)=20\\& \dfrac{1}{x-3d}+\dfrac{1}{x-d}+\dfrac{1}{x+d}+\dfrac{1}{x+3d}=\dfrac{25}{24}}
				\Leftrightarrow \heva{& 4x=20 \\& \dfrac{1}{5-3d}+\dfrac{1}{5-d}+\dfrac{1}{5+d}+\dfrac{1}{5+3d}=\dfrac{25}{24}}\\
				&\Leftrightarrow  \heva{&x=5\\& \dfrac{10}{25-9d^2}+\dfrac{10}{25-d^2}=\dfrac{25}{24}}
				\Leftrightarrow  \heva{&x=5\\& 9d^4-250d^2+241=0}\\
				&\Leftrightarrow  \heva{&x=5\\& \hoac{&d^2=1 \\& d^2=\dfrac{241}{9} }}
				\Leftrightarrow  \heva{&x=5\\& \hoac{&d= \pm 1 \ (\text{thỏa mãn})\\& d= \pm \dfrac{\sqrt{241}}{3} \,\,(\text{loại vì} \,  2d \in \mathbb{Z}).}}
			\end{align*}
			Vậy bốn  số hạng  nguyên liên tiếp của cấp số cộng là $2$; $ 4$; $ 6$; $8$.
			\item Gọi bốn số hạng liên tiếp của cấp số cộng xếp theo thứ tự tăng dần  là $x-3d$; $x-d$; $x+d$; $x+3d$ trong đó $2d>0$ là công sai.\\
			Theo đề bài ta có 
			\begin{align*}
				&\quad \heva{& (x-3d)+(x-d)+(x+d)+(x+3d)=360^\circ\\& x+3d=5(x-3d)}\\
				&\Leftrightarrow \heva{& 4x=360^\circ\\& 4x=18d}
				\Leftrightarrow  \heva{&x=90^\circ\\& 4 \cdot 90^\circ=18d}
				\Leftrightarrow  \heva{&x=90^\circ\\& d=20^\circ.}
			\end{align*}
			Vậy bốn  góc của tứ giác lồi lần lượt là  $30^\circ$; $70^\circ$; $ 110^\circ$; $150^\circ$.
		\end{enumerate}
	}
\end{bt}
% \subsubsection{Câu hỏi trắc nghiệm}
% \Opensolutionfile{ans}[ans/ans-1K2-2-Dang3]
% \begin{ex}%[Dự án DCHT-11-KNTT]%[Dao-V- Thuy]%[1K2Y5-2]
% 	Cấp số cộng $(u_n)$ có số hạng đầu $u_1=-5$ và công sai $d=3$. Tính $u_{15}$.
% 	\choice
% 	{$u_{15}=27$}
% 	{\True $u_{15}=37$}
% 	{$u_{15}=47$}
% 	{$u_{15}=57$}
% 	\loigiai{$u_{15}=u_1+14d=-5+14\times 3=37$.}
% \end{ex}

% \begin{ex}%[Dự án DCHT-11-KNTT]%[Dao-V- Thuy]%[1K2Y5-2]
% 	Cho cấp số cộng có các số hạng ban đầu là $1$; $5$; $9$; $13$; $\cdots$. Số hạng thứ $ 6 $ của cấp số cộng này là bao nhiêu?
% 	\choice{\True $ 21$}
% 	{$19 $}
% 	{$ 22$}
% 	{$ 20$}
% 	\loigiai{Ta có $u_1=1$, $d=5-1=4$ nên $u_6=1+5d=1+20=21$.
% 	}
% \end{ex}

% \begin{ex}%[Dự án DCHT-11-KNTT]%[Dao-V- Thuy]%[1K2Y5-2]
% 	Cho cấp số cộng $\left( u_n \right)$ có các số hạng lần lượt là $-4;\,1;\,6;\,x$. Tìm giá trị của $x$.
% 	\choice
% 	{$x=7$}
% 	{$x=10$}
% 	{\True $x=11$}
% 	{$x=12$}
% 	\loigiai{
% 		Dễ thấy $u_1=-4$, $d=5$ nên $u_4=-4+3\cdot 5=11$.
% 	}
% \end{ex}

% \begin{ex}%[Dự án DCHT-11-KNTT]%[Dao-V- Thuy]%[1K2B5-2]
% 	Cho cấp số cộng $(u_n)$ có $u_1=-5$ và $d=3$. Mệnh đề nào sau đây đúng?
% 	\choice
% 	{$u_{15}=34$}
% 	{$u_{15}=45$}
% 	{\True $u_{13}=31$}
% 	{$u_{10}=35$}
% 	\loigiai {
% 		$\heva{
% 			& u_1=-5 \\
% 			& d=3 \\
% 		}\Rightarrow u_n=3n-8\Rightarrow \heva{
% 			& u_{15}=37 \\
% 			& u_{13}=31 \\
% 			& u_{10}=22. \\
% 		}$}
% \end{ex}

% \begin{ex}%[Dự án DCHT-11-KNTT]%[Dao-V- Thuy]%[1K2B5-2]
% 	Cho cấp số cộng có số hạng đầu là $u_1=-\dfrac{1}{2}$, công sai $d=\dfrac{1}{2}$. Trong mỗi bộ gồm năm số hạng dưới đây, bộ năm số nào là các số hạng liên tiếp của dãy này?
% 	\choice
% 	{$-\dfrac{1}{2};\,0;\,1;\,\dfrac{1}{2};\,1$}
% 	{$-\dfrac{1}{2};\,0;\,\dfrac{1}{2};\,0;\,\dfrac{1}{2}$}
% 	{$\dfrac{1}{2};\,1;\,2;\,\dfrac{5}{2};\,\dfrac{7}{2}$}
% 	{\True $1;\,\dfrac{3}{2};\,2;\,\dfrac{5}{2};\,3$}
% 	\loigiai{
% 		Ta có $u_1=-\dfrac{1}{2}$; $u_2=0$; $u_3=\dfrac{1}{2}$, $u_4=1$; $u_5=\dfrac{3}{2}$; $u_6=2$; $u_7=\dfrac{5}{2}$; $u_8=3$.
% 	}
% \end{ex}

% \begin{ex}%[Dự án DCHT-11-KNTT]%[Dao-V- Thuy]%[1K2B5-2]
% 	Cho cấp số cộng $(u_n)$ có $u_7=\dfrac{19}{5}$ và công sai $d=\dfrac{2}{5}$. Tính $u_{10}$.
% 	\choice
% 	{$\dfrac{2}{5}$}
% 	{$\dfrac{19}{5}$}
% 	{\True $5$}
% 	{$\dfrac{27}{5}$}
% 	\loigiai{Ta có: $u_7=u_1+6d\Rightarrow u_1=u_7-6d=\dfrac{19}{5}-6\cdot \dfrac{2}{5}=\dfrac{7}{5}$.\\
% 		Suy ra $u_{10}=u_1+9d=\dfrac{7}{5}+9\cdot \dfrac{2}{5}=5$.}
% \end{ex}

% \begin{ex}%[Dự án DCHT-11-KNTT]%[Dao-V- Thuy]%[1K2B5-2]
% 	Cho cấp số cộng $(u_n)$ có số hạng đầu $u_1=-1$ và công sai $d=-3$. Số hạng thứ $20$ của cấp số cộng này là
% 	\choice
% 	{\True $u_{20}=-58$}
% 	{$u_{20}=60$}
% 	{$u_{20}=-72$}
% 	{$u_{20}=-61$}
% 	\loigiai{Số hạng thứ $20$ là: $u_{20}=u_1+19d=-1+19\cdot (-3)=-58$.}
% \end{ex}

% \begin{ex}%[Dự án DCHT-11-KNTT]%[Dao-V- Thuy]%[1K2B5-2]
% 	Cho cấp số cộng $(u_n)$ có $u_1=-5$ và $d=3$. Số $100$ là số hạng thứ mấy của cấp số cộng?
% 	\choice
% 	{Thứ $15$}
% 	{Thứ $20$}
% 	{Thứ $35$}
% 	{\True Thứ $36$}
% 	\loigiai {
% 		Ta có $\heva{
% 			&u_1=-5 \\
% 			& d=3 \\
% 		}$. Vì $u_n=100 \Rightarrow 100=u_n=u_1+(n-1 )d=3n-8\Leftrightarrow n=36$.
% 	}
% \end{ex}

% \begin{ex}%[Dự án DCHT-11-KNTT]%[Dao-V- Thuy]%[1K2B5-2]
% 	Cho cấp số cộng $(u_n)$ có $u_2=2001$ và $u_5=1995$. Khi đó $u_{1001}$ bằng
% 	\choice
% 	{$u_{1001}=4005$}
% 	{$u_{1001}=4003$}
% 	{\True $u_{1001}=3$}
% 	{$u_{1001}=1$}
% 	\loigiai {
% 		$\heva{
% 			& 2001=u_2=u_1+d \\
% 			& 1995=u_5=u_1+4d \\
% 		}\Leftrightarrow \heva{
% 			&u_1=2003 \\
% 			& d=-2 \\
% 		}\Rightarrow u_{1001}=u_1+1000d=3$.}
% \end{ex}

% \begin{ex}%[Dự án DCHT-11-KNTT]%[Dao-V- Thuy]%[1K2B5-2]
% 	Cho cấp số cộng $(u_n)$ biết $\heva{&u_1+u_3=7\\&u_2+u_4=12}$. Tính $u_{21}$.
% 	\choice
% 	{$u_{21}=1$}
% 	{\True $u_{21}=51$}
% 	{$u_{21}=31$}
% 	{$u_{21}=21$}
% 	\loigiai{
% 		Ta có $\heva{&u_1+u_3=7\\&u_2+u_4=12}\Leftrightarrow\heva{&u_1+u_1+2d=7\\&u_1+d+u_1+3d=12}\Leftrightarrow\heva{&2u_1+2d=7\\&2u_1+4d=12}\Leftrightarrow\heva{&u_1=1\\&d=\dfrac{5}{2}.}$\\
% 		Suy ra $u_{21}=u_1+20d=1+20\cdot\dfrac{5}{2}=1+50=51$.}
% \end{ex}

% \begin{ex}%[Dự án DCHT-11-KNTT]%[Dao-V- Thuy]%[1K2B5-2]
% 	Một cấp số cộng có $7$ số hạng. Biết rằng tổng của số hạng đầu và số hạng cuối bằng $30$, tổng của số hạng thứ ba và số hạng thứ sáu bằng $35$. Tìm số hạng thứ bảy của cấp số cộng đã cho.
% 	\choice
% 	{$u_7=25$}
% 	{\True $u_7=30$}
% 	{$u_7=35$}
% 	{$u_7=40$}
% 	\loigiai{
% 		Theo đề ta có: $\heva{&u_1+u_7=30\\&u_3+u_6=35}\Leftrightarrow \heva{&u_1+(u_1+6d)=30\\&(u_1+2d)+(u_1+5d)=35}\Leftrightarrow \heva{&2u_1+6d=30\\&2u_1+7d=35}\Leftrightarrow \heva{&u_1=0\\&d=5.}$\\
% 		Do đó $u_7=u_1+6d=0+6\cdot 5=30$.}
% \end{ex}

% \begin{ex}%[Dự án DCHT-11-KNTT]%[Dao-V- Thuy]%[1K2G5-2]
% 	Cho dãy số $(u_n)$ có xác định bởi $\heva{&u_1=-2,\\&u_{n+1}=\dfrac{u_n}{1-u_n}} \; (\text{với } n\in\mathbb{N}^*)$ và dãy số $(v_n)$ được xác định bởi $v_n=\dfrac{u_n+1}{u_n}$. Số hạng thứ $2023$ của dãy $(v_n)$ là
% 	\choice
% 	{$ -\dfrac{2023}{3}$}
% 	{$ -\dfrac{4046}{3}$}
% 	{\True$ -\dfrac{4043}{2}$}
% 	{$ -2023$}
% 	\loigiai{
% 		Ta có $v_{n+1}-v_n=\dfrac{u_{n+1}+1}{u_{n+1}}-\dfrac{u_n+1}{u_n}=\dfrac{\dfrac{u_n}{1-u_n}+1}{\dfrac{u_n}{1-u_n}}-\dfrac{u_n+1}{u_n}=\dfrac{1}{u_n}-\dfrac{u_n+1}{u_n}=-1$. Vậy $(v_n)$	là một CSC có công sai $d=-1$.
% 		\\Mặt khác, ta có $v_1=\dfrac{u_1+1}{u_1}=\dfrac{1}{2}$, do đó số hạng tổng quát $v_n=\dfrac{1}{2}+(n-1)(-1)=-n+\dfrac{3}{2}$. \\
% 		Do đó $v_{2023}=-2023+\dfrac{3}{2}=-\dfrac{4043}{2}$.
% 	}
% \end{ex}
% \Closesolutionfile{ans}
% \begin{indapan}{10}
% 	{ans/ans-1K2-2-Dang3}
% \end{indapan}

\begin{dang}{Các bài toán thực tế}
	Các bài toán thực tế về cấp số cộng có thể được giải bằng cách sử dụng công thức của cấp số cộng. Công thức của cấp số cộng là: $ u_n = u_1 + (n-1)d $. Trong đó:
	\begin{itemize}
		\item $ u_n $ là số hạng thứ $ n $ của cấp số cộng.
		\item $ u_1 $ là số hạng đầu tiên của cấp số cộng.
		\item $ d $ là công sai của cấp số cộng.
		\item Một số công thức thường gặp:
		\begin{enumEX}[\faCheckCircleO]{1}
			\item $u_n=\dfrac{u_{n-1}+u_{n+1}}{2}=u_1+(n-1)d$.
			\item $S_n=\dfrac{(u_1+u_n)\cdot n}{2}=\dfrac{2u_1+(n-1)d}{2}\cdot n$.
		\end{enumEX}			
	\end{itemize}
\end{dang}
\subsubsection{Ví dụ minh hoạ}
\begin{vd}%[NB]%[DCHT Toán 11 - KNTT - Nguyễn Hữu Đức] %[1K2B6-6]
	Một người có một khoản tiền gửi ngân hàng với lãi suất 10\% /năm theo hình thức lãi đơn. Nếu sau $ 5 $ năm người đó nhận được tổng số tiền là $ 550 $ triệu đồng thì số tiền gửi ban đầu của người đó là bao nhiêu?
	\dapso{$366{,}67$ triệu đồng.}
	\loigiai{
		Gọi $x$ là số tiền gửi ban đầu của người đó $ (x>0) $.\\
		Sau 5 năm, số tiền nhận được bằng số tiền gốc cộng với lãi suất:
		$$
		x + 0{,}1x \times 5 = 1{,}5x.
		$$
		Theo đề bài, tổng số tiền nhận được sau 5 năm là $550$ triệu đồng, do đó ta có phương trình:
		$$
		1{,}5x = 550.
		$$
		Giải phương trình ta có:
		$$
		x = \frac{550}{1{,}5} \approx 366{,}67.
		$$
		Vậy số tiền gửi ban đầu của người đó là $366{,}67$ triệu đồng.
	}
\end{vd}
\begin{vd}%[DCHT Toán 11 - KNTT - Nguyễn Hữu Đức] %[1K2B6-6]
	Bạn An muốn mua một món quà tặng mẹ nhân ngày mùng $8/3$. Bạn quyết định tiết kiệm từ ngày $1/2/2017$ đến hết ngày $6/3/2017$. Ngày đầu An có $5\,000$ đồng, kể từ ngày thứ hai số tiền An tiết kiệm được ngày sau cao hơn ngày trước mỗi ngày $1\,000$ đồng. Tính số tiền An tiết kiệm được để mua quà tặng mẹ.
	\dapso{$731\,000$ đồng.}
	\loigiai{
		Tính số ngày mà An tiết kiệm được từ ngày $1/2/2017$ đến hết ngày $6/3/2017$:\\
		Số ngày từ ngày $1/2/2017$ đến hết ngày $28/2/2017$ là $28$ ngày.\\
		Số ngày từ ngày $1/3/2017$ đến hết ngày $6/3/2017$ là $6$ ngày.\\
		Vậy An tiết kiệm được $28+6=34$ ngày.\\
		Gọi $u_n$ là số tiền An tiết kiệm được vào ngày thứ $n$ kể từ ngày $1/2/2017$.\\
		Theo đề ta có $u_1=5\,000$ đồng.\\
		Vì ngày sau An tiết kiệm được nhiều hơn ngày trước mỗi ngày $1\,000$ đồng nên $u_n=u_{n-1}+1\,000$, với $n\ge 2$.\\
		Vậy $(u_n)$ là một cấp số cộng với $u_1=5\,000$ và công sai $d=1\,000$.\\
		Tổng số tiền An tiết kiệm được trong $34$ ngày là:
		$$S_{34}=\dfrac{n}{2} \left(2u_1+33d\right)= \dfrac{34}{2} \left(2\cdot 5\,000+33\cdot 1\,000\right)=731\,000.$$
		Vậy số tiền An tiết kiệm được để mua quà tặng mẹ là $731\,000$ đồng.
	}
\end{vd}

\begin{vd}%[TH]%[DCHT Toán 11 - KNTT - Nguyễn Hữu Đức] %[1K2B6-6]
	[Cấp số nhân] Một hội đồng quản trị quyết định tăng lương cho nhân viên hàng năm theo tỷ lệ cố định. Ví dụ, lương của một nhân viên được tăng thêm $ 5 $\% so với năm trước. Hỏi nếu lương của một nhân viên là $ 10 $ triệu đồng/năm vào năm nay, thì lương của nhân viên đó sẽ là bao nhiêu vào năm thứ $ 5 $?
	\dapso{$12{,}1550625 $ triệu đồng/năm.}
	\loigiai{		
		Theo giả thiết, lương của nhân viên được tăng thêm $ 5 $ \% so với năm trước đó.
		\begin{itemize}
			\item Vậy lương của nhân viên vào năm thứ $ 2 $ sẽ là $ 10\cdot(1+0{,}05)=10{,}5 $ triệu đồng/năm.
			\item Tương tự, lương của nhân viên vào năm thứ $ 3 $ sẽ là $ 10{,}5 \cdot(1+0{,}05)=11{,}025 $ triệu đồng/năm.
			\item Lương của nhân viên vào năm thứ $ 4 $ sẽ là $ 11{,}025\cdot (1+0{,}05)=11{,}57625 $ triệu đồng/năm.
			\item Cuối cùng, lương của nhân viên vào năm thứ $ 5 $ sẽ là $ 11{,}57625\cdot(1+0{,}05)=12{,}1550625 $ triệu đồng/năm.
		\end{itemize}
		Vậy lương của nhân viên đó vào năm thứ $ 5 $ sẽ là $ 12{,}1550625 $ triệu đồng/năm.\\
		Chú ý: Lương của nhân viên đó vào năm thứ $ 5 $ sẽ là $ u_5=u_1+4d=10+4\cdot 10\cdot 0{,}05=12 $ triệu đồng chỉ đúng trong trường hợp lương của một nhân viên được tăng thêm $ 5 $\% so với năm đầu tiên.
	}
\end{vd}


\begin{vd}%[TH]%%[DCHT Toán 11 - KNTT - Nguyễn Hữu Đức] %[1K2B6-6]
	Hùng đang tiết kiệm để mua một cây guitar. Trong tuần đầu tiên, anh ta để dành  $ 42 $ đô la, và trong mỗi tuần tiết theo, anh ta đã thêm $ 8 $ đô la vào tài khoản tiết kiệm của mình. Cây guitar Hùng cần mua có giá $ 400 $ đô la. Hỏi vào tuần thứ bao nhiêu thì anh ấy có đủ tiền để mua cây guitar đó?
	\dapso{$ n=46 $.}
	\loigiai{
		Gọi $ n $ là số tuần anh ta đã thêm $ 8 $ đô la vào tài khoản tiết kiệm của mình.\\
		Số tiền anh ta tiết kiệm được sau $ n $ tuần đó là $ S=42+8n $. \\
		Theo bài ra $ S=42+8n\ge 400\Leftrightarrow n\ge 44,75\Rightarrow n=45 $.\\
		Vậy kể cả tuần đầu thì tuần thứ $ 46 $ anh ta có đủ tiền để mua cây guitar đó.
	}
\end{vd}

\begin{vd}%[DCHT Toán 11 - KNTT - Nguyễn Hữu Đức]%[1K2Y6-6]
	[Cấp số nhân] Hàng tháng ông An gửi vào ngân hàng một số tiền như nhau là $5\,000\,000$ đồng (vào ngày đầu mỗi tháng) với lãi suất $0{,}5\%$ một tháng, biết tiền lãi của tháng trước được nhập vào tiền gốc của tháng sau. Hỏi sau $36$ tháng ông An nhận được số tiền vốn và lãi là bao nhiêu? (làm tròn đến hàng đơn vị).
	\dapso{ $ 197\,663\,927 $ đồng.}
	\loigiai{
		Gọi $a$ là số tiền ông An gửi vào hàng tháng, $r$ là lãi suất trên một tháng và $P_n$ là số tiền vốn và lãi ông An nhận được sau $n$ tháng.
		\begin{itemize}
			\item Sau một tháng, ông An có số tiền là $P_1=a+ar=a(1+r)$.
			\item Đầu tháng thứ hai, ông An có số tiền là $P_1+a=a(1+r)+a$.
			\item Sau hai tháng, ông An có số tiền là $P_2=a(1+r)+a+\left[a(1+r)+a\right]r=a\left[(1+r)^2+(1+r)\right]$.
			\item Cuối tháng thứ $36$, ông An có số tiền là
			\begin{align*}
				P_{36}&=a\left[(1+r)^{36}+(1+r)^{35}+\ldots+(1+r)\right]\\
				&=a(1+r)\dfrac{(1+r)^{36}-1}{r}\\
				&=5000000\cdot (1+0{,}005)\cdot\dfrac{(1+0{,}005)^{36}-1}{0{,}005}\\
				&\approx 197\,663\,927 \quad \text{(đồng)}.
			\end{align*}
		\end{itemize}
	}
\end{vd}

\begin{vd}[VDT]%[DCHT Toán 11 - KNTT - Nguyễn Hữu Đức] %[1K2Y6-6]
	Một xưởng có đăng tuyển công nhân với đãi ngộ về lương như sau: Trong quý đầu tiên thì xưởng trả là $ 6 $ triệu đồng/quý và kể từ quý thứ $ 2 $ sẽ tăng lên $ 0{,}5 $ triệu cho $ 1 $ quý. Hỏi với đãi ngộ trên thì sau $ 5 $ năm làm việc tại xưởng, tổng số lương của công nhân đó là bao nhiêu?
	\dapso{ $ 215 $ triệu đồng.}
	\loigiai{
		Gọi $u_n$ (triệu đồng) là số lương của công nhân trong quý thứ $n$.\\
		Theo đề:\\
		Quý đầu: $ u_1 = 6 $ triệu.\\
		Các quý tiếp theo: $ u_{n+1} = u_{n} + 0,5 $ với $\forall n \ge 1$.\\
		Mức lương của công nhân mỗi quý là $ 1 $ số hạng của dãy số $ u_n $. Mặt khác, lương của quý sau hơn lương quý trước là $ 0,5 $ triệu nên dãy số $ u_n $ là một cấp số cộng với công sai $ d = 0{,}5 $.\\
		Ta biết $ 1 $ năm sẽ có $ 4 $ quý nên $ 5 $ năm sẽ có $ 5\cdot 4 = 20 $ quý. Theo yêu cầu của đề bài ta cần tính tổng của $ 20 $ số hạng đầu tiên của cấp số cộng ($ u_n $).\\
		Lương tháng quý $ 20 $ của công nhân: $ u_{20} = 6 + (20 - 1)\cdot 0{,}5 = 15{,}5 $ triệu đồng.\\
		Tổng số lương của công nhân nhận được sau $ 5 $ năm làm việc tại xưởng: $ S_{12}=20\cdot (6+15,5)2=215 $ (triệu đồng).
	}
\end{vd}

% \subsubsection{Bài tập tự luận}
 
%[DCHT Toán 11 - KNTT - Nguyễn Hữu Đức] %[1K2B6-6]
% \begin{bt}%[NB]%[DCHT Toán 11 - KNTT - Nguyễn Hữu Đức] %[1K2B6-6]
% 	Sinh nhật bạn của An vào ngày $ 01 $ tháng năm. An muốn mua một món quà sinh nhật cho bạn nên quyết định bỏ ống heo $ 100 $ đồng vào ngày $ 01 $ tháng $ 01 $ năm $ 2016 $, sau đó cứ liên tục ngày sau hơn ngày trước $ 100 $ đồng. Hỏi đến ngày sinh nhật của bạn, An đã tích lũy được bao nhiêu tiền? (thời gian bỏ ống heo tính từ ngày $ 01 $ tháng $ 01 $ năm $ 2016 $ đến ngày $ 30 $ tháng $ 04 $  năm $ 2016 $).\dapso{$ 738\,100 $ đồng.}
% 	\loigiai{
% 		Từ ngày $1$ tháng $1$ năm $2016$ đến ngày $30$ tháng $4$ năm $2016$ có tổng cộng $31+29+31+30=121$ ngày.\\
% 		Gọi $S$ là số tiền An tích lũy được vào ngày sinh nhật của bạn.\\
% 		Do An bỏ được $100$ đồng vào ngày đầu tiên nên số tiền An tích lũy được vào ngày thứ $n$ là 
% 		$$S= 100 + 100(n-1).$$
% 		Vậy tổng số tiền An tích lũy được là:
% 		$$
% 		S=100 + 200 + \cdots + 12\,100 = \frac{121(100 + 12\,100)}{2} = 738\,100
% 		.$$
% 		Vậy An đã tích lũy được $738\,100$ đồng vào ngày sinh nhật của bạn.}
% \end{bt}

% \begin{bt}%[TH]%[DCHT Toán 11 - KNTT - Nguyễn Hữu Đức] %[1K2Y6-6]
% 	Người ta trồng $ 3\,003 $ cây theo dạng một hình tam giác như sau: hàng thứ nhất trồng $ 1 $ cây, hàng thứ hai trồng $ 2 $ cây, hàng thứ ba trồng $ 3 $ cây... cứ tiếp tục trồng như thế cho đến khi hết số cây. Số hàng cây được trồng là bao nhiêu?
% 	\dapso{$ 77 $ hàng.}
% 	\loigiai{
% 		Tổng số cây trồng được là $1 + 2 + 3 + \cdots + n$, nghĩa là tổng của $n$ số tự nhiên đầu tiên. Ta cần tìm số $n$ để tổng này bằng $3003$.\\
% 		Ta có công thức tổng của $n$ số tự nhiên đầu tiên là:
% 		$$
% 		1 + 2 + 3 + \cdots + n = \frac{n(n+1)}{2}.
% 		$$
% 		Giải phương trình:
% 		$$
% 		\frac{n(n+1)}{2} = 3\,003.
% 		$$
% 		Ta có:
% 		$$
% 		n(n+1) = 6\,006
% 		\Rightarrow n=77.$$
% 		Vậy số hàng cây được trồng là $77$.
% 	}
% \end{bt}
% \begin{bt}%[TH]%[DCHT Toán 11 - KNTT - Nguyễn Hữu Đức] %[1K2B6-6]
% 	Một công ty định mức sản phẩm hàng tháng theo cấp số cộng. Ví dụ, sản lượng hàng tháng của một công ty được tăng thêm $10$ sản phẩm so với tháng trước. Nếu công ty sản xuất được $ 100 $ sản phẩm trong tháng này, hỏi công ty sẽ sản xuất được bao nhiêu sản phẩm trong tháng thứ $ 12 $?
% 	\dapso{$ 210 $ sản phẩm.}
% 	\loigiai{
% 		Công thức cấp số cộng được sử dụng để tính sản lượng hàng tháng của công ty. Nếu công ty sản xuất được $100$ sản phẩm trong tháng này và sản lượng hàng tháng được tăng thêm $10$ sản phẩm so với tháng trước, ta có thể sử dụng công thức sau để tính sản lượng hàng tháng của công ty trong tháng thứ $12$:
% 		$$
% 		a_n = a_1 + (n-1)d.
% 		$$
% 		Trong đó $a_1$ là sản lượng hàng tháng ban đầu, $d$ là công sai và $n$ là số tháng.\\
% 		Với bài toán này, ta có: $a_1 = 100$, $d = 10$, $n = 12$.\\
% 		Sản lượng hàng tháng của công ty trong tháng thứ $12$ là:
% 		$$
% 		a_{12} = a_1 + (n-1)d = 100 + (12-1) \times 10 = 210.
% 		$$
% 		Vậy công ty sẽ sản xuất được $210$ sản phẩm trong tháng thứ $12$.
% 	}
% \end{bt}

\subsubsection{Câu hỏi trắc nghiệm}
\Opensolutionfile{ans}[ans/ans-1K2-2-dang4]
\begin{ex}%[TH]%[DCHT Toán 11 - KNTT - Nguyễn Hữu Đức]%[1K2B6-6]
	Một công ty đang cần tuyển dụng thêm nhân viên. Công ty quyết định tăng số lượng nhân viên hàng tháng theo cấp số cộng. Nếu công ty đã có $ 20 $ nhân viên và quyết định tăng thêm $ 2 $ nhân viên hàng tháng, hỏi sau bao nhiêu tháng công ty sẽ có 50 nhân viên?
	\choice
	{$19$ tháng}
	{\True $16$ tháng}
	{$36$ tháng}
	{$26$ tháng}
	\loigiai{
		Để giải bài toán này, ta có thể sử dụng công thức cấp số cộng:
		$$
		a_n = a_1 + (n-1) \times d.
		$$
		Trong đó $a_1$ là số lượng nhân viên ban đầu, $d$ là số lượng nhân viên tăng hàng tháng và $n$ là số tháng.\\
		Ta cần tìm số tháng $n$ để công ty có được $50$ nhân viên. Thay các giá trị vào công thức cấp số cộng ta có:
		$$
		50 = 20 + (n-1) \times 2.
		$$
		Suy ra:
		$$
		n = \frac{50 - 20}{2} + 1 = 16
		.$$
		Vậy sau $16$ tháng kể từ khi công ty quyết định tăng số lượng nhân viên hàng tháng theo cấp số cộng, công ty sẽ có được $50$ nhân viên.
	}
\end{ex}
\begin{ex}%[VD]%[DCHT Toán 11 - KNTT - Nguyễn Hữu Đức] %[1K2B6-6]
	Một người đang tăng cường luyện tập thể thao hàng ngày. Anh ta quyết định tăng mức độ luyện tập theo cấp số cộng hàng tuần. Nếu anh ta bắt đầu với mức luyện tập $ 30 $ phút mỗi ngày và tăng thêm $ 5 $ phút mỗi ngày, hỏi anh ta sẽ luyện tập được bao lâu để đạt được mức luyện tập $ 60 $ phút mỗi ngày?
	\choice
	{$16$ ngày}
	{\True $6$ ngày}
	{$9$ ngày}
	{$7$ ngày}
	\loigiai{
		Gọi $n$ là số ngày liên tiếp mà người đó tăng mức độ luyện tập. Theo đó, mức độ luyện tập của người đó sau $n$ ngày là:
		$$
		30 + 5n\, \text{(phút)}.
		$$
		Vì để đạt được mức luyện tập $ 60 $ phút mỗi ngày nên:
		$$
		30 + 5n = 60.
		$$
		Từ đó suy ra:
		$$
		n = \frac{60-30}{5} = 6.
		$$
		Vậy người đó cần luyện tập liên tiếp trong $6$ ngày để đạt được mức luyện tập $60$ phút mỗi ngày.	
	}
\end{ex}
\begin{ex}%[VD]%[DCHT Toán 11 - KNTT - Nguyễn Hữu Đức] %[1K2Y6-6]
	Nếu một công ty công nghệ mới thành lập có số lượng người dùng ban đầu là $ 10\,000 $ và mỗi tháng tăng thêm cố định $ 5\,000 $ lượng người dùng, thì sau bao lâu có số lượng người dùng là $ 1 $ triệu. 
	\choice
	{\True $198$ tháng}
	{$197$ tháng}
	{$18$ tháng}
	{$98$ tháng}
	
	\loigiai{
		Ta cần tính số tháng $n$ theo công thức sau:
		$$10\,000 + 5\,000n = 1\,000\,000.$$
		$$\Rightarrow n = \frac{1\,000\,000 - 10\,000}{5\,000} = 198.$$
		Vậy sau khoảng $198$ tháng (khoảng $16$ năm và $6$ tháng), công ty sẽ đạt được $1$ triệu người dùng.
	}
\end{ex}
\begin{ex}%[VDC]%[DCHT Toán 11 - KNTT - Nguyễn Hữu Đức] %[1K2Y6-6]
	Một nhà đầu tư đang đầu tư vào một quỹ đầu tư với mức lợi nhuận cố định hàng năm. Nếu nhà đầu tư đầu tư vào quỹ đầu tư với số tiền ban đầu là $ 20 $ triệu đồng và mức lợi nhuận hàng năm là $ 10 $\%, hỏi số tiền nhà đầu tư sẽ nhận được sau $ 7 $ năm?
	\choice
	{\True $34$ triệu đồng}
	{$14$ triệu đồng}
	{$30$ triệu đồng}
	{$39$ triệu đồng}
	\loigiai{
		Với số tiền ban đầu là $ 20 $ triệu đồng và mức lợi nhuận hàng năm là $ 10 $\%, ta có thể tính được số tiền nhà đầu tư sẽ nhận được sau $ 1 $ năm, sau đó sử dụng cấp số cộng để tính số tiền nhà đầu tư sẽ nhận được sau $ 7 $ năm.
		
		Số tiền nhà đầu tư sẽ nhận được sau $ 1 $ năm là:
		
		$ 20 $ triệu đồng $\times$ $ 10 $\% = $ 2 $ triệu đồng
		
		Số tiền nhà đầu tư sẽ nhận được sau $ 7 $ năm là:
		
		$ 2 $ triệu đồng $\times$ $ 7 $ năm + $ 20 $ triệu đồng = $ 34 $ triệu đồng
		
		Vậy sau $ 7 $ năm, nhà đầu tư sẽ nhận được tổng cộng $ 34 $ triệu đồng.
	}
\end{ex}
\begin{ex}%[VDC]%[DCHT Toán 11 - KNTT - Nguyễn Hữu Đức] %[1K2Y6-6]
	Một công ty sản xuất bánh kẹo tăng sản lượng sản phẩm của mình lên mỗi tháng. Nếu sản lượng ban đầu là $ 1\,000 $ sản phẩm, một sản phẩm lợi nhuận $ 1 $ USD và tăng thêm $ 200 $ sản phẩm mỗi tháng, thì sau bao nhiêu tháng lợi nhuận công ty $ 1 $ triệu đô.
	\choice
	{$8\,000$ tháng}
	{$7\,000$ tháng}
	{$9\,000$ tháng}
	{\True $5\,000$ tháng}
	\loigiai{
		Để tính thời gian công ty đạt được lợi nhuận 1 triệu đô, chúng ta cần biết lợi nhuận của công ty đạt được bao nhiêu sau mỗi tháng.\\
		Giả sử sản lượng ban đầu là $1\,000$ sản phẩm một sản phẩm lợi nhuận $1$ USD và tăng thêm $200$ sản phẩm mỗi tháng. Ta có thể tính được lợi nhuận của công ty sau mỗi tháng như sau:
		\begin{itemize} 
			\item Tháng 1: $1\,000 \times 1 = 1000$ USD.
			\item Tháng 2: $(1\,000 + 200) \times 1 = 1200$ USD.
			\item Tháng 3: $(1\,000 + 2 \times 200) \times 1 = 1\,400$ USD.
			\item Tháng 4: $(1\,000 + 3 \times 200) \times 1 = 1\,600$ USD.
			\item Tháng $n$: $(1\,000 + (n - 1) \times 200) \times 1 = (n - 1) \times 200 + 1\,000$ USD.
		\end{itemize}
		Để tính thời gian để công ty đạt được lợi nhuận $1$ triệu đô, ta giải phương trình sau:
		
		$(n - 1) \times 200 + 1\,000 = 10^6$
		
		$\Rightarrow (n - 1) \times 200 = (10^6 - 1000)$
		
		$\Rightarrow n - 1 = \dfrac{10^6 - 1\,000}{200}$
		
		$\Rightarrow n = \dfrac{10^6 - 1\,000}{200} + 1$
		
		$\Rightarrow n = 5\,001$
		
		Vậy sau $5\,000$ tháng, công ty sẽ đạt được lợi nhuận $1$ triệu đô.
	}
\end{ex}
\begin{ex}%[VDC]%[DCHT Toán 11 - KNTT - Nguyễn Hữu Đức] %[1K2Y6-6]
	Một công ty tăng lương cho nhân viên hàng năm bằng cách thêm một số tiền cố định vào lương của họ. Ví dụ: Nếu lương ban đầu của một nhân viên là $ 10 $ triệu đồng và công ty tăng lương $ 2 $ triệu đồng mỗi năm, thì lương của nhân viên sẽ là bao nhiêu nếu làm cho công ty $ 19 $ năm?
	\choice
	{$ 16 $ triệu đồng}
	{$ 26 $ triệu đồng}
	{$ 28 $ triệu đồng}
	{\True $ 46 $ triệu đồng}
	\loigiai{
		Do tăng lương cho nhân viên hàng năm bằng cách thêm một số tiền cố định nên ta có thể sử dụng công thức tính số hạng thứ $ n $ của cấp số cộng
		$ a_n = a_1 + (n - 1)d $.\\
		Ở bài toán này, ta có:\\
		$ a_1 = 10 $ (triệu đồng) là lương ban đầu của nhân viên.\\
		$ d = 2 $ (triệu đồng) là công sai của cấp số cộng.\\
		$ n = 19  $ là số thứ tự của số hạng.\\
		Ta thay các giá trị này vào công thức trên để tính lương của nhân viên sau $ 19 $ năm:\\
		$ a_{19} = 10 + (19 - 1)2 \Rightarrow$
		$ a_{19} = 46 $ (triệu đồng).\\
		Vậy lương của nhân viên sau $ 19 $ năm làm việc cho công ty là $ 46 $ triệu đồng.
	}
\end{ex}

\begin{ex}%[VDC]%[DCHT Toán 11 - KNTT - Nguyễn Hữu Đức] %[1K2Y6-6]
	Tài sản thường bị khấu hao khiến chúng có tuổi thọ hữu ích giới hạn. Ví dụ, nếu một công ty mua một chiếc xe tải với giá $ 35\,000 $ đô la và nó bị khấu hao với tốc độ không đổi là $ 700 $ đô la mỗi tháng, thì sau bao lâu giá trị của nó còn $ 5\,000 $ đô la.
	\choice
	{$ x = 23 $ tháng}
	{\True  $ x = 43 $ tháng}
	{$ x = 41 $ tháng}
	{$ x = 40 $ tháng}
	\loigiai{
		\textit{Cách 1:} Thời gian để giá trị của chiếc xe tải trên được khấu hao xuống còn $5.000 $ đô la có thể được tính bằng cách sử dụng công thức sau:\\
		Giá trị khởi đầu của chiếc xe tải là $35\,000$
		Giá trị cuối cùng của chiếc xe tải là $5\,000$
		Tốc độ khấu hao tương ứng $700$/tháng\\
		Để tìm ra thời gian cần thiết để giá trị của chiếc xe tải giảm xuống còn $5.000$, ta cần tìm số tháng được khấu hao.\\
		Giả sử số tháng cần khấu hao là $ x $ tháng.\\
		Giá trị của chiếc xe tải sau $ x $ tháng khấu hao được tính bằng:\\ 
		$35\,000 - 700x = 5\,000$.\\
		Giải phương trình trên ta có: $ x \approx 43 $ tháng\\
		Vì vậy, sau $ 43 $ tháng, giá trị của chiếc xe tải sẽ giảm xuống còn $5\,000$.
		Ngoài ra ta có thể giải theo cấp số cộng như sau:\\
		\textit{Cách 2:} Ta có thể sử dụng cộng thức tính số hạng thứ $ n $ của cấp số cộng
		$ a_n = a_{1} + (n - 1)d $
		\begin{itemize}
			\item $ u_1 = 35\,000 $ (đô la) là giá trị ban đầu của xe tải.
			\item $ d = -700 $ (đô la) là công sai của cấp số cộng (âm vì giá trị xe tải giảm).
			\item $ a_n = 5\,000 $ (đô la) là giá trị cuối cùng của xe tải.
		\end{itemize}
		Ta thay các giá trị này vào công thức trên để tính số tháng mà xe tải bị khấu hao đến $ 5\,000 $ đô la:
		$$ 5\,000 = 35\,000 + (n - 1)(-700)\Rightarrow n = 43{,}857.$$
		Vậy sau khoảng $ 43{,}857 $ tháng, tức là khoảng $ 3 $ năm và $ 7 $ tháng, giá trị của xe tải sẽ còn khoảng $ 5\,000 $ đô la.
	}
\end{ex}
\begin{ex}%[VDC]%[DCHT Toán 11 - KNTT - Nguyễn Hữu Đức] %[1K2Y6-6]
	Các thiết bị điện tử như máy tính, điện thoại, hoặc máy ảnh thường bị khấu hao nhanh chóng do sự phát triển của công nghệ mới. Ví dụ, nếu một người mua một máy tính Macbook với giá $ 2\,000 $ đô la và nó bị khấu hao với tốc độ không đổi là $ 100 $ đô la mỗi tháng, thì giá trị của Macbook còn lại $ 1\,000 $ đô la sau bao nhiêu tháng?
	%\dapso{$ 11 $ tháng.}
	\choice
	{$ x = 12 $ tháng}
	{$ x = 43 $ tháng}
	{\True  $ x = 11 $ tháng}
	{$ x = 10 $ tháng}
	\loigiai{
		Để giải bài toán này, ta có thể sử dụng công thức tính số hạng thứ $ n $ của cấp số cộng $ a_n = a + (n - 1)d. $\\
		Ở bài toán này, ta có:\\
		$ a = 2\,000 $ (đô la) là giá trị ban đầu của máy tính Macbook.\\
		$ d = -100 $ (đô la) là công sai của cấp số cộng (âm vì giá trị máy tính giảm).\\
		$ a_n = 1\,000 $ (đô la) là giá trị cuối cùng của máy tính Macbook.\\
		Ta thay các giá trị này vào công thức trên để tính số tháng mà máy tính bị khấu hao đến $ 1\,000 $ đô la:
		$$ 1\,000 = 2\,000 + (n - 1)(-100)\Rightarrow n=11. $$
		Vậy sau $ 11 $ tháng, giá trị của máy tính Macbook sẽ còn $ 1\,000 $ đô la.
	}
\end{ex}
\begin{ex}%[VDC]%[DCHT Toán 11 - KNTT - Nguyễn Hữu Đức] %[1K2Y6-6]
	Ban đầu có 1m$^2$ bèo sinh sôi trên mặt hồ biết tốc độ sinh sôi ngày sau hơn ngày trước $ 0{,}5 $m$^2 $. Biết diện tích mặt hồ nước là $ 120 $m$^2 $ hỏi sau bao lâu bèo phủ đầy mặt hồ?
	%\dapso{$ 238 $ ngày}
	\choice
	{$ x = 120 $ tháng}
	{$ x = 143 $ tháng}
	{\True  $ x = 238 $ tháng}
	{$ x = 130 $ tháng}
	\loigiai{
		Giả sử sau $ x $ ngày, diện tích của bèo phủ đầy mặt hồ là $ S m^2 $.
		
		Theo đề bài, ta biết được rằng:
		\begin{itemize}
			\item Tốc độ sinh sôi của bèo là $ 0{,}5 $m$^2 $/ngày.
			\item Ban đầu, diện tích của bèo là  1 m$^2 $.
			\item Diện tích mặt hồ là  $ 120 $m$^2 $.
		\end{itemize}
		Vậy ta có phương trình sau đây:	$ S = 1 + 0{,}5x. $\\
		Điều kiện để bèo phủ đầy mặt hồ là $ S = 120 $.\\
		$ 1 + 0{,}5x = 120 $ hay	$ 0{,}5x = 119 $ $\Rightarrow x = 238 $ ngày.\\
		Vậy sau $ 238 $ ngày, bèo sẽ phủ đầy mặt hồ.
	}
\end{ex}
\begin{ex}%[VDC]%[DCHT Toán 11 - KNTT - Nguyễn Hữu Đức] %[1K2Y6-6]
	Nhà hát lớn Dạ Cỗ Vĩ Lan ở An Cư có hàng ghế đầu kí hiệu dãy A là $50$ chỗ hàng ghế, sau dãy B là $48$ chỗ và như thế hàng sau ít hơn hàng trước $ 2 $ ghế, biết hàng cuối cùng có $ 10 $ ghế. Tính tổng số dãy ghế và tổng số chỗ ngồi?
	%\dapso{ $21$ dãy và $ 630 $ chỗ.}
	\choice
	{\True  $21$ dãy và $ 630 $ chỗ}
	{$20$ dãy và $ 630 $ chỗ}
	{$11$ dãy và $ 630 $ chỗ}
	{$21$ dãy và $ 930 $ chỗ}
	\loigiai{
		Gọi $n$ là số dãy ghế. Theo đề bài, ta có:
		$$
		\begin{cases}
			S=50 + 48 + \cdots + 10 = \dfrac{50+10}{2}n \\
			S=\dfrac{2.50+(n-1)\cdot (-2)}{2}n
		\end{cases}
		$$
		Từ phương trình đầu tiên, ta có:
		$$
		S = 50 + 48 + \cdots + 10 = \frac{50+10}{2}n = 30n.
		$$
		Từ phương trình thứ hai, ta có:
		$$
		S = \frac{2\cdot 50+(n-1)\cdot(-2)}{2}n = (50 - n + 1)n = (51 - n)n.
		$$
		Do đó, ta có:
		$$
		30n = (51 - n)n
		\Rightarrow n=21.$$
		Vậy $n = 21$ dãy ghế và $ 30\cdot 21=630 $ ghế.
		
	}
\end{ex}
\begin{ex}%[VDC]%[DCHT Toán 11 - KNTT - Nguyễn Hữu Đức] %[1K2Y6-6]
	Người ta trồng  cây theo dạng một hình tam giác như sau: hàng thứ nhất trồng $ 1 $ cây, hàng thứ hai trồng $ 3 $ cây, hàng thứ ba trồng $ 5 $ cây,... cứ tiếp tục trồng như thế cho đến khi hết số cây là $ 6\,561 $. Số hàng cây được trồng là bao nhiêu?
	%\dapso{ $ 81 $ hàng.}
	\choice
	{\True  $ 81 $ hàng}
	{$ 16 $ hàng}
	{$ 100 $ hàng}
	{$ 89 $ hàng}
	\loigiai{
		Để giải bài toán này, ta cần tìm số hàng cây được trồng cho đến khi tổng số cây là $ 2023 $. 
		\begin{itemize}
			\item Hàng thứ nhất trồng $ 1 $ cây. 
			\item Hàng thứ hai trồng $ 3 $ cây ($ 1 $ cây $ + 2 $ cây).
			\item Hàng thứ ba trồng $ 5 $ cây ($ 1 $ cây $ + 2 $ cây $ + 2 $ cây).
			\item ...
		\end{itemize}
		Vậy ta thấy rằng số cây trồng trong hàng thứ $n$ là $(n-1)\cdot 2+1$. \\
		Số cây được trồng trong $n$ hàng đầu tiên là: 
		$$1 + 3 + 5 + ... + (2n-1) = n^2.$$ 
		Để tìm số hàng cây được trồng cho đến khi tổng số cây là $ 6561 $, ta giải phương trình sau:\\ 
		$n^2 = 6\,561.$ 
		Vậy số hàng cây được trồng là $ 81 $ hàng.
	}
\end{ex}
\begin{ex}%[VDC]%[DCHT Toán 11 - KNTT - Nguyễn Hữu Đức] %[1K2Y6-6]
	Người ta thả một $ 1 $ m$^2$ lá bèo vào một hồ nước. Kinh nghiệm cho thấy sau $ x $ giờ, bèo sẽ sinh sôi kín cả mặt hồ $ 500 $ m$^2 $. Biết rằng sau mỗi giờ, lượng lá bèo tăng thêm $ 0{,}5 $ m$^2 $ và tốc độ tăng không đổi tìm $ x $?
	%\dapso{ $999$ giờ.}
	\choice
	{$888$ giờ}
	{$777$ giờ}
	{\True  $999$ giờ}
	{$700$ giờ}
	\loigiai{
		Bài toán này có thể giải bằng cách sử dụng công thức tăng trưởng của bèo. Giả sử lượng lá bèo ban đầu là $ 1 $ m$^2$, sau mỗi giờ lượng lá bèo tăng thêm $ 0{,}5 $ m$^2$. Sau $x$ giờ, lượng lá bèo đã phủ kín mặt hồ $ 500 $ m$^2$. Ta có thể viết phương trình sau:
		$$1 + 0{,}5x = 500.$$
		Giải phương trình ta được:
		$$x = \frac{500-1}{0{,}5} \approx 999.$$
		Vậy sau khoảng $999$ giờ (khoảng 41 ngày), lượng lá bèo sẽ phủ kín mặt hồ $ 500 $ m$^2$.
	}
\end{ex}
\Closesolutionfile{ans}
% \begin{indapan}{10}
% 	{ans/ans-1K2-2-dang6}
% \end{indapan}
%%Bài 7. CSN
\def\tenchude{CẤP SỐ NHÂN}
\setcounter{section}{6}
\setcounter{dang}{0}
\setcounter{ex}{0}
\setcounter{bt}{0}
\setcounter{vd}{0}
\section{Cấp số nhân}
\subsection{Tóm tắt lý thuyết}
\begin{tomtat}
	\subsubsection{Định nghĩa} 
	Cấp số nhân là một dãy số (hữu hạn hoặc vô hạn) mà trong đó, kể từ số hạng thứ hai, mỗi số hạng đều bằng tích một số đứng ngay trước nó với một số $ q $ không đổi, nghĩa là:
	$$ u_{n}=u_{n-1}\cdot q\,\,\text{với}\,\forall n\in \mathbf{N}{,}\,n\ge 2 $$
	Số $ q $ được gọi là công bội của cấp số nhân
	\subsubsection{Số hạng tổng quát của cấp số nhân}
	Nếu cấp số nhân $ (u_n) $ có số hạng đầu là $ u_1 $ và công bội $ q $ thì số hạng tổng quát $ u_n $ của nó được xác định bởi công thức:
	$$u_n = u_1 \cdot q^{n-1}\,,n\ge 2$$
	\subsubsection{Tổng của $ n $ số hạng đầu tiên của cấp số nhân}
	Giả sử $ (u_n) $ là cấp số nhân có công bội $ q\ne 1 $. Đặt $ S_n=u_1+u_2+\cdots +u_n, $ khi đó
	$$S_n = u_1\cdot\frac{1-q^n}{1-q}.$$
	\begin{note}
		Khi $ q=1 $ thì $ S_n=n\cdot u_1 $.
	\end{note}	
	\begin{itemize}
		\item Công bội của cấp số nhân: $q = \sqrt[n-1]{\frac{u_n}{u_1}}$.
		\item Số hạng đầu tiên của cấp số nhân: $u_1 = \frac{u_n}{q^{n-1}}$.
		\item $ a,b,c $ là ba số hạng liên tiếp cấp số nhân thì $ a\cdot c=b^2 $. 
	\end{itemize}	
\end{tomtat}
\subsection{Các dạng toán thường gặp}
\begin{dang}{Nhận diện cấp số nhân, công bội $ q $}
	Để nhận diện (chứng minh) mỗi dãy số là cấp số nhân, ta làm như sau:\\
	Chứng minh $ u_{n+1}=u_nq $, $ \forall n\in\mathbb{N}^* $ và $ q $ là một số không đổi.\\
	Nếu $ u_n\ne 0 $, $ \forall n\in\mathrm{N}^* $ thì ta lập tỉ số $ \dfrac{u_{n+1}}{u_n}=k $.
	\begin{itemize}
		\item Nếu $ k $ là hằng số thì $ (u_n) $ là cấp số nhân với công bội $ q=k $.
		\item Nếu $ k $ phụ thuộc vào $ n $ thì $ (u_n) $ không phải là cấp số nhân.
	\end{itemize}
	Để chứng minh dãy $ (u_n) $ không phải là một cấp số nhân. Khi đó, ta chỉ cần chỉ ra ba số hạng liên tiếp không tạo thành một cấp số nhân, chẳng hạn $ \dfrac{u_3}{u_2}\ne \dfrac{u_2}{u_1} $.\\
	Để chứng minh ba số $ a,b,c $ theo thứ tự đó lập được một cấp số nhân, thì ta chứng minh $ ac=b^2 $ hoặc $ |b|=\sqrt{ac} $.	
\end{dang}
\subsubsection{Ví dụ minh hoạ}
\begin{vd}%[NB]%[DCHT Toán 11 - KNTT -Tên Huỳnh Thanh Chí]%[1K2Y7-1]
	Dãy số $ 1;1;1;1;\ldots $ có phải là một cấp số nhân hay không?
	\dapso{Dãy số $ 1;1;1;1;\ldots $ là một cấp số nhân.}
	\loigiai{
	Dễ thấy $ \dfrac{u_2}{u_1}=\dfrac{u_3}{u_2}=\ldots=1 $ là một số không đổi.\\
	Do đó dãy số $ 1;1;1;1;\ldots $ là một cấp số nhân.
	}
\end{vd}
\begin{vd}%[TH]%[DCHT Toán 11 - KNTT -Tên Huỳnh Thanh Chí] %[ID6 chương trình mới]
	Dãy số $ u_n=3^n $ có phải là một cấp số nhân không? Nếu có, hãy tìm công bội của cấp số nhân đó.
	\dapso{$ (u_n) $ là cấp số nhân với công bội $ q=3 $.}
	\loigiai{
	Ta có $ \dfrac{u_{n+1}}{u_n}=\dfrac{3^{n+1}}{3^n}=\dfrac{3^n\cdot 3}{3^n}=3 $ là số không đổi nên $ (u_n) $ là cấp số nhân với công bội $ q=3 $.
	}
\end{vd}
\begin{vd}%[TH]%[DCHT Toán 11 - KNTT -Tên Huỳnh Thanh Chí]%[1K2B7-1]
	Dãy số $ \heva{& u_1=3\\ & u_{n+1}=\dfrac{9}{u_n}} $ có phải là một cấp số nhân không? Nếu có, hãy tìm công bội của cấp số nhân đó.
	\dapso{$ (u_n) $ là một cấp số nhân với công bội $ q=1 $.}
	\loigiai{
	Xét dãy số $ \heva{& u_1=3\\ & u_{n+1}=\dfrac{9}{u_n}} $ có
	$ \dfrac{u_{n+1}}{u_n}=\dfrac{9}{u_n}:\dfrac{9}{u_{n-1}}=\dfrac{u_{n-1}}{u_n}\Rightarrow u_{n+1}=u_{n-1},\forall n\ge 2 $.\\
	Do đó ta có $ \heva{& u_1=u_3=u_5=\ldots=u_{2n+1}=\ldots \quad (1)\\ & u_2=u_4=u_6=\ldots=u_{2n}=\ldots \quad (2).} $\\
	Theo đề bài ta có $ u_1=3 \Rightarrow u_2=\dfrac{9}{u_1}=3 $ (3).\\
	Từ $ (1), (2) $ và $ (3) $ suy ra $ u_1=u_2=u_3=u_4=\ldots=u_{2n}=u_{2n+1}=\ldots $.\\
	Do đó $ (u_n) $ là một cấp số nhân với công bội $ q=1 $.
	}
\end{vd}
\begin{vd}%[TH]%[DCHT Toán 11 - KNTT -Tên Huỳnh Thanh Chí]%[1K2B7-1]
	Cho $ (u_n) $ là cấp số nhân có công bội $ q\ne 0,u_1\ne 0 $. Chứng minh rằng dãy số $ (v_n) $ với $ v_n=u_nu_{2n} $ cũng là một cấp số nhân.
	\dapso{$ (v_n) $ là một cấp số nhân với công bội là $ q^3 $.}
	\loigiai{
		Ta có $ \dfrac{v_n}{v_{n-1}}=\dfrac{u_nu_{2n}}{u_{n-1}u_{2(n-1)}}=\dfrac{u_1q^{n-1}\cdot u_1q^{2n-1}}{u_1q^{n-2}\cdot u_1q^{2n-3}}=q^3 $. Do đó $ (v_n) $ là một cấp số nhân với công bội là $ q^3 $.
	}
\end{vd}
\begin{vd}[VDT]%[DCHT Toán 11 - KNTT -Tên Huỳnh Thanh Chí]%[1K2K7-1]
	Cho dãy số $ (u_n) $ được xác định bởi $ \heva{& u_1=2\\ & u_{n+1}=4u_n+9},\forall n\in\mathbb{N}^* $. Chứng minh rằng dãy số $ (v_n) $ xác định bởi $ v_n=u_n+3,\forall n\in\mathbb{N}^* $ là một cấp số nhân. Hãy xác định số hạng đầu và công bội của cấp số nhân đó.
	\dapso{$ (v_n) $ là cấp số nhân với công bội $ q=4 $ và số hạng đầu $ v_1=5 $.}
	\loigiai{
	Ta có $ v_n=u_n+3 $ (1) và $ v_{n+1}=u_{n+1}+3 $ (2).\\
	Theo đề ta có $ u_{n+1}=4u_n+9 \Rightarrow u_{n+1}+3=4(u_n+3) $ (3).\\
	Thay (1) và (2)	vào (3) ta được $ v_{n+1}=4v_n \Rightarrow \dfrac{v_{n+1}}{v_n}=4,\forall n\in\mathbb{N}^* $.\\
	Suy ra $ (v_n) $ là cấp số nhân với công bội $ q=4 $ và số hàng đầu $ v_1=u_1+3=2+3=5 $.
	}
\end{vd}
\subsubsection{Bài tập tự luận}
 
\begin{bt}%[NB]%[DCHT Toán 11 - KNTT -Tên Huỳnh Thanh Chí]%[1K2Y7-1]
	Dãy số $25$; $5$; $1$; $\dfrac{1}{5}$; $\ldots$ có phải là một cấp số nhân không? Nếu có hãy tìm công bội của cấp số nhân đó.
	\dapso{Dãy số $25$; $5$; $1$; $\dfrac{1}{5}$; $\ldots$ là một cấp số nhân với công bội $ q=\dfrac{1}{5} $.}
	\loigiai{
	Ta có $ \dfrac{u_2}{u_1}=\dfrac{u_3}{u_2}=\ldots=\dfrac{1}{5} $ là một số không đổi.\\
	Do đó dãy số $25$; $5$; $1$; $\dfrac{1}{5}$; $\ldots$ là một cấp số nhân với công bội $ q=\dfrac{1}{5} $.
	}
\end{bt}
\begin{bt}%[NB]%[DCHT Toán 11 - KNTT -Tên Huỳnh Thanh Chí]%[1K2B7-1]
	Dãy số $1$; $n$; $n^2$; $n^3$; $ n^4 $; $\ldots$ (với $ n>1 $) có phải là một cấp số nhân không? Nếu có hãy tìm công bội của cấp số nhân đó.
	\dapso{Dãy số $1$; $n$; $n^2$; $n^3$; $ n^4 $; $\ldots$ (với $ n>1 $) là một cấp số nhân với công bội $ q=n $.}
	\loigiai{
		Ta có $ \dfrac{u_2}{u_1}=\dfrac{u_3}{u_2}=\ldots=n $ (với $ n>1 $) là một số không đổi.\\
		Do đó dãy số $1$; $n$; $n^2$; $n^3$; $ n^4 $; $\ldots$ (với $ n>1 $) là một cấp số nhân với công bội $ q=n $.
	}
\end{bt}
\begin{bt}%[TH]%[DCHT Toán 11 - KNTT -Tên Huỳnh Thanh Chí]%[1K2B7-1]
	Cho dãy số $ (u_n) $ được xác định bởi $ \heva{& u_1=2\\ & u_{n+1}=u_n^2} $. Hỏi dãy số $ (u_n) $ có là một cấp số nhân hay không?
	\dapso{Dãy số $ (u_n) $ không là một cấp số nhân.}
	\loigiai{
	Ta có $ u_2=u_1^2=4,u_3=u_2^2=16,u_4=u_3^2=256 $.\\
	Suy ra $ \dfrac{u_2}{u_1}=2 $; $ \dfrac{u_3}{u_2}=4 $ và $ \dfrac{u_4}{u_3}=16 $. Vì $ \dfrac{u_2}{u_1}\ne \dfrac{u_3}{u_2}\ne \dfrac{u_4}{u_3} $ nên $ (u_n) $ không là một cấp số nhân	
	}
\end{bt}
\begin{bt}%[TH]%[DCHT Toán 11 - KNTT -Tên Huỳnh Thanh Chí]%[1K2B7-1]
	Cho dãy số $ (u_n) $, biết $ u_1=2 $ và $ u_{n+1}=\dfrac{1}{3}u_n $. Chứng minh $ (u_n) $ là một cấp số nhân và tìm số hạng $ u_3 $.
	\dapso{}
	\loigiai{
	Ta có $ u_{n+1}=\dfrac{1}{3}u_n\Rightarrow \dfrac{u_{n+1}}{u_n}=\dfrac{1}{3} $ là một số không đổi nên $ (u_n) $ là một cấp số nhân với công bội là $ q=\dfrac{1}{3} $.\\
	Do đó $ u_3=u_2\cdot q=u_1\cdot q^2=2\cdot \dfrac{1}{3^2}=\dfrac{2}{9} $.
	}
\end{bt}
\begin{bt}%[TH]%[DCHT Toán 11 - KNTT -Tên Huỳnh Thanh Chí]%[1K2B7-1]
	Cho $ (u_n) $ là cấp số nhân có công bội $ q\ne 0,u_1\ne 0 $. Chứng minh rằng dãy số $ (v_n) $ với $ v_n=\dfrac{u_nu_{2n+1}}{4} $ cũng là một cấp số nhân.
	\dapso{$ (v_n) $ là một cấp số nhân với công bội là $ q^3 $.}
	\loigiai{
		Ta có $ \dfrac{v_n}{v_{n-1}}=\dfrac{\dfrac{u_nu_{2n+1}}{4}}{\dfrac{u_{n-1}u_{2(n-1)+1}}{4}}=\dfrac{u_1q^{n-1}\cdot u_1q^{2n}}{u_1q^{n-2}\cdot u_1q^{2n-2}}=q^3 $. Do đó $ (v_n) $ là một cấp số nhân với công bội là $ q^3 $.}
\end{bt}
\begin{bt}%[VD]%[DCHT Toán 11 - KNTT -Tên Huỳnh Thanh Chí]%[1K2K7-1]
	Cho dãy số $ (u_n) $ được xác định bởi $ \heva{& u_1=3\\ & u_{n+1}=2u_n-2},\forall n\in\mathbb{N}^* $. Chứng minh rằng dãy số $ (v_n) $ xác định bởi $ v_n=2u_n-4,\forall n\in\mathbb{N}^* $ là một cấp số nhân. Hãy xác định số hạng đầu và công bội của cấp số nhân đó.
	\dapso{$ (v_n) $ là cấp số nhân với công bội $ q=2 $ và số hạng đầu $ v_1=2 $.}
	\loigiai{
	Ta có $ v_n=2u_n-4 $ (1) và $ v_{n+1}=2u_{n+1}-4 $ (2).\\
	Theo đề ta có $ u_{n+1}=2u_n-2 \Rightarrow 2u_{n+1}-4=2(2u_n-4) $ (3).\\
	Thay (1) và (2)	vào (3) ta được $ v_{n+1}=2v_n \Rightarrow \dfrac{v_{n+1}}{v_n}=2,\forall n\in\mathbb{N}^* $.\\
	Suy ra $ (v_n) $ là cấp số nhân với công bội $ q=2 $ và số hàng đầu $ v_1=2u_1-4=2\cdot 3-4=2 $.
	}
\end{bt}
\subsubsection{Câu hỏi trắc nghiệm}
\Opensolutionfile{ans}[ans/ans-1K2-3-dang1]
\begin{ex}%[DCHT Toán 11 - KNTT -Tên Huỳnh Thanh Chí]%[1K2Y7-1]
	Trong các dãy số sau, dãy số nào là một cấp số nhân?
	\choice
	{\True $128$; $-64$; $32$; $-16$; $8$; $\ldots$} 
	{$\sqrt{2}$; $2$; $4$; $4\sqrt{2}$; $\ldots$}
	{$5$; $6$; $7$; $8$; $\ldots$}
	{$15$; $5$; $1$; $\dfrac{1}{5}$; $\ldots$}
	\loigiai{
	Xét phương án $128$; $-64$; $32$; $-16$; $8$; $\ldots$. \\
	Có $ \dfrac{u_2}{u_1}=\dfrac{u_3}{u_2}=\ldots=-\dfrac{1}{2} $ là một số không đổi nên dãy số $128$; $-64$; $32$; $-16$; $8$; $\ldots$ là một cấp số nhân.
	}
\end{ex}
\begin{ex}%[DCHT Toán 11 - KNTT -Tên Huỳnh Thanh Chí]%[1K2Y7-1]
	Dãy số nào sau đây \textbf{không phải} là cấp số nhân?
	\choice
	{$1$; $-1$; $1$; $-1$; $\ldots$}
	{$3$; $3^2$; $3^3$; $3^4$; $\ldots$}
	{$a$; $a^3$; $a^5$; $a^7$; $\ldots$  $(a\not =0)$}
	{\True $\dfrac{1}{\pi}$; $\dfrac{1}{{\pi}^2}$; $\dfrac{1}{{\pi}^4}$; $\dfrac{1}{{\pi}^6}$; $\ldots$}
	\loigiai{
	Xét dãy $\dfrac{1}{\pi}$; $\dfrac{1}{{\pi}^2}$; $\dfrac{1}{{\pi}^4}$; $\dfrac{1}{{\pi}^6}$; $\ldots$ có $ \dfrac{u_2}{u_1}\ne \dfrac{u_3}{u_2} \left(\dfrac{1}{\pi}\ne\dfrac{1}{\pi^2} \right) $.\\
	Do đó dãy $\dfrac{1}{\pi}$; $\dfrac{1}{{\pi}^2}$; $\dfrac{1}{{\pi}^4}$; $\dfrac{1}{{\pi}^6}$; $\ldots$ không là một cấp số nhân.
	}
\end{ex}
\begin{ex}%[DCHT Toán 11 - KNTT -Tên Huỳnh Thanh Chí]%[1K2Y7-1]
	Dãy số $1$; $2$; $4$; $8$; $16$; $32$; $\ldots$ là một cấp số nhân với 
	\choice
	{Công bội là $1$ và số hạng đầu tiên là $2$}
	{\True Công bội là $2$ và số hạng đầu tiên là $1$}
	{Công bội là $2$ và số hạng đầu tiên là $2$}
	{Công bội là $1$ và số hạng đầu tiên là $1$}
	\loigiai{
	Ta có $ q=\dfrac{u_2}{u_1}=\dfrac{u_3}{u_2}=\ldots=2 $. \\
	Vậy dãy số đã cho là một cấp số nhân với công bội là $ q=2 $ và số hạng đầu tiên là $ u_1=1 $.
	}
\end{ex}
\begin{ex}%[DCHT Toán 11 - KNTT -Tên Huỳnh Thanh Chí]%[1K2Y7-1]
	Cho cấp số nhân $(u_n)$ với $u_1=-2$ và công bội $q=-5$. Viết bốn số hạng đầu tiên của cấp số nhân.
	\choice
	{$-2$; $10$; $50$; $-250$}
	{\True $-2$; $10$; $-50$; $250$}
	{$-2$; $-10$; $-50$; $-250$}
	{$-2$; $10$; $50$; $250$}
	\loigiai{
	Vì $ (u_n) $ là một cấp số nhân nên ta có $ u_{n+1}=u_nq $. \\
	Do đó $ u_2=u_1q=(-2)\cdot (-5)=10 $, $ u_3=u_2q=10\cdot (-5)=-50 $, $ u_4=u_3q=(-50)\cdot (-5)=250 $.\\
	Vậy bốn số hạng đầu tiên của cấp số nhân đó là $ -2; 10; -50; 250 $.
	}
\end{ex}
\begin{ex}%[DCHT Toán 11 - KNTT -Tên Huỳnh Thanh Chí]%[1K2B7-1]
	Một cấp số nhân có hai số hạng liên tiếp là $3$ và $12$. Số hạng tiếp theo của cấp số nhân là
	\choice
	{$15$}
	{$21$}
	{$36$}
	{\True $48$}
	\loigiai{
		Một cấp số nhân có hai số hạng liên tiếp là $3$ và $12$, do đó ta có $ q=\dfrac{u_{n+1}}{u_n}=\dfrac{12}{3}=4 $.\\
		Vậy số hạng tiếp theo của cấp số nhân đó là $ u_{n+2}=u_{n+1}q=12\cdot 4=48 $.
	}
\end{ex}
\begin{ex}%[DCHT Toán 11 - KNTT -Tên Huỳnh Thanh Chí]%[1K2B7-1]
	Cho cấp số nhân $(u_n)$ có số hạng tổng quát là $u_n=\dfrac{3}{2}\cdot 5^n$. Khi đó số hạng đầu $u_1$ và công bội $q$ là
	\choice
	{$u_1=\dfrac{3}{2}, q=\dfrac{1}{5}$} 
	{$u_1=\dfrac{3}{2}, q=5$}
	{$u_1=\dfrac{15}{2}, q=\dfrac{1}{5}$}
	{\True $u_1=\dfrac{15}{2}, q=5$}
	\loigiai{	
	Ta có $ u_1=\dfrac{3}{2}\cdot5^1=\dfrac{15}{2}$ và $ u_2=\dfrac{3}{2}\cdot 5^2=\dfrac{75}{2} $.\\
	Vì $ (u_n) $ là một cấp số nhân nên $ q=\dfrac{u_2}{u_1}=\dfrac{75}{2}:\dfrac{15}{2}=5 $.
	}
\end{ex}
\begin{ex}%[DCHT Toán 11 - KNTT -Tên Huỳnh Thanh Chí]%[1K2B7-1]
	Trong các dãy số $(u_n)$ cho bởi số hạng tổng quát $u_n$ sau, dãy số nào là một cấp số nhân?
	\choice
	{\True $u_n=\dfrac{1}{3^{n-2}}$}
	{$u_n=\dfrac{n}{3^n}$}
	{$u_n=(n+2)\cdot 3^n$}
	{$u_n=n^2$}
	\loigiai{
		\begin{itemize}
			\item Với $u_n=\dfrac{1}{3^{n-2}}$, ta có $ q=\dfrac{u_{n+1}}{u_n}=\dfrac{1}{3^{n-3}}:\dfrac{1}{3^{n-2}}=3 $ là một số không đổi.\\
			Vậy dãy số $ (u_n) $ có số hạng tổng quát $u_n=\dfrac{1}{3^{n-2}}$ là một cấp số nhân.
			\item Với $u_n=\dfrac{n}{3^n}$, ta có $ q=\dfrac{u_{n+1}}{u_n}=\dfrac{n+1}{3^{n+1}}:\dfrac{n}{3^n}=\dfrac{n+1}{3n} $ không phải là một số không đổi.\\
			Vậy dãy số $ (u_n) $ có số hạng tổng quát $u_n=\dfrac{n}{3^n}$ không là một cấp số nhân.
			\item Với $u_n=(n+2)\cdot 3^n$, ta có $ q=\dfrac{u_{n+1}}{u_n}=\dfrac{(n+3)\cdot 3^{n+1}}{(n+2)\cdot 3^n}=\dfrac{3(n+3)}{n+2} $ không phải là một số không đổi.\\
			Vậy dãy số $ (u_n) $ có số hạng tổng quát $u_n=(n+2)\cdot 3^n$ không là một cấp số nhân.
			\item Với $u_n=n^2$, ta có $ q=\dfrac{u_{n+1}}{u_n}=\dfrac{(n+1)^2}{n^2}=\left(1+\dfrac{1}{n}\right)^2 $ không là một số không đổi.\\
			Vậy dãy số $ (u_n) $ có số hạng tổng quát $u_n=n^2$ không là một cấp số nhân.
		\end{itemize}
	}
\end{ex}
\begin{ex}%[DCHT Toán 11 - KNTT -Tên Huỳnh Thanh Chí]%[1K2B7-1]
	Trong các dãy số $(u_n)$ cho bởi số hạng tổng quát $u_n$ sau, dãy số nào là một cấp số nhân?
	\choice
	{$u_n=7-3n$}
	{$u_n=7-3^n$}
	{$u_n=\dfrac{7}{3n}$}
	{\True $u_n=7\cdot 3^n$}
	\loigiai{
	\begin{itemize}
		\item Với $u_n=7-3n$, ta có $ q=\dfrac{u_{n+1}}{u_n}=\dfrac{7-3(n+1)}{7-3n}=\dfrac{4-3n}{7-3n} $ không phải là một số không đổi.\\
		Vậy dãy số $ (u_n) $ có số hạng tổng quát $u_n=7-3n$ không là một cấp số nhân.
		\item Với $u_n=7-3^n$, ta có $ q=\dfrac{u_{n+1}}{u_n}=\dfrac{7-3^{n+1}}{7-3^n}=\dfrac{7-3\cdot 3^n}{7-3^n}=1-\dfrac{2\cdot 3^n}{7-3^n} $ không phải là một số không đổi.\\
		Vậy dãy số $ (u_n) $ có số hạng tổng quát $u_n=7-3^n$ không là một cấp số nhân.
		\item Với $u_n=\dfrac{7}{3n}$, ta có $ q=\dfrac{u_{n+1}}{u_n}=\dfrac{7}{3(n+1)}:\dfrac{7}{3n}=\dfrac{n}{n+1} $ không phải là một số không đổi.\\
		Vậy dãy số $ (u_n) $ có số hạng tổng quát $u_n=\dfrac{7}{3n}$ không là một cấp số nhân.
		\item Với $u_n=7\cdot 3^n$, ta có $ q=\dfrac{u_{n+1}}{u_n}=\dfrac{7\cdot 3^{n+1}}{7\cdot 3^n}=3 $ là một số không đổi.\\
		Vậy dãy số $ (u_n) $ có số hạng tổng quát $u_n=7\cdot 3^n$ là một cấp số nhân.
	\end{itemize}
	}
\end{ex}
\begin{ex}%[DCHT Toán 11 - KNTT -Tên Huỳnh Thanh Chí]%[1K2B7-1]
	Mệnh đề nào sau đây \textbf{sai}?
	\choice
	{Dãy số có tất cả các số hạng bằng nhau là một cấp số nhân}
	{Dãy số có tất cả các số hạng bằng nhau là một cấp số cộng}
	{Một cấp số cộng có công sai dương là một dãy số tăng}
	{\True Một cấp số nhân có công bội $q>1$ là một dãy tăng}
	\loigiai{
	\begin{itemize}
		\item Dãy số có tất cả các số hạng bằng nhau là một cấp số nhân là mệnh đề đúng. \\
		Vì xét dãy số $ (u_n) $ là một cấp số nhân.
		Khi đó $ u_{n+1}=u_n\cdot q $ với $ u_n\ne 0,q=1 $ thì $ u_{n+1}=u_n $.
		\item Dãy số có tất cả các số hạng bằng nhau là một cấp số cộng là mệnh đề đúng.\\
		Vì $ u_{n+1}=u_n+d $, với $ d=0 $ thì $ u_{n+1}=u_n $.
		\item Một cấp số cộng có công sai dương là một dãy số tăng là mệnh đề đúng.\\
		Ta xét dãy $ (u_n) $ là một cấp số cộng có công sai $ d>0 $.\\
		Vì $ u_{n+1}=u_n+d \Rightarrow u_{n+1}-u_n=d>0 $. \\
		Do đó dãy $ (u_n) $ là dãy số tăng.
		\item Một cấp số nhân có công bội $q>1$ là một dãy tăng là mệnh đề \textbf{sai}.\\
		Ta xét dãy số $ (u_n) $ là một cấp số nhân có công bội $ q>1 $.\\
		Vì $ u_{n+1}=u_nq $ với $ u_n\ne 0,q>1 $. Khi đó $ u_{n+1}-u_n=u_nq-u_n=u_n(q-1) $.\\
		Nếu $ u_n<0 $ thì $ u_{n+1}-u_n=u_nq-u_n=u_n(q-1) <0 $. \\
		Do đó dãy $ (u_n) $ là dãy số giảm.
	\end{itemize}	
	}
\end{ex}
\begin{ex}%[DCHT Toán 11 - KNTT -Tên Huỳnh Thanh Chí]%[1K2K7-1]
	Cho dãy số $(u_n)$ được xác định bởi $ u_1=2,u_n=2u_{n-1}+3n-1 $. Công thức số hạng tổng quát của dãy số đã cho là biểu thức có dạng $ a2^n+bn+c $, với $ a,b,c\in\mathbb{Z},n\ge 2, n\in\mathbb{N} $. Khi đó tổng $ a+b+c $ có giá trị bằng
	\choice
	{$ -4 $}
	{$ 4 $}
	{\True $ -3 $}
	{$ 3 $}
	\loigiai{
	Ta có $ u_n=2u_{n-1}+3n-1 \Leftrightarrow u_n+3n+5=2\left[u_{n-1}+3(n-1)+5\right] $ với $ n\ge 2, n\in\mathbb{N} $.\\
	Đặt $ v_n=u_n+3n+5 $, ta có $ v_n=2v_{n-1} $ với $ n\ge 2, n\in\mathbb{N} $.\\
	Như vậy $ (v_n) $ là cấp số nhân với công bội $ q=2 $ và $ v_1=10 $.\\
	Do đó $ v_n=10\cdot 2^{n-1}=5\cdot 2^n $.\\
	Suy ra $ u_n+3n+5=5\cdot 2^n $ hay $ u_n=5\cdot 2^n-3n-5 $ với $ n\ge 2, n\in\mathbb{N} $.\\
	Vậy $ a=5,b=-3,c=-5 $, suy ra $ a+b+c=-3 $.
	}
\end{ex}
\Closesolutionfile{ans}
% \begin{indapan}{10}
% 	{ans/ans-1K2-3-dang1}
% \end{indapan}
\begin{dang}{Số hạng tổng quát của cấp số nhân}
	Dựa vào giả thuyết, ta lập một hệ phương trình chứa công bội $ q $ và số hạng đầu $ u_n $. Giải hệ phương trình này tìm được $ u_1 $ và $ q $.\\
	Nếu cấp số nhân $ (u_n) $ có số hạng đầu $ u_1 $ và công bội $ q $ thì số hạng tổng quát $ u_n $ được xác định bởi công thức $$ u_n=u_1\cdot q^{n-1} \text{ với } n\ge 2. $$
\end{dang}
\subsubsection{Ví dụ minh hoạ}
\begin{vd}%[NB]%[DCHT Toán 11 - KNTT -Tên Huỳnh Thanh Chí]%[1K2Y7-2]
	Tìm số hạng tổng quát của dãy số $ 2;4;8;16;32;\ldots $, biết dãy $ (u_n) $ là một cấp số nhân.  
	\dapso{$ u_n=2\cdot 2^{n-1} $.}
	\loigiai{
		Vì dãy số $ (u_n) $ là một cấp số nhân nên $ q=\dfrac{u_2}{u_1}=\dfrac{u_3}{u_2}=\ldots=2 $ và số hạng đầu $ u_1=2 $.\\
		Do đó dãy số $ 2;4;8;16;32;\ldots $ là một cấp số nhân có số hạng tổng quát là $ u_n=u_1q^{n-1}=2\cdot 2^{n-1} $.
	}
\end{vd}
\begin{vd}%[TH]%[DCHT Toán 11 - KNTT -Tên Huỳnh Thanh Chí]%[1K2B7-2]
	Tìm số hạng đầu, công bội và số hạng tổng quát của cấp số nhân, biết $ \heva{& u_1+u_5=51\\ & u_2+u_6=102.} $
	\dapso{$ u_1=3 $, $ q=2 $ và $ u_n=3\cdot 2^{n-1} $.}
	\loigiai{
	Vì $ (u_n) $ là một cấp số nhân nên $ u_n=u_1\cdot q^{n-1} $.\\
	Ta có $ \heva{& u_1+u_5=51\\ & u_2+u_6=102}\Leftrightarrow\heva{& u_1+u_1q^4=51\\ & u_1q+u_1q^5=102}\Leftrightarrow\heva{& u_1(1+q^4)=51 \qquad (1)\\ & u_1q(1+q^4)=102 \qquad (2).} $\\
	Chia từng vế của $ (2) $ cho $ (1) $ ta được $ \dfrac{u_1q(1+q^4)}{u_1(1+q^4)}=\dfrac{102}{51} \Leftrightarrow q=2 $.\\
	Suy ra $ u_1=\dfrac{51}{1+q^4}=\dfrac{51}{17}=3 $, $ u_n=u_1\cdot q^{n-1}=3\cdot 2^{n-1} $.\\
	Vậy $ u_1=3 $, $ q=2 $ và $ u_n=3\cdot 2^{n-1} $.
	}
\end{vd}
\begin{vd}%[TH]%[DCHT Toán 11 - KNTT -Tên Huỳnh Thanh Chí]%[1K2B7-2]
	Tìm số hạng đầu, công bội và số hạng tổng quát của cấp số nhân, biết $ \heva{& u_1+u_6=30\\ & u_2+u_7=120.} $
	\dapso{$ u_1=\dfrac{6}{205} $, $ q=4 $ và $ u_n=\dfrac{6}{205}\cdot 4^{n-1} $.}
	\loigiai{
		Vì $ (u_n) $ là một cấp số nhân nên $ u_n=u_1\cdot q^{n-1} $.\\
		Ta có $ \heva{& u_1+u_6=30\\ & u_2+u_7=120}\Leftrightarrow\heva{& u_1+u_1q^5=30\\ & u_1q+u_1q^6=102}\Leftrightarrow\heva{& u_1(1+q^5)=30 \qquad (1)\\ & u_1q(1+q^5)=120 \qquad (2).} $\\
		Chia từng vế của $ (2) $ cho $ (1) $ ta được $ \dfrac{u_1q(1+q^5)}{u_1(1+q^5)}=\dfrac{120}{30} \Leftrightarrow q=4 $.\\
		Suy ra $ u_1=\dfrac{30}{1+q^5}=\dfrac{30}{1+4^5}=\dfrac{6}{205} $, $ u_n=u_1\cdot q^{n-1}=\dfrac{6}{205}\cdot 4^{n-1} $.\\
		Vậy $ u_1=\dfrac{6}{205} $, $ q=4 $ và $ u_n=\dfrac{6}{205}\cdot 4^{n-1} $.
	}
\end{vd}
\begin{vd}%[TH]%[DCHT Toán 11 - KNTT -Tên Huỳnh Thanh Chí]%[1K2B7-2]
	Tìm số hạng đầu, công bội và số hạng tổng quát của cấp số nhân, biết $ \heva{& u_3=40\\ & u_6=160.} $
	\dapso{$ u_1=\dfrac{40}{9} $, $ q=3 $ và $ u_n=40\cdot 3^{n-3} $.}
	\loigiai{
	Vì $ (u_n) $ là một cấp số nhân nên $ u_n=u_1\cdot q^{n-1} $.\\	
	Ta có $ \heva{& u_3=40\\ & u_6=1080}\Leftrightarrow \heva{& u_1q^2=40 \qquad (1)\\ & u_1q^5=1080\qquad (2).} $\\
	Chia từng vế của $ (2) $ cho $ (1) $ ta được $ \dfrac{u_1q^5}{u_1q^2}=\dfrac{1080}{40} \Leftrightarrow q^3=27 \Leftrightarrow q=3 $.\\
	Suy ra $ u_1=\dfrac{40}{q^2}=\dfrac{40}{3^2}=\dfrac{40}{9} $, $ u_n=u_1\cdot q^{n-1}=\dfrac{40}{9}\cdot 3^{n-1}=40\cdot 3^{n-3} $.\\
	Vậy $ u_1=\dfrac{40}{9} $, $ q=3 $ và $ u_n=40\cdot 3^{n-3} $.
	}
\end{vd}
\begin{vd}[VDT]%[DCHT Toán 11 - KNTT -Tên Huỳnh Thanh Chí]%[1K2K7-2]
	Tìm số hạng đầu, công bội và số hạng tổng quát của cấp số nhân có công bội $ q\in \mathbb{Z},q\ne 0 $, biết $ \heva{& u_2+u_4=10\\ & u_1+u_3+u_5=-21.} $
	\dapso{$ u_1=-1 $, $ q=-2 $ và $ u_n=\dfrac{(-2)^n}{2} $.}
	\loigiai{
		Vì $ (u_n) $ là một cấp số nhân nên $ u_n=u_1\cdot q^{n-1} $ với $ q\in \mathbb{Z},q\ne 0 $.\\
		Ta có $ \heva{& u_2+u_4=10\\ & u_1+u_3+u_5=-21}\Leftrightarrow\heva{& u_1q+u_1q^3=10\\ & u_1q+u_1q^2+u_1q^4=-21}\Leftrightarrow\heva{& u_1(q+q^3)=10 \qquad (1)\\ & u_1(1+q^2+q^4)=-21 \qquad (2).} $\\
		Chia từng vế của $ (2) $ cho $ (1) $ ta được 
		\allowdisplaybreaks
		\begin{eqnarray*}
			 \dfrac{u_1(1+q^2+q^4)}{u_1(q+q^3)}=\dfrac{-21}{10} 
			 &\Leftrightarrow& 10q^4+21q^3+10q^2+21q+10=0 \\
			 &\Leftrightarrow& (q+2)(2q+1)(5q^2-2q+5)=0 \\
			 &\Leftrightarrow& \hoac{& q=-2 \text{ (thỏa mãn)}\\ & q=-\dfrac{1}{2} \text{ (loại)}.}
		\end{eqnarray*}
		Suy ra $ u_1=\dfrac{10}{q+q^3}=-1 $, $ u_n=u_1\cdot q^{n-1}=(-1)\cdot (-2)^{n-1}=-(-2)^{n-1}=-\dfrac{(-2)^n}{-2}=\dfrac{(-2)^n}{2} $.\\
		Vậy $ u_1=-1 $, $ q=-2 $ và $ u_n=\dfrac{(-2)^n}{2} $.
	}
\end{vd}
\subsubsection{Bài tập tự luận}
 
% \begin{bt}%[NB]%[DCHT Toán 11 - KNTT -Tên Huỳnh Thanh Chí]%[1K2Y7-2]
% 	Tìm số hạng thứ $ 100 $ của cấp số nhân $ 8;-4;2;-1;\ldots $
% 	\dapso{$ u_{100}=-\dfrac{1}{2^{96}} $.}
% 	\loigiai{
% 	Cấp số nhân này có số hạng đầu $ u_1=8 $ và công bội $ q=\dfrac{-4}{8}=-\dfrac{1}{2} $.\\
% 	Do đó số hạng tổng quát $ u_n=8\cdot \left(-\dfrac{1}{2}\right)^{n-1} $.\\
% 	Vậy $ u_{100}=8\cdot \left(-\dfrac{1}{2}\right)^{100-1}=8\cdot \left(-\dfrac{1}{2}\right)^{99}=-\dfrac{1}{2^{96}} $.
% 	}
% \end{bt}
\begin{bt}%[NB]%[DCHT Toán 11 - KNTT -Tên Huỳnh Thanh Chí]%[1K2B7-2]
	Tìm số hạng tổng quát của dãy số $ 3;12;48;192;\ldots $, biết dãy $ (u_n) $ là một cấp số nhân.  
	\dapso{$ u_n=3\cdot 4^{n-1} $.}
	\loigiai{
		Vì dãy số $ (u_n) $ là một cấp số nhân nên $ q=\dfrac{u_2}{u_1}=\dfrac{12}{3}=4 $ và số hạng đầu $ u_1=3 $.\\
		Do đó dãy số $ 3;12;48;192;\ldots $ là một cấp số nhân có số hạng tổng quát là $ u_n=u_1q^{n-1}=3\cdot 4^{n-1} $.}
\end{bt}
\begin{bt}%[TH]%[DCHT Toán 11 - KNTT -Tên Huỳnh Thanh Chí]%[1K2B7-2]
	Tìm số hạng tổng quát của cấp số nhân, biết $ \heva{& u_1+u_3=51\\ & u_2+u_4=153.} $
	\dapso{$ u_n=\dfrac{51}{10}\cdot 3^{n-1} $.}
	\loigiai{
		Vì $ (u_n) $ là một cấp số nhân nên $ u_n=u_1\cdot q^{n-1} $.\\
		Ta có $ \heva{& u_1+u_3=51\\ & u_2+u_4=153}\Leftrightarrow\heva{& u_1+u_1q^2=51\\ & u_1q+u_1q^3=153}\Leftrightarrow\heva{& u_1(1+q^2)=51 \qquad (1)\\ & u_1q(1+q^2)=153 \qquad (2).} $\\
		Chia từng vế của $ (2) $ cho $ (1) $ ta được $ \dfrac{u_1q(1+q^2)}{u_1(1+q^2)}=\dfrac{153}{51} \Leftrightarrow q=3 $.\\
		Suy ra $ u_1=\dfrac{51}{1+q^2}=\dfrac{51}{10} $, $ u_n=u_1\cdot q^{n-1}=\dfrac{51}{10}\cdot 3^{n-1} $.\\
		Vậy số hạng tổng quát $ u_n=\dfrac{51}{10}\cdot 3^{n-1} $.
	}
\end{bt}
\begin{bt}%[TH]%[DCHT Toán 11 - KNTT -Tên Huỳnh Thanh Chí]%[1K2B7-2]
	Tìm số hạng đầu, công bội và số hạng tổng quát của cấp số nhân, biết $ \heva{& u_3=15\\ & u_6=120.} $
	\dapso{$ u_1=\dfrac{15}{4} $, $ q=2 $ và $ u_n=15\cdot 3^{n-3} $.}
	\loigiai{
		Vì $ (u_n) $ là một cấp số nhân nên $ u_n=u_1\cdot q^{n-1} $.\\	
		Ta có $ \heva{& u_3=15\\ & u_6=120}\Leftrightarrow \heva{& u_1q^2=15 \qquad (1)\\ & u_1q^5=120\qquad (2).} $\\
		Chia từng vế của $ (2) $ cho $ (1) $ ta được $ \dfrac{u_1q^5}{u_1q^2}=\dfrac{120}{15} \Leftrightarrow q^3=8 \Leftrightarrow q=2 $.\\
		Suy ra $ u_1=\dfrac{15}{q^2}=\dfrac{15}{2^2}=\dfrac{15}{4} $, $ u_n=u_1\cdot q^{n-1}=\dfrac{15}{4}\cdot 2^{n-1}=15\cdot 2^{n-3} $.\\
		Vậy $ u_1=\dfrac{15}{4} $, $ q=2 $ và $ u_n=15\cdot 3^{n-3} $.
	}
\end{bt}
\begin{bt}%[TH]%[DCHT Toán 11 - KNTT -Tên Huỳnh Thanh Chí]%[1K2B7-2]
	Tìm số hạng tổng quát của cấp số nhân, biết $ \heva{& u_4=35\\ & u_8=560.} $
	\dapso{$ u_n=35\cdot 2^{n-4} $ với $ q=2 $ hoặc $ u_n=35\cdot (-2)^{n-4} $ với $ q=-2 $.}
	\loigiai{
		Vì $ (u_n) $ là một cấp số nhân nên $ u_n=u_1\cdot q^{n-1} $.\\	
		Ta có $ \heva{& u_4=35\\ & u_8=560}\Leftrightarrow \heva{& u_1q^3=35 \qquad (1)\\ & u_1q^7=560\qquad (2).} $\\
		Chia từng vế của $ (2) $ cho $ (1) $ ta được $ \dfrac{u_1q^7}{u_1q^3}=\dfrac{560}{35} \Leftrightarrow q^4=16 \Leftrightarrow \hoac{& q=2\\ & q=-2.} $\\
		Với $ q=2 $. Suy ra $ u_1=\dfrac{35}{q^3}=\dfrac{35}{8} $, $ u_n=u_1\cdot q^{n-1}=\dfrac{35}{8}\cdot 2^{n-1}=35\cdot 2^{n-4} $.\\
		Với $ q=-2 $. Suy ra $ u_1=\dfrac{35}{q^3}=-\dfrac{35}{8} $, $ u_n=u_1\cdot q^{n-1}=-\dfrac{35}{8}\cdot (-2)^{n-1}=35\cdot (-2)^{n-4} $.\\
		Vậy $ u_n=35\cdot 2^{n-4} $ với $ q=2 $ hoặc $ u_n=35\cdot (-2)^{n-4} $ với $ q=-2 $.
	}
\end{bt}

\subsubsection{Câu hỏi trắc nghiệm}
\Opensolutionfile{ans}[ans/ans-1K2-3-dang2]
\begin{ex}%[DCHT Toán 11 - KNTT -Tên Huỳnh Thanh Chí]%[1K2Y7-2]
	Cho cấp số nhân $(u_n)$ có số hạng đầu là $u_1\ne 0$ và công bội $q\ne 0$. Số hạng tổng quát của cấp số nhân bằng
	\choice
	{$ u_{n}=u_1+(n-1)q $}
	{\True $ u_{n}=u_1\cdot q^{n-1} $}
	{$ u_{n}=u_1\cdot q^n $}
	{$ u_{n}=u_1\cdot q^{n+1} $}
	\loigiai{
	Số hạng tổng quát của cấp số nhân là $ u_{n}=u_1\cdot q^{n-1} $.
	}
\end{ex}
\begin{ex}%[DCHT Toán 11 - KNTT -Tên Huỳnh Thanh Chí]%[1K2Y7-2]
	Cấp số nhân  $\left(u_{n}\right)$ có $u_{n}=\dfrac{3}{5}\cdot 2^{n}$. Số hạng đầu tiên và công bội $ q $ là
	\choice
	{$u_1=\dfrac{6}{5},q=3$}
	{$u_1=\dfrac{6}{5},q=-2$}
	{\True $u_1=\dfrac{6}{5},q=2$}
	{$u_1=\dfrac{6}{5},q=5$}
	\loigiai{
	Ta có $u_{n}=\dfrac{3}{5}\cdot 2^{n}=\dfrac{6}{5}\cdot 2^{n-1}$, suy ra $ u_1=\dfrac{6}{5} $ và $ q=2 $.
	}
\end{ex}
\begin{ex}%[DCHT Toán 11 - KNTT -Tên Huỳnh Thanh Chí]%[1K2Y7-2]
	Cho cấp số nhân $(u_n)$ có $u_1=-3$ và công bội $q=\dfrac{2}{3}$. Chọn mệnh đề đúng?
	\choice
	{$u_5=-\dfrac{27}{16}$}
	{$u_5=-\dfrac{16}{27}$}
	{\True $u_5=\dfrac{16}{27}$}
	{$u_5=\dfrac{27}{16}$}
	\loigiai{
	Số hạng tổng quát của cấp số nhân là $ u_{n}=u_1\cdot q^{n-1}=3\cdot\left(\dfrac{2}{3}\right)^{n-1} $.\\
	Vậy $ u_5=3\cdot \left(\dfrac{2}{3}\right)^{5-1}=\dfrac{16}{27} $.
	}
\end{ex}
\begin{ex}%[DCHT Toán 11 - KNTT -Tên Huỳnh Thanh Chí]%[1K2Y7-2]
	Dãy số có số hạng tổng quát $u_{n}=\dfrac{1}{\sqrt{3}}^{2n}$ là một cấp số nhân có công bội $ q $ bằng
	\choice
	{$ \dfrac{1}{\sqrt{3}} $}
	{$ \sqrt{3} $}
	{$ \dfrac{1}{9} $}
	{\True $ \dfrac{1}{3} $}
	\loigiai{
		Ta có $u_{n}=\dfrac{1}{\sqrt{3}}^{2n}=\left[\left(\dfrac{1}{\sqrt{3}}\right)^2\right]^n=\left(\dfrac{1}{3}\right)^n=\dfrac{1}{3}\cdot\left(\dfrac{1}{3}\right)^{n-1} $.\\
		Suy ra công bội của cấp số nhân $ q=\dfrac{1}{3} $.
	}
\end{ex}
\begin{ex}%[DCHT Toán 11 - KNTT -Tên Huỳnh Thanh Chí]%[1K2Y7-2]
	Cho cấp số nhân $(u_n)$ có $u_1=1, u_2=-2$. Mệnh đề nào sau đây đúng?
	\choice
	{\True $u_{2024}=-2^{2023}$}
	{$u_{2024}=2^{2023}$}
	{$u_{2024}=-2^{2024}$}
	{$u_{2024}=2^{2024}$}
	\loigiai{
	Số hạng tổng quát của cấp số nhân là $ u_{n}=u_1\cdot q^{n-1}=\left(-2\right)^{n-1} $.\\
	Vậy $ u_{2024}=\left(-2\right)^{2024-1}=(-2)^{2023}=-2^{2023} $.
	}
\end{ex}
\begin{ex}%[DCHT Toán 11 - KNTT -Tên Huỳnh Thanh Chí]%[1K2B7-2]
	Cho cấp số nhân có $\heva{& u_4-u_2=54\\ & u_5-u_3=108}$. Số hạng đầu tiên $u_1$ và công bội $ q $ của cấp số nhân là
	\choice
	{\True $ u_1=9 $ và $ q=2 $ }
	{$ u_1=9 $ và $ q=-2 $}
	{$ u_1=-9 $ và $ q=2 $}
	{$ u_1=-9 $ và $ q=-2 $}
	\loigiai{
	Ta có $\heva{& u_4-u_2=54\\ & u_5-u_3=108}\Leftrightarrow \heva{& u_1q^3-u_1q=54\\ & u_1q^4-u_1q^2=108}\Leftrightarrow\heva{& u_1q(q^2-1)-54 \quad (1)\\ & u_1q^2(q^2-1)=108 \quad (2).} $\\
	Chia từng vế của $ (2) $ cho $ (1) $ ta được $ \dfrac{u_1q^2(q^2-1)}{u_1q(q^2-1)}=\dfrac{108}{54} \Leftrightarrow q=2 $.\\
	Suy ra $ u_1=\dfrac{54}{q^3-q}=\dfrac{54}{2^3-2}=9 $.
	}
\end{ex}
\begin{ex}%[DCHT Toán 11 - KNTT -Tên Huỳnh Thanh Chí]%[1K2B7-2]
	Cho cấp số nhân $\left( u_n\right)$ biết $\heva{& u_1+ u_2+ u_3=31\\& u_1+ u_3=26 }$. Giá trị $u_1$ và $ q $ là
	\choice
	{$ u_1=2; q=5 $ hoặc $u_1=25; q=\dfrac{1}{5}$}
	{$ u_1=5; q=1 $ hoặc $u_1=25; q=\dfrac{1}{5}$}
	{$ u_1=25; q=5 $ hoặc $u_1=1; q=\dfrac{1}{5}$}
	{\True$ u_1=1; q=5 $ hoặc $u_1=25; q=\dfrac{1}{5}$}
	\loigiai{
	Vì $ (u_n) $ là một cấp số nhân nên $ u_n=u_1\cdot q^{n-1} $.\\	
	Ta có $ \heva{& u_1+ u_2+ u_3=31\\& u_1+ u_3=26 }\Leftrightarrow\heva{& u_2=5\\ & u_1+u_3=26}\Leftrightarrow\heva{& u_1q=5 \quad (1)\\ & u_1(1+q^2)=26\quad (2).} $\\
	Chia từng vế của $ (2) $ cho $ (1) $ ta được $ \dfrac{q^2+1}{q}=\dfrac{26}{5} \Leftrightarrow 5q^2-26q+5=0 \Leftrightarrow \hoac{& q=5\\ & q=\dfrac{1}{5}.} $\\
	Với $ q=5 $. Suy ra $ u_1=\dfrac{5}{q}=\dfrac{5}{5}=1 $.\\
	Với $ q=\dfrac{1}{5} $. Suy ra $ u_1=\dfrac{5}{q}=5:\dfrac{1}{5}=25 $.\\
	Vậy $ u_1=1 $ với $ q=5 $ hoặc $ u_1=25 $ với $ q=\dfrac{1}{5} $.
	}
\end{ex}
\begin{ex}%[DCHT Toán 11 - KNTT -Tên Huỳnh Thanh Chí]%[1K2B7-2]
	Số hạng đầu tiên và công bội của cấp số nhân thỏa mãn $\heva{& u_5+u_2=36\\ & u_6-u_4=48}$ (với $ q>0 $) là
	\choice
	{$ u_1=4,q=4 $}
	{$ u_1=2,q=4 $}
	{\True$ u_1=2,q=2 $}
	{$ u_1=4,q=2 $}
	\loigiai{
	Ta có $\heva{& u_5+u_2=36\\ & u_6-u_4=48}\Leftrightarrow\heva{& u_1q^4+u_1q=36\\ & u_1q^5-u_1q^3=48}\Leftrightarrow\heva{& u_1q(q^3+1)=36\quad (1)\\ & u_1q(q^4-q^2)=48\quad (2).} $\\
	 Chia từng vế của $ (2) $ cho $ (1) $ ta được $$ \dfrac{u_1q(q^4-q^2)}{u_1q(q^3+1)}=\dfrac{48}{36}\Leftrightarrow \dfrac{q^4-q^2}{q^3+1}=\dfrac{4}{3}\Leftrightarrow 3q^4-4q^3-3q^2-4=0\hoac{& q=2\\ &q=-1.} $$
	 Từ điều kiện $ q>0 $ suy ra công bội của cấp số nhân là $ q=2 $, do đó $ u_1=\dfrac{36}{q^4+q}=2 $.\\
	 Vậy $ u_1=2 $ và $ q=2 $.
	}
\end{ex}
\begin{ex}%[DCHT Toán 11 - KNTT -Tên Huỳnh Thanh Chí]%[1K2B7-2]
	Cho cấp số nhân $u_2=\dfrac{1}{4},u_5=16$. Công bội và số hạng đầu tiên của cấp số nhân là
	\choice
	{$q=\dfrac{1}{2};u_1=\dfrac{1}{2}$}
	{$q=\dfrac{-1}{2};u_1=\dfrac{-1}{2}$}
	{\True $q=4;u_1=\dfrac{1}{16}$}
	{$q=-4;u_1=\dfrac{-1}{16}$}
	\loigiai{
	Ta có $ u_2=u_1q=\dfrac{1}{4} $ (1) và $ u_5=u_1q^4=16 $ (2).\\
	Lấy $ (2) $ chia cho $ (1) $ vế theo vế ta được $ \dfrac{u_1q^4}{u_1q}=\dfrac{16}{\tfrac{1}{4}} \Leftrightarrow q^3=64 \Leftrightarrow q=4 $.\\
	Suy ra $ u_1=\dfrac{1}{4}:q=\dfrac{1}{4}:4=\dfrac{1}{16} $.\\
	Vậy $ u_1=\dfrac{1}{16},q=4 $.
	}
\end{ex}
\begin{ex}%[DCHT Toán 11 - KNTT -Tên Huỳnh Thanh Chí]%[1K2K7-2]
	Người ta thiết kế một cái tháp gồm $ 11 $ tầng. Diện tích mặt trên của mỗi tầng bằng nửa diện tích mặt trên của tầng ngay bên dưới và diện tích mặt trên của tầng 1 bằng nửa diện tích của đế tháp (có diện tích là $12\ 288$ m$ ^2 $). Diện tích mặt trên cùng (của tầng thứ $ 11 $) có giá trị nào sau đây?
	\choice
	{\True$ 6 $ m$ ^2 $}
	{$ 8 $ m$ ^2 $}
	{$ 10 $ m$ ^2 $}
	{$ 12 $ m$ ^2 $}
	\loigiai{
	Vì diện tích của mặt trên của mỗi tầng bằng nửa diện tích mặt trên của tầng ngay bên dưới và diện tích mặt trên của tầng 1 bằng nửa diện tích của đế tháp.\\
	Do đó diện tích của mỗi tầng tạo nên dãy số và dãy số đó là một cấp số nhân có công bội $ q=\dfrac{1}{2} $.\\
	Vậy số hạng tổng quát của cấp số nhân đó là $ u_n=12\ 288\cdot \left(\dfrac{1}{2}\right)^{n-1} $.\\
	Vì từ đế tháp đến tầng thứ 11 của tháp sẽ có 12 mặt nền, do đó diện tích của mặt của tầng thứ 11 là $ u_{12}=12\ 288\cdot\left(\dfrac{1}{2}\right)^{12-1}=6 $ m$ ^2 $.
	}
\end{ex}
\Closesolutionfile{ans}
% \begin{indapan}{10}
% 	{ans/ans-1K2-3-dang2}
% \end{indapan}
\begin{dang}{Tìm số hạng cụ thể của CSN}
	Ta chuyển các số hạng của CSN về số hạng đầu $u_1$ và công bội $q$. Sử dụng công thức $u_n=u_1\cdot q^{n-1}$. \\
	Chia hai phương trình vế theo vế ta thu được phương trình theo $q$. \\
	Giải tìm $q$ và $u_1$. Từ đó tìm được số hạng cần tìm thỏa ycbt.
\end{dang}
\subsubsection{Ví dụ minh hoạ}
\begin{vd}%[NB]%[1K2Y3-3]
	Cho $u_n$ là CSN thỏa $u_1=2$; $u_4=16$. Tìm số hạng thứ $5$ của CSN.
	\loigiai{
		Do $u_n$ là CSN nên ta có $u_4=u_1\cdot q^3 \Rightarrow q^3=\dfrac{u_4}{u_1}=8 \Rightarrow q=2$. \\
		Vậy $u_5=u_1\cdot q^4=2\cdot 2^4=32$.
	}
\end{vd}
\begin{vd}%[TH]%[1K2B3-3]
	Cho cấp số nhân $(u_n)$ có $\heva{&u_4+u_6=-540 \\ &u_3+u_5=180}$. Tính số hạng đầu $u_1$ và công bội $q$ của cấp Số nhân.
	\loigiai{
		Ta có $\heva{&u_4 + u_6=-540 \\ &u_3+u_5=180}
		\Leftrightarrow \heva{&u_1q^3(1+q^2)=-540 \\ &u_1q^2(1+q^2)=180}
		\Leftrightarrow \heva{&u_1=2 \\ &q=-3.}$ \\
		Vậy $\heva{&u_1=2 \\ &q=-3}$ là số hạng cần tìm.
	}
\end{vd}
\begin{vd}%[TH]%[1K2B3-3]
	Cho cấp số nhân có $u_1=-3$, $q=\dfrac{2}{3}$. Số $\dfrac{-96}{243}$ là số hạng thứ mấy của cấp số nhân?
	\loigiai{
		Giả sử số $\dfrac{-96}{243}$ là số hạng thứ $n$ của cấp số nhân.\\
		Ta có: $u_1\cdot q^{n-1}=\dfrac{-96}{243}\Leftrightarrow(-3)\left(\dfrac{2}{3}\right)^{n-1}=\dfrac{-96}{243}\Leftrightarrow n=6$.\\
		Vậy số $\dfrac{-96}{243}$ là số hạng thứ $6$ của cấp số nhân.}
\end{vd}
\begin{vd}%[TH]%[1K2B3-3]
	Cấp số nhân $\left(u_{n}\right)$ có số hạng tổng quát là $u_n=\dfrac{3}{5} \cdot 2^{n-1}, n \in \mathbb{N}^*$. Số hạng đầu tiên và công bội của cấp số nhân đó là
	\loigiai{
		Ta có $u_{1}=\dfrac{3}{5} \cdot 2^{1-1}=\dfrac{3}{5}$ và $u_{2}=\dfrac{3}{5} \cdot 2^{2-1}=\dfrac{6}{5} \Rightarrow q=\dfrac{u_{2}}{u_{1}}=2$.\\
		Vậy $u_{1}=\dfrac{3}{5}$ và $q=2$.
	}
\end{vd}

\subsubsection{Bài tập tự luận}
 
\begin{bt}%[TH]%[1K2B3-3]
	Cho cấp số nhân $(u_n)$ biết $\heva{&u_4-u_2=25 \\ &u_3-u_1=50.}$
	\begin{enumEX}{1}
		\item Tìm số hạng đầu và công bội của cấp số nhân $(u_n)$.
		\item Tìm số hạng thứ $8$ của cấp số nhân $(u_n)$.
	\end{enumEX}
	\dapso{$\heva{&q=\dfrac{1}{2} \\ &u_1=-200}$, $u_8=-\dfrac{25}{16}$}
	\loigiai{
		\begin{enumerate}
			\item Ta có $\heva{&u_4-u_2=25 \\ &u_3-u_1=50} 
			\Leftrightarrow \heva{&u_1(q^3-q)=25 \\ &u_1(q^2-q)=50} 
			\Rightarrow \heva{&q=\dfrac{1}{2} \\ &u_1=-200.}$
			\item Ta có $u_8=u_1\cdot q^7=-200\cdot \dfrac{1}{2^7}=-\dfrac{25}{16}$.
		\end{enumerate}
	}
\end{bt}
\begin{bt}%[TH]%[1K2B3-3]
	Tìm số hạng thứ $10$ của cấp số nhân $(u_n)$ biết $\heva{&u_4-u_2=72 \\ &u_5-u_3=144.}$	
	\dapso{$u_{10}=6144$}
	\loigiai{
		Ta có $\heva{&u_4-u_2=72 \\ &u_5-u_3=144}\Leftrightarrow \heva{&u_4-u_2=72 \\ &q(u_4-u_2)=144}
		\Rightarrow \heva{&q=2 \\ &u_1(q^3-q)=72} \Leftrightarrow \heva{&q=2 \\ &u_1=12.}$ \\
		Khi đó $u_{10}=u_1\cdot q^9=6144$.
	}
\end{bt}
\begin{bt}%[TH]%[1K2B3-3]
	Cho một cấp số nhân có $5$ số hạng biết $2$ số hạng đầu là số dương, tích số hạng đầu và số hạng thứ $3$ là $1$, tích số hạng thứ $3$ và số hạng cuối là $\dfrac{1}{16}$. Tìm cấp số nhân này.	
	\dapso{$2; 1; \dfrac{1}{2}; \dfrac{1}{4}; \dfrac{1}{8}$}
	\loigiai{
		Gọi $5$ số hạng cần tìm có dạng $\dfrac{x}{q^2}$; $\dfrac{x}{q}$; $x$; $xq$; $xq^2$.\\
		Theo đề ra ta có $\heva{&\dfrac{x}{q^2}\cdot x=1 \\ &x\cdot xq^2=\dfrac{1}{16}} 
		\Leftrightarrow \heva{&x=\dfrac{1}{2} \\ &q=\dfrac{1}{2}}$ (do hai số hạng đầu dương nên $q>0$). \\
		Vậy $5$ số hạng cần tìm là $2; 1; \dfrac{1}{2}; \dfrac{1}{4}; \dfrac{1}{8}$.
	}
\end{bt}
\begin{bt}%[TH]%[1K2B3-3]
	Tìm số hạng đầu và công bội của cấp số nhân $(u_n)$ biết $\heva{&u_2+u_5-u_4=10 \\ &u_3+u_6-u_5=20.}$
	\dapso{$\heva{&q=2 \\ &u_1=1}$}
	\loigiai{
		Ta có $\heva{&u_2+u_5-u_4=10 \\ &u_3+u_6-u_5=20} 
		\Leftrightarrow \heva{&u_1(q+q^4-q^3)=10 \\ &u_1(q^2+q^5-q^4)=20}
		\Leftrightarrow \heva{&q=2 \\ &u_1=1.}$
	}
\end{bt}
\begin{bt}%[TH]%[1K2B3-4]
	Tìm $5$ số lập thành một cấp số nhân có công bội bằng $\dfrac{1}{4}$ số thứ nhất và tổng $2$ số đầu là $\dfrac{5}{4}$.
	\dapso{$1; \dfrac{1}{4}; \dfrac{1}{16}; \dfrac{1}{64}; \dfrac{1}{128}$ hoặc $-5; -\dfrac{5}{4}; -\dfrac{5}{16}; -\dfrac{5}{64}; -\dfrac{1}{128}$}
	\loigiai{
		Theo đề, ta có $\heva{&q=\dfrac{1}{4}u_1 \\ &u_1+u_2=\dfrac{5}{4}}
		\Leftrightarrow \heva{&q=\dfrac{1}{4}u_1 \\ &u_1+u_1\cdot q=\dfrac{5}{4}}
		\Leftrightarrow \heva{&q=\dfrac{1}{4}u_1 \\ &u_1^2+4u_1-5=0}
		\Leftrightarrow \heva{&q=\dfrac{1}{4} \\ &u_1=1}$ hoặc $\heva{&q=-\dfrac{5}{4} \\ &u_1=-5.}$\\
		Vậy có hai CSN là $1; \dfrac{1}{4}; \dfrac{1}{16}; \dfrac{1}{64}; \dfrac{1}{128}$ và $-5; -\dfrac{5}{4}; -\dfrac{5}{16}; -\dfrac{5}{64}; -\dfrac{1}{128}$.
	}
\end{bt}
\begin{bt}%[TH]%[1K2B3-4]
	Tìm $3$ số lập thành một cấp số nhân có tổng là $63$ và tích là $1728$.
	\dapso{$3; 12; 48$}
	\loigiai{
		Gọi ba số cần tìm là $\dfrac{x}{q}; x; xq$. Theo đề ra, ta có $x^3=1728\Rightarrow x=12$. \\
		Mặt khác $\dfrac{x}{q}+x+xq=63\Leftrightarrow 12q+12+\dfrac{12}{q}=63
		\Leftrightarrow 12q^2-51q+12=0 \Leftrightarrow \hoac{&q=4 \\ &q=\dfrac{1}{4}\cdot}$ \\
		Vậy CSN cần tìm là $3; 12; 48$.
	}
\end{bt}
\subsubsection{Câu hỏi trắc nghiệm}
\Opensolutionfile{ans}[ans/ans-1K2-3-dang3]
\begin{ex}%[1K2B3-3]
	Cho cấp số nhân $(u_n)$ có $u_{20}=8u_{17}$. Công bội của cấp số nhân là
	\choice
	{\True $q=2$}
	{$q=-2$}
	{$q=4$}
	{$q=-4$}
	\loigiai{
		Ta có $u_{20}=8u_{17}\Rightarrow u_1\cdot q^{19}=8\cdot u_1\cdot q^{16}\Rightarrow  q=2$.
	}
\end{ex}
\begin{ex}%[1K2B3-3]
	Cho cấp số nhân $\left(u_n\right)$ có $10$ số hạng với công bội $q\neq 0$ và $u_1\neq 0$. Đẳng thức nào sau đây là đúng?
	\choice
	{$u_7=u_4\cdot q^6$}
	{\True $u_7=u_4\cdot q^3$}
	{$u_7=u_4\cdot q^4$}
	{$u_7=u_4\cdot q^5$}
	\loigiai
	{Ta có $u_7=u_1\cdot q^6=\left(u_1\cdot q^3\right)\cdot q^3=u_4\cdot q^3$.
	}
\end{ex}
\begin{ex}%[1K2B3-3]
	Cho cấp số nhân $(u_n)$ có số hạng đầu $u_1=2$ và công bội $q=3$. Giá trị $u_{2019}$ bằng
	\choice
	{$3\cdot2^{2019}$}
	{$2\cdot3^{2019}$}
	{$3\cdot2^{2018}$}
	{\True $2\cdot3^{2018}$}
	\loigiai{
		Áp dụng công thức của số hạng tổng quát $u_n=u_1\cdot q^{n-1}=2\cdot 3^{2018}$.}
\end{ex}
\begin{ex}%[1K2B3-3]
	Cho cấp số nhân $(u_n)$ với công bội $q < 0$ và $u_2=4$, $u_4=9$. Tìm $u_1$.
	\choice
	{$u_1=6$}
	{\True $u_1=-\dfrac{8}{3}$}
	{$u_1=-6$}
	{$u_1=\dfrac{8}{3}$}
	\loigiai{
		Vì $q<0$, $u_2>0$ nên $u_3<0$. Do đó $u_3=-\sqrt{u_2\cdot u_4}=-\sqrt{4\cdot 9}=-6$.\\
		Ta có $u_2^2=u_1\cdot u_3\Rightarrow u_1=\dfrac{u_2^2}{u_3}=\dfrac{4^2}{-6}=-\dfrac{8}{3}$.
	}
\end{ex}
\begin{ex}%[1K2B3-3]
	Cho cấp số nhân $\left(u_{n}\right)$ có $u_{2}=-6, u_{3}=3$. Công bội $q$ của cấp số nhân đã cho bằng
	\choice
	{$2 $}
	{$\dfrac{1}{2}$}
	{\True $-\dfrac{1}{2}$}
	{$-2$}
	\loigiai{
		Công bội của cấp số nhân đã cho là $$q=\dfrac{u_3}{u_2}=-\dfrac{1}{2}.$$}
\end{ex}
\begin{ex}%[1K2Y3-3]
	Cho cấp số nhân có $u_1=-3$, $q=\dfrac{2}{3}$. Tính $u_5$?
	\choice
	{$u_5=\dfrac{27}{16}$}
	{\True $u_5=\dfrac{-16}{27}$}
	{$u_5=\dfrac{-27}{16}$}
	{$u_5=\dfrac{16}{27}$}
	\loigiai{
		Ta có: $u_5=u_1\cdot q^4=(-3)\left(\dfrac{2}{3}\right)^4=-\dfrac{16}{27}$.}
\end{ex}
\begin{ex}%[1K2Y3-3]
	Cho cấp số nhân $(u_n)$ có $u_2=\dfrac{1}{4}$; $u_5=-16$. Tìm $q$ và số hạng đầu tiên của cấp số nhân?
	\choice
	{$q=\dfrac{1}{2};u_1=\dfrac{1}{2}$}
	{$q=-\dfrac{1}{2},u_1=-\dfrac{1}{2}$}
	{\True $q=-4,u_1=\dfrac{1}{16}$}
	{$q=-4,u_1=-\dfrac{1}{16}$}
	\loigiai{
		Ta có $\heva{&u_2=\dfrac{1}{4}\\&u_5=16}\Rightarrow \heva{&u_1\cdot q=\dfrac{1}{4}\\&u_1\cdot q^4=-16}\Rightarrow q^3=-64\Rightarrow q=-4 \Rightarrow u_1=\dfrac{1}{16}$.
	}
\end{ex}
\begin{ex}%[1K2B3-3]
	Cho cấp số nhân $(u_n)$, biết: $u_n=81,u_{n+1}=9$. Lựa chọn đáp án đúng.
	\choice
	{$q=-\dfrac{1}{9}$}
	{\True $q=\dfrac{1}{9}$}
	{$q=9$}
	{$q=-9$}
	\loigiai{
		Ta có  $q=\dfrac{u_{n+1}}{u_n}=\dfrac{9}{81}=\dfrac{1}{9}$.
	}
\end{ex}
\begin{ex}%[1K2Y3-3]
	Cho cấp số nhân $\left( {u_n} \right)$ với $u_1=2$ và công bội $q=3$. Số hạng $u_2$ bằng
	\choice
	{$8$}
	{\True $6$}
	{$12$}
	{$18$}
	\loigiai{
		Ta có $u_2=u_1\cdot q=2\cdot 3=6$.}
\end{ex}
\begin{ex}%[1K2Y3-3]
	Cho cấp số nhân $(u_n)$ với $u_1=2$ và $u_3=8$. Số hạng thứ hai của cấp số nhân đã cho bằng
	\choice
	{$u_2=4$}
	{$u_2=6$}
	{\True $u_2=\pm 4$}
	{$u_2=-4$}
	\loigiai{
		Ta có $u_1 \cdot u_3=u_2^2 \Leftrightarrow u^2_2=16 \Leftrightarrow \hoac{&u_2=4\\&u_2=-4.}$
	}
\end{ex}
\begin{ex}%[1K2B3-3]
	Cho cấp số nhân $(u_n)$ có $u_1=-1; q=\dfrac{-1}{10}$. Số $\dfrac{1}{10^{103}}$ là số hạng thứ bao nhiêu?
	\choice
	{số hạng thứ $103$}
	{số hạng thứ $105$}
	{\True số hạng thứ $104$}
	{Đáp án khác}
	\loigiai{
		Ta có $u_n=u_1\cdot q^{n-1}\Leftrightarrow	\dfrac{1}{10^{103}}=-1\cdot \left(\dfrac{-1}{10}\right)^{n-1} \Leftrightarrow \left(\dfrac{-1}{10}\right)^{n-1}=\left(\dfrac{-1}{10}\right)^{103}\Rightarrow n=104$.
	}
\end{ex}
\begin{ex}%[1K2Y3-3]
	Cho cấp số nhân $\left(u_n\right)$ có các số hạng lần lượt là $3$, $9$, $27$, $81$,\ldots Khi đó $u_n$ bằng
	\choice
	{$3+3^n$}
	{$3^{n-1}$}
	{$3^{n+1}$}
	{\True $3^n$}
	\loigiai
	{Cấp số nhân đã cho có $u_1=3$ và công bội $q=3$ nên $u_n=u_1\cdot q^{n-1}=3\cdot 3^{n-1}=3^n$.
	}
\end{ex}
\begin{ex}%[1K2K3-3]
	Cho cấp số nhân $(u_n)$ có $u_1=3$ và $15u_1-4u_2+u_3$ đạt giá trị nhỏ nhất. Tìm số hạng thứ $13$ của cấp số nhân đã cho.
	\choice
	{\True $u_{13}=12288$}
	{$u_{13}=3072$}
	{$u_{13}=24567$}
	{$u_{13}=49152$}
	\loigiai{
		Gọi $q$ là công bội của cấp số nhân $(u_n)$.\\
		Ta có $15u_1-4u_2+u_3=45-12q+3q^2=3(q-2)^2+33\geq 33$ $\forall q \in \mathbb{R}$.\\
		Suy ra $15u_1-4u_2+u_3$ đạt giá trị nhỏ nhất khi $q=2$.\\
		Khi đó $u_{13}=u_1q^{12}=12288$.}
\end{ex}
\begin{ex}%[1K2B3-3]
	Cho cấp số nhân $(u_n)$ biết $u_1+u_5=51$ và $u_2+u_6=102$. Hỏi số $12288$ là số hạng thứ mấy của cấp số nhân $(u_n)$?
	\choice
	{\True Số hạng thứ $13$}
	{Số hạng thứ $10$}
	{Số hạng thứ $11$}
	{Số hạng thứ $12$}
	\loigiai{
		Gọi $q$ là công bội của cấp số nhân đã cho. Theo đề bài, ta có\\
		\centerline{$\heva{&u_1+u_5=51\\&u_2+u_6=102}\Leftrightarrow\heva{&u_1\left(1+q^4\right)=51\\&u_1q\left(1+q^4\right)=102}\Rightarrow q=2\Rightarrow u_1=3\Rightarrow u_n=3\cdot 2^{n-1.}$}
		Mặt khác $u_n=12288\Leftrightarrow 3\cdot 2^{n-1}=12288\Leftrightarrow 2^{n-1}=2^{12}\Leftrightarrow n=13$.
	}
\end{ex}
% \begin{ex}%[1K2K3-3]
% 	Một tứ giác lồi có số đo các góc lập thành một cấp số nhân. Biết rằng số đo của góc nhỏ nhất bằng $\dfrac{1}{9}$ số đo của góc nhỏ thứ ba. Hãy tính số đo của các góc trong tứ giác đó.
% 	\choice
% 	{$5^{\circ}$, $15^{\circ}$, $45^{\circ}$, $225^{\circ}$}
% 	{\True $9^{\circ}$, $27^{\circ}$, $81^{\circ}$, $243^{\circ}$}
% 	{$7^{\circ}$, $21^{\circ}$, $63^{\circ}$, $269^{\circ}$}
% 	{$8^{\circ}$, $32^{\circ}$, $72^{\circ}$, $248^{\circ}$}
% 	\loigiai{
% 		Gọi các góc của tứ giác là $a$, $aq$, $aq^2$, $aq^3,$ trong đó $q>1$.\\
% 		Theo giả thiết, ta có $a=\dfrac{1}{9}aq^2$ nên $q=3$.\\
% 		Suy ra các góc của tứ giác là $a$, $3a$, $9a$, $27a$.\\
% 		Vì tổng các góc trong tứ giác bằng $360^{\circ}$ nên ta có $a+3a+9a+27a=360^{\circ}\Leftrightarrow a=9^{\circ}$.\\
% 		Vậy số đo các góc trong tứ giác lần lượt là $9^{\circ}$, $27^{\circ}$, $81^{\circ}$, $243^{\circ}$. }
% \end{ex}
\Closesolutionfile{ans}
% \begin{indapan}{10}
% 	{ans/ans-1K2-3-dang3}
% \end{indapan}
\begin{dang}{Tìm điều kiện để một dãy số lập thành CSN}
	Dãy số $a, b, c$ lập thành CSN khi $b^2=a\cdot c$. \\
	Dãy số $a, b, c, d$ lập thành CSN khi $\heva{&b^2=a\cdot c \\ &c^2=b\cdot d.}$
\end{dang}
\subsubsection{Ví dụ minh hoạ}
\begin{vd}%[NB]%[1K2B3-4]
	Cho dãy $3,x,12,y$. Tìm $x,y$ để dãy là CSN.
	\loigiai{
		Dãy là CSN khi $\heva{&x^2=3\cdot 12 \\ &12^2=x\cdot y}\Leftrightarrow 
		\heva{&x=6 \\ &y=24}$ hoặc $\heva{&x=-6 \\ &y=-24.}$
	}
\end{vd}
\begin{vd}%[TH]%[1K2B3-4]
	Cho dãy  $x-1, 2x, 4x+3$. Tìm $x$ để dãy là CSN. 
	\loigiai{
		Dãy là CSN khi $(2x)^2=(x-1)(4x+3) \Leftrightarrow x=-3$.
	} 
\end{vd}
\begin{vd}%[VD]%[1K2K3-4]
	Các số $x+6y$, $5x+2y$, $8x+y$ theo thứ tự đó lập thành một cấp số cộng, đồng thời, các số $x+\dfrac{5}{3}$, $y-1$, $2x-3y$ theo thứ tự đó lập thành một cấp số nhân. Hãy tìm $x$ và $y$.
	\loigiai{
		\begin{itemize}
			\item Ba số $x+6y$, $5x+2y$, $8x+y$ lập thành cấp số cộng nên $(x+6y)+(8x+y)=2(5x+2y)\Leftrightarrow x=3y$.
			\item Ba số $x+\dfrac{5}{3}$, $y-1$, $2x-3y$ lập thành cấp số nhân nên $\left(x+\dfrac{5}{3}\right)(2x-3y)=\left(y-1\right)^2$.
		\end{itemize}
		Thay $x=3y$ vào ta được $8y^2+7y-1=0\Leftrightarrow y=-1$ hoặc $y=\dfrac{1}{8}$.\\
		Với $y=-1$ thì $x=-3$; với $y=\dfrac{1}{8}$ thì $x=\dfrac{3}{8}$.}
\end{vd}
\begin{vd}%[VD]%[1K2K3-4]
	Tìm tất cả các giá trị của tham số $m$ để phương trình sau có ba nghiệm phân biệt lập thành một cấp số nhân $x^3-7x^2+2\left(m^2+6m\right)x-8=0$.
	\loigiai{
		+ \textbf{Điều kiện cần:} \\
		Giả sử phương trình đã cho có ba nghiệm phân biệt $x_1$,$x_2$,$x_3$ lập thành một cấp số nhân.\\
		Theo định lý Vi-ét, ta có $x_1x_2x_3=8$.\\
		Theo tính chất của cấp số nhân, ta có $x_1x_3=x_2^2$. Suy ra  $x_2^3=8\Leftrightarrow x_2=2$.\\
		Với nghiệm $x=2$, ta có $m^2+6m-7=0\Leftrightarrow\hoac{&m=1\\&m=-7.}$ \\
		+ \textbf{Điều kiện đủ:}\\
		Với $m=1$ hoặc $m=-7$ thì $m^2+6m=7$.\\
		Khi đó phương trình ban đầu trở thành $x^3-7x^2+14x-8=0$.\\
		Giải phương trình này, ta được các nghiệm là $1$,$2$,$4$. Hiển nhiên ba nghiệm này lập thành một cấp số nhân với công bội $q=2$.\\
		Vậy $m=1$ và $m=-7$ là các giá trị cần tìm.}
\end{vd}
\begin{vd}%[VD]%[1K2K3-4]
	Các số $x+6y$, $5x+2y$, $8x+y$ theo thứ tự đó lập thành một cấp số cộng; đồng thời các số $x-1$, $y+2$, $x-3y$ theo thứ tự đó lập thành một cấp số nhân. Tính $x^2+y^2$.
	\loigiai{
		Theo giả thiết ta có
		\[
		\heva{&(x+6y)+(8x+y)=2(5x+2y)\\&(x-1)(x-3y)=\left(y+2\right)^2}\Leftrightarrow\heva{&x=3y\\&(3y-1)(3y-3y)=\left(y+2\right)^2}\Leftrightarrow \heva{&x=3y\\&0=\left(y+2\right)^2}\Leftrightarrow \heva{&x=-6\\&y=-2.}
		\]
		Vậy $x^2+y^2=40$.}
\end{vd}
\subsubsection{Bài tập tự luận}
 
\begin{bt}%[TH]%[1K2B3-4]
	Xác định $x$ dương để $2x-3$; $x$; $2x+3$ lập thành cấp số nhân.
	\dapso{$x=\sqrt{3}$}
	\loigiai{
		Ba số $2x-3$; $x$; $2x+3$ lập thành cấp số nhân khi $x^2=(2x-3)(2x+3) \Leftrightarrow x=\pm \sqrt{3}$.\\
		Do $x>0$ nên chọn $x=\sqrt{3}$.
	}
\end{bt}
\begin{bt}%[TH]%[1K2B3-4]
	Cho cấp số nhân $x, 12, y, 192$. Tìm $x$ và $y$.
	\dapso{$\heva{&x=-3 \\ &y=-48}$}	
	\loigiai{
		Bốn số $x, 12, y, 192$ lập thành CSN khi $\heva{&xy=12^2 \\ &y^2=12\cdot 192}
		\Leftrightarrow \heva{&x=3 \\ &y=48}$ hoặc $\heva{&x=-3 \\ &y=-48.}$
	}
\end{bt}
\begin{bt}%[TH]%[1K2B3-4]
	Tìm $x$ để dãy số $1$, $x^2$, $6-x^2$ lập thành cấp số nhân.
	\dapso{$x=\pm \sqrt{2}$}
	\loigiai{
		Ta có $1, x^2, 6-x^2$ lập thành cấp số nhân $\Leftrightarrow x^4=6-x^2 \Leftrightarrow x= \pm \sqrt{2}$.
	}
\end{bt}
\begin{bt}%[TH]%[1K2B3-4]
	Viết $6$ số xen giữa hai số $-2$ và $256$ để được một cấp số nhân có $8$ số hạng. Tìm cấp số nhân này.
	\dapso{$-2; 4; -8; 16; -32; 64; -128; 256$}
	\loigiai{
		Theo đề ra, ta có $\heva{&u_1=-2 \\ &u_8=256} \Leftrightarrow \heva{&u_1=-2 \\ &u_1\cdot q^7=256}
		\Leftrightarrow \heva{&u_1=-2 \\ &q=-2.}$ \\
		Cấp số nhân cần tìm là $-2; 4; -8; 16; -32; 64; -128; 256$.
	}
\end{bt}
\begin{bt}%[VD]%%[1K2K3-4]
	Bốn góc của một tứ giác lồi lập thành một cấp số nhân, góc lớn nhất gấp $8$ lần góc nhỏ nhất. Tìm $4$ góc đó.
	\dapso{$24^\circ; 48^\circ; 96^\circ; 192^\circ$}	
	\loigiai{
		Giả sử $4$ góc của tứ giác là $A\leq B\leq C\leq D$. Suy ra $A+B+C+D=360^\circ$.\\
		Theo đề, ta có $D=8A \Leftrightarrow Aq^3=8A \Leftrightarrow q=2$. Khi đó, ta được
		$$A(1+q+q^2+q^3)=360^\circ \Rightarrow A=24^\circ.$$
		Vậy $4$ góc của tứ giác lần lượt là $24^\circ; 48^\circ; 96^\circ; 192^\circ$.
	}
\end{bt}
\begin{bt}%[VD]%%[1K2K3-4]
	Tìm tất cả các giá trị của tham số $m$ để phương trình sau có ba nghiệm phân biệt lập thành một cấp số nhân $x^3-7mx^2+2(m^2+6 m)x-64=0$.
	\dapso{$m-8$}
	\loigiai{
		+ Điều kiện cần: \\
		Giả sử phương trình đã cho có ba nghiệm phân biệt $x_1; x_2; x_3$ lập thành một cấp số nhân. \\
		Theo định lý Vi-ét, ta có $x_1 \cdot x_2 \cdot x_3=64$. \\
		Theo tính chất của cấp số nhân, ta có $x_1\cdot x_3=x_2^2$. Suy ra ta có $x_2^3=64 \Leftrightarrow x_2=4$. \\
		Thay $x=4$ vào phương trình đã cho ta được 
		$$4^3-7m\cdot 4^2+2(m^2+6m) \cdot 4-64=0
		\Leftrightarrow m^2-8m=0
		\Leftrightarrow \hoac{&m=0 \\ &m=8.}$$
		+ Điều kiện đủ: \\
		Với $m=0$ thay vào phương trình đã cho ta được: $x^3-64=0$ hay $x=4$
		(nghiệm kép-loại). \\
		Với $m=8$ thay vào phương trình đã cho nên ta có phương trình $x^3-56x^2+224x-64=0$. \\
		Phương trình này có $3$ nghiệm phân biệt lập thành cấp số nhân. \\
		Vậy $m=8$ là giá trị cần tìm.
	}
\end{bt}
\subsubsection{Câu hỏi trắc nghiệm}
\Opensolutionfile{ans}[ans/ans-1K2-3-dang4]
% \begin{ex}%[1K2K3-4]
% 	Bốn góc của một tứ giác tạo thành cấp số nhân và góc lớn nhất gấp $27$ lần góc nhỏ nhất. Tổng của góc lớn nhất và góc bé nhất bằng
% 	\choice
% 	{$56^{\circ}$}
% 	{$102^{\circ}$}
% 	{$168^{\circ}$}
% 	{\True $252^{\circ}$}
% 	\loigiai{
% 		Giả sử 4 góc $A$, $B$, $C$, $D$ (với $A<B<C<D$) theo thứ tự đó lập thành cấp số nhân thỏa yêu cầu với công bội $q$.\\
% 		Theo giả thiết ta có
% 		\[\heva{&A+B+C+D=360\\&D=27A}\Leftrightarrow\heva{&A\left(1+q+q^2+q^3\right)=360\\&Aq^3=27A}\Leftrightarrow\heva{&q=3\\&A=9.}\]
% 		Suy ra $D=A\cdot q^3=9\cdot 3^3=243$.\\
% 		Vậy tổng số đo góc lớn nhất và góc bé nhất là $A+D=252^\circ$.
% 	}
% \end{ex}
\begin{ex}%[1K2B3-4]
	Xác định $x$ để $3$ số $2x-1$; $x$; $2x+1$ theo thứ tự lập thành một cấp số nhân:
	\choice
	{$x=\pm\sqrt{3}$}
	{$x=\pm\dfrac{1}{3}$}
	{\True $x=\pm\dfrac{1}{\sqrt{3}}$}
	{Không có giá trị nào của $x$}
	\loigiai{
		Ba số $2x-1$; $x$; $2x+1$ theo thứ tự lập thành cấp số nhân\\
		$\Leftrightarrow (2x-1)(2x+1)=x^2 \Leftrightarrow 3x^2=1 \Leftrightarrow x=\pm\dfrac{1}{\sqrt{3}}$.
	}
\end{ex}
\begin{ex}%[1K2K3-4]
	Cho $4$ số nguyên dương, trong đó $3$ số đầu lập thành cấp số cộng, $3$ số cuối lập thành cấp số nhân. Biết tổng số đầu và cuối là $37$, tổng $2$ số hạng giữa là $36$. Hỏi số lớn nhất thuộc khoảng nào sau đây?
	\choice
	{$\left(26;29\right)$}
	{\True $\left(24;26\right)$}
	{$\left(30;33\right)$}
	{$\left(22;25\right)$}
	\loigiai{
		Giả sử $4$ số đó là $a$, $b$, $c$, $d$ $\left(a,b,c,d\in\mathbb{N}^{\ast}\right)$. \\
		Do $a$, $b$, $c$ lập thành cấp số cộng nên ta có $a+c=2b$ $(1)$.\\
		Do $b$, $c$, $d$ lập thành cấp số nhân nên ta có $b \cdot d=c^2$ $(\ast)$. \\
		Theo giả thiết ta có $\heva{ & a+d=37 & & (2) \\ & b+c=36. & & (3)}$ \\
		Từ $(1)$, $(2)$, $(3)$ ta có $\heva{ & a=-d+37 \\ & b=\dfrac{-d+73}{3} \\ & c=\dfrac{d+35}{3}.}$ \\
		Thay vào $(\ast)$ ta có $\dfrac{-d+73}{3}\cdot d=\left(\dfrac{d+35}{3}\right)^2 \Leftrightarrow 4d^2-149d+1225=0 \Leftrightarrow \hoac{ & d=25 \\ & d=\dfrac{49}{4} & & \text{(loại)}.}$\\
		Với $d=25$, ta có $a=12$, $b=16$, $c=20$. \\
		Vậy số lớn nhất là $25\in\left(24;26\right)$.
	}
\end{ex}
\begin{ex}%[1K2K3-4]
	Ba số $x$, $y$, $z$ theo thứ tự lập thành một cấp số nhân với công bội $q$ khác $1$ đồng thời các số $x$, $2y$, $3z$ theo thứ tự lập thành một cấp số cộng với công sai khác $0$. Tìm giá trị của $q$.
	\choice
	{$q=-\dfrac{1}{3}$}
	{$q=\dfrac{1}{9}$}
	{$q=-3$}
	{\True $q=\dfrac{1}{3}$}
	\loigiai{
		Theo giả thiết ta có
		\[\heva{&y=xq \\ & z=xq^2\\&x+3z=2(2y)}\Rightarrow x+3xq^2=4xq \Rightarrow x\left(3q^2-4q+1\right)=0\Leftrightarrow \hoac{&x=0\\&3q^2-4q+1=0.}\]
		Nếu $x=0\Rightarrow y=z=0\Rightarrow$ công sai của cấp số cộng $x$, $2y$, $3z$ bằng 0 (vô lí).\\
		Nếu $3q^2-4q+1=0\Leftrightarrow\hoac{&q=1\\&q=\dfrac{1}{3}}\Leftrightarrow q=\dfrac{1}{3}$ vì $(q\not=1)$.}
\end{ex}
\begin{ex}%[1D3Y4-4]
	Trong các dãy số $(u_n)$ cho bởi số hạng tổng quát $u_n$ sau, dãy số nào là một cấp số nhân?
	\choice
	{$u_n=\dfrac{1}{3^n}-1$}
	{$u_n=n+\dfrac{1}{3}$}
	{$u_n=n^2-\dfrac{1}{3}$}
	{\True $u_n=\dfrac{1}{3^{n-2}}$}
	\loigiai{
		Từ các đáp án trên, với dãy $(u_n)$ cho bởi $u_n=\dfrac{1}{3^{n-2}}$ là một cấp số nhân, vì
		$$T=\dfrac{u_{n+1}}{u_n}=\dfrac{3^{n-2}}{3^{n-1}}=\dfrac{1}{3} \text{ (không đổi).}$$
	}
\end{ex}
\begin{ex}%[1K2B3-4]
	Trong các mệnh đề dưới đây, mệnh đề nào là \textbf{sai}?
	\choice
	{Dãy số $\left(a_n\right)$, với $a_1=3$ và $a_{n+1}=\sqrt{a_n+6}$, $\forall n\geq 1,$ vừa là cấp số cộng vừa là cấp số nhân}
	{\True Dãy số $\left(d_n\right)$, với $d_1=-3$ và $d_{n+1}=2d_n^2-15$, $\forall n\geq 1,$ vừa là cấp số cộng vừa là cấp số nhân}
	{Dãy số $\left(b_n\right)$, với $b_1=1$ và $b_{n+1}\left(2b_n^2+1\right)=3$, $\forall n\geq 1,$ vừa là cấp số cộng vừa là cấp số nhân}
	{Dãy số $\left(c_n\right)$, với $c_1=2$ và $c_{n+1}=3c_n^2-10$, $\forall n\geq 1,$ vừa là cấp số cộng vừa là cấp số nhân}
	\loigiai{
		Kiểm tra từng phương án ta có
		\begin{itemize}
			\item Ta có $a_2=3$, $a_2=3$, \ldots Bằng phương pháp quy nạp toán học chúng ra chứng minh được rằng $a_n=3$, $\forall n\geq 1$. Do đó $\left(a_n\right)$ là dãy số không đổi. Suy ra nó vừa là cấp số cộng (công sai bằng $0$) vừa là cấp số nhân (công bội bằng $1$).
			\item Tương tự như phương án trên, chúng ta chỉ ra được $b_n=1$, $\forall n\geq 1$. Do đó $\left(b_n\right)$ là dãy số không đổi. Suy ra nó vừa là cấp số cộng (công sai bằng $0$) vừa là cấp số nhân (công bội bằng $1$).
			\item Tương tự như phương án trên, chúng ta chỉ ra được $c_n=2$, $\forall n\geq 1$. Do đó $\left(c_n\right)$ là dãy số không đổi. Suy ra nó vừa là cấp số cộng (công sai bằng $0$) vừa là cấp số nhân (công bội bằng $1$).
			\item Ta có $d_1=-3$, $d_2=3$, $d_3=3$. Ba số hạng này không lập thành cấp số cộng cũng không lập thành cấp số nhân nên dãy số $\left(d_n\right)$ không phải là cấp số cộng và cũng không là cấp số nhân.
	\end{itemize}}
\end{ex}
\begin{ex}%[1K2K3-4]
	Biết rằng tồn tại hai giá trị $m_1$ và $m_2$ để phương trình \[2x^3+2\left(m^2+2m-1\right)x^2-7\left(m^2+2m-2\right)x-54=0\] có ba nghiệm phân biệt lập thành một cấp số nhân. Tính giá trị của biểu thức $P=m_1^3+m_2^3$.
	\choice
	{$P=56$}
	{$P=8$}
	{$P=-8$}
	{\True $P=-56$}
	\loigiai{
		Theo định lý Vi-ét, ta có $x_1\cdot x_2\cdot x_3 = -\dfrac{d}{a}=-\dfrac{-54}{2}=27 \Leftrightarrow x_2^3=27 \Leftrightarrow x_2=3$.\\
		Điều kiện cần để phương trình đã cho có ba nghiệm phân biệt lập thành một cấp số nhân là $x=3$ phải là nghiệm của phương trình đã cho. Suy ra
		\[m^2+2m-8=0\Leftrightarrow \hoac{&m=2 \\&m=-4.}\]
		Vì giả thiết cho biết tồn tại đúng hai giá trị của tham số $m$ nên $m=2$ và $m=-4$ là các giá trị thỏa mãn.\\
		Vậy $P=2^3+\left(-4\right)^3=-56$.
	}
\end{ex}
\begin{ex}%[1K2K3-4]
	Cho bốn số $a$, $b$, $c$, $d$ biết rằng $a$, $b$, $c$ theo thứ tự đó lập thành một cấp số nhân với công bội $q>1$; còn $b$, $c$, $d$ theo thứ tự đó lập thành cấp số cộng. Tìm $q$, biết rằng $a+d=14$ và $b+c=12$.
	\choice
	{$q=\dfrac{20+\sqrt{73}}{24}$}
	{\True $q=\dfrac{19+\sqrt{73}}{24}$}
	{$q=\dfrac{21+\sqrt{73}}{24}$}
	{$q=\dfrac{18+\sqrt{73}}{24}$}
	\loigiai{
		Giả sử $a$, $b$, $c$ lập thành cấp số cộng công bội $q$. Khi đó theo giả thiết ta có
		\[\heva{&b=aq,\;c=aq^2\\&b+d=2c\\&a+d=14 \\&b+c=12}\Rightarrow\heva{&aq+d=2aq^2&\quad(1)\\&a+d=14&(2)\\& a\left(q+q^2\right)=12.&(3)}\]
		\begin{itemize}
			\item Nếu $q=0\Rightarrow b=c=d=0$. (Vô lí!)
			\item Nếu $q=-1\Rightarrow b=-a$; $c=a\Rightarrow b+c=0$. (Vô lí!)
		\end{itemize}
		Vậy $q\not=0$, $q\not=-1$, từ $(2)$ và $(3)$ ta có $d=14-a$ và $a=\dfrac{12}{q+q^2}$. Thay vào $(1)$ ta được
		\[\begin{aligned}\dfrac{12q}{q+q^2}+\dfrac{14q^2+14q-12}{q+q^2}=\dfrac{24q^3}{q+q^2}
			&\Leftrightarrow 12q^3-7q^2-13q+6=0\\
			&\Leftrightarrow(q+1)\left(12q^2-19q+6\right)=0\\
			&\Leftrightarrow \hoac{ & q=-1 \quad \text{(loại)} \\ & q=\dfrac{19+\sqrt{73}}{24} \\ & q=\dfrac{19-\sqrt{73}}{24}.}
		\end{aligned}\]
		Vì $q>1$ nên $q=\dfrac{19+\sqrt{73}}{24}$.
	}
\end{ex}
\begin{ex}%[1K2K3-4]
	Cho dãy số tăng $a$, $b$, $c$ $\left(c\in\mathbb{Z}\right)$ theo thứ tự lập thành cấp số nhân; đồng thời $a$, $b+8$, $c$ theo thứ tự lập thành cấp số cộng và $a$, $b+8$, $c+64$ theo thứ tự lập thành cấp số nhân. Tính giá trị biểu thức $P=a-b+2c$.
	\choice
	{$P=32$}
	{$P=\dfrac{92}{9}$}
	{\True $P=64$}
	{$P=\dfrac{184}{9}$}
	\loigiai{
		Theo giả thiết, ta có hệ phương trình \[\heva{&ac=b^2\\&a+c=2(b+8)\\&a(c+64)=(b+8)^2}\Leftrightarrow\heva{&ac=b^2&\quad(1)\\&a-2b=16-c&(2)\\&ac+64a=\left(b+8\right)^2.&\quad(3)}\]
		Thay $(1)$ vào $(3)$ ta được $b^2+64a=b^2+16b+64\Leftrightarrow 4a-b=4.\qquad(4)$\\
		Kết hợp $(2)$ với $(4)$ ta được \[\heva{&a-2b=16-c\\&4a-b=4}\Leftrightarrow\heva{&a=\dfrac{c-8}{7}\\&b=\dfrac{4c-60}{7}.} \qquad (5)\]
		Thay $(5)$ vào $(1)$ ta được
		\[7(c-8)c=(4c-60)^2\Leftrightarrow 9c^2-424c+3600=0\Leftrightarrow\hoac{&c=36\\&c=\dfrac{100}{9}}\Leftrightarrow c=36.\qquad\left(\text{Vì}\;c\in\mathbb{Z}\right)\]
		Với $c=36$ $\Rightarrow a=4$, $b=12\Rightarrow P=4-12+72=64$.}
\end{ex}
\begin{ex}%[1K2K3-4]
	Cho $3$ số $a$, $b$, $c$ theo thứ tự lập thành cấp số nhân với công bội khác $1$ . Biết cũng theo thứ tự đó chúng lần lượt là số thứ nhất, thứ tư và thứ tám của một cấp số cộng công sai là $d$, $(d \neq 0)$. Tính $\dfrac{a}{d}$.
	\choice
	{$\dfrac{4}{3}$}
	{\True $9$}
	{$\dfrac{4}{9}$}
	{$3$}
	\loigiai{
		Do $a, b, c$ theo thứ tự lần lượt là số thứ nhất, thứ tư và thứ tám của một cấp số cộng công sai là $d,(d \neq 0)$ nên $\left\{\begin{array}{l}b=a+3 d \\ c=a+7 d\end{array}\right.$.\\
		Hơn nữa $a, b, c$ theo thứ tự lập thành cấp số nhân với công bội khác 1 nên $a c=b^{2}$.\\
		Khi đó \begin{eqnarray*}
			a(a+7 d)=(a+3 d)^{2} &\Leftrightarrow& a^{2}+7 a d=a^{2}+6 a d+9 d^{2}\\
			&\Leftrightarrow& 9 d^{2}-a d=0 \Leftrightarrow 9 d=a \Leftrightarrow \dfrac{a}{d}=9.
		\end{eqnarray*}
		Vậy $\dfrac{a}{d}=9$.
	}
\end{ex}
\begin{ex}%[1K2B3-4]
	Cho dãy số $(u_n)$ là một cấp số nhân với $u_n\neq 0$, $n\in\mathbb{N}^*$. Dãy số nào sau đây không phải là cấp số nhân?
	\choice
	{\True $u_1+2$; $u_2+2$; $u_3+2$; $\ldots$}
	{$3u_1$; $3u_2$; $3u_3$; $\ldots$}
	{$\dfrac{1}{u_1}$; $\dfrac{1}{u_2}$; $\dfrac{1}{u_3}$; $\ldots$}
	{$u_1$; $u_3$; $u_5$; $\ldots$}
	\loigiai{
		Giả sử $(u_n)$ là một cấp số nhân với công bội $q$.\\
		Ta có $u_2=u_1 q$, $u_3=u_1 q^2$.\\
		Dễ thấy $\dfrac{u_2+2}{u_1+2}=\dfrac{u_1 q+2}{u_1+2}$ và $\dfrac{u_3+2}{u_2+2}=\dfrac{u_1 q^2+2}{u_1 q+2}$.\\
		Do $\dfrac{u_2+2}{u_1+2}\neq\dfrac{u_3+2}{u_2+2}$ $\Rightarrow$ dãy số $u_1+2$; $u_2+2$; $u_3+2$; $\ldots$ không phải là cấp số nhân.
	}
\end{ex}
\begin{ex}%[1K2B3-4]
	Xác định $x$ để $3$ số $x-2$; $x+1$; $3-x$ theo thứ tự lập thành một cấp số nhân
	\choice
	{$x=\pm 1$}
	{\True Không có giá trị nào của $x$}
	{$x=-3$}
	{$x=2$}
	\loigiai{
		Ba số $x-2$; $x+1$; $3-x$ theo thứ tự lập thành một cấp số nhân\\
		$\Leftrightarrow(x-2)(3-x)=(x+1)^2 \Leftrightarrow 2x^2-3x+7=0$ (Phương trình vô nghiệm).
	}
\end{ex}
\begin{ex}%[1D3Y4-4]
	Trong các dãy số $(u_n)$ cho bởi số hạng tổng quát $u_n$ sau, dãy số nào là một cấp số nhân?
	\choice
	{\True $u_n=7\cdot 3^n$}
	{$u_n=\dfrac{7}{3n}$}
	{$u_n=7-3^n$}
	{$u_n=7-3n$}
	\loigiai{
		Từ các đáp án trên, với dãy $(u_n)$ cho bởi $u_n=7\cdot 3^n$ là một cấp số nhân, vì
		$$T=\dfrac{u_{n+1}}{u_n}=\dfrac{7\cdot 3^{n+1}}{7\cdot 3^n}=3 \text{ (không đổi).}$$
	}
\end{ex}
\begin{ex}%[1K2K3-4]
	Số hạng thứ hai, số hạng đầu và số hạng thứ ba của một cấp số cộng với công sai khác $0$ theo thứ tự đó lập thành một cấp số nhân với công bội $q$. Tìm $q$.
	\choice
	{\True $q=-2$}
	{$q=-\dfrac{3}{2}$}
	{$q=\dfrac{3}{2}$}
	{$q=2$}
	\loigiai{
		Giả sử ba số hạng $a$; $b$; $c$ lập thành cấp số cộng thỏa mãn yêu cầu, khi đó $b$; $a$; $c$ theo thứ tự đó lập thành cấp số nhân công bội $q\neq 1$. Ta có\\
		\[\heva{&a+c=2b\\&a=bq; c=bq^2}\Rightarrow bq+bq^2=2b\Leftrightarrow\hoac{&b=0\\&q^2+q-2=0.}\]
		\begin{itemize}
			\item Nếu $b=0\Rightarrow a=b=c=0$ nên $a$; $b$; $c$ là cấp số cộng công sai $d=0$. (Vô lí!)
			\item Nếu $q^2+q-2=0$ thì $ q=1$ hoặc $q=-2$. Dễ thấy trường hợp $q = 1$ là không thỏa mãn, vì khi đó $a = b = c$. Do đó $q = -2$.
		\end{itemize}
	}
\end{ex}
\begin{ex}%[1K2K3-4]
	Ba số $x$, $y$, $z$ lập thành một cấp số cộng và có tổng bằng $21$. Nếu lần lượt thêm các số $2$, $3$, $9$ vào ba số đó (theo thứ tự của cấp số cộng) thì được ba số lập thành một cấp số nhân. Tính $F=x^2+y^2+z^2$.
	\choice
	{$F=389$ hoặc $F=395$}
	{$F=395$ hoặc $F=179$}
	{$F=441$ hoặc $F=357$}
	{\True $F=389$ hoặc $F=179$}
	\loigiai{
		Theo tính chất của cấp số cộng, ta có $x+z=2y$.\\
		Kết hợp với giả thiết $x+y+z=21$, ta suy ra $3y=21\Leftrightarrow y=7$.\\
		Gọi $d$ là công sai của cấp số cộng thì $x=y-d=7-d$ và $z=y+d=7+d$.\\
		Sau khi thêm các số $2$, $3$, $9$ vào ba số $x$, $y$, $z$ ta được ba số là $x+2$, $y+3$, $z+9$ hay $9-d$, $10$, $16+d$.\\
		Theo tính chất của cấp số nhân, ta có $(9-d)(16+d)=10^2\Leftrightarrow d^2+7d-44=0$.\\
		Giải phương trình ta được $d=-11$ hoặc $d=4$.\\
		Với $d=-11$, cấp số cộng $18$, $7$, $-4$. Lúc này $F=389$.\\
		Với $d=4$, cấp số cộng $3$, $7$, $11$. Lúc này $F=179$.}
\end{ex}
\Closesolutionfile{ans}
% \begin{indapan}{10}
% 	{ans/ans-1K2-3-dang4}
% \end{indapan}
\begin{dang}{Tính tổng của cấp số nhân}
	Phương pháp
	\begin{itemize}
		\item Xác định số hạng đầu $u_1$, công bội $q$.
		\item Áp dụng công thức tính tổng các số hạng của cấp số nhân. 
	\end{itemize}
\end{dang}
\subsubsection{Ví dụ minh hoạ}
\begin{vd}%[NB]%[DCHT Toán 11 - KNTT -Đỗ Chí Tâm] %[1K2Y7-5]
	Tính tổng 10 số hạng đầu tiên của cấp số nhân $(u_n)$, biết $u_1=-3$ và công bội $q=-2$. \dapso{$S_{10}=1023$}
	\loigiai{
		Ta có: $S_{10}=\dfrac{u_1\left(1-q^{10}\right)}{1-q}=1023$.}
\end{vd}

\begin{vd}%[TH]%[DCHT Toán 11 - KNTT -Đỗ Chí Tâm] %[1K2B7-5]
	Tính tổng $8$ số hạng đầu tiên của cấp số nhân $(u_n),$ biết $u_1=3$ và $u_2=6$. \dapso{$S_8=765$}
	\loigiai{
		Ta có: $u_2=u_1.q \Leftrightarrow 6=3.q \Leftrightarrow q=2$\\
		$\hspace*{1.2cm}  S_8=u_1\dfrac{1-q^8}{1-q}=3.\dfrac{1-2^8}{1-2}=765. $
	}
\end{vd}

\begin{vd}[thuộc chương giới hạn]%[TH]%[DCHT Toán 11 - KNTT -Đỗ Chí Tâm] %[1K2B7-5]
	Tính tổng vô hạn $S=1+\dfrac{1}{2}+\dfrac{1}{2^2}+...+\dfrac{1}{2^n}+...$ \dapso{$S=2$}
	\loigiai{
		Đây là tổng của cấp số nhân lùi vô hạn, với $u_1=1, q=\dfrac{1}{2}$. Khi đó
		$$S=\dfrac{u_1}{1-q}=\dfrac{1}{1-\dfrac{1}{2}}=2.$$
	}
\end{vd}

\begin{vd}%[VD]%[DCHT Toán 11 - KNTT -Đỗ Chí Tâm] %[1K2K7-5]
	Tính tổng $200$ số hạng đầu tiên của dãy số $(u_n)$ biết $\heva{&u_1=1\\&u_{n+1}=3u_n}$.
	\loigiai{
		Dễ thấy dãy đã cho là một cấp số nhân với công bội $q=3; u_1=1$.\\
		Từ đó $S_{200}=u_1\dfrac{q^{200}-1}{q-1}$ $=\dfrac{3^{200}-1}{2}$.
	}
\end{vd}

\begin{vd}%[VD]%[DCHT Toán 11 - KNTT -Đỗ Chí Tâm]%[1K2K7-5]
	Một cấp số nhân có số hạng đầu $u_1=3$, công bội $q=2$. Biết $S_n=765$, tìm $n$.
	% \dapso{$n=8$}
	\loigiai{ 
		Áp dụng công thức tính tổng của cấp số nhân ta có
		$S_n=765 \Leftrightarrow \dfrac{u_1( 1-q^n )}{1-q}=765 \Leftrightarrow \dfrac{3.( 1-2^n )}{1-2}=765 \Leftrightarrow 2^n=256=2^8 \Leftrightarrow n=8$.} 
\end{vd}

\subsubsection{Bài tập tự luận}
 
\begin{bt}%[DCHT Toán 11 - KNTT -Đỗ Chí Tâm] %[1K2Y7-5]
	Một cấp số nhân có số hạng đầu $u_1=3$ và công bội $q=2 $. Tính tổng $8$ số hạng đầu của cấp số nhân.
	\dapso{$765$}		
	\loigiai{
		Ta có $S_8=\dfrac{u_1\left(1-q^8\right)}{1-q}=\dfrac{3 \left(1-2^8\right)}{1-2}=765$.	
	}	
\end{bt}

% \begin{bt}%[DCHT Toán 11 - KNTT -Đỗ Chí Tâm]%[1K2Y7-5]
% 	Một cấp số nhân có số hạng đầu $u_1=1$ và công bội $q=3$. Tính $S_{10}.$
% 	\dapso{$29524$}	
% 	\loigiai{
% 		Ta có $S_{10}=\dfrac{u_1\left(1-q^{10}\right)}{1-q}=1.\dfrac{1-3^{10}}{1-3}=29524$.	
% 	}	
% \end{bt}

% \begin{bt}%[DCHT Toán 11 - KNTT -Đỗ Chí Tâm] %[1K2Y7-5]
% 	Một cấp số nhân $(u_n)$ có $u_1=4$ và công bội $q=2$. Tính $S_{20}$.
% 	\dapso{$4194300$}	
% 	\loigiai{
% 		Ta có $S_{20}=\dfrac{u_1\left(1-q^{20}\right)}{1-q}=4194300$.}
% \end{bt}

\begin{bt}[thuộc chương giới hạn]%[DCHT Toán 11 - KNTT -Đỗ Chí Tâm] %[1K2B7-5]	
	Tính tổng $S=1+\dfrac{1}{3}+\dfrac{1}{3^2}+\cdots+\dfrac{1}{3^n}+\cdots$.
	\dapso{$\dfrac{3}{2}$}			
	\loigiai{
		Đây là tổng của một cấp số nhân lùi vô hạn với $u_1=1, q=\dfrac{1}{3}$\\
		Suy ra   $S=\dfrac{u_1}{1-q}=\dfrac{1}{1-\dfrac{1}{3}}=\dfrac{3}{2}.$
	}
\end{bt}

\begin{bt}%[DCHT Toán 11 - KNTT -Đỗ Chí Tâm] %[1K2B7-5]	
	Cho cấp số nhân có $q=-3, S_6=730$. Tính $u_1$.
	\dapso{$4$}		
	\loigiai{
		$S_6=u_1.\dfrac{1-q^6}{1-q}\Rightarrow u_1=S_6\cdot\dfrac{1-q}{1-q^6}=730\cdot \dfrac{1-(-3)}{1-(-3)^6}=4$.
	}
\end{bt}

% \begin{bt}%[DCHT Toán 11 - KNTT -Đỗ Chí Tâm] %[1K2B7-5]
% 	Một cấp số nhân $(u_n)$ có $u_1=-5$, $u_2=10$.Tính tổng của $15$ số hạng đầu của cấp số nhân đó.
% 	\dapso{$-54615$}	
% 	\loigiai{
% 		Công bội của cấp số nhân đã cho là: $q=\dfrac{u_2}{u_1}=\dfrac{10}{-5}=-2$.\\
% 		Tổng của $15$ số hạng đầu của cấp số nhân đó là $S_{15}=-5\cdot \dfrac{1-(-2)^{15}}{1-(-2)}=-54615$.
% 	}
% \end{bt}

% \begin{bt}%[DCHT Toán 11 - KNTT -Đỗ Chí Tâm] %[1K2B7-5]
% 	Một cấp số nhân $(u_n)$ có $u_1=2$, $u_2=-2$.Tính tổng của $9$ số hạng đầu của cấp số nhân đó.
% 	\dapso{$2$}	
% 	\loigiai{
% 		Công bội của cấp số nhân đã cho là: $q=\dfrac{u_2}{u_1}=\dfrac{-2}{2}=-1$.\\
% 		Tổng của $9$ số hạng đầu của cấp số nhân đó là: $S_9=2\cdot \dfrac{1-(-1)^9}{1-(-1)}=2$.
% 	}
% \end{bt}

\begin{bt}%[DCHT Toán 11 - KNTT -Đỗ Chí Tâm] %[1K2K7-5]
	Một cấp số nhân $(u_n)$ có $u_3=8$, $u_5=32$ và công bội $q>0$. Tính tổng của $10$ số hạng đầu tiên của cấp số nhân.
	\dapso{$2046$}	
	\loigiai{
		$\heva{u_3&=8\\u_5&=32}\Leftrightarrow \heva{u_1.q^2&=8\\u_1.q^4&=32}\Rightarrow q^2=\dfrac{32}{8}=4\Rightarrow q=2, u_1=2$\\
		\hspace*{5.2cm}$\Rightarrow S_{10}=u_1.\dfrac{1-q^{10}}{1-q}=2.\dfrac{2-2^{10}}{1-2}=2046$.
	}
\end{bt}

\begin{bt}%[DCHT Toán 11 - KNTT -Đỗ Chí Tâm] %[1K2K7-5]	
	Tính tổng $S=2+6+18+...+13122$.
	\dapso{$19682$}
	\loigiai{
		Xét cấp số nhân có $u_1=2, q=3$. Khi đó $13122=u_1.q^{n-1}\Leftrightarrow 13122=2.3^{n-1}\Leftrightarrow n=9$\\
		Vậy $S=S_9=u_1\dfrac{1-q^9}{1-q}=2.\dfrac{1-3^9}{1-3}=19682$
	}
\end{bt}

\begin{bt}%[DCHT Toán 11 - KNTT -Đỗ Chí Tâm] %[1K2K7-5]	
	Tính tổng $S=1+2+4+8+\cdots+1024$.	
	\dapso{$2047$}
	\loigiai{
		Xét cấp số nhân có $u_1=1, q=2$. Khi đó $1024=u_1.q^{n-1}\Leftrightarrow 1024=1.2^{n-1}\Leftrightarrow n=11$.\\
		Vậy $S=S_{11}=u_1\dfrac{1-q^{11}}{1-q}=1.\dfrac{1-2^{11}}{1-2}=2047$.
	}
\end{bt}

\begin{bt}%[DCHT Toán 11 - KNTT -Đỗ Chí Tâm] %[1K2K7-5]	
	Một cấp số nhân có $u_1=1, q=3$, biết $S_n=3280$. Tìm $n$.	
	\dapso{$8$}
	\loigiai{
		$S_n=u_1\dfrac{1-q^n}{1-q}=1.\dfrac{1-3^n}{1-3}=3280\Rightarrow n=8$.
	}
\end{bt}

% \begin{bt}%[DCHT Toán 11 - KNTT -Đỗ Chí Tâm] %[1K2K7-5]	
% 	Một cấp số nhân $(u_n)$ có $u_4+u_6=-540, u_3+u_5=180$. Tính $S_5$.
% 	\dapso{$122$}
% 	\loigiai{
% 		$\heva{u_4+u_6&=-540\\u_3+u_5&=180}\Leftrightarrow \heva{u_1.q^3+u_1.q^5&=-540\\u_1.q^2+u_1.q^4&=180} \Leftrightarrow \heva{u_1q^3(1+q^2)&=-540\\u_1q^2(1+q^2)&=180}\Leftrightarrow \heva{u_1&=2\\ q&=-3}$.\\
% 		Vậy $S_5=u_1\dfrac{1-q^5}{1-q}=2.\dfrac{1-(-3)^5}{1+3}=122$.
% 	}
% \end{bt}

\begin{bt}%[DCHT Toán 11 - KNTT -Đỗ Chí Tâm] %[1K2K7-5]
	Bốn số hạng liên tiếp của một cấp số nhân, trong đó số hạng thứ hai nhỏ hơn số hạng thứ nhất $35$, còn số hạng thứ ba lớn hơn số hạng thứ tư $560$. Tìm tổng của bốn số hạng trên, biết công bội mang giá trị dương.
	\dapso{$-\dfrac{2975}{3}$}
	\loigiai{
		Theo đề ta có $\heva{u_1-u_2&=35\\u_3-u_4&=560}\Leftrightarrow \heva{&u_1-u_1q=35\\&u_1q^2-u_1q^3=560}\Leftrightarrow \heva{&u_1(1-q)=35\,\,\,\,\,\,\, (1)\\&u_1q^2(1-q)=560\,\,\,\, (2)}$\\
		Thay $(1)$ vào $(2)$ ta được $q^2=16\Leftrightarrow q=\pm 4$.\\
		Với $q=4$ thay vào $(1)$ ta được $u_1=-\dfrac{35}{3}$.	\\
		$S_4=u_1.\dfrac{1-q^4}{1-q}=-\dfrac{2975}{3}$.
	}	
\end{bt}

\begin{bt}[thuộc chương giới hạn]%[DCHT Toán 11 - KNTT -Đỗ Chí Tâm] %[1K2G7-5]
	Tổng của một cấp số nhân lùi vô hạn bằng $ \dfrac{1}{4} $, tổng ba số hạng đầu tiên của cấp số nhân đó bằng $ \dfrac{7}{27} $. Tổng của số hạng đầu và công bội của cấp số nhân đó bằng
	\dapso{$S=0$}	
	\loigiai{Gọi $ u_1 $ và $ q $ với ($ |q|<1 $) lần lượt là số hạng đầu và cộng bội của cấp số nhân lùi vô hạn. Theo giả thiết, ta có $$\begin{cases}
			\dfrac{u_1}{q-1}=\dfrac{1}{4}\\
			u_1+u_1q+u_1q^2=\dfrac{7}{27}
		\end{cases} \Leftrightarrow \begin{cases}
			\dfrac{u_1}{q-1}=\dfrac{1}{4}\\
			u_1 (1-q^3)=\dfrac{7}{27}(1-q)
		\end{cases} \Leftrightarrow \begin{cases}\dfrac{u_1}{1-q}=\dfrac{1}{4}\\ q^3=-\dfrac{1}{27} \end{cases}\Leftrightarrow
		\heva{& u_1=\dfrac{1}{3}\\& q=-\dfrac{1}{3}.}$$ 
		Vậy $ u_1+q=0 $.
	}
\end{bt}

\begin{bt}%[DCHT Toán 11 - KNTT -Đỗ Chí Tâm] %[1K2G7-5]
	Một du khách vào trường đua ngựa đặt cược, lần đầu đặt $20.000$ đồng, mỗi lần sau tiền đặt gấp đôi số tiền lần đặt trước. Người đó thua $10$ lần liên tiếp và thắng ở lần thứ $11$. Hỏi du khách trên thắng hay thua bao nhiêu tiền?
	\dapso{$20000$ đồng}
	\loigiai{
		Số tiền du khách đặt trong mỗi lần (kể từ lần đầu) là một cấp số nhân có $u_1=20.000$ và công bội $q=2$.\\
		Du khách thua trong 10 lần liên tiếp đầu tiên nên tổng số tiền thua là
		$$S_{10}=\dfrac{u_1(1-q^{10})}{1-q}=\dfrac{20000(1-2^{10})}{1-2}=20000(2^{10}-1) \text{(đồng).}$$ 
		Số tiền du khách thắng trong lần thứ 11 là $u_{11}=u_1q^{10}=20000.2^{10}$ (đồng).\\
		Ta có $u_{11}-S_{10}=20000>0$. Vậy du khách thắng $20000$ đồng.
	}
\end{bt}
\subsubsection{Câu hỏi trắc nghiệm}
\Opensolutionfile{ans}[ans/ans-1K2-3-Dang5]
\begin{ex}%[1K2Y7-5]
	Cho cấp số nhân $u_1,u_2,u_3,\ldots,u_n$ với công bội $q$ ($q\neq 0$, $q\neq 1$). Đặt \[S_n=u_1+u_2+u_3+\cdots +u_n.\] Khẳng định nào sau đây là đúng?
	\choice
	{\True $S_n = \dfrac{u_1\left(q^n-1\right)}{q-1}$}
	{$S_n = \dfrac{u_1\left(q^n+1\right)}{q+1}$}
	{$S_n = \dfrac{u_1\left(q^{n-1}-1\right)}{q+1}$}
	{$S_n = \dfrac{u_1\left(q^{n-1}-1\right)}{q-1}$}
	\loigiai{
		Ta có $S_n=u_1+u_2+u_3+\cdots +u_n = u_1\cdot \dfrac{1-q^n}{1-q} = \dfrac{u_1\left(q^n-1\right)}{q-1}$.
	}
\end{ex}
\begin{ex}%[1K2Y7-5]
	Cho cấp số nhân $(u_n)$ có số hạng đầu $u_1=12$ và công sai $q=\dfrac{3}{2}$. Tổng $5$ số hạng đầu của cấp số nhân bằng
	\choice
	{$\dfrac{93}{4}$}
	{$\dfrac{633}{2}$}
	{\True $\dfrac{633}{4}$}
	{$\dfrac{93}{2}$}
	\loigiai{
		Gọi $S_5$ là tổng $5$ số hạng đầu của cấp số nhân đã cho. Khi đó ta có $$S_5=u_1\cdot \dfrac{1-q^5}{1-q}=12\cdot \dfrac{1-\left(\dfrac{3}{2}\right)^5}{1-\dfrac{3}{2}}=\dfrac{633}{4}.$$
	}
\end{ex}
\begin{ex}%[1K2Y7-5]
	Cho cấp số nhân $(u_n)$ có số hạng đầu $u_1=3$, công bội $q=-2$. Tính tổng $10$ số hạng đầu tiên của cấp số nhân $(u_n)$.
	\choice
	{\True $-1023$}
	{$1023$}
	{$513$}
	{$-513$}
	\loigiai{
		Tổng của $10$ số hạng đầu bằng
		$$S_{10}=u_1\cdot\dfrac{q^{10}-1}{q-1}=3\cdot\dfrac{(-2)^{10}-1}{-2-1}=-1023.$$
	}
\end{ex}
\begin{ex}%[1K2Y7-5]
	Cho cấp số nhân $(u_n)$ có $u_2=-2$ và $u_5=54$. Tính tổng $1000$ số hạng đầu tiên của cấp số nhân đã cho.
	\choice
	{$S_{1000}=\dfrac{3^{1000}-1}{2}$}
	{\True $S_{1000}=\dfrac{1-3^{1000}}{6}$}
	{$S_{1000}=\dfrac{3^{1000}-1}{6}$}
	{$S_{1000}=\dfrac{1-3^{1000}}{4}$}
	\loigiai{
		Ta có $u_5=u_2\cdot q^3\Leftrightarrow q^3=\dfrac{u_5}{u_2}=\dfrac{54}{-2}=-27=(-3)^3\Rightarrow q=-3$ và $u_1=\dfrac{u_2}{q}=\dfrac{2}{3}$.\\
		Suy ra $S_{1000}=u_1\cdot\dfrac{1-q^n}{1-q}=\dfrac{2}{3}\cdot\dfrac{1-(-3)^{1000}}{1+3}=\dfrac{1-3^{1000}}{6}$.
	}
\end{ex}
\begin{ex}%[1K2Y7-5]
	Tính tổng tất cả các số hạng của một cấp số nhân, biết số hạng đầu bằng $18$, số hạng thứ hai bằng $54$ và số hạng cuối bằng $39366$.
	\choice
	{$19674$}
	{\True $59040$}
	{$177138$}
	{$~6552$}
	\loigiai{
		$u_1=18,u_2=54 \Rightarrow q=3$.\\
		$u_n=39366 \Leftrightarrow u_1 \cdot q^{n-1}=39366 \Leftrightarrow 18 \cdot 3^{n-1}=39366 \Leftrightarrow 3^{n-1}=3^7 \Leftrightarrow n=8$.\\
		Vậy $S_8=18 \cdot \dfrac{1-3^8}{1-3}=59040$.}
\end{ex}
\begin{ex}%[1K2B7-5]
	Dãy số $\left(u_n\right)$ xác định bởi $\heva{&	u_1=1\\&u_{n + 1}= \dfrac{1}{2}u_n}$ với $n \ge 1$. Tính tổng $S = u_1 + u_2 +\cdots + u_{10}$.
	\choice
	{$S = \dfrac {1023} {2048}$}
	{$S = \dfrac{5}{2}$}
	{\True $\dfrac {1023} {512}$}
	{$S = 2$}
	\loigiai{
		Ta có các số hạng của dãy số $\left(u_n\right)$ là  $1,\dfrac{1}{2},\dfrac{1}{4},\dfrac{1}{8},\dfrac {1}{16},\dfrac{1}{32},\ldots ,\dfrac{1} {2^n}$. Khi đó $\left(u_n\right)$ lập thành một cấp số nhân có $u_1 = 1$ và công bội $q = \dfrac{1}{2}$. \\
		Suy ra $S = u_1 + u_2 +\cdots + u_{10} 
		=1+\dfrac{1}{2} + \dfrac{1}{4} +\cdots + \dfrac{1}{2^9}
		=\dfrac{1\cdot\left[1-\left(\dfrac{1}{2}\right)^{10}\right]}{1 - \dfrac{1}{2}}
		=\dfrac {1023} {512}$.}
\end{ex}
\begin{ex}%[1K2B7-5]
	Cho cấp số nhân $({{u}_{n}} )$ có ${{u}_{1}}=-6$ và $q=-2$. Tổng $n$ số hạng đầu tiên của cấp số nhân đã cho bằng $2046$. Tìm $n$.
	\choice
	{$n=9$}
	{$n=12$}
	{$n=11$}
	{\True $n=10$}
	\loigiai {
		Ta có
		$2046={{S}_{n}}={{u}_{1}}\cdot \dfrac{1-{{q}^{n}}}{1-q}=-6\cdot \dfrac{1-{{(-2 )}^{n}}}{1-(-2 )}=2({{(-2 )}^{n}}-1 )\Rightarrow {{(-2 )}^{n}}=1024\Leftrightarrow n=10$.}
\end{ex}
\begin{ex}%[1K2B7-5]
	Tổng $100$ số hạng đầu của dãy số $\left(u_n\right)$ với $u_n=2 n-1$ là
	\choice
	{$199$}
	{$2^{100}-1$}
	{\True $10000$}
	{$9999$}
	\loigiai{
		Ta có $(u_n)$ là cấp số cộng công sai $d=2$ và $u_1=1$.\\
		Do đó $S_{n}=n\cdot u_1+\dfrac{n(n-1)}{2}\cdot d=100 \cdot 1 +\dfrac{100 \cdot 99 }{2} \cdot 2 = 10000$.
	}
\end{ex}
\begin{ex}%[1K2B7-5]
	Cho dãy số $(u_n)$ với $u_n = \left(\dfrac{1}{2}\right)^n+1, \forall n \in \mathbb{N}^*$. Tính $S_{2019}=u_1+u_2+u_3+ \cdots + u_{2019}$.
	\choice
	{$S_{2019}=2019+\dfrac{1}{2^{2019}}$}
	{$S_{2019}=\dfrac{4039}{2}$}
	{$S_{2019}=\dfrac{6057}{2}$}
	{\True $S_{2019}=2020-\dfrac{1}{2^{2019}}$}
	\loigiai{
		Ta có
		\allowdisplaybreaks
		\begin{eqnarray*}
			S_{2019} &=& u_1+u_2+u_3+ \cdots + u_{2019}\\
			&=& \left(\dfrac{1}{2}+1\right) + \left[\left(\dfrac{1}{2}\right)^2+1\right] + \left[\left(\dfrac{1}{2}\right)^3+1\right] + \cdots + \left[\left(\dfrac{1}{2}\right)^{2019}+1\right]\\
			&=& 2019 + \dfrac{1}{2} + \left(\dfrac{1}{2}\right)^2 + \left(\dfrac{1}{2}\right)^3 + \cdots + \left(\dfrac{1}{2}\right)^{2019}\\
			&=& 2019 + \dfrac{1}{2} \cdot \dfrac{1-\left(\dfrac{1}{2}\right)^{2019}}{1-\dfrac{1}{2}} = 2019 + 1 - \dfrac{1}{2^{2019}}\\
			&=& 2020 - \dfrac{1}{2^{2019}}.
		\end{eqnarray*}
	}
\end{ex}
\begin{ex}%[1K2K7-5]
	Cho $S=11+101+1001+\cdots +\underbrace{1000\ldots 01}_{(n-1)\text{ chữ số 0}}$. Khẳng định nào sau đây là đúng?
	\choice
	{$S=10\left(\dfrac{10^n-1}{9}\right)$}
	{$S=10\left(\dfrac{10^n-1}{9}\right)-n$}
	{\True $S=10\left(\dfrac{10^n-1}{9}\right)+n$}
	{$S=\left(\dfrac{10^n-1}{9}\right)+n$}
	\loigiai{
		Ta có
		\begin{align*}
			S&=(10+1)+(10^2+1)+(10^3+1)+\cdots +(10^n+1)\\
			&=\left(10+10^2+10^3+\cdots +10^n\right)+\underbrace{1+1+1+\cdots+1}_{n\text{ số } 1}\\
			&=10\left(\dfrac{10^n-1}{9}\right)+n.
		\end{align*}
	}
\end{ex}
\begin{ex}%[1K2K7-5]
	Gọi $S=1+11+111+\cdots+\underbrace{111\ldots1}_{(n\text{ số }1)}$ thì $S$ nhận giá trị nào sau đây?
	\choice
	{\True $S=\dfrac{1}{9}\left[10\left(\dfrac{10^n-1}{9}\right)-n \right]$}
	{$S=\dfrac{10^n-1}{81}$}
	{$S=10\left(\dfrac{10^n-1}{81}\right)-n$}
	{$S=10\left(\dfrac{10^n-1}{81} \right)$}
	\loigiai {
		Ta có
		$S=\dfrac{1}{9}(9+99+999+\cdots+\underbrace{99\ldots9}_{\text{n số }9} )=\dfrac{1}{9}\cdot \left[ 10\cdot \dfrac{1-{{10}^{n}}}{1-10}-n \right]$.}
\end{ex}
\begin{ex}%[1K2K7-5]
	Cho dãy số $(u_n)$ thỏa mãn $\heva{&u_1=1\\&u_n=2u_{n-1}+1, n\geq2}$. Tổng $S=u_1+u_2+ \cdots +u_{20}$ là
	\choice
	{$2^{21}-20$}
	{\True $2^{21}-22$}
	{$2^{20}$}
	{$2^{20}-20$}
	\loigiai{
		Dự đoán công thức số hạng tổng quát $u_n=2^n-1$ (Chứng minh bằng phương pháp quy nạp TH).\\
		$S=2^1+2^2+\cdots+2^{20}-20=2\cdot\dfrac{1-2^{20}}{1-2}-20=2^{21}-22$.
	}
\end{ex}
\begin{ex}%[1K2K7-5]
	Biết rằng $S=1+2\cdot 3+{{3\cdot 3}^2}+\cdots+{{11\cdot 3}^{10}}=a+\dfrac{{{21\cdot 3}^{b}}}{4}$. Tính $P=a+\dfrac{b}{4}$.
	\choice
	{\True $P=3$}
	{$P=4$}
	{$P=1$}
	{$P=2$}
	\loigiai {
		Từ giả thiết suy ra $3S=3+{{2\cdot 3}^2}+{{3\cdot 3}^3}+\cdots+{{11\cdot 3}^{11}}$.\\
		Do đó
		{\allowdisplaybreaks
			\begin{eqnarray*}
				-2S&=&S-3S=1+3+{{3}^2}+\cdots+{{3}^{10}}-{{10\cdot 3}^{11}}\\
				&=&\dfrac{1-{{3}^{11}}}{1-3}-{{11\cdot 3}^{11}}=-\dfrac{1}{2}-\dfrac{{{21\cdot 3}^{11}}}{2}\Rightarrow S=\dfrac{1}{4}+\dfrac{21}{4}\cdot{{3}^{11}}.
			\end{eqnarray*}
		}
		Vì $S=\dfrac{1}{4}+\dfrac{{{21\cdot 3}^{11}}}{4}=a+\dfrac{{{21\cdot 3}^{b}}}{4}\Rightarrow a=\dfrac{1}{4},\,\,b=11\Rightarrow P=\dfrac{1}{4}+\dfrac{11}{4}=3$.}
\end{ex}
\Closesolutionfile{ans}
% \begin{indapan}{10}
% 	{ans/ans-1K2-3-Dang5}
% \end{indapan}

\begin{dang}{Kết hợp cấp số cộng và cấp số nhân}
	Nhắc lại tính chất CSC, CSN
	\begin{itemize}
		\item $3$ số $a,b,c$ theo thứ tự lập thành CSC thì $a+c=2b$.
		\item $3$ số $a,b,c$ theo thứ tự lập thành CSN thì $a.c=b^2$.
	\end{itemize}
\end{dang}
\subsubsection{Ví dụ minh hoạ}
\begin{vd}%[TH]%[DCHT Toán 11 - KNTT -Đỗ Chí Tâm] %[1K2K7-6]
	Ba số $x, y, z$ theo thứ tự đó lập thành một CSN với công bội $q (q\ne 1)$, đồng thời các số $x, 2y, 3z$ theo thứ tự đó lập thành một CSC với công sai $d$ . Hãy tìm $q$?
	\loigiai{
		Ta có $x+3z=2.2y \Leftrightarrow x+3xq^2=2.2xq\Leftrightarrow 1+3q^2=4q \Leftrightarrow \hoac{q&=\dfrac{1}{3}\\q&=1 (L)}$
	}
\end{vd}

\begin{vd}%[TH]%[DCHT Toán 11 - KNTT -Đỗ Chí Tâm] %[1K2K7-6]
	Biết rằng $a, b, c$ là ba số hạng liên tiếp của một CSC và $a, c, b$ là ba số hạng liên tiếp của một CSN, đồng thời $a+b+c=30$. Tìm $a,b,c$.
	\loigiai{
		Theo đề ta có  $\heva{&a+c=2b\,\,\,\hspace*{0.8cm}(1)\\&ab=c^2\,\,\,\hspace*{1.2cm}(2)\\&a+b+c=30\,\,\, (3)}$\\
		Từ $(1)$ và $(3)$ ta được $3b=30\Leftrightarrow b=10$\\
		Thay $b=10$ vào $(1), (2)$ ta được $\heva{a+c=20\\10a=c^2}\Leftrightarrow \hoac{&c=10, a=10\,\, (L)\\ &c=-20, a=40 (N)}$\\
		Vậy $a=40, b=10, c=-20$
	}
\end{vd}

\begin{vd}%[VD]%[DCHT Toán 11 - KNTT -Đỗ Chí Tâm] %[1K2K7-6]
	Ba số $x, y, z$ theo thứ tự đó lập thành một CSN. Ba số $x, y-4 , z$ theo thứ tự đó lập thành CSN. Đồng thời các số $x, y-4 , z-9$ theo thứ tự đó lập thành CSC. Tìm $x,y,z$.
	
	\loigiai{
		Theo đề ta có  $\heva{&xz=y^2\,\,\,\hspace*{2.2cm}(1)\\&xz=(y-4)^2\,\,\,\hspace*{1.4cm}(2)\\&x+(z-9)=2(y-4)\,\,\, (3)}$\\
		Từ $(1)$ và $(2)$ ta có $y^2=(y-4)^2\Leftrightarrow y=2$\\
		Thế $y=2$ vào $(1)$ và $(3)$ ta được $\heva{xz=4\\x+z=5}\Rightarrow x=4, z=1$ hoặc $x=1, z=4$\\
		Vậy có 2 bộ $(x,y,z)$ thỏa yêu cầu bài toán là $(1,2,4)$ và $(4,2,1)$
	}
\end{vd}

\begin{vd}%[VD]%[DCHT Toán 11 - KNTT -Đỗ Chí Tâm] %[1K2K7-6]
	Cho $a,b,c$ là ba số hạng liên tiếp của một CSN và $a,b,c-4$ là ba số hạng liên tiếp của một CSC, đồng thời $a,b-1,c-5$ là ba số hạng liên tiếp của một CSN. Tìm $a,b,c$ biết $a,b,c$ là các số nguyên. 
	
	\loigiai{
		Theo đề ta có  $\heva{&ac=b^2\,\,\,\hspace*{1.9cm}(1)\\&a+c-4=2b\,\,\,\hspace*{0.8cm}(2)\\&a(c-5)=(b-1)^2\,\,\,\,(3)}$\\
		Thay $(1)$ vào $(3)$: $b^2-5a=b^2-2b+1\Leftrightarrow b=\dfrac{5a+1}{2}$\\
		Thay vào $(2)$ ta được $a+c-4=5a+1\Leftrightarrow c=4a+5$\\
		Thế $b, c$ theo $a$ vào $(1)$ ta được $9a^2-10a+1=0\Leftrightarrow a=1 \vee a=\dfrac{1}{9} (L)$
		Vậy $a=1, b=3, c=9$
	}
\end{vd}

\begin{vd}%[VDC]%[DCHT Toán 11 - KNTT -Đỗ Chí Tâm]%[1K2G7-6]
	Cho $4$ số nguyên dương, trong đó $3$ số đầu lập thành một CSC, $3$ số hạng sau thành lập CSN.
	Biết rằng tổng của số hạng đầu và số hạng cuối là $37$, tổng của hai số hạng giữa là $36$. Tìm tổng $4$ số đó
	\loigiai{
		Gọi 4 số cần tìm lần lượt là $a, b, c, d$\\
		$a, b, c$ là $3$ số hạng liên tiếp của CSC. Ta có $a+c=2b\,\,\, (1)$\\
		$b,c,d$ là $3$ số hạng liên tiếp của CSN. Ta có $bd=c^2\,\,\, (2)$\\
		Theo giả thuyết ta có $\heva{a+d&=37\,\,\, (3)\\b+c&=36\,\,\, (4)}$\\
		Từ $(4)\Rightarrow b=36-c$ thay vào $(1)$ ta được $a=72-3c$, thay $a$ vào $(3)$ ta được $d=-35+3c$\\
		Thế $b,d$ vào $(2)$ ta được $(36-c)(-35+3c)=c^2\Rightarrow c=20 \vee c=\dfrac{63}{4} (L)$\\
		Vậy $c=20, a=12, b=16, d=95\Rightarrow S=a+b+c+d=143$
	}
\end{vd}

\subsubsection{Bài tập tự luận}
 
\begin{bt}%[DCHT Toán 11 - KNTT -Đỗ Chí Tâm]%[1K2K7-6]
	Biết $x, y, x+4$ theo thứ tự lập thành cấp số cộng và $x+1, y+1, 2y+2$ theo thứ tự lập thành cấp số nhân với $x, y$ là số thực dương. Tính $x+y$.
	\dapso{$4$}
	\loigiai{
		Theo giả thiết ta có:\\
		$\heva{&x+(x+4)=2y\\&(x+1)(2y+2)=(y+1)^2}\Leftrightarrow \heva{&y=x+2\\&(x+1)(2x+6)=(x+3)^2}\Leftrightarrow \heva{&x=1\Rightarrow y=3\\&x=-3\Rightarrow y=-1}$\\
		Do đó $x+y=4$.
	}	
\end{bt}

\begin{bt}%[DCHT Toán 11 - KNTT -Đỗ Chí Tâm] %[1K2K7-6]
	Cho $3$ số $a, b, c$ theo thứ tự tạo thành một cấp số nhân với công bội khác $1$. Biết cũng theo thứ tự đó chúng lần lượt là số hạng thứ nhất, thứ tư và thứ tám của một cấp số cộng với công sai $d\ne 0$. Tính $\dfrac{a}{d}$.
	\dapso{$9$}
	\loigiai{
		$a, b, c$ lần lượt là số hạng thứ nhất, thứ tư, thứ tám của một CSC với công sai $d$\\
		ta có $\heva{b &=a+3d\\c&=a+7d}$.\\
		Mặt khác $a, b, c$ là $3$ số hạng liên tiếp của CSN nên\\ $a.c=b^2\Leftrightarrow a(a+7d)=(a+3d)^2\Leftrightarrow a^2+7ad=a^2+6ad+9d^2\Leftrightarrow 9d^2=ad\Leftrightarrow \dfrac{a}{d}=9$.
	}	
\end{bt}

\begin{bt}%[DCHT Toán 11 - KNTT -Đỗ Chí Tâm] %[1K2K7-6]
	Tìm tích các số dương $a$ và $b$ sao cho $a, a + 2b, 2a + b$ lập thành một cấp số cộng và $(b + 1)^2, ab + 5,(a + 1)^2$ lập thành một cấp số nhân.
	\dapso{$3$}
	\loigiai{
		Theo tính chất CSC ta có $a+(2a+b)=2(a+2b)\,\,\,\, (1)$\\
		Theo tính chất CSN ta có $(b+1)^2.(a+1)^2=(ab+5)^2\,\,\,\, (2)$\\
		Từ $(1)$ ta được $a=3b$, thay vào $(2)$ ta được $(b+1)^2(3b+1)^2=(3b^2+5)^2$\\
		$\Leftrightarrow \hoac{&(b+1)(3b+1)=(3b^2+5)\\& (b+1)(3b+1)=-(3b^2+5) \,\, (\text{Vô nghiêm})}\Leftrightarrow b=1, a=3\Rightarrow ab=3.$	
	}	
\end{bt}

\begin{bt}%[DCHT Toán 11 - KNTT -Đỗ Chí Tâm] %[1K2K7-6]
	$a,b,c\,(a\ne b\ne c)$ là ba số hạng liên tiếp của một cấp số cộng và $b,c,a$ là ba số hạng liên tiếp của một cấp số nhân, đồng thời $a.b.c=125$. Tìm $a,b,c$.	
	\dapso{$5$}
	\loigiai{
		$a,b, c$ là ba số hạng liên tiếp của cấp số cộng, nên có $a+c=2b$.\\
		$b,c,a$ là ba số hạng liên tiếp của một cấp số nhân, nên có $b.a=c^2$.\\
		Ta có hệ $\heva{&a+c=2b\,\,\,\, (1)\\&b.a=c^2\,\,\,\,\,\,\,\,(2) \\&a.b.c=125\,\,\,\, (3)}$\\
		Thay $(2)$ vào $(3)$ ta được $c^3=125\Rightarrow c=5$\\
		Thay $c=5$ vào $(1), (2)$ ta được hệ $\heva{a+5&=2b\\ab&=25}\Leftrightarrow \heva{&a=2b-5\\&2b^2-5b-25=0}\Leftrightarrow \hoac{&b=5\Rightarrow a=5\\&b=-\dfrac{5}{2}\Rightarrow a=-10}$\\
		Vậy $a=-10, b=-\dfrac{5}{2}, c=5$.	
	}	
\end{bt}

\begin{bt}%[DCHT Toán 11 - KNTT -Đỗ Chí Tâm] %[1K2K7-6]
	Một cấp số cộng và một cấp số nhân đều là các dãy tăng các số hạng thứ nhất của hai dãy số đều bằng $3$, các số hạng thứ hai bằng nhau. Tỉ số giữa các số hạng thứ ba của CSN và CSC là $\dfrac{9}{5}$. Tìm tích ba số hạng của cấp số cộng thỏa mãn tính chất trên.
	\dapso{$405$}
	\loigiai{
		Gọi $u_1, u_2, u_3$ là $3$ số hạng liên tiếp của CSC.\\
		Gọi $a_1, a_2, a_3$ là $3$ số hạng liên tiếp của CSN.\\
		Theo đề ta có hệ $\heva{&u_1=a_1=3\\&u_2=a_2\\&a_3=\dfrac{9}{5}u_3}\Leftrightarrow \heva{&u_1=a_1=3\\&3+d=3q\\&5(3q^2)=9(3+2d)}\Rightarrow q=3 \vee q=\dfrac{3}{5}$\\
		Chọn $q=3$ vì dãy tăng, khi đó $d=6$. \\
		Vậy $3$ số hạng của cấp số cộng là $3; 9; 15\Rightarrow 3\cdot9\cdot15=405$	
	}	
\end{bt}

\begin{bt}%[DCHT Toán 11 - KNTT -Đỗ Chí Tâm] %[1K2K7-6]
	Một CSC và CSN đều có số hạng đầu tiên là bằng 5, số hạng thứ hai của CSC lớn hơn số hạng thứ hai của CSN là 10, còn các số hạng thứ 3 của hai cấp số thì bằng nhau. Tìm tổng các số hạng của cấp số cộng biết công bội của cấp số nhân không âm.
	\dapso{$75$}
	\loigiai{
		Gọi $u_1, u_2, u_3$ là $3$ số hạng liên tiếp của CSC với công sai $d$.\\
		Gọi $a_1, a_2, a_3$ là $3$ số hạng liên tiếp của CSN với công bội $q$.\\
		Theo đề bài ta có: $\heva{&u_1=a_1=5\\&u_2-a_2=10\\&u_3=a_3}\Leftrightarrow \heva{&u_1=a_1=5\\&u_1+d-a_1q=10\\&u_1+2d=a_1q^2}\Leftrightarrow \heva{&u_1=a_1=5\\&d=5+5q\\&5+2d=5q^2}\Rightarrow q=3\vee q=-1 (L)$\\
		Với $q=3\Rightarrow d=20$. Vậy CSC là $5;25;45 \Rightarrow S=5+25+45=75$
	}	
\end{bt}

\begin{bt}%[DCHT Toán 11 - KNTT -Đỗ Chí Tâm] %[1K2G7-6]	
	Ba số khác nhau có tổng bằng $114$ có thể coi là ba số hạng liên tiếp của một CSN, hoặc coi là số hạng thứ nhất, thứ tư và thứ hai mươi lăm của một CSC. Tìm ba số đó.
	\dapso{$2; 14; 98$}
	\loigiai{
		Gọi $u_1, u_2, u_3$ là $3$ số hạng liên tiếp của CSN với công bội $q$.\\
		Theo đề $u_1=a_1, u_2=a_4, u_3=a_{25}$ với $a_1,a_4, a_{25}$ là 3 số hạng của CSC với công sai $d$.\\
		Ta có $\heva{&a_4=a_1+3d\\&a_{25}=a_1+24d}\Rightarrow 8a_4-a_{25}=7a_1\Leftrightarrow 8u_2-u_3=7u_1\Leftrightarrow 8u_1q-u_1q^2=7u_1$\\
		\hspace*{4.2cm}$\Leftrightarrow q^2-8q+7=0 \Leftrightarrow q=1 (L) \vee q=7 (N)$\\
		Theo đề ta cũng có $u_1+u_2+u_3=114\Leftrightarrow u_1+u_1q+u_1q^2=114\Rightarrow u_1=2$\\
		Vậy $3$ số cần tìm là $2; 14; 98$.
	}	
\end{bt}

\begin{bt}%[DCHT Toán 11 - KNTT -Đỗ Chí Tâm] %[1K2G7-6]
	Ba số khác nhau có tổng là $217$ có thể coi là các số hạng liên tiếp của một CSN hoặc là các số hạng thứ $2$ thứ $9$ và thứ $44$ của một CSC. Tìm 3 số đó. 
	\dapso{$7; 35; 175$}
	\loigiai{
		Gọi $u_1, u_2, u_3$ là $3$ số hạng liên tiếp của CSN với công bội $q$.\\
		Theo đề $u_1=a_2, u_2=a_9, u_3=a_{44}$ với $a_2,a_9, a_{44}$ là 3 số hạng của CSC với công sai $d$.\\
		Ta có $\heva{&a_9=a_2+7d\\&a_{44}=a_2+42d}\Rightarrow 6a_9-a_{44}=5a_2\Leftrightarrow 6u_2-u_3=5u_1\Leftrightarrow 6u_1q-u_1q^2=5u_1$\\
		\hspace*{4.2cm}$\Leftrightarrow q^2-6q+5=0 \Leftrightarrow q=1 (L) \vee q=5 (N)$\\
		Theo đề ta cũng có $u_1+u_2+u_3=217\Leftrightarrow u_1+u_1q+u_1q^2=217\Rightarrow u_1=7$\\
		Vậy $3$ số cần tìm là $7; 35; 175$.
	}	
\end{bt}

% \begin{bt}%[DCHT Toán 11 - KNTT -Đỗ Chí Tâm] %[1K2K7-6]
% 	$a,b,c$ là ba số hạng liên tiếp của một CSN và $a,b+2,c+9$ là ba số hạng liên tiếp của một CSC, đồng thời $a,b+2,c$ là ba số hạng liên tiếp của một CSN khác. Tìm $a$.
% 	\dapso{$m=\dfrac{-7\pm3\sqrt{5}}{2}$}
% 	\loigiai{
% 		Vì $a,b,c$ là ba số hạng liên tiếp của CSN, ta có $ac=b^2\,\,\,\, (1)$\\
% 		Vì $a,b+2,c+9$ là ba số hạng liên tiếp của CSC, ta có $a+(c+9)=2(b+2)\,\,\,\, (2)$\\
% 		Vì $a,b+2,c$ là ba số hạng liên tiếp của CSN, ta có $a.c=(b+2)^2\,\,\,\, (3)$\\
% 		Thế $(1)$ vào $(3)$, ta được $b^2=(b+2)^2\Leftrightarrow b=-1$\\
% 		Thay $b=-1$ vào $(1), (2)$, ta được $\heva{&ac=1\\&a+c=-7}\Leftrightarrow a=\dfrac{-7+3\sqrt{5}}{2}\vee a=\dfrac{-7-3\sqrt{5}}{2}$
% 	}	
% \end{bt}

% \begin{bt}%[DCHT Toán 11 - KNTT -Đỗ Chí Tâm] %[1K2G7-6]
% 	Một CSC và CSN có cùng các số hạng thứ $m+1$, thứ $n+1$, thứ $p+1$ và $3$ số hạng này là $3$ số dương $a, b, c$. Tính $T=a^{b-c}.b^{c-a}.c^{a-b}$.
% 	\dapso{$m=1$}
% 	\loigiai{
% 		$a=u_1+md=q^m.v_1$\\
% 		$b=u_1+nd=q^n.v_1$\\
% 		$c=u_1+pd=q^p.v_1$\\
% 		Suy ra $T=a^{b-c}.b^{c-a}.c^{a-b}=\left(q^mv_1 \right)^{(n-p)d}. \left(q^nv_1 \right)^{(p-m)d}. \left(q^pv_1 \right)^{(m-n)d}=1$
% 	}	
% \end{bt}

% \begin{bt}%[DCHT Toán 11 - KNTT -Đỗ Chí Tâm] %[1K2G7-6]
% 	Tìm $m$ dương để phương trình $x^3+(5-m)x^2+(6-5m)x-6m=0 \,\,\,(*)$ có $3$ nghiệm phân biệt lập thành cấp số nhân.
% 	\dapso{$m=\sqrt{6}$}
% 	\loigiai{
% 		$(*)\Leftrightarrow (x+2) \left(x^2+(3-m)x-3m \right)=0\Leftrightarrow x=-2 \vee x=-3 \vee x=m$.\\
% 		Để $(*)$ có 3 nghiệm phân biệt thì $m\ne -3$ và $m\ne -2$.\\
% 		Do $3$ nghiệm này lập thành cấp số nhân, ta sắp xếp các nghiệm này theo thứ tự tăng dần được các dãy số sau
% 		\begin{itemize}
% 			\item $-3;-2; m$ lập thành CSN $\Leftrightarrow$ $-3m=(-2)^2\Leftrightarrow m=-\dfrac{4}{3}$
% 			\item $-3; m; -2$ lập thành CSN $\Leftrightarrow -3(-2)=m^2\Leftrightarrow m=\pm \sqrt{6}$
% 			\item $m; -3; -2$ lập thành CSN $\Leftrightarrow m(-2)=(-3)^2\Leftrightarrow m=-\dfrac{9}{2}$
% 		\end{itemize}	
% 		So với điều kiện thì $m=\sqrt{6}$ thỏa yêu cầu bài toán.
% 	}	
% \end{bt}
% \begin{bt}%[DCHT Toán 11 - KNTT -Đỗ Chí Tâm] %[1K2G7-6]
% 	Tìm tham số $m$ để phương trình $x^3-(2m+1)x^2+2mx=0 \,\, (*)$ có $3$ nghiệm phân biệt lập thành một cấp số cộng, biết $m<0$.
% 	\dapso{$m=-\dfrac{1}{2}$}
% 	\loigiai{
% 		$(*)\Leftrightarrow x.\left(x^2-(2m+1)x+2m \right)=0\Leftrightarrow x=0 \vee x=1 \vee x=2m$.\\
% 		Để $(*)$ có 3 nghiệm phân biệt thì $m\ne 0$ và $m\ne \dfrac{1}{2}$.\\
% 		Do $3$ nghiệm này lập thành cấp số cộng, ta sắp xếp các nghiệm này theo thứ tự tăng dần được các dãy số sau
% 		\begin{itemize}
% 			\item $2m;0; 1$ lập thành CSC $\Leftrightarrow$ $2m+1=2.0\Leftrightarrow m=-\dfrac{1}{2}$
% 			\item $0; 2m; 1$ lập thành CSC $\Leftrightarrow 0+1=4m\Leftrightarrow m=\dfrac{1}{4}$
% 			\item $0; 1; 2m$ lập thành CSC $\Leftrightarrow 0+2m=2.1\Leftrightarrow m=1$
% 		\end{itemize}	
% 		Vậy $m=-\dfrac{1}{2}$ là giá trị cần tìm.
% 	}	
% \end{bt}
% \subsubsection{Câu hỏi trắc nghiệm}
% \Opensolutionfile{ans}[ans/ans-1K2-3-Dang6]
% \begin{ex}%[1K2K7-6]
% 	Các số $x+6y$, $5x+2y$, $8x+y$ theo thứ tự đó lập thành một cấp số cộng, đồng thời các số $x-1$, $y+2$, $x-3y$ theo thứ tự đó lập thành một cấp số nhân. Tính $x^2+y^2$.
% 	\choice
% 	{$x^2+y^2=25$}
% 	{\True $x^2+y^2=40$}
% 	{$x^2+y^2=100$}
% 	{$x^2+y^2=10$}
% 	\loigiai{
% 		Theo bài ra, ta có
% 		\[ \heva{& (x+6 y)+(8 x+y)=2(5 x+2 y) \\ & (y+2)^2=(x-1)(x-3y)} \Rightarrow \heva{& x=3y \\ & (y+2)^2=0}\Rightarrow \heva{& x=-6 \\ & y=-2}\Rightarrow x^2+y^2=40. \]
% 	}
% \end{ex}

% \begin{ex}%[1K2K7-6]
% 	Cho hai số dương $ a $ và $ b $ không vượt quá $ 10 $ sao cho $ a-b $; $ 2 $; $ b $ theo thứ tự tạo thành một cấp số cộng và $ a+b $; $ 3a-2b $; $ 5a $ theo thứ tự lập thành một cấp số nhân. Tính giá trị của $ S=a+b $.
% 	\choice
% 	{$ S=8 $}
% 	{$ S=20 $}
% 	{$ S=7 $}
% 	{\True $ S=5 $}
% 	\loigiai{
% 		Vì $ a-b $; $ 2 $; $ b $ theo thứ tự tạo thành một cấp số cộng nên $ 2-(a-b)=b-2 \Leftrightarrow a=4 $.\\
% 		Vì $ a+b $; $ 3a-2b $; $ 5a $ theo thứ tự lập thành một cấp số nhân nên $$ \dfrac{3a-2b}{a+b}=\dfrac{5a}{3a-2b}\Leftrightarrow \dfrac{12-2b}{4+b}=\dfrac{20}{12-2b}\Leftrightarrow \hoac{&b=1\\&b=16 \text{ (loại).}} $$
% 		Vậy $ S=a+b=4+1=5. $
% 	}
% \end{ex}
% \begin{ex}%[1K2K7-6]
% 	Số hạng thứ hai, số hạng đầu và số hạng thứ ba của một cấp số cộng với công sai khác 0 theo thứ tự đó lập thành một cấp số nhân với công bội $q$. Tìm $q$.
% 	\choice
% 	{$q=2$}
% 	{\True $q=-2$}
% 	{$q=\dfrac{3}{2}$}
% 	{$q=-\dfrac{3}{2}$}
% 	\loigiai {
% 		Giả sử ba số hạng $a;b;c$ lập thành cấp số cộng thỏa yêu cầu, khi đó $b;a;c$ theo thứ tự đó lập thành cấp số nhân công bội $q$. Ta có\\
% 		$\heva{
% 			& a+c=2b \\
% 			& a=bq;\,c=b{{q}^2} \\
% 		}\Rightarrow bq+b{{q}^2}=2b\Leftrightarrow \hoac{
% 			& b=0 \\
% 			& {{q}^2}+q-2=0. \\
% 		}$ \\
% 		Nếu $b=0\Rightarrow a=b=c=0$ nên $a;b;c$ là cấp số cộng công sai $d=0$ (vô lí).\\
% 		Nếu ${{q}^2}+q-2=0\Leftrightarrow q=1$ hoặc $q=-2$ Nếu $q=1\Rightarrow a=b=c$ (vô lí), do đó $q=-2$.}
% \end{ex}

% \begin{ex}%[1K2K7-6]
% 	Cho ba số $a$, $b$, $c$ theo thứ tự tạo thành cấp số nhân với công bội khác $1$. Biết cũng theo thứ tự đó chúng lần lượt là số hạng thứ nhất, thứ tư và thứ tám của một cấp số cộng công sai là $s\neq 0$. Tính $\dfrac{a}{s}$.
% 	\choice
% 	{$3$}
% 	{$\dfrac{4}{9}$}
% 	{\True $9$}
% 	{$\dfrac{4}{3}$}
% 	\loigiai{
% 		Rõ ràng $a\neq 0$. Vì $s$ là công sai cấp số cộng nên $a$, $a+3s$, $a+7s$ lập thành cấp số nhân, do đó
% 		$$a(a+7s)=(a+3s)^2 \Leftrightarrow 9s^2-as=0\Leftrightarrow \hoac{& s=0\quad\text{(loại)}\\ & a=9s}\Leftrightarrow \dfrac{a}{s}=9.$$
% 	}
% \end{ex}

% \begin{ex}%[1K2K7-6]
% 	Xét các số thực dương $a, b$ sao cho $-25$, $2a$, $3b$ là cấp số cộng và $2$, $a+2$, $b-3$ là cấp số nhân. Khi đó $a^2 + b^2-3ab$ bằng
% 	\choice
% 	{$76$}
% 	{$89$}
% 	{$31$}
% 	{\True $59$}
% 	\loigiai{
% 		$-25,\,2a,\,3b$ là cấp số cộng $ \Leftrightarrow 2.2a=-25+3b \Leftrightarrow b=\dfrac{1}{3}(4a+25)$.
% 		\begin{eqnarray*}
% 			& 2,a+2,b-3 \text{ là cấp số nhân}& \Leftrightarrow (a+2)^2=2(b-3)\\
% 			& & \Leftrightarrow (a+2)^2=2\left[\dfrac{1}{3}(4a+25)-3\right]\\
% 			& & \Leftrightarrow 3a^2+4a-20=0 \\
% 			& & \Leftrightarrow \hoac{&a=2 \\&a=-\dfrac{10}{3}\,(\text{loại}).}
% 		\end{eqnarray*}
% 		Suy ra $b=11 \Rightarrow a^2+b^2-3ab=59$.
% 	}
% \end{ex}
% \begin{ex}%[1K2K7-6]
% 	Cho ba số $x$; $5$; $2y$ theo thứ tự lập thành cấp số cộng và ba số $x$; $4$; $2y$ theo thứ tự lập thành cấp số nhân thì $|x-2y|$ bằng
% 	\choice
% 	{$10$}
% 	{$8$}
% 	{$9$}
% 	{\True $6$}
% 	\loigiai{
% 		Theo giả thiết ta có
% 		\begin{itemize}
% 			\item Do $x;5;2y$ theo thứ tự lập thành một cấp số cộng nên ta có $5=\dfrac{x+2y}{2}\quad\quad (1)$.
% 			\item Do $x;4;2y$ theo thứ tự lập thành một cấp số nhân nên ta có $x\cdot 2y=4^2\quad\quad (2)$.
% 		\end{itemize}
% 		Từ $(1)$ và $(2)$ ta có $\heva{&x+2y=10\\&x\cdot 2y =16}\Rightarrow \hoac{& x=2,2y=8\\&x=8,2y=2}\Rightarrow \left|x-2y\right|=6$.
% 	}
% \end{ex}

% \begin{ex}%[1K2K7-6]
% 	Cho dãy số tăng $a, b, c\,\,(c\in \mathbb{Z} )$ theo thứ tự lập thành cấp số nhân; đồng thời $a,b+8,c$ theo thứ tự lập thành cấp số cộng và $a, b+8, c+64$ theo thứ tự lập thành cấp số nhân. Tính giá trị biểu thức $P=a-b+2c$.
% 	\choice
% 	{$P=\dfrac{184}{9}$}
% 	{\True $P=64$}
% 	{$P=\dfrac{92}{9}$}
% 	{$P=32$}
% 	\loigiai{
% 		Ta có $\heva{
% 			& ac=b^2 \\
% 			& a+c=2(b+8 ) \\
% 			& a(c+64 )={{(b+8 )}^2} \\
% 		}\Leftrightarrow \heva{
% 			& ac=b^2\quad \quad(1 ) \\
% 			& a-2b=16-c\quad(2 ) \\
% 			& ac+64a=(b+8 )^2\quad(3 )\\
% 		}$.
% 		Thay $(1)$ vào $(3)$ ta được: $$b^2+64a=b^2+16b+64\Leftrightarrow 4a-b=4. \quad
% 		(4 )$$
% 		Kết hợp $(2)$ với $(4)$ ta được: $\heva{
% 			& a-2b=16-c \\
% 			& 4a-b=4 \\
% 		}\Leftrightarrow \heva{
% 			& a=\dfrac{c-8}{7} \\
% 			& b=\dfrac{4c-60}{7}. \\
% 		}\,\,\,\,\,(5 )$ \\
% 		Thay $(5)$ vào $(1)$ ta được:\\
% 		$7(c-8 )c={{(4c-60 )}^2}\Leftrightarrow 9{{c}^2}-424c+3600=0\Leftrightarrow \hoac{
% 			& c=36 \\
% 			& c=\dfrac{100}{9} \\
% 		}\Leftrightarrow c=36\,\,(c\in \mathbb{Z} )$. \\
% 		Với $c=36\Rightarrow a=4,\,\,b=12\Rightarrow P=4-12+72=64$.}
% \end{ex}

% \begin{ex}%[1K2K7-6]
% 	Cho bốn số $a$, $b$, $c$, $d$ theo thứ tự đó tạo thành cấp số nhân với công bội khác $1$. Biết tổng ba số hạng đầu bằng $\dfrac{148}{9}$, đồng thời theo thứ tự đó $a$, $b$, $c$ lần lượt là số hạng thứ nhất, thứ tư và thứ tám của một cấp số cộng. Tính giá trị của biểu thức $T=a-b+c-d$.
% 	\choice
% 	{$T=-\dfrac{101}{27}$}
% 	{$T=\dfrac{100}{27}$}
% 	{\True $T=-\dfrac{100}{27}$}
% 	{$T=\dfrac{101}{27}$}
% 	\loigiai{
% 		Gọi $s$ ($s\neq 0$) là công sai của cấp số cộng. Vì $a$, $b$, $c$ theo thứ tự lần lượt là số hạng thứ nhất, thứ tư và thứ tám của cấp số cộng nên $b=a+3s$ và $c=a+7s$.\\
% 		Mặt khác, $a$, $b$, $c$ theo thứ tự tạo thành cấp số nhân với công bội khác $1$ nên
% 		\[ac=b^2 \Leftrightarrow a(a+7s)=(a+3s)^2 \Leftrightarrow as=9s^2 \Leftrightarrow a=9s \,\,(\text{vì } s\neq 0).\]
% 		Suy ra $b=12s$, $c=16s$.\\
% 		Theo giả thiết
% 		\[a+b+c=\dfrac{148}{9} \Leftrightarrow 9s+12s+16s=\dfrac{148}{9} \Leftrightarrow 37s=\dfrac{148}{9} \Leftrightarrow s=\dfrac{4}{9}.\]
% 		Suy ra $a=4$, $b=\dfrac{16}{3}$, $c=\dfrac{64}{9}$. Từ đó ta tính được $d=4\cdot \left(\dfrac{4}{3}\right)^3 = \dfrac{256}{27}$.\\
% 		Vậy $T=a-b+c-d = 4-\dfrac{16}{3}+\dfrac{64}{9}-\dfrac{256}{27} = -\dfrac{100}{27}$.
% 	}
% \end{ex}
% \begin{ex}%[1K2K7-6]
% 	Cho $x$ và $y$ là các số nguyên thỏa mãn các số $x+6y$ ,$5x+2y$, $8x+y$ theo thứ tự lập thành cấp cộng và các số $x-\dfrac{5}{3}y$, $y-1$, $2x-3y$ theo thứ tự lập thành cấp số nhân. Tính tổng $S=2x+3y$.
% 	\choice
% 	{$9$}
% 	{$6$}
% 	{$-6$}
% 	{\True $-9$}
% 	\loigiai{
% 		Vì các số $x+6y$ ,$5x+2y$, $8x+y$ theo thứ tự lập thành cấp cộng nên ta có
% 		$$ (x+6y)+(8x+y)=2(5x+2y)\Leftrightarrow x=3y. $$
% 		Vì các số $x-\dfrac{5}{3}y$, $y-1$, $2x-3y$ theo thứ tự lập thành cấp số nhân nên ta có
% 		$$ \left(x-\dfrac{5}{3}y\right) (2x-3y)=(y-1)^2.  $$
% 		Thay $x=3y$ vào phương trình trên, ta được
% 		\begin{eqnarray*}
% 			& & \left(3y-\dfrac{5}{3}y\right) (6y-3y)=(y-1)^2\\
% 			&\Leftrightarrow & 4y^2=y^2-2y+1\\
% 			&\Leftrightarrow & \hoac{& y=-1\\& y=\dfrac{1}{3}}.
% 		\end{eqnarray*}
% 		Ta loại trường hợp $y=\dfrac{1}{3}$ vì $y$ là số nguyên. Suy ra $x=3y=3(-1)=-3$. Vậy $$S=2x+3y=2(-3)+3(-1)=-9.$$
% 	}
% \end{ex}
\Closesolutionfile{ans}
% \begin{indapan}{10}
% 	{ans/ans-1K2-3-Dang6}
% \end{indapan}
\begin{dang}{Bài toán thực tế}
	\textit{Bài toán lãi kép:} Một người gửi tiết kiệm vào ngân hàng một số tiền $A$ với lãi suất $r\%$ mỗi kì hạn. Số tiền lãi sẽ được nhập vào vốn ban đầu để tính lãi cho kì hạn tiếp theo. Hỏi sau $n$ kì hạn thì người đó có tất cả bao nhiêu tiền?\\
	\textit{Lời giải:} Gọi $u_n$ là số tiền người đó có sau $n$ kì hạn. Ta có:
	\begin{itemize}
		\item Số tiền người đó có sau kì hạn thứ nhất là: $u_1=A+A\cdot r\%=A\left(1+r\%\right)$.
		\item Số tiền người đó có sau $n$ kì hạn là: $u_n=u_{n-1}+u_{n-1}\cdot r\%=u_n\left(1+r\%\right)$.
	\end{itemize}
	Suy ra dãy số $(u_n)$ là một cấp số nhân với số hạng đầu $u_1=A\left(1+r\%\right)$ và công bội $q=1+r\%$.\\
	Vậy số tiền người đó có sau $n$ kì hạn là: \fbox{$u_n=A\left(1+r\%\right)^n$}.
\end{dang}
\subsubsection{Ví dụ minh hoạ}
\begin{vd}%[VD]%[DCHT Toán 11 - KNTT -Tên Huỳnh Thanh Chí]%[1K2K7-2]
	Trong một lọ nuôi cấy vi khuẩn, ban đầu có $ 5\ 000 $ con vi khuẩn và số lượng vi khuẩn tăng lên thêm $ 8\% $ mỗi giờ. Hỏi sau $ 5 $ giờ thì số lượng vi khuẩn là bao nhiêu?
	\dapso{}
	\loigiai{
	Ta có $ A=5\ 000 $ là số lượng vi khuẩn ban đầu, $ r=8\%=0{,}08 $ là tỉ lệ gia tăng vi khuẩn sau một giờ.
	\begin{itemize}
	\item Tại thời điểm sau $ 1 $ giờ: $ u_1=5000+5000\cdot 0{,}08= 5\ 000\cdot(1{,}08)$.
	\item Tại thời điểm sau $ n $ giờ: $ u_n=u_{n-1}+u_{n-1}\cdot0{,}08=u_{n-1}\cdot (1{,}08)$.
	\end{itemize}
	Do đó ta có thể nhận thấy rằng, số lượng vi khuẩn ở thời gian $ n $ giờ là một cấp số nhân có số hạng đầu $ u_1=5000.1,08 $ và công bội $ q=1{,}08 $.\\
	% Suy ra số hạng tổng quát $ u_n=5\ 000\cdot (1{,}08)^{n} $.\\
	Vậy số lượng vi khuẩn sau $ 5 $ giờ là $ u_5=5\ 000\cdot (1{,}08)^{5}\approx 7346 $ (vi khuẩn).
	}
\end{vd}
\begin{vd}%[VD]%[1K2K3-3]
	Người ta thiết kế một cái tháp gồm $10$ tầng theo cách: Diện tích bề mặt trên của mỗi tầng bằng nửa diện tích bề mặt trên của tầng ngay bên dưới và diện tích bề mặt của tầng 1 bằng nửa diện tích bề mặt đế tháp. Biết diện tích bề mặt đế tháp là $12\, 288$ m$^2$, tính diện tích bề mặt trên cùng của tháp.
	\loigiai{
		Gọi $S$ là diện tích mặt đế và $T_1, T_2, \ldots, T_{10}$ là diện tích bề mặt của tầng 1, tầng 2, \ldots, tầng 10.\\
		Khi đó, ta có
		\allowdisplaybreaks
		\begin{eqnarray*}
			&T_1&=\dfrac{1}{2}\cdot S;\\
			&T_n&=\dfrac{1}{2}\cdot T_{n-1}
		\end{eqnarray*}
		Suy ra $\left(T_n\right)$ là cấp số nhân có số hạng đầu $T_1=\dfrac{1}{2}\cdot 12288=6144$ m$^2$ và công bội $q=\dfrac{1}{2}$.\\
		Vậy diện tích bề mặt trên cùng của tháp là $T_{10}=\dfrac{1}{2^{10}}\cdot 12288=12$ m$^2$.
	}
\end{vd}
\begin{vd}%[TH] %[DCHT Toán 11 - KNTT - Dung Phuong] %[1K2B7-7]
	Dân số trung bình của Việt Nam năm $2020$ là $97{,}6$ triệu người, tỉ lệ tăng dân số là $1{,}14 \% /$năm.
	\begin{flushright}
		\textit{(Nguồn: Niên giám thống kê của Việt Nam năm 2020, NXB Thống kê, 2021)}
	\end{flushright}
	Giả sử tỉ lệ tăng dân số không đổi qua các năm.
	\begin{enumerate}
		\item Sau 1 năm, dân số của Việt Nam sẽ là bao nhiêu triệu người (làm tròn kết quả đến hàng phần mười)?
		\item Viết công thức tính dân số Việt Nam sau $n$ năm kể từ năm $2020$.
	\end{enumerate}
	\dapso{$\approx 98{,}7$ triệu người}
	\loigiai{
		\begin{enumerate}
			\item Sau 1 năm, dân số của Việt Nam sẽ là
			\allowdisplaybreaks
			\begin{eqnarray*}
				u_1&=&97{,}6+97{,}6 \cdot 0{,}0114=97{,}6 \cdot(1+0{,}0114)\\
				&=&97{,}6 \cdot 1{,}0114 \approx 98{,}7 (\text{triệu người}).
			\end{eqnarray*} 
			\item Gọi $u_n$ là dân số của Việt Nam sau $n$ năm.\\
			Do tỉ lệ tăng dân số hàng năm là $1{,}14 \%$ nên ta có
			\allowdisplaybreaks
			\begin{eqnarray*}
				u_n &=&u_{n-1}+u_{n-1} \cdot 0{,}0114=u_{n-1} \cdot(1+0{,}0114) \\
				&=&u_{n-1} \cdot 1{,}0114\ \text{với}\ n \geq 2.
			\end{eqnarray*}
			Do đó, $\left(u_n\right)$ là cấp số nhân có số hạng đầu $u_1=97{,}6\cdot 1{,}0114$, công bội $q=1{,}0114$.\\
			Vậy dân số của Việt Nam sau $n$ năm kể từ năm $2020$ là
			\[u_n=97{,}6 \cdot 1{,}0114 \cdot 1{,}0114^{n-1}=97{,}6 \cdot 1{,}0114^n\ (\text{triệu người}). \]
		\end{enumerate}
	}
\end{vd}

\begin{vd}%[TH] %[DCHT Toán 11 - KNTT - Dung Phuong] %[1K2B7-7]
	Bác Linh gửi vào ngân hàng $100$ triệu đồng tiền tiết kiệm với hình thức lãi kép, kì hạn 1 năm với lãi suất $6 \% /$năm. Viết công thức tính số tiền (cả gốc và lãi) mà bác Linh có được sau $n$ năm (giả sử lãi suất không thay đổi qua các năm).
	\dapso{$100 \cdot 1{,}06^{n-1}$ triệu đồng}
	\loigiai{
		Gọi $u_n$ là số tiền (cả gốc lẫn lãi) mà bác Linh có được sau $n$ năm.\\
		Do lãi suất 1 năm là $6\%$ nên ta có
		\allowdisplaybreaks
		\begin{eqnarray*}
			u_n &=&u_{n-1}+u_{n-1} \cdot 0{,}06=u_{n-1} \cdot(1+0{,}06) \\
			&=&u_{n-1} \cdot 1{,}06\ \text{với}\ n \geq 2.
		\end{eqnarray*}
		Do đó, $\left(u_n\right)$ là cấp số nhân có số hạng đầu $u_1=100\cdot 1{,}06$ (triệu đồng), công bội $q=1{,}06$.\\
		Vậy số tiền mà bác Linh có được sau $n$ năm là
		\[u_n=100 \cdot 1{,}06^{n}\ (\text{triệu đồng}). \]
	}
\end{vd}
\begin{vd}%[VD] %[DCHT Toán 11 - KNTT - Dung Phuong] %[1K2K7-7]
	Một hình vuông có cạnh $1$ đơn vị dài được chia thành chín hình vuông nhỏ hơn và hình vuông ở chính giữa được tô màu xanh như hình. Mỗi hình vuông nhỏ hơn lại được chia thành chín hình vuông con, và mỗi hình vuông con ở chính giữa lại được tô màu xanh. Nếu quá trình này được tiếp tục lặp lại năm lần, thì tổng diện tích các hình vuông được tô màu xanh là bao nhiêu?
	\dapso{$\dfrac{26281}{39366}$}
	\begin{center}
		\begin{tikzpicture}[>=stealth,thick,scale=0.7]
			\def\n{1}
			\def\a{3}
			\pgfmathsetmacro{\m}{int(3^(\n))}
			\def\hv#1{
				\ifnum#1>0
				\fill[blue!50] (-\a/3,-\a/3) rectangle (\a/3,\a/3);
				\pgfmathtruncatemacro{\k}{#1-1}
				\foreach \i in {0,...,3}{\begin{scope}[shift={(90*\i:2)},scale=1/3]\hv{\k}\end{scope}}
				\foreach \i in {0,...,3}{\begin{scope}[shift={(45+90*\i:{4/sqrt(2)})},scale=1/3]\hv{\k}\end{scope}}
				\fi
			}
			\draw(-\a,-\a) rectangle (\a,\a);
			\hv{\n}
			\foreach \i in {0,1,...,\m}{
				\draw[blue!50] 
				({-\a+2*\i *\a/\m},\a)--++(270:2*\a)
				(\a,{-\a+2*\i *\a/\m})--++(180:2*\a)
				;
			}
		\end{tikzpicture}
		\begin{tikzpicture}[>=stealth,thick,scale=0.7]
			\def\n{2}
			\def\a{3}
			\pgfmathsetmacro{\m}{int(3^(\n))}
			\def\hv#1{
				\ifnum#1>0
				\fill[blue!50] (-\a/3,-\a/3) rectangle (\a/3,\a/3);
				\pgfmathtruncatemacro{\k}{#1-1}
				\foreach \i in {0,...,3}{\begin{scope}[shift={(90*\i:2)},scale=1/3]\hv{\k}\end{scope}}
				\foreach \i in {0,...,3}{\begin{scope}[shift={(45+90*\i:{4/sqrt(2)})},scale=1/3]\hv{\k}\end{scope}}
				\fi
			}
			\draw(-\a,-\a) rectangle (\a,\a);
			\hv{\n}
			\foreach \i in {0,1,...,\m}{
				\draw[blue!50] 
				({-\a+2*\i *\a/\m},\a)--++(270:2*\a)
				(\a,{-\a+2*\i *\a/\m})--++(180:2*\a)
				;
			}
		\end{tikzpicture}
		\begin{tikzpicture}[>=stealth,thick,scale=0.7]
			\def\n{4}
			\def\a{3}
			\def\hv#1{
				\ifnum#1>0
				\fill[blue!50] (-\a/3,-\a/3) rectangle (\a/3,\a/3);
				\pgfmathtruncatemacro{\k}{#1-1}
				\foreach \i in {0,...,3}{\begin{scope}[shift={(90*\i:2)},scale=1/3]\hv{\k}\end{scope}}
				\foreach \i in {0,...,3}{\begin{scope}[shift={(45+90*\i:{4/sqrt(2)})},scale=1/3]\hv{\k}\end{scope}}
				\fi
			}
			\draw (-\a,-\a) rectangle (\a,\a);
			\hv{\n}
		\end{tikzpicture}
	\end{center}
	\loigiai{
		Lần phân chia thứ nhất, $1$ hình vuông thành $9$ hình vuông con, diện tích hình vuông tô màu xanh là $u_1=\dfrac{1}{9}$.\\
		Lần phân chia thứ hai, $8$ hình vuông thành $9$ hình vuông con, diện tích hình vuông tô màu xanh tăng thêm là $u_2=\dfrac{1}{9}\left(\dfrac{8}{9}\right)$.\\
		Lần phân chia thứ ba, $8^2$ hình vuông thành $9$ hình vuông con, diện tích hình vuông tô màu xanh tăng thêm là $u_3=\dfrac{1}{9}\left(\dfrac{8}{9}\right)^2$.\\
		Lần phân chia thứ tư, $8^3$ hình vuông thành $9$ hình vuông con, diện tích hình vuông tô màu xanh tăng thêm là $u_4=\dfrac{1}{9}\left(\dfrac{8}{9}\right)^3$.\\
		Lần phân chia thứ năm, $8^4$ hình vuông thành $9$ hình vuông con, diện tích hình vuông tô màu xanh tăng thêm là $u_5=\dfrac{1}{9}\left(\dfrac{8}{9}\right)^4$.\\
		Như vậy diện tích các hình vuông tăng thêm sau mỗi lần chia  tạo thành cấp số nhân có công bội là $q=\dfrac{8}{9}$, số hạng đầu là $u_1=\dfrac{1}{9}$.\\
		Do đó, tổng diện tích hình vuông tô màu xanh sau $5$ lần chia là\\
		\[u_1+u_2+u_3+u_4+u_5=\dfrac{1-q^5}{1-q}\cdot u_1=\dfrac{1-\left(\dfrac{8}{9}\right)^5}{1-\dfrac{8}{9}}\cdot \dfrac{1}{9}=\dfrac{26281}{39366}.\]	
		
	}
\end{vd}

\begin{vd}%[TH] %[DCHT Toán 11 - KNTT - Dung Phuong] %[1K2B7-7]
	Một khay nước có nhiệt độ $23^\circ$ được đặt vào ngăn đá của tủ lạnh. Biết sau mỗi giờ, nhiệt độ của nước giảm $20\%$. Tính nhiệt độ của khay nước đó sau $6$ giờ theo đơn vị độ $C$.
	\dapso{$ \approx 7,5^\circ $ gam}
	\loigiai{
		Nhiệt độ sau mỗi giờ của khay nước theo thứ tự lập thành cấp số nhân với $u_1=23.(1-20\%)$ và $q=(1-20\%)$.\\
		Ta có $u_6=u_1.q^5=23.(1-20\%)^6 \approx 7,5$.\\
		Nhiệt độ của khay nước sau $6$ giờ là $ \approx 6,0^\circ $.
	}
	
\end{vd}
\begin{vd}%[TH] %[DCHT Toán 11 - KNTT - Dung Phuong] %[1K2B7-7]
	Chu kì bán rã của nguyên tố phóng xạ poloni $210$ là $138$ ngày, nghĩa là sau $138$ ngày, khối lượng của nguyên tố đó chi còn một nửa (theo: https://vi.wikipedia.org/wiki/Poloni-210). Tính khối lượng còn lại của $20$ gam poloni $210$ sau:
	\begin{listEX}[2]
		\item [a)]  $690$ ngày;
		\item [b)] $7314$ ngày (khoảng $20$ năm).
	\end{listEX}
	\dapso{$\dfrac{20}{2^{53}}$ gam}
	\loigiai{
		\begin{listEX}[1]
			\item [a)] Ta có $\dfrac{690}{138}=5$ suy ra khối lượng còn lại sau 690 này là $\dfrac{20}{2^5}=0{,}625$ gam;
			\item [b)] Ta có $\dfrac{7314}{138}=53$ suy ra khối lượng còn lại sau 7314 này là $\dfrac{20}{2^{53}}$ gam.
		\end{listEX}
	}
\end{vd}	
\begin{vd}%[TH] %[DCHT Toán 11 - KNTT - Dung Phuong] %[1K2K7-7]
	Tế bào E.Coli trong điều kiện nuôi cấy thích hợp cứ $20$ phút lại phân đôi một lần. Hỏi sau $24$ giờ, tế bào ban đầu sẽ phân chia thành bao nhiếu tế bào?
	\dapso{$2^{72}$}
	\loigiai{
		Lần phân chia thứ nhất, $1$ tế bào thành $2$ tế bào, số tế bào lần $1$ phân chia là $u_1 = 2$.\\
		Lần phân chia thứ hai $2$, số tế bào lần $2$ phân chia là  $u_2=2\cdot 2 = u_1 \cdot 2$.\\
		Lần phân chia thứ $3$ có  $4$ tế bào phân chia, số tế bào lần $3$ phân chia là $u_3=2\cdot u_2$.\\
		Như vậy một tế bào phân đôi sẽ tạo thành cấp số nhân có công bội là $2$, số hạng đầu là $u_1=2$.\\
		Sau $n$ lần phân chia từ một tế bào phân được thành $u_n=2^{n-1}u_1$.\\
		Đổi $24$ giờ $=24 \cdot 60 =  72 \cdot 20$ (phút)  $\Rightarrow 24$ giờ gấp $72$ lần $20$ phút. \\
		Do đó, sau $24$ giờ số tế bào nhận được là $u_{72}=2^{71}\cdot 2 = 2^{72}$ (tế bào).
	}
\end{vd}

\subsubsection{Bài tập tự luận}
 


\begin{bt}%[TH] %[DCHT Toán 11 - KNTT - Dung Phuong] %[1K2B7-7]
	Một quốc gia có dân số năm 2011 là $P$ triệu người. Trong $10$ năm tiếp theo, mỗi năm dân số tăng $a \%$. Chứng minh rằng dân số các năm từ năm 2011 đến năm 2021 của quốc gia đó tạo thành cấp số nhân. Tìm công bội của cấp số nhân này.
	\loigiai{
		Coi ngày điều tra dân số năm 2011 và năm 2021 trùng nhau thì từ năm 2011 đến năm 2021 là 10 năm. Vậy dân số nước ta tính đến năm 2021 là 
		\[u_{10} = P\cdot \left(1+a\%\right)^{10}.\]
		Ta có \[u_{1} = P\cdot \left(1+a\%\right)^{1}.\]
		\[u_{2} = P\cdot \left(1+a\%\right)^{2}.\]
		Và công bội của cáp số nhân này là $\, \dfrac{u_2}{u_1} = q = \dfrac{P\cdot \left(1+a\%\right)^{2}}{P\cdot \left(1+a\%\right)^{1}} = 1+a\%.$
	}
\end{bt}
\begin{bt}%[TH] %[DCHT Toán 11 - KNTT - Dung Phuong] %[1K2B7-7]
	Vào năm 2020, dân số của một quốc gia là khoảng $97$ triệu người và tốc độ tăng trưởng dân số là $0{,}91 \%$. Nếu tốc độ tăng trưởng dân số này được giữ nguyên hằng năm, hãy ước tính dân số của quốc gia đó vào năm 2030.
	\dapso{$106{,}1973784$}
	\loigiai{
		Dân số năm 2021 tăng lên so với năm 2020 là $97 \cdot 0{,}91 \% $ triệu người.\\
		Dân số năm 2021 là 
		\begin{center}
			$97 + 97 \cdot 0{,}91 \% = 97\cdot (1+0{,}91 \%)$ triệu người.
		\end{center}
		Dân số năm 2022 tăng lên so với năm 2021 là $97\cdot (1+0{,}91 \%)\cdot 0{,}91 \% $ triệu người.\\
		Dân số năm 2022 là 
		\begin{center}
			$97\cdot (1+0{,}91 \%) + 97\cdot (1+0{,}91 \%) \cdot 0{,}91 \% = 97\cdot (1+0{,}91 \%)^2$ triệu người.
		\end{center}
		Dân số năm 2023 tăng lên so với năm 2021 là $97\cdot (1+0{,}91 \%)^2\cdot 0{,}91 \% $ triệu người.\\
		Dân số năm 2023 là
		\begin{center}
			$97\cdot (1+0{,}91 \%)^2 + 97\cdot (1+0{,}91 \%)^2\cdot 0{,}91 \% = 97\cdot (1+0{,}91 \%)^3$ triệu người.
		\end{center}
		Tương tự vậy ta có dân số năm 2030 là $97\cdot (1+0{,}91 \%)^{10} = 106{,}1973784$ triệu người.
	}
\end{bt}
\begin{bt}%[TH] %[DCHT Toán 11 - KNTT - Dung Phuong] %[1K2B7-7]
	Một tỉnh có $2$ triệu dân vào năm 2020 với tỉ lệ tăng dân số là $1$ \%/năm. Gọi $u_n$ là số dân của tỉnh đó sau $n$ năm. Giả sử tỉ lệ tăng dân số là không đổi.
	\begin{enumEX}[a)]{1} 
		\item Viết công thức tính số dân của tỉnh đó sau $n$ năm kể từ năm 2020.
		\item Tính số dân của tỉnh đó sau $10$ năm kể từ năm 2020.  
	\end{enumEX}
	\loigiai{
		\begin{enumEX}[a)]{1} 
			\item Với $u_n$ là số dân của tỉnh đó sau $n$ năm. \\
			Ta có $u_1=2 \cdot 1,01$ (triệu dân).\\
			$u_{n+1}=u_n+u_n\cdot 0{,}01 = 1{,}01u_n$. \\
			Do đó, $(u_n)$ là cấp số nhân với số hạng đầu $u_1=2 \cdot 1,01$ và công bội $q=1{,}01$. \\
			Vậy công thức tính số dân của tỉnh đó sau $n$ năm là $u_n=u_1q^{n-1}\Rightarrow u_n=2\cdot 1{,}01^{n}$.
			\item Số dân của tỉnh đó sau $10$ năm kể từ năm 2020 là $u_{10}=2\cdot 1{,}01^10 = 2{,}209$ (triệu dân).
		\end{enumEX}
	}
\end{bt}
\begin{bt}%[TH] %[DCHT Toán 11 - KNTT - Dung Phuong] %[1K2B7-7]
	Giả sử một thành phố có dân số năm 2022 là khoảng $2{,}1$ triệu người và tốc độ gia tăng dân số trung bình mỗi năm là $0{,}75 \%$.
	\begin{listEX}[1]
		\item [a)]  Dự đoán dân số của thành phố đó vào năm $2032$; \dapso{$\approx2262924$ (người)}
		\item [b)]  Nếu tốc độ gia tăng dân số vẫn giữ nguyên như trên thì uớc tính vào năm nào dân số của thành phố đó sẽ tăng gấp đôi so với năm $2022$?  \dapso{$2116$}	\end{listEX}
	
	\loigiai{
		\begin{listEX}
			\item [a)] 	Giả sử dân số năm $2022$ là $u_1=2{,}1\cdot 10^6$ thì dân số năm 
			$2023$ 
			là 
			$u_2=u_1+ 0{,}0075u_1=1{,}0075u_1$.\\
			Tương tự dân số năm $2024$ là $u_3=1{,}0075u_2$.\\
			Do đó dân số của thành phố qua các năm lập thành một cấp số nhân với 
			$u_1=2{,}1\cdot10^6$; $q=1{,}0075$.\\
			Vậy dân số năm $2032$ tương ứng với $u_{11}=u_1\cdot q^{10}=2,1\cdot 
			10^6\cdot1{,}0075^{10}\approx2262924$ (người).
			\item [b)] Giả sử đến năm thứ $n$ thì dân số gấp đôi năm $2022$. \\
			Suy ra 
			$u_n=2u_1 \Leftrightarrow q^{n-1}=2\Leftrightarrow  1{,}0075^{n-1}=2 
			\Leftrightarrow n \approx 93{,}7.$\\
			Vậy $94$ năm sau tức là năm $2116$ thì dân số thành phố sẽ gấp đôi năm $2022$.
	\end{listEX}}
\end{bt}
\begin{bt}%[TH] %[DCHT Toán 11 - KNTT - Dung Phuong] %[1K2B7-7]
	Giả sử anh Tuấn kí hợp đồng lao động trong $10$ năm với điều khoản về tiền lương như sau: Năm thứ nhất, tiền lương của anh Tuấn là $60$ triệu. Kể từ năm thứ hai trở đi, mỗi năm tiền lương của anh Tuấn được tăng lên $8 \%$. Tính tổng số tiền lương anh Tuấn lĩnh được trong $10$ năm đi làm (đơn vị: triệu đồng, làm tròn đến hàng phần nghìn).
	\dapso{$\approx 869{,}194$ triệu người}
	\loigiai{
		Gọi $u_n$ là số tiền lương (triệu đồng) anh Tuấn được lĩnh ở năm làm việc thứ $n$. Ta có: $u_1=60$;
		\[u_n=u_{n-1}+u_{n-1} \cdot 0{,}08=u_{n-1} \cdot(1+0{,}08)=u_{n-1} \cdot 1{,}08. \]
		Do đó, $\left(u_n\right)$ là cấp số nhân có số hạng đầu $u_1=60$, công bội $q=1{,}08$. Áp dụng công thức tính tổng $S_n$, ta có tổng số tiền lương anh Tuấn lĩnh được trong $10$ năm đi làm là
		\[S_{10}=\dfrac{60\cdot\left(1-1{,}08^{10}\right)}{1-1{,}08} \approx 869{,}194\ (\text{triệu người}). \]
	}
\end{bt}
\begin{bt}%[TH] %[DCHT Toán 11 - KNTT - Dung Phuong] %[1K2B7-7]
	Một công ty xây dựng mua một chiếc máy ủi với giá $3$ tỉ đồng. Cứ sau mỗi năm sử dụng, giá trị của chiếc máy ủi này lại giảm $20 \%$ so với giá trị của nó trong năm liền trước đó. Tìm giá trị còn lại của chiếc máy ủi đó sau $5$ năm sử dụng.
	\dapso{$983$ triệu đồng}
	\loigiai{
		Gọi $u_n$ là Giá trị của máy ủi sau $n$ sử dụng. \\
		Dãy số ($u_n$) là một cấp số nhân có $u_1=3.0,8$, $q=0{,}8$.\\
		Số hạng tổng quát của cấp số nhân này là $u_n=3\cdot 0{,}2^{n}$.\\
		Ta có $u_5=3\cdot 0{,}8^5=0{,}98304$.\\
		Tương ứng giá trị của chiếc máy ủi sau $5$ năm xấp xỉ $983$ triệu đồng.
	}
\end{bt}
\begin{bt}%[TH] %[DCHT Toán 11 - KNTT - Dung Phuong] %[1K2B7-7]
	Một gia đình mua một chiếc ô tô giá $800$ triệu đồng. Trung bình sau mỗi năm sử dụng, giá trị còn lại của ô tô giảm đi $4 \%$ (so với năm trước đó).
	\begin{enumEX}[a)]{1} 
		\item Viết công thức tính giá trị của ô tô sau $1$ năm, $2$ năm sử dụng.
		\item Viết công thức tính giá trị của ô tô sau $n$ năm sử dụng.
		\item Sau $10$ năm, giá trị của ô tô ước tính còn bao nhiêu triệu đồng? 
	\end{enumEX} 
	\dapso{$\approx 531{,}87$ triệu đồng}
	\loigiai{
		Gọi $u_n$ là giá trị còn lại của ô tô sau $n$ năm sử dụng. 
		\begin{enumEX}[a)]{1} 
			\item Giá trị của ô tô sau $1$ năm sử dụng là $u_1=800-800\cdot0{,}04=800\cdot0{,}96=768$ triệu đồng.\\
			Giá trị của ô tô sau $2$ năm sử dụng là $u_2=u_1-u_1\cdot0{,}04=u_1\cdot0{,}96=737{,}28$ triệu đồng.
			\item Ta có $u_n=u_{n-1}-u_{n-1}\cdot0{,}04=u_{n-1}\cdot0{,}96$. \\
			Do đó, $(u_n)$ là cấp số nhân với số hạng đầu $u_1=768$ và công bội $q=0{,}96$. \\
			Vậy sau $n$ năm sử dụng, giá trị còn lại của chiếc ô tô là $u_n=u_1q^{n-1}\Rightarrow u_n=768\cdot0{,}96^{n-1}$.
			\item Sau $10$ năm, ước tính giá trị của ô tô còn lại là $u_{10}=768\cdot0{,}96^9\approx 531{,}87$ triệu đồng.
		\end{enumEX} 
	}
\end{bt}
% \begin{bt} [VD] %[DCHT Toán 11 - KNTT - Dung Phuong] %[1K2K7-7]
% 	Ông An vay ngân hàng $1$ tỉ đồng với lãi suất $12\%/$năm. Ông đã trả nợ theo cách: Bắt đầu từ tháng thứ nhất sau khi vay, cuối tháng ông trả ngân hàng số tiền là $a$ (đồng) và đã trả hết nợ sau đúng $2$ năm kể từ ngày vay. Hỏi số tiền mỗi tháng mà ông An phải trả là bao nhiêu đồng (làm tròn kết quả đến hàng nghìn)?
% 	\dapso{$47073500$}
% 	\loigiai{
% 		Do lãi suất là $12\%$/năm tương đương với lãi là $1\%$/tháng.\\
% 		Sau $1$ tháng, ông An còn nợ là: $10^9.(1+1\%)-a=10^9.(1,01)-S_1$.\\
% 		Sau $2$ tháng, ông An còn nợ là: $10^9.(1.01)^2-a.(1.01)-a=10^9(1,01)^2-S_2$.\\
% 		Sau $3$ tháng, ông An còn nợ là: $10^9.(1.01)^3-a(1.01)^2-a(1.01)-a=10^9.(1.01)^3-S_3$.\\
% 		Sau $24$ tháng, ông An còn nợ là: $10^9.(1.01)^{24}-S_{24}=0$.\\
% 		Do đó $S_{24}=10^9.(1.01)^{24}  \Leftrightarrow a.\dfrac{1-(1.01)^{24}}{1-(1.01)}=10^9.(1.01)^{24} \Leftrightarrow a =\dfrac{10^9.(1.01)^{24}.0.01}{(1.01)^{24}-1}\approx 47073472,22$.\\
% 		Vậy mỗi tháng ông An phải trả $47073500$.
% 	}
% \end{bt}
\begin{bt}%[VD] %[DCHT Toán 11 - KNTT - Dung Phuong] %[1K2K7-7]
	\immini
	{
		Một người nhảy bungee (một trò chơi mạo hiểm mà người chơi nhảy từ một nơi có địa thế cao xuống với dây đai an toàn buộc xung quanh người) từ một cây cầu và căng một sợi dây dài $100$ m. Sau mỗi lần rơi xuống, nhờ sự đàn hồi của dây, người nhảy được kéo lên một quãng đường có độ dài bằng $75$\% so với lần rơi trước đó và lại bị rơi xuống đúng bằng quãng đường vừa được kéo lên. Tính tổng quãng đường người đó đi được sau $10$ lần kéo lên và lại rơi xuống. 
	}
	{
		\begin{tikzpicture}[xscale=.5, font=\small, line join=round, line cap=round, >=stealth,yscale=1]
			\def\a{-0.12} % Hệ số a phải khác 0
			\def\b{0.86}
			\def\c{0}
			\def\m{-0.12} % Hệ số a phải khác 0
			\def\n{0.86}
			\def\p{-0.3}
			\clip (-2,-2)rectangle(9,2);
			\fill[red!40] (-2,1.8)--(9,1.8)--(9,1.55)--(-2,1.55)--cycle;%Mặt phẳng của cây cầu
			\fill[green!50] (-2,1.55)--(0,0)--(1.5,-1)--(-1,-2)--(-2,-2)--cycle;
			\fill[green!50] (9,1.55)--(7,0)--(5.5,-1)--(8,-2)--(9,-2)--cycle;
			\fill[blue!40] (-1,-2)--(8,-2)--(5.5,-1)--(1.5,-1)--cycle;
			\draw[color=blue!50,line width=2pt,<->] (2,-.5)--(2,1.5)node[left,midway]{$100$ m};	
			\draw[color=blue!50,line width=2pt,<->] (5,-.5)--(5,1)node[midway,right]{$0{,}75\cdot100$ m};
			\draw(3.5,1.55)--(3.5,-.25)node[rotate=200]{\faChild};
			\draw[color=white,<->] (3.5,1.8)--(3.5,1.55); 
		\end{tikzpicture}
	}
	\dapso{$\approx666{,}2 \text{ m}$}
	\loigiai{
		Gọi $u_n$ là quãng đường người đó được kéo lên ở lần thứ $n$ được kéo lên và lại rơi xuống (đơn vị tính: mét). \\
		Ta có $u_1=0{,}75\cdot100=100\cdot1{,}5=75$ m và $u_n=0{,}75\cdot u_{n-1}$. \\
		Vậy $(u_n)$ là cấp số nhân với số hạng đầu $u_1=75$ và công bội $q=0{,}75$. \\
		Tổng quãng đường người đó đi được sau $10$ lần kéo lên và lại rơi xuống là 
		$$\begin{aligned}
			S&=100+2u_1+2u_2+\cdots+2u_{10}\\
			&=100+2S_{10}
			=100+2\cdot\dfrac{75\left(1-0{,}75^{10}\right)}{1-0{,}75}\\
			&\approx666{,}2 \text{ m}.
		\end{aligned}$$ 
	}
\end{bt}
\begin{bt} [TH] %[DCHT Toán 11 - KNTT - Dung Phuong] %[1K2B7-7]
	Một cái tháp có $11$ tầng. Diện tích của mặt sàn tầng $2$ bằng nửa diện tích của mặt đáy tháp và diện tích của mặt sàn mỗi tầng bằng nửa diện tích của mặt sàn mỗi tầng ngay bên dưới. Biết mặt đáy tháp có diện tích là $12 288m^2$. Tính diện tích của mặt sàn tầng trên cùng của tháp theo đơn vị mét vuông.
	\dapso{$12m^2$}
	\loigiai{ (Lưu ý: Một số nơi xem tầng 1 là tầng trệt. Nên bài toán này giống bài toán tháp 10 tầng ở phần trên)
		Do diện tích của mặt sàn tính từ tầng một lập thành một cấp số nhân với $u_2=\dfrac{1}{2}.12288=6144$ và $q=\dfrac{1}{2}$.\\
		Ta có $\heva{u_2&=6144 \\ q&=\dfrac{1}{2}}  \Leftrightarrow \heva{u_1&=12288 \\ q&=\dfrac{1}{2}}$.\\
		Ta có $u_{11}=u_1.q^{10}=12288.\dfrac{1}{2^{10}}=12m^2$.
		Vậy diện tích của mặt sàn tầng trên cùng là	$12m^2$.
	}
\end{bt}

\begin{bt}%[TH]%[DCHT Toán 11 - KNTT - Dung Phuong]%[1K2B7-7]
	\immini{Cho hình vuông $C_1$ có cạnh bằng $4$. Người ta chia mỗi cạnh hình vuông thành bốn phần bằng nhau và nối các điểm chia một cách thích hợp để có hình vuông $C_2$ . Từ hình vuông $C_2$ lại làm tiếp tục như trên để có hình vuông $C_3$. Cứ tiếp tục quá trình như trên, ta nhận được dãy các hình vuông $C_1, C_2, C_3, \ldots , C_n, \ldots$ Gọi $a_n$ là độ dài cạnh hình vuông $C_n$. Chứng minh rằng dãy số $\left(a_n\right)$ là cấp số nhân.}{
		\begin{tikzpicture}[scale=.8]
			\def\a{2}  %cạnh hình vuông
			\def\t{.7}  % tỷ lệ điểm cho vòng lặp tiếp
			\path 
			(-\a,-\a) coordinate (A1)
			(-\a,\a) coordinate (B1)
			(\a,\a) coordinate (C1)
			(\a,-\a) coordinate (D1);
			\draw (A1)--(B1)--(C1)--(D1)--cycle;
			\foreach \i[count=\j from 2] in {1,...,10}
			\draw
			(barycentric cs:A\i=\t,B\i=1-\t) coordinate (A\j)--
			(barycentric cs:B\i=\t,C\i=1-\t) coordinate (B\j)--
			(barycentric cs:C\i=\t,D\i=1-\t) coordinate (C\j)--
			(barycentric cs:D\i=\t,A\i=1-\t) coordinate (D\j)--cycle
			;	
			% \node at (0,-2.2) [below]{\textit{Hình 4}};
		\end{tikzpicture}	
	}
	\loigiai{
		\immini{Gọi cạnh một hình vuông thứ $n$, $n+1$ lần lượt là $a_n, a_{n+1}$.\\
			Do $MN=\sqrt{MB^2+BN^2}=\sqrt{\left(\dfrac{AB}{4}\right)^2+\left(\dfrac{3AB}{4}\right)^2 }=AB\cdot\dfrac{\sqrt{10}}{4}$.\\
			Nên ta có cạnh hình vuông thứ $n+1$ là:\\ $a_{n+1}=a_n.\dfrac{\sqrt{10}}{4}$.\\
			Vậy dãy số $\left(a_n\right)$ là cấp số nhân.	
		}{
			\begin{tikzpicture}[scale=0.8,>=stealth, font=\footnotesize, line join=round, line cap=round]
				\path
				(0,0) coordinate (A)
				(4,0) coordinate (B)
				(4,4) coordinate (C)
				(0,4) coordinate (D)
				($(A)!0.75!(B)$) coordinate (M)
				($(B)!0.75!(C)$) coordinate (N)
				($(C)!0.75!(D)$) coordinate (P)
				($(D)!0.75!(A)$) coordinate (Q)
				;
				\draw (A)--(B)--(C)--(D)--cycle (M)--(N)--(P)--(Q)--cycle;
				\node at ($(A)!0.5!(B)$)[below]{$a_n$};
				\node at ($(M)!0.5!(N)$)[left]{$a_{n+1}$};
				\foreach \p/\q in {A/180,B/0,C/0,D/180,M/-90,N/0,P/90,Q/180}
				\fill[black] (\p) circle (1.0pt) ($(\p)+(\q:2.5mm)$) node{$\p$};
		\end{tikzpicture}
	}
	}
\end{bt}

\begin{bt}%[TH] %[DCHT Toán 11 - KNTT - Dung Phuong] %[1K2B7-7]
	Một cây đàn organ có tần số âm thanh các phím liên tiếp tạo thành một cấp số nhân. Cho biết tần số phím La trung là $400$ Hz và tần số của phím La cao cao hơn $12$ phím là $800$ Hz (nguồn: https://vi.wikipedia.org/wikiOrgan). Tìm công bội của cấp số nhân nói trên (làm tròn kết quả đến hàng phần nghìn).
	\dapso{$q = \pm \sqrt[12]{2}$}
	\loigiai{
		Theo đề ta có $\heva{&u_1=400\\&u_{13}=800} \Leftrightarrow \heva{&u_1=400\\&u_1q^{12}=800} \Rightarrow q^{12} = 2 \Rightarrow q = \pm \sqrt[12]{2}$.
	}
\end{bt}


\begin{bt}%[VD]%[DCHT Toán 11 - KNTT - Dung Phuong] %[1K2K7-7] 
	Một loại thuốc được dùng mỗi ngày một lần. Lúc đầu nồng độ thuốc trong máu của bệnh nhân tăng nhanh, nhưng mỗi liều kế tiếp có tác dụng ít hơn liều trước đó. Lượng thuốc trong máu ở ngày thứ nhất là $50 \,\mathrm{mg}$, và mỗi ngày sau đó giảm chỉ còn một nửa so với ngày kề trước đó. Tính tổng lượng thuốc (tính bằng $\mathrm{mg}$) trong máu của bệnh nhân sau khi dùng thuốc $10$ ngày liên tiếp.
	\dapso{$99{,}902$ mg.}
	\loigiai{
		Gọi $u_n$ là giá trị của lượng thuốc trong máu của bệnh nhân trong ngày thứ $n$. \\
		Dãy số này là một cấp số nhân có $u_1=50$, $q=\dfrac{1}{2}$.\\
		Tổng của $n$ số hạng đầu tiên của cấp số nhân là $S_n=u_1\dfrac{1-q^n}{1-q}$.\\
		Theo bài toán, ta có $S_{10}=50 \cdot\dfrac{1-\left(\dfrac{1}{2}\right)^{10}}{1-\dfrac{1}{2}} \approx 99{,}902$.\\
		Vậy tổng lượng thuốc trong máu của bệnh nhân sau khi dùng thuốc $10$ ngày liên tiếp là $99{,}902$ mg.	
	}
\end{bt}
\subsubsection{Câu hỏi trắc nghiệm}
\Opensolutionfile{ans}[ans/ans-1K2-2-Dang7]
\begin{ex}%[1K2K7-7]
	\immini{
		Cho hình vuông có cạnh là $1$. Nối các trung điểm của hình vuông trên ta được một hình vuông có diện tích $S_1$, tiếp tục quá trình trên với các hình vuông với diện tích là $S_2$; $S_3$; $\ldots ;S_n;\ldots$. Tính tổng vô hạn $S_1+ S_2+ S_3+\cdots+S_n+\cdots$.
		\choice
		{$2$}
		{$\dfrac{1}{2}$}
		{\True $1$}
		{$\dfrac{3}{2}$}
	}
	{\hspace*{1 cm}
		\begin{tikzpicture}[scale=0.8,line cap=round,line join=round]
			\path
			(0,0) coordinate (A)
			(4,0) coordinate (B)
			(0,4) coordinate (D);			
			\coordinate (C) at ($(B)-(A)+(D)$);
			\coordinate (H) at ($(A)!0.5!(B)$);
			\coordinate (I) at ($(A)!0.5!(D)$);
			\coordinate (J) at ($(D)!0.5!(C)$);
			\coordinate (K) at ($(B)!0.5!(C)$);
			\coordinate (E) at ($(I)!0.5!(H)$);
			\coordinate (F) at ($(H)!0.5!(K)$);
			\coordinate (G) at ($(K)!0.5!(J)$);
			\coordinate (O) at ($(J)!0.5!(I)$);
			\coordinate (M) at ($(E)!0.5!(F)$);
			\coordinate (N) at ($(F)!0.5!(G)$);
			\coordinate (P) at ($(G)!0.5!(O)$);
			\coordinate (Q) at ($(O)!0.5!(E)$);
			\draw (A)--(B)--(C)--(D)--cycle (I)--(H)--(K)--(J)--cycle
			(E)--(F)--(G)--(O)--cycle (M)--(N)--(P)--(Q)--cycle;
			\foreach \p in {A,B,C,D,E,F,G,H,I,J,K,M,N,P,Q,O}
			\fill[black] (\p) circle (1.0pt);			
		\end{tikzpicture}
	}
	\loigiai{
		Ta có $S_1=\dfrac{1}{2}$, $S_2=\dfrac{1}{4}$, $S_3=\dfrac{1}{8},\cdots  S_n=\dfrac{1}{2^n},\ldots$ tạo thành $1$ cấp số nhân với công bội $q=\dfrac{1}{2}<1$. \\
		Vậy $S_1+ S_2+ S_3+\cdots+S_n+\cdots=\dfrac{\dfrac{1}{2}}{1-\dfrac{1}{2}}=1$.
	}
\end{ex}
\begin{ex}%[1K2K7-7]
	Cho $n$ là số nguyên dương và $n$ tam giác $A_1B_1C_1,A_2B_2C_2,\ldots,A_nB_nC_n$, trong đó các điểm lần ${A}_{i+1},{B}_{i+1},{C}_{i+1}$ lượt nằm trên các cạnh $B_iC_i,A_iC_i,A_iB_i(i=1,2,\ldots,n-1)$ sao cho ${A}_{i+1}C_i=3{A}_{i+1}B_i,{B}_{i+1}A_i=3{B}_{i+1}C_i,{C}_{i+1}B_i=3{C}_{i+1}A_i$. Gọi $S$ là tổng tất cả các diện tích của tam giác $A_1B_1C_1,A_2B_2C_2,\ldots,A_nB_nC_n$ biết rằng tam giác $A_1B_1C_1$ có diện tích bằng $\dfrac{9}{16}$. Tìm số nguyên dương sao cho $S=\dfrac{{16}^{29}-7^{29}}{{16}^{29}}$.
	\choice
	{$n=28$}
	{$n=2018$}
	{$n=30$}
	{\True $n=29$}
	\loigiai{
		Gọi $S_i(i=1,2,3,...,n)$ là diện tích của $\Delta A_iB_iC_i$. Ta có $\dfrac{S_{A_1B_2C_2}}{S_{A_1B_1C_1}}=\dfrac{A_1B_2}{A_1C_1}\cdot \dfrac{A_1C_2}{A_1B_1}=\dfrac{1}{4}\cdot \dfrac{3}{4}=\dfrac{3}{16}$. Tương tự, ta có $\dfrac{S_{A_2B_1C_2}}{S_{A_1B_1C_1}}=\dfrac{S_{A_2B_2C_1}}{S_{A_1B_1C_1}}=\dfrac{3}{16}$. Do đó $\dfrac{S_{A_2B_2C_2}}{S_{A_1B_1C_1}}=1-3\cdot \dfrac{3}{16}=\dfrac{7}{16}\Rightarrow S_2=\dfrac{7}{16}S_1$.\\
		Tương tự, ta có ${S}_{i+1}=\dfrac{7}{16}S_i,i=1,2,\ldots,n$.
		Khi đó $$S=S_1\left[1+\dfrac{7}{16}+\cdots+{\left(\dfrac{7}{16}\right)}^{n-1}\right]=\dfrac{9}{16}\cdot \dfrac{1-{\left(\dfrac{7}{16}\right)}^n}{1-\dfrac{7}{16}}=1-{\left(\dfrac{7}{16}\right)}^n.$$
		Theo giả thiết ta có $1-{\left(\dfrac{7}{16}\right)}^n=1-{\left(\dfrac{7}{16}\right)}^{29}\Leftrightarrow n=29$.}
\end{ex}
\begin{ex}%[1K2K7-7]
	Người ta thiết kế một cái tháp gồm $11$ tầng. Diện tích bề mặt trên của mỗi tầng bằng nửa diện của mặt trên tầng ngay bên dưới và diện tích tầng $1$ bằng nửa diện tích của đế tháp. Biết đế tháp có diện tích là $12288\, \mathrm{m}^2$. Tính diện tích mặt trên cùng.
	\choice
	{$12\, \mathrm{m}^2$}
	{\True $6\, \mathrm{m}^2$}
	{$10\, \mathrm{m}^2$}
	{$8\, \mathrm{m}^2$}
	\loigiai{
		Gọi $S_{i}$ là diện tích của tầng thứ $i$ với $i = 1,2,\ldots,11$.\\
		Do giả thiết suy ra $S_{i + 1} = \dfrac{1}{2}S_{i}$ với $i = 1,2,\ldots,10$.\\
		Do đó $\left\{S_{i}\right\}$ là một cấp số nhân với công bội $q = \dfrac{1}{2}$. Do đó  $S_{11} = \dfrac{1}{2^{10}}S_{1} = \dfrac{1}{2^{11}}\cdot 12288 = 6\left(\mathrm{m}^2\right)$.
	}
\end{ex}

\begin{ex}%[1K2K7-7]
	Cho tứ giác $ABCD$ có bốn góc tạo thành cấp số nhân có công bội $ q=2 $. Góc có số đo nhỏ nhất trong bốn góc đó là
	\choice
	{\True $ 24^\circ $}
	{$ 1^\circ $}
	{$ 12^\circ $}
	{$ 30^\circ $}
	\loigiai{
		Gọi số đo bốn góc của tứ giác $ ABCD $ là $ x $, $ 2x $, $ 4x $, $ 8x $.
		\\ Có $ x+2x+4x+8x=360 \Leftrightarrow 15x=360 \Leftrightarrow x=24 $.}
\end{ex}

\begin{ex}%[1K2K7-7]
	Một du khách vào chuồng đua ngựa đặt cược, lần đầu tiên đặt $20000$ đồng, mỗi lần sau tiền đặt gấp đôi lần tiền đặt cược trước. Người đó thua lần $9$ liên tiếp và thắng ở lần thứ $10$. Hỏi du khách đó thắng hay thua bao nhiêu tiền?
	\choice
	{\True Thắng $20000$ đồng}
	{Thua $40000$ đồng}
	{Hòa vốn}
	{Thua $20000$ đồng}
	\loigiai{
		Số tiền đặt cược lần thứ $n$ là $u_n=u_1\cdot 2^{n-1}$ với $u_1=20000$. \\
		Ta có: $u_{10}-\displaystyle\sum_{n=1}^9 u_1\cdot 2^{n-1}=20000\cdot 2^9-\displaystyle\sum_{n=1}^9 20000\cdot 2^{n-1}=20000$. \\
		Vậy du khách thắng $20000$ đồng.
	}
\end{ex}

% \begin{ex}%[1K2K7-7]
% 	Một người gửi tiết kiệm vào ngân hàng với lãi suất $7{,}5$ \%/năm. Biết rằng nếu không rút tiền ra khỏi ngân hàng thì cứ sau mỗi năm số tiền lãi sẽ được nhập vào vốn để tính lãi cho năm tiếp theo. Hỏi sau ít nhất bao nhiêu năm người đó thu được (cả số tiền gửi ban đầu và lãi) gấp đôi số tiền đã gửi, giả định trong khoảng thời gian này lãi suất không thay đổi và người đó không rút tiền ra?
% 	\choice
% 	{$12$ năm}
% 	{$11$ năm}
% 	{\True $10$ năm}
% 	{$9$ năm}
% 	\loigiai{
% 		Áp dụng công thức: $S_n=A(1+r)^n \Rightarrow n=\log_{(1+r)}\left(\dfrac{S_n}{A}\right) \Rightarrow n=\log_{\left(1+7{,}5\%\right)}(2)\approx 9{,}6$.}
% \end{ex}

\begin{ex}%[1K2K7-7]
	Cho tam giác $ ABC $ cân tại $ A $ có cạnh đáy $ BC $,  đường cao $ AH $ và cạnh bên $ AB $ theo thứ tự đó lập thành cấp số nhân công bội $ q $. Giá trị của $ q $ là
	\choice
	{$ q=\dfrac{1}{2}\sqrt{\sqrt{2}+1} $ }
	{$ q=\sqrt{2}+1 $ }
	{$ q=\sqrt{2(\sqrt{2}+1)} $}
	{\True $ q=\dfrac{1}{2}\sqrt{2(\sqrt{2}+1)} $ }
	\loigiai{
		Giả sử $ BC=u_1 $, $ AH=u_1\cdot q $ và $ AB=u_1\cdot q^2 $ với $ u_1> 0, q> 0 $.\\
		Do $ \triangle ABC $ cân tại $ A $ suy ra
		\begin{align*}
			AB^2=AH^2+\dfrac{BC^2}{4}\Leftrightarrow
			& u_1^2\cdot q^4=\dfrac{u_1^2}{4}+u_1^2\cdot q^2\\
			\Leftrightarrow & 4q^4-4q^2-1=0\\
			\Leftrightarrow & q^2=\dfrac{1\pm \sqrt{2}}{2}.
		\end{align*}
		Kết hợp với điều kiện bài toán ta có $ q=\sqrt{\dfrac{1+ \sqrt{2}}{2}}=\dfrac{1}{2}\sqrt{2(\sqrt{2}+1)} $.
	}
\end{ex}
\begin{ex}%[1K2K7-7]
	Giả sử một người đi làm được lĩnh lương khởi điểm là $2.000.000$ đồng/tháng. Cứ $3$ năm người ấy lại được tăng lương một lần với mức tăng bằng $7\%$ của tháng trước đó. Hỏi sau $36$ năm làm việc người ấy lĩnh được tất cả bao nhiêu tiền?
	\choice
	{\True $ 1.287.968.492 $ đồng}
	{$ 10.721.769.110 $ đồng}
	{$ 7{,}068289036\cdot 10^8 $ đồng}
	{$ 429.322.830{,}5 $ đồng}
	\loigiai{
		Ta có $36$ năm tương ứng với $12$ kỳ lương; mỗi kỳ lương có $36$ tháng và kỳ sau tăng $7\%$ so với kỳ trước. Do đó tổng số tiền mỗi kỳ lương là một cấp số nhân với $u_1=36\times 2=72$ (triệu đồng) và công bội $q=1{,}07$.\\
		Vậy tổng số tiền sau $36$ năm là $T=\dfrac{72\cdot \left[(1{,}07)^{12}-1\right]}{1{,}07-1}=1287{,}968492$ (triệu đồng).
	}
\end{ex}

\begin{ex}%[1K2K7-7]
	Từ độ cao $55{,}8$ (mét) của tháp nghiên Pisa nước Italia người ta thả một quả bóng cao su chạm xuống đất. Giả sử mỗi lần chạm đất bóng lại nảy lên độ cao bằng $\dfrac{1}{10}$ độ cao mà bóng đạt trước đó. Tổng độ dài hành trình (mét) của bóng được thả từ lúc ban đầu cho đến khi nó nằm yên trên mặt đất thuộc khoảng nào trong các khoảng sau đây?
	\choice
	{$(69;72)$}
	{$(60;63)$}
	{\True $(67;69)$}
	{$(64;66)$}
	\loigiai{
		Đặt $u_1=55{,}8$ (mét) là quãng đường bóng rơi khi thả xuống, $u_{n+1}=\dfrac{1}{10^{n}} u_1, n\ge 1$ là quãng đường bóng rơi sau lần nảy lên thứ $n$. \\
		Ta có $(u_n)$ là dãy cấp số nhân với $u_1=55{,}8$ và công bội $q=\dfrac{1}{10}$.\\
		Suy ra tổng quãng đường quả bóng rơi xuống là $\displaystyle \lim \limits_{n \rightarrow +\infty} u_1 \cdot \dfrac{1-q^n}{1-q}=\displaystyle \lim \limits_{n \rightarrow +\infty}55{,}8\cdot\dfrac{1-\left( \dfrac{1}{10}\right)^n}{1-\dfrac{1}{10}}=62 $.\\
		Ngoài ra ta còn phải tính tổng quãng đường mà bóng nảy lên. Ta có tổng quãng đường bóng nảy lên bằng tổng quãng đường rơi của bóng trừ đi quãng đường thả rơi xuống.\\
		Vậy tổng quãng đường hành trình của quả bóng là $62+62-55{,}8=68{,}2$ (mét).
	}
\end{ex}

\begin{ex}%[1K2K7-7]
	Một gia đình lập kế hoạch tiết kiệm như sau: Họ lập một sổ tiết kiệm tại một ngân hàng và cứ đầu mỗi tháng họ gửi
	vào sổ tiết kiệm đó $15$ triệu đồng. Giả sử lãi suất tiền gửi không đổi là $0{,}6$ \%/tháng và tiền gửi được tính lãi theo hình thức lãi
	kép. Hỏi sau $3$ năm gia đình đó tiết kiệm được số tiền gần nhất với con số nào dười đây?
	\choice
	{$543240000$ đồng}
	{$589269000$ đồng}
	{$669763000$ đồng}
	{\True $604359000$ đồng}
	\loigiai{
		Gọi $S_0$ triệu đồng là số tiền gia đình đó định kỳ gửi tiết kiệm vào đầu hằng tháng, $r$ là lãi suất tiền gửi hằng tháng. Ta có $S_0=15$ triệu đồng, $r=0{,}6$
		\%/tháng.\\
		Gọi $S_i$, $i=\overline{1,n}$ là số tiền trong sổ tiết kiệm cuối tháng thứ $i$.\\
		Ta có \begin{itemize}
			\item $S_1=S_0+S_0\cdot r=S_0(1+r)$,
			\item  $S_2=\left[ S_0+S_0(1+r)\right]+\left[ S_0+S_0(1+r)\right]r=S_0 (1+r)+S_0(1+r)^2$,
			\item  $\begin{aligned}[t]
				S_3=&\ \left[S_0+S_0(1+r)+S_0(1+r)^2 \right] +\left[S_0+S_0(1+r)+S_0(1+r)^2 \right]r\\
				=&\ S_0(1+r)+S_0(1+r)^2+S_0(1+r)^3,\end{aligned}$,
			\item \ldots
			\item$\begin{aligned}[t]S_n=&\ S_0(1+r)+S_0(1+r)^2+S_0(1+r)^3+\cdots +S_0(1+r)^n\\=&\ S_0\left[ (1+r)+(1+r)^2+(1+r)^3+\cdots+(1+r)^n\right]\\
				=&\  S_0(1+r)\cdot \dfrac{(1+r)^{n}-1}{(1+r)-1}=S_0(1+r)\cdot \dfrac{(1+r)^{n}-1}{r}.
			\end{aligned}$
		\end{itemize}
		Vậy sau $3$ năm, tức cuối tháng thứ $36$ thì gia đình tiết kiệm được số tiền là
		\[S_{36}=15\cdot 10^6(1+0{,}6\cdot 10^{-2})\cdot \dfrac{(1+0{,}6\cdot 10^{-2})^{36}-1}{0{,}6\cdot 10^{-2}}=604358538{,}2 \ \text{đồng}.\]
	}
\end{ex}
\Closesolutionfile{ans}
% \begin{indapan}{10}
% 	{ans/ans-1K2-2-Dang7}
% \end{indapan}

%%Ôn tập chương II
\setcounter{dang}{0}
\setcounter{ex}{0}
\setcounter{bt}{0}
\setcounter{vd}{0}
\section*{Ôn tập chương 2}
\Opensolutionfile{ans}[ans/ans-1K2-Ontapchuong2]
\begin{ex}%[1K2Y5-2]
	Cho dãy số $\left(u_n\right)$, biết $u_n=\left(-1\right)^n.2n$. Mệnh đề nào sau đây sai?
	\choice
	{$u_1=-2$}
	{$u_2=4$}
	{$u_3=-6$}
	{\True $u_4=-8$}
	\loigiai{
		Thay trực tiếp vào kiểm tra, ta có
		\begin{eqnarray*}
			u_1&=&-2.1=-2\\
			u_2&=&(-1)^2.2.2=4\\
			u_3&=&(-1)^3.2.3=-6\\
			u_4&=&(-1)^4.2.4=8.
		\end{eqnarray*}
	}
\end{ex}
\begin{ex}%[1K2Y5-2]
	Cho dãy số $\left(u_n\right)$, biết $u_n=\left(-1\right)^n.\dfrac{2^n}{n}$. Tìm số hạng $u_3$.
	\choice
	{$u_3=\dfrac{8}{3}$}
	{$u_3=2$}
	{$u_3=-2$}
	{\True $u_3=-\dfrac{8}{3}$}
	\loigiai{
		Thay trực tiếp vào kiểm tra, ta có
		\begin{center}
			$u_3=(-1)^3.\dfrac{2^3}{3}=-\dfrac{8}{3}$.
		\end{center}
	}
\end{ex}
\begin{ex}%[1K2Y5-2]
	Cho dãy số $\left(u_n\right)$, biết $u_n=\dfrac{2n+5}{5n-4}$. Số $\dfrac{7}{12}$ là số hạng thứ mấy của dãy số?
	\choice
	{\True $8$}
	{$6$}
	{$9$}
	{$10$}
\end{ex}
\loigiai{
	Ta có
	\allowdisplaybreaks
	\begin{eqnarray*}
		&&u_n=\dfrac{2n+5}{5n-4}\\
		&\Leftrightarrow&\dfrac{7}{12}=	\dfrac{2n+5}{5n-4}\\
		&\Leftrightarrow&24n+60=35n-28\\
		&\Leftrightarrow&11n=88\\
		&\Leftrightarrow&n=8.
	\end{eqnarray*}
	Vậy số $\dfrac{7}{12}$ là số hạng thứ 8.
}
\begin{ex}%[1K2Y5-2]
	Cho dãy số $\left(u_n\right)$, biết $u_n=2^n$. Tìm số hạng $u_{n+1}$.
	\choice
	{\True $u_{n+1}=2^n.2$}
	{$u_{n+1}=2^n+1$}
	{$u_{n+1}=2\left(n+1\right)$}
	{$u_{n+1}=2^n+2$}
	\loigiai{
		Ta có
	}
	\loigiai{
		Thay $n$ bằng $n+1$ trong công thức $u_n$ ta được
		\allowdisplaybreaks
		\begin{eqnarray*}
			u_{n+1}&=&2^{n+1}\\
			& =&2.2^n.
		\end{eqnarray*}
	}
\end{ex}
\begin{ex}%[1K2B5-2]
	Cho dãy số $\left(u_n\right)$, biết $u_n=5^{n+1}$. Tìm số hạng $u_{n-1}$.
	\choice
	{$u_{n-1}=5^{n-1}$}
	{\True $u_{n-1}=5^{n}$}
	{$u_{n-1}=5.5^{n+1}$}
	{$u_{n-1}=5.5^{n-1}$}
	\loigiai{
		Thay $n$ bằng $n-1$ trong công thức $u_n$ ta được
		\allowdisplaybreaks
		\begin{eqnarray*}
			u_{n-1}& = &5^{n-1+1}\\
			& = &5^n.
		\end{eqnarray*}
	}
\end{ex}
\begin{ex}%[1K2Y5-1]
	Cho dãy số có các số hạng đầu là $-2;0;2;4;6;...$. Số hạng tổng quát của dãy số này là công thức nào dưới đây?
	\choice
	{$u_n=-2n$}
	{$u_n=n-2$}
	{$u_n=-2\left(n+1\right)$}
	{\True $u_n=2n-4$}
	\loigiai{
		Kiểm tra $u_1=-2$ ta loại các đáp án B và C. Tương tự kiểm tra $u_2=0$ ta loại đáp án A.
	}
\end{ex}
\begin{ex}%[1K2B5-1]
	Cho dãy số $\left(u_n\right)$, được xác định $\heva{&u_1=\dfrac{1}{2}\\&u_{n+1}=u_n-2}$. Số hạng tổng quát $u_n$ của dãy số là số hạng nào dưới đây?
	\choice
	{$u_n=\dfrac{1}{2}+2\left(n-1\right)$}
	{\True $u_n=\dfrac{1}{2}-2\left(n-1\right)$}
	{$u_n=\dfrac{1}{2}-2n$}
	{$u_n=\dfrac{1}{2}+2n$}
	\loigiai{
		Ta có
		\begin{center}
			$\heva{&u_1=\dfrac{1}{2}\\&u_{n+1}=u_n-2}\Rightarrow\heva{&u_1=\dfrac{1}{2}\\&u_2=-\dfrac{3}{2}\\&u_3=-\dfrac{7}{2}}$
		\end{center} 
		Ta thấy chỉ có đáp án B đều thoả mãn.
	}
\end{ex}
\begin{ex}%[1K2B5-1]
	Cho dãy số $\left(u_n\right)$, được xác định $\heva{&u_1=-2\\&u_{n+1}=-2-\dfrac{1}{u_n}}$. Số hạng tổng quát $u_n$ của dãy số là số hạng nào dưới đây?
	\choice
	{$u_n=\dfrac{-n+1}{n}$}
	{$u_n=\dfrac{n+1}{n}$}
	{\True $u_n=-\dfrac{n+1}{n}$}
	{$u_n=-\dfrac{n}{n+1}$}
	\loigiai{
		Ta có
		\begin{center}
			$\heva{&u_1=-2\\&u_{n+1}=-2-\dfrac{1}{u_n}}\Rightarrow\heva{&u_1=-2\\&u_2=-\dfrac{3}{2}}$
		\end{center}
		Ta thấy chỉ có đáp án C thoả mãn.
	}
\end{ex}
\begin{ex}%[1K2Y5-2]
	Cho cấp số cộng có số hạng đầu $u_1=-\dfrac{1}{2}$, công sai $d=\dfrac{1}{2}$. Năm số hạng liên tiếp đầu tiên của cấp số này là.
	\choice
	{$-\dfrac{1}{2};0;1;\dfrac{1}{2};1$}
	{$-\dfrac{1}{2};0;\dfrac{1}{2};0;\dfrac{1}{2}$}
	{$\dfrac{1}{2};1;\dfrac{3}{2};2;\dfrac{5}{2}$}
	{\True $-\dfrac{1}{2};0;\dfrac{1}{2};1;\dfrac{3}{2}$}
	\loigiai{
		Ta dùng công thức tổng quát $u_n=u_1+(n-1)d=-\dfrac{1}{2}+(n-1)\dfrac{1}{2}=-1+\dfrac{n}{2}$ để tính các số hạng của một cấp số cộng. Ta có
		\begin{center}
			$u_1=-\dfrac{1}{2},u_2=0,u_3=\dfrac{1}{2},u_4=1,u_5=\dfrac{3}{2}$.
		\end{center}
	}
\end{ex}
\begin{ex}%[1K2B6-3]
	Viết ba số hạng xen giữa các số $2$ và $22$ để được một cấp số cộng có năm số hạng.
	\choice
	{\True 
		$7;12;17$}
	{$6;10;14$}
	{$8;13;18$}
	{$6;12;18$}
	\loigiai{
		Giữa $2$ và $22$ có thêm ba số hạng nữa lập thành cấp số cộng, xem như ta có một cấp số cộng có năm số hạng với $u_1=22;u_5=22$, ta cần tìm $u_2,u_3,u_4$. Ta có
		\begin{eqnarray*}
			&&u_5=u_1+4d\\
			&\Leftrightarrow&d=\dfrac{u_5-u_1}{4}\\&\Leftrightarrow&d=5\\
			&\Rightarrow&\heva{&u_2=7\\&u_3=12\\&u_4=17}.
		\end{eqnarray*}
	}
\end{ex}
\begin{ex}%[1K2K6-3]
	Biết các số $C_n^1;C_n^2;C_n^3$ theo thứ tự lập thành một cấp số cộng với $n>3$. Tìm $n$.
	\choice
	{$n=5$}
	{\True $n=7$}
	{$n=9$}
	{$n=11$}
	\loigiai{
		Ba số $C_n^1;C_n^2;C_n^3$ theo thứ tự $u_1;u_2;u_3$ lập thành một cấp số cộng nên
		\begin{eqnarray*}
			&&u_1+u_3=2u_2\\
			&\Leftrightarrow&C_n^1+C_n^3=2C_n^2\\
			&\Leftrightarrow&n+\dfrac{(n-2)(n-1)n}{6}=2.\dfrac{(n-1)n}{2}\\
			&\Leftrightarrow&1+\dfrac{n^2-3n+2}{6}=n-1\\
			&\Leftrightarrow&n^2-9n+14\\
			&\Leftrightarrow&\hoac{&n=2\\&n=7}	
		\end{eqnarray*}
		Kết hợp với điều kiện $n>3$, do đó $n=7$ thoả mãn yêu cầu bài toán.
	}
\end{ex}
\begin{ex}%[1K2B6-2]
	Cho cấp số cộng $\left(u_n\right)$ có các số hạng đầu lần lượt là $5; 9; 13; 17;...$. Tìm số hạng tổng quát $u_n$ của cấp số cộng.
	\choice
	{$u_n=5n+1$}
	{$u_n=5n-1$}
	{\True $u_n=4n+1$}
	{$u_n=4n-1$}
	\loigiai{
		Các số $5; 9; 13; 17 th;...$ theo thứ tự đó lập thành cấp số cộng $\left(u_n\right)$ nên
		\begin{center}
			$\heva{&u_1=5\\&d=u_2-u_1=4}\Rightarrow u_n=u_1+(n-1)d=5+4(n-1)=4n+1$.
		\end{center}
	}
\end{ex}
\begin{ex}%[1K2K6-1]
	Cho cấp số cộng $\left(u_n\right)$ có $u_1=3$ và $d=\dfrac{1}{2}$. Khẳng định nào sau đây đúng?
	\choice
	{$u_n=-3+\dfrac{1}{2}(n+1)$}
	{$u_n=-3+\dfrac{1}{2}n-1$}
	{\True $u_n=-3+\dfrac{1}{2}(n-1)$}
	{$u_n=-3+\dfrac{1}{4}(n-1)$}
	\loigiai{
		Ta có
		\begin{center}
			$\heva{&u_1=-3\\&d=\dfrac{1}{2}}\Rightarrow u_n=u_1+(n-1)d=-3+\dfrac{1}{2}(n-1)$.
		\end{center}
	}
\end{ex}
\begin{ex}%[1K2K6-1]
	Trong các dãy số được cho dưới đây, dãy số nào là cấp số cộng?
	\choice
	{\True $u_7=7-3n$}
	{$u_7=7-3^n$}
	{$u_7=\dfrac{7}{3n}$}
	{$u_7=7.3^n$}
	\loigiai{
		Dãy $\left(u_n\right)$ là cấp số cộng khi và chỉ khi $u_n=an+b$ với $a,b$ là hằng số.
	}
\end{ex}
\begin{ex}%[1K2B6-3]
	Cho cấp số cộng $\left(u_n\right)$ có $u_1=-5$ và $d=3$. Mệnh đề nào sau đây đúng?
	\choice
	{$u_{15}=34$}
	{$u_{15}=45$}
	{\True $u_{13}=31$}
	{$u_{10}=35$}
	\loigiai{
		Ta có
		\begin{center}
			$\heva{&u_1=-5\\&d=3}\Rightarrow u_n=3n-8\Rightarrow\heva{&u_{15}=37\\&u_{13}=31\\&u_{10}=22}$.
		\end{center}
	}
\end{ex}
\begin{ex}%[1K2B6-3]
	Cho cấp số cộng $\left(u_n\right)$ có $d=-2$ và $S_8=72$. Tìm số hạng đầu tiên $u_1$.
	\choice
	{\True $u_1=16$}
	{$u_1=-16$}
	{$u_1=\dfrac{1}{16}$}
	{$u_1=-\dfrac{1}{16}$}
	\loigiai{
		Ta có $\heva{&d=-2\\&S_8=72}\Leftrightarrow\heva{&d=-2\\&8u_1+\dfrac{8.7}{2}d=72}\Rightarrow 8u_1+28.(-2)=72\Leftrightarrow u_1=16$.
	}
\end{ex}
\begin{ex}%[1K2K6-3]
	Một cấp số cộng có số hạng đầu là $1$, công sai là $4$, tổng của n số hạng đầu là $561$. Khi đó số
	hạng thứ $n$ của cấp số cộng đó là $u_n$ có giá trị là bao nhiêu?
	\choice
	{$u_n=57$}
	{$u_n=61$}
	{\True $u_n=65$}
	{$u_n=69$}
	\loigiai{
		Ta có $\heva{&u_1=1,d=4\\&S_n=561}\Leftrightarrow\heva{&u_=1,d=4\\&nu_1+\dfrac{n(n-1)}{2}d=561}\Rightarrow n+\dfrac{n^2-n}{2}.4=561\Leftrightarrow 2n^2-n-561=0\Leftrightarrow n=17$.\\
		Từ đây suy ra $u_{17}=u_1+16d=1+16.4=65$.
	}
\end{ex}
\begin{ex}%[1K2K6-5]
	Tổng $n$ số hạng đầu tiên của một cấp số cộng là $S_n=\dfrac{3n^2-19n}{4}$ với $n\in\mathbb{N}^*$. Tìm số hạng đầu
	tiên $u_1$ và công sai $d$ của cấp số cộng đã cho.
	\choice
	{$u_1=2,d=-\dfrac{1}{2}$}
	{\True $u_1=-4,d=\dfrac{3}{2}$}
	{$u_1=-\dfrac{3}{2},d=-2$}
	{$u_1=\dfrac{5}{2},d=\dfrac{1}{2}$}
	\loigiai{
		Ta có $\dfrac{3n^2-19n}{4}=\dfrac{3}{4}n^2-\dfrac{19n}{4}=S_n=nu_1+\dfrac{n^2-n}{2}d=\dfrac{d}{2}n^2+\left(u_1-\dfrac{d}{2}\right)n$.\\
		Đồng nhất hai vế của phương trình, ta có $\heva{&\dfrac{d}{2}=\dfrac{3}{4}\\&u_1-\dfrac{d}{2}=-\dfrac{19}{4}}\Leftrightarrow\heva{&u_1=-4\\&d=\dfrac{3}{2}}$.
	}
\end{ex}
\begin{ex}%[1K2K6-3]
	Cho cấp số cộng $\left(u_n\right)$ có $u_2=2001$ và $u_5=1995$. Khi đó $u_{1001}$ bằng.
	\choice
	{$u_{1001}=4005$}
	{$u_{1001}=4003$}
	{\True $u_{1001}=3$}
	{$u_{1001}=1$}
	\loigiai{
		Ta có $\heva{&u_2=2001\\&u_5=1995}\Leftrightarrow\heva{&u_1+d=2001\\&u_1+4d=1995}\Leftrightarrow \heva{&u_1=2003\\&d=-2}\Rightarrow u_{1001}=u_1+1000d=3$.
	}
\end{ex}
\begin{ex}%[1K2B6-1]
	Cho cấp số cộng $\left(u_n\right)$ biết $u_n=-1,u_{n+1}=8$. Tính công sai $d$ của cấp số cộng đó.
	\choice
	{$d=-9$}
	{$d=7$}
	{$d=-7$}
	{\True $d=9$}
	\loigiai{
		Ta có $d=u_{n+1}-u_n=8-(-1)=9$.
	}
\end{ex}
\begin{ex}%[1K2K6-5]
	Cho cấp số cộng $\left(u_n\right)$ thỏa mãn $u_2+u_{23}=60$. Tính tổng $S_24$ của $24$ số hạng đầu tiên của
	cấp số cộng đã cho.
	\choice
	{$S_{24}=60$}
	{$S_{24}=120$}
	{\True $S_{24}=720$}
	{$S_{24}=1440$}
	\loigiai{
		Ta có $u_2+u_{23}=60\Leftrightarrow u_1+d+u_1+22d=60\Leftrightarrow 2u_1+23d=60$.\\
		Khi đó $S_{24}=\dfrac{24}{2}\left(u_1+u_{24}\right)=12\left(u_1+u_1+23d\right)=12.60=720$.
	}
\end{ex}
\begin{ex}%[1K2K6-1]
	Một cấp số cộng có $6$ số hạng. Biết rằng tổng của số hạng đầu và số hạng cuối bằng $17$, tổng
	của số hạng thứ hai và số hạng thứ tư bằng $14$. Tìm công sai $d$ của câp số cộng đã cho.
	\choice
	{$d=2$}
	{\True $d=-3$}
	{$d=4$}
	{$d=5$}
	\loigiai{
		Ta có $\heva{&u_1+u_6=17\\&u_2+u_4=14}\Leftrightarrow\heva{&2u_1+5d=17\\&2u_1+6d=14}\Leftrightarrow\heva{&u_1=16\\&d=-3}$.
	}
\end{ex}
\begin{ex}%[1K2K6-1]
	Cho cấp số cộng $\left(u_n\right)$ thỏa mãn $\heva{&u_7-u_3=8\\&u_2u_7=75}$. Tìm công sai $d$ của cấp số cộng đã cho.
	\choice
	{$d=\dfrac{1}{2}$}
	{$d=\dfrac{1}{3}$}
	{\True $d=2$}
	{$d=3$}
	\loigiai{
		Ta có $\heva{&u_7-u_3=8\\&u_2u_7=75}\Leftrightarrow\heva{&u_1+6d-u_1-2d=8\\&(u_1+d)(u_1+6d)=75}\Leftrightarrow\heva{&d=2\\&(u_1+2)(u_1+12)=75}$.
	}
\end{ex}
\begin{ex}%[1K2K6-3]
	Ba góc của một tam giác vuông tạo thành cấp số cộng. Hai góc nhọn của tam giác có số đo
	(độ) là
	\choice
	{$20^\circ$ và $70^\circ$}
	{$45^\circ$ và $45^\circ$}
	{$20^\circ$ và $45^\circ$}
	{\True $30^\circ$ và $60^\circ$}
	\loigiai{
		Ba góc $A,B,C$ của một tam giác vuông theo thứ tự đó $(A<B<C)$ lập thánh cấp số cộng nên $C=90,C+A=2B$.\\
		Ta có $\heva{&A+B+C=180\\&A+C=2B\\&C=90}\Leftrightarrow\heva{&A=30\\&B=60\\&C+90}$.
	}
\end{ex}
\begin{ex}%[1K2K6-3]
	Một tam giác vuông có chu vi bằng $3$ và độ dài các cạnh lập thành một cấp số cộng. Độ dài các
	cạnh của tam giác đó là
	\choice
	{$\dfrac{1}{2};1;\dfrac{3}{2}$}
	{$\dfrac{1}{3};1;\dfrac{5}{3}$}
	{\True $\dfrac{3}{4};1;\dfrac{5}{4}$}
	{$\dfrac{1}{4};1;\dfrac{7}{4}$}
	\loigiai{
		Ba cạnh $a,b,c,(a<b<c)$ của một tam giác theo thứ tự đó lập thành một cấp số cộng.\\
		Ta có $\heva{&a^2+b^2=c^2\\&a+b+c=3\\&a+c=2b}\Leftrightarrow\heva{&a^2+b^2=c^2\\&3b=3\\&a+c=2b}\Leftrightarrow\heva{&a^2+b^2=c^2\\&b=1\\&a=2-c}$.\\
		Từ đây suy ra $a^2+b^2=c^2\Rightarrow (2-c)^2+1=c^2\Leftrightarrow c=\dfrac{5}{4}\Leftrightarrow\heva{&a=\dfrac{3}{4}\\&b=1\\&c=\dfrac{5}{4}}$.
	}
\end{ex}
\begin{ex}%[1K2K6-6]
	Một rạp hát có $30$ dãy ghế, dãy đầu tiên có $25$ ghế. Mỗi dãy sau có hơn dãy trước $3$ ghế. Hỏi rạp
	hát có tất cả bao nhiêu ghế?
	\choice
	{$1635$}
	{$1792$}
	{\True $2055$}
	{$3125$}
	\loigiai{
		Số ghế của mỗi dãy (bắt đầu từ dãy đầu tiên) theo thứ tự đó lập thành một cấp số cộng có $30$ số hạng có công sai $d=3$ và $u_1=25$.\\
		Tổng số ghế là $S_{30}=30u_1+\dfrac{30.29}{2}d=2055$.
	}
\end{ex}
\begin{ex}%[1K2K6-6]
	Người ta trồng $3003$ cây theo một hình tam giác như sau: hàng thứ nhất trồng $1$ cây, hàng thứ hai trồng $2$ cây, hàng thứ ba trồng $3$ cây,... .Hỏi có tất cả bao nhiêu hàng cây?
	\choice
	{$73$}
	{$75$}
	{\True $77$}
	{$79$}
	\loigiai{
		Số cây mỗi hàng (bắt đầu từ hàng thứ nhất) lập thành một cấp số cộng $(u_n)$ có $u_1=1,d=1$. Giả sử có $n$ hàng cây thì $u_1+u_2+...+u_n=S_n$.\\
		Ta có $S_n=1.n+\dfrac{n(n-1)}{2}.1=3003\Leftrightarrow n=77$.
	}
\end{ex}
\begin{ex}%[1K2G6-6]
	Một chiếc đồng hồ đánh chuông, kể từ thời điểm $0$ (giờ) thì sau mỗi giờ thì số tiếng chuông được đánh đúng bằng số giờ mà đồng hồ chỉ tại thời điểm đánh chuông. Hỏi một ngày đồng hồ đó đánh bao nhiêu tiếng chuông?
	\choice
	{$78$}
	{$156$}
	{\True $300$}
	{$48$}
	\loigiai{
		Kể từ lúc $1$ (giờ) đến $24$ (giờ) số tiếng chuông được đánh lập thành cấp số cộng có $24$ số hạng với $u_1=1$, công sai $d=1$. Vậy số tiếng chuông được đánh trong $1$ ngày là $S_{24}=1.24+\dfrac{24.23}{2}.1=300$.
	}
\end{ex}
\begin{ex}%[1K2G6-6]
	Trên một bàn cờ có nhiều ô vuông, người ta đặt $7$ hạt dẻ vào ô đầu tiên, sau đó đặt tiếp vào ô thứ
	hai số hạt nhiều hơn ô thứ nhất là $5$, tiếp tục đặt vào ô thứ ba số hạt nhiều hơn ô thứ hai là $5$,...
	và cứ thế tiếp tục đến ô thứ $n$. Biết rằng đặt hết số ô trên bàn cờ người ta phải sử dụng $25450$
	hạt. Hỏi bàn cờ đó có bao nhiêu ô vuông?
	\choice
	{$98$}
	{\True $100$}
	{$102$}
	{$104$}
	\loigiai{
		Số hạt dẻ trên mỗi ô (bắt đầu từ ô thứ nhất) theo thứ tự đó lập thành cấp số cộng $(u_n)$ có $u_1=7,d=5$. Gọi $n$ là số ô trên bàn cờ thì $u_1+u_2+...+u_n=S_n$.\\
		Ta có $S_n=25450\Leftrightarrow 7n+\dfrac{n(n-1)}{2}.7=25450\Leftrightarrow n=100$.
	}
\end{ex}
\begin{ex}%[1K2G6-6]
	Một gia đình cần khoan một cái giếng để lấy nước. Họ thuê một đội khoan giếng nước đến để khoan giếng nước. Biết giá của mét khoan đầu tiên là $80.000$ đồng, kể từ mét khoan thứ $2$ giá của mỗi mét khoan tăng thêm $5000$ đồng so với giá của mét khoan trước đó. Biết cần phải khoan sâu xuống $50$ mét mới có nước. Vậy hỏi phải trả bao nhiêu tiền để khoan cái giếng đó?
	\choice
	{$5.250.000$ đồng}
	{\True $10.125.000$ đồng}
	{$4.00.000$ đồng}
	{$4.245.000$ đồng}
	\loigiai{
		Giá tiền khoang mỗi mét (bắt đầu từ mét đầu tiên) lập thành cấp số cộng $(u_n)$ có $u_1=80000,d=5000$. Do cần khoang $50$ mét nên tổng số tiền cần trả là $S_{50}=80000.50+\dfrac{50.49}{2}.5000=10125000$.
	}
\end{ex}
\begin{ex}%[1K2Y7-3]
	Một cấp số nhân có hai số hạng liên tiếp là $16$ và $36$. Số hạng tiếp theo là
	\choice
	{$720$}
	{\True$81$}
	{$64$}
	{$56$}
	\loigiai{
		Ta có cấp số nhân $(u_n)$ có $\heva{&u_n=36\\&u_{n+1}=36}\Rightarrow q=\dfrac{u_{n+1}}{u_n}=\dfrac{9}{4}$. Từ đây suy ra $u_{n+2}=u_{n+1}.q=36.\dfrac{9}{4}=81$.
	}
\end{ex}
\begin{ex}%[1K2B7-3]
	Tìm x để các số $2;8;x;128$ theo thứ tự đó lập thành một cấp số nhân.
	\choice
	{$x=14$}
	{\True 
		$x=32$}
	{$x=64$}
	{$x=68$}
	\loigiai{
		Cấp số nhân$ 2;8;x;128$ theo thứ tự đó sẽ là $u_1,u_2,u_3,u_4$.\\
		Ta có $\heva{&\dfrac{u_2}{u_1}=\dfrac{u_3}{u_2}\\&\dfrac{u_3}{u_2}=\dfrac{u_4}{u_3}}\Leftrightarrow\heva{&\dfrac{8}{2}=\dfrac{x}{8}\\&\dfrac{128}{x}=\dfrac{x}{8}}\Leftrightarrow\heva{&x=32\\&x^2=1024}\Rightarrow x=32$.
	}
\end{ex}
\begin{ex}%[1K2K7-3]
	Tìm tất cả giá trị của $x$ để ba số $2x-11;x;2x+1$ theo thứ tự đó lập thành một cấp số nhân.
	\choice
	{\True $x=\pm\dfrac{1}{\sqrt{3}}$}
	{$x=\pm\dfrac{1}{3}$}
	{$x=\pm\sqrt{3}$}
	{$x=\pm3$}
	\loigiai{
		Cấp số nhân $2x-1;x;2x+1$, suy ra $(2x-1)(2x+1)=x^2\Leftrightarrow x=\pm \dfrac{1}{\sqrt{3}}$.
	}
\end{ex}
\begin{ex}%[1K2K7-3]
	Với giá trị $x,y$ nào dưới đây thì các số hạng lần lượt là $-2;x;-18;y$ theo thứ tự đó lập thành cấp số nhân?
	\choice
	{$\heva{&x=6\\&y=-54}$}
	{$\heva{&x=-10\\&y=-26}$}
	{\True $\heva{&x=-6\\&y=-54}$}
	{$\heva{&x=-6\\&y=54}$}
	\loigiai{
		Cấp số nhân $-2;x;-18;y$, suy ra $\heva{&\dfrac{x}{-2}=\dfrac{-18}{x}\\&\dfrac{-18}{x}=\dfrac{y}{-18}}\Leftrightarrow\heva{&x=\pm6\\& y=\pm 54}$. Vậy $(x,y)=(6;54)$ hoặc $(x;y)=(-6;-54)$.
	}
\end{ex}
\begin{ex}%[1K2K7-3]
	Hai số hạng đầu của của một cấp số nhân là $2x+1$ và $4x^2-1$. Số hạng thứ ba của cấp số nhân là.
	\choice
	{$2x-1$}
	{$2x+1$}
	{\True $8x^3-4x^2-2x+1$}
	{$8x^3+4x^2-2x-1$}
	\loigiai{
		Công bội của cấp số nhân là $q=\dfrac{4x^2-1}{2x+1}=2x-1$. Vậy số hạng thứ ba của cấp số nhân là $(4x^2-1)(2x-1)=8x^3-4x^2-2x+1$.
	}
\end{ex}
\begin{ex}%[1K2B7-1]
	Trong các dãy số $(u_n)$ cho bởi số hạng tổng quát nu sau, dãy số nào là một cấp số nhân
	\choice
	{\True 
		$u_n=\dfrac{1}{3^{n-2}}$}
	{$u_n=\dfrac{1}{3^{n}}-1$}
	{$u_n=n+\dfrac{1}{3}$}
	{$u_n=n^2-\dfrac{1}{3}$}
	\loigiai{
		Dãy $u_n=\dfrac{1}{3^{n-2}}=3\left(\dfrac{1}{3}\right)^{n-1}$ là cấp số nhân có $u_1=3,q=\dfrac{1}{3}$.
	}
\end{ex}
\begin{ex}%[1K2B7-1]
	Một cấp số nhân có $6$ số hạng, số hạng đầu bằng $2$ và số hạng thứ sáu bằng $486$. Tìm công bội $q$ của cấp số nhân đã cho.
	\choice
	{\True $q=3$}
	{$q=-3$}
	{$q=2$}
	{$q=-2$}
	\loigiai{
		Ta có $\heva{&u_1=2\\&u_6=486}\Rightarrow u_6=u_1q^5\Leftrightarrow 486=2.q^5\Leftrightarrow q=3$.
	}
\end{ex}
\begin{ex}%[1K2B7-1]
	Cho cấp số nhân $\left(u_n\right)$ có $u_1=-3$ và $q=\dfrac{2}{3}$ Mệnh đề nào sau đây đúng.
	\choice
	{$u_5=-\dfrac{27}{16}$}
	{\True $u_5=-\dfrac{16}{27}$}
	{$u_5=\dfrac{16}{27}$}
	{$u_5=\dfrac{27}{16}$}
	\loigiai{
		Ta có $\heva{&u_1=-3\\&q=\dfrac{2}{3}}\Rightarrow u_5=u_1.q^4=-3.\left(\dfrac{2}{3}\right)^4=-\dfrac{16}{27}$.
	}
\end{ex}
\begin{ex}%[1K2K7-3]
	Cho cấp số nhân $\left(u_n\right)$ có $u_1=3$ và $q=-2$. Số $192$ là số hạng thứ mấy của cấp số nhân đã cho.
	\choice
	{$5$}
	{$6$}
	{\True $7$}
	{Không là số hạng của cấp số đã cho}
	\loigiai{
		Ta có $u_n=u_1.q^{n-1}\Leftrightarrow 192=3.(-2)^{n-1}\Leftrightarrow n=7$.
	}
\end{ex}
\begin{ex}%[1K2K7-3]
	Một cấp số nhân có công bội bằng $3$ và số hạng đầu bằng $5$. Biết số hạng chính giữa là $32805$. Hỏi cấp số nhân đã cho có bao nhiêu số hạng?
	\choice
	{$18$}
	{\True $17$}
	{$16$}
	{$9$}
	\loigiai{
		Ta có $u_n=u_1.q^{n-1}\Leftrightarrow 32805=3.5^{n-1}\Leftrightarrow n=9$. Vậy $u_9$ là số hạng chính giữa của cấp số nhân, nên cấp số nhân đã cho có $17$ số hạng.
	}
\end{ex}
\begin{ex}%[1K2K7-5]
	Cho cấp số nhân $\left(u_n\right)$ có $u_1=-3$ và $q=-2$. Tính tổng $10$ số hạng đầu tiên của cấp số nhân đã cho.
	\choice
	{$S_{10}=-511$}
	{$S_{10}=-1025$}
	{$S_{10}=1025$}
	{\True $S_{10}=1023$}
	\loigiai{
		Ta có $\heva{&u_1=-3\\&q=-2}\Rightarrow S_{10}=u_1.\dfrac{q^{n}-1}{q-1}=(-3).\dfrac{(-2)^{10}-1}{-2-1}=1023$.
	}
\end{ex}
\begin{ex}%[1K2G7-5]
	Cho cấp số nhân có các số hạng lần lượt là $1;4;16;64;...$. Gọi $S_n$ là tổng của $n$ số hạng đầu tiên của cấp số nhân đó. Mệnh đề nào sau đây đúng?
	\choice
	{$S_n=4^{n-1}$}
	{$S_n=\dfrac{n\left(1+4^{n-1}\right)}{2}$}
	{\True $S_n=\dfrac{4^n-1}{3}$}
	{$S_n=\dfrac{4\left(4^n-1\right)}{3}$}
	\loigiai{
		Ta có $\heva{&u_1=-3\\&q=4}\Rightarrow S_n=u_1.\dfrac{q^{n}-1}{q-1}=\dfrac{4^n-1}{3}$.
	}
\end{ex}
\begin{ex}%[1K2G7-1]
	Số hạng thứ hai, số hạng đầu và số hạng thứ ba của một cấp số cộng với công sai khác $0$ theo thứ tự đó lập thành một cấp số nhân với công bội $q$. Tìm $q$.
	\choice
	{$q=2$}
	{\True $q=-2$}
	{$q=-\dfrac{3}{2}$}
	{$q=\dfrac{3}{2}$}
	\loigiai{
		Giả sử ba số hạng $a;b;c$ lập thành cấp số cộng thỏa yêu cầu, khi đó $b;a;c$ theo thứ tự đó lập thành cấp số nhân công bội $q$. Ta có $\heva{&a+c=2b\\&a=bq\\&c=bq^2}\Rightarrow bq+bq^2=2b\Leftrightarrow\heva{&b=0\\&q^2+q-2=0}$.\\
		Nếu $b=0\Rightarrow a=b=c=0$ nên $a;b;c$ là cấp số cộng công sai $d=0$ (vô lí).\\
		Nếu $q^2+q-2=0\Leftrightarrow\hoac{&q=1\\&q=-2}$. Nếu $q=1\Rightarrow a=b=c$ (vô lí), do đó $q=-2$.
	}
\end{ex}
\begin{ex}%[1K2G7-1]
	Cho bố số $a,b,c,d$ biết rằng $a,b,c$ theo thứ tự đó lập thành một cấp số nhân công bội $q>1$,
	còn $b,c,d$ theo thứ tự đó lập thành cấp số cộng. Tìm $q$ biết rằng $a+d=14$ và $b+c=12$.
	\choice
	{$q=\dfrac{18+\sqrt{73}}{24}$}
	{\True $q=\dfrac{19+\sqrt{73}}{24}$}
	{$q=\dfrac{20+\sqrt{73}}{24}$}
	{$q=\dfrac{21+\sqrt{73}}{24}$}
	\loigiai{
		Giả sử $a,b,c$ lập thành cấp số cộng công bội $q$. Khi đó theo giả thiết ta có\\
		$\heva{&b=aq\\&c=aq^2\\&b+d=2c\\&a+d=14\\&c+d=12}\Rightarrow\heva{&aq+d=aq^2,&(1)\\&a+d=14,&(2)\\&a\left(q+q^2\right)=12,&(3)}$.\\
		Nếu $q=0\Rightarrow b=c=d=0$ (vô lý).\\
		Nếu $q=-1\Rightarrow b=-a=-c\Rightarrow b+c=0$ (vô lý).\\
		Vậy $q\ne 0,q\ne -1$, từ $(2)$ và $(3)$, ta có $d=14-a$ và $a=\dfrac{12}{q+q^2}$, thay vào $(1)$, ta được\\
		$\dfrac{12q}{q+q^2}+\dfrac{14q^2+14q-12}{q+q^2}=\dfrac{24q^3}{q+q^2}\Leftrightarrow 12q^3-7q^2-13q+6=0\Leftrightarrow q=\dfrac{19\pm \sqrt{73}}{24}$.\\
		Mà $q>1$ nên $q=\dfrac{19+\sqrt{73}}{24}$.
	}
\end{ex}
\begin{ex}%[1K2G7-5]
	Gọi $S=1+11+111+\cdots+111\ldots1$ ($n$ số $1$) thì $S$ nhận giá trị nào sau đây?
	\choice
	{$S=\dfrac{10^n-1}{81}$}
	{$S=10\cdot\dfrac{10^n-1}{81}$}
	{$S=10\cdot\dfrac{10^n-1}{81}-1$}
	{\True $S=\dfrac{1}{9}\left[10\cdot\dfrac{10^n-1}{9}-1\right]$}
	\loigiai{
		Ta có $S=\dfrac{1}{9}\left(9+99+999+\cdots+999\ldots9\right)=\dfrac{1}{9}\left(10+100+1000+\cdots+100\ldots0-n\right)=\dfrac{1}{9}\left[10\cdot\dfrac{10^n-1}{9}-1\right]$.
	}
\end{ex}
\begin{ex}%[1K2G7-7]
	Biết rằng $S=1+2\cdot3+3\cdot3^2+\cdots+11.3^{10}=a+\dfrac{21\cdot3^b}{4}$. Tính $P=a+\dfrac{b}{4}$.
	\choice
	{$P=1$}
	{$P=2$}
	{\True $P=3$}
	{$P=4$}
	\loigiai{
		Từ giả thiết suy ra $3S=3+2\cdot3^2+3\cdot3^3+\cdots+11\cdot3^{11}$.\\
		Do đó $-2S=S-3S=1+3+3^2+3^3+\cdots+3^{10}-10.3^{11}=\dfrac{1-3^{11}}{1-3}-11\cdot3^{11}\Rightarrow S=\dfrac{1}{4}+\dfrac{21}{4}\cdot3^{11}$.\\
		Vậy $a=\dfrac{1}{4},b=11$, suy ra $P=3$.
	}
\end{ex}
\begin{ex}%[1K2K7-1]
	Một cấp số nhân có ba số hạng là $a,b,c$ (theo thứ tự đó) trong đó các số hạng đều khác $0$ và công bội $q\ne 0$. Mệnh đề nào sau đây là đúng.
	\choice
	{$\dfrac{1}{a^2}=\dfrac{1}{bc}$}
	{\True $\dfrac{1}{b^2}=\dfrac{1}{ac}$}
	{$\dfrac{1}{c^2}=\dfrac{1}{ba}$}
	{$\dfrac{1}{a}+\dfrac{1}{b}=\dfrac{2}{c}$}
	\loigiai{
		Ta có $ac=b^2\Rightarrow \dfrac{1}{b^2}=\dfrac{1}{ac}$
	}
\end{ex}
\begin{ex}%[1K2K7-3]
	Bốn góc của một tứ giác tạo thành cấp số nhân và góc lớn nhất gấp $27$ lần góc nhỏ nhất. Tổng của góc lớn nhất và góc bé nhất bằng.
	\choice
	{$56^\circ$}
	{$102^\circ$}
	{\True $252^\circ$}
	{$168^\circ$}
	\loigiai{
		Giả sử $4$ góc $A, B, C, D$ (với $A<B<C<D$) theo thứ tự đó lập thành cấp số nhân thỏa yêu cầu với công bội $q$.\\
		Ta có $\heva{&A+B+C+D=360\\&D=27A}\Leftrightarrow\heva{&A\left(1+q+q^2+q^3\right)=360\\&Aq^3=27A}\Leftrightarrow\heva{&q=3\\&A=9\\&D=243}\Rightarrow A+D=252$.
	}
\end{ex}
\begin{ex}%[1K2G7-7]
	Người ta thiết kế một cái tháp gồm $11$ tầng. Diện tích bề mặt trên của mỗi tầng bằng nữa diện tích của mặt trên của tầng ngay bên dưới và diện tích mặt trên của tầng $1$ bằng nửa diện tích của đế tháp (có diện tích là $12288m^2$). Tính diện tích mặt trên cùng.
	\choice
	{\True $6m^2$}
	{$8m^2$}
	{$10m^2$}
	{$12m^2$}
	\loigiai{
		Diện tích bề mặt của mỗi tầng (kể từ $1$) lập thành một cấp số nhân có công bội $q=\dfrac{1}{2}$ và
		$u_1=\dfrac{12288}{2}=6144$. Khi đó diện tích mặt trên cùng là $u_{11}=u_1\cdot q^{10}=6144\cdot\left(\dfrac{1}{2}\right)^{10}=6$.
	}
\end{ex}
\begin{ex}%[1K2G7-7]
	Một du khách vào chuồng đua ngựa đặt cược, lần đầu đặt $20000$ đồng, mỗi lần sau tiền đặt gấp
	đôi lần tiền đặt cọc trước. Người đó thua $9$ lần liên tiếp và thắng ở lần thứ $10$. Hỏi du khác trên thắng hay thua bao nhiêu?
	\choice
	{Hoà vốn}
	{Thua $20000$ đồng}
	{\True Thắng $20000$ đồng}
	{Thua $40000$ đồng}
	\loigiai{
		Số tiền du khác đặt trong mỗi lần (kể từ lần đầu) là một cấp số nhân có $u_1=20000$ và công bội $q=2$. Du khách thua trong $9$ lần đầu tiên nên tổng số tiền thua là $S_9=u_1.\dfrac{q^9-1}{q-1}=20000\cdot\dfrac{2^9-1}{2-1}=10220000$.\\
		Số tiền mà du khách thắng trong lần thứ $10$ là $u_{10}=u_1\cdot q^9=20000\cdot2^9=10240000$.\\
		Ta có $u_{10}-S_9=20000>0$ nên du khách thắng $20000$.
	}
\end{ex}
\Closesolutionfile{ans}
% \begin{indapan}{10}
% 	{ans/ans-1K2-Ontapchuong2}
% \end{indapan}

%Chương III
% \setcounter{chapter}{2}
\setcounter{subsubsection}{0}
\setcounter{ex}{0}
\setcounter{bt}{0}
\chap{Một số yếu tố thống kê và xác suất}
% \section{Các số đặc trưng đo xu thế trung tâm cho mẫu số liệu ghép nhóm}

\subsection{Tóm tắt lý thuyết}
\begin{tomtat}
	\subsubsection{Mẫu số liệu ghép nhóm}
		\begin{enumerate}
			\item \textbf{\textit{Mẫu số liệu ghép nhóm}} là mẫu số liệu cho dưới dạng bảng tần số ghép nhóm.
			\item Mỗi số liệu gồm một số giá trị của mẫu số liệu được ghép nhóm theo một tiêu chí xác định có dạng $\left[a;b\right)$, trong đó $a$ là \textit{đầu mút trái}, $b$ là \textit{đầu mút phải}. Độ dài nhóm là $b-a$.
			\item \textbf{\textit{Tần số tích luỹ}} của một nhóm là số số liệu trong mẫu số liệu có giá trị nhỏ hơn giá trị đầu mút phải của nhóm đó. Tần số tích luỹ của nhóm $1$, nhóm $2$, $\ldots$, nhóm $m$ kí hiệu lần lượt là $cf_1$, $cf_2$, $\ldots$, $cf_m$.
		\end{enumerate}
	\subsubsection{Số trung bình cộng (số trung bình)}
		\begin{enumerate}
			\item Trung điểm $x_i$ của nửa khoảng (tính bằng trung bình cộng của hai đầu mút) ứng với nhóm $i$ là \textit{giá trị đại diện} của nhóm đó.
			\item \textit{Số trung bình cộng} của mẫu số liệu ghép nhóm, kí hiệu $\overline{x}$, được tính theo công thức 
				$$\overline{x} = \dfrac{n_1x_1 + n_2x_2 + \ldots + n_mx_m}{n}.$$
		\end{enumerate}
	\subsubsection{Trung vị}
	Để tính trung vị của mẫu số liệu ghép nhóm, ta làm như sau:
\begin{itemize}
    \item \textbf{Bước 1:} Xác định nhóm chứa trung vị. Giả sử đó là nhóm thứ $p : [a_p; a_{p + 1})$.
    \item \textbf{Bước 2:} Trung vị là 
    $$M_e = a_p + \dfrac{\dfrac{n}{2} - (m_1 + \cdots + m_{p-1})}{m_p} \cdot ( a_{p + 1} - a_p)$$
    trong đó $n$ là cỡ mẫu, $m_p$ là tần số nhóm $p$. Với $p=1$ ta quy ước $m_1 + \cdots + m_{p-1} = 0$.
\end{itemize}
		\begin{note}
			Nhóm chứa trung vị là nhóm đầu tiên có tần số tích luỹ $cf_p=m_1 + \cdots + m_{p}$ lớn hơn hoặc bằng $\dfrac{n}{2}$
		\end{note}
	\subsubsection{Tứ phân vị}Để tính tứ phân vị thứ nhất $Q_1$ của mẫu số liệu ghép nhóm, trước hết ta xác định nhóm chứa $Q_1$, giả sử đó là nhóm thứ  $p:\left[a_p;a_{p+1} \right)$. Khi đó 
	$$Q_1=a_p+\dfrac{\dfrac{n}{4}-\left(m_1+\cdots+m_{p-1}\right)}{m_p}\cdot \left(a_{p+1}-a_p\right).$$
	trong đó, $n$ là cỡ mẫu, $m_p$ là tần số nhóm $p$. Với $p=1$, ta quy ước $m_1+\cdots+m_{p-1}=0$.\\
	Để tính tứ phân vị thứ ba $Q_3$ của mẫu số liệu ghép nhóm, trước hết ta xác định nhóm chứa $Q_3$, giả sử đó là nhóm thứ  $p:\left[a_p;a_{p+1} \right)$. Khi đó 
	$$Q_3=a_p+\dfrac{\dfrac{3n}{4}-\left(m_1+\cdots+m_{p-1}\right)}{m_p}\cdot \left(a_{p+1}-a_p\right).$$
	Trong đó $n$ là cỡ mẫu, $m_p$ là tần số nhóm $p$. Với $p=1$, ta quy ước $m_1+\cdots+m_{p-1}=0$.\\
	Tứ phân vị thứ hai $Q_2$ chính là trung vị $M_e$.\\
\subsubsection{Mốt của mẫu số liệu ghép nhóm}
Để tìm mốt của mẫu số liệu ghép nhóm, ta thực hiện theo các bước sau:
\begin{enumerate}
	\item [Bước 1.] Xác định nhóm có tần số lớn nhất (gọi là nhóm chứa mốt), giả sử là nhóm $j:\left[a_j;a_{j+1} \right)$.
	\item [Bước 2.] Mốt được xác định là $M_o=a_j+\dfrac{m_i-m_{j-1}}{\left(m_i-m_{j-1}\right)+\left(m_i-m_{j+1}\right)}\cdot h$.\\
\end{enumerate}
trong đó, $m_j$ là tần số nhóm $j$ (quy ước $m_0=m_{k+1}=0$) và $h$ là độ dài của nhóm.	

\end{tomtat}
%=================================================
\setcounter{subsubsection}{0}
\setcounter{ex}{0}
\setcounter{bt}{0}
\subsection{Các dạng toán thường gặp}
\begin{dang}{Mẫu số liệu ghép nhóm}
\end{dang}
\subsubsection{Ví dụ minh hoạ}
\begin{vd}%[Cánh Diều]%[1C5Y1-1]
	\immini{
		\textbf{Bảng bên} biểu diễn mẫu số liệu ghép nhóm được cho dưới dạng bảng tần số ghép nhóm. Hãy cho biết 
		\begin{enumerate}
			\item Mẫu số liệu có bao nhiêu số liệu; bao nhiêu nhóm?
			\item Tần số của mỗi nhóm.
		\end{enumerate}
	}{
		\begin{tabular}{|c|c|}
			\hline
			\textbf{Nhóm} & \textbf{Tần số}\\ 
			\hline
			$\left[0;5\right)$ & $11$\\
			\hline
			$\left[5;10\right)$ & $31$\\
			\hline
			$\left[10;15\right)$ & $45$\\
			\hline
			$\left[15;20\right)$ & $21$\\
			\hline
			$\left[20;26\right)$ & $12$\\
			\hline
			& $n = 120$ \\
			\hline
		\end{tabular}
	}
	\loigiai{
		\begin{enumerate}
			\item Mẫu số liệu gồm $120$ số liệu và $5$ nhóm.
			\item Tần số lần lượt của các nhóm $1$, $2$, $3$, $4$, $5$ lần lượt là $11$, $31$, $45$, $21$, $12$.
		\end{enumerate}	
	}
\end{vd}
\begin{vd}%[CTST]%[1T5B1-1]
	Một cửa hàng đã thống kê số ba lô bán được mỗi ngày trong tháng 9 với kết quả cho như sau: \begin{center}
		\begin{tabular}{lllllllllllllll}
			$12$ & $29$ & $12$ & $19$ & $15$ & $21$ & $19$ & $29$ & $28$ & $12$ & $15$ & $25$ & $16$ & $20$ & $29$\\
			$21$ & $12$ & $24$ & $14$ & $10$ & $12$ & $10$ & $23$ & $27$ & $28$ & $18$ & $16$ & $10$ & $20$ & $21$
		\end{tabular}
	\end{center}
	Hãy chia mẫu số liệu trên thành 5 nhóm, lập bảng tần số ghép nhóm, hiệu chỉnh bảng tần số ghép nhóm và xác định giá trị đại diện cho mỗi nhóm.
	\loigiai{
		Khoảng biến thiên của mẫu số liệu trên là $R=29-10=19$.\\
		Độ dài mỗi nhóm $L>\dfrac{R}{k}=\dfrac{19}{5}=3{,}8$.\\
		Ta chọn $L=4$ và chia dữ liệu thành các nhóm $[10; 14)$, $[14; 18)$, $[18; 22)$, $[22; 26)$, $[26; 30)$.\\
		Khi đó ta có bảng tần số ghép nhóm sau
		\begin{center}
			\begin{tabular}{|c|c|c|c|c|c|}
				\hline \textbf{Cân nặng} &{$[10; 14)$} &{$[14; 18)$} &{$[18; 22)$} &{$[22; 26)$} &{$[26; 30)$} \\
				\hline \textbf{Giá trị đại diện} & $12$ & $16$ & $20$ & $24$ & $28$ \\
				\hline \textbf{Số ba lô bán được} & $8$ & $5$ & $8$ & $3$ & $6$ \\
				\hline
			\end{tabular}
		\end{center}
	}
\end{vd}
\begin{vd}%[KNTT]%[Ngọc Hiếu]%[1K3B8-1]
	Bảng thống kê sau cho biết thời gian chạy (phút) của $30$ vận động viên (VĐV) trong một giải chạy Marathon.
	\begin{center}
		\begin{tabular}{|c|c|c|c|c|c|c|c|c|c|c|c|c|}
			\hline
			Thời gian&$129$&$130$&$133$&$134$&$135$&$136$&$138$&$141$&$142$&$143$&$144$&$145$\\
			\hline
			Số VĐV&$1$&$2$&$1$&$1$&$1$&$2$&$3$&$3$&$4$&$5$&$2$&$5$\\
			\hline
		\end{tabular}
	\end{center}
	Hãy chuyển mẫu số liệu trên sang mẫu số liệu ghép nhóm gồm sáu nhóm có độ dài bằng nhau và bằng $3$.
	\loigiai{
		Giá trị nhỏ nhất là $129$, giá trị lớn nhất là $145$ nên khoảng biến thiên là $145-129=16$. Tổng độ dài của sáu nhóm là $18$. Để cho đối xứng, ta chọn đầu mút trái của nhóm đầu tiên là $127{,}5$ và đầu mút phải của nhóm cuối cùng là $145{,}5$ ta được các nhóm là $[127{,}5;130{,}5),\; [130{,5};133{,5}],\ldots , [142{,}5;145{,}5]$. Đếm số giá trị thuộc mỗi nhóm, ta có mẫu số liệu ghép nhóm như sau
		\begin{center}
			\fontsize{9}{1pt}
			{\begin{tabular}{|c|c|c|c|c|c|c|}
					\hline
					Thời gian&$[125{,}5;130{,}5)$&$[130{,}5;133{,}5)$&$[133{,}5;136{,}5)$&$[136{,}5;139{,}5)$&$[139{,}5;142{,}5)$&$[142{,}5;145{,}5)$\\
					\hline
					Số VĐV&$3$&$1$&$4$&$3$&$7$&$12$\\
					\hline
			\end{tabular}}
		\end{center}
	}
\end{vd}
\begin{vd}%[Cánh Diều]%[1C5B1-1]
	Một trường trung học phổ thông chọn $36$ học sinh nam của khối $11$, do chiều cao của các bạn học sinh đó và thu được mẫu số liệu sau (đơn vị: centimét):
	$$
	\begin{array}{llllllllllll}
		160 & 161 & 161 & 162 & 162 & 162 & 163 & 163 & 163 & 164 & 164 & 164 \\
		164 & 165 & 165 & 165 & 165 & 165 & 166 & 166 & 166 & 166 & 167 & 167 \\
		168 & 168 & 168 & 168 & 169 & 169 & 170 & 171 & 171 & 172 & 172 & 174
	\end{array}
	$$
	Lập bảng tần số ghép nhóm bao gồm cả tần số tích luỹ cho mẫu số liệu trên có $5$ nhóm ứng với $5$ nửa khoảng:
	$$
	\left[160;163 \right),\ \left[163;169 \right),\ \left[166;169 \right),\ \left[169;172 \right),\ \left[172;175 \right).
	$$
	\loigiai{
		Bảng tần số ghép nhóm bao gồm cả tần số tích luỹ như sau:
		\begin{center}
			\begin{tabular}{|c|c|c|}
				\hline
				\textbf{Nhóm} & \textbf{Tần số} & \textbf{Tần số tích luỹ}\\ 
				\hline
				$\left[169;163\right)$ & $6$ & $6$\\
				\hline
				$\left[163;166\right)$ & $12$ & $18$\\
				\hline
				$\left[166;169\right)$ & $10$ & $28$\\
				\hline
				$\left[169;172\right)$ & $5$ & $33$\\
				\hline
				$\left[172;175\right)$ & $3$ & $36$\\
				\hline
				& $n = 36$ &\\
				\hline
			\end{tabular}
		\end{center}
	}
\end{vd}
\subsubsection{Bài tập rèn luyện}
\begin{bt}%[KNTT]%[1K3B8-1]
	Trong các mẫu số liệu sau, mẫu nào là mẫu số liệu ghép nhóm? Đọc và giải thích mẫu số liệu ghép nhóm đó.
	\begin{enumerate}
		\item Số tiền mà sinh viên chi cho thanh toán cước điện thoại trong tháng.
		\begin{center}
			\begin{tabular}{|c|c|c|c|c|c|}
				\hline
				Số tiền (nghìn đồng)&$[0;50)$&$[50;100)$&$[100;150)$&$[150;200)$&$[200;250)$\\
				\hline
				Số sinh viên&$5$&$12$&$23$&$17$&$3$\\
				\hline
			\end{tabular}
		\end{center}
		\item Thống kê nhiệt độ tại một điểm trong $40$ ngày, ta có bảng số liệu sau
		\begin{center}
			\begin{tabular}{|c|c|c|c|c|}
				\hline
				Nhiệt độ $(^\circ$ C)&$[19;22)$&$[22;25)$&$[25;28)$&$[28;31)$\\
				\hline
				Số ngày&$7$&$15$&$12$&$6$\\
				\hline
			\end{tabular}
		\end{center}
	\end{enumerate}
	\loigiai{
		Cả hai mẫu số liệu trên đều là mẫu số lớp ghép nhóm.
		\begin{enumerate}
			\item Có năm nhóm là
			\begin{itemize}
				\item Dưới $50$ nghìn đồng có $5$ sinh viên.
				\item Từ $50$ đến dưới $100$ nghìn đồng có $12$ sinh viên.
				\item Từ $100$ đến dưới $150$ nghìn đồng có $23$ sinh viên.
				\item Từ $150$ đến dưới $200$ nghìn đồng có $17$ sinh viên.
				\item Từ $200$ đến dưới $250$ nghìn đồng có $3$ sinh viên.
			\end{itemize}
			\item Có bốn nhóm là
			\begin{itemize}
				\item Từ $19^\circ$ C đến dưới $22^\circ$ C có $7$ ngày.
				\item Từ $22^\circ$ C đến dưới $25^\circ$ C có $15$ ngày.
				\item Từ $25^\circ$ C đến dưới $28^\circ$ C có $12$ ngày.
				\item Từ $128^\circ$ C đến dưới $31^\circ$ C có $6$ ngày.
			\end{itemize}
		\end{enumerate}
	}
\end{bt}	
\begin{bt}%[KNTT]%[1K3B8-1]
	Số sản phẩm một công nhân làm được trong một ngày được cho như sau:
	\begin{center}
		\begin{tabular}{c c c c c c c c c c c c c}
			$18$&$25$&$39$&$12$&$54$&$27$&$46$&$25$&$19$&$8$&$36$&$22$&\\
			$20$&$19$&$17$&$44$&$5$&$18$&$23$&$28$&$25$&$34$&$46$&$27$&$16$
		\end{tabular}
	\end{center}
	Hãy chuyển mẫu số liệu sang dạng ghép nhóm với sáu nhóm có độ dài bằng nhau.
	\loigiai{
		Khoảng biến thiên là $54-5=49$.\\
		Ta chia thành các nhóm sau $[4{,}5;13); [13;21{,}5);[21{,}5;30);\ldots ;[47;55{,}5)$.\\
		Đếm số giá trị của mỗi nhóm, ta có bảng ghép nhóm sau:
		\begin{center}
			\begin{tabular}{|c|c|c|c|c|c|c|}
				\hline
				Số sản phẩm &$[4{,}5;13)$&$[13;21{,}5)$&$[21{,}5;30)$&$[30;38{,}5)$&$[38{,}5;47)$&$[47;55{,}5)$\\
				\hline
				Số công nhân&$3$&$7$&$8$&$2$&$4$&$1$\\
				\hline
			\end{tabular}
		\end{center}
	}
\end{bt}
\begin{bt}%[KNTT]%[1K3B8-1]
	Thời gian ra sân (giờ) của một số cựu cầu thủ ở giải ngoại hạng Anh qua các thời kì được cho như sau:
	\begin{center}
		\begin{tabular}{c c c c c c c c}
			$653$ & $632$ & $609$ & $572$ & $565$ & $535$ & $516$ & $514$ \\
			$508$ & $505$ & $504$ & $504$ & $503$ & $499$ & $496$ & $492$ 
		\end{tabular}
	\end{center}
	Hãy chuyển mẫu số liệu trên sang dạng ghép nhóm với bảy nhóm có độ dài bằng nhau.
	\loigiai{
		Khoảng biến thiên là $653-492=161$.\\
		Ta chia thành các nhóm sau $[492;515); [515;538);[538;561);\ldots; [630;653]$.\\
		Đếm số giá trị của mỗi nhóm, ta có bảng ghép nhóm sau:
		\begin{center}
			\begin{tabular}{|c|c|c|c|c|c|c|c|}
				\hline
				Thời gian &$[492;515)$&$[515;538)$&$[538;561)$&$[561;584)$&$[584;607)$&$[607;630)$&$[630;653]$\\
				\hline
				Số cầu thủ &$9$&$2$&$0$&$2$&$0$&$1$&$2$\\
				\hline
			\end{tabular}
		\end{center}
	}
\end{bt}

%===================================
\setcounter{subsubsection}{0}
\setcounter{ex}{0}
\setcounter{bt}{0}
\begin{dang}{Số trung bình cộng (số trung bình)}
\end{dang}
\subsubsection{Ví dụ minh hoạ}
\begin{vd}%[Cánh Diều]%[1C5Y1-2]
	\immini
	{
		Một nhà thực vật học đo chiều dài của $74$ lá cây (đơn vị: milimét) và thu được bảng tần số như bảng bên. Tính chiều dài trung bình của $74$ lá cây trên theo đơn vị milimét (làm tròn kết quả đến hàng phần trăm).
	}
	{
		\begin{tabular}{|c|c|c|}
			\hline
			\textbf{Nhóm} & \textbf{Giá trị đại diện} & \textbf{Tần số}\\ 
			\hline
			$\left[5{,}45;5{,}85\right)$ & $5{,}65$ & $5$\\
			$\left[5{,}85;6{,}25\right)$ & $6{,}05$ & $9$\\
			$\left[6{,}25;6{,}65\right)$ & $6{,}45$ & $15$\\
			$\left[6{,}65;7{,}05\right)$ & $6{,}85$ & $19$\\
			$\left[7{,}05;7{,}45\right)$ & $7{,}25$ & $16$\\
			$\left[7{,}45;7{,}85\right)$ & $7{,}65$ & $8$\\
			$\left[7{,}85;8{,}25\right)$ & $8{,}05$ & $2$\\
			\hline
			&  & $n = 74$\\
			\hline
		\end{tabular}
	}
	\loigiai{
		Chiều dài trung bình của $74$ lá cây mà nhà thực vật học đo xấp xỉ là 
		\[
		\overline{x} = \dfrac{5\cdot 5{,}65 + 9 \cdot 6{,}05 + 15\cdot 6{,}45 + 19\cdot 6{,}85 + 16 \cdot 7{,}25 + 8\cdot 7{,}65 + 2\cdot 8{,}05}{74} \approx 6{,}80\ (\text{mm}).
		\]
	}
\end{vd}
\begin{vd}%[CTST]%[1T5B1-2]
	Kết quả khảo sát cân nặng của $25$ quả cam ở mỗi lô hàng $A$ và $B$ được cho ở bảng sau:
	\begin{center}
		\begin{tabular}{|c|c|c|c|c|c|}
			\hline \multicolumn{1}{|c|}{Cân nặng $(\mathrm{g})$} &{$[150; 155)$} &{$[155; 160)$} &{$[160; 165)$} &{$[165; 170)$} &{$[170; 175)$} \\
			\hline Số quả cam ở lô hàng $A$ & 2 & 6 & 12 & 4 & 1 \\
			\hline Số quả cam ở lô hàng $B$ & 1 & 3 & 7 & 10 & 4 \\
			\hline
		\end{tabular}
	\end{center}
	\begin{enumerate}
		\item Hãy ước lượng cân nặng trung bình của mỗi quả cam ở lô hàng $A$ và lô hàng $B$.
		\item Nếu so sánh theo số trung bình thì cam ở lô hàng nào nặng hơn?
	\end{enumerate}
	\loigiai{
		Ta có bảng thống kê số lượng cam theo giá trị đại diện:
		\begin{center}
			\begin{tabular}{|c|c|c|c|c|c|}
				\hline \multicolumn{1}{|c|}{Cân nặng $(\mathrm{g})$} &{$152{,}5$} &{$157{,}5$} &{$162{,}5$} &{$167{,}5$} &$172{,}5$\\
				\hline Số quả cam ở lô hàng $A$ & 2 & 6 & 12 & 4 & 1 \\
				\hline Số quả cam ở lô hàng $B$ & 1 & 3 & 7 & 10 & 4 \\
				\hline
			\end{tabular}
		\end{center}
		\begin{enumerate}
			\item Cân nặng trung bình của mỗi quả cam ở lô hàng $A$ xấp xỉ bằng
			\[(2\cdot 152{,}5+6\cdot 157{,}5+12\cdot 162{,}5+4\cdot 167{,}5+1\cdot 172{,}5): 25=161{,}7\ (\mathrm{g}). \]
			Cân nặng trung bình của mỗi quả cam ở lô hàng $B$ xấp xỉ bằng
			\[(1\cdot 152{,}5+3\cdot 157{,}5+7\cdot 162{,}5+10\cdot 167{,}5+4\cdot 172{,}5): 25=165{,}1\ (\mathrm{g}). \]
			\item Nếu so sánh theo số trung bình thì cam ở lô hàng $B$ nặng hơn cam ở lô hàng $A$.
		\end{enumerate}
	}
\end{vd}
\begin{vd}%[KNTT]%[1K3B9-1]
	Tìm cân nặng trung bình của học sinh lớp $11D$ cho trong Bảng $3.5$.
	\begin{center}
		\begin{tabular}{|c|c|c|c|c|c|c|}
			\hline
			Cân nặng	& $\left[40{,}5;45{,}5 \right)$ & $\left[45{,}5;50{,}5 \right)$ & $\left[50{,}5;55{,}5 \right)$ & $\left[55{,}5;60{,}5 \right)$ & $\left[60{,}5;65{,}5 \right)$ & $\left[65{,}5;70{,}5 \right)$ \\
			\hline
			Số học sinh&$10$	& $7$ & $16$ &$4$  & $2$ & $3$ \\
			\hline
		\end{tabular}

		Bảng $3.5$. Cân nặng của học sinh lớp $11D$.	
	\end{center}	
	\loigiai{
		Trong mỗi khoảng cân nặng, giá trị đại diện là trung bình cộng của hai giá trị đầu mút nên ta có bảng sau
		\begin{center}
			\begin{tabular}{|c|c|c|c|c|c|c|}
				\hline
				Cân nặng (kg)	& $43$ & $48$ & $53$ & $58$ & $63$ & $68$ \\
				\hline
				Số học sinh &$10$ & $7$ & $16$ &$4$  & $2$ & $3$ \\
				\hline
			\end{tabular}
		\end{center}	
		Tổng số học sinh là $n=42$. Cân nặng trung bình của học sinh lớp $11D$ là $$\overline{x}=\dfrac{10\cdot 43+7\cdot 48+16\cdot 53+4\cdot 58+2\cdot 63+3\cdot 68}{42}\approx51{,}81\,\mathrm{(kg)}.$$
	}
\end{vd}
\subsubsection{Bài tập rèn luyện}
\begin{bt}%[Cánh diều]%[1C5B1-5]
	Mẫu số liệu dưới đây ghi lại tốc độ của $40$ ô tô khi đi qua một trạm đo tốc độ (đơn vị: km/h)
	\[
	\begin{array}{cccccccccc}
		48{,}5 & 43 & 50 & 55 & 45 & 60 & 53 & 55,5 & 44 & 65 \\
		51 & 62,5 & 41 & 44,5 & 57 & 57 & 68 & 49 & 46{,}5 & 53{,}5 \\
		61 & 49{,}5 & 54 & 62 & 59 & 56 & 47 & 50 & 60 & 61 \\
		49{,}5 & 52{,}5 & 57 & 47 & 60 & 55 & 45 & 47,5 & 48 & 61{,}5
	\end{array}
	\]
	\begin{enumerate}
		\item Lập bảng tần số ghép nhóm cho mẫu số liệu trên có sáu nhóm ứng với sáu nửa khoảng:
		\[
		[40 ; 45),[45 ; 50),[50 ; 55),[55 ; 60),[60 ; 65),[65 ; 70).
		\]
		\item Xác định số trung bình cộng của mẫu số liệu ghép nhóm trên.
	\end{enumerate}
	\loigiai{
		\begin{enumerate}
			\item Ta có bảng tần số ghép nhóm của mẫu số liệu trên như sau:
			\begin{center}
				\begin{tabular}{|c|c|c|c|}
					\hline
					\textbf{Nhóm} & \textbf{Giá trị đại diện} & \textbf{Tần số} & \textbf{Tần số tích luỹ}\\ 
					\hline
					$\left[40;45\right)$ & $42{,}5$ & $4$ & $4$\\
					$\left[45;50\right)$ & $47{,}5$ & $11$ & $15$\\
					$\left[50;55\right)$ & $52{,}5$ & $7$ & $22$\\
					$\left[55;60\right)$ & $57{,}5$ & $8$ & $30$\\
					$\left[60;65\right)$ & $62{,}5$ & $8$ & $38$\\
					$\left[65;70\right)$ & $67{,}5$ & $2$ & $40$\\
					\hline
					&  & $n = 40$ &\\
					\hline
				\end{tabular}
			\end{center}
			\item Trung bình cộng của mẫu số liệu trên là
			\[
			\overline{x} = \dfrac{42{,}5 \cdot 4 + 47{,}5 \cdot 11 + 52{,}5 \cdot 7+ 57{,}5 \cdot 8+ 62{,}5 \cdot 8 + 67{,}5 \cdot 2}{40} = 53{,}875\text{ (km/h)}.
			\]
			\item Ta thấy: Nhóm $2$ ứng với nửa khoảng $\left[45;50\right)$ là nhóm có tần số lớn nhất với $u=45$, $g=5$, $n_2 = 11$. Nhóm $1$ có tần số $n_1 = 4$, nhóm $3$ có tần số $n_3 = 7$.
		\end{enumerate}
	}
\end{bt}
\begin{bt}%[KNTT]%[1K3B9-4]
	Tuổi thọ (năm) của 50 bình ắc quy ô tô được cho như sau:
	\begin{center}
		\begin{tabular}{|c|c|c|c|c|c|c|}
			\hline
			Tuổi thọ (năm)	& $\left[2;2{,}5 \right)$ & $\left[2{,}5;3 \right)$ & $\left[3;3{,}5 \right)$&$\left[3{,}5;4 \right)$&$\left[4;4{,}5 \right)$&$\left[4{,}5;5 \right)$  \\
			\hline
			Tần số &$4$	& $9$ & $14$ &$11$  & $7$&$5$ \\
			\hline
		\end{tabular}
	\end{center}
	Tính tuổi thọ trung bình của $50$ bình ắc quy ô tô này.
	\loigiai{
		Ta có bảng sau
		\begin{center}
			\begin{tabular}{|c|c|c|c|c|c|c|}
				\hline
				Tuổi thọ (năm)	& $2{,}25$ & $2{,}75$ & $3{,}25$&$3{,}75$&$4{,}25$&$4{,}75$  \\
				\hline
				Tần số &$4$	& $9$ & $14$ &$11$  & $7$&$5$\\
				\hline
			\end{tabular}	
		\end{center}
		Tuổi thọ trung bình của 50 bình ắc quy ô tô này là
		$$\overline{x}=\dfrac{2{,}25\cdot 4+2{,}75\cdot 9+3{,}25\cdot 14+3{,}75\cdot 11+4{,}25\cdot 7+4{,}75\cdot 5}{50}=3{,}48 \, \text{(năm)}.$$
	}
\end{bt}
\begin{bt}%[KNTT]%[1K3B9-4]
	\immini{
		Phỏng vấn một số học sinh lớp $11$ về thời gian (giờ) ngủ của một buổi tối, thu được bảng số liệu ở bên. So sánh thời gian ngủ trung bình của các bạn học sinh nam và nữ.
	}
	{
		\begin{tabular}{|c|c|c|}
			\hline
			Thời gian	& Số học sinh nam & Số học sinh nữ\\
			\hline
			$\left[4;5 \right)$	& $6$ & $4$ \\
			\hline
			$\left[5;6 \right)$	& $10$ & $8$ \\
			\hline
			$\left[6;7 \right)$	& $13$ & $10$ \\
			\hline
			$\left[7;8 \right)$	& $9$ & $11$ \\
			\hline
			$\left[8;9 \right)$	& $7$ & $8$ \\
			\hline
		\end{tabular}
	}
	\loigiai{
		Trong mỗi khoảng thời gian, giá trị đại diện là trung bình cộng của giá trị hai đầu mút nên ta có bảng sau:
		\begin{center}
			\begin{tabular}{|c|c|c|}
				\hline
				Thời gian	& Số học sinh nam & Số học sinh nữ\\
				\hline
				$4{,}5$	& $6$ & $4$ \\
				\hline
				$5{,}5$	& $10$ & $8$ \\
				\hline
				$6{,}5$	& $13$ & $10$ \\
				\hline
				$7{,}5$	& $9$ & $11$ \\
				\hline
				$8{,}5$	& $7$ & $8$ \\
				\hline
			\end{tabular}	
		\end{center}
		Tổng số học sinh nam là $n_1=6+10+13+9+7=45$.\\ Thời gian ngủ trung bình của học sinh nam là:
		$$\overline{x_1}=\dfrac{4{,}5\cdot 6+5{,}5\cdot10+6{,}5\cdot13+7{,}5\cdot9+8{,}5\cdot7}{45}=\dfrac{587}{90}\approx 6{,}52\,\, \text{(giờ)}.$$
		Tổng số học sinh nữ là $n_2=4+8+10+11+8=41$. Thời gian ngủ trung bình của học sinh nữ là:
		$$\overline{x_2}=\dfrac{4,5\cdot4+5,5\cdot8+6,5\cdot10+7,5\cdot11+8,5\cdot8}{41}=\dfrac{555}{82}\approx 6{,}77 \,\,\text{(giờ)}.$$
		Vì $\overline{x_2}>\overline{x_1}$ nên thời gian ngủ trung bình của các bạn học sinh nữ lớn hơn thời gian ngủ trung bình của các bạn nam.
	}
\end{bt}
\begin{bt}%[KNTT]%[1K3B9-4]
	Quãng đường (km) từ nhà đến nơi làm việc của 40 công nhân một nhà máy được ghi lại như sau:
	\begin{center}
		\begin{tabular}{cccccccccccccccccccc}
			$5$	& $3$ &$10$ & $20$ & $25$ & $11$ & $13$ & $7$ & $12$ & $31$\\
			$19$ &$10$  &$12$  & $17$ & $18$ & $11$ & $32$ & $17$ &$16$  &$2$ \\
			$7$	& $9$ &$7$ & $8$ & $3$ & $5$ & $12$ & $15$ & $18$ & $3$\\
			$12$ &$14$  &$2$  & $9$ & $6$ & $15$ & $15$ & $7$ &$6$  &$12$
		\end{tabular}
	\end{center}
	\begin{enumerate}
		\item [a)] Ghép nhóm dãy số liệu trên thành các khoảng có độ rộng bằng nhau, khoảng đầu tiên là $\left[0;5\right)$. Tìm giá trị đại diện cho mỗi nhóm.
		\item [b)] Tính số trung bình của mẫu số liệu không ghép nhóm và mẫu số liệu ghép nhóm. Giá trị nào chính xác hơn?
	\end{enumerate}
	\loigiai{
		\begin{enumerate}
			\item [a)] Giá trị nhỏ nhất của mẫu số liệu là $2$, giá trị lớn nhất là $32$, khoảng đầu tiên của mẫu số liệu ghép nhóm là $\left[0;5\right)$ nên ta ghép nhóm mẫu số liệu như sau
			\begin{center}
				\begin{tabular}{|c|c|c|c|c|c|c|c|}
					\hline
					Quãng đường		 & $\left[0;5\right)$ & $\left[5;10\right)$ & $\left[10;15\right)$ & $\left[15;20\right)$ & $\left[20;25\right)$& $\left[25;30\right)$& $\left[30;35\right)$\\
					\hline
					Số công nhân		& $5$ & $11$ & $11$ & $9$ & $1$ & $1$ & $2$ \\
					\hline
				\end{tabular}
			\end{center}
			Trong mỗi khoảng, giá trị đại điện là trung bình cộng của hai giá trị đầu mút nên ta có bảng sau
			\begin{center}
				\begin{tabular}{|c|c|c|c|c|c|c|c|}
					\hline
					Quãng đường		 & $2{,}5$ & $7{,}5$ & $12{,}5$ & $17{,}5$ & $22{,}5$& $27{,}5$& $32{,}5$\\
					\hline
					Số công nhân		& $5$ & $11$ & $11$ & $9$ & $1$ & $1$ & $2$ \\
					\hline
				\end{tabular}
			\end{center}
			\item [b)] Số trung bình của mẫu số liệu không ghép nhóm là
			$$\overline{x}=\dfrac{5+3+10+\cdots +12}{40}=11{,}9.$$
			Số trung bình của mẫu số liệu ghép nhóm là
			$$\overline{x}=\dfrac{5\cdot 2{,}5+11\cdot 7{,}5+11\cdot 12{,}5+9\cdot 17{,}5+1\cdot 22{,}5+1\cdot 27{,}5+2\cdot 32{,}5}{40}=12{,}625.$$
			Số trung bình của mẫu số liệu không ghép nhóm sẽ chính xác hơn số trung bình của mẫu số liệu ghép nhóm vì số trung bình của dữ liệu không ghép nhóm sử dụng chính xác các số liệu, còn số trung bình của dữ liệu ghép nhóm sử dụng giá trị đại diện của mỗi khoảng ghép nhóm.
		\end{enumerate}
	}
\end{bt}
\begin{bt}%[CTST]%[1T5B1-2]
	Anh Văn ghi lại cự li 30 lần ném lao của mình ở bảng sau (đơn vị: mét):
	\begin{center}
		\begin{tabular}{|c|c|c|c|c|c|c|c|c|c|}
			\hline $72{,}1$ & $72{,}9$ & $70{,}2$ & $70{,}9$ & $72{,}2$ & $71{,}5$ & $72{,}5$ & $69{,}3$ & $72{,}3$ & $69{,}7$ \\
			\hline $72{,}3$ & $71{,}5$ & $71{,}2$ & $69{,}8$ & $72{,}3$ & $71{,}1$ & $69{,}5$ & $72{,}2$ & $71{,}9$ & $73{,}1$ \\
			\hline $71{,}6$ & $71{,}3$ & $72{,}2$ & $71{,}8$ & $70{,}8$ & $72{,}2$ & $72{,}2$ & $72{,}9$ & $72{,}7$ & $70{,}7$ \\
			\hline
		\end{tabular}
	\end{center}
	\begin{enumerate}
		\item Tính cự li trung bình của mỗi lần ném.
		\item Tổng hợp lại kết quả ném của anh Văn vào bảng tần số ghép nhóm theo mẫu sau:
		\begin{center}
			\begin{tabular}{|c|c|c|c|c|c|}
				\hline Cự li $(\mathrm{m})$ &{$[69{,}2; 70)$} &{$[70; 70{,}8)$} &{$[70{,}8; 71{,}6)$} &{$[71{,}6; 72{,}4)$} &{$[72{,}4; 73{,}2)$} \\
				\hline Số lần & $?$ & $?$ & $?$ & $?$ & $?$ \\
				\hline
			\end{tabular}
		\end{center}
		\item Hãy ước lượng cự li trung bình mỗi lần ném từ bảng tần số ghép nhóm trên.
		\item Khả năng anh Văn ném được khoảng bao nhiêu mét là cao nhất?
	\end{enumerate}
	\loigiai{
		\begin{enumerate}
			\item Điểm tổng của mỗi đợt gồm 10 lần ném
			\begin{center}
				\begin{tabular}{|c|c|c|c|c|c|c|c|c|c|c|}
					\hline Điểm &Điểm &Điểm &Điểm &Điểm &Điểm &Điểm &Điểm &Điểm &Điểm &Tổng \\
					\hline $72{,}1$ & $72{,}9$ & $70{,}2$ & $70{,}9$ & $72{,}2$ & $71{,}5$ & $72{,}5$ & $69{,}3$ & $72{,}3$ & $69{,}7$ &$713{,}6$\\
					\hline $72{,}3$ & $71{,}5$ & $71{,}2$ & $69{,}8$ & $72{,}3$ & $71{,}1$ & $69{,}5$ & $72{,}2$ & $71{,}9$ & $73{,}1$ &$714{,}9$\\
					\hline $71{,}6$ & $71{,}3$ & $72{,}2$ & $71{,}8$ & $70{,}8$ & $72{,}2$ & $72{,}2$ & $72{,}9$ & $72{,}7$ & $70{,}7$ &$718{,}4$\\
					\hline
				\end{tabular}
			\end{center}
			Cự li trung bình của mỗi lần ném của anh Văn
			\[\overline{x}=\dfrac{713{,}6+714{,}9+718{,}4}{30}\approx71{,}56\ (\mathrm{m}). \]
			\item Bảng tần số ghép nhóm kết quả ném của anh Văn:
			\begin{center}
				\begin{tabular}{|c|c|c|c|c|c|}
					\hline Cự li $(\mathrm{m})$ &{$[69{,}2; 70)$} &{$[70; 70{,}8)$} &{$[70{,}8; 71{,}6)$} &{$[71{,}6; 72{,}4)$} &{$[72{,}4; 73{,}2)$} \\
					\hline Số lần & $4$ & $2$ & $7$ & $12$ & $5$ \\
					\hline
				\end{tabular}
			\end{center}
			\item Bảng tần số ghép nhóm kết quả ném của anh Văn (theo giá trị đại diện):
			\begin{center}
				\begin{tabular}{|c|c|c|c|c|c|}
					\hline Cự li $(\mathrm{m})$ &{$[69{,}2; 70)$} &{$[70; 70{,}8)$} &{$[70{,}8; 71{,}6)$} &{$[71{,}6; 72{,}4)$} &{$[72{,}4; 73{,}2)$} \\
					\hline Giá trị đại diện &$69{,}6$ &$70{,}4$ &$71{,}2$ &$72{,}0$ &$72{,}8$\\
					\hline Số lần & $4$ & $2$ & $7$ & $12$ & $5$ \\
					\hline
				\end{tabular}
			\end{center}
			Cự li trung bình mỗi lần ném của anh Văn qua bảng tần số ghép nhóm
			\[(69{,}6\cdot 4+70{,}4\cdot 2+71{,}2\cdot 7+72\cdot 12+72{,}8\cdot 5):30=71{,}52\ (\mathrm{m}).  \]
			\item Nhóm chứa mốt của mẫu số liệu trên là nhóm $[71{,}6; 72{,}4)$.\\
			Do đó $u_m=71{,}6$; $n_{m-1}=7$; $n_m=12$; $n_{m+1}=5$; $u_{m+1}-u_m=72{,}4-71{,}6=0{,}8$.\\
			Mốt của mẫu số liệu ghép nhóm là
			\[M_0=71{,}6+\dfrac{12-7}{(12-7)+(12-5)} \cdot 0{,}8=\dfrac{101}{14} \approx 71{,}93. \]
			Dựa vào kết quả trên thì khả năng anh Văn ném được cao nhất là khoảng $71{,}93$ mét.
		\end{enumerate}
	}
\end{bt}
\begin{bt}%[CTST]%[1T5B1-2]
	Người ta đếm số xe ô tô đi qua một trạm thu phí mỗi phút trong khoảng thời gian từ $9$ giờ đến $9$ giờ $30$ phút sáng. Kết quả được ghi lại ở bảng sau:
	\begin{center}
		\begin{tabular}{|c|c|c|c|c|c|c|c|c|c|c|c|c|c|c|}
			\hline $15$ & $16$ & $13$ & $21$ & $17$ & $23$ & $15$ & $21$ & $6$ & $11$ & $12$ & $23$ & $19$ & $25$ & $11$ \\
			\hline $25$ & $7$ & $29$ & $10$ & $28$ & $29$ & $24$ & $6$ & $11$ & $23$ & $11$ & $21$ & $9$ & $27$ & $15$ \\
			\hline
		\end{tabular}
	\end{center}
	\begin{enumerate}
		\item Tính số xe trung bình đi qua trạm thu phí trong mỗi phút.
		\item Tổng hợp lại số liệu trên vào bảng tần số ghép nhóm theo mẫu sau:
		\begin{center}
			\begin{tabular}{|c|c|c|c|c|c|}
				\hline Số xe &{$[6; 10]$} &{$[11; 15]$} &{$[16; 20]$} &{$[21; 25]$} &{$[26; 30]$} \\
				\hline Số lần & $?$ & $?$ & $?$ & $?$ & $?$ \\
				\hline
			\end{tabular}
		\end{center}
		\item Hãy ước lượng trung bình số xe đi qua trạm thu phí trong mỗi phút từ bảng tần số ghép nhóm trên.
	\end{enumerate}
	\loigiai{
		\begin{enumerate}
			\item 
%			Bảng tần số
%			\begin{center}
%				\begin{tabular}{|c|c|c|c|c|c|c|c|c|c|c|c|c|c|c|c|c|c|c|c|}
%					\hline Giá trị &$6$ & $7$ & $9$ & $10$ & $11$ & $12$ & $13$ & $15$ & $16$ & $17$ & $19$ & $21$ & $23$ & $24$ & $25$ & $27$ & $28$ & $29$ &\\
%					\hline Tần số &$2$ & $1$ & $1$ & $1$ & $4$ & $1$ & $1$ & $3$ & $1$ & $1$ & $1$ & $3$ & $3$ & $1$ & $2$ & $1$ & $1$ & $2$ &$N=30$\\
%					\hline
%				\end{tabular}
%			\end{center}
			Số xe trung bình đi qua trạm thu phí trong mỗi phút là
			\allowdisplaybreaks
			\begin{eqnarray*}
				\overline{x}&=&\dfrac{6\cdot 2+7+9+10+11\cdot 4+12+13+15\cdot 3}{30}\\
				&&+\dfrac{16+17+19+21\cdot 3+23\cdot 3+24+25\cdot 2+27+28+29\cdot 2}{30}\\
				&\approx& 17{,}43\ (\text{xe}).
			\end{eqnarray*}	
			\item Bảng tần số ghép nhóm
			\begin{center}
				\begin{tabular}{|c|c|c|c|c|c|}
					\hline Số xe &{$[6; 10]$} &{$[11; 15]$} &{$[16; 20]$} &{$[21; 25]$} &{$[26; 30]$} \\
					\hline Số lần & $5$ & $9$ & $3$ & $9$ & $4$ \\
					\hline
				\end{tabular}
			\end{center}
			\item Bảng tần số ghép nhóm (theo giá trị đại diện) được hiệu chỉnh lại như sau
			\begin{center}
				\begin{tabular}{|c|c|c|c|c|c|}
					\hline Số xe &{$[5{,}5; 10{,}5)$} &{$[10{,}5; 15{,}5)$} &{$[15{,}5; 20{,}5)$} &{$[20{,}5; 25{,}5)$} &{$[25{,}5; 30{,}5)$} \\
					\hline Giá trị đại diện &{$8$} &{$13$} &{$18$} &{$23$} &{$28$} \\
					\hline Số lần & $5$ & $9$ & $3$ & $9$ & $4$ \\
					\hline
				\end{tabular}
			\end{center}
			Số xe trung bình đi qua trạm qua bảng tần số ghép nhóm là
			\[\overline{x}=\dfrac{8\cdot 5+13\cdot 9+18\cdot 3+23\cdot 9+28\cdot 4}{30}\approx 17{,}67\ (\text{xe}). \]
		\end{enumerate}
	}
\end{bt}
\begin{bt}%[CTST]%[1T5B1-2]
	Một thư viện thống kê số lượng sách được mượn mỗi ngày trong ba tháng ở bảng sau:
	\begin{center}
		\begin{tabular}{|c|c|c|c|c|c|c|c|}
			\hline Số sách &{$[16; 20]$} &{$[21; 25]$} &{$[26; 30]$} &{$[31; 35]$} &{$[36; 40]$} &{$[41; 45]$} &{$[46; 50]$} \\
			\hline Số ngày & 3 & 6 & 15 & 27 & 22 & 14 & 5 \\
			\hline
		\end{tabular}
	\end{center}
	Hãy ước lượng số trung bình của mẫu số liệu ghép nhóm trên.
	\loigiai{
		Vì số lượng sách được mượn là số nguyên nên ta hiệu chỉnh bảng tần số ghép nhóm (theo giá trị đại diện) như sau
		\begin{center}
			{\footnotesize \begin{tabular}{|c|c|c|c|c|c|c|c|}
					\hline Số sách &{$[15{,}5; 20{,}5)$} &{$[20{,}5; 25{,}5)$} &{$[25{,}5; 30{,}5)$} &{$[30{,}5; 35{,}5)$} &{$[35{,}5; 40{,}5]$} &{$[40{,}5; 45{,}5)$} &{$[45{,}5; 50{,}5)$} \\
					\hline Giá trị đại diện &{$18$} &{$23$} &{$28$} &{$33$} &{$38$} &{$43$} &{$48$} \\
					\hline Số ngày & 3 & 6 & 15 & 27 & 22 & 14 & 5 \\
					\hline
			\end{tabular}}
		\end{center}
		Trung bình số lượng sách được mượn mỗi ngày trong 3 tháng của thư viện là
		\[\overline{x}=\dfrac{18\cdot 3+23\cdot 6+28\cdot 15+33\cdot 27+38\cdot 22+43\cdot 14+48\cdot 5}{92}\approx 34{,}58. \]
	}
\end{bt}
\begin{bt}%[CTST]%[1T5B1-2]
	Kết quả đo chiều cao của $200$ cây keo $3$ năm tuổi ở một nông trường được biểu diễn ở biểu đồ dưới đây.
	\begin{center}
		\begin{tikzpicture}[scale=1,font=\scriptsize]
			\def\hoanh{11.5};
			\def\tung{6.5};
			\def\mau{cyan};
			\foreach \x/\n in{1/20,3/35,5/60,7/55,9/30}{\draw[line width=16mm,\mau] (\x,0)--++(0,{\n/10});
				\draw[dashed] (\x,{\n/10})node[above]{$\n$}--(0,{\n/10}) node[left]{$\n$};}
			\foreach \x/\p in {1/[8{,}5;8{,}8),3/[8{,}8;9{,}1),5/[9{,}1;9{,}4),7/[9{,}4;9{,}7),9/[9{,}7;10{,}0)}{\node[below] at (\x,0){\scriptsize $\p$};}
			\draw[->] (0,0)--(\hoanh,0) node[below]{($m$)};
			\draw[->] (0,0)node[below left]{$O$}--(0,\tung) node[left]{(Số cây)};
			\path (current bounding box.north) node[above]		{\textbf{Chiều cao 200 cây keo 3 năm tuổi}};
		\end{tikzpicture}
	\end{center}
	Hãy ước lượng số trung bình của mẫu số liệu ghép nhóm trên.
	\loigiai{
		Bảng tần số ghép nhóm (theo giá trị đại diện)
		\begin{center}
			\begin{tabular}{|c|c|c|c|c|c|}
				\hline Chiều cao &$[8{,}5; 8{,}8)$ &{$[8{,}8; 9{,}1)$} &{$[9{,}1; 9{,}4)$} &{$[9{,}4; 9{,}7)$} &{$[9{,}7; 10{,}0)$} \\
				\hline Giá trị đại diện &$8{,}65$ &$8{,}95$ &$9{,}25$ &$9{,}55$ &$9{,}85$ \\
				\hline Số cây & $20$ & $35$ & $60$ & $55$ & $30$\\
				\hline
			\end{tabular}
		\end{center}
		Chiều cao trung bình của $200$ cây keo 3 năm tuổi là
		\[\overline{x}=\dfrac{8{,}65\cdot 20+8{,}95\cdot 35+9{,}25\cdot 60+9{,}55\cdot 55+9{,}85\cdot 30}{200}\approx 9{,}31. \]
	}
\end{bt}
\begin{bt}%[CTST]%[1T5K2-2]
	Kiểm tra điện lượng của một số viên pin tiểu do một hãng sản xuất thu được kết quả như sau:
	\begin{center}
		\begin{tabular}{|c|c|c|c|c|c|}
			\hline 
			\begin{tabular}{c}
				\textbf{Điện lượng} \\	\textbf{(nghìn mAh)}
			\end{tabular} 
			& $ \left[ 0{,}9 ; 0{,}95\right)  $ & $ \left[ 0{,}95 ; 1{,}0\right)  $ & $ \left[ 1{,}0 ; 1{,}05\right)  $ &$ \left[ 1{,}05 ; 1{,}1\right)  $  &  $ \left[ 1{,}1 ; 1{,}15\right)  $\\ 
			\hline 
			\textbf{Số viên pin}& $ 10 $ & $ 20 $ & $ 35 $ & $ 15 $ & $ 5 $ \\ 
			\hline 
		\end{tabular} 
	\end{center}
	Hãy ước lượng số trung bình của mẫu số liệu ghép nhóm trên.
	\loigiai{
		Tìm số trung bình của mẫu số liệu ghép nhóm.\\
		Ta có bảng thống kê điện lượng của pin theo giá trị đại diện là:
		\begin{center}
			\begin{tabular}{|c|c|c|c|c|c|}
				\hline 
				\begin{tabular}{c}
					\textbf{Điện lượng} \\	\textbf{(nghìn mAh)}
				\end{tabular} 
				& $ \left[ 0{,}9 ; 0{,}95\right)  $ & $ \left[ 0{,}95 ; 1{,}0\right)  $ & $ \left[ 1{,}0 ; 1{,}05\right)  $ &$ \left[ 1{,}05 ; 1{,}1\right)  $  &  $ \left[ 1{,}1 ; 1{,}15\right)  $\\ 
				\hline 
				\textbf{Giá trị đại diện}& $ 0{,}925 $ & $ 0{,}975 $ & $ 1{,}025 $ & $ 1{,}075 $ & $ 1{,}125 $ \\ 
				\hline
				\textbf{Số viên pin}& $ 10 $ & $ 20 $ & $ 35 $ & $ 15 $ & $ 5 $ \\ 
				\hline 
			\end{tabular} 
		\end{center}
		Số trung bình của mẫu số liệu ghép nhóm theo dõi điện lượng của một số viên pin xấp xỉ bằng $$\dfrac{0{,}925\cdot 10 + 0{,}975\cdot 20 +1{,}025 \cdot 35 +1{,}075 \cdot 15+1{,}125 \cdot 5}{10+20+35+15+5}\approx 1{,}016.$$
	}
\end{bt}

%===================================
\setcounter{subsubsection}{0}
\setcounter{ex}{0}
\setcounter{bt}{0}
\begin{dang}{Trung vị}
\end{dang}
\subsubsection{Ví dụ minh hoạ}
\begin{vd}%[Cánh Diều]%[1C5B1-3]
	\immini
	{
		Sau khi kiểm tra về số học sinh trong $100$ lớp học, người ta chia mẫu số liệu đó thành năm nhóm căn cứ vào số lượng học sinh của mỗi lớp (đơn vị: học sinh) và lập bảng tần số ghép nhóm bao gồm tần số tích luỹ như bảng bên. Tìm trung vị của mẫu số liệu đó.
	}
	{
		\begin{tabular}{|c|c|c|}
			\hline
			\textbf{Nhóm} & \textbf{Tần số} & \textbf{Tần số tích luỹ}\\ 
			\hline
			$\left[36;38\right)$ & $9$ & $9$\\
			$\left[38;40\right)$ & $15$ & $24$\\
			$\left[40;42\right)$ & $25$ & $49$\\
			$\left[42;44\right)$ & $30$ & $79$\\
			$\left[44;46\right)$ & $21$ & $100$\\
			\hline
			& $n = 100$ &\\
			\hline
		\end{tabular}
	}
	\loigiai{
		Số phần tử của mẫu là $n=100$. Ta có $\dfrac{n}{2} = \dfrac{100}{2} = 50$.\\
		Do $cf_3 = 49 < 50 < cf_4 = 79$ nên nhóm $4$ là nhóm đầu tiên có tần số tích luỹ lớn hơn hoặc bằng $50$.\\
		Xét nhóm $4$ là nhóm $\left[42;44\right)$ có $r=42$; $d=2$ và $n_4=30$ và nhóm $3$ là nhóm $\left[40;42\right)$ có $cf_3 = 49$.\\
		Khi đó trung vị của mẫu số liệu là 
		\[
		M_e = 42 + \dfrac{50 - 49}{30} \cdot 2 \approx 42\text{ (học sinh)}.
		\]
	}
\end{vd}
\begin{vd}%[KNTT]%[1K3B9-2]
	Thời gian (phút) truy cập internet mỗi buổi tối của một số học sinh được cho trong bảng sau:
	\begin{center}
		\begin{tabular}{|c|c|c|c|c|c|c|}
			\hline
			Thời gian (phút)	& $\left[9{,}5;12{,}5 \right)$ & $\left[12{,}5;15{,}5 \right)$ & $\left[15{,}5;18{,}5 \right)$ & $\left[18{,}5;21{,}5 \right)$ & $\left[21{,}5;24{,}5 \right)$ \\
			\hline
			Số học sinh&$3$	& $12$ & $15$ &$24$  & $2$  \\
			\hline
		\end{tabular}	
	\end{center}
	Tính trung vị của mẫu số liệu ghép nhóm này.
	\loigiai{
		Cỡ mẫu là $n=3+12+15+24+2=56$.\\
		Gọi $x_1,\,\ldots,\,x_{56}$ là thời gian vào internet của $56$ học sinh và giả sử dãy này đã được sắp xếp theo thứ tự tăng dần. Khi đó, trung vị là $\dfrac{x_{28}+x_{29}}{2}$. Do $2$ giá trị $x_{28},\,x_{29}$ thuộc nhóm $\left[15{,}5;18{,}5 \right)$ nên nhóm này chứa trung vị. Do đó, $p=3$; $a_3=15{,}5$; $m_3=15$; $m_1+m_2=3+12=15$; $a_4-a_3=3$ và ta có $$M_e=15{,}5+\dfrac{\dfrac{56}{2}-15}{15}\cdot 3=18{,}1.$$
	}
\end{vd}
\begin{vd}%[CTST]%[1T5B2-1]
	Kết quả khảo sát cân nặng của $ 25 $ quả bơ ở một lô hàng cho trong bảng sau:
	\begin{center}
		\begin{tabular}{|c|c|c|c|c|c|}
			\hline 
			\textbf{Cân nặng}\textbf{ (g)}	& $ \left[150 ; 155 \right) $ & $ \left[ 155 ; 160\right)  $ & $ \left[160 ; 165\right)  $ & $ \left[ 165 ; 170\right)  $ & $ \left[170 ; 175 \right)  $ \\ 
			\hline 
			\textbf{Số quả bơ}	& $ 1 $ & $ 7 $ & $ 12 $ & $ 3 $ & $ 2 $ \\ 
			\hline 
		\end{tabular} 
	\end{center}
	Hãy tìm trung vị của mẫu số liệu ghép nhóm trên.
	\loigiai{
		Gọi $ x_1; x_2; \ldots ; x_{25} $ là cân nặng của $ 25$ quả bơ xếp theo thứ tự không giảm.\\
		Do $ x_1\in \left[150 ; 155 \right) $; $ x_2, \ldots, x_8 \in \left[ 155 ; 160\right) $; $ x_9, \ldots, x_{20} \in \left[ 160 ; 165\right) $ nên trung vị của mẫu số liệu $ x_1; x_2; \ldots; x_{25} $ là $ x_{13}\in \left[ 160 ; 165\right)$.\\
		Ta xác định được $ n=25 $, $ n_m=12 $, $ C=1+7=8 $, $ u_m=160 $, $ u_{m+1}=165 $.\\
		Vậy trung vị của mẫu số liệu ghép nhóm là $$ M_e=160+\dfrac{\dfrac{25}{2}-8}{12}\cdot(165-160) =161{,}875.$$
	}
\end{vd}
\begin{vd}%[CTST]%[1T5K2-1]
	Trong tuần lễ bảo vệ môi trường, các học sinh khối $ 11 $ tiến hành thu nhặt vỏ chai nhựa để tái chế. Nhà trường thống kê kết quả thu nhặt vỏ chai của học sinh khối $ 11 $ ở bảng sau
	\begin{center}
		\begin{tabular}{|c|c|c|c|c|c|}
			\hline 
			\textbf{Số vỏ chai nhựa}	& $ \left[ 11 ; 15\right]  $ & $ \left[ 16 ; 29\right]  $ & $ \left[21 ; 25 \right]  $ & $ \left[ 26 ; 30\right]  $ & $ \left[31 ; 35 \right]  $ \\ 
			\hline 
			\textbf{Số học sinh}	& $ 53 $ & $ 82 $ & $ 48 $ & $ 39 $ & $ 18 $ \\ 
			\hline 
		\end{tabular} 
	\end{center}
	Hãy tìm trung vị của mẫu số liệu ghép nhóm trên.
	\loigiai{
		Do số vỏ chai là số nguyên nên ta hiệu chỉnh lại như sau:
		\begin{center}
			\begin{tabular}{|c|c|c|c|c|c|}
				\hline 
				\textbf{Số vỏ chai nhựa}	& $ \left[ 10{,}5 ; 15{,}5\right) $ & $ \left[ 15{,}5 ; 20{,}5\right) $ & $ \left[ 20{,}5 ; 25{,}5\right) $ & $ \left[ 25{,}5 ; 30{,}5\right) $ & $ \left[30{,}5 ; 35{,}5 \right)  $ \\ 
				\hline 
				\textbf{Số học sinh}& $ 53 $ & $ 82 $ & $ 48 $ & $ 39 $ & $ 18 $ \\ 
				\hline 
			\end{tabular} 
		\end{center}
		Số học sinh tham gia thu nhặt vỏ chai nhựa là $$ n=53+82+48+39+18=240.$$
		Gọi $ x_1; x_2; \ldots ; x_{240} $ lần lượt là số vỏ chai $ 240 $ học sinh khối $ 11 $ thu nhặt được xếp theo thứ tự không giảm.\\
		Do $ x_1, \ldots, x_{53}\in \left[10{,}5 ; 15{,}5 \right) $; $ x_{54}, \ldots, x_{135}\in \left[ 15{,}5 ; 20{,}5\right)$ nên trung vị của mẫu số liệu $ x_1; x_2; \ldots;x_{240} $ là $$ \dfrac{1}{2}\left( x_{120}+x_{121}\right)\in \left[ 15{,}5 ; 20{,}5\right).$$
		Ta xác định được $ n=240$; $ n_m=82 $; $ C=53 $; $ u_m=15{,}5 $; $ u_{m+1}=20{,}5 $.\\
		Trung vị của mẫu số liệu ghép nhóm là $$ M_e=15{,}5+\dfrac{\dfrac{240}{2}-53}{82}\cdot \left( 20{,}5-15{,}5\right)=\dfrac{803}{41}\approx 19{,}59. $$
	}
\end{vd}
\begin{vd}%[CTST]%[1T5K2-1]
	Trong một hội thao, thời gian chạy $200$m của một nhóm các vận động viên được ghi lại ở bảng sau
	\begin{center}
		\begin{tabular}{|c|c|c|c|c|c|}
			\hline 
			\textbf{Thời gian} \textbf{(giây)}& $ \left[21 ; 21{,}5 \right)  $ & $ \left[ 21{,}5 ; 22\right)  $ & $ \left[ 22 ; 22{,}5\right)  $ & $ \left[ 22{,}5 ; 23\right)  $ & $ \left[ 23 ; 23{,}5\right)  $ \\ 
			\hline 
			\textbf{Số vận động viên} & $ 5 $ & $ 12 $ & $ 32 $ & $ 45 $ & $ 30 $ \\ 
			\hline 
		\end{tabular} 
	\end{center}
	Dựa vào bảng số liệu trên, ban tổ chức muốn chọn ra khoảng $ 50 \% $ số vận động viên chạy nhanh nhất để tiếp tục thi vòng $ 2 $. Ban tổ chức nên chọn các vận động viên có thời gian chạy không quá bao nhiêu giây?
	\loigiai{
		Số vận động viên tham gia là $$n=5+12+32+45+30=124.$$
		Gọi $ x_1; x_2; \ldots ; x_{124} $ lần lượt là thời gian chạy $ 200 $ m của $ 124 $ vận động viên được xếp theo thứ tự không giảm.\\
		Do $ x_1, \ldots, x_5 \in \left[ 21 ; 21{,}5 \right)$, $ x_6, \ldots, x_{17} \in \left[ 21{,}5 ; 22\right) $, $ x_{18}, \ldots, x_{49} \in \left[22 ; 22{,}5\right) $, $ x_{50},\ldots, x_{94} \in \left[ 22{,}5 ; 23\right) $ nên trung vị của mẫu số liệu $ x_1; x_2; \ldots ;x_{124} $ là
		$$\dfrac{1}{2}\cdot \left( x_{62}+x_{63}\right) \in  \left[ 22{,}5 ; 23\right).$$
		Ta xác định được $ n=124 $; $ n_m=45$; $ C=5+12+32=49 $; $ u_m= 22{,}5$; $ u_{m+1}=23$.\\
		Trung vị của mẫu số liệu ghép nhóm là $$M_e=22{,}5 +\dfrac{\dfrac{124}{2}-49}{45}\cdot \left( 23-22{,}5\right)= \dfrac{1019}{45}\approx 22{,}64.$$
		Vậy ban tổ chức nên chọn các vận động viên  có thời gian chạy không quá $ 22{,}64$ (giây) được tiếp tục thi vòng hai.
	}
\end{vd}
\subsubsection{Bài tập rèn luyện}
\begin{bt}%[Cánh diều]%[1C5B1-5]
	\immini
	{
		Bảng bên cho ta bảng tần số ghép nhóm số liệu thống kê chiều cao của $40$ mẫu cây ở một vườn thực vật (đơn vị: centimét). Xác định trung vị của mẫu số liệu ghép nhóm trên.
	}
	{
		\begin{tabular}{|c|c|c|}
			\hline
			\textbf{Nhóm} & \textbf{Tần số} & \textbf{Tần số tích luỹ}\\ 
			\hline
			$\left[30;40\right)$ & $4$ & $4$\\
			$\left[40;50\right)$ & $10$ & $14$\\
			$\left[50;60\right)$ & $14$ & $28$\\
			$\left[60;70\right)$ & $6$ & $34$\\
			$\left[70;80\right)$ & $4$ & $38$\\
			$\left[80;90\right)$ & $2$ & $40$\\
			\hline
			& $n = 40$ &\\
			\hline
		\end{tabular}
	}
	\loigiai{
		Ta có $\dfrac{n}{2} = 20$, mà $14<20<28$ nên nhóm $3$ là nhóm đầu tiên có tần số tích luỹ lớn hơn hoặc bằng $20$. \\
		Xét nhóm $3$ là nhóm $\left[50;60\right)$ có $r=50$, $d=10$, $n_3=14$ và nhóm $2$ có $cf_2 = 14$.\\
		Khi đó, tứ phân vị thứ hai (cũng là trung vị) là
		\[
		M_e = 50 + \dfrac{20 - 14}{14} \cdot 10 = 54{,}3\text{ (cm)}.
		\]
	}
\end{bt}
\begin{bt}%[Cánh diều]%[1C5B1-5]
	Mẫu số liệu sau ghi lại cân nặng của $30$ bạn học sinh (đơn vị: kilôgam)
	\[
	\begin{array}{cccccccccc}
		17 & 40 & 39 & 40{,}5 & 42 & 51 & 41{,}5 & 39 & 41 & 30\\
		40 & 42 & 40{,}5 & 39{,}5 & 41 & 40{,}5 & 37 & 39{,}5 & 40 & 41\\
		38{,}5 & 39{,}5 & 40 & 41 & 39 & 40{,}5 & 40 & 38{,}5 & 39{,}5 & 41{,}5
	\end{array}
	\]
	\begin{enumerate}
		\item Lập bảng tần số ghép nhóm cho mẫu số liệu trên có tám nhóm ứng với tám nửa khoảng:
		\[
		[15 ; 20),[20 ; 25),[25 ; 30),[30 ; 35),[35 ; 40),[40 ; 45),[45 ; 50),[50 ; 55).
		\]
		\item Xác định trung vị của mẫu số liệu ghép nhóm trên.
	\end{enumerate}
	\loigiai{
		\begin{enumerate}
			\item Ta có bảng tần số ghép nhóm của mẫu số liệu trên như sau:
			\begin{center}
				\begin{tabular}{|c|c|c|c|}
					\hline
					\textbf{Nhóm} & \textbf{Giá trị đại diện} & \textbf{Tần số} & \textbf{Tần số tích luỹ}\\ 
					\hline
					$\left[15;20\right)$ & $17{,}5$ & $1$ & $1$\\
					$\left[20;25\right)$ & $22{,}5$ & $0$ & $1$\\
					$\left[25;30\right)$ & $27{,}5$ & $0$ & $1$\\
					$\left[30;35\right)$ & $32{,}5$ & $1$ & $2$\\
					$\left[35;40\right)$ & $37{,}5$ & $10$ & $12$\\
					$\left[40;45\right)$ & $42{,}5$ & $17$ & $29$\\
					$\left[45;50\right)$ & $47{,}5$ & $0$ & $29$\\
					$\left[50;55\right)$ & $52{,}5$ & $1$ & $30$\\
					\hline
					&  & $n = 30$ &\\
					\hline
				\end{tabular}
			\end{center}
			\item Ta có $\dfrac{n}{2} = 15$, mà $12<15<29$ nên nhóm $6$ là nhóm đầu tiên có tần số tích luỹ lớn hơn hoặc bằng $15$. \\
			Xét nhóm $6$ là nhóm $\left[40;45\right)$ có $r=40$, $d=5$, $n_6=17$ và nhóm $5$ có $cf_5 = 12$.\\
			Khi đó, trung vị là
			\[
			M_e = 40 + \dfrac{15-12}{17} \cdot 5 = 40{,}9\text{ (kg)}.
			\]
		\end{enumerate}
	}
\end{bt}
\begin{bt}%[CTST]%[1T5G2-2]
	Cân nặng của một số lợn con mới sinh thuộc hai giống $ A $ và $ B $ được cho ở biểu đồ dưới đây (đơn vị: kg).
	\begin{center}
		\begin{tikzpicture}[>=stealth,line join=round,line cap=round,font=\footnotesize,scale=0.85,line width=1pt]
			\draw[->] (0,0)--(0,5)node[left]{(\text{Số con})};
			\foreach \y in {1,2,3,4}
			\draw[shift={(0,\y)}] (0,0)--(-2pt,0) node[left]{\scriptsize ${\y}0$};
			%	\path (4.5,6) node {\normalsize{\textbf{Cân nặng của một số lợn con mới sinh}}};
			\path (4.5,5.5) node {
				$\begin{array}{c}
					\normalsize{\textbf{Cân nặng của một số}}\\
					\normalsize{\textbf{lợn con mới sinh}}
				\end{array}$
			};
			%% nhãn
			\path (2.5,-1.5) node[rectangle,fill=cyan,draw=none]{};
			\path (3.6,-1.5) node {\text{Giống $ A $}};
			\path (5,-1.5) node[rectangle,fill=orange,draw=none]{};
			\path (6.1,-1.5) node {\text{Giống $ B $}};
			% đường gióng
			\foreach \y in {1,2,3,4}{
				\draw[line width=0.2pt] (0,\y)--(8.4,\y);
			}
			%% cột
			\draw[fill=cyan,draw=none] (0,0)--(0,0.8)--(1,0.8)node[midway,above]{$ 8 $}--(1,0)--cycle;
			\draw[fill=orange,draw=none] (1,0)--(1,1.3)--(2,1.3)node[midway,above]{$ 13 $}--(2,0)--cycle;
			\draw[fill=cyan,draw=none] (2,0)--(2,2.8)--(3,2.8)node[midway,above]{$ 28 $}--(3,0)--cycle;
			\draw[fill=orange,draw=none] (3,0)--(3,1.4)--(4,1.4)node[midway,above]{$ 14 $}--(4,0)--cycle;
			\draw[fill=cyan,draw=none] (4,0)--(4,3.2)--(5,3.2)node[midway,above]{$ 32 $}--(5,0)--cycle;
			\draw[fill=orange,draw=none] (5,0)--(5,2.4)--(6,2.4)node[midway,above]{$ 24 $}--(6,0)--cycle;
			\draw[fill=cyan,draw=none] (6,0)--(6,1.7)--(7,1.7)node[midway,above]{$ 17 $}--(7,0)--cycle;
			\draw[fill=orange,draw=none] (7,0)--(7,1.4)--(8,1.4)node[midway,above]{$ 14 $}--(8,0)--cycle;
			%% miền
			\node [below] at (1,0){$ \left[1{,}0 ; 1{,}1 \right)$};
			\node [below] at (3,0){$ \left[1{,}1 ; 1{,}2 \right)$};
			\node [below] at (5,0){$ \left[1{,}2 ; 1{,}3 \right)$};
			\node [below] at (7,0){$ \left[1{,}3 ; 1{,}4 \right)$};
			\draw[->] (0,0)node [below left=-2pt]{$ O $}--(9,0)node[below]{(\text{kg})};
		\end{tikzpicture}
	\end{center}
	Hãy so sánh cân nặng của lợn con mới sinh giống $ A $ và giống $ B $ theo số trung bình và trung vị.
	\loigiai{
		Bảng tần số ghép nhóm thống kê cân nặng của lợn con mới sinh giống $ A $ và giống $ B $ như sau:
		\begin{center}
			\begin{tabular}{|c|c|c|c|c|}
				\hline 
				\textbf{Cân nặng (kg)}	& $ \left[1{,}0 ; 1{,}1 \right)$  &$ \left[1{,}1 ; 1{,}2 \right)$  &$ \left[1{,}2 ; 1{,}3 \right)$  & $ \left[1{,}3 ; 1{,}4 \right)$ \\ 
				\hline 
				\textbf{Giá trị đại diện (kg)}	& $1{,}05 $ & $ 1{,}15 $ & $ 1{,}25 $ & $ 1{,}35 $ \\ 
				\hline 
				\begin{tabular}{c}
					\textbf{Giống A}
					\\ 
					\textbf{(đơn vị: con)}
				\end{tabular} 	& $ 8 $ & $ 28 $ & $ 32 $ & $ 17 $ \\ 
				\hline 
				\begin{tabular}{c}
					\textbf{Giống B}
					\\ 
					\textbf{(đơn vị: con)}
				\end{tabular} 	& $ 13 $ & $ 14 $ & $ 24 $ & $ 14 $ \\ 
				\hline 
			\end{tabular} 
		\end{center}
		Cân nặng trung bình của lợn con mới sinh giống $ A $ là $$ \dfrac{1{,}05\cdot 8 + 1{,}15 \cdot 28 + 1{,}25 \cdot 32 + 1{,}35 \cdot 17}{8+28+32+17}=\dfrac{2071}{1700}\approx 1{,}218.$$
		Cân nặng trung bình của lợn con mới sinh giống $ B $ là $$ \dfrac{1{,}05\cdot 13 + 1{,}15 \cdot 14 + 1{,}25 \cdot 24 + 1{,}35 \cdot 14}{13+14+24+14}=\dfrac{121}{100} \approx 1{,}21.$$
		Suy ra cân nặng trung bình của lợn con mới sinh giống $ A $  lớn hơn cân nặng trung bình của lợn con mới sinh giống $ B $. \\
		Trung vị của mẫu số liệu ghép nhóm cân nặng của lợn con  giống $ A $ là $$M_e=1{,}2+\dfrac{\dfrac{85}{2}-(8+28)}{32}\cdot (1{,}3-1{,}2)=\dfrac{781}{640}\approx 1{,}22.$$
		Trung vị của mẫu số liệu ghép nhóm cân nặng của lợn con  giống  $ B $ là $$M_e=1{,}2+\dfrac{\dfrac{65}{2}-(13+14)}{24}\cdot (1{,}3-1{,}2)=\dfrac{587}{480}\approx 1{,}22.$$
		Suy ra trung vị của mẫu số liệu ghép nhóm cân nặng của của lợn con giống $ A $ bằng trung vị của mẫu số liệu ghép nhóm cân nặng của của lợn con giống $ B $.
	}
\end{bt}

\setcounter{subsubsection}{0}
\setcounter{ex}{0}
\setcounter{bt}{0}
\begin{dang}{Tứ phân vị}
	
\end{dang}
\subsubsection{Ví dụ minh hoạ}
\begin{vd}
	\immini
	{
		Bảng bên cho biết tần số ghép nhóm số liệu thống kê cân nặng của $40$ học sinh lớp $11A$ trong một trường trung học phổ thông (đơn vị: kilôgam). Xác định tứ phân vị của mẫu số liệu ghép nhóm.
	}
	{
		\begin{tabular}{|c|c|c|}
			\hline
			\textbf{Nhóm} & \textbf{Tần số} & \textbf{Tần số tích luỹ}\\ 
			\hline
			$\left[30;40\right)$ & $2$ & $2$\\
			$\left[40;50\right)$ & $10$ & $12$\\
			$\left[50;60\right)$ & $16$ & $28$\\
			$\left[60;70\right)$ & $8$ & $36$\\
			$\left[70;80\right)$ & $2$ & $38$\\
			$\left[80;90\right)$ & $2$ & $40$\\
			\hline
			& $n = 40$ &\\
			\hline
		\end{tabular}
	}
	\loigiai{
		Số phần tử của mẫu là $n=40$.
		\begin{itemize}
			\item Ta có $\dfrac{n}{4} = 10$, mà $2<10<12$ nên nhóm $2$ là nhóm đầu tiên có tần số tích luỹ lớn hơn hoặc bằng $10$. \\
			Xét nhóm $2$ là nhóm $\left[40;50\right)$ có $s=40$, $h=10$, $n_2=10$ và nhóm $1$ có $cf_1 = 2$.\\
			Khi đó, tứ phân vị thứ nhất là
			\[
			Q_1 = 40 + \dfrac{10-2}{10} \cdot 10 = 48\text{ (kg)}.
			\]
			\item Ta có $\dfrac{n}{2} = 20$, mà $12<20<28$ nên nhóm $3$ là nhóm đầu tiên có tần số tích luỹ lớn hơn hoặc bằng $20$. \\
			Xét nhóm $3$ là nhóm $\left[50;60\right)$ có $r=50$, $d=10$, $n_3=16$ và nhóm $2$ có $cf_2 = 12$.\\
			Khi đó, tứ phân vị thứ hai là
			\[
			Q_2 = 50 + \dfrac{20-12}{16} \cdot 10 = 55\text{ (kg)}.
			\]
			\item Ta có $\dfrac{3n}{4} = 30$, mà $28<30<36$ nên nhóm $4$ là nhóm đầu tiên có tần số tích luỹ lớn hơn hoặc bằng $30$. \\
			Xét nhóm $4$ là nhóm $\left[60;70\right)$ có $t=50$, $l=10$, $n_4=8$ và nhóm $3$ có $cf_3 = 28$.\\
			Khi đó, tứ phân vị thứ ba là
			\[
			Q_3 = 60 + \dfrac{30-28}{8} \cdot 10 = 62{,}5\text{ (kg)}.
			\]
		\end{itemize}
		Vậy tứ phân vị của mẫu số liệu trên là $48$, $55$ và $62{,}5$.
	}
\end{vd}
\subsubsection{Bài tập rèn luyện}
\begin{bt}%[1K3B9-3]
	Thời gian (phút) truy cập internet mỗi buổi tối của một số học sinh được cho trong bảng sau:
	\begin{center}
		\begin{tabular}{|c|c|c|c|c|c|c|}
			\hline
			Thời gian (phút)	& $\left[9{,}5;12{,}5 \right)$ & $\left[12{,}5;15{,}5 \right)$ & $\left[15{,}5;18{,}5 \right)$ & $\left[18{,}5;21{,}5 \right)$ & $\left[21{,}5;24{,}5 \right)$ \\
			\hline
			Số học sinh&$3$	& $12$ & $15$ &$24$  & $2$  \\
			\hline
		\end{tabular}	
	\end{center}
	Tìm tứ phân vị thứ nhất $Q_1$ và tứ phân vị thứ ba $Q_3$ của mẫu số liệu ghép nhóm.
	\loigiai{
		Cỡ mẫu là $n=3+12+15+24+2=56$.\\
		Tứ phân vị thứ nhất $Q_1$ là $\dfrac{x_{14}+x_{15}}{2}$. Do $2$ giá trị $x_{28},\,x_{29}$ thuộc nhóm $\left[12{,}5;15{,}5 \right)$ nên nhóm này chứa $Q_1$. Do đó, $p=2$; $a_2=12{,}5$; $m_2=12$; $m_1=3$; $a_3-a_2=3$ và ta có $$Q_1=12{,}5+\dfrac{\dfrac{56}{4}-3}{12}\cdot 3=15{,}25.$$
		Với tứ phân vị thứ ba $Q_3$ là $\dfrac{x_{42}+x_{43}}{2}$. Do $2$ giá trị $x_{42},\,x_{43}$ thuộc nhóm $\left[18{,}5;21{,}5 \right)$ nên nhóm này chứa $Q_3$. Do đó, $p=4$; $a_4=18{,}5$; $m_4=24$; $m_1+m_2+m_3=3+12+15=30$; $a_5-a_4=3$ và ta có $$Q_3=18{,}5+\dfrac{\dfrac{3\cdot 56}{4}-30}{24}\cdot 3=20.$$
	}
\end{bt}
\begin{bt}%[1K3B9-4]
	Điểm thi môn Toán (thang điểm 100, điểm được làm tròn đến 1) của 60 thí sinh được cho trong bảng sau:
	\begin{center}
		\begin{tabular}{|c|c|c|c|c|c|}
			\hline
			Điểm		& $0-9$ & $10-19$ & $20-29$ & $30-39$ & $40-49$ \\
			\hline
			Số thí sinh	& $1$ & $2$ & $4$ & $6$ & $15$ \\
			\hline
			Điểm	& $50-59$ & $60-69$ & $70-79$ & $80-89$ & $90-99$  \\
			\hline
			Số thí sinh	& $12$ & $10$ & $6$ & $3$ & $1$  \\
			\hline
		\end{tabular}
	\end{center}
	\begin{enumerate}
		\item [a)] Hiệu chỉnh để thu được mẫu số liệu ghép nhóm dạng Bảng $3.2$.
		\item [b)] Tìm các tứ phân vị và giải thích ý nghĩa của chúng.
	\end{enumerate}
	\loigiai{
		\begin{enumerate}
			\item [a)] Bảng số liệu ghép nhóm về điểm thi môn Toán của 60 thí sinh
			\begin{center}
				\begin{tabular}{|c|c|c|c|c|c|}
					\hline
					Điểm		& $\left[0;20\right)$ & $\left[20;40\right)$ & $\left[40;60\right)$ & $\left[60;80\right)$ & $\left[80;100\right)$ \\
					\hline
					Số thí sinh	& $3$ & $10$ & $27$ & $16$ & $4$ \\
					\hline
				\end{tabular}
			\end{center}
			
			\item [b)] Cỡ mẫu $n=60$. Gọi $x_1$, $x_2$,$\ldots$, $x_{60}$ là điểm thi môn Toán của 60 học sinh và giả sử dãy này đã được sắp xếp theo thứ tự tăng dần. Khi đó, trung vị là $\dfrac{x_{30}+x_{31}}{2}$.\\
			Do hai giá trị $x_{30}$, $x_{31}$ thuộc nhóm $\left[40;60\right)$ nên nhóm này chứa trung vị. Do đó, $p=3;a_3=40;m_3=27;m_1+m_2=13;a_4-a_3=20$ và ta có
			$$Q_2=M_e=40+\dfrac{\dfrac{60}{2}-13}{27}\cdot 20\approx 52{,}6.$$
			Tứ phân vị thứ nhất $Q_1=\dfrac{x_{15}+x_{16}}{2}$. Do hai giá trị $x_{15}$, $x_{16}$ thuộc nhóm $\left[40;60\right)$ nên nhóm này chứa $Q_1$. Do đó, $p=3;\,a_3=40;\,m_3=27;\,m_1+m_2=13;\,a_4-a_3=20$ và ta có
			$$Q_1=40+\dfrac{\dfrac{60}{4}-13}{27}\cdot 20\approx 41{,}5.$$
			Tứ phân vị thứ ba $Q_3=\dfrac{x_{45}+x_{46}}{2}$. Do hai giá trị $x_{45}$, $x_{46}$ thuộc nhóm $\left[60;80\right)$ nên nhóm này chứa $Q_3$. Do đó, $p=4;\,a_4=60;\,m_4=16;\,m_1+m_2+m_3=40;\,a_5-a_4=20$ và ta có
			$$Q_3=60+\dfrac{\dfrac{3\cdot 60}{4}-40}{16}\cdot 20\approx 66{,}3.$$
			Khoảng cách từ $Q_1$ đến $Q_2$ là $11{,}1$ còn khoảng cách từ $Q_2$ và $Q_3$ là $13{,}7$. Điều này cho thấy mẫu số liệu tập trung với mật độ cao hơn ở bên trái $Q_2$ và mật độ thấp hơn ở bên phải $Q_2$.
		\end{enumerate}	
	}
\end{bt}
\begin{bt}%[1K3B9-4]
	\immini{
		Phỏng vấn một số học sinh lớp $11$ về thời gian (giờ) ngủ của một buổi tối, thu được bảng số liệu ở bên.
		\begin{enumerate}
			\item [a)] So sánh thời gian ngủ trung bình của các bạn học sinh nam và nữ.
			\item [b)] Hãy cho biết $75\%$ học sinh khối $11$ ngủ ít nhất bao nhiêu giờ?
		\end{enumerate}
	}
	{
		\begin{tabular}{|c|c|c|}
			\hline
			Thời gian	& Số học sinh nam & Số học sinh nữ\\
			\hline
			$\left[4;5 \right)$	& $6$ & $4$ \\
			\hline
			$\left[5;6 \right)$	& $10$ & $8$ \\
			\hline
			$\left[6;7 \right)$	& $13$ & $10$ \\
			\hline
			$\left[7;8 \right)$	& $9$ & $11$ \\
			\hline
			$\left[8;9 \right)$	& $7$ & $8$ \\
			\hline
		\end{tabular}
	}
	\loigiai{
		\begin{enumerate}
			\item [a)] Trong mỗi khoảng thời gian, giá trị đại diện là trung bình cộng của giá trị hai đầu mút nên ta có bảng sau:
			\begin{center}
				\begin{tabular}{|c|c|c|}
					\hline
					Thời gian	& Số học sinh nam & Số học sinh nữ\\
					\hline
					$4{,}5$	& $6$ & $4$ \\
					\hline
					$5{,}5$	& $10$ & $8$ \\
					\hline
					$6{,}5$	& $13$ & $10$ \\
					\hline
					$7{,}5$	& $9$ & $11$ \\
					\hline
					$8{,}5$	& $7$ & $8$ \\
					\hline
				\end{tabular}	
			\end{center}
			Tổng số học sinh nam là $n_1=6+10+13+9+7=45$.\\ Thời gian ngủ trung bình của học sinh nam là:
			$$\overline{x_1}=\dfrac{4{,}5\cdot 6+5{,}5\cdot10+6{,}5\cdot13+7{,}5\cdot9+8{,}5\cdot7}{45}=\dfrac{587}{90}\approx 6{,}52\,\, \text{(giờ)}.$$
			Tổng số học sinh nữ là $n_2=4+8+10+11+8=41$. Thời gian ngủ trung bình của học sinh nữ là:
			$$\overline{x_2}=\dfrac{4,5\cdot4+5,5\cdot8+6,5\cdot10+7,5\cdot11+8,5\cdot8}{41}=\dfrac{555}{82}\approx 6{,}77 \,\,\text{(giờ)}.$$
			Vì $\overline{x_2}>\overline{x_1}$ nên thời gian ngủ trung bình của các bạn học sinh nữ lớn hơn thời gian ngủ trung bình của các bạn nam.
			\item [b)] Tổng số học sinh được điều tra là $n=n_1+n_2=45+41=86$.\\
			Giả sử $x_1;x_2;x_3;\cdot \cdot;x_{86}$ là dãy giá trị được sắp xếp theo thứ tự không giảm.\\
			Ta có bảng sau:
			\begin{center}
				\begin{tabular}{|c|c|c|}
					\hline
					Thời gian	& Số học sinh \\
					\hline
					$\left[4;5 \right)$	& $10$  \\
					\hline
					$\left[5;6 \right)$	& $18$  \\
					\hline
					$\left[6;7 \right)$	& $23$  \\
					\hline
					$\left[7;8 \right)$	& $20$  \\
					\hline
					$\left[8;9 \right)$	& $15$  \\
					\hline
				\end{tabular}
			\end{center}
			Tứ phân vị thứ nhất $Q_1$ là $x_{22}$. Do $x_{22}$ thuộc nhóm $\left[5;6\right)$ nên nhóm này chứa $Q_1$.\\ Do đó, $p=2;\,a_2=5;\,m_2=18;\,m_1=10;\,a_3-a_2=1$ và ta có
			$$Q_1=5+\dfrac{\dfrac{86}{4}-10}{18}\cdot 1=\dfrac{203}{36}\approx 5{,}64 \text{(giờ)}.$$
			Nghĩa là có $25\%$ học sinh khối $11$ ngủ ít hơn $5{,}64$ giờ.\\ Vậy $75\%$ học sinh khối $11$ ngủ ít nhất $5{,}64$ giờ.
		\end{enumerate}
	}
\end{bt}
\setcounter{subsubsection}{0}
\setcounter{ex}{0}
\setcounter{bt}{0}
\begin{dang}{Mốt}
	
\end{dang}
\subsubsection{Ví dụ minh hoạ}
\begin{vd}%[Tex hóa SGK CD, TVN-223]%[1C5B1-5]
	Kết quả kiểm tra môn Toán của lớp $11D$ như sau
	\[
	\begin{array}{cccccccccccccccccccc}
	5 & 6 & 7 & 5 & 6 & 9 & 10 & 8 & 5 & 5 & 4 & 5 & 4 & 5 & 7 & 4 & 5 & 8 & 9 & 10 \\
	5 & 3 & 5 & 6 & 5 & 7 & 5 & 8 & 4 & 9 & 5 & 6 & 5 & 6 & 8 & 8 & 7 & 9 & 7 & 9
	\end{array}
	\]
	\begin{enumerate}
		\item Lập bảng tần số ghép nhóm của mẫu số liệu trên có bốn nhóm ứng với bốn nửa khoảng $\left[3;5\right)$, $\left[5;7\right)$, $\left[7;9\right)$, $\left[9;11\right)$.
		\item Mốt của bảng số liệu ghép nhóm trên là bao nhiêu (làm tròn kết quả đến hàng phần mười)?
	\end{enumerate}
	\loigiai{
		\immini
		{
			\begin{enumerate}
				\item Bảng bên là bảng tần số ghép nhóm cho kết quả kiểm tra môn Toán của lớp $11D$.
				\item Ta thấy: Nhóm $2$ ứng với nửa khoảng $\left[5;7\right)$ là nhóm có tần số lớn nhất với $u=5$, $g=2$, $n_2 = 18$. Nhóm $1$ có tần số $n_1 = 5$, nhóm $3$ có tần số $n_3=10$.\\
				Khi đó, mốt của mẫu số liệu là 
				\[
				M_o = 5 + \left( \dfrac{18- 5}{2\cdot 18 - 5 - 10} \right) \cdot 2 \approx 6{,}2.
				\]
			\end{enumerate}
		}
		{
			\begin{tabular}{|c|c|}
				\hline
				\textbf{Nhóm} & \textbf{Tần số}\\ 
				\hline
				$\left[3;5\right)$ & $5$\\
				\hline
				$\left[5;7\right)$ & $18$\\
				\hline
				$\left[7;9\right)$ & $10$\\
				\hline
				$\left[9;11\right)$ & $7$\\
				\hline
				& $n = 40$ \\
				\hline
			\end{tabular}
		}
	}
\end{vd}
\subsubsection{Bài tập rèn luyện}
\begin{bt}%[Tex hóa SGK CD, TVN-223]%[1C5B1-5]
	Mẫu số liệu dưới đây ghi lại tốc độ của $40$ ô tô khi đi qua một trạm đo tốc độ (đơn vị: km/h):
	\[
	\begin{array}{cccccccccc}
	48{,}5 & 43 & 50 & 55 & 45 & 60 & 53 & 55,5 & 44 & 65 \\
	51 & 62,5 & 41 & 44,5 & 57 & 57 & 68 & 49 & 46{,}5 & 53{,}5 \\
	61 & 49{,}5 & 54 & 62 & 59 & 56 & 47 & 50 & 60 & 61 \\
	49{,}5 & 52{,}5 & 57 & 47 & 60 & 55 & 45 & 47,5 & 48 & 61{,}5
	\end{array}
	\]
	\begin{enumerate}
		\item Lập bảng tần số ghép nhóm cho mẫu số liệu trên có sáu nhóm ứng với sáu nửa khoảng:
		\[
		[40 ; 45),[45 ; 50),[50 ; 55),[55 ; 60),[60 ; 65),[65 ; 70).
		\]
		\item Mốt của mẫu số liệu ghép nhóm trên là bao nhiêu?
	\end{enumerate}
	\loigiai{
		\begin{enumerate}
			\item Ta có bảng tần số ghép nhóm của mẫu số liệu trên như sau:
			\begin{center}
				\begin{tabular}{|c|c|c|c|}
					\hline
					\textbf{Nhóm} & \textbf{Giá trị đại diện} & \textbf{Tần số} & \textbf{Tần số tích luỹ}\\ 
					\hline
					$\left[40;45\right)$ & $42{,}5$ & $4$ & $4$\\
					$\left[45;50\right)$ & $47{,}5$ & $11$ & $15$\\
					$\left[50;55\right)$ & $52{,}5$ & $7$ & $22$\\
					$\left[55;60\right)$ & $57{,}5$ & $8$ & $30$\\
					$\left[60;65\right)$ & $62{,}5$ & $8$ & $38$\\
					$\left[65;70\right)$ & $67{,}5$ & $2$ & $40$\\
					\hline
					&  & $n = 40$ &\\
					\hline
				\end{tabular}
			\end{center}
			
			\item Ta thấy: Nhóm $2$ ứng với nửa khoảng $\left[45;50\right)$ là nhóm có tần số lớn nhất với $u=45$, $g=5$, $n_2 = 11$. Nhóm $1$ có tần số $n_1 = 4$, nhóm $3$ có tần số $n_3 = 7$.\\
			Khi đó, mốt của mẫu số liệu là 
			\[
			M_o = 45 + \left( \dfrac{11 - 4}{2\cdot 11 - 4 - 7} \right) \cdot 5 \approx 48{,}2\text{ (km/h)}.
			\]
		\end{enumerate}
	}
\end{bt}
\begin{bt}%[Tex hóa SGK CD, TVN-223]%[1C5B1-5]
	Mẫu số liệu sau ghi lại cân nặng của $30$ bạn học sinh (đơn vị: kilôgam):
	\[
	\begin{array}{cccccccccc}
	17 & 40 & 39 & 40{,}5 & 42 & 51 & 41{,}5 & 39 & 41 & 30\\
	40 & 42 & 40{,}5 & 39{,}5 & 41 & 40{,}5 & 37 & 39{,}5 & 40 & 41\\
	38{,}5 & 39{,}5 & 40 & 41 & 39 & 40{,}5 & 40 & 38{,}5 & 39{,}5 & 41{,}5
	\end{array}
	\]
	\begin{enumerate}
		\item Lập bảng tần số ghép nhóm cho mẫu số liệu trên có tám nhóm ứng với tám nửa khoảng:
		\[
		[15 ; 20),[20 ; 25),[25 ; 30),[30 ; 35),[35 ; 40),[40 ; 45),[45 ; 50),[50 ; 55).
		\]
		
		\item Mốt của mẫu số liệu ghép nhóm trên là bao nhiêu?
	\end{enumerate}
	\loigiai{
		\begin{enumerate}
			\item Ta có bảng tần số ghép nhóm của mẫu số liệu trên như sau:
			\begin{center}
				\begin{tabular}{|c|c|c|c|}
					\hline
					\textbf{Nhóm} & \textbf{Giá trị đại diện} & \textbf{Tần số} & \textbf{Tần số tích luỹ}\\ 
					\hline
					$\left[15;20\right)$ & $17{,}5$ & $1$ & $1$\\
					$\left[20;25\right)$ & $22{,}5$ & $0$ & $1$\\
					$\left[25;30\right)$ & $27{,}5$ & $0$ & $1$\\
					$\left[30;35\right)$ & $32{,}5$ & $1$ & $2$\\
					$\left[35;40\right)$ & $37{,}5$ & $10$ & $12$\\
					$\left[40;45\right)$ & $42{,}5$ & $17$ & $29$\\
					$\left[45;50\right)$ & $47{,}5$ & $0$ & $29$\\
					$\left[50;55\right)$ & $52{,}5$ & $1$ & $30$\\
					\hline
					&  & $n = 30$ &\\
					\hline
				\end{tabular}
			\end{center}
		
			\item Ta thấy: Nhóm $6$ ứng với nửa khoảng $\left[40;45\right)$ là nhóm có tần số lớn nhất với $u=40$, $g=5$, $n_6 = 17$. Nhóm $5$ có tần số $n_5 = 10$, nhóm $7$ có tần số $n_7 = 0$.\\
			Khi đó, mốt của mẫu số liệu là 
			\[
			M_o = 40 + \left( \dfrac{17 - 10}{2\cdot 17 - 10 - 0} \right) \cdot 5 \approx 41{,}5\text{ (kg)}.
			\]
		\end{enumerate}
	}
\end{bt}
\begin{bt}%[Tex hóa SGK CD, TVN-223]%[1C5B1-5]
	\immini
	{
		Bảng bên cho ta bảng tần số ghép nhóm số liệu thống kê chiều cao của $40$ mẫu cây ở một vườn thực vật (đơn vị: centimét).
		
		
			 Mốt của mẫu số liệu ghép nhóm trên là bao nhiêu?
		
	}
	{
		\begin{tabular}{|c|c|c|}
			\hline
			\textbf{Nhóm} & \textbf{Tần số} & \textbf{Tần số tích luỹ}\\ 
			\hline
			$\left[30;40\right)$ & $4$ & $4$\\
			$\left[40;50\right)$ & $10$ & $14$\\
			$\left[50;60\right)$ & $14$ & $28$\\
			$\left[60;70\right)$ & $6$ & $34$\\
			$\left[70;80\right)$ & $4$ & $38$\\
			$\left[80;90\right)$ & $2$ & $40$\\
			\hline
			& $n = 40$ &\\
			\hline
		\end{tabular}
	}
	\loigiai{
	Ta thấy: Nhóm $3$ ứng với nửa khoảng $\left[50;60\right)$ là nhóm có tần số lớn nhất với $u=50$, $g=10$, $n_3 = 14$. Nhóm $2$ có tần số $n_2 = 10$, nhóm $4$ có tần số $n_4 = 6$.\\
			Khi đó, mốt của mẫu số liệu là 
			\[
			M_o = 50 + \left( \dfrac{14 - 10}{2\cdot 14 - 10 - 6} \right) \cdot 10 \approx 53{,}3\text{ (cm)}.
			\]
	
	}
\end{bt}
\subsection{Bài tập trắc nghiệm}
\Opensolutionfile{ans}[ans/ansOC3]
% \begin{ex}%[1C5Y1-1]
% 	Một cuộc khảo sát đã tiến hành xác định tuổi (theo năm) của $120$ chiếc ô-tô. Kết quả điều tra được cho trong bảng sau
% 	\begin{center}
% 		\begin{tabular}{ |c|c|c|c|c|c|c| }
% 			\hline
% 			Nhóm & $[0;4)$ & $[4;8)$ & $[8;12)$ & $[12;16)$ & $[16;20)$ &  \\
% 			\hline
% 			Tần số & $23$ & $25$ & $27$ & $26$ & $19$ & $n=120$ \\
% 			\hline
% 		\end{tabular}
% 	\end{center}
% 	Mẫu số liệu trên có bao nhiêu nhóm?
% 	\choice
% 	{$10$}
% 	{$11$}
% 	{\True $5$}
% 	{$7$}
% 	\loigiai{
% 		Từ bảng, ta thấy mẫu số liệu trên có $5$ nhóm.
% 	}
% \end{ex}
% \begin{ex}%[1C5Y1-1]
% 	Một cuộc khảo sát đã tiến hành xác định tuổi (theo năm) của $120$ chiếc ô-tô. Kết quả điều tra được cho trong bảng sau
% 	\begin{center}
% 		\begin{tabular}{ |c|c|c|c|c|c|c| }
% 			\hline
% 			Nhóm & $[0;4)$ & $[4;8)$ & $[8;12)$ & $[12;16)$ & $[16;20)$ &  \\
% 			\hline
% 			Tần số & $23$ & $25$ & $27$ & $26$ & $19$ & $n=120$ \\
% 			\hline
% 		\end{tabular}
% 	\end{center}
% 	Nhóm có tần số bằng $19$ là
% 	\choice
% 	{$[0;4)$}
% 	{$[8;12)$}
% 	{$[12;16)$}
% 	{\True $[16;20)$}
% 	\loigiai{
% 		Từ bảng, ta thấy nhóm có tần số bằng $19$ là $[16;20)$.
% 	}
% \end{ex}
\begin{ex}%[1C5Y1-1]
	Một cuộc khảo sát đã tiến hành xác định tuổi (theo năm) của $120$ chiếc ô-tô. Kết quả điều tra được cho trong bảng sau
	\begin{center}
		\begin{tabular}{ |c|c|c|c|c|c|c| }
			\hline
			Nhóm & $[0;4)$ & $[4;8)$ & $[8;12)$ & $[12;16)$ & $[16;20)$ &  \\
			\hline
			Tần số & $23$ & $25$ & $27$ & $26$ & $19$ & $n=120$ \\
			\hline
		\end{tabular}
	\end{center}
	Số ô-tô có độ tuổi dưới $12$ là
	\choice
	{\True $75$}
	{$27$}
	{$48$}
	{$26$}
	\loigiai{
		Từ bảng, ta thấy số ô-tô có độ tuổi dưới $12$ là $23+25+27=75$.
	}
\end{ex}
% \begin{ex}%[1C5Y1-1]
% 	Cho mẫu số liệu ghép nhóm sau
% 	\begin{center}
% 		\begin{tabular}{ |c|c|c|c|c|c|c|c| }
% 			\hline
% 			Thời gian & $[15;20)$ & $[20;25)$ & $[25;30)$ & $[30;35)$ & $[35;40)$ & $[40;45)$ & $[45;50)$ \\
% 			\hline
% 			Số nhân viên & $6$ & $14$ & $25$ & $37$ & $21$ & $13$ & $9$ \\
% 			\hline
% 		\end{tabular}
% 	\end{center}
% 	Tần số của nhóm $[15;20)$ là bao nhiêu?
% 	\choice
% 	{\True $6$}
% 	{$7$}
% 	{$14$}
% 	{$25$}
% 	\loigiai{
% 		Ta thấy tần số của nhóm $[15;20)$ là $6$.
% 	}
% \end{ex}
\begin{ex}%[1C5Y1-1]
	Khảo sát thời gian tập thể dục trong ngày của một số học sinh khối $11$ thu được mẫu số liệu ghép nhóm sau
	\begin{center}
		\begin{tabular}{ |c|c|c|c|c|c| }
			\hline
			Thời gian (phút)& $[0;20)$ & $[20;40)$ & $[40;60)$ & $[60;80)$ & $[80;100)$ \\
			\hline
			Số học sinh & $5$ & $9$ & $12$ & $10$ & $6$ \\
			\hline
		\end{tabular}
	\end{center}
	Giá trị đại diện của nhóm $[20;40)$ là
	\choice
	{$10$}
	{\True $30$}
	{$20$}
	{$40$}
	\loigiai{
		Giá trị đại diện của nhóm $[20;40)$ là $\dfrac{20+40}{2}=30$.
	}
\end{ex}
\begin{ex}%[1C5Y1-1]
	Doanh thu bán hàng trong $20$ ngày được lựa chọn ngẫu nhiên của một cửa hàng được ghi lại ở bảng sau (đơn vị: triệu đồng):
	\begin{center}
		\begin{tabular}{ |c|c|c|c|c|c| }
			\hline
			Doanh thu & $[5;7)$ & $[7;9)$ & $[9;11)$ & $[11;13)$ & $[13;15)$ \\
			\hline
			Số ngày & $2$ & $7$ & $7$ & $3$ & $1$ \\
			\hline
		\end{tabular}
	\end{center}
	Doanh thu bán hàng của cửa hàng trong ngày $A$ là $7$ triệu đồng thì được xếp vào nhóm nào?
	\choice
	{$[5;7)$}
	{\True $[7;9)$}
	{$[9;11)$}
	{$[13;15)$}
	\loigiai{
		Doanh thu bán hàng là $7$ triệu đồng thì được xếp vào nhóm $[7;9)$.
	}
\end{ex}
\begin{ex}%[1C5Y1-1]
	Doanh thu bán hàng trong $20$ ngày được lựa chọn ngẫu nhiên của một cửa hàng được ghi lại ở bảng sau (đơn vị: triệu đồng):
	\begin{center}
		\begin{tabular}{ |c|c|c|c|c|c| }
			\hline
			Doanh thu & $[5;7)$ & $[7;9)$ & $[9;11)$ & $[11;13)$ & $[13;15)$ \\
			\hline
			Số ngày & $2$ & $7$ & $7$ & $3$ & $1$ \\
			\hline
		\end{tabular}
	\end{center}
	Các nhóm có độ dài bằng
	\choice
	{\True $2$}
	{$3$}
	{$4$}
	{$5$}
	\loigiai{
		Các nhóm có độ dài bằng nhau, và bằng $2$.
	}
\end{ex}
\begin{ex}%[1C5B1-1]
	Cho bảng số liệu về khối lượng của $30$ củ khoai tây thu hoạch từ một thửa ruộng như hình bên dưới. Tần suất của lớp $[100;110)$ là bao nhiêu?
	\begin{center}
		\begin{tabular}{ |c|c|c|c|c|c| }
			\hline
			Lớp khối lượng (gam) & $[70;80)$ & $[80;90)$ & $[90;100)$ & $[100;110)$ & $[110;120]$ \\
			\hline
			Tần số & $3$ & $6$ & $12$ & $6$ & $3$ \\
			\hline
		\end{tabular}
	\end{center}
	\choice
	{\True $20\%$}
	{$10\%$}
	{$40\%$}
	{$90\%$}
	\loigiai{
		Tần suất ghép lớp $[100;110)$ là $\dfrac{6}{30}\cdot 100\%=20\%$.
	}
\end{ex}
\begin{ex}%[1C5B1-1]
	Cân nặng của $28$ học sinh nam lớp $11$ được cho ở bảng sau
	\begin{center}
		\begin{tabular}{ |c|c|c|c|c|c|c| }
			\hline
			Cân nặng & $[45;49)$ & $[49;53)$ & $[53;57)$ & $[57;61)$ & $[61;65)$\\
			\hline
			Số học sinh & $4$ & $5$ & $7$ & $7$ & $5$ \\
			\hline
		\end{tabular}
	\end{center}
	Tấn số tích lũy của nhóm $[49;53)$ là bao nhiêu?
	\choice
	{$5$}
	{$4$}
	{\True $9$}
	{$20$}
	\loigiai{
		Tần số tích lũy của nhóm $[49;53)$ là $4+5=9$.
	}
\end{ex}
% \begin{ex}%[1C5K1-1]
% 	Một thư viện thống kê sô người đến đọc sách vào buổi tối trong $30$ ngày của tháng vừa qua như sau
% 	\begin{center}
% 		\begin{tabular}{ cccccccccc }
% 			$26$ & $35$ & $68$ & $84$ & $33$ & $84$ & $62$ & $45$ & $57$ & $46$ \\
% 			$35$ & $29$ & $28$ & $50$ & $26$ & $34$ & $75$ & $74$ & $43$ & $49$ \\
% 			$54$ & $55$ & $83$ & $82$ & $81$ & $54$ & $27$ & $36$ & $41$ & $52$ \\
% 		\end{tabular}
% 	\end{center}
% 	Bạn An lập bảng tần số mẫu số liệu trên như sau
% 	\begin{center}
% 		\begin{tabular}{ |c|c|c|c|c|c|c|c| }
% 			\hline
% 			Nhóm & $[25;35)$ & $[35;45)$ & $[45;55)$ & $[55;65)$ & $[65;75]$ & $[75;85)$ & \\
% 			\hline
% 			Tần số & $7$ & $4$ & $7$ & $3$ & $3$ & $6$ & $n=30$ \\
% 			\hline
% 		\end{tabular}
% 	\end{center}
% 	Bạn Khuê lập bảng tần số mẫu số liệu trên như sau
% 	\begin{center}
% 		\begin{tabular}{ |c|c|c|c|c|c|c|c|c| }
% 			\hline
% 			Nhóm & $[23;31)$ & $[31;39)$ & $[39;47)$ & $[47;55)$ & $[55;63]$ & $[71;79)$ & $[79;87)$ & \\
% 			\hline
% 			Tần số & $5$ & $5$ & $4$ & $5$ & $3$ & $1$ & $2$ & $n=30$ \\
% 			\hline
% 		\end{tabular}
% 	\end{center}
% 	Hỏi bảng tần số của bạn nào đúng?
% 	\choice
% 	{Bảng tần số của bạn An}
% 	{\True Bảng tần số của bạn Khuê}
% 	{Cả hai bạn đều đúng}
% 	{Cả hai bạn đều sai}
% 	\loigiai{
% 		Từ các bảng tần số mẫu số liệu trên, ta thấy bảng tần số của bạn An sai ở hai nhóm $[35;45)$ (tần số đúng bằng $5$) và nhóm $[65;75)$ (tần số đúng bằng $2$).
% 	}
% \end{ex}
% \begin{ex}%[0D5Y3-1]
% 	Cho dãy số liệu thống kê: $21$, $23$, $24$, $25$, $22$, $20$. Số trung bình cộng của các số liệu thống kê đã cho là
% 	\choice
% 	{$23{,}5$}
% 	{$22$}
% 	{\True $22{,}5$}
% 	{$14$}
% 	\loigiai{
% 		Ta có $\overline{x}=\dfrac{21+23+24+25+22+20}{6}=22{,}5$.
% 	}
% \end{ex}
% \begin{ex}%[0D5Y3-1]
% 	Điều tra về số con của $30$ gia đình ở khu vực, kết quả thu được như sau
% 	\begin{center}
% 		\begin{tabular}{|c|c|c|c|c|c|c|}
% 			\hline 
% 			Giá trị (số con) & $0$ &$1$ & $2$ &$3$ & $4$ & Tổng \\ 
% 			\hline 
% 			Tần số & $1$ & $7$ & $15$ & $5$ & $2$ & $N=30$ \\ 
% 			\hline 
% 		\end{tabular}
% 	\end{center}
% 	Tìm số trung bình $\overline{x}$ của mẫu số liệu trên.
% 	\choice
% 	{\True $\overline{x}=2$}
% 	{$\overline{x}=1$}
% 	{$\overline{x}=1{,}5$}
% 	{$\overline{x}=3$}
% 	\loigiai{
% 		Ta có $\overline{x}=\dfrac{0\cdot 1+1\cdot 7+2\cdot 15+3\cdot 5+4\cdot 2}{30}=2$.
% 	}
% \end{ex}
% \begin{ex}%[0D5Y3-1]
% 	Điểm môn Toán của lớp $11A$ được cho trong bảng sau
% 	\begin{center}
% 		\begin{tabular}{|l|c|c|c|c|c|c|c|c|c|c|}
% 			\hline
% 			Điểm&$1$&$2$&$3$&$4$&$5$&$6$&$7$&$8$&$9$&$10$\\
% 			\hline
% 			Tần số&$2$&$1$&$4$&$3$&$9$&$7$&$5$&$5$&$3$&$1$\\
% 			\hline
% 		\end{tabular}
% 	\end{center}
% 	Điểm trung bình của các học sinh lớp $10A$ là bao nhiêu?
% 	\choice
% 	{$5$}
% 	{$5{,}5$}
% 	{$5{,}6$}
% 	{\True $5{,}7$}
% 	\loigiai{
% 		Điểm trung bình lớp $11A$ là
% 		$$\overline{x}= \dfrac{1 \cdot 2 + 2 \cdot 1 + \cdots + 10 \cdot 1}{40}= \dfrac{227}{40} \approx 5{,}7.$$
% 	}
% \end{ex}
% \begin{ex}%[0D5Y3-1]
% 	Kết quả điểm kiểm tra môn Toán của $40$ học sinh lớp $10A$ được trình bày ở bảng sau:
% 	\begin{center}
% 		\begin{tabular}{|c|c|c|c|c|c|c|c|c|}
% 			\hline 
% 			Điểm&$4$  &$5$  &$6$  &$7$  &$8$  &$9$  &$10$  &Cộng  \\ 
% 			\hline 
% 			Tần số&$2$  &$8$  &$7$  &$10$  &$8$  &$3$  &$2$  &$40$  \\ 
% 			\hline 
% 		\end{tabular} 
% 	\end{center}
% 	Tính số trung bình cộng của bảng trên. (làm tròn kết quả đến một chữ số thập phân).
% 	\choice
% 	{\True $6{,}8$}
% 	{$6{,}4$}
% 	{$7{,}0$}
% 	{$6{,}7$}
% 	\loigiai{
% 		Ta có $\overline{x}=\dfrac{4\cdot 2+ 5\cdot 8+ 6\cdot 7+ 7\cdot 10+ 8\cdot 8+ 9\cdot 3+ 10\cdot 2}{40}\approx 6{,}8$.
% 	}
% \end{ex}
\begin{ex}%[0D5B3-1]
	Điểm môn Toán của lớp $10$A được cho như bảng sau
	\begin{center}
		\begin{tabular}{|c|c|c|c|c|c|}
			\hline
			Điểm &$[0;2)$& $[2;4)$& $[4;6)$& $[6;8)$& $[8;10)$\\\hline
			Tần số& $3$& $5$& $12$& $12$& 8\\ \hline
		\end{tabular}
	\end{center}
	Điểm trung bình của các học sinh lớp $10$A là bao nhiêu?
	\choice
	{$5$}
	{\True $5{,}85$}
	{$5{,}65$}
	{$5{,}45$}	
	\loigiai{
		Điểm trung bình $\overline{x}=\dfrac{1\cdot 3+3\cdot 5+5\cdot 12+7\cdot 12+9\cdot 8}{40}=5{,}85$.		
	}
\end{ex}
\begin{ex}%[0D5B3-1]
	Cho bảng phân bố tần số ghép lớp
	\begin{center}
		\begin{tabular}{|l|c|c|c|c|}
			\hline
			{Lớp các giá trị $x$}&{[8; 10)}&{[10; 12)}&{[12; 14]}&{Cộng}\\
			\hline
			{Tần số $n_i$}&{15}&{30}&{55}&{100}\\
			\hline
		\end{tabular}	
	\end{center}
	Số trung bình của các giá trị trong bảng trên là
	\choice
	{$9$}
	{$13$}
	{$11$}
	{\True $11{,}8$}	
	\loigiai{
		Giá trị đại diện của lớp $\left[8; 10\right)$: $c_1=\dfrac{8 + 10}{2}=9$.\\ 
		Giá trị đại diện của lớp $\left[10; 12\right)$: $c_2=\dfrac{10 + 12}{2}=11$.\\ 
		Giá trị đại diện của lớp $\left[12; 14\right)$: $c_3=\dfrac{12 + 14}{2}=13$.\\ 
		Vậy số trung bình cộng $\overline{x}=\dfrac{9\cdot 15 + 11\cdot 30 + 13\cdot 55}{15 + 30 + 55}=\dfrac{59}{5}$.
	}
\end{ex}
\begin{ex}%[0D5B3-1]
	Kết quả khảo sát cân nặng của $25$ quả cam ở lô hàng $A$ được cho như sau
	\begin{center}
		\begin{tabular}{|l|c|c|c|c|c|}
			\hline
			Cân nặng (g) & $[150;155)$ & $[155;160)$ & $[160;165)$ & $[165;170)$ & $[170;175)$ \\
			\hline
			Số quả cam & $2$ & $6$ & $12$ & $4$ & $1$ \\
			\hline
		\end{tabular}	
	\end{center}
	Tính cân nặng trung bình của mỗi quả cam ở lô hàng $A$.
	\choice
	{\True $161{,}7$ (g)}
	{$161{,}7$ (kg)}
	{$155$ (g)}
	{$160$ (kg)}
	\loigiai{
		Ta có giá trị đại diện của các nhóm lần lượt là $152{,}5$; $157{,}5$; $162{,}5$; $167{,}5$; $172{,}5$.\\
		Vậy cân nặng trung bình của mỗi quả cam là
		$$\overline{x}=\dfrac{152{,}5\cdot 2+ 157{,}5\cdot 6+162{,}5\cdot 12+167{,}5\cdot 4+172{,}5\cdot 1}{25}=161{,}7 \text{(g).}$$
	}
\end{ex}
% \begin{ex}%[0D5K3-1]
% 	Ba nhóm học sinh gồm $10$ người, $15$ người, $25$ người. Khối lượng trung bình của mỗi nhóm lần lượt là $50$ kg, $38$ kg, $40$ kg. Khối lượng trung bình của cả ba nhóm học sinh là
% 	\choice
% 	{\True $41{,}4$ kg}
% 	{$42{,}4$ kg}
% 	{$26$ kg}
% 	{$37$ kg}
% 	\loigiai{
% 		Tổng khối lượng nhóm thứ nhất là $50\cdot 10=500$ (kg).\\
% 		Tổng khối lượng nhóm thứ hai là $38\cdot 15=570$ (kg).\\
% 		Tổng khối lượng nhóm thứ ba là $40\cdot 25=1000$ (kg).\\
% 		Tổng khối lượng cả ba nhóm là $500+570+1000=2070$ (kg).\\
% 		Tổng số người cả ba nhóm là $10+15+25=50$ (người).\\
% 		Khối lượng trung bình của cả ba nhóm học sinh là $\dfrac{2070}{50}=41{,}4$ (kg).
% 	}
% \end{ex}
\begin{ex}%[0D5K3-1]
	Sau một kì thi học sinh giỏi Toán, người ta thống kê kết quả (thang điểm $20$) và thu được bảng tần số sau
	\begin{center}
		\begin{tabular}{|l|c|c|c|c|}
			\hline
			Lớp điểm & $[6;10]$ & $[11;15]$ & $[16;20]$ & Cộng\\
			\hline
			Tần số & $22$ & $12$ & $6$ & $40$ \\
			\hline
		\end{tabular}
	\end{center}
	Nếu những học sinh chỉ cần đạt điểm trung bình của bảng điểm trên đều được nhận Giấy Khen của ban tổ chức, thì số học sinh được nhận Giấy Khen là bao nhiêu?
	\choice
	{$11$}
	{\True $18$}
	{$12$}
	{$6$}
	\loigiai{
		Ta lập lại bảng với thêm dòng giá trị đại diện
		\begin{center}
			\begin{tabular}{|l|c|c|c|c|}
				\hline
				Lớp điểm & $[6;10]$ & $[11;15]$ & $[16;20]$ & Cộng\\
				\hline
				Giá trị đại diện & $8$ & $13$ & $18$ & \\
				\hline
				Tần số & $22$ & $12$ & $6$ & $40$ \\
				\hline
			\end{tabular}
		\end{center}
		Điểm trung bình là $\overline{x}=\dfrac{22\cdot 8+12\cdot 13+6\cdot 18}{40}=11$.\\
		Vậy số học sinh được nhận thưởng là $12+6=18$ (học sinh).
	}
\end{ex}
% \begin{ex}%[0D5G3-1]
% 	Cho biết tình hình thu hoạch lúa vụ mùa năm $1980$ của ba hợp tác xã ở địa phương $V$ như sau
% 	\begin{center}
% 		\begin{tabular}{|l|c|c|c|}
% 			\hline
% 			Hợp tác xã & $A$ & $B$ & $C$ \\
% 			\hline
% 			Năng suất lúa (tạ/ha) & $40$ & $38$ & $36$ \\
% 			\hline
% 			Diện tích trồng lúa (ha) & $150$ & $130$ & $120$ \\
% 			\hline
% 		\end{tabular}	
% 	\end{center}
% 	Hãy tính năng suất lúa trung bình của vụ mùa năm $1980$ trong toàn bộ ba hợp tác xã kể trên.
% 	\choice
% 	{\True $38{,}15$ tạ/ha}
% 	{$38{,}05$ tạ/ha}
% 	{$38{,}10$ tạ/ha}
% 	{$38{,}20$ tạ/ha}
% 	\loigiai{
% 		Sản lượng lúa của hợp tác xã $A$ là $40\cdot 150=6000$ (tạ).\\
% 		Sản lượng lúa của hợp tác xã $B$ là $38\cdot 130=4940$ (tạ).\\
% 		Sản lượng lúa của hợp tác xã $C$ là $36\cdot 120=4320$ (tạ).\\
% 		Tổng sản lượng lúa của cả ba hợp tác xã là $6000+4940+4320=15260$ (tạ).\\
% 		Tổng diện tích trồng lúa của cả ba hợp tác xã là $150+130+120=400$ (ha).\\
% 		Vậy năng suất lúa trung bình của cả ba hợp tác xã là $\dfrac{15260}{400}=38{,}15$ tạ/ha.\\
% 		\textbf{\underline{Lưu ý}:} Không thể tính năng suất trung bình bằng cách $\dfrac{40+38+36}{3}=38$ (tạ/ha), vì khi chênh lệch diện tích càng lớn thì số trung bình càng không chính xác.
% 	}
% \end{ex}
% \begin{ex}%[0D5Y3-2]
% 	Tiền lương hàng tháng của $7$ nhân viên trong một công ty du lịch là $650$, $840$, $690$, $2500$, $720$, $670$, $3000$ (đơn vị: nghìn đồng). Tìm số trung vị của các số liệu thống kê đã cho.
% 	\choice
% 	{$690$}
% 	{$2500$}
% 	{\True $720$}
% 	{$670$}
% 	\loigiai{ 
% 		Sắp xếp thứ tự các số liệu thống kê, ta thu được dãy tăng các số liệu như sau: $650$, $670$, $690$, $720$, $840$, $2500$, $3000$ (nghìn đồng).\\
% 		Ta có số các số liệu thống kê là $n=7=2\cdot 3+1$ nên số trung vị là $M_e=x_4=720$.
% 	}
% \end{ex}
% \begin{ex}%[0D5Y3-2]
% 	Điểm học kì một của một học sinh được cho bởi bảng số liệu sau (Đơn vị: điểm)
% 	\begin{center}
% 		\begin{tabular}{|c|c|c|c|c|c|c|c|c|}
% 			\hline
% 			5& 6 &6&7& 7 &8 &8& 8,5&9\\
% 			\hline
% 		\end{tabular}
% 	\end{center}
% 	Số trung vị của bảng nói trên là
% 	\choice
% 	{$6$}
% 	{$9$}
% 	{\True $7$}
% 	{$8$}
% 	\loigiai{ Ta có $N=9$ là số lẻ. Số liệu thứ $\dfrac{N+1}{2} = 5$ là số trung vị.\\
% 		Do đó số trung vị là $M_e = 7$ (điểm).
% 	}
% \end{ex}
% \begin{ex}%[0D5Y3-2]
% 	Điều tra số học sinh giỏi khối $10$ của $15$ trường cấp ba trên địa bản tỉnh $A$, ta được bảng số liệu như sau
% 	\begin{center}
% 		\begin{tabular}{|c|c|c|c|c|c|c|c|c|c|c|c|c|c|c|}
% 			\hline
% 			22& 29 &29&29& 30 &31 &32& 32&33 &34 &34 &35 &35 &35 &36  \\
% 			\hline	
% 		\end{tabular}
% 	\end{center}
% 	Số trung vị của bảng nói trên là
% 	\choice
% 	{\True $8$}
% 	{$9$}
% 	{$6$}
% 	{$7$}
% 	\loigiai{ Ta có $N=15$ là số lẻ $\Rightarrow$ số liệu thứ $\dfrac{15+1}{2}=8$ là số trung vị.\\
% 		Vậy số trung vị là $M_e = 8$.
% 	}
% \end{ex}
% \begin{ex}%[0D5Y3-2]
% 	Cho mẫu số liệu thống kê $\{6;4;4;1;9;10;7\}$. Số liệu trung vị của mẫu số liệu thống kê trên là
% 	\choice
% 	{$1$}
% 	{\True $6$}
% 	{$4$}
% 	{$10$}
% 	\loigiai{
% 		Sắp xếp thành dãy không giảm: $1$, $4$, $4$, $6$, $7$, $9$, $10$.\\
% 		Từ dãy trên ta có số trung vị là số $6$ trong dãy trên.
% 	}
% \end{ex}
% \begin{ex}%[0D5B3-2]
% 	Điểm kiểm tra môn Toán của $10$ học sinh được cho như sau: $6$; $7$; $7$; $6$; $7$; $8$; $8$; $7$; $9$; $9$. Số trung vị của mẫu số liệu trên là	
% 	\choice
% 	{$6$}
% 	{\True $7$}
% 	{$8$}
% 	{$9$}
% 	\loigiai{
% 		Ta sắp xếp số liệu theo thứ tự không giảm như sau: $6$; $6$; $7$; $7$; $7$; $7$; $8$; $8$; $9$; $9$.\\
% 		Dãy số trên có tất cả $10$ giá trị, và $2$ giá trị chính giữa bằng $7$.\\
% 		Vậy số trung vị của mẫu số liệu trên là $\dfrac{7+7}{2}=7$.
% 	}
% \end{ex}
% \begin{ex}%[0D5B3-2]
% 	Một cửa hàng dép da đã thống kê cỡ dép của một số khách hàng nam cho kết quả như sau: $39$; $38$; $39$; $40$; $41$; $41$; $43$; $37$; $38$; $40$; $43$; $41$; $42$; $41$; $42$. Tìm trung vị của mẫu số liệu trên.
% 	\choice
% 	{$37$}
% 	{$39$}
% 	{\True $41$}
% 	{$43$}
% 	\loigiai{
% 		Ta sắp xếp số liệu theo thứ tự không giảm: $37$; $38$; $38$; $39$; $39$; $40$; $40$; $41$; $41$; $41$; $41$; $42$; $42$; $42$; $43$.\\
% 		Vì $n=15$ là số lẻ nên số trung vị là số chính giữa của dãy số liệu.\\
% 		Vậy trung vị là $M_e=41$.
% 	}
% \end{ex}
% \begin{ex}%[0D5B3-2]
% 	Điều tra số học sinh của $30$ lớp học, ta được bảng số liệu như sau
% 	\begin{center}
% 		\begin{tabular}{|c|c|c|c|c|c|c|c|c|c|c|c|c|c|c|}
% 			\hline
% 			35& 39 &39&40& 40 &41 &41& 41&41 &44 &44 &45 &45 &45 &46  \\
% 			\hline
% 			48 &48 &48&48& 49 &49 &49&49 &49 &49 &50 &50 &50 &50 &51  \\
% 			\hline	
% 		\end{tabular}
% 	\end{center}
% 	Số trung vị của bảng nói trên là
% 	\choice
% 	{$46$}
% 	{$49$}
% 	{\True $47$}
% 	{$48$}
% 	\loigiai{ Ta có $N=30$ là số chẵn. Số liệu thứ $15$ và $16$ lần lượt là $46$, $48$ là số trung vị.\\
% 		Vậy số trung vị là $M_e = \dfrac{46+48}{2}=47$ (học sinh).
% 	}
% \end{ex}
% \begin{ex}%[0D5B3-2]
% 	Cho bảng phân bố tần số
% 	\begin{center}
% 		\begin{tabular}{|l|c|c|c|c|c|c|}
% 			\hline
% 			Tuổi & $18$ & $19$ & $20$ & $21$ & $22$ & Cộng \\
% 			\hline
% 			Tần số & $10$ & $50$ & $70$ & $29$ & $10$ & $169$ \\
% 			\hline	
% 		\end{tabular}
% 	\end{center}
% 	Số trung vị của bảng phân bố tần số đã cho là
% 	\choice
% 	{$18$ tuổi}
% 	{\True $20$ tuổi}
% 	{$19$ tuổi}
% 	{$21$ tuổi}
% 	\loigiai{
% 		Sau khi sắp xếp các tuổi trên thành dãy không giảm, do có $169$ số nên số trung vị là số thứ $85$ trong dãy trên.\\
% 		Mà số thứ $85$ trong dãy là $20$. Vậy $M_e=20$.
% 	}
% \end{ex}
% \begin{ex}%[0D5B3-2]
% 	Để khảo sát kết quả thi tuyển sinh môn Toán trong kì thi tuyển sinh Đại học năm vừa qua của trường $A$, người điều tra chọn một mẫu gồm $100$ học sinh tham gia kì thi tuyển sinh đó. Điểm môn Toán (thang điểm $10$) của các học sinh này được cho ở bảng phân bố tần số sau đây
% 	\begin{center}
% 		\begin{tabular}{|l|c|c|c|c|c|c|c|c|c|c|c|c|c|}
% 			\hline
% 			Điểm & $0$ & $1$ & $2$ & $3$ & $4$ & $5$ & $6$ & $7$ & $8$ & $9$ & $10$ &  \\
% 			\hline
% 			Tần số & $1$ & $1$ & $3$ & $5$ & $8$ & $13$ & $19$ & $24$ & $14$ & $10$ & $2$ & $n=100$ \\
% 			\hline	
% 		\end{tabular}
% 	\end{center}
% 	Số trung vị của mẫu số liệu trên.
% 	\choice
% 	{\True $M_e=6{,}5$}
% 	{$M_e=7{,}5$}
% 	{$M_e=5{,}5$}
% 	{$M_e=6$}
% 	\loigiai{
% 		Do kích thước mẫu $n=100$ là một số chẵn nên số trung vị là trung bình cộng của hai giá trị đứng thứ $\dfrac{n}{20}=50$ và $\dfrac{n}{2}+1=51$.\\
% 		Do đó $M_e=\dfrac{6+7}{2}=6{,}5$.
% 	}
% \end{ex}
% \begin{ex}%[0D5B3-2]
% 	Số áo bán được trong một quý ở một cửa hàng bán áo sơ-mi nam được cho trong bảng sau
% 	\begin{center}
% 		\begin{tabular}{|l|c|c|c|c|c|c|c|c|}
% 			\hline
% 			Cỡ số & $36$ & $37$ & $38$ & $39$ & $40$ & $41$ & $42$ & Cộng \\
% 			\hline
% 			Số áo bán được & $13$ & $45$ & $126$ & $110$ & $126$ & $40$ & $5$ & $465$ \\
% 			\hline	
% 		\end{tabular}
% 	\end{center}
% 	Hãy tìm số trung vị của các số liệu thống kê trên.
% 	\choice
% 	{$37$}
% 	{$38$}
% 	{\True $39$}
% 	{$40$}
% 	\loigiai{
% 		Ta sắp xếp dãy số áo bán được theo dãy không giảm
% 		$$36, 36, 36, \ldots, 36, 37, 37, \ldots, 37, 38, 38, \ldots, 38, \ldots, 42, 42.$$
% 		Dãy trên gồm $465$ số nên số trung vị là số thứ $233$.\\
% 		Mà số thứ $233$ là số $39$. Vậy $M_e=233$.
% 	}
% \end{ex}
\begin{ex}
	Cho bảng tần số về cân nặng của 180 người dân trong một xã như sau: (đơn vị: kg)
	\begin{center}
	\begin{tabular}{|c|c|c|}
	\hline
	\textbf{Nhóm} & \textbf{Tần số} & \textbf{Tần số tích luỹ}\\ 
	\hline
	$\left[0;10\right)$ & $6$ & $6$\\
	$\left[10;20\right)$ & $15$ & $21$\\
	$\left[20;30\right)$ & $37$ & $58$\\
	$\left[30;40\right)$ & $48$ & $106$\\
	$\left[40;50\right)$ & $22$ & $128$\\
	$\left[50;60\right)$ & $29$ & $157$\\
	$\left[60;70\right)$ & $23$ & $180$\\

	\hline
	& $n = 180$ &\\
	\hline
\end{tabular}	
	\end{center}
	Tứ phân vị thứ nhất của mẫu số liệu trên là
	\choice
	{$56{,}486$ kg}
	{\True $26{,}486$ kg}
	{$25{,}496$ kg}
	{$36{,}486$ kg}
	\loigiai{
	Số phần tử của mẫu là $n=180$  và $\dfrac{n}{4}=\dfrac{\cdot 180}{4}=45$.\\ Ta có  $21<135<58$ nên nhóm $3$ là nhóm  đầu tiên có   tần số tích luỹ  lớn hơn hoặc bằng $45$.\\
	Xét nhóm $3$ là  nhóm $\left[20;30\right)$ có $s=20$, $h=10$, $n_3=37$, nhóm $2$ là nhóm có $cf_2=21$.\\
	Vậy $Q_1=20+\dfrac{45-21}{37}\cdot 10\approx 26{,}49$ (kg).
}
	\end{ex}

\begin{ex}
	Cho bảng tần số chiều cao của 46 học sinh nam của khối lớp $11$ như sau
	\begin{center}
		\begin{tabular}{|c|c|}
			\hline
			\textbf{Nhóm} & \textbf{Tần số} \\
			\hline
			$\left[155;160\right)$ & $3$ \\
			$\left[160;165\right)$ & $18$ \\
			$\left[165;170\right)$ & $10$ \\
			$\left[170;175\right)$ & $15$ \\
			
			\hline
			& $n = 46$ \\
			\hline
		\end{tabular}	
	\end{center}
	Xác định tứ phân vị thứ nhất của mẫu số liệu trên
\choice
{$161{,}36$}
{$161{,}63$}
{\True $162{,}36$}
{$162{,}63$}
\loigiai{
	Số phần tử của mẫu là $n=46$; $\dfrac{n}{4}=11{,}5$.\\
		Ta có $cf_1=3$, $cf_2=3+18=21$ và $6<11{,}5<21$ nên nhóm $2$ là nhóm đầu tiên có   tần số tích luỹ  lớn hơn hoặc bằng $11{,}5$.\\
		Xét nhóm $2$ là nhóm $\left[160;165\right)$, ta có $s=160$, $h=5$, $n_6=21$; nhóm $1$ có tần số tích luỹ bằng $6$.\\
		Vậy $Q_1=160+\dfrac{11{,}5-3}{18}\cdot 5=162{,}36$.
}
\end{ex}
\begin{ex}
	Cho bảng tần số chiều cao của 46 học sinh nam của khối lớp $11$ như sau
	\begin{center}
		\begin{tabular}{|c|c|}
			\hline
			\textbf{Nhóm} & \textbf{Tần số} \\
			\hline
			$\left[155;160\right)$ & $3$ \\
			$\left[160;165\right)$ & $18$ \\
			$\left[165;170\right)$ & $10$ \\
			$\left[170;175\right)$ & $15$ \\
			
			\hline
			& $n = 46$ \\
			\hline
		\end{tabular}	
	\end{center}
	Xác định tứ phân vị thứ ba của mẫu số liệu trên
	\choice
	{$162{,}36$}
	{$166{,}5$}
	{ $166$}
	{\True $171{,}16$}
	\loigiai{
		Số phần tử của mẫu là $n=46$; $\dfrac{3n}{4}=34{,}5$.\\
		Ta có $cf_3=3+18+10=31$, $cf_4=31+15=46$ và $31<34{,}5<46$ nên nhóm $4$ là nhóm đầu tiên có   tần số tích luỹ  lớn hơn hoặc bằng $34{,}5$.\\
		Xét nhóm $4$ là nhóm $\left[170;175\right)$, ta có $t=170$, $l=5$, $n_4=15$; nhóm $3$ có tần số tích luỹ bằng $31$.\\
		Vậy $Q_3=170+\dfrac{34{,}5-31}{15}\cdot 5=171{,}16$.
	}
\end{ex}
\begin{ex}
	\immini{Cho bảng tần số ghép nhóm số liệu thống kê  chiều cao  của $40$ mẫu cây ở  một vườn thực vật 	(đơn vị: centimét).\\
		Xác định tứ phân vị thứ hai của  số liệu ghép nhóm trên
\choice
{\True $56{,}43$}
{$56{,}34$}
{$46{,}43$}
{$36{,}43$}		

}{\begin{tabular}{|c|c|c|}
	\hline
	\textbf{Nhóm} & \textbf{Tần số} & \textbf{Tần số tích luỹ}\\ 
	\hline
	$\left[30;40\right)$ & $4$ & $4$\\
	$\left[40;50\right)$ & $10$ & $14$\\
	$\left[50;60\right)$ & $14$ & $28$\\
	$\left[60;70\right)$ & $6$ & $34$\\
	$\left[70;80\right)$ & $4$ & $38$\\
	$\left[80; 	90\right)$ & $2$ & $40$\\

	
	\hline
	& $n = 40$ &\\
	\hline
\end{tabular}}
\loigiai{
	Số phần tử của mẫu là $n=46$; $\dfrac{n}{2}=23$.\\
	Ta có $cf_2=14<23<cf_3=28$ nên nhóm $3$ là nhóm đầu tiên có tần số tích lũy lớn hơn hoặc bằng $23$.\\
	Xét nhóm $3$ là nhóm $[50;60)$ có $r=50$, $d=10$, $n_3=14$ và nhóm $2$ là nhóm $\left[40;50\right)$ có $cf_2=14$.\\
	Do đó $Q_2=50+\dfrac{23-14}{14}\cdot 10=56{,}43$.
}
\end{ex}
\begin{ex}
Một bảng xếp hạng đã tính điềm chuẩn hoá cho chỉ số nghiên cứu của một số trường đại học ở
Việt Nam và thu được kết quả sau
\begin{center}
\begin{tabular}{|c|c|c|c|c|c|c|}
	\hline
	\textbf{Điểm} & Dưới $20$  &  $[20;30)$&  $[30;40)$ &  $[40;60)$ &  $[60;80)$ &  $[80;100)$\\
	\hline
	Số điểm & $4$ & $19$&$6$&$2$&$3$&$1$\\

	
	
	\hline

\end{tabular}	
\end{center}
Xác định điểm ngưỡng đề đưa ra danh sách $25$\% trường đại học có chỉ số nghiên cứu tốt nhất Việt Nam.	
\choice
{$25{,}26$}
{\True $35{,}42$}
{$45{,}35$}
{$45{,}42$}	
\loigiai{
Điểm ngưỡng để đưa ra danh sách $25$\% trường đại học có chỉ số nghiên cứu tốt nhất Việt Nam là tứ phân
vị thứ ba.\\
Ta có  $n=35$ và $\dfrac{3n}{4}=26{,}25$.\\
Do $cf_2=4+19=23<26{,}25<cf_3=23+6=29$ nên nhóm $[30;40])$ là nhóm đầu tiên có tần số tích lũy lớn hơn hoặc bằng $23{,}25$.\\
Nhóm $[30;40])$ có $r=30$, $d=10$, $n_3=6$; nhóm $2$ có $cf_2=23$. Do đó
$$Q_3=30+\dfrac{26{,}25-23}{6}\cdot 10\approx 35{,}41.$$ 
Vậy để đưa ra danh sách $25$\% trường đại học có chỉ số nghiên cứu tốt nhất Việt Nam ta lấy các trường có
điểm chuẩn hóa trên $35{,}42$.
}
\end{ex}
% \begin{ex}%1%[1K3B9-3]
% 	Điểm thi môn Toán (thang điểm 100, điểm được làm tròn đến 1) của 60 thí sinh được cho trong bảng sau
% 	\begin{center}
% 		\begin{tabular}{|l|c|c|c|c|c|}
% 			\hline Điểm & $[0-9,5)$ & $[9,5-19,5)$ & $[19,5-29,5)$ & $[29,5-39,5)$ & $[39,5-49,5)$ \\
% 			\hline Số thí sinh & $1 $& $2$ & $4$ & $6$ & $15$ \\
% 			\hline Điểm & $[49,5-59,5)$ & $[59,5-69,5)$ & $[69,5-79,5)$ & $[79,5-89,5)$ & $[89,5-99,5)$ \\
% 			\hline Số thí & $12$ & $10$ & $6$ & $3$ & $1$ \\
% 			\hline
% 		\end{tabular}    
% 	\end{center}
% 	Tìm  tứ phân vị thứ hai của mẫu số liệu.
% 	\choice
% 	{\True $Q_2\approx 51,17$}
% 	{$Q_2\approx 51,67$}
% 	{$Q_2\approx 49,5$}
% 	{$Q_2\approx 41,3$}
% 	\loigiai{Cỡ mẫu là $n=60$.\\
% 		Tứ phân vị thứ nhất $Q_2$ là $\dfrac{x_{30}+x_{31}}{2}$. Do $x_{30}$, $x_{31}$ đều thuộc nhóm $[49,5 ; 59,5)$ nên nhóm này chứa $Q_2$. \\Do đó, $p=6 ; \;a_6=49,5 ;\; m_6=12 ; \;m_1+\ldots+m_5=28, \;a_7-a_6=10$ và ta có
% 		$$
% 		Q_2=49,5+\dfrac{\frac{60}{2}-28}{12}\cdot 10\approx51,17.
% 		$$
% 	}
% \end{ex}
% %2
% \begin{ex}%[1K3B9-3]
% 	Điểm thi môn Toán (thang điểm 100, điểm được làm tròn đến 1) của $60$ thí sinh được cho trong bảng sau
% 	\begin{center}
% 		\begin{tabular}{|l|c|c|c|c|c|}
% 			\hline Điểm & $[0-9,5)$ & $[9,5-19,5)$ & $[19,5-29,5)$ & $[29,5-39,5)$ & $[39,5-49,5)$ \\
% 			\hline Số thí sinh & $1 $& $2$ & $4$ & $6$ & $15$ \\
% 			\hline Điểm & $[49,5-59,5)$ & $[59,5-69,5)$ & $[69,5-79,5)$ & $[79,5-89,5)$ & $[89,5-99,5)$ \\
% 			\hline Số thí & $12$ & $10$ & $6$ & $3$ & $1$ \\
% 			\hline
% 		\end{tabular}    
% 	\end{center}
% 	Tìm  tứ phân vị thứ nhất của mẫu số liệu.
% 	\choice
% 	{\True $Q_1\approx 41,3$}
% 	{$Q_1\approx 51,67$}
% 	{$Q_1\approx 40,83$}
% 	{$Q_1\approx 51,17$}
% 	\loigiai{Cỡ mẫu là $n=60$.\\
% 		Tứ phân vị thứ nhất $Q_1$ là $\dfrac{x_{15}+x_{16}}{2}$. Do $x_{15}$, $x_{16}$ đều thuộc nhóm $[39,5-49,5)$ nên nhóm này chứa $Q_1$. \\Do đó, $p=5 ; \;a_5=39,5 ;\; m_5=15 ; \;m_1+\ldots+m_4=13, \;a_6-a_5=10$ và ta có
% 		$$
% 		Q_1=39,5+\dfrac{\frac{60}{4}-13}{15}\cdot 10\approx 40,83.
% 		$$
% 	}
% \end{ex}
%3
\begin{ex}%[1K3B9-3]
	Phỏng vấn một số học sinh khối 11 vể thời gian (giờ) ngủ của một buổi tối, thu được bảng số liệu như sau.
	\begin{center}
		\begin{tabular}{|l|c|c|c|c|c|}
			\hline Thời gian  (giờ)  &{$[4 ; 5)$}&{$[5 ; 6)$}&{$[6 ; 7)$}&{$[7 ; 8)$}&{$[8 ; 9)$}\\
			\hline Số học sinh & $6$ & $10$ & $13$ & $9$ & $7$ \\
			\hline
		\end{tabular}     
	\end{center}
	Hãy cho biết $75 \%$ học sinh khối 11 ngủ ít nhất bao nhiêu giờ?
	\choice
	{$7,675$}
	{\True $7,53$}
	{$8$}
	{ $7,9$}
	\loigiai{
		Cỡ mẫu là $n=45$.\\
		Gọi $x_1, \ldots, x_{45}$ là mẫu số liệu được sắp xếp theo thứ tự không giảm. Khi đó, trung vị là $x_{23}$. Do đó, tứ phân vị thứ ba $Q_3$ là $x_{34}$. Do $x_{34}$ đều thuộc nhóm $[7;8)$ nên nhóm này chứa $Q_3$. Do đó, $p=4 ; \;a_4=7 ;\; m_4=9 ; \;m_1+m_2+m_3=29 ; \;a_5-a_4=1$ và ta có
		$$
		Q_3=7+\dfrac{\frac{3 \cdot 45}{4}-29}{9}\cdot 1\approx7,53.
		$$ 
		Vậy $75\%$ học sinh khối 11 ngủ ít nhất $7,53$ giờ.
	}
\end{ex}
%4
% \begin{ex}%[1K3B9-3]
% 	Điểm thi môn Toán (thang điểm 100, điểm được làm tròn đến 1) của 60 thí sinh được cho trong bảng sau
% 	\begin{center}
% 		\begin{tabular}{|l|c|c|c|c|c|}
% 			\hline Điểm & $[0-9,5)$ & $[9,5-19,5)$ & $[19,5-29,5)$ & $[29,5-39,5)$ & $[39,5-49,5)$ \\
% 			\hline Số thí sinh & $1 $& $2$ & $4$ & $6$ & $15$ \\
% 			\hline Điểm & $[49,5-59,5)$ & $[59,5-69,5)$ & $[69,5-79,5)$ & $[79,5-89,5)$ & $[89,5-99,5)$ \\
% 			\hline Số thí & $12$ & $10$ & $6$ & $3$ & $1$ \\
% 			\hline
% 		\end{tabular}    
% 	\end{center}
% 	Tìm  tứ phân vị thứ ba của mẫu số liệu.
% 	\choice
% 	{$Q_3=41,3$}
% 	{$Q_3=51,67$}
% 	{$Q_3=45$}
% 	{\True $Q_3=65$}
% 	\loigiai{Cỡ mẫu là $n=60$.\\
% 		Với tứ phân vị thứ ba $Q_3$ là $\dfrac{x_{45}+x_{46}}{2}$. Do $x_{45}$, $x_{46}$ đều thuộc nhóm $[60 ; 70)$ nên nhóm này chứa $Q_3$. Do đó, $p=7 ; \;a_7=60 ;\; m_7=10 ; \;m_1+\ldots+m_6=40 ; \;a_8-a_7=10$ và ta có
% 		$$
% 		Q_3=59,5+\dfrac{\frac{3 \cdot 60}{4}-40}{10}\cdot 10=64,5.
% 		$$
% 	}
% \end{ex}
%5
\begin{ex}%[1K3B9-3]
	Một hãng xe ô tô thống kê lại số lần gặp sự cố về động cơ về động cơ của $100$ chiếc xe cùng loại sau 2 năm sử dụng đầu tiên ở dảng sau
	\begin{center}
		\begin{tabular}{|l|c|c|c|c|c|}
			\hline Số lần gặp sự cố  &{$[0,5;2,5)$}&{$[2,5;4,5)$}&{$[4,5;6,5)$}&{$[6,5 ; 8,5)$}&{$[8,5;10,5)$}\\
			\hline Số xe & $17$ & $33$ & $25$ & $20$ & $5$ \\
			\hline
		\end{tabular}     
	\end{center}
	Tìm tứ phân vị thứ nhất của mẫu số liệu.
	\choice
	{\True $Q_1\approx 4$}
	{$Q_1\approx 2,98$}
	{$Q_1\approx 2,5$}
	{$Q_1\approx 3,5$}
	\loigiai{
		Cỡ mẫu là $n=100$.\\
		Gọi $x_1, \ldots, x_{100}$ là mẫu số liệu được sắp xếp theo thứ tự không giảm. Khi đó, trung vị là $\dfrac{x_{50}+x_{51}}{2}$. 
		Do đó, tứ phân vị thứ nhất $Q_1$ là $\dfrac{x_{25}+x_{26}}{2}$. Do $x_{25}$, $x_{26}$ đều thuộc nhóm $[2,5;4,5)$ nên nhóm này chứa $Q_1$. \\Do đó, $p=2 ; \;a_2=2,5;\; m_2=33 ; \;m_1=17, \;a_3-a_2=2$ và ta có
		$$
		Q_1=2,5+\dfrac{\frac{100}{4}-17}{33}\cdot 2\approx 2,98.
		$$
	}    
\end{ex}
%6
% \begin{ex}%[1K3B9-3]
% 	Một hãng xe ô tô thống kê lại số lần gặp sự cố về động cơ về động cơ của $100$ chiếc xe cùng loại sau 2 năm sử dụng đầu tiên ở dảng sau
% 	\begin{center}
% 		\begin{tabular}{|l|c|c|c|c|c|}
% 			\hline Số lần gặp sự cố  &{$[0,5;2,5)$}&{$[2,5;4,5)$}&{$[4,5;6,5)$}&{$[6,5 ; 8,5)$}&{$[8,5;10,5)$}\\
% 			\hline Số xe & $17$ & $33$ & $25$ & $20$ & $5$ \\
% 			\hline
% 		\end{tabular}     
% 	\end{center}
% 	Tìm   tứ phân vị thứ hai của mẫu số liệu.
% 	\choice
% 	{\True $Q_2=4,5$}
% 	{$Q_2\approx 5,12$}
% 	{$Q_2\approx 4,89$}
% 	{$Q_2\approx 5,2$}
% 	\loigiai{
% 		Cỡ mẫu là $n=100$.\\
% 		Gọi $x_1, \ldots, x_{100}$ là mẫu số liệu được sắp xếp theo thứ tự không giảm. Khi đó, trung vị là $\dfrac{x_{50}+x_{51}}{2}$. Do $x_{50} \in [2,5;4,5)$, $x_{51} \in [4,5;6,5)$  nên tứ phân vị thứ hai của mẫu số liệu ghép nhóm là  $Q_2=4,5$. 
% 	}
% \end{ex}
%7
\begin{ex}%[1K3B9-3]
	Một hãng xe ô tô thống kê lại số lần gặp sự cố về động cơ về động cơ của $100$ chiếc xe cùng loại sau 2 năm sử dụng đầu tiên ở dảng sau
	\begin{center}
		\begin{tabular}{|l|c|c|c|c|c|}
			\hline Số lần gặp sự cố  &{$[0,5;2,5)$}&{$[2,5;4,5)$}&{$[4,5;6,5)$}&{$[6,5 ; 8,5)$}&{$[8,5;10,5)$}\\
			\hline Số xe & $17$ & $33$ & $25$ & $20$ & $5$ \\
			\hline
		\end{tabular}     
	\end{center}
	Tìm   tứ phân vị thứ ba của mẫu số liệu.  
	\choice
	{$Q_3=6,3$}
	{$Q_3=6,8$}
	{$Q_3=7,2$}
	{\True $Q_3=6,5$}
	\loigiai{ Cỡ mẫu là $n=100$.\\
		Với tứ phân vị thứ ba $Q_3$ là $\dfrac{x_{75}+x_{76}}{2}$. Do $x_{75} \in [4,5;6,5)$, $x_{76} \in [6,5 ; 8,5)$  nên tứ phân vị thứ ba của mẫu số liệu ghép nhóm là $Q_3=6,5$. 
		
	}
\end{ex}
%8
% \begin{ex}%[1K3B9-3]
% 	Lương tháng của một số nhân viên văn phòng được ghi lại như sau (đơn vị: triệu đồng)
% 	\begin{center}
% 		\begin{tabular}{|l|c|c|c|c|c|}
% 			\hline Lương tháng (triệu đồng)  &{$[6;8)$}&{$[8;10)$}&{$[10;12)$}&{$[12;14)$}\\
% 			\hline Số nhân viên & $3$ & $6$ & $8$ & $7$  \\
% 			\hline
% 		\end{tabular}     
% 	\end{center}  
% 	Tìm tứ phân vị thứ nhất của mẫu số liệu.
% 	\choice
% 	{\True $Q_1= 9$}
% 	{$Q_1= 8,5$}
% 	{$Q_1= 9,5$}
% 	{$Q_1= 8,2$}
% 	\loigiai{
% 		Cỡ mẫu là $n=24$.\\
% 		Gọi $x_1, \ldots, x_{24}$ là mẫu số liệu được sắp xếp theo thứ tự không giảm. Khi đó, trung vị là $\dfrac{x_{12}+x_{13}}{2}$. 
% 		Do đó, tứ phân vị thứ nhất $Q_1$ là $\dfrac{x_{6}+x_{7}}{2}$. Do $x_{6}$, $x_{7}$ đều thuộc nhóm $[8;10)$ nên nhóm này chứa $Q_1$. \\Do đó, $p=2 ; \;a_2=8;\; m_2=6 ; \;m_1=3, \;a_3-a_2=2$ và ta có
% 		$$
% 		Q_1=8+\dfrac{\frac{24}{4}-3}{6}\cdot 2=9.
% 		$$
% 	}    
% \end{ex}
% %9
% \begin{ex}%[1K3B9-3]
% 	Lương tháng của một số nhân viên văn phòng được ghi lại như sau (đơn vị: triệu đồng)
% 	\begin{center}
% 		\begin{tabular}{|l|c|c|c|c|c|}
% 			\hline Lương tháng (triệu đồng)  &{$[6;8)$}&{$[8;10)$}&{$[10;12)$}&{$[12;14)$}\\
% 			\hline Số nhân viên & $3$ & $6$ & $8$ & $7$  \\
% 			\hline
% 		\end{tabular}     
% 	\end{center}   
% 	Tìm   tứ phân vị thứ hai của mẫu số liệu.
% 	\choice
% 	{\True $Q_2=10,75$}
% 	{$Q_2= 10,5$}
% 	{$Q_2= 11$}
% 	{$Q_2=11,5$}
% 	\loigiai{
% 		Cỡ mẫu là $n=24$.\\
% 		Gọi $x_1, \ldots, x_{24}$ là mẫu số liệu được sắp xếp theo thứ tự không giảm. Khi đó, trung vị là $\dfrac{x_{12}+x_{13}}{2}$. 
% 		Do đó,  tứ phân vị thứ hai $Q_2$ là $\dfrac{x_{12}+x_{13}}{2}$. Do $x_{12}$, $x_{13}$ đều thuộc nhóm $[10;12)$ nên nhóm này chứa $Q_2$. \\Do đó, $p=3 ; \;a_3=10;\; m_3=8 ; \;m_1+m_2=9, \;a_4-a_3=2$ và ta có
% 		$$
% 		Q_2=10+\dfrac{\frac{24}{2}-9}{8}\cdot 2=10,75.
% 		$$
% 	}
% \end{ex}
% %10
% \begin{ex}%[1K3B9-3]
% 	Lương tháng của một số nhân viên văn phòng được ghi lại như sau (đơn vị: triệu đồng)
% 	\begin{center}
% 		\begin{tabular}{|l|c|c|c|c|c|}
% 			\hline Lương tháng (triệu đồng)  &{$[6;8)$}&{$[8;10)$}&{$[10;12)$}&{$[12;14)$}\\
% 			\hline Số nhân viên & $3$ & $6$ & $8$ & $7$  \\
% 			\hline
% 		\end{tabular}     
% 	\end{center}  
% 	Tìm   tứ phân vị thứ ba của mẫu số liệu.
% 	\choice 
% 	{$Q_3\approx 12,5$}
% 	{$Q_3\approx 13,2$}
% 	{$Q_3\approx 13,5$}
% 	{\True $Q_3\approx 12,3$}
% 	\loigiai{
% 		Cỡ mẫu là $n=24$.\\
% 		Gọi $x_1, \ldots, x_{24}$ là mẫu số liệu được sắp xếp theo thứ tự không giảm. Khi đó, trung vị là $\dfrac{x_{12}+x_{13}}{2}$. 
% 		Do đó,  tứ phân vị thứ ba $Q_3$ là $\dfrac{x_{18}+x_{19}}{2}$. Do $x_{18}$, $x_{19}$ đều thuộc nhóm $[12;14)$ nên nhóm này chứa $Q_3$. \\Do đó, $p=4 ; \;a_4=12;\; m_4=7 ; \;m_1+m_2+m_3=17, \;a_4-a_3=2$ và ta có
% 		$$
% 		Q_2=12+\dfrac{\frac{24\cdot 3}{4}-17}{7}\cdot 2\approx 12,3.
% 		$$
% 	}
% \end{ex}
%11
% \begin{ex}%[1K3B9-3]
% 	Số điểm một cầu thủ bóng rổ ghi được trong 20 trận đấu được cho ở bảng sau
% 	\begin{center}
% 		\begin{tabular}{|l|c|c|c|c|c|}
% 			\hline Điểm số  &{$[5,5;10,5)$}&{$[10,5;15,5)$}&{$[15,5;20,5)$}&{$[20,5;25,5)$}\\
% 			\hline Số trận & $3$ & $9$ & $2$ & $6$  \\
% 			\hline
% 		\end{tabular}     
% 	\end{center}  
% 	Tìm   tứ phân vị thứ ba của mẫu số liệu.
% 	\choice 
% 	{$Q_3\approx 23,5$}
% 	{$Q_3\approx 22,2$}
% 	{$Q_3\approx 21,6$}
% 	{\True $Q_3\approx 21,3$}
% 	\loigiai{
% 		Cỡ mẫu là $n=20$.\\
% 		Gọi $x_1, \ldots, x_{20}$ là mẫu số liệu được sắp xếp theo thứ tự không giảm. Khi đó, trung vị là $\dfrac{x_{10}+x_{11}}{2}$. 
% 		Do đó,  tứ phân vị thứ ba $Q_3$ là $\dfrac{x_{15}+x_{16}}{2}$. Do $x_{15}$, $x_{16}$ đều thuộc nhóm $[20,5;25,5)$ nên nhóm này chứa $Q_3$. \\Do đó, $p=4 ; \;a_4=20,5;\; m_4=6 ; \;m_1+m_2+m_3=14, \;a_5-a_4=25,5-20,5=5$ và ta có
% 		$$
% 		Q_2=20,5+\dfrac{\frac{20\cdot 3}{4}-14}{6}\cdot 5 \approx 21,3.
% 		$$
% 	}
% \end{ex}
% %12
% \begin{ex}%[1K3B9-3]
% 	Số điểm một cầu thủ bóng rổ ghi được trong 20 trận đấu được cho ở bảng sau
% 	\begin{center}
% 		\begin{tabular}{|l|c|c|c|c|c|}
% 			\hline Điểm số  &{$[5,5;10,5)$}&{$[10,5;15,5)$}&{$[15,5;20,5)$}&{$[20,5;25,5)$}\\
% 			\hline Số trận & $3$ & $9$ & $2$ & $6$  \\
% 			\hline
% 		\end{tabular}     
% 	\end{center} 
% 	Tìm tứ phân vị thứ nhất của mẫu số liệu.
% 	\choice
% 	{\True $Q_1\approx 11,6$}
% 	{$Q_1\approx 11,3$}
% 	{$Q_1\approx 21,6$}
% 	{$Q_1\approx 21,3$}
% 	\loigiai{
% 		Cỡ mẫu là $n=20$.\\
% 		Gọi $x_1, \ldots, x_{20}$ là mẫu số liệu được sắp xếp theo thứ tự không giảm. Khi đó, trung vị là $\dfrac{x_{10}+x_{11}}{2}$. 
% 		Do đó, tứ phân vị thứ nhất $Q_1$ là $\dfrac{x_{5}+x_{6}}{2}$. Do $x_{5}$, $x_{6}$ đều thuộc nhóm $[10,5;15,5)$ nên nhóm này chứa $Q_1$. \\Do đó, $p=2 ; \;a_2=10,5;\; m_2=9 ; \;m_1=3, \;a_3-a_2=5$ và ta có
% 		$$
% 		Q_1=10,5+\dfrac{\frac{20}{4}-3}{9}\cdot 5\approx 11,6.
% 		$$
% 	}
% \end{ex}
%13
\begin{ex}%[1K3B9-3]
	Số điểm một cầu thủ bóng rổ ghi được trong 20 trận đấu được cho ở bảng sau
	\begin{center}
		\begin{tabular}{|l|c|c|c|c|c|}
			\hline Điểm số  &{$[5,5;10,5)$}&{$[10,5;15,5)$}&{$[15,5;20,5)$}&{$[20,5;25,5)$}\\
			\hline Số trận & $3$ & $9$ & $2$ & $6$  \\
			\hline
		\end{tabular}     
	\end{center} 
	Tìm tứ phân vị thứ nhất của mẫu số liệu.
	\choice
	{\True $Q_1\approx 11,6$}
	{$Q_1\approx 14,4$}
	{$Q_1\approx 15,6$}
	{$Q_1\approx 21,3$}
	\loigiai{
		Cỡ mẫu là $n=20$.\\
		Gọi $x_1, \ldots, x_{20}$ là mẫu số liệu được sắp xếp theo thứ tự không giảm. Khi đó, trung vị là $\dfrac{x_{10}+x_{11}}{2}$. 
		Do đó, tứ phân vị thứ nhất $Q_2$ là $\dfrac{x_{10}+x_{11}}{2}$. Do $x_{10}$, $x_{11}$ đều thuộc nhóm $[10,5;15,5)$ nên nhóm này chứa $Q_2$. \\Do đó, $p=2 ; \;a_2=10,5;\; m_2=9 ; \;m_1=3, \;a_3-a_2=5$ và ta có
		$$
		Q_1=10,5+\dfrac{\frac{20}{2}-3}{9}\cdot 5\approx 14,4.
		$$   
	}
\end{ex}
%14
% \begin{ex}%[1K3B9-3]
% 	Một người thống kê lại thời gian thực hiện các cuộc gọi điện thoại của người đó trong một tuần cho trong bảng sau
% 	\begin{center}
% 		\begin{tabular}{|l|c|c|c|c|c|}
% 			\hline Số bệnh nhân &{$[0;60)$}&{$[60;120)$}&{$[120;180)$}&{$[180;240)$}&{$[240;300)$}\\
% 			\hline Số ngày & $8$ & $10$ & $7$ & $5$ & $2$ \\
% 			\hline
% 		\end{tabular}     
% 	\end{center}
% 	Tìm   tứ phân vị thứ ba của mẫu số liệu.  
% 	\choice
% 	{$Q_3\approx 175,28$}
% 	{$Q_3\approx 150,32 $}
% 	{$Q_3=175$}
% 	{\True $Q_3\approx 171,43$}
% 	\loigiai{ Cỡ mẫu là $n=32$.\\
% 		Gọi $x_1, \ldots, x_{32}$ là mẫu số liệu được sắp xếp theo thứ tự không giảm. Khi đó, trung vị là $\dfrac{x_{16}+x_{17}}{2}$.\\
% 		Do đó, tứ phân vị thứ ba $Q_3$ là $\dfrac{x_{24}+x_{25}}{2}$. Do $x_{24} $, $x_{25} \in [120;180)$   nên nhóm này chứa $Q_3$. \\Do đó, $p= 3; \;a_3=120 ;\; m_3=7 ; \;m_1+m_2=18 ; \;a_4-a_3=60$ và ta có
% 		$$
% 		Q_3=120+\dfrac{\frac{3 \cdot 32}{4}-18}{7}\cdot 60\approx 171,43.
% 		$$
% 	}
% \end{ex}
% %15
% \begin{ex}%[1K3B9-3]
% 	Một người thống kê lại thời gian thực hiện các cuộc gọi điện thoại của người đó trong một tuần cho trong bảng sau
% 	\begin{center}
% 		\begin{tabular}{|l|c|c|c|c|c|}
% 			\hline Số bệnh nhân &{$[0;60)$}&{$[60;120)$}&{$[120;180)$}&{$[180;240)$}&{$[240;300)$}\\
% 			\hline Số ngày & $8$ & $10$ & $7$ & $5$ & $2$ \\
% 			\hline
% 		\end{tabular}     
% 	\end{center}
% 	Tìm   tứ phân vị thứ hai của mẫu số liệu.  
% 	\choice
% 	{$Q_2\approx 80,25$}
% 	{$Q_2\approx 100,32$}
% 	{$Q_2=115$}
% 	{\True $Q_2=108$}
% 	\loigiai{ Cỡ mẫu là $n=32$.\\
% 		Gọi $x_1, \ldots, x_{32}$ là mẫu số liệu được sắp xếp theo thứ tự không giảm. Khi đó, trung vị là $\dfrac{x_{16}+x_{17}}{2}$.\\
% 		Do đó, tứ phân vị thứ hai $Q_2$ là $\dfrac{x_{16}+x_{17}}{2}$. Do $x_{16} $, $x_{17} \in [60;120)$   nên nhóm này chứa $Q_2$. \\Do đó, $p= 2; \;a_2=60 ;\; m_2=10 ; \;m_1=8 ; \;a_3-a_2=60$ và ta có
% 		$$
% 		Q_3=60+\dfrac{\frac{ 32}{2}-8}{10}\cdot 60=108.
% 		$$
% 	}    
% \end{ex}
% \begin{ex}
% 	Một công ty xây dựng khảo sát khách hàng xem họ có nhu cầu mua nhà ở mức giá nào. Kết quả khảo sát được ghi lại ở bảng sau
% 	\begin{center}
% 		\begin{tabular}{|c|c|c|c|c|c|}
% 			\hline \begin{tabular}{c}
% 				\textbf{Mức giá} \\
% 				\textbf{(triệu đồng/$\mathrm{m}^2)$}
% 			\end{tabular} &{$[10; 14)$} &{$[14; 18)$} &{$[18; 22)$} &{$[22; 26)$} &{$[26; 30)$} \\
% 			\hline \textbf{Số khách hàng} & 54 & 78 & 120 & 45 & 12 \\
% 			\hline
% 		\end{tabular}
% 	\end{center}
	
% 	 Công ty nên xây nhà ở mức giá nào để nhiều người có nhu cầu mua nhất?
% 	\choice
% 	{\True $19{,}4$ triệu đồng$/ \mathrm{m}^2$ }
% 	{$20{,}4$ triệu đồng$/ \mathrm{m}^2$ }
% 	{$19{,}6$ triệu đồng$/ \mathrm{m}^2$ }
% 	{$20{,}6$ triệu đồng$/ \mathrm{m}^2$ }	
% 	\loigiai{
% 		\ Nhóm chứa mốt của mẫu số liệu trên là nhóm $[18; 22)$.\\ Do đó $u_m=18$, $n_{m-1}=78$, $n_m=120$, $n_{m+1}=45$, $u_{m+1}-u_m=22-18=4$.\\
% 			Mốt của mẫu số liệu ghép nhóm là
% 			\[M_0=18+\dfrac{120-78}{(120-78)+(120-45)} \cdot 4=\dfrac{758}{39} \approx 19{,}4. \]
% 		 Dựa vào kết quả trên ta có thể dự đoán rằng nếu công ty xây nhà ở mức giá $19{,}4$ triệu đồng$/ \mathrm{m}^2$ thì sẽ có nhiều người có nhu cầu mua nhất.
		
% 	}
% \end{ex}
\begin{ex}%[1T5B1-3]
	Số cuộc gọi điện thoại một nguời thực hiện mỗi ngày trong $30$ ngày được lựa chọn ngẫu nhiên được thống kê trong bảng sau:
\begin{center}
	\begin{tabular}{|c|c|c|c|c|c|}
		\hline Số cuộc gọi &{$[3; 5]$} &{$[6; 8]$} &{$[9; 11]$} &{$[12; 14]$} &{$[15; 17]$} \\
		\hline Số ngày & 5 & 13 & 7 & 3 & 2 \\
		\hline
	\end{tabular}
\end{center}
 Tìm mốt của mẫu số liệu ghép nhóm trên.
 Hãy dự đoán xem khả năng người đó thực hiện bao nhiêu cuộc gọi mỗi ngày là cao nhất.
\choice
{$4$}
{$6$}
{$5$}
{\True $7$}	
\loigiai{
	Do số cuộc gọi là số nguyên nên ta hiệu chỉnh lại như sau:
	\begin{center}
		\begin{tabular}{|c|c|c|c|c|c|}
			\hline Số cuộc gọi &{$[2{,}5; 5{,}5)$} &{$[5{,}5; 8{,}5)$} &{$[8{,}5; 11{,}5)$} &{$[11{,}5; 14{,}5)$} &{$[14{,}5; 17{,}5)$} \\
			\hline Số ngày & 5 & 13 & 7 & 3 & 2 \\
			\hline
		\end{tabular}
	\end{center}	
		 Nhóm chứa mốt của mẫu số liệu trên là nhóm $[5{,}5; 8{,}5)$.\\
		Do đó $u_m=5{,}5$; $n_{m-1}=5$; $n_m=13$; $n_{m+1}=7$; $u_{m+1}-u_m=8{,}5-5{,}5=3$.\\
		Mốt của mẫu số liệu ghép nhóm là
		\[M_0=5{,}5+\dfrac{13-5}{(13-5)+(13-7)} \cdot 3=\dfrac{101}{14} \approx 7{,}2. \]
	 Dựa vào kết quả trên ta có thể dự đoán rằng khả năng người đó thực hiện $7$ cuộc gọi mỗi ngày là cao nhất.
	
}
\end{ex}
\begin{ex}
	Một thư viện thống kê số lượng sách được mượn mỗi ngày trong ba tháng ở bảng sau:
	\begin{center}
		\begin{tabular}{|c|c|c|c|c|c|c|c|}
			\hline Số sách &{$[16; 20]$} &{$[21; 25]$} &{$[26; 30]$} &{$[31; 35]$} &{$[36; 40]$} &{$[41; 45]$} &{$[46; 50]$} \\
			\hline Số ngày & 3 & 6 & 15 & 27 & 22 & 14 & 5 \\
			\hline
		\end{tabular}
	\end{center}
	Hãy ước lượng  mốt của mẫu số liệu ghép nhóm trên.
	\choice
	{$34{,}33$}
	{\True $34{,}03$}
	{$35{,}63$}
	{$34{,}13$}	
	\loigiai{
		Vì số lượng sách được mượn là số nguyên nên ta hiệu chỉnh bảng tần số ghép nhóm (theo giá trị đại diện) như sau
		\begin{center}
			{\footnotesize \begin{tabular}{|c|c|c|c|c|c|c|c|}
					\hline Số sách &{$[15{,}5; 20{,}5)$} &{$[20{,}5; 25{,}5)$} &{$[25{,}5; 30{,}5)$} &{$[30{,}5; 35{,}5)$} &{$[35{,}5; 40{,}5]$} &{$[40{,}5; 45{,}5)$} &{$[45{,}5; 50{,}5)$} \\
					\hline Giá trị đại diện &{$18$} &{$23$} &{$28$} &{$33$} &{$38$} &{$43$} &{$48$} \\
					\hline Số ngày & 3 & 6 & 15 & 27 & 22 & 14 & 5 \\
					\hline
			\end{tabular}}
		\end{center}
		Trung bình số lượng sách được mượn mỗi ngày trong 3 tháng của thư viện là
		\[\overline{x}=\dfrac{18\cdot 3+23\cdot 6+28\cdot 15+33\cdot 27+38\cdot 22+43\cdot 14+48\cdot 5}{92}\approx 34{,}58. \]
		Nhóm chứa mốt của mẫu số liệu trên là nhóm $[30{,}5; 35{,}5)$.\\
		Do đó $u_m=30{,}5$; $n_{m-1}=15$; $n_m=27$; $n_{m+1}=22$; $u_{m+1}-u_m=35{,}5-30{,}5=5$.\\
		Mốt của mẫu số liệu ghép nhóm là
		\[M_0=30{,}5+\dfrac{27-15}{(27-15)+(27-22)} \cdot 5\approx 34{,}03. \]
	}
\end{ex}
\begin{ex}
	Kết quả đo chiều cao của $200$ cây keo $3$ năm tuổi ở một nông trường được biểu diễn ở biểu đồ dưới đây.
	\begin{center}
		\begin{tikzpicture}[scale=1,font=\scriptsize]
		\def\hoanh{11.5};
		\def\tung{6.5};
		\def\mau{cyan};
		\foreach \x/\n in{1/20,3/35,5/60,7/55,9/30}{\draw[line width=16mm,\mau] (\x,0)--++(0,{\n/10});
			\draw[dashed] (\x,{\n/10})node[above]{$\n$}--(0,{\n/10}) node[left]{$\n$};}
		\foreach \x/\p in {1/[8{,}5;8{,}8),3/[8{,}8;9{,}1),5/[9{,}1;9{,}4),7/[9{,}4;9{,}7),9/[9{,}7;10{,}0)}{\node[below] at (\x,0){\scriptsize $\p$};}
		\draw[->] (0,0)--(\hoanh,0) node[below]{($m$)};
		\draw[->] (0,0)node[below left]{$O$}--(0,\tung) node[left]{(Số cây)};
		\path (current bounding box.north) node[above]		{\textbf{Chiều cao 200 cây keo 3 năm tuổi}};
		\end{tikzpicture}
	\end{center}
	Mốt của mẫu số liệu ghép nhóm trên là
	\choice
	{$9{,}35$}
	{$10{,}53$}
	{$10{,}35$}
	{$9{,}53$}	
	\loigiai{
		Bảng tần số ghép nhóm (theo giá trị đại diện)
		\begin{center}
			\begin{tabular}{|c|c|c|c|c|c|}
				\hline Chiều cao &$[8{,}5; 8{,}8)$ &{$[8{,}8; 9{,}1)$} &{$[9{,}1; 9{,}4)$} &{$[9{,}4; 9{,}7)$} &{$[9{,}7; 10{,}0)$} \\
				\hline Giá trị đại diện &$8{,}65$ &$8{,}95$ &$9{,}25$ &$9{,}55$ &$9{,}85$ \\
				\hline Số cây & $20$ & $35$ & $60$ & $55$ & $30$\\
				\hline
			\end{tabular}
		\end{center}
		Chiều cao trung bình của $200$ cây keo 3 năm tuổi là
		\[\overline{x}=\dfrac{8{,}65\cdot 20+8{,}95\cdot 35+9{,}25\cdot 60+9{,}55\cdot 55+9{,}85\cdot 30}{200}\approx 9{,}31. \]
		Nhóm chứa mốt của mẫu số liệu trên là nhóm $[9{,}1; 9{,}4)$.\\
		Do đó $u_m=9{,}1$; $n_{m-1}=35$; $n_m=60$; $n_{m+1}=55$; $u_{m+1}-u_m=9{,}4-9{,}1=0{,}3$.\\
		Mốt của mẫu số liệu ghép nhóm là
		\[M_0=9{,}1+\dfrac{60-35}{(60-35)+(60-55)} \cdot 0{,}3= 9{,}35. \]
	}
\end{ex}
\begin{ex}%[1K3B9-4]
	Bảng số liệu ghép nhóm sau cho biết chiều cao (cm) của $50$ học sinh lớp $11A$.
	\begin{center}
		\begin{tabular}{|c|c|c|c|c|c|c|}
			\hline
			Khoảng chiều cao (cm)	& $\left[145;150 \right)$ & $\left[150;155 \right)$ & $\left[155;160 \right)$ & $\left[160;165 \right)$&$\left[165;170 \right)$  \\
			\hline
			Số học sinh&$7$	& $14$ & $10$ &$10$  & $9$ \\
			\hline
		\end{tabular}
	\end{center}
Mốt của mẫu số liệu ghép nhóm là
	\choice
	{$154{,}20$}
	{\True $153{,}18$}
	{$155{,}12$}
	{$158{,}36$}	
	\loigiai{
		Tần số lớn nhất là $14$ nên nhóm chứa mốt là nhóm $\left[150;155 \right)$. Ta có $j=2$, $a_2=150$, $m_2=14$, $m_1=7$, $m_3=10$, $h=5$. Do đó $$M_o=150+\dfrac{14-7}{\left(14-7\right)+\left(14-10\right)\cdot 5}\approx 153{,}18.$$	
		Số học sinh có chiều cao khoảng $153{,}18$ là nhiều nhất.
	}
\end{ex}
\begin{ex}%[1K3Y9-4]
	Chọn khẳng định \textbf{sai}.
	\choice
	{ Mốt của mẫu số liệu không ghép nhóm là giá trị có khả năng xuất hiện cao nhất khi lấy mẫu}
	{Mốt của mẫu số liệu sau khi ghép nhóm xấp xỉ với mốt của mẫu số liệu không ghép nhóm}
	{\True Một mẫu số liệu ghép nhóm chỉ có một mốt}
	{Một mẫu số liệu ghép nhóm có thể có nhiều nhóm chứa mốt và nhiều mốt}
	\loigiai{
		Mốt của mẫu số liệu không ghép nhóm là giá trị có khả năng xuất hiện cao nhất khi lấy mẫu.\\ Mốt của mẫu số liệu sau khi ghép nhóm xấp xỉ với mốt của mẫu số liệu không ghép nhóm. \\
		Một mẫu số liệu ghép nhóm có thể có nhiều nhóm chứa mốt và nhiều mốt.\\
		Do đó khẳng định sai là: Một mẫu số liệu ghép nhóm chỉ có một mốt.
	}    
\end{ex}
% \begin{ex}%[1K3Y9-4]
% 	Người ta ghi lại tuổi thọ của một số con ong cho kết quả như sau:
% 	\begin{center}
% 		\begin{tabular}{|l|c|c|c|c|c|c|}
% 			\hline Tuồi thọ (ngày) &{$[0;20)$}&{$[20;40)$}&{$[40;60)$}&{$[60;80)$}&{$[80;100)$}\\
% 			\hline Số lượng & $5$ & $12$ & $23$ & $31$ & $29$  \\
% 			\hline
% 		\end{tabular}
% 	\end{center}
% 	Nhóm chứa mốt của mẫu số liệu này là
% 	\choice
% 	{ $[20;40)$}
% 	{$[40;60)$}
% 	{\True $[60;80)$}
% 	{$[80;100)$}
% 	\loigiai{
% 		Nhóm chứa mốt của mẫu số liệu này là $[60;80)$.
% 	}    
% \end{ex}
% %6
% \begin{ex}%[1K3B9-4]
% 	Người ta ghi lại tuổi thọ của một số con ong cho kết quả như sau:
% 	\begin{center}
% 		\begin{tabular}{|l|c|c|c|c|c|c|}
% 			\hline Tuồi thọ (ngày) &{$[0;20)$}&{$[20;40)$}&{$[40;60)$}&{$[60;80)$}&{$[80;100)$}\\
% 			\hline Số lượng & $5$ & $12$ & $23$ & $31$ & $29$  \\
% 			\hline
% 		\end{tabular}
% 	\end{center}
% 	Mốt của mẫu số liệu này là
% 	\choice
% 	{ $M_0=70$}
% 	{$M_0=60$}
% 	{\True $M_0=76$}
% 	{$M_0=31$}
% 	\loigiai{
% 		Tần số lớn nhất là $31$ nên nhóm chứa mốt là nhóm $[60;80)$. \\
% 		Ta có, $j=4, a_4=60, m_4=31$, $m_3=23, m_5=29, h=20$. Do đó
% 		$$
% 		M_0=60+\frac{31-23}{(31-29)+(14-11)}\cdot 20 =76.
% 		$$
% 	}   
% \end{ex}
% %7
% \begin{ex}%[1K3Y9-4]
% 	Doanh thu bán hàng 20 ngày được lựa chọn ngẫu nhiên của một cửa hàng được ghi lại ở bảng sau (đơn vị: triệu đồng)
% 	\begin{center}
% 		\begin{tabular}{|l|c|c|c|c|c|c|}
% 			\hline Doanh thu &{$[5;7)$}&{$[7;9)$}&{$[9;11)$}&{$[11;13)$}&{$[13;15)$}\\
% 			\hline Số ngày & $2$ & $9$ & $7$ & $3$ & $1$ \\
% 			\hline
% 		\end{tabular}
% 	\end{center}
% 	Nhóm chứa mốt của mẫu số liệu này là
% 	\choice
% 	{ $[9;11)$}
% 	{$[5;7)$}
% 	{\True $[7;9)$}
% 	{$[11;13)$}
% 	\loigiai{
% 		Tần số lớn nhất là $9$ nên nhóm chứa mốt là nhóm $[7;9)$. \\
% 	}   
% \end{ex}
% %8
% \begin{ex}%[1K3B9-4]
% 	Doanh thu bán hàng 20 ngày được lựa chọn ngẫu nhiên của một cửa hàng được ghi lại ở bảng sau (đơn vị: triệu đồng)
% 	\begin{center}
% 		\begin{tabular}{|l|c|c|c|c|c|c|}
% 			\hline Doanh thu &{$[5;7)$}&{$[7;9)$}&{$[9;11)$}&{$[11;13)$}&{$[13;15)$}\\
% 			\hline Số ngày & $2$ & $9$ & $7$ & $3$ & $1$ \\
% 			\hline
% 		\end{tabular}
% 	\end{center}
% 	Xác định mốt của mẫu số liệu.
% 	\choice
% 	{ $M_0\approx 8$}
% 	{$M_0\approx 8,5$}
% 	{\True $M_0\approx 8,56$}
% 	{$M_0\approx 9$}
% 	\loigiai{
% 		Tần số lớn nhất là $9$ nên nhóm chứa mốt là nhóm $[7;9)$. \\
% 		Ta có, $j=2, a_2=7, m_2=9$, $m_1=2, m_3=7, h=2$. Do đó
% 		$$
% 		M_0=7+\frac{9-2}{(9-2)+(9-7)}\cdot 2\approx 8,56.
% 		$$
% 	}    
% \end{ex}
% %9
% \begin{ex}%[1K3B9-4]
% 	Điểm kiểm tra môn Toán của lớp 12A được cho trong bảng sau
% 	\begin{center}
% 		\begin{tabular}{|l|c|c|c|c|c|c|c|c|}
% 			\hline Khoảng điểm &{$[6,5;7)$}&{$[7;7,5)$}&{$[7,5;8)$}&{$[8;8,5)$}&{$[8,5;9)$}&{$[9;9,5)$}&{$[9,5;10)$}\\
% 			\hline Tần số & $8$ & $10$ & $16$ & $24$& $13$ & $7$ & $4$ \\
% 			\hline
% 		\end{tabular}
% 	\end{center}
% 	Xác định mốt của mẫu số liệu ghép nhóm này.    
% 	\choice
% 	{ $M_0\approx 8,4$}
% 	{$M_0\approx 8,5$}
% 	{\True $M_0\approx 8,21$}
% 	{$M_0\approx 24$}
% 	\loigiai{
% 		Tần số lớn nhất là $24$ nên nhóm chứa mốt là nhóm $[8;8,5)$. \\
% 		Ta có, $j=4, a_4=8, m_4=24$, $m_3=16, m_5=13, h=0,5$. Do đó
% 		$$
% 		M_0=8+\frac{24-16}{(24-16)+(24-13)}\cdot 0,5\approx 8,21.
% 		$$
% 	}
% \end{ex}
% %10
% \begin{ex}%[1K3Y9-4]
% 	Điểm kiểm tra môn Toán của lớp 12A được cho trong bảng sau
% 	\begin{center}
% 		\begin{tabular}{|l|c|c|c|c|c|c|c|c|}
% 			\hline Khoảng điểm &{$[6,5;7)$}&{$[7;7,5)$}&{$[7,5;8)$}&{$[8;8,5)$}&{$[8,5;9)$}&{$[9;9,5)$}&{$[9,5;10)$}\\
% 			\hline Tần số & $8$ & $10$ & $16$ & $24$& $13$ & $7$ & $4$ \\
% 			\hline
% 		\end{tabular}
% 	\end{center}
% 	Nhóm chứa mốt của mẫu số liệu này là
% 	\choice
% 	{ $[7;7,5)$}
% 	{$[7,5;8)$}
% 	{\True $[8;8,5)$}
% 	{$[8,5;9)$}
% 	\loigiai{
% 		Tần số lớn nhất là $24$ nên nhóm chứa mốt là nhóm $[8,5;9)$. \\
% 	}       
% \end{ex}
% %11
% \begin{ex}%[1K3Y9-4]
% 	Để kiểm tra thời gian sử dụng pin của   chiếc điện thoại mới, bạn A thống kê thời gian sử dụng điện thoại của mình từ lúc sạc đầy cho đến khi hết pin ở bảng sau
% 	\begin{center}
% 		\begin{tabular}{|l|c|c|c|c|c|c|c|c|}
% 			\hline Thời gian sử dụng (giờ) &{$[7;9)$}&{$[9;11)$}&{$[11;13)$}&{$[13;15)$}&{$[15;17)$}\\
% 			\hline Số lần & $2$ & $5$ & $7$ & $6$& $3$  \\
% 			\hline
% 		\end{tabular}
% 	\end{center}
% 	Nhóm chứa mốt của mẫu số liệu này là
% 	\choice
% 	{ $[9;11)$}
% 	{\True $[11;13)$}
% 	{ $[13;15)$}
% 	{$[15;17)$}
% 	\loigiai{
% 		Tần số lớn nhất là $7$ nên nhóm chứa mốt là nhóm $[11;13)$. \\
% 	}        
% \end{ex}
% %12
% \begin{ex}%[1K3B9-4]
% 	Để kiểm tra thời gian sử dụng pin của   chiếc điện thoại mới, bạn A thống kê thời gian sử dụng điện thoại của mình từ lúc sạc đầy cho đến khi hết pin ở bảng sau
% 	\begin{center}
% 		\begin{tabular}{|l|c|c|c|c|c|c|c|c|}
% 			\hline Thời gian sử dụng (giờ) &{$[7;9)$}&{$[9;11)$}&{$[11;13)$}&{$[13;15)$}&{$[15;17)$}\\
% 			\hline Số lần & $2$ & $5$ & $7$ & $6$& $3$  \\
% 			\hline
% 		\end{tabular}
% 	\end{center}   
% 	Xác định mốt của mẫu số liệu ghép nhóm này.    
% 	\choice
% 	{ $M_0\approx 11,67$}
% 	{$M_0\approx 12$}
% 	{\True $M_0\approx 12,33$}
% 	{$M_0\approx 7$}
% 	\loigiai{
% 		Tần số lớn nhất là $7$ nên nhóm chứa mốt là nhóm $[11;13)$. \\
% 		Ta có, $j=3, a_3=11, m_3=7$, $m_2=5, m_4=6, h=2$. Do đó
% 		$$
% 		M_0=11+\frac{7-5}{(7-5)+(7-6)}\cdot 2\approx 12,33.
% 		$$
% 	}
% \end{ex}
% %13
% \begin{ex}%[1K3Y9-4]
% 	Tổng lượng mưa trong tháng 8 đo được tại một trạm quan trắc đặt tại Vũng Tàu từ năm 2002 đến năm 2020 được ghi lại như sau (đơn vị: mm)
% 	\begin{center}
% 		\begin{tabular}{|l|c|c|c|c|c|c|c|c|}
% 			\hline Tổng lượng mưa trong tháng 8 (mm) &{$[120;175)$}&{$[175;230)$}&{$[230;285)$}&{$[285;340)$}\\
% 			\hline Số năm & $10$ & $5$ & $3$ & $1$  \\
% 			\hline
% 		\end{tabular}
% 	\end{center}  
% 	Nhóm chứa mốt của mẫu số liệu này là
% 	\choice
% 	{ $[175;230)$}
% 	{ $[230;285)$}
% 	{\True $[120;175)$}
% 	{$[285;340)$}
% 	\loigiai{
% 		Tần số lớn nhất là $10$ nên nhóm chứa mốt là nhóm $[120;175)$. \\
% 	}        
% \end{ex}
% %14
% \begin{ex}%[1K3B9-4]
% 	Tổng lượng mưa trong tháng 8 đo được tại một trạm quan trắc đặt tại Vũng Tàu từ năm 2002 đến năm 2020 được ghi lại như sau (đơn vị: mm)
% 	\begin{center}
% 		\begin{tabular}{|l|c|c|c|c|c|c|c|c|}
% 			\hline Tổng lượng mưa trong tháng 8 (mm) &{$[120;175)$}&{$[175;230)$}&{$[230;285)$}&{$[285;340)$}\\
% 			\hline Số năm & $10$ & $5$ & $3$ & $1$  \\
% 			\hline
% 		\end{tabular}
% 	\end{center}  
% 	Xác định mốt của mẫu số liệu ghép nhóm này.    
% 	\choice
% 	{ $M_0\approx 172,25$}
% 	{$M_0\approx 146,125$}
% 	{\True $M_0\approx 156,67$}
% 	{$M_0\approx 10$}
% 	\loigiai{
% 		Tần số lớn nhất là $10$ nên nhóm chứa mốt là nhóm $[120;175)$. \\
% 		Ta có, $j=1, a_1=120, m_1=10$, $m_2=5, m_0=0, h=55$. Do đó
% 		$$
% 		M_0=120+\frac{10-0}{(10-0)+(10-5)}\cdot 55\approx 156.67.
% 		$$
% 	}
% \end{ex}
% %15
% \begin{ex}%[1K3B9-4]
% 	Một công ty xây dựng khảo sát khách hàng xem họ có nhu cầu mua nhà ở mức giá nào. Kết quả khảo sát được ghi lại ở bảng sau (đơn vị: triệu đồng/$\mathrm{m}^2$
% 	\begin{center}
% 		\begin{tabular}{|l|c|c|c|c|c|c|c|c|}
% 			\hline Mức giá &{$[10;14)$}&{$[14;18)$}&{$[18;22)$}&{$[22;26)$}&{$[26;30)$}\\
% 			\hline Số khách hàng & $54$ & $78$ & $120$ & $45$& $12$  \\
% 			\hline
% 		\end{tabular}
% 	\end{center}   
% 	Xác định mốt của mẫu số liệu ghép nhóm này.    
% 	\choice
% 	{ $M_0\approx 18$}
% 	{\True $M_0\approx 19,4$}
% 	{ $M_0\approx 20$}
% 	{$M_0\approx 120$}
% 	\loigiai{
% 		Tần số lớn nhất là $120$ nên nhóm chứa mốt là nhóm $[18;22)$. \\
% 		Ta có, $j=3, a_3=18, m_3=120$, $m_2=78, m_4=45, h=4$. Do đó
% 		$$
% 		M_0=18+\frac{120-78}{(120-78)+(120-45)}\cdot 4\approx 19,4.
% 		$$
% 	}
% \end{ex}
% \begin{ex}%[1K3K9-4]
% 	Một công ty xây dựng khảo sát khách hàng xem họ có nhu cầu mua nhà ở mức giá nào. Kết quả khảo sát được ghi lại ở bảng sau (đơn vị: triệu đồng/$\mathrm{m}^2$
% 	\begin{center}
% 		\begin{tabular}{|l|c|c|c|c|c|c|c|c|}
% 			\hline Mức giá &{$[10;14)$}&{$[14;18)$}&{$[18;22)$}&{$[22;26)$}&{$[26;30)$}\\
% 			\hline Số khách hàng & $54$ & $78$ & $120$ & $45$& $12$  \\
% 			\hline
% 		\end{tabular}
% 	\end{center}   
% 	Công ty nên xây nhà ở mức giá nào để nhiều người có nhu cầu mua nhất?    
% 	\choice
% 	{ $ 18$ triệu đồng/$\mathrm{m}^2$}
% 	{\True $19,4$ triệu đồng/$\mathrm{m}^2$}
% 	{ $20$ triệu đồng/$\mathrm{m}^2$}
% 	{$21$ triệu đồng/$\mathrm{m}^2$}
% 	\loigiai{
% 		Tần số lớn nhất là $120$ nên nhóm chứa mốt là nhóm $[18;22)$. \\
% 		Ta có, $j=3, a_3=18, m_3=120$, $m_2=78, m_4=45, h=4$. Do đó
% 		$$
% 		M_0=18+\frac{120-78}{(120-78)+(120-45)}\cdot 4\approx 19,4.
% 		$$
% 		Dựa vào kết quả trên ta dự đoán rằng nếu công ty xây nhà ở mức giá $19,4$ triệu đồng/$\mathrm{m}^2$ thì sẽ có nhiều người có nhu cầu mua nhất.
% 	}
% \end{ex}
% \begin{ex}%[1K3Y9-4]
% 	Số cuộc gọi điện thoại một người thực hiện mỗi ngày trong 30 ngày được lựa chọn ngẫu nhiên được thống kê trong bảng sau
% 	\begin{center}
% 		\begin{tabular}{|l|c|c|c|c|c|c|c|c|}
% 			\hline Số cuộc gọi &{$[2,5;5,5)$}&{$[5,5;8,5)$}&{$[8,5;11,5)$}&{$[11,5;14,5)$}&{$[14,5;17,5)$}\\
% 			\hline Số ngày & $5$ & $13$ & $7$ & $3$& $2$  \\
% 			\hline
% 		\end{tabular}
% 	\end{center}   
% 	Nhóm chứa mốt của mẫu số liệu này là
% 	\choice
% 	{ $[2,5;5,5)$}
% 	{\True $[5,5;8,5)$}
% 	{ $[8,5;11,5)$}
% 	{$[11,5;14,5)$}
% 	\loigiai{
% 		Tần số lớn nhất là $13$ nên nhóm chứa mốt là nhóm $[5,5;8,5)$. \\
% 	}
% \end{ex}
% \begin{ex}%[1K3K9-4]
% 	Số cuộc gọi điện thoại một người thực hiện mỗi ngày trong 30 ngày được lựa chọn ngẫu nhiên được thống kê trong bảng sau
% 	\begin{center}
% 		\begin{tabular}{|l|c|c|c|c|c|c|c|c|}
% 			\hline Số cuộc gọi &{$[2,5;5,5)$}&{$[5,5;8,5)$}&{$[8,5;11,5)$}&{$[11,5;14,5)$}&{$[14,5;17,5)$}\\
% 			\hline Số ngày & $5$ & $13$ & $7$ & $3$& $2$  \\
% 			\hline
% 		\end{tabular}
% 	\end{center}   
% 	Hãy dự đoán xem khả năng người đó thực hiện bao nhiêu cuộc gọi mỗi ngày là cao nhất?   
% 	\choice
% 	{ $5$}
% 	{\True $7$}
% 	{ $6$}
% 	{$8$}
% 	\loigiai{
% 		Tần số lớn nhất là $13$ nên nhóm chứa mốt là nhóm $[5,5;8,5)$. \\
% 		Ta có, $j=2, a_2=5,5, m_2=13$, $m_1=5, m_3=7, h=3$. Do đó
% 		$$
% 		M_0=5,5+\frac{13-5}{(13-5)+(13-7)}\cdot 3\approx 7,2.
% 		$$
% 		Do đó ta có thể dự đoán khả năng người đó thực hiện $7$ cuộc gọi mỗi ngày là cao nhất.
% 	}
% \end{ex}

\Closesolutionfile{ans}

% %--------Lời giải chi tiết
\FULLWIDTH \hienLG \anDA 
% % % --
% \setcounter{deso}{0}
\chap{LỜI GIẢI CHI TIẾT}
\setcounter{section}{0} \setcounter{ex}{0} \setcounter{dang}{0}
%%Bài 1
% 
\section{Giá trị lượng giác của một góc lượng giác}
\subsection{Tóm tắt lý thuyết}
\begin{tomtat}
	\subsubsection{Khái niệm góc lượng giác và số đo của góc lượng giác}
	Trong mặt phẳng, cho hai tia $Ou$, $Ov$. Xét tia $Om$ cùng nằm trong mặt phẳng này. Nếu tia $Om$ quay quanh điểm $O$, theo một chiều nhất định từ $Ou$ đến $Ov$, thì ta nói nó quét một góc lượng giác với tia đầu  $Ou$, tia cuối $Ov$ và kí hiệu là ($Ou$, $Ov$).\\
	Mỗi góc lượng giác gốc $O$ được xác định bởi tia đầu $Ou$, tia cuối $Ov$ và số đo của nó.
	\begin{center}
		\begin{minipage}[H]{0.3\textwidth}
			\begin{tikzpicture}[scale=.7]	
				\draw (0,0) -- (4,0)node[below] {$u$};
				\draw[red] (0,0) -- (45:4)node[below right] {$v$};
				\draw[dashed,green!50!black] (0,0) -- (20:4)node[below right] {$m$};
				\draw[-stealth,red] (0:1) arc (0:45:1);
				\draw[-stealth] (2.75,1) arc (0:45:1);
				\path (30:3.5) node[below=-2pt]{$+$};
				\path (0:0) node[below left]{$O$};
			\end{tikzpicture}
		\end{minipage}
		\begin{minipage}[H]{0.3\textwidth}
			\begin{tikzpicture} [scale=.7]	
				\draw (0,0) -- (4,0)node[below] {$u$};
				\draw[red] (0,0) -- (45:4)node[below right] {$v$};
				\draw[dashed,green!50!black] (0,0) -- (75:3.5)node[below right] {$m$};
				\draw[red,-stealth,smooth,samples=100] plot[domain =0:2.25*pi]({.5*(1.1)^(\x) *cos(\x r)},{.5*(1.1)^(\x) *sin(\x r)});
				\draw[-stealth] (0.5,2) arc (75:110:2);
				\path (90:2.5) node[below=-2pt]{$+$};
				\path (0:0) node[below left]{$O$};
			\end{tikzpicture}
		\end{minipage}
		\begin{minipage}[H]{0.3\textwidth}
			\begin{tikzpicture}[scale=.7]		
				\draw (0,0) -- (4,0)node[below] {$u$};
				\draw[red] (0,0) -- (45:4)node[below right] {$v$};
				\draw[dashed,green!50!black] (0,0) -- (120:3)node[above right] {$m$};
				\draw[-stealth,red] (0:.8) arc (0:-315:.8);
				\draw[-stealth] (-1,1.7) arc (120:60:1);
				\path (105:2.4) node[below=-2pt]{$-$};
				\path (0:0) node[below left]{$O$};
			\end{tikzpicture}
		\end{minipage}
	\end{center}
	\subsubsection{Hệ thức Chasles}
	\immini{Hệ thức Chasles: Với ba tia $Ou$, $Ov$, $Ow$ bất kì, ta có 	
	$$
		\text{sđ}(Ou, Ov)+\text{sđ}(Ov,Ow)=\text{sđ}(Ou,Ow)+k 360^{\circ}(k \in \mathbb{Z}). 
		$$}
	{\begin{tikzpicture}[scale=0.77, font=\footnotesize, line join=round, line cap=round, >=stealth]		
			\draw (0,0) -- (4,0)node[below] {$u$};
			\draw[red] (0,0) -- (75:3.5)node[below right] {$w$};
			\draw[green!50!black] (0,0) -- (30:4)node[below right] {$v$};
			\draw[-stealth,red] (0:1) arc (0:-285:1);
			\draw[-stealth] (0:.9) arc (0:30:.9);
			\draw[-stealth] (.86,.5) arc (30:75:1);
			\path (0:0) node[below left]{$O$};
	\end{tikzpicture}}
	% Nhận xét. Từ hệ thức Chasles, ta suy ra:
	% Với ba tia tuỳ ý $Ox$, $Ou$, $Ov$ ta có
	% $$
	% \text{sđ}(Ou, Ov)=\text{sđ}(Ox, Ov)-\text{sđ}(Ox,Ou)+k360^{\circ}(k \in \mathbb{Z}). 
	% $$
	% Hệ thức này đóng vai trò quan trọng trong việc tính toán số đo của góc lượng giác.
	\subsubsection{Đơn vị đo góc và cung tròn}
	\textbf{Đơn vị độ}: Góc $1^{\circ}$ bằng $\dfrac{1}{180}$ góc bẹt.\\
	Đơn vị độ được chia thành những đơn vị nhỏ hơn: $1^{\circ}=60'; 1'=60"$.\\
	% Đối với các góc lượng giác, khi mà số vòng quay trong chuyển động tương ứng từ tia đầu đến tia cuối là khá lớn thì số đo của chúng tính bằng độ sẽ trở nên cồng kềnh. Do đó, trong khoa học và kĩ thuật, bên cạnh việc đo bằng độ, người ta còn sử dụng đơn vị đo góc bằng rađian.\\
	\immini{\textbf{Đơn vị rađian}: Cho đường tròn $(O)$ tâm $O$, bán kính $R$ và một cung $AB$ trên $(O)$.
		Ta nói cung tròn $AB$ có số đo bằng 1 rađian nếu độ dài của nó đúng bằng bán kính $R$.
		Khi đó ta cũng nói rằng góc $AOB$ có số đo bằng 1 rađian và viết: $\overset\frown{AOB}=1$ rad.}
	{\begin{tikzpicture}[scale=0.77, font=\footnotesize, line join=round, line cap=round, >=stealth]		
			\draw[green!50!black] (30:2) arc (30:-270:2);
			\draw[red] (30:2) arc (30:90:2);
			\draw (90:2)node[above]{$B$}--(0:0)--(30:2)node[above right]{$A$};
			\path (0:0) node[below left]{$O$};
			\draw(60:2) node[above right]{1 rad};
	\end{tikzpicture}}
	\textbf{Quan hệ giữa độ và rađian:}
	$$
	1 \text{ góc bẹt }=180^\circ = 1 \mathrm{rad} \Leftrightarrow  1^\circ=\dfrac{\pi}{180} \mathrm{rad} \quad \text { và }\quad 1\,  \mathrm{rad}=\left(\dfrac{180}{\pi}\right)^\circ.
	$$
	\begin{note}
		Khi viết số đo của một góc theo đơn vị rađian, người ta thường không viết chữ rad sau số đo. Chẳng hạn góc $\dfrac{\pi}{2}$ được hiểu là góc $\dfrac{\pi}{2}$ rad.
	\end{note}
	\begin{note}
		Dưới đây là bảng tương ứng giữa số đo bằng độ và số đo bằng rađian của các góc đặc biệt trong phạm vi từ $0^{\circ}$ đến $180^{\circ}$.
	\end{note}
	\begin{center}
		\renewcommand{\arraystretch}{2}
		\begin{tabular}{|l|c|c|c|c|c|c|c|c|c|}
			\hline Độ & $0^{\circ}$ & $30^{\circ}$ & $45^{\circ}$ & $60^{\circ}$ & $90^{\circ}$ & $120^{\circ}$ & $135^{\circ}$ & $150^{\circ}$ & $180^{\circ}$ \\
			\hline Rađian & 0 & $\dfrac{\pi}{6}$ & $\dfrac{\pi}{4}$ & $\dfrac{\pi}{3}$ & $\dfrac{\pi}{2}$ & $\dfrac{2 \pi}{3}$ & $\dfrac{3 \pi}{4}$ & $\dfrac{5 \pi}{6}$ & $\pi$ \\
			\hline
		\end{tabular}
	\end{center}
	\subsubsection{Độ dài cung tròn}
	Một cung của đường tròn bán kính $R$ và có số đo $\alpha$ rad thì có độ dài $l=R \alpha$.
	\subsubsection{Đường tròn lượng giác}
	\immini{\begin{itemize}
			\item Đường tròn lượng giác là đường tròn có tâm tại gốc toạ độ, bán kính bằng $1$, được định hướng và lấy điểm $A(1 ; 0)$ làm điểm gốc của đường tròn.
			\item Điểm trên đường tròn lượng giác biểu diễn góc lượng giác có số đo $\alpha$ là điểm $M$ trên đường tròn lượng giác sao cho sđ$(OA, OM)=\alpha$.
	\end{itemize}
	\begin{note}
		Góc $\alpha$ và $\beta$ có chung điểm biểu diễn khi \fbox{$\alpha - \beta = k2\pi$} (chẵn lần $\pi$)
		\end{note}}
	{
		\begin{tikzpicture}[line join = round, line cap = round, >=stealth, font=\footnotesize, scale=0.6]
			\tikzset{label style/.style={font=\footnotesize}}
			\path (0,0) coordinate (O)
			(3,0) coordinate (A)
			(0,3) coordinate (B)
			(0,-3) coordinate (B')
			(-3,0) coordinate (A')
			(0:0)++(150:3) coordinate (M)
			($(O)!(M)!(A')$) coordinate (H)
			($(O)!(M)!(B)$) coordinate (K)
			;
			\draw[->] (-4,0) -- (4,0) node[above,blue]{$x$};
			\draw[->] (0,-4) -- (0.,4) node[left,blue]{$y$};
			\draw[orange] (O) circle (3cm);
			\draw[rotate=0,->,green!50!black] (0.5,0) arc (0:150:0.5cm);
			\draw (0.35,0.25) node[above,blue] {$\alpha$};
			\draw[dashed] (H)--(M)--(K);
			\draw[green!50!black] (M)--(O);
			\foreach \p/\r in {A/-45,M/150,H/-90,O/-150,A'/-135,B'/-45,B/45,K/0}
			\fill (\p) circle (1pt) node[shift={(\r:3mm)},blue]{$\p$};
		\end{tikzpicture}
	}
	\subsubsection{Các giá trị lượng giác của góc lượng giác}
	\immini{Gọi $M(x;y)$ là điểm biểu diễn của góc lượng giác $\alpha$ trên đường tròn lượng giác. Khi đó, ta có:
		\begin{itemize}
			\item $\cos\alpha=x.$
			\item $\sin\alpha=y.$
			\item $\tan\alpha=\dfrac{\sin\alpha}{\cos\alpha}=\dfrac{y}{x} ~(x\neq0).$
		\item $\cot\alpha=\dfrac{\cos\alpha}{\sin\alpha}=\dfrac{x}{y} ~(y\neq0).$
	\end{itemize}}
	{\begin{tikzpicture}[line join = round, line cap = round, >=stealth, font=\footnotesize, scale=0.6]
			\tikzset{label style/.style={font=\footnotesize}}
			\path (0,0) coordinate (O)
			(3,0) coordinate (A)
			(0,3) coordinate (B)
			(0,-3) coordinate (B')
			(-3,0) coordinate (A')
			(0:0)++(40:3) coordinate (M)
			($(O)!(M)!(A')$) coordinate (H)
			($(O)!(M)!(B)$) coordinate (K)
			;
			\draw[->] (-4,0) -- (4,0) node[above,blue]{$x$};
			\draw[->] (0,-4) -- (0.,4) node[left,blue]{$y$};
			\draw[orange] (O) circle (3cm);
			\draw[rotate=0,->,green!50!black] (0.7,0) arc (0:40:0.7cm);
			\draw (1,0) node[above,blue] {$\alpha$};
			\draw[dashed] (H)--(M)--(K);
			\draw[green!50!black] (M)--(O);
			\draw[blue,fill=black] (0,2) node[left]{$\sin\alpha$}(2,0) circle(1pt) node[below]{$\cos\alpha$}(3,2.3) node{$M(x;y)$};
			\foreach \p/\r in {A/-45,O/-135,A'/-135,B'/-45,B/45}
			\fill (\p) circle (1pt) node[shift={(\r:3mm)},blue]{$\p$};
	\end{tikzpicture}}
	\begin{note}
		a) Ta còn gọi trục tung là trục sin, trục hoành là trục côsin.\\
		b) Từ định nghĩa ta suy ra:
		\begin{itemize}
			\item $\sin\alpha$, $\cos\alpha$ xác định với mọi giá trị của $\alpha$ và ta có:
			$$-1\leq \sin\alpha\leq 1; \quad -1\leq \cos\alpha\leq 1; \quad \sin(\alpha+k2\pi)=\sin\alpha;\quad \cos(\alpha+k2\pi)=\cos\alpha\,\, (k\in\mathbb{Z}).$$
			\item $\tan\alpha$ xác định khi $\alpha\neq\dfrac{\pi}{2}+k\pi\,\,  (k\in\mathbb{Z})$.
			\item $\cot\alpha$ xác định khi $\alpha\neq k\pi\,\,  (k\in\mathbb{Z})$.
			\item Dấu của các giá trị lượng giác của một góc lượng giác phụ thuộc vào vị trí điểm biểu diễn $M$ trên đường tròn lượng giác.
		\end{itemize}
	\end{note}
	\begin{minipage}[h]{0.6\textwidth}
		\begin{tabular}{c|c|c|c|c|}
			\cline{2-5}
			& \multicolumn{4}{c|}{Góc phần tư} \\ \hline
			\multicolumn{1}{|c|}{Giá trị lượng giác} & I     & II     & III     & IV    \\ \hline
			\multicolumn{1}{|c|}{$\sin \alpha$}     &   $+$    &  $ +$      &    $-$    &   $-$  \\ \hline
			\multicolumn{1}{|c|}{$\cos \alpha$}     &   $+$    &  $ -$      &    $-$    &   $+$  \\ \hline
			\multicolumn{1}{|c|}{$\tan \alpha$}     &   $+$    &  $ -$      &    $+$    &   $-$  \\ \hline
			\multicolumn{1}{|c|}{$\cot \alpha$}     &   $+$    &  $ -$      &    $+$    &   $-$  \\ \hline
		\end{tabular}
	\end{minipage}
	\begin{minipage}[h]{0.6\textwidth}
		\begin{tikzpicture}[line join = round, line cap = round, >=stealth, font=\footnotesize, scale=0.6]
			\tikzset{label style/.style={font=\footnotesize}}
			\path (0,0) coordinate (O)
			(3,0) coordinate (A)
			(0,3) coordinate (B)
			(0,-3) coordinate (B')
			(-3,0) coordinate (A')
			(0:0)++(-60:3) coordinate (M)
			($(O)!(M)!(A')$) coordinate (H)
			($(O)!(M)!(B)$) coordinate (K)
			;
			\draw[->] (-4,0) -- (4,0) node[above,blue]{$x$};
			\draw[->] (0,-4) -- (0.,4) node[left,blue]{$y$};
			\draw[orange] (O) circle (3cm);
			\draw[rotate=0,->,green!50!black] (0.5,0) arc (0:-60:0.5cm);
			\draw (0.75,-0.35) node[blue] {$\alpha$};
			\draw[dashed] (H)--(M)--(K);
			\draw[green!50!black] (M)--(O);
			\draw[blue] (2.5,2.5) node{$I$}(-2.5,2.5) node{$II$}(-2.5,-2.5) node{$III$}(2.5,-2.5) node{$IV$};
			\foreach \p/\r in {A/-45,M/-60,H/90,O/-150,A'/-135,B'/-45,B/45,K/180}
			\fill (\p) circle (1pt) node[shift={(\r:3mm)},blue]{$\p$};
		\end{tikzpicture}
	\end{minipage}
	% \subsubsection{Giá trị lượng giác của các góc đặc biệt}
	% \begin{center}
	% 	\renewcommand{\arraystretch}{2}
	% 	\begin{tabular}{|c|c|c|c|c|c|}
	% 		\hline
	% 		\multirow{2}{*}{Góc $\alpha$} & $0$              & $\dfrac{\pi}{6}$  & $\dfrac{\pi}{4}$  & $\dfrac{\pi}{3}$  & $\dfrac{\pi}{2}$              \\ \cline{2-6} 
	% 		& $0^\circ$              & $30^\circ$  & $45^\circ$  & $60^\circ$  & $90^\circ$             \\ \hline
	% 		$\sin\alpha$                  & $0$             & $\dfrac{1}{2}$ & $\dfrac{\sqrt{2}}{2}$ & $\dfrac{\sqrt{3}}{2}$ & 1              \\ \hline
	% 		$\cos\alpha$                  & $1$             & $\dfrac{\sqrt{3}}{2}$ & $\dfrac{\sqrt{2}}{2}$ & $\dfrac{1}{2}$ & 0              \\ \hline
	% 		$\tan\alpha$                 & $0$             & $\dfrac{1}{\sqrt{3}}$ & 1 & $\sqrt{3}$ & Không xác định \\ \hline
	% 		$\cot\alpha$                  & Không xác định & $\sqrt{3}$ & 1 & $\dfrac{1}{\sqrt{3}}$ & 0              \\ \hline
	% 	\end{tabular}
	% \end{center}
	\subsubsection{Các công thức lượng giác cơ bản}
	Đối với các giá trị lượng giác, ta có các hệ thức cơ bản sau
	\begin{enumEX}[$\bullet$]{2}
		\item $\sin^2 \alpha  + \cos^2 \alpha =1$
		\item $ 1+ \tan^2 \alpha= \dfrac{1}{\cos^2 \alpha}$ $\left(\alpha \neq \dfrac{\pi}{2}+k\pi , k\in \mathbb{Z}\right)$
		\item $ 1+ \cot^2 \alpha= \dfrac{1}{\sin^2 \alpha}$ $\left(\alpha \neq k\pi , k\in \mathbb{Z}\right)$
		\item $\tan \alpha \cdot \cot \alpha =1 $ $\left(\alpha \neq \dfrac{k\pi}{2}, k\in \mathbb{Z}\right)$
	\end{enumEX}
	\newpage
	\subsubsection{Giá trị lượng giác của các góc có liên quan đặc biệt}
	\begin{enumerate}
		\item Góc đối nhau ($\alpha$ và $-\alpha$)
		\immini{\begin{itemize}
				\item $\cos (-\alpha)=\cos \alpha$
				\item $\sin (-\alpha) =-\sin \alpha$
				\item $\tan (-\alpha) =-\tan \alpha$
				\item $\cot (-\alpha) =-\cot \alpha$
		\end{itemize}}
		{\vspace*{-1cm}\begin{tikzpicture}[line join = round, line cap = round, >=stealth, font=\footnotesize, scale=0.5]
				\tikzset{label style/.style={font=\footnotesize}}
				\path (0,0) coordinate (O)
				(3,0) coordinate (A)
				(0:0)++(120:3) coordinate (M)
				(0:0)++(-120:3) coordinate (N)
				(0,4) coordinate (C)
				(0,-4) coordinate (D)
				($(O)!(M)!(C)$) coordinate (E)
				($(O)!(N)!(D)$) coordinate (F)
				;
				\draw[->] (-4,0) -- (4,0) node[above,blue]{$x$};
				\draw[->] (0,-4) -- (0.,4) node[left,blue]{$y$};
				\draw[orange] (O) circle (3cm);
				\draw[rotate=0,->,red] (0.5,0) arc (0:120:0.5cm);
				\draw[rotate=0,->,green!50!black] (0.6,0) arc (0:-120:0.6cm);
				\draw (0,0) node[above right=2pt,blue] {$\alpha$} (0,-0) node[below right=2pt,blue]{$-\alpha$};
				\draw[dashed] (E)--(M)--(N)--(F);
				\draw[green!50!black] (M)--(O);
				\draw[red] (N)--(O);
				\foreach \p/\r in {A/-45,M/120,N/-120,O/-150}
				\fill (\p) circle (1pt) node[shift={(\r:3mm)},blue]{$\p$};
		\end{tikzpicture}}
		\item Góc bù nhau ($\alpha$ và $\pi-\alpha$)
		\immini{\begin{itemize}
				\item $\sin (\pi -\alpha)=\sin \alpha$
				\item $\cos (\pi -\alpha) =-\cos \alpha$
				\item $\tan (\pi -\alpha) =-\tan \alpha$
				\item $\cot (\pi -\alpha) =-\cot \alpha$
		\end{itemize}}
		{\vspace*{-0.5cm}\begin{tikzpicture}[line join = round, line cap = round, >=stealth, font=\footnotesize, scale=0.5]
				\tikzset{label style/.style={font=\footnotesize}}
				\path (0,0) coordinate (O)
				(3,0) coordinate (A)
				(0:0)++(30:3) coordinate (M)
				(0:0)++(150:3) coordinate (N)
				(4,0) coordinate (C)
				(-4,0) coordinate (D)
				($(O)!(M)!(C)$) coordinate (E)
				($(O)!(N)!(D)$) coordinate (F)
				;
				\draw[->] (-4,0) -- (4,0) node[above,blue]{$x$};
				\draw[->] (0,-4) -- (0.,4) node[left,blue]{$y$};
				\draw[orange] (O) circle (3cm);
				\draw[rotate=0,->,red] (0.5,0) arc (0:150:0.5cm);
				\draw[rotate=0,->,green!50!black] (1.6,0) arc (0:30:1.6cm);
				\draw (2,0) node[above,blue] {$\alpha$} (0.3,1.5) node[below,blue]{$\pi-\alpha$};
				\draw[dashed] (E)--(M)--(N)--(F);
				\draw[red] (O)--(N);
				\draw[green!50!black] (O)--(M);
				\foreach \p/\r in {A/-45,M/30,N/150,O/-130}
				\fill (\p) circle (1pt) node[shift={(\r:3mm)},blue]{$\p$};
		\end{tikzpicture}}
		\item Góc phụ nhau ($\alpha$ và $\dfrac{\pi}{2}-\alpha$)
		\immini{\begin{itemize}
				\item $\sin \left( \dfrac{\pi}{2}-\alpha\right)=\cos \alpha$
				\item $\cos \left( \dfrac{\pi}{2}-\alpha\right)=\sin \alpha$
				\item $\tan \left( \dfrac{\pi}{2}-\alpha\right)=\cot \alpha$
				\item $\cot \left( \dfrac{\pi}{2}-\alpha\right)=\tan \alpha$
		\end{itemize}}
		{\vspace*{-0.5cm}\begin{tikzpicture}[line join = round, line cap = round, >=stealth, font=\footnotesize, scale=0.5]
				\tikzset{label style/.style={font=\footnotesize}}
				\path (0,0) coordinate (O)
				(3,0) coordinate (A)
				(0:0)++(20:3) coordinate (M)
				(0:0)++(70:3) coordinate (N)
				(0,4) coordinate (C)
				(4,0) coordinate (D)
				($(O)!(M)!(C)$) coordinate (E)
				($(O)!(N)!(D)$) coordinate (F)
				(2.82,0) coordinate (G)
				(0,2.82) coordinate (H)
				;
				\draw[->] (-4,0) -- (4,0) node[above,blue]{$x$};
				\draw[->] (0,-4) -- (0.,4) node[left,blue]{$y$};
				\draw[orange] (O) circle (3cm);
				\draw[rotate=0,->,red] (0.7,0) arc (0:70:0.7cm);
				\draw[rotate=0,->,green!50!black] (1.6,0) arc (0:20:1.6cm);
				\draw (2,0) node[above,blue] {$\alpha$} (1,-.2) node[below,blue]{$\frac{\pi}{2}-\alpha$};
				\draw[dashed] (E)--(M) (F)--(N) (G)--(M) (H)--(N);
				\draw[dashed] (-3,-3)--(3,3);
				\draw[->] (0.8,-.5)--(0.5,0.45);
				\draw[red] (O)--(N);
				\draw[green!50!black] (O)--(M);
				\foreach \p/\r in {A/-45,M/20,N/70,O/-220}
				\fill (\p) circle (1pt) node[shift={(\r:3mm)},blue]{$\p$};
		\end{tikzpicture}}
		\item Góc hơn kém $\pi$ ($\alpha$ và $\pi+\alpha$)
		\immini{\begin{itemize}
				\item $\sin (\pi +\alpha)=-\sin \alpha$
				\item $\cos (\pi +\alpha)=-\cos \alpha$
				\item $\tan (\pi +\alpha)=\tan \alpha$
				\item $\cot (\pi +\alpha)=\cot \alpha$
		\end{itemize}}
		{\vspace*{-0.5cm}\begin{tikzpicture}[line join = round, line cap = round, >=stealth, font=\footnotesize, scale=0.5]
				\tikzset{label style/.style={font=\footnotesize}}
				\path (0,0) coordinate (O)
				(3,0) coordinate (A)
				(0:0)++(60:3) coordinate (M)
				(0:0)++(240:3) coordinate (N)
				;
				\draw[->] (-4,0) -- (4,0) node[above,blue]{$x$};
				\draw[->] (0,-4) -- (0.,4) node[left,blue]{$y$};
				\draw[orange] (O) circle (3cm);
				\draw[rotate=0,->,red] (1.7,0) arc (0:240:1.7cm);
				\draw[rotate=0,->,green!50!black] (0.6,0) arc (0:60:0.6cm);
				\draw (1,0) node[above,blue] {$\alpha$};
				\draw (-1.2,1) node[below,blue,rotate=60]{$\pi+\alpha$};
				\draw[red] (O)--(N);
				\draw[green!50!black] (O)--(M);
				\foreach \p/\r in {A/-45,M/60,N/240,O/-150}
				\fill (\p) circle (1pt) node[shift={(\r:3mm)},blue]{$\p$};
		\end{tikzpicture}}
	\end{enumerate}
\end{tomtat}
 \foreach \i in {1,2,...,7} {\input{data/11KNTT/data1/1K1-2-\i.tex}}

%%Bài 2
% %\chapter{Hàm số  lượng giác và phương trình lượng giác}
\setcounter{section}{1}
\section{Công thức lượng giác}
\subsection{Tóm tắt lý thuyết}
\begin{tomtat}
% 	\begin{center}
% 		\begin{tikzpicture}[scale = 2.5]
% 			\path (0,0) coordinate (O) (1.5,0) coordinate (x) (0,1.5) coordinate (y);
% 			\draw[thick,->] (-1.5,0)--(x);
% 			\draw[thick,->] (0,-1.5)--(y);
% 			\draw (O) circle (1);
% 			\path ($(O)+(55:1)$) coordinate (M) 
% 			($(O)+(30:1)$) coordinate (N);
% 			\path ($(O)!(M)!(x)$) coordinate (x_M)
% 			($(O)!(M)!(y)$) coordinate (y_M)
% 			($(O)!(N)!(x)$) coordinate (x_N)
% 			($(O)!(N)!(y)$) coordinate (y_N);
% 			\draw[dashed] (x_M)--(M)--(y_M) (x_N)--(N)--(y_N);
% 			\foreach \x/\g in {O/-135,x/-90,y/180,x_M/-90,x_N/-90,y_M/180,y_N/180}
% 			\fill ($(\g:1mm)+(\x)$) node {$\x$};
% 			\fill 	(M) circle (0.5pt)
% 			($(15:4mm)+(M)$) node {$M\left(x_M,y_M\right)$};
% 			\fill (N) circle (0.5pt)
% 			($(15:4mm)+(N)$) node {$N\left(x_N,y_N\right)$};
% 	\draw (M)--(O)--(N);		
% 	\draw pic[draw,,angle radius=6mm,->,red]{angle=x--O--M};
% 	\fill[red] (45:3mm) node {$\alpha$};
% 	\draw pic[red,draw,,angle radius=10mm,->]{angle=x--O--N};
% 	\fill[red] (15:5mm) node {$\beta$};
% 		\end{tikzpicture}
% 	\end{center}
% Trong mặt phẳng $Oxy$ cho hai điểm $M,N$ trên đường tròn lượng giác.\\ Đặt $\alpha = \text{sđ} (Ox,OM), \beta = \text{sđ} (Ox,ON)$, ta có $M(\cos \alpha,\sin \alpha)$ và $N(\cos \beta, \sin \beta)$. Khi đó ta tính được $\overrightarrow{OM}.\overrightarrow{ON}$ bằng hai cách
% \begin{align*}
% 	\overrightarrow{OM}.\overrightarrow{ON}&=\left|\overrightarrow{OM}\right|.\left|\overrightarrow{ON}\right|.\cos \left(\overrightarrow{OM},\overrightarrow{ON}\right) = \cos (\alpha-\beta),\\
% 	\overrightarrow{OM}.\overrightarrow{ON} &= x_Mx_N+y_My_N= \cos \alpha \cos\beta +\sin\alpha\sin\beta.
% \end{align*}
% Từ đó dẫn tới công thức
% \begin{align*}
% 	\cos (\alpha-\beta) = \cos \alpha \cos \beta + \sin \alpha\sin \beta \tag{$\star$}
% \end{align*}
% Tất cả các công thức trong bài học được xây dựng dựa trên công thức $(\star)$.\\
% Trong suốt bài học, khi không nói gì thêm, chỉ xét các góc lượng giác mà trong đó giá trị lượng giác được để cập có nghĩa. 
	\subsubsection{Công thức cộng}
	\begin{khung4}{Công thức cộng}
	\begin{tasks}[style=itemize](2)
		\task $\cos (a-b) = \cos a \cos b + \sin a\sin b$.
		\task $\cos (a+b) = \cos a \cos b - \sin a\sin b$.
		\task $\sin (a-b) = \sin a \cos b - \sin b \cos a$.
		\task $\sin (a+b) = \sin a \cos b + \sin b \cos a$.
		\task $\tan (a-b) = \dfrac{\tan a - \tan b}{1+\tan a \tan b}$.
		\task $\tan (a+b) = \dfrac{\tan a + \tan b}{1-\tan a \tan b}$.
	\end{tasks}
	\end{khung4}
	\subsubsection{Công thức nhân đôi}
	Công thức nhân đôi được xây dựng bằng cách thay $b=a$ trong công thức cộng.
	\begin{khung4}{Công thức nhân đôi}
		\begin{tasks}[style=itemize]
			\task $\sin 2a = 2\sin a \cos a$.
			\task $\cos 2a = \cos^2a-\sin^2a = 2\cos^2a-1 = 1-2\sin^2a$.
			\task $\tan 2a = \dfrac{2\tan a}{1-\tan^2a}$.
		\end{tasks}
		\end{khung4}
\begin{note}
	Từ công thức nhân đôi, ta có công thức hạ bậc:
\end{note}
	\begin{khung4}{Công thức hạ bậc}
		\begin{tasks}[style=itemize](3)
			\task $\sin^2a= \dfrac{1-\cos 2a}{2}$.
		\task $\cos^2a = \dfrac{1+\cos 2a}{2}$.
		\task $\tan^2a=\dfrac{1-\cos2a}{1+\cos 2a}$.
		\end{tasks}
		\end{khung4}

\begin{note}
	Áp dụng công thức cộng cho $3a = a +2a$, ta có công thức nhân ba:
\end{note}
	\begin{khung4}{Công thức nhân ba}
		\begin{tasks}[style=itemize](2)
			\task $\sin3a= 3\sin a -4\sin^3a$.
		\task $\cos3a= 4\cos^3a-3\cos a$.
		\task $\tan3a = \dfrac{3\tan a - \tan^3 a}{1-3\tan^2a}$.
		\end{tasks}
		\end{khung4}

	\subsubsection{Công thức biến đổi tích thành tổng}
	\begin{khung4}{Công thức tích thành tổng}
		\begin{tasks}[style=itemize]
			\task $\cos a \cos b = \dfrac{1}{2}\left[\cos (a-b) + \cos (a+b)\right]$.
		\task $\sin a \sin b = \dfrac{1}{2}\left[\cos (a-b)-\cos(a+b)\right]$.
		\task $\sin a \cos b = \dfrac{1}{2}\left[\sin (a-b)+\sin (a+b)\right]$.
		\end{tasks}
		\end{khung4}
	\subsubsection{Công thức biến đổi tổng thành tích}
	Công thức biến đổi tổng thành tích được xây dựng bằng cách $a=\dfrac{a+b}{2}, b = \dfrac{a-b}{2}$ trong công thức biến đổi tích thành tổng.
	\begin{khung4}{Công thức tổng thành tích}
		\begin{tasks}[style=itemize](2)
			\task $\cos a+ \cos b = 2\cos\dfrac{a+b}{2}\cos \dfrac{a-b}{2}$.
		\task $\cos a- \cos b = -2\sin\dfrac{a+b}{2}\sin \dfrac{a-b}{2}$.
		\task $\sin a+ \sin b = 2\sin\dfrac{a+b}{2}\cos \dfrac{a-b}{2}$.
		\task $\sin a -\sin b = 2\cos\dfrac{a+b}{2}\sin \dfrac{a-b}{2}$.
		\end{tasks}
		\end{khung4}
\end{tomtat}

 \foreach \i in {1,2,...,5} {\input{data/11KNTT/data1/1K1-2-\i.tex}}

%%Bài 3
% \setcounter{section}{2}
\section{Hàm số lượng giác}
\subsection{Tóm tắt lý thuyết}
\begin{tomtat}
	% \subsubsection{Định nghĩa hàm số lượng giác}
	% \begin{dn}
	% 	\begin{itemize}
	% 		\item Hàm số sin $y=\sin x$ có tập xác định là $\mathbb{R}$.
	% 		\item Hàm số cos $y=\cos x$ có tập xác định là $\mathbb{R}$.
	% 		\item Hàm số tan  $y=\tan x$ có tập xác định là $\mathbb{R} \setminus\left\{\dfrac{\pi}{2}+k \pi \Big| k \in \mathbb{Z} \right\}$.
	% 		\item  Hàm số cot $y=\cot x$ có tập xác định là $\mathbb{R} \setminus \left\{k \pi \Big| k \in \mathbb{Z} \right\}$.
	% 	\end{itemize}
	% \end{dn}
	\subsubsection{Hàm số chẵn, hàm số lẻ}
	\begin{dn}
	Cho hàm số $y=f(x)$ có tập xác định là $\mathscr{D}$.
	\begin{itemize}
		\item Hàm số $f(x)$ được gọi là \textbf{hàm số chẵn} nếu $\forall x \in \mathscr{D}$ thì $-x \in \mathscr{D}$ và $f(-x)=f(x)$. Đồ thị của một hàm số chẵn nhận trục tung là trục đối xứng.
		\item Hàm số $f(x)$ được gọi là \textbf{hàm số lẻ} nếu $\forall x \in \mathscr{D}$ thì $-x \in \mathscr{D}$ và $f(-x)=-f(x)$. Đồ thị của một hàm số lẻ nhận gốc toạ độ là tâm đối xứng.
	\end{itemize}
	\end{dn}
	\subsubsection{Hàm số tuần hoàn}
	\begin{dn}
		Hàm số $y=f(x)$ có tập xác định $\mathscr{D}$ được gọi là \textbf{hàm số tuần hoàn} nếu tồn tại số $T \neq 0$ sao cho với mọi $x \in \mathscr{D}$ ta có:
		\begin{enumerate}[i)]
			\item $x+T \in \mathscr{D}$ và $x-T \in \mathscr{D}$;
			\item $f(x+T)=f(x)$.
		\end{enumerate}
		Số $T$ dương nhỏ nhất thỏa mãn các điều kiện trên (nếu có) được gọi là \textbf{chu kì} của hàm số tuần hoàn đó.
	\end{dn}
	\begin{nx}
		\
		\begin{itemize}
			\item  Các hàm số $y=\sin x$ và $y=\cos x$ tuần hoàn với chu kì $2 \pi$. Các hàm số $y=\tan x$ và $y=\cot x$ tuần hoàn với chu kì $\pi$.
		\end{itemize}
	\end{nx}
	\begin{note} 
		Tổng quát, người ta chứng minh được các hàm số $y=A \sin \omega x$ và $y=A \cos \omega x$ $(\omega>0)$ là những hàm số tuần hoàn với chu kì \fbox{$T=\dfrac{2 \pi}{\omega}$}.
	\end{note}
	\subsubsection{Đồ thị và tính chất của hàm số $y=\sin x$}
	\begin{tc}
		Hàm số $y=\sin x$:
		\begin{itemize}
			\item   Có tập xác định là $\mathbb{R}$ và tập giá trị là $[-1 ; 1]$;
			\item   Là hàm số lẻ và tuần hoàn với chu kì $2 \pi$;
			\item    Đồng biến trên mỗi khoảng $\left(-\dfrac{\pi}{2}+k 2 \pi ; \dfrac{\pi}{2}+k 2 \pi\right)$ và nghịch biến trên mỗi khoảng \\
			$\left(\dfrac{\pi}{2}+k 2 \pi ; \dfrac{3 \pi}{2}+k 2 \pi\right)$, $k \in \mathbb{Z}$;
			\item    Có đồ thị đối xứng qua gốc toạ độ và gọi là một \textbf{đường hình sin}.
		\end{itemize}
	\begin{center}
		\begin{tikzpicture}[>=stealth,scale=0.7,transform shape] 
			\path
			({-2.5*pi},0) coordinate (X1)
			({-2*pi},0) coordinate (X2)
			({-1.5*pi},0) coordinate (X3)
			({-pi},0) coordinate (X4)
			({-0.5*pi},0) coordinate (X5)
			(0,0) coordinate (O)
			({0.5*pi},0) coordinate (X6)
			({pi},0) coordinate (X7)
			({1.5*pi},0) coordinate (X8)
			({2*pi},0) coordinate (X9)
			({2.5*pi},0) coordinate (X10)
			({-pi},-2) coordinate (A)
			({pi},-2) coordinate (B)
			;
			\draw[->] (-9.5,0) -- (9.5,0) node[below] {\small $x$};
			\draw[->] (0,-1.5) -- (0,1.8) node[right] {\small $y$};
			\draw [dotted] (X3)--({-1.5*pi},1)--({2.5*pi},1)--({2.5*pi},0)  ({0.5*pi},1)--({0.5*pi},0)
			(X1)--({-2.5*pi},-1)--({1.5*pi},-1)--({1.5*pi},0)  ({-0.5*pi},-1)--({-0.5*pi},0)
			({-pi},0) -- (A) ({pi},0) -- (B);
			\foreach \x/\g/\z in {X1/90/-\tfrac{5\pi}{2},X2/140/-2\pi,X3/-90/-\tfrac{3\pi}{2},X4/-135/-\pi,X5/90/-\tfrac{\pi}{2},X6/-90/\tfrac{\pi}{2},X7/60/\pi,X8/90/\tfrac{3\pi}{2},X9/-40/2\pi,X10/-90/\tfrac{5\pi}{2}} 
			\fill[black] (\x) circle(1pt) +(\g:5mm) node {$\z$};
			\draw [<->] ({-pi},-1.7)--({pi},-1.7) ; 
			\draw (0,0) node[below right]{$O$} (0,-1.7) node[below]{$T=2\pi$}
			(0,1) node[above right]{$1$} (0,-1) node[below right]{$-1$};
			\clip (-9.5,-1.4) rectangle (9.5,1.6) ;
			\draw[thick,samples=100,domain=-9.3:9.3] plot(\x,{sin((\x)*180/pi)});
			
		\end{tikzpicture}
	\end{center}
	\end{tc}
	\subsubsection{Đồ thị và tính chất của hàm số $y=\cos x$}
	\begin{tc}
		Hàm số $y=\cos x$:
		\begin{itemize}
			\item    Có tập xác định là $\mathbb{R}$ và tập giá trị là $[-1 ; 1]$;
			\item    Là hàm số chẵn và tuần hoàn với chu kì $2 \pi$;
			\item    Đồng biến trên mỗi khoảng $(-\pi+k 2 \pi ; k 2 \pi)$ và nghịch biến trên mỗi khoảng $(k 2 \pi ; \pi+k 2 \pi), k \in \mathbb{Z}$;
			\item    Có đồ thị là một đường hình sin đối xứng qua trục tung.
		\end{itemize}
	\begin{center}
		\begin{tikzpicture}[>=stealth,scale=0.77,transform shape] 
			\path
			({-2.5*pi},0) coordinate (X1)
			({-2*pi},0) coordinate (X2)
			({-1.5*pi},0) coordinate (X3)
			({-pi},0) coordinate (X4)
			({-0.5*pi},0) coordinate (X5)
			(0,0) coordinate (O)
			({0.5*pi},0) coordinate (X6)
			({pi},0) coordinate (X7)
			({1.5*pi},0) coordinate (X8)
			({2*pi},0) coordinate (X9)
			({2.5*pi},0) coordinate (X10)
			({-pi},-2) coordinate (A)
			({pi},-2) coordinate (B)
			;
			\draw[->] (-9.5,0) -- (9.5,0) node[below] {\small $x$};
			\draw[->] (0,-1.5) -- (0,1.8) node[right] {\small $y$};
			\draw [dotted] (X2)--({-2*pi},1)--({2*pi},1)--({2*pi},0) (X4)--({-pi},-1)--({pi},-1)--({pi},0) 
			({-pi},0) -- (A) ({pi},0) -- (B);
			\foreach \x/\g/\z in {X1/125/-\tfrac{5\pi}{2},X2/-90/-2\pi,X3/-120/-\tfrac{3\pi}{2},X4/90/-\pi,X5/110/-\tfrac{\pi}{2},X6/-120/\tfrac{\pi}{2},X7/90/\pi,X8/120/\tfrac{3\pi}{2},X9/-90/2\pi,X10/-120/\tfrac{5\pi}{2}} 
			\fill[black] (\x) circle(1pt) +(\g:5mm) node {$\z$};
			\draw [<->] ({-pi},-1.7)--({pi},-1.7) ; 
			\draw (0,0) node[below right]{$O$} (0,-1.7) node[below]{$T=2\pi$}
			(0,1) node[above right]{$1$} (0,-1) node[below right]{$-1$}
			;
			\clip (-9.5,-1.4) rectangle (9.5,1.6) ;
			\draw[thick,samples=100,domain=-9.3:9.3] plot(\x,{cos((\x)*180/pi)});
			
		\end{tikzpicture}
	\end{center}
	\end{tc}
	\subsubsection{Đồ thị và tính chất của hàm số $y=\tan x$}
	\begin{tc}
		Hàm số $y=\tan x$:
		\begin{itemize}
			\item    Có tập xác định là $\mathbb{R} \setminus\left\{\dfrac{\pi}{2}+k \pi \Big| k \in \mathbb{Z} \right\}$ và tập giá trị là $\mathbb{R}$;
			\item    Là hàm số lẻ và tuần hoàn với chu kì $\pi$;
			\item    Đồng biến trên mỗi khoảng $\left(-\dfrac{\pi}{2}+k \pi ; \dfrac{\pi}{2}+k \pi\right)$, $k \in \mathbb{Z}$;
			\item    Có đồ thị đối xứng qua gốc toạ độ.
		\end{itemize}
	\begin{center}
		\begin{tikzpicture}[>=stealth,scale=0.77,transform shape] 
			\path
			({-1.5*pi},0) coordinate (X1)
			({-pi},0) coordinate (X2)
			({-0.5*pi},0) coordinate (X3)
			({0.5*pi},0) coordinate (X4)
			({pi},0) coordinate (X5)
			({1.5*pi},0) coordinate (X6)
			;
			\draw[->] (-6.5,0) -- (6.5,0) node[below] {\small $x$};
			\draw[->] (0,-3.5) -- (0,3.5) node[right] {\small $y$};
			\draw [dashed] ({-3*pi/2},3.5)--({-3*pi/2},-3.5) ({-pi/2},3.5)--({-pi/2},-3.5) ({pi/2},3.5)--({pi/2},-3.5) ({3*pi/2},3.5)--({3*pi/2},-3.5) ;
			\foreach \x/\g/\z in {X1/-150/-\tfrac{3\pi}{2},X2/-40/-\pi,X3/-40/-\tfrac{\pi}{2},X4/-40/\tfrac{\pi}{2},X5/-400/\pi,X6/-40/\tfrac{3\pi}{2}} 
			\fill[black] (\x) circle(1pt) +(\g:5mm) node {$\z$};
			\draw (0,0) node[below right]{$O$};
			\clip (-6.5,-3.5) rectangle (6.5,3.5) ;
			\draw[thick,samples=100,domain={-pi/2+0.2}:{pi/2-0.2}] plot(\x,{tan((\x)*180/pi)});
			\draw[thick,samples=100,domain={pi/2+0.2}:{3*pi/2-0.2}] plot(\x,{tan((\x)*180/pi)});
			\draw[thick,samples=100,domain={-3*pi/2+0.2}:{-pi/2-0.2}] plot(\x,{tan((\x)*180/pi)});
		\end{tikzpicture}
	\end{center}
	\end{tc}
	\subsubsection{Đồ thị và tính chất của hàm số $y=\cot x$}
	\begin{tc}
		Hàm số $y=\cot x$:
		\begin{itemize}
			\item    Có tập xác định là $\mathbb{R} \setminus\{k \pi \mid k \in \mathbb{Z} \}$ và tập giá trị là $\mathbb{R}$; 
			\item    Là hàm số lẻ và tuần hoàn với chu kì $\pi$;
			\item    Nghịch biến trên mỗi khoảng $(k \pi ; \pi+k \pi), k \in \mathbb{Z}$;
			\item    Có đồ thị đối xứng qua gốc toạ độ.
		\end{itemize}
	\begin{center}
		\begin{tikzpicture}[>=stealth,scale=0.77,transform shape] 
			\path
			({-2*pi},0) coordinate (X1)
			({-1.5*pi},0) coordinate (X2)
			({-pi},0) coordinate (X3)
			({-0.5*pi},0) coordinate (X4)
			({0.5*pi},0) coordinate (X5)
			({pi},0) coordinate (X6)
			({1.5*pi},0) coordinate (X7)
			({2*pi},0) coordinate (X8)
			;
			\draw[->] (-7.5,0) -- (7.5,0) node[below] {\small $x$};
			\draw[->] (0,-3.5) -- (0,3.5) node[left] {\small $y$};
			\draw [dashed] ({-2*pi},3.5)--({-2*pi},-3.5) ({-pi},3.5)--({-pi},-3.5) ({pi},3.5)--({pi},-3.5) ({2*pi},3.5)--({2*pi},-3.5) ;
			\foreach \x/\g/\z in {X1/-150/-2\pi,X2/-130/-\tfrac{3\pi}{2},X3/-140/-\pi,X4/-100/-\tfrac{\pi}{2},X5/-100/\tfrac{\pi}{2},X6/-120/\pi,X7/-100/\tfrac{3\pi}{2}, X8/-120/2\pi}
			\fill[black] (\x) circle(1pt) +(\g:5mm) node {$\z$};
			\draw (0,0) node[below left]{$O$};
			\clip (-6.5,-3.5) rectangle (6.5,3.5) ;
			\draw[thick,samples=100,domain={0.2}:{pi-0.2}] plot(\x,{cot((\x)*180/pi)});
			\draw[thick,samples=100,domain={pi+0.2}:{2*pi-0.2}] plot(\x,{cot((\x)*180/pi)});
			\draw[thick,samples=100,domain={-0.2}:{-pi+0.2}] plot(\x,{cot((\x)*180/pi)});
			\draw[thick,samples=100,domain={-pi-0.2}:{-2*pi+0.2}] plot(\x,{cot((\x)*180/pi)});
		\end{tikzpicture}
	\end{center}
	\end{tc}
\end{tomtat}
\subsection{Các dạng toán thường gặp}
 \foreach \i in {1,2,...,5} {\input{data/11KNTT/data1/1K1-3-\i.tex}}

%%Bài 4
% 
\setcounter{section}{3}
\section{Phương trình lượng giác cơ bản}
\subsection{Tóm tắt lý thuyết}
\begin{tomtat}
\subsubsection{Phương trình $\sin x=m$}
\begin{itemize}
	\item Với $|m|>1$ thì phương trình $\sin x=m$ vô nghiệm.
	\item Với $|m|\leq 1$, sẽ tồn tại duy nhất $\alpha \in \left[-\dfrac{\pi}{2}; \dfrac{\pi}{2}\right]$ thỏa mãn $\sin\alpha=m$. Khi đó
	\begin{center}
		$\sin x=m\Leftrightarrow\sin x=\sin\alpha\Leftrightarrow\hoac{&x=\alpha+k2\pi\\&x=\pi-\alpha+k2\pi}$ ($k\in \mathbb{Z}$).
	\end{center}
\item Nếu số đo của góc $\alpha$ được đo bằng đơn vị độ thì
\begin{center}
	$\sin x=\sin\alpha^\circ\Leftrightarrow\hoac{&x=\alpha^\circ+k360^\circ\\&x=180^\circ-\alpha^\circ+k360^\circ}$ ($k\in\mathbb{Z}$).
\end{center}
\item Tổng quát,
\begin{center}
	$\sin f(x)=\sin g(x)\Leftrightarrow\hoac{&f(x)=g(x)+k2\pi\\&f(x)=\pi - g(x)+k2\pi}$ ($k\in\mathbb{Z}$).
\end{center}
\item Một số trường hợp đặt biệt:
\begin{enumEX}[\faCheckCircleO]{1}
	\item $\sin x=0\Leftrightarrow x=k\pi$, $k\in\mathbb{Z}$.
	\item $\sin x=1\Leftrightarrow x=\dfrac{\pi}{2}+k2\pi$, $k\in\mathbb{Z}$.
	\item $\sin x=-1\Leftrightarrow x=-\dfrac{\pi}{2}+k2\pi$, $k\in \mathbb{Z}$.
\end{enumEX}
\end{itemize}
	\subsubsection{Phương trình $\cos x=m$}
\begin{itemize}
	\item Với $|m|>1$ thì phương trình $\cos x=m$ vô nghiệm.
	\item Với $|m|\leq 1$, sẽ tồn tại duy nhất $\alpha\in\left[0; \pi\right]$ thỏa mãn $\cos x=m$. Khi đó
	\begin{center}
		$\cos x=m\Leftrightarrow\cos x=\cos \alpha\Leftrightarrow\hoac{&x=\alpha+k2\pi\\&x=-\alpha+k2\pi}$ ($k\in\mathbb{Z}$).
	\end{center}
\item Nếu số đo của góc $\alpha$ được đo bằng đơn vị độ thì
\begin{center}
	$\cos x=\cos\alpha\Leftrightarrow\hoac{&x=\alpha^\circ+k360^\circ\\&x=-\alpha^\circ+k360^\circ}$ ($k\in\mathbb{Z}$).
\end{center}
\item Tổng quát,
\begin{center}
	$\cos f(x)=\cos g(x)\Leftrightarrow\hoac{&f(x)=g(x)+k2\pi\\&f(x)=-g(x)+k2\pi}$ ($k\in \mathbb{Z}$)
\end{center}
\item Một số trường hợp đặc biệt:
\begin{enumEX}[\faCheckCircleO]{1}
\item $\cos x=0\Leftrightarrow x=\dfrac{\pi}{2}+k\pi$, $k\in\mathbb{Z}$.
\item $\cos x=1\Leftrightarrow x=k2\pi$, $k\in\mathbb{Z}$.
\item $\cos x=-1\Leftrightarrow x=\pi+k2\pi$, $k\in\mathbb{Z}$.
\end{enumEX}
\end{itemize}
\subsubsection{Phương trình $\tan x=m$}
\begin{itemize}
	\item Với mọi $m\in\mathbb{R}$, tồn tại duy nhất $\alpha\in\left(-\dfrac{\pi}{2};\dfrac{\pi}{2}\right)$ thỏa mãn $\tan \alpha=m$. Khi đó
	\begin{center}
		$\tan x=m\Leftrightarrow\tan x=\tan \alpha\Leftrightarrow x=\alpha+k\pi$ ($k\in\mathbb{Z}$).
	\end{center}
\item Nếu số đo của góc $\alpha$ được đo bằng đơn vị độ thì
\begin{center}
	$\tan x=\tan\alpha^\circ\Leftrightarrow x=\alpha^\circ+k180^\circ$, $k\in \mathbb{Z}$
\end{center}
\item Tổng quát,
\begin{center}
	$\tan f(x)=\tan g(x)\Leftrightarrow f(x)=g(x)+k\pi$, $k\in\mathbb{Z}$.
\end{center}
\end{itemize}
\subsubsection{Phương trình $\cot x=m$}
\begin{itemize}
	\item Với mọi $m\in\mathbb{R}$, tồn tại duy nhất $\alpha\in\left(0;\pi\right)$ thỏa mãn $\cot\alpha=m$. Khi đó
	\begin{center}
		$\cot x=m\Leftrightarrow\cot x=\cot\alpha\Leftrightarrow x=\alpha+k\pi$ $k\in\mathbb{Z}$.
	\end{center}
\item Nếu số đo của góc $\alpha$ được đo bằng đơn vị độ thì
\begin{center}
	$\cot x=\cot\alpha^\circ\Leftrightarrow x=\alpha^\circ+k180^\circ$, $k\in\mathbb{Z}$.
\end{center}
\item Tổng quát,
\begin{center}
	$\cot f(x)=\cot g(x)\Leftrightarrow f(x)=g(x)+k\pi$, $k\in\mathbb{Z}$.
\end{center}
\end{itemize}
\end{tomtat}

 \foreach \i in {2,3,4,5,6,7,11} {\input{data/11KNTT/data1/1K1-4-\i.tex}}

%%Ôn tập chương I
% 
% \section{BÀI TẬP ÔN TẬP CHƯƠNG I}
\Opensolutionfile{ans}[ans/ans-1K1-4-OTC]
\begin{ex}%[Câu 1]%[1K1Y1-1]
	Đổi $225^\circ$ sang rađian.
	\choice
	{$\dfrac{4\pi}{5}$}
	{$\dfrac{6\pi}{5}$}
	{$\dfrac{3\pi}{7}$}
	{\True $\dfrac{5\pi}{4}$}
	%<MyLT2>
	\loigiai{
		Ta có $225^\circ = \dfrac{225}{180}\pi=\dfrac{5\pi}{4}$ (rađian).
	}
\end{ex}
\begin{ex}%[Câu 2]%[1K1Y1-3]
	Một đường tròn có bán kính $R=10$ cm. Độ dài cung $40^\circ$ trên đường tròn gần bằng
	\choice
	{$11$ cm}
	{$13$ cm}
	{\True $7$ cm}
	{$9$ cm}
	\loigiai{
		Ta có $40^\circ = 40\cdot \dfrac{\pi}{180} =\dfrac{2\pi}{9}$ rađian.\\
		Độ dài cung $l=\dfrac{2\pi}{9}\cdot 10=\dfrac{20\pi}{9}\approx 7$ cm.
	}
\end{ex}
\begin{ex}%[Câu 3]%[1K1B1-4]
	Bánh xe của người đi xe đạp quay được $2$ vòng trong $6$ giây. Hỏi trong $1$ giây, bánh xe quay được bao nhiêu độ?
	\choice
	{$60^\circ$}
	{$72^\circ$}
	{$240^\circ$}
	{\True $120^\circ$}
	\loigiai{
		Trong $6$ giây, bánh xe quay được $2\cdot 360^\circ=720^\circ$.\\
		Trong $1$ giây, bánh xe quay được $720^\circ\colon 6=120^\circ$.
	}
\end{ex}
\begin{ex}%[Câu 4]%[1K1Y1-6]
	Cho góc $\alpha$ thỏa mãn $90^\circ <\alpha <180^\circ$. Khẳng định nào sau đây đúng?
	\choice
	{$\cos\alpha>0$}
	{\True $\sin\alpha>0$}
	{$\tan\alpha>0$}
	{$\cot\alpha>0$}
	\loigiai
	{Vì $90^\circ <\alpha <180^\circ$ nên $\sin\alpha>0$, $\cos\alpha<0$, $\tan\alpha<0$ và $\cot\alpha<0$.}
\end{ex}
\begin{ex}%[Câu 5]%[1K1B1-6]
	Cho $\sin \alpha =\dfrac{1}{3}$ và $\dfrac{\pi}{2}<\alpha<\pi$. Khi đó $\cos \alpha$ có giá trị là
	\choice
	{$\cos \alpha =-\dfrac{2}{3}$}
	{$\cos \alpha =\dfrac{2\sqrt{2}}{3}$}
	{$\cos \alpha =\dfrac{8}{9}$}
	{\True $\cos \alpha =-\dfrac{2\sqrt{2}}{3}$}
	\loigiai{
		Ta có $\cos^2 \alpha =1-\sin^2 \alpha =1-\left(\dfrac{1}{3}\right)^2=\dfrac{8}{9}$.\\
		Vì $\dfrac{\pi}{2}<\alpha<\pi$ nên $\cos \alpha <0$.\\
		Do đó $\cos \alpha =-\dfrac{2\sqrt{2}}{3}$.
	}
\end{ex}
\begin{ex}%[Câu 6]%[1K1B1-7]
	Cho $A$, $B$, $C$ là ba góc của tam giác $ABC$. Trong các khẳng định sau, khẳng định nào \textbf{sai}?
	\choice
	{$\sin (B+C)=\sin A$}
	{$\cos (B+C)=-\cos A$}
	{\True $\tan (B+C)=\tan A$}
	{$\cot (B+C)=-\cot A$}
	\loigiai{
		Ta có $B+C=180^\circ-A$.\\Suy ra $\tan(B+C)=\tan (180^\circ-A)=-\tan A$.
	}
\end{ex}
\begin{ex}%[Câu 7]%[1K1B1-8]
	Tính giá trị biểu thức $P=\cos ^2\dfrac{\pi}{8}+\cos ^2\dfrac{{3\pi}}{8}+\cos ^2\dfrac{{5\pi}}{8}+\cos ^2\dfrac{{7\pi}}{8}.$
	\choice
	{$P=-1$}
	{$P=0$}
	{$P=1$}
	{\True $P=2$}
	\loigiai{Ta có $\cos ^2\dfrac{7\pi}{8}=\cos ^2\dfrac{\pi}{8}$ và $\cos ^2\dfrac{5\pi}{8}=\cos ^2\dfrac{3\pi}{8}$
		$\Rightarrow P=2\left({{\cos}^2\dfrac{\pi}{8}+{\cos}^2\dfrac{{3\pi}}{8}}\right)$.\\
		Vì $\dfrac{\pi}{8}+\dfrac{{3\pi}}{8}=\dfrac{\pi}{2}\Rightarrow \cos \dfrac{\pi}{8}=\sin \dfrac{{3\pi}}{8}\Rightarrow \cos ^2\dfrac{\pi}{8}=\sin ^2\dfrac{{3\pi}}{8}.$\\
		Do đó $ P=2 \left({{\sin}^2\dfrac{{3\pi}}{8}+{\cos}^2\dfrac{{3\pi}}{8}}\right)=2\cdot1=2.$
	}
\end{ex}
\begin{ex}%[Câu 8]%[1K1B1-8]
	Cho $\sin a + \cos a = - \dfrac 54$, khi đó giá trị của $\sin a \cos a$ bằng
	\choice
	{$1$}
	{$\dfrac{5}{4}$}
	{$\dfrac{3}{16}$}
	{\True $\dfrac{9}{32}$}
	\loigiai{
		$\sin a\cos a = \dfrac{(\sin a + \cos a)^2 -1}{2} = \dfrac{9}{32}$.
	}
\end{ex}
\begin{ex}%[Câu 9]%[1K1Y1-8]
	Cho $\tan x=\dfrac{1}{2}$. Tính $\tan \left(x+\dfrac{\pi}{4}\right)$.
	\choice
	{$2$}
	{$\dfrac{3}{2}$}
	{$6$}
	{\True $3$}
	\loigiai
	{
		Ta có $\tan \left(x+\dfrac{\pi}{4}\right)=\dfrac{\tan x+\tan \dfrac{\pi}{4}}{1-\tan x\cdot \tan \dfrac{\pi}{4}}=\dfrac{\dfrac{1}{2}+1}{1-\dfrac{1}{2}} = 3$.
	}
\end{ex}
\begin{ex}%[Câu 10]%[1K1Y1-3]
	Biểu diễn các góc lượng giác $\alpha=-\dfrac{5\pi}{6}$, $\beta=\dfrac{\pi}{3}$, $\gamma=\dfrac{25\pi}{3}$, $\delta=\dfrac{17\pi}{6}$ trên đường tròn lượng giác. Các góc nào có điểm biểu diễn trùng nhau?
	\choice
	{\True $\beta$ và $\gamma$}
	{$\alpha$, $\beta$, $\gamma$}
	{$\beta$, $\gamma$, $\delta$}
	{$\alpha$ và $\beta$}
	\loigiai{
		Ta có $\beta+8\pi=\dfrac{\pi}{3}+8\pi=\dfrac{25\pi}{3}=\gamma$.\\
		Do đó, $\beta$ và $\gamma$ có điểm biểu diễn trùng nhau trên đường tròn lượng giác.
	}
\end{ex}
\begin{ex}%[Câu 11]%[1K1Y1-7]
	Trong các khẳng định sau, khẳng định  nào là \textbf{sai}?
	\choice
	{$\sin(\pi-\alpha)=\sin\alpha$}
	{\True $\cos(\pi-\alpha)=\cos \alpha$}
	{$\sin(\pi+\alpha)=-\sin\alpha$}
	{$\cos(\pi+\alpha)=-\cos \alpha$}
	\loigiai{
		Ta có $\cos(\pi-\alpha)=-\cos \alpha$ nên $\cos(\pi-\alpha)=\cos \alpha$ là khẳng định \textbf{sai}.
	}
\end{ex}
\begin{ex}%[Câu 12]%[1K1Y1-2]
	Góc lượng giác nào tương ứng với chuyển động quay $3\dfrac{1}{5}$ vòng ngược chiều kim đồng hồ?
	\choice
	{$\dfrac{16 \pi}{5}$}
	{$\left(\dfrac{16}{5}\right)^\circ$}
	{\True $1152^\circ$}
	{$1152 \pi$}
	\loigiai{
		Chuyển động quay ngược chiều kim đồng hồ là quay theo chiều dương; góc tương ứng là
		$$3\dfrac{1}{5}\cdot 2\pi=\dfrac{32\pi}{5}, \text{ tương ứng với } 1152^\circ.$$
	}
\end{ex}

\begin{ex}%[Câu 13]%[1K1Y2-1]
	Trong các khẳng định sau, khẳng định nào \textbf{sai}?
	\choice
	{\True $\cos (a-b)=\cos a\cos b-\sin a\sin b$}
	{$\sin (a-b)=\sin a\cos b-\cos a\sin b$}
	{$\cos (a+b)=\cos a\cos b-\sin a\sin b$}
	{$\sin (a+b)=\sin a\cos b+\cos a\sin b$}
	\loigiai{
		Ta có $\cos (a-b)=\cos a\cos b+\sin a\sin b$ nên $\cos (a-b)=\cos a\cos b-\sin a\sin b$ là khẳng định \textbf{sai}.
	}
\end{ex}
\begin{ex}%[Câu 14]%[1K1Y1-7]
	Trong trường hợp nào dưới đây $\cos \alpha=\cos \beta$ và $\sin \alpha=-\sin \beta$?
	\choice
	{\True $\beta=-\alpha$}
	{$\beta=\pi-\alpha$}
	{$\beta=\pi+\alpha$}
	{$\beta=\dfrac{\pi}{2}+\alpha$}
	\loigiai{
		Trong trường hợp hai cung đối nhau thì các giá trị $\cos$ của chúng bằng nhau, các giá trị $\sin$ của chúng đối nhau.
	}
\end{ex}

\begin{ex}%[Câu 15]%[1K1B2-2]
	Nếu $\cos a=\dfrac{1}{4}$ thì $\cos 2 a$ bằng
	\choice
	{$\dfrac{7}{8}$}
	{\True $-\dfrac{7}{8}$}
	{$\dfrac{15}{16}$}
	{$-\dfrac{15}{16}$}
	\loigiai{
		Ta có  $\cos 2 a =2 \cos^2 a-1=2 \cdot \left( \dfrac{1}{4} \right)^2-1=-\dfrac{7}{8}$.
	}
\end{ex}
\begin{ex}%[Câu 16]%[1K1K2-2]
	Nếu $\tan (a+b)=3, \tan (a-b)=-3$ thì $\tan 2 a$ bằng
	\choice
	{\True $0$}
	{$\dfrac{3}{5}$}
	{$1$}
	{$-\dfrac{3}{4}$}
	\loigiai{
		Ta có $\tan (a+b)=3 \Leftrightarrow \tan a+ \tan b= 3- 3 \tan a \tan b$ \quad (1)\\
		và 
		$\tan (a-b)=-3 \Leftrightarrow \tan a- \tan b= -3- 3 \tan a \tan b$. \quad (2) \\
		Lấy vế trừ vế của (1) và (2) ta được $2\tan b=6\Leftrightarrow \tan b =3$.\\
		Thay $\tan b =3$ vào (1) ta được $\tan a= 0$.\\
		Khi đó $\tan 2 a = \dfrac{2 \tan a}{1- \tan^2 a}=0$.
	}
\end{ex}

\begin{ex}%[Câu 17]%[1K1K2-3]
	Nếu $\cos a=\dfrac{3}{5}$ và $\cos b=-\dfrac{4}{5}$ thì $\cos (a+b) \cos (a-b)$ bằng
	\choice
	{\True $0$}
	{$2$}
	{$4$}
	{$5$}
	\loigiai{
		Do  $\cos a=\dfrac{3}{5}$ và $\cos b=-\dfrac{4}{5}$ nên  $\cos 2a=-\dfrac{7}{25}$ và $\cos 2b=\dfrac{7}{25}$.\\
		Ta có $2 \cos (a+b) \cos (a-b)= \cos 2a +\cos 2b=-\dfrac{7}{25}+ \dfrac{7}{25}=0$.\\
		Do đó $\cos (a+b) \cos (a-b)=0$.
	}
\end{ex}




\begin{ex}%[Câu 18]%[1K1B2-3]
	Rút gọn biểu thức $M=\cos(a+b)\cos(a-b)-\sin (a+b)\sin(a-b)$, ta được
	\choice
	{$M=\sin 4a$}
	{$M=1-2\cos^2a$}
	{\True $M=1-2\sin^2a$}
	{$M=\cos 4a$}
	\loigiai{
		Ta có
		\allowdisplaybreaks
		\begin{eqnarray*}
			M&=&\cos(a+b)\cos(a-b)-\sin (a+b)\sin(a-b)\\
			&=&\dfrac{1}{2}\left(\cos2a+\cos 2b\right)+\dfrac{1}{2}\left(\cos2a-\cos 2b\right)\\
			&=&\cos 2a\\
			&=&1-2\sin^2a.
		\end{eqnarray*}
	}
\end{ex}
\begin{ex}%[Câu 19]%[1K1B2-2]
	Nếu $\sin x +\cos x = \dfrac{1}{2}$ thì $\sin 2x$ bằng
	\choice
	{$\dfrac{3}{4}$}
	{$\dfrac{3}{8}$}
	{$\dfrac{\sqrt{2}}{2}$}
	{\True $-\dfrac{3}{4}$}
	\loigiai{
		Ta có $\sin 2x=\left( {\sin x+\cos x} \right)^2-\left( {\sin^2 x+\cos ^2x} \right)=\left( {\dfrac{1}{2}} \right)^2-1=-\dfrac{3}{4}$.}
\end{ex}
\begin{ex}%[Câu 20]%[1K1Y2-3]
	Mệnh đề nào dưới đây đúng?
	\choice
	{\True $\cos3x\cdot\cos5x=\dfrac{1}{2}(\cos8x+\cos2x)$}
	{$\cos3x\cdot\cos5x=\dfrac{1}{2}(\cos8x-\cos2x)$}
	{$\cos3x\cdot\cos5x=\dfrac{1}{2}(\cos2x-\cos8x)$}
	{$\cos3x\cdot\cos5x=\dfrac{1}{2}(\sin8x+\sin2x)$}
	\loigiai{Ta có $\cos3x\cdot\cos5x=\dfrac{1}{2}[\cos(3x+5x)+\cos(3x-5x)]=\dfrac{1}{2}(\cos8x+\cos2x)$.}
\end{ex}
\begin{ex}%[Câu 21]%[1K1B2-4]
	Giả sử $3\sin ^4x-\cos ^4x=\dfrac{1}{2}$ thì $\sin ^4x+3\cos ^4x$ có giá trị bằng
	\choice
	{$2$}
	{\True $1$}
	{$4$}
	{$3$}
	\loigiai{
		\begin{eqnarray*}
			3\sin ^4x-\cos ^4x=\dfrac{1}{2}\Leftrightarrow 6\sin ^4x-2\cos ^4x=1&\Leftrightarrow& 6\sin ^4x-2\left(1-\sin^2\alpha\right)^2=1\\
			&\Leftrightarrow& 4\sin ^4x-4\sin ^2\alpha-3=0\\
			&\Leftrightarrow& \left(2\sin^2\alpha+3\right)\left(2\sin^2\alpha-1\right)=0\\
			&\Rightarrow&{\sin ^2}\alpha=\dfrac{1}{2}.
		\end{eqnarray*}
		Ta có $\sin ^4x+3\cos ^4x$ $=\sin ^4\alpha+3\left(1-\sin^2\alpha\right)^2$ $=\dfrac{1}{4}+3\left(1-\dfrac{1}{2}\right)^2=1$.}
\end{ex}
\begin{ex}%[Câu 22]%[1K1B3-2]
	Hàm số $y=\sin x$ đồng biến trên khoảng
	\choice
	{$(0 ; \pi)$}
	{$\left(-\dfrac{3 \pi}{2} ;-\dfrac{\pi}{2}\right)$}
	{\True $\left(-\dfrac{ \pi}{2} ;\dfrac{\pi}{2}\right)$}
	{$(-\pi ; 0)$}
	\loigiai{ 
		Do hàm số $y=\sin x$ đồng biến trên mỗi khoảng $\left( -\dfrac{\pi}{2}+k2 \pi;\dfrac{\pi}{2}+k2 \pi \right) $ nên ứng với $k=0$, ta có hàm số $y=\sin x$ đồng biến trên khoảng $\left(-\dfrac{ \pi}{2} ;\dfrac{\pi}{2}\right)$.
	}
\end{ex}

\begin{ex}%[Câu 23]%[1K1B3-2]
	Hàm số nghịch biến trên khoảng $(\pi ; 2 \pi)$ là
	\choice
	{$y=\sin x$}
	{$y=\cos x$}
	{$y=\tan x$}
	{\True $y=\cot x$}
	\loigiai{
		Do hàm số $y=\cot x$ nghịch biến trên mỗi khoảng $\left( k \pi; \pi +k \pi \right) $ nên  ứng với $k=1$, ta có hàm số $y=\cot x$ nghịch biến trên khoảng $(\pi ; 2 \pi)$.
	}
\end{ex}
\begin{ex}%[Câu 24]%[1K1B3-1]
	Tập xác định của hàm số $y=\dfrac{\cos x}{\sin x-1}$ là
	\choice
	{$\mathbb{R}\setminus \left\{k2\pi| k\in\mathbb{Z}\right\}$}
	{\True $\mathbb{R}\setminus \left\{\dfrac{\pi}{2}+k2\pi| k\in\mathbb{Z}\right\}$}
	{$\mathbb{R}\setminus \left\{\dfrac{\pi}{2}+k\pi| k\in\mathbb{Z}\right\}$}
	{$\mathbb{R}\setminus \left\{k\pi| k\in\mathbb{Z}\right\}$}
	\loigiai{
		Hàm số xác định khi và chỉ khi $\sin x-1\ne 0\Leftrightarrow\sin x\ne 1\Leftrightarrow x\ne \dfrac{\pi}{2}+k2\pi$ với $k\in \mathbb{Z}$.\\
		Vậy tập xác định của hàm số là $\mathbb{R}\setminus \left\{\dfrac{\pi}{2}+k2\pi| k\in\mathbb{Z}\right\}$.
	}
\end{ex}
\begin{ex}%[Câu 25]%[1K1Y3-1]
	Khẳng định nào sau đây là \textbf{sai}?
	\choice
	{Hàm số $y=\cos x$ có tập xác định là $\mathbb{R}$}
	{Hàm số $y=\cos x$ có tập giá trị là $[-1;1]$}
	{\True Hàm số $y=\cos x$ là hàm số lẻ}
	{Hàm số $y=\cos x$ tuần hoàn với chu kì $2\pi$}
	\loigiai{
		Hàm số $y=\cos x$ là hàm số chẵn.
	}
\end{ex}
\begin{ex}%[Câu 26]%[1K1Y3-4]
	Trong các hàm số sau đây, hàm số nào là hàm tuần hoàn?
	\choice
	{$y=\tan x+x$}
	{$y=x^2+1$}
	{\True $y=\cot x$}
	{$y=\dfrac{\sin x}{x}$}
	\loigiai{
		Hàm số $y=\cot x$ là hàm số tuần hoàn với chu kỳ $T=\pi$.
	}
\end{ex}
\begin{ex}%[Câu 27]%[1K1Y3-3]
	Khẳng định nào sau đây đúng?
	\choice
	{Hàm số $y=\sin x$ là hàm số chẵn}
	{\True Hàm số $y=\cos x$ là hàm số chẵn}
	{Hàm số $y=\tan x$ là hàm số chẵn}
	{Hàm số $y=\cot x$ là hàm số chẵn}
	\loigiai{
		Hàm số $y=\cos x$ là hàm số chẵn;  các hàm số còn lại là hàm số lẻ.
	}
\end{ex}
\begin{ex}%[Câu 28]%[1K1B3-3]
	Khẳng định nào sau đây là đúng?
	\choice
	{Hàm số $y=\cos x$ là hàm số lẻ}
	{\True Hàm số $y=\tan 2x- \sin x$ là hàm số lẻ}
	{Hàm số $y=\sin x$ là hàm số chẵn}
	{Hàm số $y=\tan x \cdot \sin x$ là hàm số lẻ}
	\loigiai{
		Xét hàm số $y=f\left( {x} \right)=\tan 2x-\sin x$.
		\\Hàm số xác định khi $\cos 2x \ne 0 \Leftrightarrow x \ne \dfrac{\pi}{4}+k\dfrac{\pi}{2}$, $\left( {k \in \mathbb{Z}} \right)$.
		\\Tập xác định $\mathscr{D}=\mathbb{R} \setminus \left\{ {\dfrac{\pi}{4}+k\dfrac{\pi}{2},k \in \mathbb{Z}} \right\}$.
		\\ Với mọi $x \in \mathscr{D}$ thì $-x \in \mathscr{D}$ và $f\left( {-x} \right)=\tan \left( {-2x} \right)- \sin \left( {-x} \right)=-\tan 2x + \sin x=-f\left( {x} \right)$.\\ Do đó hàm số $y=\tan 2x-\sin x$ là hàm số lẻ.}
\end{ex}
\begin{ex}%[Câu 29]%[1K1B3-1]
	Tập xác định của hàm số $y=\dfrac{\cot x}{\cos x-1}$ là
	\choice
	{$\mathbb{R}\setminus\left\{\dfrac{k\pi}{2}, k\in\mathbb{Z}\right\}$}
	{$\mathbb{R}\setminus\left\{\dfrac{k}{2}+k\pi, k\in\mathbb{Z}\right\}$}
	{\True $\mathbb{R}\setminus\left\{k\pi, k\in\mathbb{Z}\right\}$}
	{$\mathbb{R}\setminus\left\{k2\pi, k\in\mathbb{Z}\right\}$}
	\loigiai{
		Hàm số xác định khi và chỉ khi $\heva{&\sin x\ne 0\\ &\cos x\ne 1}\Leftrightarrow\heva{&x\ne k\pi\\ &x\ne l2\pi}\ (k,l\in\mathbb{Z})\Leftrightarrow x\ne k\pi, k\in\mathbb{Z}$.\\
		Vậy, tập xác định của hàm số $y=\dfrac{\cot x}{\cos x-1}$ là $\mathbb{R}\setminus\left\{k\pi, k\in\mathbb{Z}\right\}$.
	}
\end{ex}
\begin{ex}%[Câu 30]%[1K1K3-2]
	Cho đồ thị hàm số $y=\sin x$ như hình vẽ sau
	\begin{center}
		
		\begin{tikzpicture}[>=stealth,scale=0.7]
			\draw [->] (-11,0)--(0,0)
			node[below right]{$O$}--(11,0)node[below]{$x$}; % Hệ trục tọa độ
			\draw[->] (0,-1.5) --(0,2) node[left]{$y$};
			\draw[dashed] (-5*pi/2,0)node[above]{$-\tfrac{5\pi}{2}$}--(-5*pi/2,-1)--(3*pi/2,-1)--(3*pi/2,0)node[above]{$\tfrac{3\pi}{2}$};
			\draw[dashed] (-3*pi/2,0)node[below]{$-\tfrac{3\pi}{2}$} --(-3*pi/2,1)--(5*pi/2,1)--(5*pi/2,0)node[below]{$\tfrac{5\pi}{2}$};
			\draw[dashed] (-pi/2,0)node[above]{$-\tfrac{\pi}{2}$}--(-pi/2,-1);
			\draw[dashed] (pi/2,0)node[below]{$\tfrac{\pi}{2}$}--(pi/2,1);
			\draw(-2.9*pi,0) node[below left]{$-3\pi$}(-1.9*pi,0) node[below]{$-2\pi$}(-1.1*pi,0) node[below]{$-\pi$}(3*pi,0) node[below]{$3\pi$}(2*pi,0) node[below]{$2\pi$}(pi,0) node[above]{$\pi$}(0,-1.6)node[below]{$2\pi$}(0,1)node[above right]{$1$};
			\draw[dashed] (-pi,0)--(-pi,-1.6)(pi,0)--(pi,-1.6);
			\draw[<->](-pi,-1.6)--(pi,-1.6);
			\draw [domain=-3.5*pi:3.4*pi,samples=100] plot (\x, {sin(\x r)});
		\end{tikzpicture}
	\end{center}
	Mệnh đề nào dưới đây \textbf{sai}?
	\choice
	{Hàm số $y=\sin x$ tăng trên khoảng $\left(-\dfrac{\pi}{2};\dfrac{\pi}{2}\right)$}
	{Hàm số $y=\sin x$ giảm trên khoảng $\left(\dfrac{\pi}{2};\dfrac{3\pi}{2}\right)$}
	{Hàm số $y=\sin x$ giảm trên khoảng $\left(-\dfrac{3\pi}{2};-\pi \right)$}
	{\True Hàm số $y=\sin x$ tăng trên khoảng $\left(0;\pi \right)$}
	\loigiai{
		\begin{itemize}
			\item Hàm số $y=\sin x$ tăng trên $\left(0;\dfrac{\pi}{2}\right)$ và giảm trên $\left(\dfrac{\pi}{2};\pi \right)$.
			\item Vậy trên khoảng $\left(0;\pi \right)$, hàm số $y=\sin x$ vừa tăng vừa giảm nên khẳng định hàm số $y=\sin x$ tăng trên khoảng $\left(0;\pi \right)$ là khẳng định \textbf{sai}.
	\end{itemize}}
\end{ex}
\begin{ex}%[Câu 31]%[1K1B3-2]
	Chọn khẳng định đúng trong các khẳng định sau
	\choice
	{Hàm số $y = \tan x$ tuần hoàn với chu kì $2\pi$}
	{Hàm số $y = \cos x$ tuần hoàn với chu kì $\pi$}
	{\True Hàm số $y = \sin x$ đồng biến trên khoảng $\left(0; \dfrac{\pi}{2}\right)$}
	{Hàm số $y = \cot x$ nghịch biến trên $\mathbb{R}$}
	\loigiai{Ta xét $y = \sin x$ suy ra  $y'  = \cos x$. Dễ thấy $\cos x > 0\ ,\  \forall x\in \left(0; \dfrac{\pi}{2}\right)$. Do đó hàm số $y = \sin x$ đồng biến trên khoảng $\left(0; \dfrac{\pi}{2}\right)$.
	}
\end{ex}
\begin{ex}%[Câu 32]%[1K1K3-6]
	Đồ thị của hàm số $y=\sin x$ và $y=\cos x$ cắt nhau tại bao nhiêu điểm có hoành độ thuộc đoạn $\left[-2\pi;\dfrac{5\pi}{2}\right]$?
	\choice
	{\True $5$}
	{$6$}
	{$4$}
	{$7$}
	\loigiai{
		Xét phương trình hoành độ giao điểm của hai đồ thị hàm số $\sin x=\cos x$.\\
		Nếu $\cos x=0$ thì $\sin x=0$ nên vô lý.\\
		Do đó, $\cos x\ne 0$. Ta có
		\allowdisplaybreaks
		\begin{eqnarray*}
			\sin x=\cos x&\Leftrightarrow&\tan x=1\\
			&\Leftrightarrow&x=\dfrac{\pi}{4}+k\pi ,\quad \left(k\in\mathbb{Z}\right).
		\end{eqnarray*}
		Ta lại có
		\allowdisplaybreaks
		\begin{eqnarray*}
			-2\pi \le x\le \dfrac{5\pi}{2}&\Leftrightarrow& -2\pi \le \dfrac{\pi}{4}+k\pi\le \dfrac{5\pi}{2}\\
			&\Leftrightarrow& -2 \le \dfrac{1}{4}+k\le \dfrac{5}{2}\\
			&\Leftrightarrow& \dfrac{-9}{4} \le k\le \dfrac{9}{4}.
		\end{eqnarray*}
		Do $k\in\mathbb{Z}$ nên $k\in\left\{-2;-1;0;1;2\right\}$.\\
		Vậy hai đồ thị hàm số cắt nhau tại $5$ điểm có hoành độ thuộc đoạn $\left[-2\pi;\dfrac{5\pi}{2}\right]$.
	}
\end{ex}
\begin{ex}%[Câu 33]%[1K1B3-5]
	Tìm tập giá trị của hàm số $y=2\cos3x +1$.
	\choice
	{$[-3;1]$}
	{$[-3;-1]$}
	{\True $[-1;3]$}
	{$[1;3]$}
	\loigiai{
		$\forall x\in \mathbb{R}$ ta có
		\begin{eqnarray*}
			&& -1\leq\cos3x\leq1 \\
			&\Leftrightarrow& -2\leq2\cos3x\leq2 \\
			&\Leftrightarrow& -1\leq2\cos3x+1\leq3.
		\end{eqnarray*}
	}
\end{ex}
\begin{ex}%[Câu 34]%[1K1B3-6]
	Đường cong trong hình bên là đồ thị trên đoạn $\left[-\pi ;\pi\right]$ của một hàm số trong bốn hàm số được liệt kê ở bốn phương án $\textbf{A, B, C, D}$ dưới đây. Hỏi đó là hàm số nào?
	\begin{center}
		\definecolor{x}{rgb}{0.75,0.75,0.75}
		\begin{tikzpicture}[scale=1, line join=round, line cap=round,>=stealth]
			\draw[->] (-4,0.) -- (4,0.)node [above] { $x$};
			\draw[shift={(-3.14,0)}] node[below left] {\footnotesize $-\pi$};
			\draw[shift={(-1.57,0)}] node[above left] {\footnotesize $-\dfrac{\pi}{2}$};
			\draw[shift={(1.57,0)}] node[below left] {\footnotesize $\dfrac{\pi}{2}$};
			\draw[shift={(3.14,0)}] node[below left] {\footnotesize $\pi$};
			\draw[->] (0.,-1.3) -- (0.,1.5)node [right] { $y$};
			\draw (0,1) node[above left] {\footnotesize $1$};
			\draw (0,-1) node[above left] {\footnotesize $-1$};
			\draw (0pt,-10pt) node[right] {\footnotesize $O$};
			\clip(-4.2,-1.3) rectangle (4.2,1.5);
			\draw[line width=1.2pt,smooth,samples=100,domain=-3.14:3.14] plot(\x,{sin(((\x))*180/pi)});
			\draw [dashed] (-1.57,0)--(-1.57,-1)--(0,-1)(1.57,0)--(1.57,1)--(0,1);
		\end{tikzpicture}
	\end{center}
	\choice
	{\True $y=\sin x$}
	{$y=\cos x$}
	{$y=\tan x$}
	{$y=\cot x$}
	\loigiai{
		Đồ thị hàm số đi qua các điểm $(0;0),(\pi;0), \left(\dfrac{\pi}{2};1\right)$ và nhận $O$ làm tâm đối xứng.
	}
\end{ex}
\begin{ex}%[Câu 35]%[1K1Y4-3]
	Phương trình $\cot x=-1$ có nghiệm là
	\choice
	{\True $-\dfrac{\pi}{4}+k \pi(k \in \mathbb{Z})$}
	{$\dfrac{\pi}{4}+k \pi(k \in \mathbb{Z})$}
	{$\dfrac{\pi}{4}+k 2 \pi(k \in \mathbb{Z})$}
	{$-\dfrac{\pi}{4}+k 2 \pi(k \in \mathbb{Z})$}
	\loigiai{
		Ta có $\cot x=-1 \Leftrightarrow \cot x=\cot \left(-\dfrac{\pi}{4}\right) \Leftrightarrow x =-\dfrac{\pi}{4}+k \pi(k \in \mathbb{Z}) $.
	}
\end{ex}
\begin{ex}%[Câu 36]%[1K1Y4-3]
	Trong các phép biến đổi sau, phép biến đổi nào \textbf{sai}?
	\choice
	{$\sin x=1\Leftrightarrow x=\dfrac{\pi}{2}+k2\pi,(k\in \mathbb{Z})$}
	{$\tan x=1\Leftrightarrow x=\dfrac{\pi}{4}+k\pi,(k\in \mathbb{Z})$}
	{$\cos x=\dfrac{1}{2}\Leftrightarrow \hoac{
			& x=\dfrac{\pi}{3}+k2\pi,(k\in \mathbb{Z}) \\
			& x=-\dfrac{\pi}{3}+k2\pi,(k\in \mathbb{Z})}$}
	{\True $\sin x=0\Leftrightarrow x=k2\pi,(k\in \mathbb{Z})$}
	\loigiai{
		Ta có $\sin x=0\Leftrightarrow x=k\pi,(k\in \mathbb{Z})$, nên đáp án $\sin x=0\Leftrightarrow x=k2\pi,(k\in \mathbb{Z})$ sai.}
\end{ex}
\begin{ex}%[Câu 37]%[1K1B4-5]
	Nghiệm của phương trình $\sin x\cdot \cos x=\dfrac{1}{2}$ là
	\choice
	{$x=k2\pi$; $k\in \mathbb{Z}$}
	{$x=\dfrac{k\pi}{4}$; $k\in \mathbb{Z}$}
	{\True $x=\dfrac{\pi}{4}+k\pi$; $k\in \mathbb{Z}$}
	{$x=k\pi$; $k\in \mathbb{Z}$}
	\loigiai{
		Ta có $\sin x\cdot \cos x=\dfrac{1}{2}\Leftrightarrow \sin 2x=1\Leftrightarrow 2x=\dfrac{\pi}{2}+k2\pi\Leftrightarrow x=\dfrac{\pi}{4}+k\pi$ với $k\in\mathbb{Z}$.
	}
\end{ex}
\begin{ex}%[Câu 38]%[1K1Y4-3]
	Họ nghiệm của phương trình $\sin2x=1$ là
	\choice
	{$x=\dfrac{\pi}{2}+k\pi,\,k\in\mathbb{Z}$}
	{$x=\dfrac{\pi}{2}+k2\pi,\,k\in\mathbb{Z}$}
	{\True $x=\dfrac{\pi}{4}+k\pi,\,k\in\mathbb{Z}$}
	{$x=\dfrac{\pi}{4}+\dfrac{k\pi}{2},\,k\in\mathbb{Z}$}
	\loigiai{
		Ta có $\sin2x=1\Leftrightarrow 2x=\dfrac{\pi}{2}+k2\pi\Leftrightarrow x=\dfrac{\pi}{4}+k\pi,\, k\in\mathbb{Z}$.
	}
\end{ex}
\begin{ex}%[Câu 39]%[1K1B4-5]
	Phương trình $\sin 2x \cos x = \sin 7x \cos 4x$ có các họ nghiệm là
	\choice
	{$x=\dfrac{k2\pi}{5};x=\dfrac{\pi}{12}+\dfrac{k\pi}{6} (k \in \Bbb{Z})$}
	{$x=\dfrac{k\pi}{5};x=\dfrac{\pi}{12}+\dfrac{k\pi}{3} (k \in \Bbb{Z})$}
	{\True $x=\dfrac{k\pi}{5};x=\dfrac{\pi}{12}+\dfrac{k\pi}{6} (k \in \Bbb{Z})$}
	{$x=\dfrac{k2\pi}{5};x=\dfrac{\pi}{12}+\dfrac{k\pi}{3} (k \in \Bbb{Z})$}
	\loigiai{
		Ta có \begin{eqnarray*}
			\sin 2x \cos x = \sin 7x \cos 4x &\Leftrightarrow & \dfrac{1}{2}(\sin 3x+\sin x)=\dfrac{1}{2}(\sin 11x+\sin 3x)\\
			&\Leftrightarrow & \sin 11x = \sin x\\
			&\Leftrightarrow & \hoac{&x=\dfrac{k\pi}{5}\\&x=\dfrac{\pi}{12}+\dfrac{k\pi}{3} }(k \in \Bbb{Z}).
		\end{eqnarray*}
	}
\end{ex}
\begin{ex}%[Câu 40]%[1K1K4-3]
	Số nghiệm của phương trình $\cos x=0$ trên đoạn $[0 ; 10 \pi]$ là
	\choice
	{$5$}
	{$9$}
	{\True $10$}
	{$11$}
	\loigiai{
		Ta có $\cos x=0 \Leftrightarrow x =\dfrac{\pi}{2}+ k \pi (k \in \mathbb{Z})$.\\
		Do $0 \leq x \leq 10 \pi \Leftrightarrow 0 \leq \dfrac{\pi}{2}+ k \pi \leq 10 \Leftrightarrow -\dfrac{1}{2} \leq k \leq \dfrac{19}{2}\Leftrightarrow 0\leq k \leq 9( k \in  \mathbb{Z} )$.\\
		Do đó phương trình $\cos x=0$ có $10$ nghiệm.
	}
\end{ex}

\begin{ex}%[Câu 41]%[1K1B4-3]
	Số nghiệm của phương trình $\sin x=0$ trên đoạn $[0 ; 10 \pi]$ là
	\choice
	{$10$}
	{$6$}
	{$5$}
	{\True $11$}
	\loigiai{
		Ta có $\sin x=0 \Leftrightarrow x = k \pi (k \in \mathbb{Z})$.\\
		Do $0 \leq x \leq 10 \pi \Leftrightarrow 0 \leq k\leq 10$.\\
		Do đó phương trình $\sin x=0$ có $11$ nghiệm.
	}
\end{ex}

\begin{ex}%[Câu 42]%[1K1B4-3]
	Số nghiệm của phương trình $\sin \left(x+\dfrac{\pi}{4}\right)=\dfrac{\sqrt{2}}{2}$ trên đoạn $[0; \pi]$ là
	\choice
	{$4$}
	{$1$}
	{\True $2$}
	{$3$}
	\loigiai{
		Ta có $\sin \left(x+\dfrac{\pi}{4}\right)=\dfrac{\sqrt{2}}{2} \Leftrightarrow \sin \left(x+\dfrac{\pi}{4}\right)=\sin \left( \dfrac{\pi}{4}\right) \Leftrightarrow  \hoac{&x= k 2 \pi\\&x=\dfrac{\pi}{2}+ k 2 \pi} (k \in \mathbb{Z})$.\\
		Do $x \in [0 ; \pi]$ nên $x=0$ hoặc $x=\dfrac{\pi}{2}$.
	}
\end{ex}
\begin{ex}%[Câu 43]%[1K1B4-5]
	Phương trình $ \sin{2x}+3\cos x=0 $ có bao nhiêu nghiệm trong khoảng $ (0;\pi)$?
	\choice
	{$ 0 $}
	{\True $ 1 $}
	{$ 2 $}
	{$ 3 $}
	\loigiai
	{
		Ta có $ \sin{2x}+3\cos x=0 \Leftrightarrow \hoac{& \cos x=0 \\ &\sin x=-\dfrac{3}{2}}\Leftrightarrow \cos x=0 \Leftrightarrow x= \dfrac{\pi}{2}+k\pi$. Do $ x \in (0;\pi) $ nên có một nghiệm là $ x=\dfrac{\pi}{2}$.
	}
\end{ex}

\begin{ex}%[Câu 44]%[1K1B1-9]
	Một bánh xe có $72$ răng. Số đo góc mà bánh xe đã quay được khi di chuyển $10$ răng là
	\choice
	{$40^\circ	$}
	{\True $50^\circ$}
	{$60^\circ$}
	{$30^\circ$}
	\loigiai{
		1 bánh răng tương ứng với $\dfrac{360^\circ}{72}=5^\circ$$\Rightarrow 10$ bánh răng là $50^\circ$.}
\end{ex}

\begin{ex}%[Câu 45]%[1K1K1-9]
	\immini{Người ta muốn làm một cánh diều hình quạt có bán kính là $a$, độ dài cung tròn là $b$ và có chu vi là $80$ cm (như hình vẽ). Khi diện tích cánh diều đạt giá trị lớn nhất, tổng $a+b$ bằng
		\choice
		{$50$ cm}
		{$40$ cm}
		{$70$ cm}
		{\True $60$ cm}}{
		\begin{tikzpicture}
			\draw (0,3) arc (150:210:3);
			\coordinate [label=below:$A$] (A) at (0,0);
			\coordinate [label=right:$O$] (O) at (30:3);
			\coordinate [label=above:$C$] (C) at (0,3);
			\foreach \point in {O,A,C} \fill[black] (\point) circle (1pt);
			\draw (O)--(C) (O)--(A);
		\end{tikzpicture}}
	\loigiai{
		Gọi $\varphi$ (rad) là số đo cung của hình quạt. Khi đó $\varphi =\dfrac{b}{a}$.\\
		Chu vi cánh diều bằng $b+2a=80$.\\
		Diện tích cánh diều bằng $S=\dfrac{\varphi a^2}{2}=\dfrac{ab}{2}=\dfrac{1}{4}(b \cdot 2a) \le \dfrac{1}{4} \cdot \left(\dfrac{b+2a}{2}\right)^2=400$.\\
		Dấu bằng xảy ra khi và chỉ khi $\heva{&b=2a \\& b+2a=80}\Leftrightarrow \heva{&b=40 \\& a=20.}$\\
		Do vậy $a+b=60$ cm.}
\end{ex}
\begin{ex}%[Câu 46]%[1K1K1-9]
	\immini{
		Khi một tia sáng truyền từ không khí vào mặt nước thì một phần tia sáng bị phản xạ trên bề mặt, phần còn lại bị khúc xạ như hình bên. Góc tới $i$ liên hệ với góc khúc xạ $r$ bởi Định luật khúc xạ ánh sáng
		$$\dfrac{\sin i}{\sin r}=\dfrac{n_2}{n_1}.$$
		Ở đây, $n_1$ và $n_2$ tương ứng với chiết suất của môi trường $1$ (không khí) và môi trường $2$ (nước). Cho biết góc tới $i=50^\circ$ và  chiết suất của không khí bằng $1$ còn chiết suất của nước là $1{,}33$. Khi đó  góc khúc xạ gần với kết quả nào sau đây.
		\choice
		{\True$35{,}17^\circ$}
		{$55{,}47^\circ$}
		{$31{,}42^\circ$}
		{$12{,}35^\circ$}
	}
	{
		\begin{tikzpicture}[>=stealth,line join=round,line cap=round,font=\footnotesize,scale=.7]
			\path
			(0,0)coordinate(I)++(90:3)coordinate(N)++(-90:6)coordinate(N')
			(I)++(0:3)coordinate(B)++(180:6)coordinate(A)
			(I)++(30:3)coordinate(S')
			(I)++(150:3)coordinate(S)
			(I)++(-55:3)coordinate(R)
			;
			\fill[cyan!20!](-3,-3)rectangle(3,0)
			;
			\draw (A)--(B)
			;
			\draw[dashed](N)--(N')
			;
			\draw[->,midway](S)--(I)
			;
			\draw[->](I)--(S')
			;
			\draw[->](I)--(R)
			;
			\foreach \p/\r in {N/180,N'/180,S/160,S'/90,R/0,I/-135}
			\fill (\p) node[shift={(\r:3mm)}]{$\p$}
			;
			\draw pic[angle radius=3mm,draw=red,fill=green!50,angle eccentricity=1.5] {angle = N--I--S}
			;
			\draw pic[angle radius=4mm,draw=orange,fill=orange!50,angle eccentricity=1.5] {angle = S'--I--N}
			;
			\draw pic[angle radius=4mm,draw=blue,fill=blue!50,angle eccentricity=1.5] {angle = N'--I--R}
			;
			\draw (-2.5,.5)circle(7pt)node{$1$}
			(-2.5,-.5)circle(8pt)node{$2$}
			;	
		\end{tikzpicture}
	}
	\loigiai{
		Ta có $\dfrac{\sin i}{\sin r}=\dfrac{n_2}{n_1}\Leftrightarrow \dfrac{\sin 50^\circ}{\sin r}=\dfrac{1{,}33}{1}\Leftrightarrow \sin r=\dfrac{\sin 50^\circ}{1{,}33}\Rightarrow r\approx 35{,}17^\circ$.
	}
\end{ex}
\begin{ex}%[Câu 47]%[1K1G2-4]
	Giả sử $a, b, c$ lần lượt là ba cạnh đối diện với ba góc $A, B, C$ của tam giác $ABC$ thỏa điều kiện $2\cos\dfrac{B}{2}\cos\dfrac{C}{2}=\dfrac{1}{2}+\dfrac{b+c}{a}\sin\dfrac{A}{2}$. Tính góc $A$ của tam giác $ABC$.
	\choice
	{$30^\circ$}
	{$45^\circ$}
	{\True $60^\circ$}
	{$90^\circ$}
	\loigiai
	{\noindent Đặt $2\cos\dfrac{B}{2}\cos\dfrac{C}{2}=\dfrac{1}{2}+\dfrac{b+c}{a}\sin\dfrac{A}{2}\;(\star)$. Ta có
		\begin{align*}
			(\star)&\Leftrightarrow  2\cos\dfrac{B}{2}\cos\dfrac{C}{2}=\dfrac{1}{2}+\dfrac{\sin B+\sin C}{\sin A}\sin\dfrac{A}{2}\\
			&\Leftrightarrow  \cos\dfrac{B+C}{2}+\cos\dfrac{B-C}{2}=\dfrac{1}{2}+\dfrac{2\sin\dfrac{B+C}{2}\cos\dfrac{B-C}{2}}{2\sin\dfrac{A}{2}\cos\dfrac{A}{2}}\sin\dfrac{A}{2}\\
			&\Leftrightarrow  \sin\dfrac{A}{2}+\cos\dfrac{B-C}{2}=\dfrac{1}{2}+\cos\dfrac{B-C}{2}\;\left(\text{vì}\;\sin\dfrac{A}{2}>0, \cos\dfrac{A}{2}=\sin\dfrac{B+C}{2}\right)\\
			&\Leftrightarrow  \sin\dfrac{A}{2}=\dfrac{1}{2}\Leftrightarrow  A=\dfrac{\pi}{3}.
		\end{align*}
	}
\end{ex}
\begin{ex}%[Câu 48]%[1K1G4-5]
	Phương trình $2\sqrt{3}\sin\left(x-\dfrac{\pi}{8}\right)\cos\left(x-\dfrac{\pi}{8}\right)+2\cos^2\left(x-\dfrac{\pi}{8}\right) = \sqrt{3}+1$ có nghiệm là
	\choice
	{\True $x=\dfrac{5\pi}{24}+k\pi$, $x=\dfrac{3\pi}{8}+k\pi$ với $k\in\mathbb{Z}$}
	{$x=\dfrac{5\pi}{12}+k\pi$, $x=\dfrac{3\pi}{4}+k\pi$ với $k\in\mathbb{Z}$}
	{$x=\dfrac{5\pi}{4}+k\pi$, $x=\dfrac{5\pi}{16}+k\pi$ với $k\in\mathbb{Z}$}
	{$x=\dfrac{5\pi}{8}+k\pi$, $x=\dfrac{7\pi}{24}+k\pi$ với $k\in\mathbb{Z}$}
	\loigiai
	{
		Ta có
		\allowdisplaybreaks
		\begin{eqnarray*}
			&& 2\sqrt{3}\sin\left(x-\dfrac{\pi}{8}\right)\cos\left(x-\dfrac{\pi}{8}\right)+2\cos^2\left(x-\dfrac{\pi}{8}\right) = \sqrt{3}+1\\
			&\Leftrightarrow & \sqrt{3}\sin\left(2x-\dfrac{\pi}{4}\right)+1+\cos\left(2x-\dfrac{\pi}{4}\right) = \sqrt{3}+1\\
			&\Leftrightarrow & \sqrt{3}\sin\left(2x-\dfrac{\pi}{4}\right)+\cos\left(2x-\dfrac{\pi}{4}\right) = \sqrt{3}\\
			&\Leftrightarrow & \dfrac{\sqrt{3}}{2}\sin\left(2x-\dfrac{\pi}{4}\right)+\dfrac{1}{2}\cos\left(2x-\dfrac{\pi}{4}\right) = \dfrac{\sqrt{3}}{2}\\
			&\Leftrightarrow & \sin\left(2x-\dfrac{\pi}{12}\right) = \dfrac{\sqrt{3}}{2}\\
			&\Leftrightarrow & \left[\begin{aligned}&2x-\dfrac{\pi}{12}=\dfrac{\pi}{3}+k2\pi,k\in\mathbb{Z} \\&2x-\dfrac{\pi}{12}=\dfrac{2\pi}{3}+k2\pi,k\in\mathbb{Z}\end{aligned}\right.\\
			&\Leftrightarrow & \left[\begin{aligned}&x=\dfrac{5\pi}{24}+k\pi,k\in\mathbb{Z} \\&x=\dfrac{3\pi}{8}+k\pi,k\in\mathbb{Z}.\end{aligned}\right.
		\end{eqnarray*}
		Vậy phương trình đã cho có nghiệm $x=\dfrac{5\pi}{24}+k\pi$, $x=\dfrac{3\pi}{8}+k\pi$ với $k\in\mathbb{Z}$.
	}
\end{ex}
\begin{ex}%[Câu 49]%[1K1G4-5]
	Nghiệm dương nhỏ nhất của phương trình $\sin x+\sin 2x=\cos x+2\cos^2 x$ là
	\choice{$\dfrac{\pi}{6}$}{$\dfrac{\pi}{3}$}{$2\dfrac{\pi}{3}$}{\True$\dfrac{\pi}{4}$}
	\loigiai{\begin{eqnarray*}
			& &\sin x+\sin 2x=\cos x+2\cos^2 x\\
			&\Leftrightarrow & \sin x + 2\sin x\cos x = \cos x\left(2\cos x + 1\right)\\
			&\Leftrightarrow & \sin x\left(2\cos x + 1\right) = \cos x\left(2\cos x + 1\right)\\
			&\Leftrightarrow & \hoac{&\cos x = - \dfrac{1}{2}\\&\sin x = \cos x}\\
			&\Leftrightarrow & \hoac{&x = \pm \dfrac{2\pi}{3} + k2\pi\\&x = \dfrac{\pi}{4} + k\pi} \quad \left(k \in \mathbb{Z}\right).
		\end{eqnarray*}
		Khi đó nghiệm dương nhỏ nhất của phương trình là $x = \dfrac{\pi}{4}$.}
\end{ex}
\begin{ex}%[Câu 50]%[1K1G4-5]
	Số nghiệm của phương trình $ \dfrac{2\sin x-1}{2\sin^2x+\sin x-1}=2 $ trong khoảng $ \left(\dfrac{\pi}{2}; \dfrac{7\pi}{2}\right) $ là
	\choice
	{$ 5 $}
	{$ 2 $}
	{$ 4 $}
	{\True $ 3 $}
	\loigiai{
		Điều kiện $ 2\sin^2 x+\sin x-1\neq 0\Leftrightarrow\heva{& \sin x\neq -1\\ & \sin x\neq \dfrac{1}{2}} $.\\
		Khi đó phương trình đã cho tương đương với \begin{eqnarray*}
			& &2\sin x-1=4\sin^2 x+2\sin x-2\\
			& \Leftrightarrow & 4\sin^2 x=1\\
			&\Leftrightarrow &\hoac{& \sin x=\dfrac{1}{2}\ (\text{không thỏa mãn điều kiện})\\
				&\sin x=-\dfrac{1}{2}\ (\text{thỏa mãn điều kiện})}\\
			&\Leftrightarrow & \hoac{& x=-\dfrac{\pi}{6}+k2\pi\\ & x=\dfrac{7\pi}{6}+k2\pi},\ k\in\mathbb{Z}.
		\end{eqnarray*}
		\begin{itemize}
			\item Trường hợp $ x=-\dfrac{\pi}{6}+k2\pi $. Khi đó, $\begin{aligned}[t]
				x\in \left(\dfrac{\pi}{2}; \dfrac{7\pi}{2}\right)&\Leftrightarrow \dfrac{\pi}{2}< -\dfrac{\pi}{6}+k2\pi<\dfrac{7\pi}{2}\\
				&\Leftrightarrow \dfrac{2\pi}{3}<k2\pi<\dfrac{11\pi}{3}\\
				&\Leftrightarrow \dfrac{1}{3}<k<\dfrac{11}{6}\\
				&\Leftrightarrow k=1 \ (\text{vì}\; k\in\mathbb{Z}).
			\end{aligned} $
			\item Trường hợp $ x=\dfrac{7\pi}{6}+k2\pi $. Khi đó, $\begin{aligned}[t]
				x\in \left(\dfrac{\pi}{2}; \dfrac{7\pi}{2}\right)&\Leftrightarrow \dfrac{\pi}{2}< \dfrac{7\pi}{6}+k2\pi<\dfrac{7\pi}{2}\\
				&\Leftrightarrow -\dfrac{\pi}{3}<k2\pi<\dfrac{7\pi}{3}\\
				&\Leftrightarrow -\dfrac{1}{6}<k<\dfrac{7}{6}\\
				&\Leftrightarrow k\in\{0; 1\} \ (\text{vì}\; k\in\mathbb{Z}).
			\end{aligned} $
		\end{itemize}
		Vậy phương trình đã cho có tất cả 3 nghiệm thuộc khoảng $ \left(\dfrac{\pi}{2}; \dfrac{7\pi}{2}\right) $.
	}
\end{ex}
\Closesolutionfile{ans}
% \setcounter{deso}{0}
\begin{name}
	{\tenchude}
	{ĐỀ ÔN TẬP CHƯƠNG I}
	{LỚP TOÁN THẦY PHÁT}
	{\thoigian}
\end{name}
\TN
%Câu 1
\begin{ex}
	Cho góc lượng giác $\alpha $. Mệnh đề nào sau đây đúng?
	\choice
	{$\sin \left(-\alpha\right)=\sin \alpha $}
	{$\cos \left(-\alpha\right)=-\cos \alpha $}
	{$\tan \left(-\alpha\right)=\tan \alpha $}
	{\True $\cot \left(-\alpha\right)=-\cot \alpha $}
	\loigiai{
		Dựa vào tính chất của hai góc đối nhau nên $\cot \left(-\alpha\right)=-\cot \alpha $
	}
\end{ex}
%Câu 2
\begin{ex}
	Giá trị $\cos 75^\circ $ là :
	\choice
	{$\dfrac{\sqrt{6}+\sqrt{2}}{4}$}
	{$\dfrac{\sqrt{6}-\sqrt{2}}{2}$}
	{\True $\dfrac{\sqrt{6}-\sqrt{2}}{4}$}
	{$\dfrac{\sqrt{6}+\sqrt{2}}{2}$}
	\loigiai{
		Ta có $\cos 75^\circ =\cos \left(30^\circ+45^\circ \right)=\cos 30^\circ \cos 45^\circ -\sin 30^\circ \sin 45^\circ=\dfrac{\sqrt{6}-\sqrt{2}}{4}$
	}
\end{ex}
%Câu 3
\begin{ex}
	Cho $\sin \alpha =\dfrac{5}{13}$ với $\dfrac{\pi }{2}<\alpha <\pi $. Mệnh đề nào sau đây đúng?
	\choice
	{$\cos \alpha =\dfrac{12}{13}$}
	{$\cos \alpha =\dfrac{8}{13}$}
	{$\cos \alpha =-\dfrac{8}{13}$}
	{\True $\cos \alpha =-\dfrac{12}{13}$}
	\loigiai{
	Ta có $\cos \alpha =\pm \sqrt{1-{{\sin }^2}\alpha }=\pm \dfrac{12}{13}$. Do $\dfrac{\pi }{2}<\alpha <\pi $ nên $\cos \alpha =-\dfrac{12}{13}$
	}
\end{ex}
%Câu 4
\begin{ex}
	Cho các góc $\alpha $, $\beta $ thỏa mãn $\alpha ,\beta \in \left(\dfrac{\pi }{2};\pi\right)$ và $\sin \alpha =\dfrac{1}{3}$, $\cos \beta =-\dfrac{2}{3}$. Tính $\sin \left(\alpha +\beta\right)$.
	\choice
	{\True $\sin \left(\alpha +\beta\right)=-\dfrac{2+2\sqrt{10}}{9}$}
	{$\sin \left(\alpha +\beta\right)=\dfrac{2\sqrt{10}-2}{9}$}
	{$\sin \left(\alpha +\beta\right)=\dfrac{\sqrt{5}-4\sqrt{2}}{9}$}
	{$\sin \left(\alpha +\beta\right)=\dfrac{\sqrt{5}+4\sqrt{2}}{9}$}
	\loigiai{
	Do $\alpha ,\beta \in \left(\dfrac{\pi }{2};\pi\right)$ nên có: $\heva{& \cos \alpha <0 \\& \sin \beta >0}$.\\
	Ta có $\cos \alpha =-\ \sqrt{1-{{\sin }^2}\alpha }=-\ \sqrt{1-\dfrac{1}{9}}=-\ \dfrac{2\sqrt{2}}{3}$ và $\sin \beta =\sqrt{1-{{\cos }^2}\beta }=\sqrt{1-\dfrac{4}{9}}=\dfrac{\sqrt{5}}{3}$.\\
	Suy ra $\sin \left(\alpha +\beta\right)=\sin \alpha \cdot \cos \beta +\cos \alpha \cdot \sin \beta =\dfrac{1}{3} \cdot \left(-\dfrac{2}{3}\right)+\left(-\dfrac{2\sqrt{2}}{3}\right) \cdot \dfrac{\sqrt{5}}{3}=-\ \dfrac{2+2\sqrt{10}}{9}$.\\
	Vậy $\sin \left(\alpha +\beta\right)=-\ \dfrac{2+2\sqrt{10}}{9}$
	}
\end{ex}
%Câu 5
\begin{ex}
	Biết $\sin \alpha +\text{cos}\alpha =m$. Tính $P=\text{cos}\left(\alpha -\dfrac{\pi }{4}\right)$ theo $m$.
	\choice
	{$P=2m$}
	{$P=\dfrac{m}{2}$}
	{\True $P=\dfrac{m}{\sqrt{2}}$}
	{$P=m\sqrt{2}$}
	\loigiai{
	Ta có $P=\text{cos}\left(\alpha -\dfrac{\pi }{4}\right)=\text{cos}\alpha \cdot \cos \dfrac{\pi }{4}+\sin \alpha \sin \dfrac{\pi }{4}=\dfrac{1}{\sqrt{2}}\text{cos}\alpha +\dfrac{1}{\sqrt{2}}\sin \alpha $\\
	$\Rightarrow P=\dfrac{1}{\sqrt{2}}\left(\sin \alpha +\text{cos}\alpha\right)=\dfrac{m}{\sqrt{2}}$
	}
\end{ex}
%Câu 6
\begin{ex}
	Cho $x=\tan \alpha $. Tính $\sin 2\alpha $ theo $x$.
	\choice
	{$2x\sqrt{1+x^2}$}
	{$\dfrac{1-x^2}{1+x^2}$}
	{$\dfrac{2x}{1-x^2}$}
	{\True $\dfrac{2x}{1+x^2}$}
	\loigiai{
	Ta có $\sin 2\alpha =2\sin \alpha \cdot \cos\alpha =2\dfrac{\sin \alpha }{\cos\alpha }\cdot \cos^2\alpha =2\tan \alpha \cdot \dfrac{1}{1+{{\tan }^2}\alpha }=\dfrac{2x}{1+x^2}$
	}
\end{ex}
%Câu 7
\begin{ex}
	Tập xác định của hàm số $y=\cot x$ là
	\choice
	{$D=\mathbb{R}$}
	{$D=\mathbb{R}\backslash \left\{ k\dfrac{\pi }{2}\left| k\in \mathbb{Z} \right. \right\}$}
	{$D=\mathbb{R}\backslash \left\{ \pi +k\dfrac{\pi }{2}\left| k\in \mathbb{Z} \right. \right\}$}
	{\True $D=\mathbb{R}\backslash \left\{ k\pi \left| k\in \mathbb{Z} \right. \right\}$}
	\loigiai{
		Điều kiện: $\sin x\ne 0\Leftrightarrow x\ne k\pi \left(k\in \mathbb{Z}\right)$.\\
		Do đó, tập xác định của hàm số $y=\cot x$ là $D=\mathbb{R}\backslash \left\{ k\pi \left| k\in \mathbb{Z} \right. \right\}$
	}
\end{ex}
%Câu 8
\begin{ex}
	Trên khoảng $\left(-\pi ;\pi\right)$, hàm số $y=\sin x$ nghịch biến trên khoảng nào sau đây?
	\choice
	{$\left(-\pi ;0\right)$}
	{$\left(-\dfrac{\pi }{2};\dfrac{\pi }{2}\right)$}
	{$\left(0;\pi\right)$}
	{\True $\left(\dfrac{\pi }{2};\pi\right)$}
	\loigiai{
		Hàm số $y=\sin x$ nghịch biến trong khoảng $\left(\dfrac{\pi }{2};\pi\right)$
	}
\end{ex}
%Câu 9
\begin{ex}
	Hàm số $y={{\sin }^2}2x-{{\cos }^2}2x$ tuần hoàn với chu kỳ bằng
	\choice
	{$2\pi $}
	{$\pi $}
	{\True $\dfrac{\pi }{2}$}
	{$\dfrac{\pi }{4}$}
	\loigiai{
	Ta có $y={{\sin }^2}2x-{{\cos }^2}2x=-\cos 4x$. Vậy hàm số đã cho tuần hoàn với chu kỳ $\dfrac{2\pi }{4}=\dfrac{\pi }{2}$
	}
\end{ex}
%Câu 10
\begin{ex}
	Nghiệm của phương trình $2\sin x+1=0$ là
	\choice
	{$x=\dfrac{\pm \pi }{6}+k2\pi ,k\in \mathbb{Z}$}
	{$x=\dfrac{\pi }{6}+k2\pi ,k\in \mathbb{Z}$}
	{$x=\dfrac{7\pi }{6}+k2\pi ,k\in \mathbb{Z}$}
	{\True $\hoac{& x=\dfrac{-\pi }{6}+k2\pi \\& x=\dfrac{7\pi }{6}+k2\pi},k\in \mathbb{Z}$}
	\loigiai{
		Ta có: $2\sin x+1=0\Leftrightarrow \sin x=\dfrac{-1}{2}\Leftrightarrow \hoac{& x=\dfrac{-\pi }{6}+k2\pi \\& x=\dfrac{7\pi }{6}+k2\pi},k\in \mathbb{Z}$
	}
\end{ex}
%Câu 11
\begin{ex}
	Phương trình nào dưới đây vô nghiệm.
	\choice
	{$\cos x=\dfrac{1}{2}$}
	{\True $\sin x-\cos x=2$}
	{$\sin (5x+1)=1$}
	{$\sin x+\sqrt{3}\cos x=1$}
	\loigiai{
		Chú ý\\
		- $\left| \sin \alpha \right|\le 1,\forall \alpha \in \mathbb{R}$ và $\left| \cos \alpha \right|\le 1,\forall \alpha \in \mathbb{R}$ nên các phương trình ở đáp án A, C có nghiệm.\\
		- Phương trình $a\sin x+b\cos x=c$ có nghiệm khi $a^2+b^2\ge c^2$, ta kiểm tra được phương trình đáp án B vô nghiệm, đáp án D có nghiệm
	}
\end{ex}
%Câu 12
\begin{ex}
	Cho phương trình $2\tan x-3=\dfrac{-2}{\tan x+1}$. Gọi $S$ là tập hợp các nghiệm của phương trình thuộc khoảng $\left(0;\dfrac{\pi }{2}\right)$. Tổng các phần tử của $S$ là
	\choice
	{$0$}
	{$\dfrac{\pi }{3}$}
	{\True $\dfrac{\pi }{4}$}
	{$1$}
	\loigiai{
		Điều kiện : $\cos x\ne 0,\tan x\ne -1$.\\
		Vì $x\in \left(0;\dfrac{\pi }{2}\right)\Rightarrow \tan x>0$.\\
		Phương trình ban đầu tương đương\\
		$\begin{aligned}
				& \Leftrightarrow \left(2\tan x-3\right)\left(\tan x+1\right)=-2\Leftrightarrow 2{{\tan }^2}x-\tan x-3=-2 \\& \Leftrightarrow 2{{\tan }^2}x-\tan x-1=0 \end{aligned}$\\
		$\Leftrightarrow \hoac{& \tan x=1\begin{matrix}\\
					{} & (TM) \\\\
				\end{matrix} \\& \tan x=\dfrac{-1}{2}(L)}$\\
		+ Với $\tan x=1\Leftrightarrow x=\dfrac{\pi }{4}+k\pi ,k\in \mathbb{Z}$. Vì $x\in \left(0;\dfrac{\pi }{2}\right)$ nên $x=\dfrac{\pi }{4}$.\\
		Vậy $S=\left\{ \dfrac{\pi }{4} \right\}$ và tổng các phần tử của $S$ là $\dfrac{\pi }{4}$
	}
\end{ex}
\TNTF
%Câu 13
\begin{ex}
	Xét tính đúng sai của các mệnh đề sau:
	\choiceTF
	{${{\sin }^2}x=\dfrac{1+\sin 2x}{2}$}
	{\True Nếu $\cos \alpha =\dfrac{1}{3}$ thì $\cos 2\alpha =-\dfrac{7}{9}$}
	{\True Nếu $\sin x=\dfrac{3}{4}$ với $x\in \left(0;\dfrac{\pi }{2}\right)$ thì $\sin 2x=\dfrac{3\sqrt{7}}{8}$}
	{\True Cho $\cos \alpha =\dfrac{2}{3}$ với $\alpha \in \left(-\dfrac{\pi }{2};0\right)$ biết $\tan \left(\alpha +\dfrac{\pi }{4}\right)=a+b\sqrt{c}$, $c$ là số nguyên tố $\left(a,b,c\in \mathbb{Z},c\ge 0\right)$ Khi đó $a+b+c=0$}
	\loigiai{
	a) ${{\sin }^2}x=\dfrac{1-\cos 2x}{2}$\\
	b) $\cos 2\alpha =2{{\cos }^2}\alpha -1=2{{\left(\dfrac{1}{3}\right)}^2}-1=\dfrac{-7}{9}$\\
	c) Ta có ${{\cos }^2}x=1-{{\sin }^2}x=1-{{\left(\dfrac{3}{4}\right)}^2}=\dfrac{7}{16}$.\\
	Vì $x\in \left(0;\dfrac{\pi }{2}\right)$ nên $\cos x>0\Rightarrow \cos x=\dfrac{\sqrt{7}}{4}$ suy ra $\sin 2x=2\sin x \cdot \cos x=2\cdot \dfrac{\sqrt{7}}{4}\cdot \dfrac{3}{4}=\dfrac{3\sqrt{7}}{8}$\\
	d) Ta có ${{\tan }^2}\alpha =\dfrac{1}{{{\cos }^2}\alpha }-1=\dfrac{1}{{{\left(\dfrac{2}{3}\right)}^2}}-1=\dfrac{5}{4}$\\
	Vì $\alpha \in \left(-\dfrac{\pi }{2};0\right)$ nên $\tan \alpha <0\Rightarrow \tan \alpha =\dfrac{-\sqrt{5}}{2}$\\
	$\tan \left(\alpha +\dfrac{\pi }{4}\right)=\dfrac{\tan \alpha +\tan \dfrac{\pi }{4}}{1-\tan \alpha \cdot \tan \dfrac{\pi }{4}}=\dfrac{\dfrac{-\sqrt{5}}{2}+1}{1-\left(\dfrac{-\sqrt{5}}{2}\right) \cdot 1}=-9+4\sqrt{5}$\\
	Vậy $a=-9,b=4,c=5$ nên mệnh đề đúng
	}
\end{ex}
%Câu 14
\begin{ex}
	Biết $\cos x=\dfrac{1}{3}$ và $-\dfrac{\pi }{2}<x<0$. Khi đó: Các mệnh đề sau đúng hay sai?
	\choiceTF
	{\True $\sin \left(\dfrac{\pi }{2}-x\right)>0$}
    {$\sin 2x=\dfrac{4\sqrt{2}}{9}$}
	{\True $\cos \left(x+\dfrac{4\pi }{3}\right)=-\dfrac{1+3\sqrt{6}}{6}$}
	{\True $\sin x+\sin 3x=-\dfrac{8\sqrt{2}}{27}$}
	\loigiai{
	a) Ta có $\sin \left(\dfrac{\pi }{2}-x\right)=\cos x=\dfrac{1}{3}>0$\\
	b) Ta có ${{\sin }^2}x=1-{{\cos }^2}x=1-{{\left(\dfrac{1}{3}\right)}^2}=\dfrac{8}{9}\Rightarrow \sin x=\pm \dfrac{2\sqrt{2}}{3}$.\\
	Vì $-\dfrac{\pi }{2}<x<0$ nên $\sin x=-\dfrac{2\sqrt{2}}{3}$.\\
	Áp dụng công thức nhân đôi ta có: $\sin 2x=2\sin x\cos x=2 \cdot \left(-\dfrac{2\sqrt{2}}{3}\right) \cdot \dfrac{1}{3}=-\dfrac{4\sqrt{2}}{9}$\\
	c) $\cos \left(x+\dfrac{4\pi }{3}\right)=\cos x \cdot \cos \dfrac{4\pi }{3}-\sin x \cdot \sin \dfrac{4\pi }{3}=\dfrac{1}{3} \cdot \left(-\dfrac{1}{2}\right)-\left(-\dfrac{2\sqrt{2}}{2}\right) \cdot \left(-\dfrac{\sqrt{3}}{2}\right)=-\dfrac{1+3\sqrt{6}}{6}$\\
	d) Áp dụng công thức ta có:\\
	$\sin x+\sin 3x=2\sin 2x \cdot \cos x=2 \cdot \left(-\dfrac{4\sqrt{2}}{9}\right) \cdot \dfrac{1}{3}=-\dfrac{8\sqrt{2}}{27}$
	}
\end{ex}
%Câu 15
\begin{ex}
	Cho hàm số $f(x)=-2\sin \left(2x-\dfrac{\pi }{2}\right)+2025$. Các mệnh đề sau đúng hay sai?
	\choiceTF
	{\True Hàm số $f(x)$ có tập xác định là $\mathbb{R}$}
	{Hàm số $f(x)$ tuần hoàn với chu kì $T=2\pi $}
	{Hàm số $f(x)$ không chẵn, không lẻ}
	{\True Hàm số $f(x)$ đạt giá trị lớn nhất tại $x=k\pi ,k\in \mathbb{Z}$}
	\loigiai{
		a). Vì tập xác định của hàm $\sin $ là $\mathbb{R}$ nên hàm số $f(x)$ có tập xác định là $\mathbb{R}$.\\
		b). Ta có $-2\sin \left(2x-\dfrac{\pi }{2}\right)+2025=2\sin \left(\dfrac{\pi }{2}-2x\right)+2025=2\cos 2x+2025$.\\
		Do đó $f(x)=2\cos 2x+2025$ nên hàm số $f(x)$ tuần hoàn với chu kì $T=\dfrac{2\pi }{2}=\pi $.\\
		c) Ta có $\forall x\in \mathbb{R},-x\in \mathbb{R}$ và $f(-x)=2\cos (-2x)+2025=2\cos 2x+2025=f(x)$ nên hàm số $f(x)$ là hàm số chẵn.\\
		d) Ta có $-2\le 2\cos 2x\le 2,\forall x\in \mathbb{R}$ hay $2023\le 2\cos 2x+2025\le 2027,\forall x\in \mathbb{R}$.\\
		Do đó $f(x)=2027\Leftrightarrow \cos 2x=1\Leftrightarrow x=k\pi ,k\in \mathbb{Z}$.\\
		Vậy hàm số $f(x)$ đạt giá trị lớn nhất tại $x=k\pi ,k\in \mathbb{Z}$
	}
\end{ex}
%Câu 16
\begin{ex}
	Cho hàm số $f(x)=\dfrac{1}{{{\cos }^2}x}+\dfrac{1}{{{\sin }^2}x}$. Xét tính đúng sai của các mệnh đề sau
	\choiceTF
	{\True Hàm số đã cho là hàm số tuần hoàn}
	{\True Hàm số đã cho là hàm số chẵn}
	{Tập xác định của hàm số là $D=\mathbb{R}\backslash \left\{ \dfrac{\pi }{2}+k\pi ,k\in \mathbb{Z} \right\}$}
	{\True Giá trị nhỏ nhất của hàm số là 4}
	\loigiai{
		a) Hàm số tuần hoàn do hai hàm $y=\operatorname{sinx}$ và $y=\cos x$ cùng tuần hoàn với chu kì $2\pi $.\\
		b) Ta có $f(-x)=\dfrac{1}{{{\cos }^2}(-x)}+\dfrac{1}{{{\sin }^2}(-x)}=\dfrac{1}{{{\left(\operatorname{cosx}\right)}^2}}+\dfrac{1}{{{\left(-\sin x\right)}^2}}=\dfrac{1}{{{\cos }^2}x}+\dfrac{1}{{{\sin }^2}x}=f(x)$.\\
		Do đó hàm số đã cho là hàm số chẵn\\
		c) Hàm số xác định khi $\heva{& \operatorname{sinx}\ne 0 \\& \operatorname{cosx}\ne 0}\Leftrightarrow \sin 2x\ne 0\Leftrightarrow 2x\ne k\pi \Leftrightarrow x\ne \dfrac{k\pi }{2},k\in \mathbb{Z}$\\
		Tập xác định của hàm số là $D=\mathbb{R}\backslash \left\{ \dfrac{k\pi }{2},k\in \mathbb{Z} \right\}$.\\
		d) Khi $x\ne \dfrac{k\pi }{2},k\in \mathbb{Z}$ ta có\\
		$f(x)=\dfrac{1}{{{\cos }^2}x}+\dfrac{1}{{{\sin }^2}x}\ge 2\sqrt{\dfrac{1}{{{\cos }^2}x} \cdot \dfrac{1}{{{\sin }^2}x}}=2\sqrt{\dfrac{4}{{{\sin }^2}2x}}=\dfrac{4}{\left| \sin 2x \right|}\ge \dfrac{4}{1}=4$.\\
		Nên giá trị nhỏ nhất của hàm số là 4
	}
\end{ex}
\TNSA
%Câu 19
\begin{ex}
    Tìm tập giá trị của các hàm số $y=\sqrt{2+\cos x}-5$ là đoạn $[a;b]$. Giá trị $a+b$ (làm tròn đến hàng phần chục) là
    \shortans{-7,3}
    \loigiai{
    Vì $\cos x\ge -1\Leftrightarrow 2+\cos x\ge 1>0,\forall x\in \mathbb{R}$ nên tập xác định của hàm số là $D=\mathbb{R}$.\\
    $\forall x\in \mathbb{R}$, ta có:
    \begin{eqnarray*}
        & & -1\le \cos x\le 1 \\
        & \Leftrightarrow & 1\le 2+\cos x\le 3 \\
        & \Leftrightarrow & 1\le \sqrt{2+\cos x}\le \sqrt{3} \\
        & \Leftrightarrow & -4\le \sqrt{2+\cos x}\,-5\le \sqrt{3}-5
    \end{eqnarray*}
    Vậy tập giá trị của hàm số là $T=\left[-4;\sqrt{3}-5\right]$. Suy ra $a+b \approx -7,3$.
    }    
\end{ex}
%Câu 18
\begin{ex}
	Tổng số giờ ban ngày của ngày thứ $x$ trong một năm không nhuận được tính bởi công thức
	$g(x)=3\sin (0,0172x-1,376)+12$.
	Trong đó $x$ đại diện cho ngày trong năm, $1\le x\le 365$. Ngày $\overline{ab}$ tháng $\overline{cd}$ có số giờ ban ngày dài nhất. Số $\overline{abcd}$ bằng
	\shortans{2006}
	\loigiai{
		Ta có $-1\le \sin (0,0172x-1,376)\le 1$\\
		$-3\le 3\sin (0,0172x-1,376)\le 3$\\
		$9\le 3\sin (0,0172x-1,376)+12\le 15$\\
		Suy ra $9\le g(x)\le 15$\\
		Do đó, số giờ ban ngày dài nhất trong một ngày là 15 giờ.\\
		Ta có phương trình $3\sin (0,0172x-1,376)+12=15$\\
		$\sin (0,0172x-1,376)=1$\\
		$x\approx 171{,}3$\\
		Vậy vào khoảng ngày thứ 171 trong năm (ngày 20 tháng 6) thì số giờ ban ngày dài nhất
	}
\end{ex}
%Câu 19
\begin{ex}
	Hai thành phố có cùng kinh độ. Vĩ tuyến của thành phố A là $10^\circ $ Bắc và vĩ tuyến của thành phố B là $40^\circ $ Bắc. Giả sử bán kính trái đất là 3960 dặm. Tìm khoảng cách giữa hai thành phố (làm tròn đến chữ số hàng đơn vị)
	\shortans{2073}
	\loigiai{
		Khoảng cách từ điểm trên đường xích đạo đến thành phố B ở cùng kinh độ là $3960 \cdot \dfrac{40}{180} \cdot \pi =880\pi $ (dặm)\\
		Khoảng cách từ điểm trên đường xích đạo đến thành phố A ở cùng kinh độ là $3960 \cdot \dfrac{10}{180}\pi =220\pi $ (dặm)\\
		Khoảng cách giữa hai thành phố A và B là $880\pi -220\pi =660\pi \approx 2073$ (dặm)
	}
\end{ex}
%Câu 20
\begin{ex}
	Giả sử vận tốc $v$ (tính bằng lít/ giây) của luồng khí trong một chu kì hô hấp (tức là thời gian từ lúc bắt đầu của một nhịp thở đến khi bắt đầu của nhịp thở tiếp theo) của một người nào đó ở trạng thái nghỉ ngơi được cho bởi công thức $v=0{,}85\sin \dfrac{\pi t}{3}$, trong đó $t$ là thời gian (tính bằng giây). Biết rằng quá trình hít vào xảy ra khi $v>0$ và quá trình thở ra xảy ra khi $v<0$. Trong khoảng thời gian từ 5 đến 10 giây, khoảng thời điểm sau $a$ giây đến trước $b$ giây thì người đó hít vào. Tính $\sqrt{a+b}$ (làm tròn đến hàng phần trăm).
	\shortans{3,87}
	\loigiai{
		+) Vì quá trình hít vào xảy ra khi $v>0$ nên ta có\\
		$0{,}85\sin \dfrac{\pi t}{3}>0\Leftrightarrow \sin \dfrac{\pi t}{3}>0\Leftrightarrow \dfrac{\pi t}{3}\in \left(k2\pi ;\pi +k2\pi\right)(k\in \mathbb{Z})$\\
		$\Leftrightarrow t\in \left(6k;3+6k\right)\,\left(k\in \mathbb{Z}\right)$\\
		+) Vì $t\in [5;10]$ nên $k=1$ suy ra $t\in \left(6;9\right)$.\\
		Trong khoảng thời gian từ 5 đến 10 giây, khoảng thời điểm sau $6$ giây đến trước $9$ giây thì người đó hít vào nên $\sqrt{a+b}=\sqrt{15}\approx 3,87$.
	}
\end{ex}
%Câu 21
\begin{ex}
	Nghiệm phương trình lượng giác $\sqrt{3}\sin x-\cos x=0$ có dạng $x=\dfrac{\pi }{a}+k \cdot b\pi $ ($a$, $b$, $k\in \mathbb{Z}$, $a\ne 0$). Tính $(a+b)^4$.
	\shortans{2401}
	\loigiai{
		Phương trình tương đương\\
		$\dfrac{\sqrt{3}}{2}\sin x-\dfrac{1}{2}\cos x=0$\\
		$\Leftrightarrow \sin x\cos \dfrac{\pi }{6}-\cos x\sin \dfrac{\pi }{6}=0$\\
		$\Leftrightarrow \sin \left(x-\dfrac{\pi }{6}\right)=0$\\
		$\Leftrightarrow x-\dfrac{\pi }{6}=k\pi $ ($k\in \mathbb{Z}$)\\
		$\Leftrightarrow x=\dfrac{\pi }{6}+k\pi $.\\
		Phương trình có nghiệm là: $x=\dfrac{\pi }{6}+k\pi $ ($k\in \mathbb{Z}$).\\
		Suy ra $a=6$; $b=1$. Vậy $(a+b)^4=7^4=2401$.
	}
\end{ex}
%Câu 22
\begin{ex}
	Một vật $M$ được gắn vào đầu lò xo và dao động quanh vị trí cân bằng, toạ độ $x$ (đơn vị: cm) tại thời điểm $t$ (giây) được tính bởi công thức $x=8{,}6\sin \left(8t+\dfrac{\pi }{2}\right)$. Có $n$ thời điểm trong khoảng 2 giây đầu tiên thì $s=4{,}3$ cm. Giá trị $\sqrt[3]{n}$ (làm tròn đến hàng phần trăm)
	\shortans{1,71}
	\loigiai{
	Khi $x=4{,}3$ thì $8{,}6\sin \left(8t+\dfrac{\pi }{2}\right)=4{,}3\Rightarrow \sin \left(8t+\dfrac{\pi }{2}\right)=\dfrac{1}{2}$\\
	$\Leftrightarrow \hoac{&8t+\dfrac{\pi }{2}=\dfrac{\pi }{6}+k2\pi  \\
			&8t+\dfrac{\pi }{2}=\dfrac{5\pi }{6}+l2\pi}(k,l\in \mathbb{Z})
            \Leftrightarrow \hoac{& t=-\dfrac{\pi }{24}+k\dfrac{\pi }{4} \\
            & t=\dfrac{\pi }{24}+l\dfrac{\pi }{4} }(k,l\in \mathbb{Z})$.\\
	Vì $t\in (0;2)$ nên $\heva{& 0<-\dfrac{\pi }{24}+k\dfrac{\pi }{4}<2 \\ & 0<\dfrac{\pi }{24}+l\dfrac{\pi }{4}<2} \Leftrightarrow \heva{& \dfrac{1}{6}<k<\dfrac{8}{\pi }+\dfrac{1}{6}  \\ & -\dfrac{1}{6}<l<\dfrac{8}{\pi }-\dfrac{1}{6}}$\\
	Mà $k,l\in \mathbb{Z}$ nên $k\in \left\{ 1;2 \right\}$; $l\in \left\{ 0;1;2 \right\}$.\\
	Vậy có $5$ thời điểm thỏa mãn đề bài nên $\sqrt[3]{n}\approx 1,71$
	}
\end{ex}

% \begin{name}
	{\tenchude}
	{ĐỀ ÔN TẬP CHƯƠNG I}
	{LỚP TOÁN THẦY PHÁT}
	{\thoigian}
\end{name}
\TN
\setcounter{ex}{0}
\Opensolutionfile{ans}[ans/ans-TN-C1-De1]
\TN
%Câu 1
\begin{ex}
	Rút gọn biểu thức $M=\cos 2x \cdot \cos x+\sin 2x \cdot \sin x$ ta được kết quả là:
	\choice
	{\True $M=\cos x$}
	{$M=\cos 3x$}
	{$M=\sin x$}
	{$M=\sin 3x$}
	\loigiai{
		Ta có: $M=\cos 2x \cdot \cos x+\sin 2x \cdot \sin x=\cos (2x-x)=\cos x$
	}
\end{ex}
%Câu 2
\begin{ex}
	Đẳng thức nào không đúng với mọi $x$?
	\choice
	{$\cos^2 3x=\dfrac{1+\cos 6x}{2}$}
	{$\cos 2x=1-2\sin^2x$}
	{$\sin 2x=2\sin x\cos x$}
	{\True $\sin^2 2x=\dfrac{1+\cos 4x}{2}$}
	\loigiai{
		Ta có $\sin^2 2x=\dfrac{1-\cos 4x}{2}$
	}
\end{ex}
%Câu 3
\begin{ex}
	Góc có số đo $\dfrac{\pi }{24}$ đổi sang độ bằng
	\choice
	{$7^\circ $}
	{\True $7^\circ 3{0}'$}
	{$8^\circ $}
	{$8^\circ 3{0}'$}
	\loigiai{
		Ta có: $\dfrac{\pi }{24}=\dfrac{180^\circ }{24}=7^\circ 30'$
	}
\end{ex}
%Câu 4
\begin{ex}
	Một đường tròn có đường kính là $50$ (cm). Độ dài của cung tròn trên đường tròn có số đo là $\dfrac{\pi }{4}$ bằng (làm tròn đến hàng đơn vị)
	\choice
	{$40$ (cm)}
	{$39$ (cm)}
	{$19$ (cm)}
	{\True $20$ (cm)}
	\loigiai{
		Độ dài của cung tròn $l=\alpha \cdot R=\dfrac{\pi }{4} \cdot 25=\dfrac{25}{4}\pi \approx 20$ (cm)
	}
\end{ex}
%Câu 5
\begin{ex}
	Chọn phát biểu đúng:
	\choice
	{Các hàm số $y=\sin x$, $y=\cos x$, $y=\cot x$ đều là hàm số chẵn}
	{Các hàm số $y=\sin x$, $y=\cos x$, $y=\cot x$ đều là hàm số lẻ}
	{Các hàm số $y=\sin x$, $y=\cot x$, $y=\tan x$ đều là hàm số chẵn}
	{\True Các hàm số $y=\sin x$, $y=\cot x$, $y=\tan x$ đều là hàm số lẻ}
	\loigiai{
		Hàm số $y=\cos x$ là hàm số chẵn, hàm số $y=\sin x$, $y=\cot x$, $y=\tan x$ là các hàm số lẻ
	}
\end{ex}
%Câu 6
\begin{ex}
	Nếu $\sin x+\cos x=\dfrac{1}{2}$ thì $\sin 2x$ bằng
	\choice
	{$\dfrac{3}{4}$}
	{$\dfrac{3}{8}$}
	{$\dfrac{\sqrt{2}}{2}$}
	{\True $\dfrac{-3}{4}$}
	\loigiai{
	Do $\sin x+\cos x=\dfrac{1}{2}\Rightarrow \dfrac{1}{4}={{\left(\sin x+\cos x\right)}^2}={{\left(\sin x\right)}^2}+{{\left(\text{cosx}\right)}^2}+2\sin x \cdot \cos x$\\
	$\Rightarrow \dfrac{1}{4}=1+\sin 2x\Rightarrow \sin 2x =\dfrac{-3}{4}$
	}
\end{ex}
%Câu 7
\begin{ex}
	Một con lắc lò xo sau khi được kéo xuống dưới vị trí cân bằng $4$ cm và thả ra thì nó dao động điều hòa với phương trình: $y=-4\cos 8t$ (cm). Biên độ $A$ cm và chu kỳ $T$ của dao động là
	\choice
	{\True $A=4$ cm, $T=\dfrac{\pi }{4}$}
	{$A=4$ cm, $T=\dfrac{\pi }{2}$}
	{$A=8$ cm, $T=\dfrac{\pi }{4}$}
	{$A=4$ cm, $T=2\pi $}
	\loigiai{
		Biên độ của dao động là: $A=|-4|=4$ (cm).\\
		Chu kỳ của dao động là:$T=\dfrac{2\pi }{|8|}=\dfrac{\pi }{4}$
	}
\end{ex}
%Câu 8
\begin{ex}
	Hãy tìm tập tất cả các giá trị của $m$ để phương trình $\left| \sin x \right|=m$ có nghiệm?
	\choice
	{$-1\le m\le 1$}
	{$-1\le m\le 0$}
	{$-1<m<0$}
	{\True $0\le m\le 1$}
	\loigiai{
		Vì $0<= |\sin x|<=1, \forall x \in \mathbb{R}$ nên phương trình $\left| \sin x \right|=m$ có nghiệm khi và chỉ khi $0\le m\le 1$.
	}
\end{ex}
%Câu 9
\begin{ex}
	Nghiệm của phương trình $2\sin \left(4x-\dfrac{\pi }{3}\right)-1=0$ là:
	\choice
	{$x=\pi +k2\pi ;x=k\dfrac{\pi }{2}\ (k \in \mathbb{Z})$}
	{$x=\dfrac{\pi }{8}+k\dfrac{\pi }{2};x=\dfrac{7\pi }{24}+k\dfrac{\pi }{2}\ (k \in \mathbb{Z})$}
	{$x=k2\pi ;x=\dfrac{\pi }{2}+k2\pi\ (k \in \mathbb{Z})$}
	{$x=k\pi ;x=\pi +k2\pi\ (k \in \mathbb{Z})$}
	\loigiai{
		$2\sin \left(4x-\dfrac{\pi }{3}\right)-1=0\Leftrightarrow \sin \left(4x-\dfrac{\pi }{3}\right)=\dfrac{1}{2}\Leftrightarrow \hoac{& 4x-\dfrac{\pi }{3}=\dfrac{\pi }{6}+k2\pi \\& 4x-\dfrac{\pi }{3}=\pi -\dfrac{\pi }{6}+k2\pi}\Leftrightarrow \hoac{& x=\dfrac{\pi }{8}+k\dfrac{\pi }{2} \\& x=\dfrac{7\pi }{24}+k\dfrac{\pi }{2}}\left(k\in \mathbb{Z}\right)$
	}
\end{ex}
%Câu 10
\begin{ex}
	Biết $\sin \left(\alpha +\dfrac{3\pi }{2}\right)+\cos \left(\alpha +\dfrac{3\pi }{2}\right)=\sqrt{2}$. Tính $\sin \left(\alpha +\pi\right)-2\cos \left(\alpha -\pi\right)$.
	\choice
	{$\dfrac{3}{\sqrt{2}}$}
	{\True $-\dfrac{3}{\sqrt{2}}$}
	{$-\dfrac{1}{\sqrt{2}}$}
	{$\dfrac{1}{\sqrt{2}}$}
	\loigiai{
	Ta có $\sin \left(\alpha +\dfrac{3\pi }{2}\right)=\sin \left(\alpha +2\pi -\dfrac{\pi }{2}\right)=\sin \left(\alpha -\dfrac{\pi }{2}\right)=-\sin \left(\dfrac{\pi }{2}-\alpha\right)=-\cos \alpha $.\\
	$\cos \left(\alpha +\dfrac{3\pi }{2}\right)=\cos \left(\alpha +2\pi -\dfrac{\pi }{2}\right)=\cos \left(\alpha -\dfrac{\pi }{2}\right)=\cos \left(\dfrac{\pi }{2}-\alpha\right)=\sin \alpha $.\\
	Suy ra $\sin \alpha -\cos \alpha =\sqrt{2}\Rightarrow \sin \alpha =\cos \alpha +\sqrt{2}$.\\
	Vì ${{\sin }^2}\alpha +{{\cos }^2}\alpha =1\Rightarrow 2{{\cos }^2}\alpha +2\sqrt{2}\cos \alpha +2=1$\\
	$\Leftrightarrow 2{{\cos }^2}\alpha +2\sqrt{2}\cos \alpha +1=0\Leftrightarrow \cos \alpha =-\dfrac{1}{\sqrt{2}}\Rightarrow \sin \alpha =\dfrac{1}{\sqrt{2}}$.\\
	Do đó $\sin \left(\alpha +\pi\right)-2\cos \left(\alpha -\pi\right)=-\sin \alpha +2\cos \alpha =-\dfrac{3}{\sqrt{2}}$
	}
\end{ex}
%Câu 11
\begin{ex}
	Hằng ngày mực nước của con kênh lên xuống theo thủy triều. Độ sâu $h$(mét) của mực nước trong kênh được tính tại thời điểm $t$ (giờ) trong một ngày bởi công thức $h=3\cos \left(\dfrac{\pi t}{7=8}+\dfrac{\pi }{4}\right)+12$. Mực nước của kênh cao nhất khi:
	\choice
	{$t=13$(giờ)}
	{\True $t=14$(giờ)}
	{$t=15$(giờ)}
	{$t=16$(giờ)}
	\loigiai{
		Mực nước của kênh cao nhất khi $h$ lớn nhất\\
		$\Leftrightarrow \cos \left(\dfrac{\pi t}{8}+\dfrac{\pi }{4}\right)=1\Leftrightarrow \dfrac{\pi t}{8}+\dfrac{\pi }{4}=k2\pi $ với $0<t\le 24$ và $k\in \mathbb{Z}$.\\
		Lần lượt thay các đáp án, ta được đáp án B thỏa mãn.\\
		Vì với $t=14$ thì $\dfrac{\pi t}{8}+\dfrac{\pi }{4}=2\pi $ (đúng với $k=1\in \mathbb{Z}$)
	}
\end{ex}
%Câu 12
\begin{ex}
	Số giờ có ánh sáng mặt trời của một thành phố A ở vĩ độ ${{40}^{\text{o}}}$ bắc trong ngày thứ t của một năm không nhuận được cho bởi hàm số $d(t)=3\sin \left[\dfrac{\pi }{180}(t-80)\right]+12$ với $t\in \mathbb{Z}$ và $0<t\le 365$. Vào ngày nào trong năm thì thành phố A có nhiều giờ có ánh sáng mặt trời nhất?
	\choice
	{\True 170}
	{171}
	{172}
	{173}
	\loigiai{
		Ta có $d(t)=3\sin \left[\dfrac{\pi }{180}(t-80)\right]+12\le 3 \cdot 1+12=15$.\\
		Vậy thành phố A có nhiều giờ có ánh sáng mặt trời nhất khi $\sin \left[\dfrac{\pi }{180}(t-80)\right]=1\Leftrightarrow \dfrac{\pi }{180}(t-80)=\dfrac{\pi }{2}+k2\pi \Leftrightarrow t=170+360k (k\in \mathbb{Z})$.\\
		Vì $0<t\le 365$ nên $0<170+360k\le 365\Leftrightarrow -\dfrac{17}{36}<k\le \dfrac{39}{72}\Rightarrow k=0\Rightarrow t=170$.
	}
\end{ex}

\TNTF
%Câu 13
\begin{ex}
	Cho phương trình $\sin x=a$ (1).
	\choiceTF
	{\True Nếu $a>1$ thì phương trình (1) vô nghiệm}
	{Nếu $a=1$ thì phương trình (1) có nghiệm $\alpha =\dfrac{\pi }{2}+k\pi ,\left(k\in \mathbb{Z}\right)$}
	{\True Nếu $-1\le a\le 1$ thì phương trình (1) có nghiệm $\hoac{& x=\alpha +k2\pi \\ & x=\pi -\alpha +k2\pi} \left(k\in \mathbb{Z}\right)$}
	{Phương trình (1) luôn có hai điểm biểu diễn nghiệm trên đường tròn lượng giác}
	\loigiai{
		Nếu $a=1\Rightarrow \sin \alpha =1\Leftrightarrow \alpha =\dfrac{\pi }{2}+k2\pi ,\left(k\in \mathbb{Z}\right)$
	}
\end{ex}
%Câu 14
\begin{ex}
	Các mệnh đề sau đúng hay sai?
	\choiceTF
	{\True Hàm số $y=\sin \sqrt{x+4}$ có tập xác định là $D=\left[-4;+\infty\right)$}
	{Hàm số $y=\cot \left(\dfrac{\pi }{2}+x\right)$ có tập xác định là $D=\mathbb{R}$}
	{\True Hàm số $y=\sqrt{3-2\cos x}$ có tập xác định là $D=\mathbb{R}$}
	{Hàm số $y=\dfrac{1-3\cos x}{\sin x}$ có tập xác định là $D=\mathbb{R}\backslash \left\{ k\dfrac{\pi }{2},k\in \mathbb{Z} \right\}$}
	\loigiai{
	a) Hàm số xác định khi và chỉ khi $x+4\ge 0\Leftrightarrow x\ge -4$.\\
	Vậy tập xác định của hàm số là $D=\left[-4;+\infty\right)$.\\
	b) Hàm số xác định khi và chỉ khi $\sin \left(x+\dfrac{\pi }{2}\right)\ne 0\Leftrightarrow x+\dfrac{\pi }{2}\ne k\pi \Leftrightarrow x\ne -\dfrac{\pi }{2}+k\pi ;k\in \mathbb{Z}$.\\
	Vậy tập xác định của hàm số là $D=\mathbb{R}\backslash \left\{ -\dfrac{\pi }{2}+k\pi ;k\in \mathbb{Z} \right\}$.\\
	c) Hàm số xác định khi $3-2\cos x\ge 0\Leftrightarrow \cos x\le \dfrac{3}{2}$ (đúng $\forall x\in \mathbb{R}$), vì $-1\le \cos x\le 1,\forall x\in \mathbb{R}$.\\
	Vậy tập xác định của hàm là $D=\mathbb{R}$.\\
	d) Hàm số xác định khi và chỉ khi $\sin x\ne 0\Leftrightarrow x\ne k\pi \left(k\in \mathbb{Z}\right)$.\\
	Vậy tập xác định của hàm số là $D=\mathbb{R}\backslash \left\{ k\pi ,k\in \mathbb{Z} \right\}$
	}
\end{ex}
%Câu 15
\begin{ex}
	Hằng ngày mực nước của con kênh lên xuống theo thủy triều. Độ sâu $h$ (mét) của mực nước trong kênh tính theo thời gian $t$ (giờ) được cho bởi công thức $h(t)=3\cos \left(\dfrac{\pi t}{6}+\dfrac{\pi }{4}\right)+14$.
	\choiceTF
	{Công thức tuần hoàn với chu kì $T=2\pi $}
	{\True Chiều sâu của mực nước thấp nhất là $11 \text{m}$}
	{Chiều sâu của mực nước cao nhất là $14 \text{m}$}
	{\True Thời gian để mực nước cao nhất là $t=9$}
	\loigiai{
		a) Công thức có dạng $y=\cos (ax+b)$ tuần hoàn với chu kì $T=\dfrac{2\pi }{|a|}$ nên chu kì cần tìm là $T=\dfrac{2\pi }{\left| \dfrac{\pi }{6} \right|}=12$.\\
		b) Ta có $\forall t\colon -1\le \cos \left(\dfrac{\pi t}{6}+\dfrac{\pi }{4}\right)\le 1\Leftrightarrow -3\le 3\cos \left(\dfrac{\pi t}{6}+\dfrac{\pi }{4}\right)\le 3\Leftrightarrow 11\le 3\cos \left(\dfrac{\pi t}{6}+\dfrac{\pi }{4}\right)+14\le 17\Leftrightarrow 11\le h\le 17$. Vậy chiều sâu của mực nước thấp nhất là $11 \text{m}$.\\
		c) Ta có $\forall t\colon -1\le \cos \left(\dfrac{\pi t}{6}+\dfrac{\pi }{4}\right)\le 1\Leftrightarrow -3\le 3\cos \left(\dfrac{\pi t}{6}+\dfrac{\pi }{4}\right)\le 3\Leftrightarrow 11\le 3\cos \left(\dfrac{\pi t}{6}+\dfrac{\pi }{4}\right)+14\le 17\Leftrightarrow 11\le h\le 17$. Chiều sâu của mực nước cao nhất là $17 \text{m}$.\\
		d) Ta có $\forall t\colon -1\le \cos \left(\dfrac{\pi t}{6}+\dfrac{\pi }{4}\right)\le 1\Leftrightarrow -3\le 3\cos \left(\dfrac{\pi t}{6}+\dfrac{\pi }{4}\right)\le 3\Leftrightarrow 11\le 3\cos \left(\dfrac{\pi t}{6}+\dfrac{\pi }{4}\right)+14\le 17\Leftrightarrow 11\le h\le 17$. Chiều sâu của mực nước cao nhất là $17 \text{m}$.\\
		Max $h=17\Leftrightarrow \cos \left(\dfrac{\pi t}{6}+\dfrac{\pi }{4}\right)=1\Leftrightarrow \dfrac{\pi t}{6}+\dfrac{\pi }{4}=k2\pi \Leftrightarrow t=-3+12k,k\in \mathbb{Z}$.\\
		Vì thời gian không âm và $k\in \mathbb{Z}$ nên ta chọn $t=1$. Vậy thời gian ngắn nhất $t=-3+12=9$
	}
\end{ex}
%Câu 16
\begin{ex}
	Cho phương trình $\left(2\cos x-1\right)\left(\sin 2x-m\right)=0$ (1).
	\choiceTF
	{\True $x=\dfrac{7\pi }{3}$ là một nghiệm của phương trình $(1)$}
	{Khi $m=2$ thì phương trình $(1)\Leftrightarrow \hoac{& x=\pm\dfrac{\pi }{3}+k2\pi \\& x=\dfrac{\pi }{2}+l2\pi} (k,l \in \mathbb{Z})$}
	{\True Khi $m=1$ thì tập nghiệm của phương trình $(1)$ có tất cả 4 điểm biểu diễn trên đường tròn lượng giác}
	{Chỉ tìm được một giá trị của $m$ để phương trình $(1)$ có đúng hai nghiệm thuộc $\left(-\dfrac{\pi }{4};\dfrac{3\pi }{4}\right]$}
			\loigiai{
			Ta có $\left(2\cos x-1\right)\left(\sin 2x-m\right)=0\Leftrightarrow \hoac{& \cos x=\dfrac{1}{2} \\& \sin 2x=m}\Leftrightarrow \hoac{& x=\dfrac{\pi }{3}+k2\pi \\& x=-\dfrac{\pi }{3}+k2\pi \\& \sin 2x=m}$\\
			a) Thay $x=\dfrac{7\pi }{3}$ phương trình $(1)$ ta thấy thỏa mãn nên $x=\dfrac{7\pi }{3}$ là một nghiệm của phương trình $(1)$.\\
			b) Khi $m=2$ thì phương trình $(1)\Leftrightarrow \hoac{& x=\dfrac{\pi }{3}+k2\pi \\& x=-\dfrac{\pi }{3}+k2\pi} (k \in \mathbb{Z})$\\
			c) Khi $m=1$ phương trình $(1)\Leftrightarrow \hoac{& x=\dfrac{\pi }{3}+k2\pi \\& x=-\dfrac{\pi }{3}+k2\pi \\& \sin 2x=1}\Leftrightarrow \hoac{& x=\dfrac{\pi }{3}+k2\pi \\& x=-\dfrac{\pi }{3}+k2\pi \\& x=\dfrac{\pi }{4}+l\pi}$.\\
			Do đó tập nghiệm của phương trình $(1)$ có tất cả $4$ điểm biểu diễn trên đường tròn lượng giác.\\
			d) Do phương trình $(2)$ có một nghiệm $x=\dfrac{\pi }{3}$ thuộc $\left(-\dfrac{\pi }{4};\dfrac{3\pi }{4}\right]$.\\
			Do đó để phương trình $(1)$ có đúng hai nghiệm thuộc $\left(-\dfrac{\pi }{4};\dfrac{3\pi }{4}\right]$ thì phương trình $\sin 2x=m$ có 1 nghiệm thuộc $\left(-\dfrac{\pi }{4};\dfrac{3\pi }{4}\right]$ khác $\dfrac{\pi }{3}$ (*)\\
			Ta có $x\in \left(-\dfrac{\pi }{4};\dfrac{3\pi }{4}\right]\Rightarrow 2x\in \left(-\dfrac{\pi }{2};\dfrac{3\pi }{2}\right]$ hay $2x\in \left[0;2\pi\right]$\\
			Từ (*) suy ra $m=1$ hoặc $m=-1$\\
	}
\end{ex}

\TNSA
%Câu 17
\begin{ex}
	Cho góc $\alpha $ thỏa mãn $\sin \alpha =\dfrac{1}{5}$. Khi đó giá trị biểu thức $P={{\cos }^2}2x+{{\cos }^2}x$ bằng $\dfrac{a}{b}$. Tính $a+b$. Biết rằng phân số $\dfrac{a}{b}$ là phân số tối giản
	\shortans{1754}
	\loigiai{
		Biến đổi biểu thức $P$ rồi thay giá trị $\sin \alpha =\dfrac{1}{5}$ vào $P$, ta được:\\
		$\begin{aligned}
				& P={{\cos }^2}2x+{{\cos }^2}x \\& \text{ }={{\left(1-2{{\sin }^2}\alpha\right)}^2}+\left(1-{{\sin }^2}\alpha\right)={{\left(1-2 \cdot {{\left(\dfrac{1}{5}\right)}^2}\right)}^2}+\left(1-{{\left(\dfrac{1}{5}\right)}^2}\right)=\dfrac{1129}{625} \end{aligned}$\\
		$\Rightarrow \heva{& a=1129 \\& b=625}\Rightarrow a+b=1754$
	}
\end{ex}
%Câu 18
\begin{ex}
	Số điểm chung của đồ thị hàm số $y=\sin x$ và $y=\cos x$ trên $\left[ -\dfrac{\pi }{2};\dfrac{3\pi }{2} \right]$ là $n$. Giá trị $\sqrt{n}$ (làm tròn đến hàng phần trăm) bằng
	\shortans{1,41}
	\loigiai{
		Số điểm chung của đồ thị hàm số $y=\sin x$ và $y=\cos x$ trên $\left[ -\dfrac{\pi }{2};\dfrac{3\pi }{2} \right]$ bằng số nghiệm phương trình $\sin x = \cos x$ trên $\left[ -\dfrac{\pi }{2};\dfrac{3\pi }{2} \right]$.\\
		Ta có $\sin x = \cos x \Leftrightarrow \sin x - \cos x =0 \Leftrightarrow \sin \left(x-\dfrac{\pi}{4} \right)=0 \Leftrightarrow x-\dfrac{\pi}{4}=k \pi \Leftrightarrow x= \dfrac{\pi}{4} +k\pi \ (k \in \mathbb{Z})$.\\
		$x \in \left[ -\dfrac{\pi }{2};\dfrac{3\pi }{2} \right]$ nên $x \in \left\{ \dfrac{\pi}{4}; \dfrac{5\pi}{4}\right\}$.\\
		Vậy $n=2$ nên $\sqrt{n} \approx 1,41$.
	}
\end{ex}
%Câu 19
\begin{ex}
	Biết có $n$ giá trị nguyên của tham số $m$ để phương trình $\cos x=m$ có nghiệm. Giá trị $\sqrt{n}$ (làm tròn đến hàng phần trăm) bằng
	\shortans{1,73}
	\loigiai{
		$\cos x=m$ có nghiệm $\Leftrightarrow -1\le m\le 1$. Mà $m\in \mathbb{Z}\Rightarrow m\in \left\{ -1;0;1 \right\}$. Vậy $\sqrt{n}\approx 1{,}73$
	}
\end{ex}
%Câu 20
\begin{ex}
	Biết $x=x_0$ là nghiệm duy nhất của phương trình $2\sin \left(x-\dfrac{\pi }{6}\right)+2=0$ trên khoảng $\left(0;2\pi\right)$. Giá trị $x_0$ (làm tròn đến hàng phần trăm) bằng
	\shortans{5,24}
	\loigiai{
		Ta có: $2\sin \left(x-\dfrac{\pi }{6}\right)+2=0\Leftrightarrow \sin \left(x-\dfrac{\pi }{6}\right)=-1\Leftrightarrow x=-\dfrac{\pi }{3}+k2\pi ,k\in \mathbb{Z}$\\
		Do $x\in \left(0;2\pi\right)$ nên $0<-\dfrac{\pi }{3}+k2\pi <2\pi \Leftrightarrow \dfrac{1}{6}<k<\dfrac{7}{6}\Leftrightarrow k=1$.\\
		Vậy phương trình có một nghiệm $x=\dfrac{5\pi }{3}\approx 5{,}24$
	}
\end{ex}
%Câu 21
\begin{ex}
	Gọi $M$ và $m$ lần lượt là giá trị lớn nhất và giá trị nhỏ nhất của hàm số $y=\sin x+\sqrt{3}\cos x+\sqrt{2}$. Tính $M^2m$ (làm tròn đến hàng phần trăm)
	\shortans{6,83}
	\loigiai{
		Ta có $y=\sin x+\sqrt{3}\cos x+\sqrt{2}=2\left(\dfrac{1}{2}\sin x+\dfrac{\sqrt{3}}{2}\cos x\right)+\sqrt{2}=2\sin \left(x+\dfrac{\pi }{3}\right)+\sqrt{2}$.\\
		Suy ra $M=2+\sqrt{2}$, $m=-2+\sqrt{2}$. Nên $M^2m\approx 6{,}83$
	}
\end{ex}
%Câu 22
\begin{ex}
	Mùa xuân ở Hội Lim (tỉnh Bắc Ninh) thường có trò chơi đu. Khi người chơi đu nhún đều, cây đu sẽ đưa người chơi đu dao động qua lại vị trí cân bằng. Nghiên cứu trò chơi này, người ta thấy khoảng cách $h$ (mét) được tính từ vị trí chân người chơi đu đến vị trí cân bằng được biểu diễn bởi hệ thức $h=|d|$ với $d=3\cos \left[\dfrac{\pi }{3}(2t-1)\right]$ ($t\ge 0$ và được tính bằng giây), trong đó ta quy ước $d>0$ khi vị trí cân bằng ở về phía sau lưng người chơi đu và $d<0$ trong trường hợp ngược lại.
	Biết $t_1$, $t_2$ lần lượt là thời điểm đầu tiên người đu ở vị trí phía sau lưng và vị trí phía trước vị trí cân bằng $1{,}5$ mét. Giá trị $t_1+t_2^2$ (làm tròn đến hàng phần trăm) bằng
	\shortans{3,25}
	\loigiai{
	Người chơi cách vị trí cân bằng 1 mét khi $3\cos \left[\dfrac{\pi }{3}(2t-1)\right]=\pm 1{,}5$\\
	$\Leftrightarrow \cos^2\left[\dfrac{\pi }{3}(2t-1)\right]=\dfrac{1}{4}\Leftrightarrow \cos \left[\dfrac{2\pi }{3}(2t-1)\right]=-\dfrac{1}{2}$ $\Leftrightarrow \hoac{& \dfrac{2\pi }{3}(2t-1)=\dfrac{2\pi }{3}+k2\pi \\& \dfrac{2\pi }{3}(2t-1)=-\dfrac{2\pi }{3}+k2\pi} \left(k\in \mathbb{Z}\right)
		\Leftrightarrow \hoac{& t=1+\dfrac{3k}{2} \\& t=\dfrac{3k}{2}}\left(k\in \mathbb{Z}\right)$.\\
	Vì $t>0$ nên $t_1=1$ và $t_2=1{,}5$. Vậy $t_1+t_2^2=3{,}25$
	}
\end{ex}

\Closesolutionfile{ans}

\indapan{10}{ans/ans-SA-C1-De1}
% \section*{BT ÔN TẬP CHƯƠNG 1}
\setcounter{ex}{0}\setcounter{bt}{0}
\Opensolutionfile{ans}[ans/ans1C3-CD-1]
\noindent\textbf{I. PHẦN TRẮC NGHIỆM:}
\begin{ex}%[1T1B2-3]
        Tính tổng $S=\sin^25^{\circ}+\sin^210^{\circ}+\sin^215^{\circ}+ \cdots +\sin^285^{\circ}$.
        \choice
        {$S=\dfrac{19}{2}$}
        {\True $S=\dfrac{17}{2}$}
        {$S=8$}
        {$S=9$}
        \loigiai{
            \begin{align*}
                S&=\sin^25^{\circ}+\sin^210^{\circ}+\sin^215^{\circ}+ \cdots +\sin^285^{\circ}\\
                &=\left(\sin^25^{\circ}+\sin^285^{\circ}\right)+\left(\sin^210^{\circ}+\sin^280^{\circ}\right)+ \cdots +\left(\sin^240^{\circ}+\sin^250^{\circ}\right)+\sin^245^{\circ} \\
                &=8+\dfrac{1}{2}=\dfrac{17}{2}.
        \end{align*}}
    \end{ex}

\begin{ex}%[1C1Y1-2]
Cho góc lượng giác với tia đầu và tia cuối như trong hình. Tên của góc lượng giác là
    \begin{center}
        \begin{tikzpicture}[scale=1, font=\footnotesize, line join=round, line cap=round, >=stealth]
            \begin{axis}[
                axis line style={draw=none},
                axis lines=middle,
                axis equal image,
                enlargelimits,
                xtick=\empty,
                ytick=\empty,
                data cs=polar,
                samples=200,
                thick,
                line cap=round,
                line join=round,
                >=stealth
                ]
                \addplot [smooth, domain=0:390,->] {1+x/5000};
                \addplot [smooth, domain=420:450,->] {1.5+x/5000} node[above,midway]{$+$};
                \addplot [mark=none] (0.475,0) node [below left] {$O$};
                \addplot [mark=none] coordinates {(0,0) (390,3+390/5000)} node[below]{$y$};
                \addplot [mark=none] coordinates {(0,0) (0,3+0/5000)} node[below]{$x$};
                \addplot [mark=none,dashed] coordinates {(0,0) (420,3+420/5000)} node[right]{$m$};
            \end{axis}
        \end{tikzpicture}
    \end{center}
\choice
{\True $(Ox,Oy)$}
{$(Oy,Ox)$}
{$(Om,Oy)$}
{$(Om,Ox)$}
    \loigiai{
        Trong hình, góc lượng giác là $(Ox,Oy)$ với tia đầu $Ox$ và tia cuối $Oy$.
    }
    \end{ex}

\begin{ex}%[1T1B2-2]
        Cho $\tan a=\dfrac{2}{3}$, $5\pi <a<\dfrac{11\pi}{2}$. Khi đó $\cos \left(a+\dfrac{\pi}{3}\right)$ bằng
        \choice
        {$\dfrac{2\sqrt{3}+3}{2\sqrt{13}}$}
        {\True $\dfrac{2\sqrt{3}-3}{2\sqrt{13}}$}
        {$\dfrac{-2\sqrt{3}+3}{2\sqrt{13}}$}
        {$\dfrac{-2\sqrt{3}-3}{2\sqrt{13}}$}
        \loigiai{
            Ta có $\cos^2a=\dfrac{1}{1+\tan^2 a}=\dfrac{1}{1+\dfrac{4}{9}}=\dfrac{9}{13}$.\\
            Vì $5\pi <a<\dfrac{11\pi}{2}$ nên $\cos a<0$ và $\sin a<0$.\\
             Do đó, $\cos a=-\dfrac{3\sqrt{13}}{13}$ và $\sin a=-\dfrac{2\sqrt{13}}{13}$.\\
            Vậy $\cos \left(a+\dfrac{\pi}{3}\right)=\dfrac{1}{2}\cos a-\dfrac{\sqrt{3}}{2}\sin a=\dfrac{-3+2\sqrt{3}}{2\sqrt{13}}$.}
    \end{ex}

\begin{ex}%[Tex hóa SGK CD-CT,T12/22, TVN-006]%[1K1Y1-8]
    Trong các khẳng định sau, khẳng định  nào là \textbf{sai}?
    \choice
    {$\sin(\pi-\alpha)=\sin\alpha$}
    {\True $\cos(\pi-\alpha)=\cos \alpha$}
    {$\sin(\pi+\alpha)=-\sin\alpha$}
    {$\cos(\pi+\alpha)=-\cos \alpha$}
    \loigiai{
        Ta có $\cos(\pi-\alpha)=-\cos \alpha$ nên $\cos(\pi-\alpha)=\cos \alpha$ là khẳng định \textbf{sai}.
    }
\end{ex}

\begin{ex}%[1C1B1-2]
    Cho góc lượng giác gốc $O$ có tia đầu $Ou$, tia cuối $Ov$ và có số đo $\dfrac{2\pi}{3}$. Cho góc lượng giác $(O'u',O'v')$ có tia đầu $O'u'\equiv Ou$, tia cuối $O'v'\equiv Ov$. Viết công thức biểu thị số đo góc lượng giác $(O'u',O'v')$.
\choice
{$(O'u',Ov')=\dfrac{\pi}{3}+k2\pi\ (k\in \mathbb{Z})$}
{$(O'u',Ov')=\dfrac{4\pi}{3}+k2\pi\ (k\in \mathbb{Z})$}
{\True $(O'u',Ov')=\dfrac{2\pi}{3}+k2\pi\ (k\in \mathbb{Z})$}
{$(O'u',Ov')=-\dfrac{\pi}{3}+k2\pi\ (k\in \mathbb{Z})$}
    \loigiai{
        Ta có $(O'u',Ov')=(Ou,Ov)+k2\pi=\dfrac{2\pi}{3}+k2\pi\ (k\in \mathbb{Z})$.
    }
\end{ex}

\begin{ex}%[Tex hóa SGK CD-CT,T12/22, TVN-006]%[1K1B2-3]
    Rút gọn biểu thức $M=\cos(a+b)\cos(a-b)-\sin (a+b)\sin(a-b)$, ta được
    \choice
    {$M=\sin 4a$}
    {$M=1-2\cos^2a$}
    {\True $M=1-2\sin^2a$}
    {$M=\cos 4a$}
    \loigiai{
        Ta có
        \allowdisplaybreaks
        \begin{eqnarray*}
            M&=&\cos(a+b)\cos(a-b)-\sin (a+b)\sin(a-b)\\
            &=&\dfrac{1}{2}\left(\cos2a+\cos 2b\right)+\dfrac{1}{2}\left(\cos2a-\cos 2b\right)\\
            &=&\cos 2a\\
            &=&1-2\sin^2a.
        \end{eqnarray*}
    }
\end{ex}

\begin{ex}%[1T1B5-3]
        Tập nghiệm của phương trình $3\cos\left(3x-\dfrac{\pi}{3}\right)=0$ là
        \choice
        {$\left\{\dfrac{\pi}{2}+k\pi, k \in \mathbb{Z}\right\}$}
        {$\left\{\dfrac{5\pi}{6}+k 2\pi, k \in \mathbb{Z}\right\}$}
        {$\left\{\dfrac{5\pi}{18}+\dfrac{k 2\pi}{3}, k \in \mathbb{Z}\right\}$}
        {\True $\left\{\dfrac{5\pi}{18}+\dfrac{k\pi}{3}, k \in \mathbb{Z}\right\}$}
        \loigiai{
            $3\cos\left(3x-\dfrac{\pi}{3}\right)=0\Leftrightarrow 3x-\dfrac{\pi}{3}=\dfrac{\pi}{2}+k\pi\Leftrightarrow x=\dfrac{5\pi}{18}+\dfrac{k\pi}{3}, k\in\mathbb{Z}$.
            Tập nghiệm phương trình $S=\left\{\dfrac{5\pi}{18}+\dfrac{k\pi}{3}, k\in\mathbb{Z}\right\}$.
        }
    \end{ex}

\begin{ex}%[1T1B6-3]
        Phương trình $\sqrt{3}\sin x+\cos x=1$ tương đương với phương trình nào sau đây?
        \choice
        {$\cos \left( x+\dfrac{\pi}{6}\right) =\dfrac{1}{2}$}
        {$\sin \left( x+\dfrac{\pi}{3}\right) =\dfrac{1}{2}$}
        {\True $\cos \left( x-\dfrac{\pi}{3}\right) =\dfrac{1}{2}$}
        {$\sin \left( x-\dfrac{\pi}{6}\right) =\dfrac{1}{2}$}
        \loigiai{
            Chia hai vế của phương trình cho $2$, ta được
            \begin{eqnarray*}
                &\sqrt{3}\sin x+\cos x=1&\Leftrightarrow\dfrac{\sqrt{3}}{2}\sin x+\dfrac{1}{2}\cos x=\dfrac{1}{2}\\
                &&\Leftrightarrow\sin\dfrac{\pi}{3}\sin x+\cos\dfrac{\pi}{3}\cos x=\dfrac{1}{2}\\
                &&\Leftrightarrow\cos\left( x-\dfrac{\pi}{3}\right) =\dfrac{1}{2}.
            \end{eqnarray*}
        }
    \end{ex}

\begin{ex}%[1T1Y4-1]
        Tìm điều kiện xác định của hàm số  $y=\cot x$.
        \choice
        {$x \neq \dfrac{\pi}{4}+k \pi, k \in \mathbb{Z}$}
        { $x \neq k 2 \pi, k \in \mathbb{Z}$}
        {\True $x \neq k \pi, k \in \mathbb{Z}$}
        {$x\neq \dfrac{\pi}{2}+k \pi, k \in \mathbb{Z}$}
        \loigiai{
            Hàm số $y=\cot x$ xác định khi và chỉ khi $\sin x \ne 0 \Leftrightarrow x\neq k \pi, k \in \mathbb{Z} .$}
    \end{ex}

\begin{ex}%[1T1B4-2]
        Hàm số nào sau đây đồng biến trên khoảng $(0;\pi)$?
        \choice
        {\True $y=x^2$}
        {$y=\cos x$}
        {$y=\sin x$}
        {$y=\tan x$}
        \loigiai{
            Hàm số $y = x^2$ đồng biến khi $x > 0 \Rightarrow$ hàm số đồng biên trên khoảng $\left(0;\pi\right)$.}
    \end{ex}

\begin{ex}%[1C1B1-2]
    Cho góc lượng giác gốc $O$ có tia đầu $Ou$, tia cuối $Ov$ và có số đo $-\dfrac{5\pi}{6}$. Cho góc lượng giác $(O'u',O'v')$ có tia đầu $O'u'\equiv Ou$, tia cuối $O'v'\equiv Ov$. Viết công thức biểu thị số đo góc lượng giác $(O'u',O'v')$.
    \choice
    {$(O'u',Ov')=\dfrac{\pi}{6}+k2\pi\ (k\in \mathbb{Z})$}
    {$(O'u',Ov')=\dfrac{4\pi}{3}+k2\pi\ (k\in \mathbb{Z})$}
    {$(O'u',Ov')=-\dfrac{\pi}{6}+k2\pi\ (k\in \mathbb{Z})$}
    {\True $(O'u',Ov')=-\dfrac{5\pi}{6}+k2\pi\ (k\in \mathbb{Z})$}
    \loigiai{
        Ta có $(O'u',Ov')=(Ou,Ov)+k2\pi=-\dfrac{5\pi}{6}+k2\pi\ (k\in \mathbb{Z})$.
    }
\end{ex}

\begin{ex}%[1T1B4-6]
        Hình bên dưới là đồ thị của hàm số nào dưới đây?
        \begin{center}
            \begin{tikzpicture}[>=stealth,line join=round,line cap=round,font=\footnotesize,scale=0.7]
                \def\a{3.141592654}
                \draw[color=gray,dash pattern=on 1pt off 1pt,xstep=3.14cm,ystep=1.0cm] (-9.424,-3) grid (9.424,3);
                \draw[->] (-9.424,0) -- (9.7,0)node[below]{\scriptsize $x$};
                \draw[->] (0,-3.5) -- (0,3.5) node[left] {\scriptsize $y$};
                \draw (0,0)node[below left]{\scriptsize $O$};
                \clip (-9.58,-3.5)rectangle(9.58,3.5);
                \draw[samples=300,smooth,domain=-9.424:9.424] plot(\x,{-3*cos(\x*180/pi)});
                \path(-2*\a,0)node[shift={(-150:12pt)}]{$-2\pi$}
                (-\a,0)node[shift={(-150:12pt)}]{$-\pi$}
                (\a,0)node[shift={(-135:10pt)}]{$\pi$}
                (2*\a,0)node[shift={(-140:10pt)}]{$2\pi$}
                (0,3)node[shift={(-140:10pt)}]{$3$}
                (0,-3)node[shift={(-140:10pt)}]{$-3$};
                \foreach \x/\y in{-2*\a/0,-\a/0,\a/0,2*\a/0,-3*\a/3,3*\a/3,-2*\a/-3,2*\a/-3,\a/3,-\a/3,0/3,0/-3,0/0}\draw(\x,\y) circle (1pt);
            \end{tikzpicture}
        \end{center}
        \choice
        {\True $y=-3\cos x$}
        {$y=-2-\cos x$}
        {$y=2+|\cos x|$}
        {$y=\cos x-4$}
        \loigiai{
            \begin{itemize}
                \item $y(0)=-3\Rightarrow $ loại $y=\cos x-4$ và $y=2+|\cos x|$.
                \item $y(\pi)=3\Rightarrow $ loại $y=-2-\cos x$.
            \end{itemize}
        }
    \end{ex}

\begin{ex}%[1T1Y4-1]
        Điều kiện xác định của hàm số $y=\cot x$ là
        \choice
        { $x \ne \dfrac{\pi}{8}+k\dfrac{\pi}{2}$}
        {$x \ne \dfrac{\pi}{2}+k\pi$}
        {\True $x \ne k\pi$}
        {$x \ne \dfrac{\pi}{4}+k\pi$}
        \loigiai{
            Hàm số xác định khi và chỉ khi $\sin x \ne 0 \Leftrightarrow x \ne k\pi$, $k \in \mathbb{Z}$.
        }
    \end{ex}

\begin{ex}%[1T1B4-5]
        Cho hàm số $y=\sin^2x-\sin x+2$. Gọi $M,N$ lần lượt là GTLN và GTNN của hàm số đã cho. Khi đó $M+N$ bằng
        \choice
        {$k=-\dfrac{1}{2}$}
        {\True $\dfrac{23}{4}$}
        {$\dfrac{15}{4}$}
        {$6$}
        \loigiai{
            Ta có $y=\sin^2x-\sin x+2=\left(\sin x-\dfrac{1}{2}\right)^2+\dfrac{7}{4}$. \\
            Vì $-1 \leq \sin x \leq 1,\,\forall x \in \mathbb{R}$ nên $-\dfrac{3}{2} \leq \sin x-\dfrac{1}{2} \leq \dfrac{1}{2},\,\forall x \in \mathbb{R}$.\\
            Suy ra $0 \leq \left(\sin x-\dfrac{1}{2}\right)^2 \leq \dfrac{9}{4},\,\forall x \in \mathbb{R}$.\\
            Suy ra $\dfrac{7}{4} \leq \left(\sin x-\dfrac{1}{2}\right)^2+\dfrac{7}{4} \leq 4,\,\forall x \in \mathbb{R}$.\\
            Suy ra $\dfrac{7}{4} \leq y \leq 4,\,\forall x \in \mathbb{R}$.\\
            Vậy $M+N=\dfrac{7}{4}+4=\dfrac{23}{4}$.}
    \end{ex}

\begin{ex}%[Tex hóa SGK CD-CT,T12/22, TVN-006]%[1K1Y3-4]
    Trong các hàm số sau đây, hàm số nào là hàm tuần hoàn?
    \choice
    {$y=\tan x+x$}
    {$y=x^2+1$}
    {\True $y=\cot x$}
    {$y=\dfrac{\sin x}{x}$}
    \loigiai{
        Hàm số $y=\cot x$ là hàm số tuần hoàn với chu kỳ $T=\pi$.
    }
\end{ex}

\begin{ex}%[1T1Y1-1]
        Góc $18^\circ$ có số đo bằng rađian là bao nhiêu?
        \choice
        {$\pi$}
        {$\dfrac{\pi}{360}$}
        {\True $\dfrac{\pi}{10}$}
        {$\dfrac{\pi}{18}$}
        \loigiai{
            Ta có $18^\circ=\dfrac{\pi}{10}$ rad.
        }
    \end{ex}

\begin{ex}%[Tex hóa SGK CD-CT,T12/22, TVN-006]%[1K1Y1-4]
    Biểu diễn các góc lượng giác $\alpha=-\dfrac{5\pi}{6}$, $\beta=\dfrac{\pi}{3}$, $\gamma=\dfrac{25\pi}{3}$, $\delta=\dfrac{17\pi}{6}$ trên đường tròn lượng giác. Các góc nào có điểm biểu diễn trùng nhau?
    \choice
    {\True $\beta$ và $\gamma$}
    {$\alpha$, $\beta$, $\gamma$}
    {$\beta$, $\gamma$, $\delta$}
    {$\alpha$ và $\beta$}
    \loigiai{
        Ta có $\beta+8\pi=\dfrac{\pi}{3}+8\pi=\dfrac{25\pi}{3}=\gamma$.\\
        Do đó, $\beta$ và $\gamma$ có điểm biểu diễn trùng nhau trên đường tròn lượng giác.
    }
\end{ex}

\begin{ex}%[1C1B1-2]
    Cho góc lượng giác $(Ou,Ov)$ có số đo là $\dfrac{3\pi}{4}$, góc lượng giác $(Ou,Ow)$ có số đo là $\dfrac{5\pi}{4}$. Số đo của góc lượng giác $(Ov,Ow)$ là
\choice
{\True $(Ov,Ow)=\dfrac{\pi}{2}+k2\pi\ (k\in \mathbb{Z})$}
{$(Ov,Ow)=2\pi+k2\pi\ (k\in \mathbb{Z})$}
{$(Ov,Ow)=-\dfrac{\pi}{2}+k2\pi\ (k\in \mathbb{Z})$}
{$(Ov,Ow)=-\dfrac{\pi}{6}+k2\pi\ (k\in \mathbb{Z})$}
    \loigiai{
        Theo hệ thức Chasles, ta có
        \begin{eqnarray*}
            (Ov,Ow)&=&(Ou,Ow)-(Ou,Ov)+k2\pi\\
            &=&\dfrac{5\pi}{4}-\dfrac{3\pi}{4}+k2\pi\\
            &=&\dfrac{\pi}{2}+k2\pi\ (k\in \mathbb{Z}).
        \end{eqnarray*}
    }
\end{ex}

\begin{ex}%[1C1B1-2]
    Cho góc lượng giác gốc $O$ có tia đầu $Ou$, tia cuối $Ov$ và có số đo $45^\circ$. Cho góc lượng giác $(O'u',O'v')$ có tia đầu $O'u'\equiv Ou$, tia cuối $O'v'\equiv Ov$. Công thức biểu thị số đo góc lượng giác $(O'u',O'v')$ là
\choice
{$(O'u',Ov')=-45^\circ+k360^\circ\ (k\in \mathbb{Z})$}
{\True $(O'u',Ov')=45^\circ+k360^\circ\ (k\in \mathbb{Z})$}
{$(O'u',Ov')=135^\circ+k360^\circ\ (k\in \mathbb{Z})$}
{$(O'u',Ov')=-135^\circ+k360^\circ\ (k\in \mathbb{Z})$}
    \loigiai{
        Ta có $(O'u',Ov')=(Ou,Ov)+k360^\circ=45^\circ+k360^\circ\ (k\in \mathbb{Z})$.
    }
\end{ex}

\begin{ex}%[1T1B4-5]
        Hàm số $y=3-5\sin x$ có giá trị lớn nhất bằng
        \choice
        {$6$}
        {$2$}
        {\True $8$}
        {$4$}
        \loigiai{
            Ta có
            $$-1\le \sin x\le 1 \Leftrightarrow 5\ge-5\sin x\ge-5\Leftrightarrow  8\ge 3-5\sin x\ge -2\Rightarrow -2\le y\le 8.$$
            Suy ra giá trị lớn nhất của hàm số là $8$, đạt được khi $x=\dfrac{\pi}{2}+k2\pi,k\in\mathbb{Z}$.
        }
    \end{ex}

\begin{ex}%[1T1B2-4]
        Rút gọn biểu thức $M=\sin(\pi-a)+\tan\left(\dfrac{\pi}{2}-a\right)+\sin(-a)+\cot(\pi+a)$ được
        \choice
        {$M=2\cos a$}
        {$M=2\tan a$}
        {\True $M=2\cot a$}
        {$M=0$}
        \loigiai{
            Ta có $M=\sin a+\cot a-\sin a+\cot a=2\cot a$.
        }
    \end{ex}

\begin{ex}%[1T1Y4-6]
        Đồ thị hàm số $y=\cos x$ đi qua điểm nào sau đây?
        \choice
        {$P(-1;\pi)$}
        {$M(\pi;1)$}
        {$Q(3\pi; 1)$}
        {\True $N(0;1)$}
        \loigiai{
            Điểm $N(0;1)$ thuộc đồ thị hàm số.
        }
    \end{ex}

\begin{ex}%[1T1B4-1]
        Tập xác định của hàm số $y=2017\tan^{2018} \left( 2x+\dfrac{\pi}{3}\right)$ là
        \choice
        {\True $\mathscr{D}=\mathbb{R}\setminus\left\lbrace\dfrac{\pi}{12}+k\dfrac{\pi}{2}, k\in\mathbb{Z} \right\rbrace $}
        {$\mathscr{D}=\mathbb{R}\setminus\left\lbrace\dfrac{\pi}{2}+k\dfrac{\pi}{2}, k\in\mathbb{Z} \right\rbrace $}
        {$\mathscr{D}=\mathbb{R}\setminus\left\lbrace\dfrac{\pi}{2}+k\dfrac{\pi}{2}, k\in\mathbb{Z} \right\rbrace $}
        {$\mathscr{D}=\mathbb{R}\setminus\left\lbrace\dfrac{\pi}{2}+k\dfrac{\pi}{2}, k\in\mathbb{Z} \right\rbrace $}
        \loigiai{
            Hàm số xác định khi $2x+\dfrac{\pi}{3}\ne \dfrac{\pi}{2}+k\pi\Leftrightarrow x\ne\dfrac{\pi}{12}+k\dfrac{\pi}{2}, k\in\mathbb{Z}.$
        }
        \end{ex}

\begin{ex}%[1K1Y1-7]
        Tìm khẳng định đúng (với điều kiện các hệ thức đã xác định).
        \choice
        {$\cos \left(\pi -\alpha \right)=\cos \alpha$}
        {\True $\cos \left(-\alpha \right)=\cos \alpha$}
        {$\sin \left(\pi -\alpha \right)=-\sin \alpha$}
        {$\sin \left(-\alpha \right)=\sin \alpha$}
        \loigiai{
            Ta có
            \begin{itemize}
                \item $\sin \left(-\alpha \right)=-\sin \alpha$.
                \item $\cos \left(\pi -\alpha \right)=-\cos \alpha$.
                \item $\cos \left(-\alpha \right)=\cos \alpha$.
                \item $\sin \left(\pi -\alpha \right)=\sin \alpha$.
            \end{itemize}
        }
    \end{ex}


\noindent\textbf{II. PHẦN TỰ LUẬN:}
\begin{ex}%[Cánh Diều]%[1C1B4-3]
    Giải các phương trình
    \begin{multicols}{3}
        \begin{enumerate}[a)]
            \item $\sin x=-\dfrac{1}{2}$;
            \item $\sin x=\dfrac{\sqrt{2}}{2}$;
            \item $\sin3x=\sin2x$;
            \item $\sin x=\cos3x$;
            \item $\cos x=\dfrac{\sqrt{3}}{2}$;
            \item $\cos x=-\dfrac{\sqrt{2}}{2}$;
            \item $\cos x=-\dfrac{1}{2}$;
            \item $\cos3x=\cos\left(x+\dfrac{\pi}{3}\right)$;
            \item $\tan x=\dfrac{1}{\sqrt{3}}$;
            \item $\tan x=-1$;
            \item $\cot2x=-\sqrt{3}$.
        \end{enumerate}
    \end{multicols}
\loigiai{
\begin{enumerate}[a)]
    \item Do $\sin\left(-\dfrac{\pi}{6}\right)=-\dfrac{1}{2}$ nên $$\sin x=\sin\left(-\dfrac{\pi}{6}\right)\Leftrightarrow\hoac{&x=-\dfrac{\pi}{6}+k2\pi\\&x=\pi-\left(-\dfrac{\pi}{6}\right)+k2\pi}\Leftrightarrow\hoac{&x=-\dfrac{\pi}{6}+k2\pi\\&x=\dfrac{7\pi}{6}+k2\pi}\,(k\in\mathbb{Z}).$$
    \item Do $\sin\dfrac{\pi}{4}=\dfrac{\sqrt{2}}{2}$ nên $$\sin x=\sin\dfrac{\pi}{4}\Leftrightarrow\hoac{&x=\dfrac{\pi}{4}+k2\pi\\&x=\pi-\dfrac{\pi}{4}+k2\pi}\Leftrightarrow\hoac{&x=\dfrac{\pi}{4}+k2\pi\\&x=\dfrac{3\pi}{4}+k2\pi}\,(k\in\mathbb{Z}).$$
    \item $\sin3x=\sin2x\Leftrightarrow\hoac{&3x=2x+k2\pi\\&3x=\pi-2x+k2\pi}\Leftrightarrow\hoac{&x=k2\pi\\&x=\dfrac{\pi}{5}+k\dfrac{2\pi}{5}}\,(k\in\mathbb{Z})$.
    \item $\sin x=\cos3x\Leftrightarrow\sin x=\sin\left(\dfrac{\pi}{2}-3x\right)\Leftrightarrow\hoac{&x=\dfrac{\pi}{2}-3x+k2\pi\\&x=\pi-\left(\dfrac{\pi}{2}-3x\right)+k2\pi}\Leftrightarrow\hoac{&x=\dfrac{\pi}{8}+k\dfrac{\pi}{2}\\&x=-\dfrac{\pi}{4}+k\pi}\,(k\in\mathbb{Z})$.
    \item $\cos x=\dfrac{\sqrt{3}}{2}\Leftrightarrow\cos x=\cos\dfrac{\pi}{6}\Leftrightarrow x=\pm\dfrac{\pi}{6}+k2\pi,\,(k\in\mathbb{Z})$;
    \item $\cos x=-\dfrac{\sqrt{2}}{2}\Leftrightarrow \cos x=\cos\dfrac{3\pi}{4}\Leftrightarrow x=\pm\dfrac{3\pi}{4}+k2\pi,\,(k\in\mathbb{Z})$;
    \item $\cos x=-\dfrac{1}{2}\Leftrightarrow\cos x=\cos\dfrac{2\pi}{3}\Leftrightarrow x=\pm\dfrac{2\pi}{3}+k2\pi,\,(k\in\mathbb{Z})$;
    \item $\cos3x=\cos\left(x+\dfrac{\pi}{3}\right)\Leftrightarrow\hoac{&3x=x+\dfrac{\pi}{3}+k2\pi\\&3x=-x-\dfrac{\pi}{3}+k2\pi}\Leftrightarrow\hoac{&x=\dfrac{\pi}{6}+k\pi\\&x=-\dfrac{\pi}{12}+\dfrac{k\pi}{2}},\,(k\in\mathbb{Z})$;
    \item $\tan x=\dfrac{1}{\sqrt{3}}\Leftrightarrow\tan x=\tan\dfrac{\pi}{6}\Leftrightarrow x=\dfrac{\pi}{6}+k\pi,\,(k\in\mathbb{Z})$;
    \item $\tan x=-1\Leftrightarrow\tan x=\tan\left(-\dfrac{\pi}{4}\right)\Leftrightarrow x=-\dfrac{\pi}{4}+k\pi,\,(k\in\mathbb{Z})$;
    \item $\cot2x=-\sqrt{3}\Leftrightarrow\cot2x=\cot\left(-\dfrac{\pi}{6}\right)\Leftrightarrow x=-\dfrac{\pi}{6}+k\pi,\,(k\in\mathbb{Z})$.
\end{enumerate}}
\end{ex}

\begin{ex}%[1C1B4-3]
    Giải phương trình:
    \begin{listEX}[3]
        \item $\sin \left(2x-\dfrac{\pi}{3}\right)=-\dfrac{\sqrt{3}}{2}$;
        \item $\sin \left(3x+\dfrac{\pi}{4}\right)=-\dfrac{1}{2}$;
        \item $\cos \left(\dfrac{x}{2}+\dfrac{\pi}{4}\right) =\dfrac{\sqrt{3}}{2}$;
        \item $2\cos 3x+5=3$;
        \item $3\tan x=-\sqrt{3}$;
        \item $\cot x-3=\sqrt{3}\left(1-\cot x\right)$.
    \end{listEX}
    \loigiai{
        \begin{enumerate}[a)]
            \item Ta có 
            \begin{eqnarray*}
                &&\sin \left(2x-\dfrac{\pi}{3}\right)=-\dfrac{\sqrt{3}}{2}\\
                &\Leftrightarrow& \sin \left(2x-\dfrac{\pi}{3}\right) =\sin \left(-\dfrac{\pi}{3}\right)\\
                &\Leftrightarrow& \hoac{
                    &2x-\dfrac{\pi}{3} =-\dfrac{\pi}{3}+k2\pi\\
                    &2x-\dfrac{\pi}{3} = \pi+\dfrac{\pi}{3}+k2\pi}\\
                &\Leftrightarrow&
                \hoac{&2x=k2\pi\\&2x=\dfrac{5\pi}{3} +k2\pi}\\
                &\Leftrightarrow&
                \hoac{&x=k\pi\\ &x=\dfrac{5\pi}{6}+k\pi} (k\in \mathbb{Z}).
            \end{eqnarray*}         
            \item Ta có 
            \begin{eqnarray*}
                &&\sin \left(3x+\dfrac{\pi}{4}\right)=-\dfrac{1}{2} \\
                &\Leftrightarrow& \sin \left(3x+\dfrac{\pi}{4}\right) =\sin \left(-\dfrac{\pi}{6}\right)\\  
                &\Leftrightarrow& \hoac{
                    &3x+\dfrac{\pi}{4} = -\dfrac{\pi}{6}+k2\pi\\
                    &3x+\dfrac{\pi}{4}=\pi -\left(-\dfrac{\pi}{6}\right)+k2\pi} \\
                &\Leftrightarrow&
                \hoac{
                    &3x= -\dfrac{5\pi}{12}+k2\pi\\
                    &3x=\dfrac{11\pi}{12}+k2\pi} \\
                &\Leftrightarrow&
                \hoac{&x=-\dfrac{5}{36}+\dfrac{k2\pi}{3}\\
                    &x=\dfrac{11\pi}{36}+\dfrac{k2\pi}{3}} (k\in \mathbb{Z}).
            \end{eqnarray*}                     
            \item Ta có 
            \begin{eqnarray*}
                &&\cos \left(\dfrac{x}{2}+\dfrac{\pi}{4}\right) =\dfrac{\sqrt{3}}{2} \\ &\Leftrightarrow& \cos \left(\dfrac{x}{2}+\dfrac{\pi}{4}\right)=\cos \dfrac{\pi}{6}\\
                &\Leftrightarrow& \hoac{
                    &\dfrac{x}{2}+\dfrac{\pi}{4} = \dfrac{\pi}{6}+k2\pi\\
                    &\dfrac{x}{2}+\dfrac{\pi}{4}= -\dfrac{\pi}{6}+k2\pi}\\
                &\Leftrightarrow&
                \hoac{
                    &\dfrac{x}{2}=-\dfrac{\pi}{12}+k2\pi\\
                    &\dfrac{x}{2}=-\dfrac{5\pi}{12}+k2\pi}\\
                &\Leftrightarrow&
                \hoac{
                    &x=-\dfrac{\pi}{6}+k4\pi\\
                    &x=-\dfrac{5\pi}{6}+k4\pi} (k\in \mathbb{Z}).
            \end{eqnarray*} 
            \item Ta có $2\cos 3x+5=3 \Leftrightarrow \cos 3x =-1 \Leftrightarrow 3x=\pi+k2\pi \Leftrightarrow x = \dfrac{\pi}{3}+\dfrac{k2\pi}{3}\,\,(k\in \mathbb{Z})$.
            \item Ta có $3\tan x=-\sqrt{3} \Leftrightarrow \tan x =-\dfrac{\sqrt{3}}{3} \Leftrightarrow 
            \tan x=\tan \left(-\dfrac{\pi}{6}\right) \Leftrightarrow x=-\dfrac{\pi}{6}+k\pi\,\, (k\in \mathbb{Z}).$
            \item Ta có 
            \begin{eqnarray*}
                &&\cot x-3=\sqrt{3}\left(1-\cot x\right)\\
                &\Leftrightarrow& \cot x-3 =\sqrt{3}-\sqrt{3}\cot x\\
                &\Leftrightarrow& (1+\sqrt{3})\cot x=\sqrt{3}(1+\sqrt{3})\\
                &\Leftrightarrow& \cot x=\sqrt{3}\\
                &\Leftrightarrow& \cot x=\cot \dfrac{\pi}{6}\\
                &\Leftrightarrow& x=\dfrac{\pi}{6}+k\pi\,\, (k\in \mathbb{Z}).
            \end{eqnarray*} 
        \end{enumerate}
    }
\end{ex}

\begin{ex}%[1C1B4-3]
    Giải phương trình:
    \begin{listEX}[3]
        \item $\sin \left(2x+\dfrac{\pi}{4}\right)=\sin x$;
        \item $\sin 2x=\cos 3x$;
        \item $\cos^2 2x =\cos^2 \left(x+\dfrac{\pi}{6}\right)$.
    \end{listEX}
    \loigiai{
        \begin{enumerate}[a)]
            \item Ta có 
            \[\sin \left(2x+\dfrac{\pi}{4}\right)=\sin x
            \Leftrightarrow 
            \hoac{&2x+\dfrac{\pi}{4}=x+k2\pi\\&2x+\dfrac{\pi}{4}=\pi-x+k2\pi} 
            \Leftrightarrow \hoac{&x=-\dfrac{\pi}{4}+k2\pi\\&3x=-\dfrac{\pi}{4}+k2\pi} \Leftrightarrow \hoac{&x=-\dfrac{\pi}{4}+k2\pi\\&x=-\dfrac{\pi}{12}+\dfrac{k2\pi}{3}
            },\, (k\in \mathbb{Z}).\]
            \item Ta có 
            \begin{eqnarray*}
                \sin 2x=\cos 3x &\Leftrightarrow& \cos 3x =\cos \left(\dfrac{\pi}{2}-2x\right)\\
                &\Leftrightarrow& \hoac{
                    &3x=\dfrac{\pi}{2}-2x+k2\pi\\
                    &3x=\pi-\left(\dfrac{\pi}{2}-2x\right) +k2\pi}\\
                &\Leftrightarrow&
                \hoac{&5x=\dfrac{\pi}{2}+k2\pi\\&x=\dfrac{\pi}{2}+k2\pi}\\
                &\Leftrightarrow&
                \hoac{&x=\dfrac{\pi}{12}+\dfrac{k2\pi}{5}\\
                    &x=\dfrac{\pi}{2}+k2\pi}\, (k\in \mathbb{Z}).
            \end{eqnarray*} 
            \item Ta có $\cos^2 2x =\cos^2 \left(x+\dfrac{\pi}{6}\right) \Leftrightarrow 
            \hoac{
                &\cos 2x=\cos \left(x+\dfrac{\pi}{6}\right) &(1)\\
                &\cos 2x=-\cos \left(x+\dfrac{\pi}{6}\right). &(2)
            }$\\
            +) $(1) \Leftrightarrow \hoac{
                &2x=x+\dfrac{\pi}{6}+k2\pi\\
                &2x=-\left(x+\dfrac{\pi}{6}\right)+k2\pi
            } \Leftrightarrow
            \hoac{
                &x=\dfrac{\pi}{6}+k2\pi\\
                &3x=-\dfrac{\pi}{6}+k2\pi
            }
            \Leftrightarrow
            \hoac{
                &x=\dfrac{\pi}{6}+k2\pi\\
                &x=-\dfrac{\pi}{18}+\dfrac{k2\pi}{3}
            }(k\in \mathbb{Z})$.\\
            +) $(2) \Leftrightarrow 
            \cos 2x=\cos\left[\pi- \left(x+\dfrac{\pi}{6}\right) \right]
            \Leftrightarrow
            \hoac{
                &2x=\pi- \left(x+\dfrac{\pi}{6}\right)+k2\pi\\
                &2x=-\left[\pi- \left(x+\dfrac{\pi}{6}\right)\right]+k2\pi
            } $
            \[\Leftrightarrow
            \hoac{
                &3x=\dfrac{5\pi}{6}+k2\pi\\
                &x=-\dfrac{5\pi}{6}+k2\pi
            } \Leftrightarrow
            \hoac{
                &x=\dfrac{5\pi}{18}+\dfrac{k2\pi}{3}\\
                &x=-\dfrac{5\pi}{6}+k2\pi
            } \,(k\in\mathbb{Z}).\]
        \end{enumerate}
    }
\end{ex}

\begin{ex}Giải các phương trình sau
    \begin{listEX}[2]
        \item $2\sin x+\sqrt{2}=0$;
        \item $\sin2x-\cos x+2\sin x=1$;
        \item $3\sin ^2 x-5\sin x+2=0$;
        \item $\sqrt{3}\tan^2 x-2\tan x+\sqrt{3}=0$;
        \item $2\cos^2 2x-5\cos 2x+2=0$;
        \item $\sin^2\dfrac{x}{2}+\sin\dfrac{x}{2}-2=0$.
    \end{listEX}
    \loigiai{
        \begin{enumerate}[a)]
            \item $2\sin x+\sqrt{2}=0\Leftrightarrow\sin x=-\dfrac{\sqrt{2}}{2}\Leftrightarrow\sin x=\sin\left(-\dfrac{\pi}{4}\right)\Leftrightarrow\hoac{&x=-\dfrac{\pi}{4}+k2\pi\\&x=\dfrac{5\pi}{4}+k2\pi}\,(k\in\mathbb{Z})$;
            \item $\sin2x-\cos x+2\sin x=1\Leftrightarrow2\sin x\cos x-\cos x+2\sin x-1=0\Leftrightarrow(2\sin x-1)(\cos x+1)=0\\\Leftrightarrow\hoac{&\sin x=\dfrac{1}{2}\\&\cos x=-1}\Leftrightarrow\hoac{&\sin x=\sin\dfrac{\pi}{6}\\&x=(2k+1)\pi}\Leftrightarrow\hoac{&x=\dfrac{\pi}{6}+k2\pi\\&x=\dfrac{5\pi}{6}+k2\pi\\&x=(2k+1)\pi}\,(k\in\mathbb{Z})$;
            \item Đặt $t=\cos x$, $-1\le t\le 1$, phương trình đã cho trở thành $3t^2-5t+2=0$, ta được $t=1$ hoặc $t=\dfrac{2}{3}$.\\
            Với $t=1$ ta có $\cos x=1 \Leftrightarrow x= k2\pi,\, (k\in\mathbb{Z})$.\\
            Với $t=\dfrac{2}{3}$ ta có $\cos x=\dfrac{2}{3}=\cos\alpha\Leftrightarrow x=\pm \alpha +k2\pi,\, (k\in\mathbb{Z})$.\\
            Vậy tập nghiệm của phương trình đã cho là $S=\left\{ k2\pi, \pm \alpha+k2\pi, k\in\mathbb{Z} \right\}$.
            \item Đặt $t=\tan x$, phương trình đã cho trở thành $\sqrt{3} t^2-2t+\sqrt{3}=0$. Phương trình này vô nghiệm. Vậy phương trình đã cho vô nghiệm.
            \item $2\cos^2 2x-5\cos 2x+2=0\Leftrightarrow\hoac{&\cos2x=2\\&\cos2x=\dfrac{1}{2}}\Leftrightarrow\cos2x=\cos\dfrac{\pi}{3}\Leftrightarrow x=\pm\dfrac{\pi}{6}+k\pi,\, (k\in\mathbb{Z})$;
            \item $\sin^2\dfrac{x}{2}+\sin\dfrac{x}{2}-2=0\Leftrightarrow\hoac{&\sin\dfrac{x}{2}=1\\&\sin\dfrac{x}{2}=-2}\Leftrightarrow\dfrac{x}{2}=\dfrac{\pi}{2}+k2\pi\Leftrightarrow x=\pi+k4\pi,\, (k\in\mathbb{Z})$.
        \end{enumerate}
    }
\end{ex}

\begin{ex}%[1D1G1-5]
    Tìm giá trị lớn nhất và giá trị nhỏ nhất của hàm số  $y=2(\sin x+\cos x)+\sin 2 x+3$.
    \loigiai
    {Tập xác định $\mathscr{D}=\mathbb{R}$.\\
        Đặt $t=\sin x+ \cos x=\sqrt{2}\sin \left(x+\dfrac{\pi}{4}\right)$, $t\in \left[-\sqrt{2};\sqrt{2}\right]$.\\
        Ta có $t^2=\left(\sin x+ \cos x\right)^2=1+2\sin x\cos x=1+\sin 2x\Rightarrow \sin 2x =t^2-1$.\\
        Hàm số trở thành $y=g(t)=t^2+2t+2$. \\
        Bảng biến thiên của hàm số $y=g(t)$ trên đoạn $ \left[-\sqrt{2};\sqrt{2}\right]$
        \begin{center}
            \begin{tikzpicture}
                \tkzTabInit[lgt=1,espcl=3,deltacl=1]%nocadre,
                {$t$/1, $g(t)$ /2}
                {$-\sqrt{2}$ , $-1$ ,  $\sqrt{2}$}
                \tkzTabVar {+/$4-2\sqrt{2}$,-/$1$ ,+/ $4+2\sqrt{2}$}
            \end{tikzpicture}
        \end{center}
        Vậy $\max\limits_{x \in \mathbb{R}} y=4+2\sqrt{2}$ và $\min\limits_{x \in \mathbb{R}} y=1$.
    }
\end{ex}

\begin{ex}%[1D1K1-5]
    Tìm giá trị lớn nhất và giá trị nhỏ nhất của hàm số $y=\sqrt{3} \sin x-\cos x+5$.
    \loigiai
    {
        Tập xác định $\mathscr{D}=\mathbb{R}$.\\
        Biến đổi $y=\sqrt{3} \sin x-\cos x+5=2\left(\dfrac{\sqrt{3}}{2}\cdot\sin x-\dfrac{1}{2}\cdot\cos x\right)+5=2\sin\left(x-\dfrac{\pi}{6}\right)+5$.\\
        Với mọi $x\in \mathbb{R}$ ta có
        \allowdisplaybreaks
        \begin{eqnarray*}
            & & -1\leq \sin\left(x-\dfrac{\pi}{6}\right)\leq 1\\
            &\Leftrightarrow& -2\leq 2\sin\left(x-\dfrac{\pi}{6}\right)\leq 2\\
            &\Leftrightarrow&3\leq  2\sin\left(x-\dfrac{\pi}{6}\right)+5\leq 7.
        \end{eqnarray*}
        Vậy $\max\limits_{x \in \mathbb{R}} y=7$ khi $x=\dfrac{2\pi}{3}$ và $\min\limits_{x \in \mathbb{R}} y=3$ khi $x=-\dfrac{\pi}{3}$.
    }
\end{ex}
\Closesolutionfile{ans}

%Chương II
% %%Bài 5. DS

\setcounter{section}{4}
\setcounter{dang}{0}
\setcounter{ex}{0}
\setcounter{bt}{0}
\setcounter{vd}{0}
\section{Dãy số}
\subsection{Tóm tắt lý thuyết}
\begin{tomtat}
	\subsubsection{Định nghĩa dãy số} 
	\begin{itemize}
		\item Mỗi hàm $u$ xác định trên tập các số nguyên dương $\mathbb{N^{*}}$ được gọi là một dãy vô hạn (gọi tắt là dãy số), kí hiệu $u=u(n)$.
		\item 	Ta thường viết $u_n$ thay cho $u(n)$ và kí hiệu dãy số $u=u(n)$ bởi $(u_n)$, do đó dãy số $(u_n)$ được viết dưới dạng khai triển $u_1, u_2, u_3, \ldots, u_n, \ldots$\\
		Số $u_1$ gọi là số hạng đầu, $u_n$ gọi là số hạng thứ $n$ và gọi là số hạng tổng quát của dãy số.
		\item Nếu $\forall n \in \mathrm{N^*}, u_n=c$ thì $(u_n)$ được gọi là dãy số không đổi.
		\item Mỗi hàm $u$ xác định trên tập $\mathrm{M}=\left\{1;2;3;\ldots;m\right\}, \forall m \in \mathrm{N^*}$ được gọi là một dãy số hữu hạn.
		\item Dạng khai triển của dãy hữu hạn là $u_1, u_2, u_3, \ldots, u_m$.\\
		Số $u_1$ gọi là số hạng đầu, số $u_m$ gọi là số hạng cuối.
		
	\end{itemize}
	\subsubsection{Các cách cho một dãy số}
	Một dãy số có thể cho bằng:
	\begin{itemize}
		\item Liệt kê các số hạng (chỉ dùng cho các dãy hữu hạn và có ít số hạng);
		\item Công thức của số hạng tổng quát;
		\item Phương pháp mô tả;
		\item Phương pháp truy hồi.
	\end{itemize}
	\subsubsection{Dãy số tăng, dãy số giảm, dãy số bị chặn}
	\begin{itemize}
		\item Dãy số $(u_n)$ được gọi là dãy số tăng nếu ta có $u_{n+1}>u_n, \forall n \in \mathrm{N^*}$.
		\item Dãy số $(u_n)$ được gọi là dãy số giảm nếu ta có $u_{n+1}<u_n, \forall n \in \mathrm{N^*}$.
		\item Dãy số $(u_n)$ được gọi là bị chặn trên nếu tồn tại số $M$ sao cho $u_n \le M, \forall n \in \mathrm{N^*}$.
		\item Dãy số $(u_n)$ được gọi là bị chặn dưới nếu tồn tại số $m$ sao cho $u_n \ge m, \forall n \in \mathrm{N^*}$.
		\item Dãy số $(u_n)$ được gọi là bị chặn nếu nó vừa bị chặn trên vừa bị chặn dưới, tức là  tồn tại các số $m, M$ sao cho $m \le u_n \le M, \forall n \in \mathrm{N^*}$.
	\end{itemize}
\end{tomtat}

\subsection{Các dạng toán thường gặp}
\begin{dang}{Số hạng tổng quát, biểu diễn dãy số}
	Để tìm số hạng tổng quát của một dãy bất kỳ khi biết một vài số hạng đầu của dãy số ta làm như sau
	\begin{itemize}
		\item Phân tích các số hạng sau theo các số hạng đã biết theo một quy luật nào đó.
		\item Dự đoán số hạng tổng quát 
		\item Kiểm tra bằng cách thay lần lượt các giá trị $n\in \mathrm{N^*}$ vào công thức tổng quát (Chứng minh bằng phương pháp quy nạp).
	\end{itemize}
	Để biểu diễn một dãy số khi biết công thức tổng quát ta lần lượt thay $n\in \mathrm{N^*}$ vào công thức tổng quát để tìm các số hạng thứ nhất, thứ hai, $\ldots$
\end{dang}
\subsubsection{Ví dụ minh hoạ}
\begin{vd}[NB]%[DCHT Toán 11 - KNTT -Tên GV]%[1K2Y1-1]
	Xác định số hạng đầu và số hạng tổng quát của dãy số $(u_n)$ các số tự nhiên lẻ $1, 3, 5, 7, \ldots $
	\dapso{$u_n=2n-1$}	
	\loigiai{Dãy $(u_n)$ có số hạng đầu $u_1=1$ và số hạng tổng quát $u_n=2n-1$.}
\end{vd}
\begin{vd}[NB]%[DCHT Toán 11 - KNTT -Tên GV]%[1K2Y1-1]
	Xác định số hạng đầu và số hạng tổng quát của dãy số $(v_n)$ các số nguyên dương chia hết cho $5$: $5,10,15,20,\ldots$
	\dapso{$v_n=5n$}
	\loigiai{Dãy $(v_n)$ có số hạng đầu $v_1=5$ và số hạng tổng quát $v_n=5n$.}
\end{vd}
% \begin{vd}[NB]%[DCHT Toán 11 - KNTT -Tên GV]%[1K2Y1-1]
% 	Viết năm số hạng đầu và số hạng thứ $100$ của dãy số $(u_n)$ có số hạng tổng quát $u_n=3n-2$.
% 	\dapso{$u_{100}=298$}
% 	\loigiai{Năm số hạng đầu của dãy số là $1,4,7,10,13$.\\
% 		Số hạng thứ $100$ của dãy là $u_{100}=3\cdot100-2=298$.}
% \end{vd}
% \begin{vd}[NB]%[DCHT Toán 11 - KNTT -Tên GV]%[1K2Y1-1]
% 	Cho dãy số xác định bằng hệ thức truy hồi: $u_1=1, u_n=3u_{n-1}+2$ với $n\ge 2$. Viết ba số hạng đầu của dãy số này.
% 	\dapso{$u_1=1, u_2=5, u_3=17$}
% 	\loigiai{Ta có $u_1=1, u_2=3u_1+2=5, u_3=3u_2+2=17$.}
% \end{vd}
% \begin{vd}[NB]%[DCHT Toán 11 - KNTT -Tên GV]%[1K2Y1-1]
% 	Dãy số $(u_n)$ cho bởi hệ thức truy hồi: $u_1=1, u_n=n \cdot u_{n-1}$ với $n \ge 2$. Viết năm số hạng đầu của dãy số và dự đoán công thức tổng quát $u_n$.
% 	\dapso{$u_n=n!$}
% 	\loigiai{Năm số hạng đầu của dãy là
% 		$u_1=1, u_2=2\cdot u_1=2, u_3=3\cdot u_2=6, u_4=4 \cdot u_3= 24, u_5=5 \cdot u_4=124$.\\
% 		Số hạng tổng quát\\
% 		Ta có $ u_2=2\cdot 1, u_3=6=3\cdot 2\cdot 1, u_4=24=4\cdot 3\cdot 2\cdot 1, u_5=124= 5\cdot4\cdot3\cdot2\cdot1  $.\\
% 		Vậy số hạng tổng quát $u_n=n!$.}
% \end{vd}
\subsubsection{Bài tập tự luận}
 
\begin{bt}[NB]%[DCHT Toán 11 - KNTT -Tên GV]%[1K2Y1-1]
	Xét dãy số hữu hạn gồm các số tự nhiên lẻ nhỏ hơn 20, sắp xếp theo thứ tự từ bé đến lớn. Liệt kê tất cả các số hạng của dãy số này, tìm số hạng đầu và số hạng cuối của dãy. 
	\dapso{$u_1=1$, $u_{11}=19$}
	\loigiai{Các số hạng của dãy là $1,3,5,7,9,10,11,13,15,17,19$.\\
		Số hạng đầu của dãy là $u_1=1$.\\
		Số hạng cuối của dãy là $u_{11}=19$.}
\end{bt}
\begin{bt}[TH]%[DCHT Toán 11 - KNTT -Tên GV]%[1K2Y1-1]
	Xét dãy số gồm tất cả các số tự nhiên chia cho $5$ dư $1$. Xác định số hạng tổng quát của dãy số.
	\dapso{$u_n=5n+1$}
	\loigiai{Các số tự nhiên chia cho $5$ dư $1$ gồm các số sau:
		$6,11,16,21, \ldots $\\
		Số hạng tổng quát $u_n=5n+1$.}
\end{bt}
% \begin{bt}[NB]%[DCHT Toán 11 - KNTT -Tên GV] %[1K2Y1-1]
% 	Tìm năm số hạng đầu và số hạng thứ $100$ của dãy $(u_n)$ có số hạng tổng quát $u_n= \dfrac{(-1)^n}{n}$.
% 	\dapso{$u_1=-1, \dfrac{1}{2},-\dfrac{1}{3}, \dfrac{1}{4}, -\dfrac{1}{5} $, $u_{100}=\dfrac{1}{100}$}
% 	\loigiai{
% 		Năm số hạng đầu của dãy là $u_1=-1, \dfrac{1}{2},-\dfrac{1}{3}, \dfrac{1}{4}, -\dfrac{1}{5} $.\\
% 		Số hạng thứ $100$ là $u_{100}=\dfrac{1}{100}$.}
% \end{bt}
\begin{bt}[NB]%[DCHT Toán 11 - KNTT -Tên GV]%[1K2Y1-1]
	Viết năm số hạng đầu của dãy số gồm các số nguyên tố theo thứ tự tăng dần.
	\dapso{$2,3,5,7,11$}
	\loigiai {Năm số hạng đầu của dãy số trên là $2,3,5,7,11$.} 
\end{bt}
% \begin{bt}[NB]%[DCHT Toán 11 - KNTT-Tên GV]%[1K2Y1-1]
% 	Viết năm số hạng đầu của dãy $(u_n)$ với số hạng tổng quát là $u_n=n!$.
% 	\dapso{$1,2,6,24,120$}
% 	\loigiai{Năm số hạng đầu của dãy trên là $1,2,6,24,120$.}
% \end{bt}
\subsubsection{Câu hỏi trắc nghiệm}
\Opensolutionfile{ans}[ans/ans-1K2-1-Dang1]
%Cau1
\begin{ex}%[DCHT Toán 11 - KNTT -Tên GV]%[1K2B1-1]
	Cho dãy số có các số hạng đầu là $5,10,15,20,25, \ldots$ Số hạng tổng quát của dãy số này là
	\choice
	{$ u_n=5(n-1) $}
	{\True$ u_n=5n $}
	{$ u_n=5+n $}
	{$ u_n=5n+1 $}
	\loigiai
	{Ta có $5=5\cdot 1, 10=5 \cdot 2, 15 = 5\cdot 3, 20=5 \cdot 4, 25 = 5\cdot 5, \ldots$\\
		Vậy dãy trên có số hạng tổng quát là $u_n=5n$.
	}
\end{ex}
%Cau2
\begin{ex}%[DCHT Toán 11 - KNTT -Tên GV]%[1K2B1-1]
	Cho dãy số $(u_n)$ với $u_n=\dfrac{an^2}{n+1}$, $a$ là hằng số. $u_{n+1}$ là số hạng nào trong các số hạng sau
	\choice
	{\True $u_{n+1}=\dfrac{a(n+1)^2}{n+2} $}
	{$u_{n+1}=\dfrac{a(n+1)^2}{n+1}$}
	{$u_{n+1}=\dfrac{an^2+1}{n+1}$}
	{$u_{n+1}=\dfrac{an^2}{n+2} $}
	\loigiai
	{Ta có $u_{n+1}=\dfrac{a(n+1)^2}{n+1+1}=\dfrac{a(n+1)^2}{n+2}$.
	}
\end{ex}
%cau3
\begin{ex}%[DCHT Toán 11 - KNTT -Tên GV]%[1K2B1-1]
	Cho dãy số có các số hạng đầu là $8,15,22,29,36, \ldots$ Số hạng tổng quát của dãy số này là
	\choice
	{$ u_n=7n+7 $}
	{$ u_n=7n $}
	{\True $ u_n=7n+1 $}
	{$ u_n$ không viết được dưới dạng công thức }
	\loigiai
	{Ta có $8=7\cdot 1+1, 15=7 \cdot 2+1, 22 = 7\cdot 3+1, 29=7 \cdot 4+1, 36 = 7\cdot 5+1, \ldots$\\
		Vậy dãy trên có số hạng tổng quát là $u_n=7n+1$.
	}
\end{ex}
%Cau4
\begin{ex}%[DCHT Toán 11 - KNTT -Tên GV]%[1K2B1-1]
	Cho dãy số có các số hạng đầu là $0,\dfrac{1}{2},\dfrac{2}{3},\dfrac{3}{4},\dfrac{4}{5}, \ldots$ Số hạng tổng quát của dãy số này là
	\choice
	{$ u_n=\dfrac{n+1}{n}$}
	{\True $ u_n=\dfrac{n}{n+1} $}
	{$ u_n=\dfrac{n-1}{n}$}
	{$ u_n=\dfrac{n^2-n}{n+1}$  }
	\loigiai
	{Ta có $0=\dfrac{0}{0+1}, \dfrac{1}{2}=\dfrac{1}{1+1} ,\dfrac{2}{3} = \dfrac{2}{2+1}, \dfrac{3}{4}=\dfrac{3}{3+1}, \dfrac{4}{5} = \dfrac{4}{4+1}, \ldots$\\
		Vậy dãy trên có số hạng tổng quát là $u_n=\dfrac{n}{n+1}$.
	}
\end{ex}
%Cau5
\begin{ex}%[DCHT Toán 11 - KNTT -Tên GV]%[1K2B1-1]
	Cho dãy số $(u_n)$ với $u_1=1, u_{n+1}=u_n-1$. Số hạng tổng quát $u_n$ của dãy số là số hạng nào dưới đây?
	\choice
	{\True $ u_n=2-n$}
	{$ u_n$ không xác định}
	{$ u_n=1-n$}
	{$ u_n=-n$, với mọi $n$ }
	\loigiai
	{Ta có $u_1=1, u_2=0 ,u_3 = -1, u_4=-2,  \ldots$\\
		Dễ dàng dự đoán được số hạng tổng quát là $u_n=2-n$.
	}
\end{ex}
% %cau6
% \begin{ex}%[DCHT Toán 11 - KNTT -Tên GV]%[1K2B1-1]
% 	Cho dãy số $(u_n)$ với $u_n=\dfrac{2n^2-1}{n^2+3}, \forall n \in \mathrm{N*}$. Số hạng đầu tiên của dãy số là 
% 	\choice
% 	{$ u_1=-\dfrac{1}{3}$}
% 	{$ u_1=\dfrac{2}{3}$}
% 	{$ u_1=\dfrac{1}{3}$}
% 	{\True $ u_1=\dfrac{1}{4}$ }
% 	\loigiai
% 	{Ta có $u_1=\dfrac{2\cdot 1^2-1}{1^2+3}=\dfrac{1}{4}$.
% 	}
% \end{ex}
% %cau7
% \begin{ex}%[DCHT Toán 11 - KNTT -Tên GV]%[1K2B1-1]
% 	Cho dãy số $(u_n)$ với $u_1=-1, u_{n+1}=u_n+3$ với $n \ge 1$. Ba số hạng đầu tiên của dãy số lần lượt là 
% 	\choice
% 	{\True $-1, 2, 5$}
% 	{$ 1, 4, 7$}
% 	{$ 4,7,10$}
% 	{$-1,3,7$ }
% 	\loigiai
% 	{Ta có $u_1=-1, u_2=-1+3=2 ,u_3 = 2+3=5$.
% 	}
% \end{ex}
\Closesolutionfile{ans}
\begin{indapan}{10}
	{ans/ans-1K2-1-Dang1}
\end{indapan}

\begin{dang}{Tìm số hạng cụ thể của dãy số}
	Để tìm số hạng cụ thể của dãy số ta làm như sau
	\begin{itemize} 
		\item Với trường hợp dãy số đã cho biết công thức tổng quát của dãy số thì ta chỉ cần thay giá trị tương ứng của số hạng đó vào công thức tổng quát.
		\item  Với trường hợp dãy số cho bởi công thức truy hồi hoặc dưới dạng thì ta phải tìm lần lượt từ những số hạng đầu tiên cho đến số đứng trước số cần tìm trong dãy.
	\end{itemize}
\end{dang}
\subsubsection{Ví dụ minh hoạ}
\begin{vd}[NB]%[1K2Y5-2]
	Cho dãy số $(u_n),$ biết $u_n=(-1 )^n\cdot \dfrac{2^n}{n}$. Tìm số hạng $u_3$.
	\dapso{$u_3=-\dfrac{8}{3}$}
	\choice
	{\True $u_3=-\dfrac{8}{3}$}
	{$u_3=2$}
	{$u_3=-2$}
	{$u_3=\dfrac{8}{3}$}
	\loigiai{
		Ta có
		$$u_3=(-1)^3\cdot \dfrac{2^3}{3}=-\dfrac{8}{3}.$$}
\end{vd}
\begin{vd}[NB]%[DCHT Toán 11 - KNTT -Nguyễn Long]%[1K2Y5-2]
	Cho dãy số $(u_n)$, biết $u_n=\dfrac{2n^2-1}{n^2+3}$. Tìm số hạng $u_5$.
	\dapso{$u_5=\dfrac{7}{4}$}
	\choice
	{$u_5=\dfrac{1}{4}$}
	{\True $u_5=\dfrac{7}{4}$}
	{$u_5=\dfrac{17}{12}$}
	{$u_5=\dfrac{71}{39}$}
	\loigiai{
		Ta có $u_5=\dfrac{2\cdot 5^2-1}{5^2+3}=\dfrac{49}{28}=\dfrac{7}{4}$.}
	
\end{vd}
\begin{vd}[NB]%[1K2Y5-2]
	Cho dãy số $u_n$ bao gồm các số nguyên tố. Tìm số hạng thứ $5$ của dãy số.
	\dapso{$u_5=11$}
	\loigiai{ 
		Ta có
		$u_1=2,u_2=3,u_3=5,u_4=7,u_5=11$. \\
		Vậy số hạng thứ $5$ của dãy số là $11$.
	}
\end{vd}
\begin{vd}[NB]%[1K2Y5-2]
	Cho dãy số $(u_n) $ thỏa mãn $ \heva{& u_1 = 5 \\& u_{n+1} = u_n+n}$. Tìm số hạng thứ $5$ của dãy số.
	\dapso{$u_5=15$}
	\choice
	{$ 11 $}
	{\True $ 15 $}
	{$ 16 $}
	{$ 12 $}
	\loigiai{
		Ta có $ u_2=u_1+1=6$, $ u_3=u_2+2=8$, $ u_4=u_3+3=11$,  $ u_5=u_4+4=15$.
	}
\end{vd}

\begin{vd}[TH]%[VD 5 SGK KNTT]%[1K2B5-2]
	Cho dãy số xác định bằng hệ thức truy hồi
	$$
	u_1=1, u_n=3 u_{n-1}+2 \text { với } n \geq 2
	$$
	Viết ba số hạng đầu của dãy số này.
	\dapso{$u_5=17$}
	\loigiai{
		Ta có: $u_1=1, u_2=3 u_1+2=3 \cdot 1+2=5, u_3=3 u_2+2=3 \cdot 5+2=17$.
	}
\end{vd}

\begin{vd}[VD]%[1K2B5-2]
	Cho dãy số $\left(u_n\right)\colon\heva{&u_1=5 \\ &u_{n+1}=u_n+n}$. Số $20$ là số hạng thứ mấy trong dãy?
	\dapso{số hạng thứ $6$}
	\loigiai{
		Ta có $u_1=5, u_2=6, u_3=8, u_4=11, u_5=16, u_6=20$.\\
		Vậy số $20$ là số hạng thứ $6$.}
\end{vd}

\subsubsection{Bài tập tự luận}
 
\begin{bt}[NB]%[1K2Y5-2]
	Cho dãy số $u_n=\dfrac{1}{\sqrt{n}+1}$. Tìm số hạng $u_4$.	
	\dapso{$u_4=\dfrac{1}{3}$}
	\loigiai{ Ta có
		$u_4=\dfrac{1}{\sqrt{4}}+1=\dfrac{1}{3}.$		
	}
\end{bt}
%%%
\begin{bt}[NB]%[1K2Y5-2]
	Cho dãy số $(u_n)$ có số hạng tổng quát: $u_n=2 n+\sqrt{n^2+4}$. Tìm số hạng thứ $6$ của dãy số.
	\dapso{$u_6=12+2\sqrt{10}$}
	\loigiai{
		Ta có $u_6=12+2 \sqrt{10}$.
	}
\end{bt}
%%%
\begin{bt}[NB]%[1K2Y5-2]
	Cho dãy số $(u_n)$ xác định bởi: $\heva{&u_1=-1 ; u_2=3 \\&u_{n+1}=5 u_n-6 u_{n-1} \forall n \geq 2}.$ Tìm số hạng thứ $7$ của dãy.
	\dapso{$3261$}
	\loigiai{
		Ta có
		$$
		u_3=5 u_2-6 u_1=21 ;~ u_4=5 u_3-6 u_2=87 ;~ u_5=309 ;~ u_6=1023 ;~ u_7=3261
		$$
		Vậy số hạng thứ $7$ của dãy là $3261$.
	}
\end{bt}
%%%%%%%
\begin{bt}[NB]%[1K2Y5-2]
	Viết năm số hạng đầu của dãy số Fibonacci $\left(F_n\right)$ cho bởi hệ thức truy hồi
	$$
	\heva{
		&F_1=1, F_2=1 \\
		&F_n=F_{n-1}+F_{n-2}~(n \geq 3) .
	}
	$$
	\dapso{$F_3=2,~F_4=3,~F_5=5$}
	\loigiai{
		Ta có $F_3=2,~F_4=3,~F_5=5.$
	}
\end{bt}
%%%
\begin{bt}[NB]%[1K2T5-2]
	Người ta nuôi cấy $5$ con vi khuẩn E-coli trong môi trường nhân tạo. Cứ $30$ phút thì vi khuẩn E-coli sẽ nhân đôi 1 lần. Tính số lượng vi khuẩn thu được sau $1,2,3$ lần nhân đôi.
	\dapso{$u_2=10, u_3=20, u_4=40$}
	\loigiai{
		Đặt $u_1=5$, gọi số vi khuẩn sau $n$ lần phân chia là $u_{n+1}$, khi đó ta có dãy số $(u_n)$ thỏa mãn $$u_1=5, \; u_{n+1}=2u_n$$
		Ta có $u_2=10, u_3=20, u_4=40$.
	}	
\end{bt}
%%%%%
\begin{bt}[TH]%[1K2B5-2]
	Cho dãy số $(u_n)$ được xác định bởi $u_n=\dfrac{n^2+3n+7}{n+1}$.
	\begin{listEX}
		\item Viết năm số hạng đầu của dãy.
		\item Dãy số có bao nhiêu số hạng nhận giá trị nguyên.
	\end{listEX}
	\dapso{$u_1=\dfrac{11}{2}$; $u_2=\dfrac{17}{3}$; $u_3=\dfrac{25}{4}$; $u_4=7$; $u_5=\dfrac{47}{6}$. $u_4=7 $}
	\loigiai{		
		\begin{listEX}
			\item Ta có năm số hạng đầu của dãy
			$u_1=\dfrac{1^2+3.1+7}{1+1}=\dfrac{11}{2}$; $u_2=\dfrac{17}{3}$; $u_3=\dfrac{25}{4}$; $u_4=7$; $u_5=\dfrac{47}{6}$.
			\item Ta có: $u_n=n+2+\dfrac{5}{n+1}$, do đó $u_n$ nguyên khi và chỉ khi $ \dfrac{5}{n+1}$ nguyên hay $ n+1 $ là ước của 5. Điều đó xảy ra khi $ n+1=5\Leftrightarrow n=4 $. Vậy dãy số có duy nhất một số hạng nguyên là $u_4=7 $.
		\end{listEX}		
	}
\end{bt}
\begin{bt}[VD]%[1K2K5-2]
	Cho dãy số $\left(x_n\right)$ thỏa mãn điều kiện $x_1=1, x_{n+1}-x_n=\dfrac{1}{n(n+1)}, n=1,2,3, \ldots$. Số hạng $x_{2023}$ bằng
	\dapso{$x_{2023}=\dfrac{4045}{2023}$}
	\loigiai{
		Ta có
		$$
		\begin{aligned}
			x_{n+1}-x_n=\dfrac{1}{n(n+1)}=\dfrac{1}{n}-\dfrac{1}{n+1} & \Leftrightarrow \sum_{k=1}^{n-1}\left(x_{k+1}-x_k\right)=\sum_{k=1}^{n-1}\left(\dfrac{1}{k}-\dfrac{1}{k+1}\right) \\
			& \Leftrightarrow x_n-x_1=1-\dfrac{1}{n} \\
			& \Leftrightarrow x_n=\dfrac{2n-1}{n} .
		\end{aligned}
		$$
	}
\end{bt}
\begin{bt}[VDC]%[1K2G5-2]
	Cho dãy số $\left(u_n\right)$ biết $\heva{&u_1=99 \\&u_{n+1}=u_n-2 n-1, n \geq 1}$. Hỏi số $-861$ là số hạng thứ mấy?
	\dapso{$-861$ là số hạng thứ $31$}
	\loigiai{
		Ta có
		$$
		\begin{aligned}
			&u_n &=& &u_{n-1}-2 n+1 \\
			&u_{n-1} & = & &u_{n-2}-2 n+3 \\
			&\vdots &\vdots&  &\vdots \\
			&u_3 & = & &u_2-2 n+2 n-5 \\
			&u_2 & = & &u_1-2 n+2 n-3
		\end{aligned}
		$$
		Suy ra
		$$
		\begin{aligned}
			& u_n=u_1-2 n \cdot(n-1)+1+3+5+\cdots+(2 n-5)+(2 n-3) \\
			& u_n=99-2 n^2+2 n+\dfrac{n-1}{2}\cdot[2 \cdot 1+(n-2) \cdot 2]=100-n^2
		\end{aligned}
		$$
		Giả sử $u_n=-861 \Rightarrow n^2=961 \Rightarrow n=31$ (vì $n \in \mathbb{N}$).
		Vậy số $-861$ là số hạng thứ $31$ .}
\end{bt}
\subsubsection{Câu hỏi trắc nghiệm}
\Opensolutionfile{ans}[ans/ans-1K2-1-Dang2]
%Câu 1
\begin{ex}%[1K2Y5-2]
	Cho dãy số $({{u}_{n}} )$, biết ${{u}_{n}}=\dfrac{n}{{{3}^{n}}-1}$. Ba số hạng đầu tiên của dãy số đó lần lượt là những số nào dưới đây?
	\choice
	{$\dfrac{1}{2};\dfrac{1}{4};\dfrac{1}{16}$}
	{$\dfrac{1}{2};\dfrac{2}{3};\dfrac{3}{4}$}
	{ \True $\dfrac{1}{2};\dfrac{1}{4};\dfrac{3}{26}$}
	{$\dfrac{1}{2};\dfrac{1}{4};\dfrac{1}{8}$}
	\loigiai {
		Ta có
		${{u}_{1}}=\dfrac{1}{2};\,\,{{u}_{2}}=\dfrac{2}{{{3}^2}-1}=\dfrac{2}{8}=\dfrac{1}{4};\,\,{{u}_{3}}=\dfrac{3}{{{3}^3}-1}=\dfrac{3}{26}$.}
	
\end{ex}
%%%%%%%%%%
%Câu 2
\begin{ex}%[1K2Y5-2]
	Cho dãy số $({{u}_{n}} ),$ biết ${{u}_{n}}={{(-1 )}^{n}}\cdot 2n$. Mệnh đề nào sau đây {\bf sai}?
	\choice
	{${{u}_{3}}=-6$}
	{${{u}_{2}}=4$}
	{ \True ${{u}_{4}}=-8$}
	{${{u}_{1}}=-2$}
	\loigiai {
		Ta có\\
		${{u}_{1}}=-2\cdot 1=-2;\,\,{{u}_{2}}={{(-1 )}^2}\cdot 2\cdot 2=4,\,\,{{u}_{3}}={{(-1 )}^3}\cdot 2\cdot 3=-6;\,\,{{u}_{4}}={{(-1 )}^4}\cdot 2\cdot 4=8$.\\
		\textbf{Nhận xét:} Dễ thấy ${{u}_{n}}>0$ khi $n$ chẵn và ngược lại nên đáp án $u_4=-8$ sai.}
	
\end{ex}
%%%%%%%%%%
%Câu 3
\begin{ex}%[1K2Y5-2]
	Cho dãy số $({{u}_{n}} )$ xác định bởi $\heva{
		& {{u}_{1}}=2 \\
		& {{u}_{n+1}}=\dfrac{1}{3}({{u}_{n}}+1 ) \\
	}.$ Tìm số hạng ${{u}_{4}}$.
	\choice
	{${{u}_{4}}=\dfrac{2}{3}$}
	{${{u}_{4}}=1$}
	{${{u}_{4}}=\dfrac{14}{27}$}
	{ \True ${{u}_{4}}=\dfrac{5}{9}$}
	\loigiai {
		Ta có
		${{u}_{2}}=\dfrac{1}{3}({{u}_{1}}+1 )=\dfrac{1}{3}(2+1 )=1;\,\,{{u}_{3}}=\dfrac{1}{3}({{u}_{2}}+1 )=\dfrac{2}{3};\,\,{{u}_{4}}=\dfrac{1}{3}({{u}_{3}}+1 )=\dfrac{1}{3}\cdot\left(\dfrac{2}{3}+1\right)=\dfrac{5}{9}$. \\
	}
\end{ex}
%%%%%%%%%%
%Câu 4
\begin{ex}%[1K2Y5-2]
	Cho dãy số $({{u}_{n}} )$, biết $\heva{
		& {{u}_{1}}=-1 \\
		& {{u}_{n+1}}={{u}_{n}}+3 \\
	}$ với $n\ge 0$. Ba số hạng đầu tiên của dãy số đó là lần lượt là những số nào dưới đây?
	\choice
	{\True $-1;\,2;\,5$}
	{$-1;3;7$}
	{$1;\,4;\,7$}
	{$4;\,7;\,10$}
	\loigiai {
		Ta có ${{u}_{1}}=-1;\,\,{{u}_{2}}={{u}_{1}}+3=2;\,\,{{u}_{3}}={{u}_{2}}+3=5$. \\
		\textbf{Nhận xét.} (i) Dùng chức năng “lặp” của MTCT để tính:\\
		Nhập vào màn hình: $X=X+3$ \\
		Bấm CALC và cho $X=-1$ (ứng với ${{u}_{1}}=-1)$ \\
		Để tính ${{u}_{n}}$ cần bấm “=” ra kết quả liên tiếp $n-1$ lần. Ví dụ để tính ${{u}_{2}}$ ta bấm “=” ra kết quả lần đầu tiên, bấm “=” ra kết quả thứ hai chính là ${{u}_{3}},\ldots$\\
		(ii) Vì ${{u}_{1}}=-1$ nên loại các đáp án $u_1=1$, $u_1=4$.\\
		Còn lại các đáp án có $u_1=-1$; để biết đáp án nào ta chỉ cần kiểm tra ${{u}_{2}}$ (vì ${{u}_{2}}$ ở hai đáp án là khác nhau): ${{u}_{2}}={{u}_{1}}+3=2$.
	}
	
\end{ex}
%%%%%%%%%%
%Câu 5
\begin{ex}%[1K2B5-2]
	Cho dãy số $({{u}_{n}} ),$ biết ${{u}_{n}}=\dfrac{2n+5}{5n-4}$. Số $\dfrac{7}{12}$ là số hạng thứ mấy của dãy số?
	\choice
	{$9$}
	{$6$}
	{$10$}
	{\True $8$}
	\loigiai {
		Ta có
		$${{u}_{n}}=\dfrac{2n+5}{5n-4}=\dfrac{7}{12}\Leftrightarrow 24n+60=35n-28\Leftrightarrow 11n=88\Leftrightarrow n=8.$$}
	
\end{ex}
%%%%%%%%%%
%Câu 6
\begin{ex}%[1K2B5-2]
	Cho dãy $(u_n)$ xác định bởi $\heva{& u_1=3 \\& u_{n+1}=\dfrac{u_n}{2}+2}$. Mệnh đề nào sau đây {\bf sai}?
	\choice
	{\True ${{u}_{2}}=\dfrac{5}{2}$}
	{${{u}_{4}}=\dfrac{31}{8}$}
	{${{u}_{3}}=\dfrac{15}{4}$}
	{${{u}_{5}}=\dfrac{63}{16}$}
	\loigiai {
		Ta có $\heva{
			& {{u}_{2}}=\dfrac{{{u}_{1}}}{2}+2=\dfrac{3}{2}+2=\dfrac{7}{2};\,\,{{u}_{3}}=\dfrac{{{u}_{2}}}{2}+2=\dfrac{7}{4}+2=\dfrac{15}{4}. \\
			& {{u}_{4}}=\dfrac{{{u}_{3}}}{2}+2=\dfrac{15}{8}+2=\dfrac{31}{8};\,\,{{u}_{5}}=\dfrac{{{u}_{4}}}{2}+2=\dfrac{31}{16}+2=\dfrac{63}{16}. \\
		}$}
\end{ex}
%%%%%%%%%%
%Câu 7
\begin{ex}%[1K2B5-2]
	Cho dãy số $({{u}_{n}} ),$ với ${{u}_{n}}={{\left(\dfrac{n-1}{n+1} \right)}^{2n+3}}$. Tìm số hạng ${{u}_{n+1}}$.
	\choice
	{${{u}_{n+1}}={{\left(\dfrac{n-1}{n+1} \right)}^{2(n-1 )+3}}$}
	{${{u}_{n+1}}={{\left(\dfrac{n-1}{n+1} \right)}^{2(n+1 )+3}}$ }
	{\True ${{u}_{n+1}}={{\left(\dfrac{n}{n+2} \right)}^{2n+5}}$}
	{${{u}_{n+1}}={{\left(\dfrac{n}{n+2} \right)}^{2n+3}}$}
	\loigiai {
		${{u}_{n}}={{\left(\dfrac{n-1}{n+1} \right)}^{2n+3}}\Rightarrow {{u}_{n+1}}={{\left(\dfrac{(n+1 )-1}{(n+1 )+1} \right)}^{2(n+1 )+3}}={{\left(\dfrac{n}{n+2} \right)}^{2n+5}}$.}
	
\end{ex}
%%%%%%%%%%
%Câu 8
\begin{ex}%[1K2K5-2]
	Cho dãy số $({{a}_{n}} ),$ được xác định $\heva{
		& {{a}_{1}}=3 \\
		& {{a}_{n+1}}=\dfrac{1}{2}{{a}_{n}},~n\ge 1 \\
	}$. Mệnh đề nào sau đây {\bf sai}?
	\choice
	{${{a}_{1}}+{{a}_{2}}+{{a}_{3}}+{{a}_{4}}+{{a}_{5}}=\dfrac{93}{16}$}
	{${{a}_{10}}=\dfrac{3}{512}$}
	{ \True ${{a}_{n}}=\dfrac{3}{{{2}^{n}}}$}
	{${{a}_{n+1}}+{{a}_{n}}=\dfrac{9}{{{2}^{n}}}$}
	\loigiai {
		Ta có ${{a}_{1}}=3;\,{{a}_{2}}=\dfrac{{{u}_{1}}}{2};\,\,{{a}_{3}}=\dfrac{{{u}_{2}}}{2}=\dfrac{{{u}_{1}}}{{{2}^2}};\,\,{{a}_{4}}=\dfrac{{{u}_{3}}}{2}=\dfrac{{{u}_{1}}}{{{2}^3}},\ldots \\
		\Rightarrow {{u}_{n}}=\dfrac{{{u}_{1}}}{{{2}^{n-1}}}=\dfrac{3}{{{2}^{n-1}}}$ nên suy ra đáp án ${{a}_{n}}=\dfrac{3}{{{2}^{n}}}$ sai. \\
		Xét đáp án\\
		${{a}_{1}}+{{a}_{2}}+{{a}_{3}}+{{a}_{4}}+{{a}_{5}}=3\left(1+\dfrac{1}{2}+\dfrac{1}{{{2}^2}}+\dfrac{1}{{{2}^3}}+\dfrac{1}{{{2}^4}}\right)=3.\dfrac{1-{{(\dfrac{1}{2} )}^5}}{1-\dfrac{1}{2}}=\dfrac{93}{16}\Rightarrow $  đúng.\\
		Xét đáp án ${{a}_{10}}=\dfrac{3}{{{2}^{9}}}=\dfrac{3}{512}\Rightarrow $  đúng.\\
		Xét đáp án ${{a}_{n+1}}+{{a}_{n}}=\dfrac{3}{{{2}^{n}}}+\dfrac{3}{{{2}^{n-1}}}=\dfrac{3+3\cdot 2}{{{2}^{n}}}=\dfrac{9}{{{2}^{n}}}\Rightarrow $ đúng.}
	
\end{ex}
%%%%%%%%%%
%Câu 9
\begin{ex}%[1K2K5-2]
	Cho dãy số $(u_n)$ biết $\heva{&u_1=1\\&u_2=4\\&u_{n+2}=3u_{n+1}-2u_n}$ với mọi $n \ge 1$. Giá trị $u_{101}-u_{100}$ là 
	\choice
	{$3\cdot 2^{102} $}
	{$3\cdot 2^{101} $}
	{$3\cdot 2^{100} $}
	{\True $ 3\cdot 2^{99}$}
	\loigiai{
		Theo bài  ta có 
		\begin{eqnarray*}
			&u_{n+2}=3u_{n+1}-2u_n\\
			\Leftrightarrow \,& u_{n+2}=u_{n+1}+2(u_{n+1}-u_n)\\
			\Leftrightarrow \,& u_{n+2}-u_{n+1}=2(u_{n+1}-u_n).
		\end{eqnarray*}
		Với $n=99$ ta có 
		\begin{align*}
			u_{101}-u_{100}&=2(u_{100}-u_{99})\\
			&=2\cdot 2 (u_{99}-u_{98})\\
			&= \ldots\\
			&=2^{99}\cdot(u_2-u_1)=3\cdot2^{99}.
		\end{align*}
	}
\end{ex}
%%%%%%%%%%
%Câu 10
\begin{ex}%[1K2G5-2]
	Cho dãy số $\left(u_n\right)$ thoả mãn $u_1=\sqrt{2}$ và $u_{n+1}=\sqrt{2+u_n}$ với mọi $n\geq 1$. Tìm $u_{2023}$.
	\choice
	{$u_{2023}=\sqrt{2}\cos\dfrac{\pi}{2^{2022}}$}
	{\True $u_{2023}=\sqrt{2}\cos\dfrac{\pi}{2^{2024}}$}
	{$u_{2023}=\sqrt{2}\cos\dfrac{\pi}{2^{2023}}$}
	{$u_{2023}=2$}
	\loigiai{Ta chứng minh bằng phương pháp quy nạp số hạng tổng quát của dãy là $u_n=2\cos\dfrac{\pi}{2^{n+1}}$.\\
		Dễ thấy, với $n=1$ ta có $u_1=\sqrt{2}$ (đúng).\\
		Giả sử mệnh đề đúng với $n=k, \forall k\in \mathbb{N}^\ast$ nghĩa là $u_k=2\cos\dfrac{\pi}{2^{k+1}}$ ta phải chứng minh mệnh đề đúng với $n=k+1$ nghĩa là $u_{k+1}=2\cos\dfrac{\pi}{2^{k+2}}$.\\
		Thật vậy, $u_{k+1}=\sqrt{2+u_k}=\sqrt{2+2\cos\dfrac{\pi}{2^{k+1}}}=\sqrt{4\cos^2\dfrac{\pi}{2^{k+2}}}=2\cos\dfrac{\pi}{2^{k+2}}$.\\
		Áp dụng công thức tổng quát trên ta có $u_{2023}=\sqrt{2}\cos\dfrac{\pi}{2^{2024}}$.
	}
\end{ex}
%%%%%%%%%%
\Closesolutionfile{ans}
\begin{indapan}{10}
	{ans/ans-1K2-1-Dang2}
\end{indapan}
\begin{dang}{Xét tính tăng giảm của dãy số}
	\begin{enumerate}
		\item Phương pháp 1. Xét dấu của hiệu số $u_{n+1}-u_n$.
		\begin{enumerate}
			\item Nếu $u_{n+1}-u_n>0, \forall n \in \mathbb{N}^\ast$ thì $(u_n)$ là dãy số tăng.
			\item Nếu $u_{n+1}-u_n<0, \forall n \in \mathbb{N}^\ast$ thì $(u_n)$ là dãy số giảm.
		\end{enumerate}
		\item Phương pháp 2. Nếu $u_n>0, \forall n\in \mathbb{N}^\ast$ thì ta có thể so sánh thương $\dfrac{u_{n+1}}{u_n}$ với $1$.
		\begin{enumerate}
			\item Nếu $\dfrac{u_{n+1}}{u_n}>1$ thì $(u_n)$ là dãy số tăng.
			\item Nếu $\dfrac{u_{n+1}}{u_n}<1$ thì $(u_n)$ là dãy số giảm.
		\end{enumerate}
		Nếu $u_n<0, \forall n\in \mathbb{N}^\ast$ thì ta có thể so sánh thương $\dfrac{u_{n+1}}{u_n}$ với $1$.
		\begin{enumerate}
			\item Nếu $\dfrac{u_{n+1}}{u_n}<1$ thì $(u_n)$ là dãy số tăng.
			\item Nếu $\dfrac{u_{n+1}}{u_n}>1$ thì $(u_n)$ là dãy số giảm.
		\end{enumerate}
		\item Phương pháp 3. Nếu dãy số $(u_n)$ cho bởi hệ thức truy hồi thì thường dùng phương pháp quy nạp để chứng minh $u_{n+1}>u_n, \forall n \in \mathbb{N}^\ast$ (hoặc $u_{n+1}<u_n \forall n \in \mathbb{N}^\ast$).
	\end{enumerate}
\end{dang}
\subsubsection{Ví dụ minh hoạ}
\begin{vd}[NB]%[1K2Y5-3]
	Xét sự tăng giảm của dãy số $(u_n)$ với $u_n=(-1)^n$.
	\dapso{dãy không tăng không giảm}
	\loigiai{
		Ta có:\\ $u_1=(-1)^1=-1,\,
		u_2=(-1)^2=1,\,
		u_3=(-1)^3=-1.$\\
		Vậy $(u_n)$ là dãy không tăng không giảm.
	}
\end{vd}
\begin{vd}[NB]%[1K2Y5-3]
	Xét tính tăng giảm của dãy số sau $(u_n)$ với $u_n=\dfrac{2n+1}{n+1}$.
	\dapso{dãy số tăng}
	\loigiai
	{
		Ta có: $u_n=\dfrac{2n+1}{n+1}=2-\dfrac{1}{n+1}$.\\
		$u_{n+1}-u_n=\left(2-\dfrac{1}{n+1+1}\right)-\left(2-\dfrac{1}{n+1}\right)=\dfrac{1}{n+1}-\dfrac{1}{n+2}>0, \forall n \in \mathbb{N}^\ast$.\\
		Vậy dãy số $(u_n)$ là dãy số tăng.
	}
\end{vd}
\begin{vd}[TH]%[1K2B5-3]
	Xét tính tăng giảm của dãy số $(u_n)$ với $u_n=\sqrt{n}-\sqrt{n+2}$.
	\dapso{dãy số tăng}
	\loigiai{
		Ta có $u_n=\sqrt{n}-\sqrt{n+2}=\dfrac{-2}{\sqrt{n}+\sqrt{n+2}}$.\\
		Xét hiệu\\ 
		$\begin{aligned}
			u_{n+1}-u_n&=\dfrac{-2}{\sqrt{n+1}+\sqrt{n+3}}-\dfrac{-2}{\sqrt{n}+\sqrt{n+2}}\\
			&=\dfrac{2}{\sqrt{n}+\sqrt{n+2}}-\dfrac{2}{\sqrt{n+1}+\sqrt{n+3}}>0, \forall n\in \mathbb{N}^\ast.
		\end{aligned}$\\
		Vậy $(u_n)$ là dãy số tăng.
	}
\end{vd}

\begin{vd}[TH]%[1K2B5-3]
	Xét tính tăng giảm của dãy số $(u_n)$ với $u_n=\dfrac{n}{3^n}$.
	\dapso{dãy số giảm}
	\loigiai{
		Ta có $u_n=\dfrac{n}{3^n}>0, \forall n \in \mathbb{N}^\ast$.\\
		Xét thương $\dfrac{u_{n+1}}{u_n}=\dfrac{n+1}{3^{n+1}}:\dfrac{n}{3^n}=\dfrac{n+1}{3.n}<1, \forall  n \in \mathbb{N}^\ast$.\\
		Vậy $(u_n)$ là dãy số giảm.
	}
\end{vd}
\begin{vd}[VD]%[1K2K5-3]
	Xét tính tăng giảm của dãy số $(u_n)$ với $ \heva{& u_1=2\\
		& u_{n+1}=\dfrac{3u_n+1}{u_n+1}, n\in \mathbb{N}^\ast.}$
	\dapso{dãy số tăng}
	\loigiai{
		Giả sử $u_{n+1}>u_n , \forall n \in \mathbb{N}^\ast. \qquad (*)$\\
		Ta chứng minh $(*)$ bằng phương pháp quy nạp.
		\begin{itemize}
			\item Với $n=1, u_2=\dfrac{3.2+1}{2+1}=\dfrac{6}{3}=\dfrac{7}{3}>u_1=2.$
			\item Giả sử $(*)$ đúng khi $n=k, k\in \mathbb{N}^\ast$, tức là $u_{k+1}>u_k$.\\
			Ta sẽ chứng minh $(*)$ đúng với $n=k+1$, tức là
			$u_{k+2}>u_{k+1}$.\\
			Thật vậy\\ $u_{k+2}-u_{k+1}=\left(3-\dfrac{2}{u_{k+1}+1}\right)-\left(3-\dfrac{2}{u_k+1}\right)=\dfrac{2}{u_k+1}-\dfrac{2}{u_{k+1}+1}.$\\
			Theo giả thiết quy nạp ta có:\\ $u_{k+1}>u_k \Rightarrow u_{k+1}+1>u_k+1 \Rightarrow \dfrac{2}{u_k+1}>\dfrac{2}{u_{k+1}+1}$.\\
			Vậy $u_{k+2}-u_{k+1}>0$.\\
			Do đó, $(*)$ đúng với mọi số nguyên dương $n$.
		\end{itemize}
		Vậy $(u_n)$ là dãy số tăng.}
\end{vd}
\subsubsection{Bài tập tự luận}
 
\begin{bt}[NB]%[1K2Y5-3]
	Xét tính tăng giảm của dãy số $(u_n)$ với $u_n=\dfrac{\sqrt{2}}{3^n}$.
	\dapso{dãy số giảm}
	\loigiai{
		Ta có $u_n>0, \forall n \in \mathbb{N}^\ast$.\\
		Xét thương $$\dfrac{u_{n+1}}{u_n}=\dfrac{\sqrt{2}}{3^{n+1}}: \dfrac{\sqrt{2}}{\sqrt{3^2}}=\dfrac{3^n}{3^{n+1}}=\dfrac{1}{3}<1.$$
		Vậy $\left(u_n\right)$ là dãy số giảm.
	}
\end{bt}
\begin{bt}[NB]%[1K2Y5-3]
	Xét tính tăng giảm của dãy số $\left(u_n\right)$ với $u_n=\dfrac{1}{n(n+1)}$.
	\dapso{dãy số tăng}
	\loigiai{
		Ta có $u_n=\dfrac{1}{n(n+1)}=\dfrac{1}{n}-\dfrac{1}{n+1}$.
		Xét hiệu:
		$$
		\begin{aligned}
			u_{n+1}-u_n & =\left(\dfrac{1}{n}-\dfrac{1}{n+1}\right)-\left(\dfrac{1}{n+1}-\dfrac{1}{n+2}\right) \\
			& =\dfrac{1}{n}-\dfrac{1}{n+2}>0, \forall n \in \mathbb{N}^\ast
		\end{aligned}
		$$
		Vậy  $\left(u_n\right)$ là dãy số tăng.
	}
\end{bt}
\begin{bt}[TH]%[1K2B5-3]
	Xét tính tăng giảm của dãy số $\left(u_n\right)$ với $u_n=n+\cos ^2 n$.
	\dapso{dãy số tăng}
	\loigiai{
		Xét hiệu
		$$
		\begin{aligned}
			u_{n+1}-u_n & =\left(n+1+\cos ^2(n+1)\right)-\left(n+\cos ^2 n\right) \\
			& =1+\cos ^2(n+1)-\cos ^2 n \\
			& =\cos ^2(n+1)+\sin ^2 n>0, \forall n \in \mathbb{N}^\ast .
		\end{aligned}
		$$
		Vậy $\left(u_n\right)$ là dãy số tăng.
	}
\end{bt}
\begin{bt}[TH]%[1K2B5-3]
	Xét tính tăng giảm của dãy số $(u_n)$ với $u_n=\dfrac{1}{n+1}+\dfrac{1}{n+2}+\ldots+\dfrac{1}{2n}$.
	\dapso{dãy số giảm}
	\loigiai{
		Xét hiệu\\
		$\begin{aligned}
			u_{n+1}-u_n&=\left(\dfrac{1}{n+2}+\dfrac{1}{n+3}+\ldots+\dfrac{1}{2(n+1)}\right)-\left(\dfrac{1}{n+1}+\dfrac{1}{n+2}+\ldots+\dfrac{1}{2n}\right)\\
			&=\dfrac{1}{n+2}-\dfrac{1}{2n+1}-\dfrac{1}{2n+2}\\
			&=\dfrac{1}{2n+2}-\dfrac{1}{2n+1}<0, \forall n\in \mathbb{N}^\ast.
		\end{aligned}$\\
		Vậy $(u_n)$ là dãy số giảm.
	}
	
\end{bt}

\begin{bt}[TH]%[1K2B5-3]
	Xét tính tăng giảm của dãy số $\left(u_n\right)$ với $u_n=\dfrac{1}{n+1}+\dfrac{1}{n+2}+\ldots+\dfrac{1}{2 n}$.
	\dapso{dãy số giảm}
	\loigiai{
		Xét hiệu
		$$
		\begin{aligned}
			u_{n+1}-u_n & =\left(\dfrac{1}{n+2}+\dfrac{1}{n+3}+\ldots+\dfrac{1}{2(n+1)}\right)-\left(\dfrac{1}{n+1}+\dfrac{1}{n+2}+\ldots+\dfrac{1}{2 n}\right) \\
			& =\dfrac{1}{n+2}-\dfrac{1}{2 n+1}-\dfrac{1}{2 n+2} \\
			& =\dfrac{1}{2 n+2}-\dfrac{1}{2 n+1}<0, \forall n \in \mathbb{N}^\ast
		\end{aligned}
		$$
		Vậy $\left(u_n\right)$ là dãy số giảm.
	}
\end{bt}
\begin{bt}[VD]%[1K2K5-3]
	Xét tính tăng giảm của dãy số $\left(u_n\right)$ cho bởi
	$$
	\left(u_n\right)\colon\heva{
		&u_1=1 ; u_2=2 \\
		&u_{n+1}=\sqrt{u_n}+\sqrt{u_{n-1}} \forall n \geq 2
	}
	$$
	\dapso{dãy số tăng}
	\loigiai{
		Ta chứng minh dãy $\left(u_n\right)$ là dãy tăng bằng phương pháp quy nạp.\\
		Dễ thấy $u_1<u_2<u_3$.\\
		Giả sử $u_{k-1}<u_k ~\forall k \geq 2$, ta chứng minh $u_{k+1}>u_k$.\\
		Thật vậy ta có $u_{k+1}=\sqrt{u_k}+\sqrt{u_{k-1}}>\sqrt{u_{k-1}}+\sqrt{u_{n-2}}=u_k$.\\ Vậy $\left(u_n\right)$ là dãy tăng.
	}
	
\end{bt}
\begin{bt}[VD]%[1K2K5-3]
	Cho dãy số $\left(u_n\right)$ biết $u_n=\dfrac{b \cdot2 n^2+1}{n^2+3}$ và $b \in \mathbb{R}$. Hãy xác định $b$ để
	\begin{listEX}[2]
		\item $\left(u_n\right)$ là dãy số giảm.
		\item $\left(u_n\right)$ là dãy số tăng.
	\end{listEX}
	\dapso{$b<\dfrac{1}{6}$ dãy số giảm; $b>\dfrac{1}{6}$ dãy số tăng}
	\loigiai{
		Ta có
		$$
		u_n=2 b+\dfrac{1-6 b}{n^2+3}
		$$
		Xét hiệu $$u_{n+1}-u_n=\dfrac{1-6 b}{(n+1)^2+3}-\dfrac{1-6 b}{n^2+3}=(1-6 b) \cdot\left(\dfrac{1}{(n+1)^2+3}-\dfrac{1}{n^2+3}\right)=A_n.$$
		\begin{listEX}
			\item Để $\left(u_n\right)$ là dãy sỗ giảm thì $A_n<0, \forall n \in \mathbb{N}^\ast$.
			$$
			A_n<0 \Leftrightarrow 1-6 b>0 \Leftrightarrow b<\dfrac{1}{6}
			$$
			\item Để $\left(u_n\right)$ là dãy số tăng thì $A_n>0, \forall n \in \mathbb{N}^\ast$.
			$$
			A_n>0 \Leftrightarrow 1-6 b<0 \Leftrightarrow b>\dfrac{1}{6}.
			$$
		\end{listEX}
		
	}
\end{bt}
\begin{bt}[VDC]%[1K2G5-3]
	Xét tính tăng giảm của dãy số $\left(u_n\right)$ với $u_n=\sin n+\cos n$.
	\dapso{dãy khōng tăng, không giảm}
	\loigiai{
		Ta có: $u_n=\sin n+\cos n=\sqrt{2} \sin \left(n+\dfrac{\pi}{4}\right)$.
		Xét hiệu
		$$
		\begin{aligned}
			u_{n+1}-u_n&=\sqrt{2} \sin \left(n+1+\dfrac{\pi}{4}\right)-\sqrt{2} \sin \left(n+\dfrac{\pi}{4}\right) \\
			&=2 \sqrt{2} \cdot \cos \left(2 n+\dfrac{1}{2}+\dfrac{\pi}{4}\right) \cdot \sin \dfrac{1}{2}=A_n . \\
		\end{aligned}
		$$
		Với  $n=1, A_1>0$. Với  $n=100, A_{100}<100$ . \\
		Vậy $\left(u_n\right)$ là dãy khōng tăng, không giảm.
	}
\end{bt}
\subsubsection{Bài tập trắc nghiệm}
\Opensolutionfile{ans}[ans/ans-1K2-1-Dang3]
%Câu 1
\begin{ex}%[1K2Y5-3]
	Cho các dãy số sau. Dãy số nào là dãy số tăng?
	\choice
	{$1;1;1;1;1;1;\ldots $}
	{$1;\dfrac{1}{2};\dfrac{1}{4};\dfrac{1}{8};\dfrac{1}{16};\ldots $}
	{$1;-\dfrac{1}{2};\dfrac{1}{4};-\dfrac{1}{8};\dfrac{1}{16};\ldots $}
	{ \True $1;3;5;7;9;\ldots $}
	\loigiai {
		Xét đáp án $1;1;1;1;1;1;\ldots$ đây là dãy hằng nên không tăng không giảm.\\
		Xét đáp án $1;-\dfrac{1}{2};\dfrac{1}{4};-\dfrac{1}{8};\dfrac{1}{16};\ldots \Rightarrow {{u}_{1}}>{{u}_{2}}<{{u}_{3}}\Rightarrow $ loại.\\
		Xét đáp án $1;3;5;7;9;\ldots \Rightarrow {{u}_{n}}<{{u}_{n+1}},\,\,n\in {{\mathbb{N}}^{*}}\Rightarrow $ chọn.\\
		Xét đáp án $1;\dfrac{1}{2};\dfrac{1}{4};\dfrac{1}{8};\dfrac{1}{16};\ldots \Rightarrow {{u}_{1}}>{{u}_{2}}>{{u}_{3}}\ldots >{{u}_{n}}>\ldots \Rightarrow $ loại.}
	
\end{ex}
%%%%%%%%%%
%Câu 2
\begin{ex}%[1K2Y5-3]
	Với giá trị nào của $a$ thì dãy số $\left(u_n\right)$ với $u_n=\dfrac{a n-1}{n+2}, \forall n \geq 1$ là dãy số tăng?
	\choice
	{$a>2$}
	{$a<-2$}
	{\True $a>-\dfrac{1}{2}$}
	{$a<-\dfrac{1}{2}$}
	\loigiai{
		Ta có $u_n=a-\dfrac{1+2 a}{n+2}$.\\
		$u_{n+1}-u_n=(1+2 a)\left(\dfrac{1}{n+2}-\dfrac{1}{n+3}\right)$.\\
		Suy ra dãy số đã cho tăng khi $a>-\dfrac{1}{2}$.
	}
\end{ex}
%%%%%%%%%%%%
%Câu 3
\begin{ex}%[1K2Y5-3]
	Trong các dãy $\left(u_n\right)$ sau đây dãy nào là dãy số giảm ?
	\choice
	{$u_n=(-1)^n$}
	{$u_n=2^n$}
	{$u_n=3 n+1$}
	{\True $u_n=\dfrac{1}{3^n}$}
	\loigiai{
		Xét dãy số $\left(u_n\right)$ có $u_n=\dfrac{1}{3^n}$, ta thấy $u_n>0, \forall n \in \mathbb{N}^\ast$ và $\dfrac{u_{n+1}}{u_n}=\dfrac{\dfrac{1}{3^{n+1}}}{\dfrac{1}{3^n}}=\dfrac{1}{3}<1$ nên dãy số $\left(u_n\right)$ này là dãy số giảm.
	}
\end{ex}
%%%%%%%%%%
%Câu 4
\begin{ex}%[1K2B5-3]
	Trong các dãy số $({{u}_{n}} )$ cho bởi số hạng tổng quát ${{u}_{n}}$ sau, dãy số nào là dãy số tăng?
	\choice
	{${{u}_{n}}=\dfrac{1}{n}$}
	{${{u}_{n}}=\dfrac{1}{{{2}^{n}}}$}
	{${{u}_{n}}=\dfrac{n+5}{3n+1}$}
	{ \True ${{u}_{n}}=\dfrac{2n-1}{n+1}$ }
	\loigiai {
		Vì ${{2}^{n}};\,n$ là các dãy dương và tăng nên $\dfrac{1}{{{2}^{n}}};\,\,\dfrac{1}{n}$ là các dãy giảm, do đó loại các đáp án ${{u}_{n}}=\dfrac{1}{{{2}^{n}}}$ và ${{u}_{n}}=\dfrac{1}{n}$.\\
		Xét đáp án ${{u}_{n}}=\dfrac{n+5}{3n+1}\Rightarrow \heva{
			& {{u}_{1}}=\dfrac{3}{2} \\
			& {{u}_{2}}=\dfrac{7}{6} \\
		}\Rightarrow {{u}_{1}}>{{u}_{2}}\Rightarrow $ loại.\\
		Xét đáp án ${{u}_{n}}=\dfrac{2n-1}{n+1}=2-\dfrac{3}{n+1}\Rightarrow  {{u}_{n+1}}-{{u}_{n}}=3\left(\dfrac{1}{n+1}-\dfrac{1}{n+2}\right)>0\Rightarrow$ nhận.}
	
\end{ex}
%%%%%%%%%%
%%%%%%%%%%
%Câu 5
\begin{ex}%[1K2B5-3]
	Trong các dãy số $({{u}_{n}} )$ cho bởi số hạng tổng quát ${{u}_{n}}$ sau, dãy số nào là dãy số giảm?
	\choice
	{${{u}_{n}}={{n}^2}$}
	{${{u}_{n}}=\dfrac{3n-1}{n+1}$}
	{${{u}_{n}}=\sqrt{n+2}$}
	{ \True ${{u}_{n}}=\dfrac{1}{{{2}^{n}}}$}
	\loigiai {
		Vì ${{2}^{n}}$ là dãy dương và tăng nên $\dfrac{1}{{{2}^{n}}}$ là dãy giảm. \\
		Xét ${{u}_{n}}=\dfrac{3n-1}{n+1}\Rightarrow \heva{
			& {{u}_{1}}=1 \\
			& {{u}_{2}}=\dfrac{5}{3} \\
		}\Rightarrow {{u}_{1}}<{{u}_{2}},$ loại.\\
		Hoặc
		${{u}_{n+1}}-{{u}_{n}}=\dfrac{3n+2}{n+2}-\dfrac{3n-1}{n+1}=\dfrac{4}{(n+1 )(n+2 )}>0$ nên $({{u}_{n}} )$ là dãy tăng.\\
		Xét ${{u}_{n}}={{n}^2}\Rightarrow {{u}_{n+1}}-{{u}_{n}}={{(n+1 )}^2}-{{n}^2}=2n+1>0,$ loại.\\
		Xét ${{u}_{n}}=\sqrt{n+2}\Rightarrow {{u}_{n+1}}-{{u}_{n}}=\sqrt{n+3}-\sqrt{n+2}=\dfrac{1}{\sqrt{n+3}+\sqrt{n+2}}>0,$ loại.}
	
\end{ex}

%Câu 6

\begin{ex}%[1K2B5-3]
	Trong các dãy số $(u_{n})$ sau, hãy chọn dãy số tăng.
	\choice
	{\True $u_{n}=(-1)^{2n}(5^{n}+1)$, $n\in \mathbb N^*$}
	{$u_{n}=\dfrac{n}{n^{2}+1}$, $n\in \mathbb N^*$}
	{$u_{n}=(-1)^{n+1}\sin \dfrac{\pi}{n}$, $n\in \mathbb N^*$}
	{$u_{n}=\dfrac{1}{\sqrt{n+1}+n}$, $n\in \mathbb N^*$}
	\loigiai
	{
		Xét dãy số $(u_n)$ với $u_{n}=(-1)^{2n}(5^{n}+1)$, ta có
		\[u_{n+1}-u_n = (-1)^{2n+2}(5^{n+1}+1)-(-1)^{2n}(5^{n}+1) = 5^{n+1}+1-5^n-1 = 4\cdot 5^n>0, \forall n\in\mathbb{N}^\ast.\]
		Vậy dãy trên là dãy số tăng.\\
		Xét các dãy số còn lại
		\begin{itemize}
			\item Với $u_{n}=(-1)^{n+1}\sin \dfrac{\pi}{n}$ ta có $u_1=0$, $u_2=-1$ hay $u_1>u_2$. Vậy dãy số này không là dãy số tăng.
			\item Với $u_{n}=\dfrac{1}{\sqrt{n+1}+n}$ ta có $u_1=\sqrt{2}-1$, $u_2=2-\sqrt{3}$ hay $u_1>u_2$. Vậy dãy số này không là dãy số tăng.
			\item Với $u_{n}=\dfrac{n}{n^{2}+1}$ ta có $u_1=\dfrac{1}{2}$, $u_2=\dfrac{2}{5}$ hay $u_1>u_2$. Vậy dãy số này không là dãy số tăng.
		\end{itemize}
	}
\end{ex}
%%%%%%%%%%
%Câu 7
\begin{ex}%[1K2K5-3]
	Trong các dãy số $({{u}_{n}} )$ cho bởi số hạng tổng quát ${{u}_{n}}$ sau, dãy số nào là dãy số giảm?
	\choice
	{${{u}_{n}}=\dfrac{{{n}^2}+1}{n}$}
	{${{u}_{n}}={{(-1 )}^{n}}\cdot ({{2}^{n}}+1 )$}
	{\True ${{u}_{n}}=\sqrt{n}-\sqrt{n-1}\,$}
	{${{u}_{n}}=\sin n$}
	\loigiai {
		Xét ${{u}_{n}}=\sin n\Rightarrow  {{u}_{n+1}}-{{u}_{n}}=2\cos \left(n+\dfrac{1}{2} \right)\sin \dfrac{1}{2}$ có thể dương hoặc âm phụ thuộc $n$ nên đáp án sai. Hoặc dễ thấy $\sin n$ có dấu thay đổi trên ${{\mathbb{N}}^{*}}$ nên dãy $\sin n$ không tăng, không giảm.\\
		Xét ${{u}_{n}}=\dfrac{{{n}^2}+1}{n}=n+\dfrac{1}{n}\Rightarrow  {{u}_{n+1}}-{{u}_{n}}=1+\dfrac{1}{n+1}-\dfrac{1}{n}=\dfrac{{{n}^2}+n-1}{n(n+1 )}>0$ nên dãy đã cho tăng nên đáp án sai.\\
		Xét ${{u}_{n}}=\sqrt{n}-\sqrt{n-1}=\dfrac{1}{\sqrt{n}+\sqrt{n+1}},$ dãy $\sqrt{n}+\sqrt{n-1}>0$ là dãy tăng nên suy ra ${{u}_{n}}$ giảm. \\
		Xét ${{u}_{n}}={{(-1 )}^{n}}({{2}^{n}}+1 )$ là dãy thay dấu nên không tăng không giảm, nên đáp án đúng.\\
		Cách trắc nghiệm\\
		Xét ${{u}_{n}}=\sin n$ có dấu thay đổi trên ${{\mathbb{N}}^{*}}$ nên dãy này không tăng không giảm.\\
		Xét ${{u}_{n}}=\dfrac{{{n}^2}+1}{n}$, ta có $\heva{
			& n=1\to {{u}_{1}}=2 \\
			& n=2\to {{u}_{2}}=\dfrac{5}{2} \\
		}\Rightarrow {{u}_{1}}<{{u}_{2}}\Rightarrow {{u}_{n}}=\dfrac{{{n}^2}+1}{n}$ không giảm.\\
		Xét ${{u}_{n}}=\sqrt{n}-\sqrt{n-1}$, ta có $\heva{
			& n=1\to {{u}_{1}}=1 \\
			& n=2\to {{u}_{2}}=\sqrt{2}-1 \\
		}\Rightarrow {{u}_{1}}>{{u}_{2}}$ nên dự đoán dãy này giảm.\\
		Xét ${{u}_{n}}={{(-1 )}^{n}}({{2}^{n}}+1 )$ là dãy thay dấu nên không tăng không giảm.\\
		Cách CASIO.\\
		Các dãy $\sin n;\,\,{{(-1 )}^{n}}({{2}^{n}}+1 )$ có dấu thay đổi trên ${{\mathbb{N}}^{*}}$ nên các dãy này không tăng không giảm nên loại các đáp án này.\\
		Xét hai đáp án còn lại, ta chỉ cần kiểm tra một đáp án bằng chức năng $TABLE$.\\
		Chẳng hạn kiểm tra đáp án ${{u}_{n}}=\dfrac{{{n}^2}+1}{n}$, ta vào chức năng $TABLE$ nhập $F(X )=\dfrac{X^2+1}{X}$ với thiết lập $\text{Start}=1,\text{ End}=10,\text{ Step}=1$.\\
		Nếu thấy cột $F(X )$ các giá trị tăng thì loại ${{u}_{n}}=\dfrac{{{n}^2}+1}{n}$ nếu ngược lại nếu thấy cột $F(X )$ các giá trị giảm dần thị chọn ${{u}_{n}}=\dfrac{{{n}^2}+1}{n}$.}
	
\end{ex}
%%%%%%%%%%
%Câu 8
\begin{ex}%[1K2K5-3]
	Mệnh đề nào sau đây đúng?
	\choice
	{Dãy số ${{u}_{n}}=\dfrac{1}{n}-2$ là dãy tăng}
	{\True Dãy số ${{u}_{n}}=2n+\cos \dfrac{1}{n}$ là dãy tăng}
	{Dãu số ${{u}_{n}}=\dfrac{n-1}{n+1}$ là dãy giảm}
	{Dãy số ${{u}_{n}}={{(-1 )}^{n}}({{2}^{n}}+1 )$ là dãy giảm}
	\loigiai {
		Xét đáp án ${{u}_{n}}=\dfrac{1}{n}-2\Rightarrow {{u}_{n+1}}-{{u}_{n}}=\dfrac{1}{n+1}-\dfrac{1}{n}<0\Rightarrow $loại.\\
		Xét đáp án ${{u}_{n}}={{(-1 )}^{n}}({{2}^{n}}+1 )$ là dãy có dấu thay đổi nên không giảm nên loại.\\
		Xét đáp án ${{u}_{n}}=\dfrac{n-1}{n+1}=1-\dfrac{2}{n+1}\Rightarrow {{u}_{n+1}}-{{u}_{n}}=2\left(\dfrac{1}{n+1}-\dfrac{1}{n+2}\right)>0\Rightarrow $ loại.\\
		Xét đáp án ${{u}_{n}}=2n+\cos \dfrac{1}{n}\Rightarrow {{u}_{n+1}}-{{u}_{n}}=\left(2-\cos \dfrac{1}{n+1}\right)+\cos \dfrac{1}{n+2}>0$ chọn.}
	
\end{ex}
%%%%%%%%%%
%Câu 9
\begin{ex}%[1K2K5-3]
	Mệnh đề nào sau đây {\bf sai}?
	\haicot
	{Dãy số ${{u}_{n}}=\dfrac{1-n}{\sqrt{n}}$ là dãy giảm}
	{Dãy số ${{u}_{n}}=n+\sin ^2n$ là dãy tăng}
	{\True Dãy số ${{u}_{n}}={{\left(1+\dfrac{1}{n}\right)}^{n}}$ là dãy giảm}
	{Dãy số ${{u}_{n}}=2{{n}^2}-5$ là dãy tăng}
	\loigiai {
		Xét đáp án \\ ${{u}_{n}}=\dfrac{1-n}{\sqrt{n}}=\dfrac{1}{\sqrt{n}}-\sqrt{n}\Rightarrow {{u}_{n+1}}-{{u}_{n}}=\dfrac{1}{\sqrt{n+1}}-\dfrac{1}{\sqrt{n}}+\sqrt{n}-\sqrt{n+1}<0$ nên dãy $({{u}_{n}} )$ là dãy giảm nên đúng.\\
		Xét đáp án ${{u}_{n}}=2{{n}^2}-5$ là dãy tăng vì ${{n}^2}$ là dãy tăng nên đúng. \\
		Hoặc
		${{u}_{n+1}}-{{u}_{n}}=2(2n+1 )>0$ nên $({{u}_{n}} )$ là dãy tăng.\\
		Xét đáp án ${{u}_{n}}={{\left(1+\dfrac{1}{n}\right)}^{n}}={{\left(\dfrac{n+1}{n} \right)}^{n}}>0\Rightarrow\dfrac{{{u}_{n+1}}}{{{u}_{n}}}=\dfrac{n+2}{n+1}\cdot {{\left(\dfrac{n+2}{n}\right)}^{n}}>1\Rightarrow ({{u}_{n}} )$ là dãy tăng nên sai.\\
		Xét đáp án ${{u}_{n}}=n+\sin ^2n\Rightarrow {{u}_{n+1}}-{{u}_{n}}=(1-\sin ^2(n+1 ) )+\sin ^2n>0$.}
	
\end{ex}
%%%%%%%%%%
%Câu 10%VDC 
\begin{ex}%[Nguyễn Long]%[1K2G5-3]
	Cho dãy $(u_n)\colon\heva{&u_1=1\\&u_{n+1}=\dfrac{n}{2(n+1)}u_n+\dfrac{3(n+2)}{2(n+1)}},n \in \mathbb{N^*}$. Nhận xét nào sau đây đúng
	\choice
	{\True Dãy số $(u_n)$ là dãy số tăng}
	{Dãy số $(u_n)$ là dãy số giảm}
	{Dãy số $(u_n)$ là dãy số không tăng, không giảm}
	{Tất cả các đáp án còn lại đều sai}
	\loigiai{ Ta chứng minh quy nạp $u_n<3, \forall n \in N^*$.\\
		Giả sử mđ đúng với $\mathrm{n}=\mathrm{k}$ khi đó có:
		$$
		u_{k+1}=\dfrac{k}{2(k+1)} u_k+\dfrac{3(k+2)}{2(k+1)}<\dfrac{3 k}{2(k+2)}+\dfrac{3(k+2)}{2(k+1)}=3 .
		$$
		Vậy mệnh đề đúng với $\mathrm{n}=\mathrm{k}+1$.
		Từ đó ta có $$u_{n+1}-u_n=\dfrac{\left(3-u_n\right)(n+2)}{n+1}>0.$$
		Vậy dãy $\left(u_n\right)$ tăng }
\end{ex}
\Closesolutionfile{ans}
\begin{indapan}{10}
	{ans/ans-1K2-1-Dang3}
\end{indapan}

\begin{dang}{Xét tính bị chặn của dãy số}
	\begin{itemize}
		\item Để chứng minh dãy số $(u_n)$ bị chặn trên bởi $M$, ta chứng minh $u_n\le M$, $\forall n\in\mathbb{N}^\ast$.
		\item Để chứng minh dãy số $(u_n)$ bị chặn dưới bởi $m$, ta chứng minh $u_n\ge m$, $\forall n\in\mathbb{N}^\ast$.
		\item Để chứng minh dãy số bị chặn ta chứng minh nó bị chặn trên và bị chặn dưới.
		\begin{itemize}
			\item Nếu dãy số $(u_n)$ tăng thì bị chặn dưới bởi $u_1$.
			\item Nếu dãy số $(u_n)$ giảm thì bị chặn trên bởi $u_1$.
		\end{itemize}
	\end{itemize}
\end{dang}
\subsubsection{Ví dụ minh hoạ}

%ví dụ 1
\begin{vd}[NB]%[1K2Y5-4]%[Trương Đăng Khoa]
	Chứng minh rằng dãy số $(u_n)$ với $u_n=\dfrac{3n}{n^2+9}$ bị chặn trên bởi $\dfrac{1}{2}$.
	\dapso{dãy số đã cho bị chặn trên bởi $\dfrac{1}{2}$}
	\loigiai{
		Với mọi $n\ge 1$, ta có $\dfrac{3n}{n^2+9}\le\dfrac{1}{2}\Leftrightarrow n^2+9\le 6n\Leftrightarrow(n-3)^2\le 0$ (đúng).\\
		Vậy dãy số đã cho bị chặn trên bởi $\dfrac{1}{2}$.
	}
\end{vd}

%ví dụ 2
\begin{vd}[NB]%[1K2Y5-4]%[Trương Đăng Khoa]
	Chứng minh rằng dãy số $(u_n)$ xác đinh bởi $u_n=\dfrac{8n+3}{3n+5}$ là một dãy số bị chặn.
	\dapso{dãy số bị chặn}
	\loigiai{
		Ta có $u_n>0$, $\forall n\ge 1$. Suy ra dãy số bị chặn dưới.\\
		Mặt khác $u_n=\dfrac{8n+3}{3n+5}<\dfrac{8n+3}{3n}=\dfrac{8}{3}+\dfrac{1}{n}<\dfrac{8}{3}+1=\dfrac{11}{3}$. Do đó dãy số bị chặn trên bởi $\dfrac{11}{3}$.\\
		Vậy dãy số đã cho bị chặn.
	}
\end{vd}
%ví dụ 4
\begin{vd}[TH]%[1K2B5-4]%[Trương Đăng Khoa]
	Xét tính bị chặn của dãy số $\left(u_n\right)$ với $u_n=\dfrac{3n+1}{n+3}$.
	\dapso{dãy số bị chặn}
	\loigiai{
		Với $n\in \mathbb{N}^\ast$ ta có $u_n=\dfrac{3n+1}{n+3}>0$.\\
		Nên dãy $\left(u_n\right)$ bị chặn dưới bởi $0$.\\
		Mặt khác $u_n=\dfrac{3n+1}{n+3}=\dfrac{3n+9-8}{n+3}=3-\dfrac{8}{n+3}<3$, $\forall n\in\mathbb{N}^\ast$.\\
		Nên dãy $\left(u_n\right)$ bị chặn trên bởi $3$.\\
		Vậy dãy số $\left(u_n\right)$ bị chặn.
	}
\end{vd}
%ví dụ 3
\begin{vd}[VD]%[1K2K5-4]%[Trương Đăng Khoa]
	Cho dãy số $(u_n)$ xác định bởi $u_1=1$ và $u_{n+1}=\dfrac{u_n+2}{u_n+1}$, $\forall n\ge 1$. Chứng minh rằng dãy $(u_n)$ bị chặn trên bởi sô $\dfrac{3}{2}$ và bị chặn dưới bởi số $1$.	
	\loigiai{
		Ta chứng minh $1\le u_n\le\dfrac{3}{2},\forall n\ge 1$ bằng phương pháp quy nạp.
		\begin{itemize}
			\item Với $n=1$ ta có $1\le u_1\le\dfrac{3}{2}$.
			\item Giả sử $1\le u_n\le\dfrac{3}{2}$ với mọi $n=k\ge 1$, tức là $1\le u_k\le\dfrac{3}{2}$. Ta cần chứng minh $1\le u_{k+1}\le\dfrac{3}{2}$.
		\end{itemize}
		Thật vậy 
		$u_{k+1}=1+\dfrac{1}{u_k+1}$.\\
		Vì $u_k+1>0$ nên $u_{k+1}=1+\dfrac{1}{u_k+1}>1$.\\
		Vì $u_k+1\ge 2$ nên $u_{k+1}=1+\dfrac{1}{u_k+1}\le 1+\dfrac{1}{2}=\dfrac{3}{2}$.\\
		Vậy $1\le u_n\le\dfrac{3}{2}$, $\forall n\ge 1$ hay dãy $(u_n)$ bị chặn trên bởi số $\dfrac{3}{2}$ và bị chặn dưới bởi số $1$.
	}
\end{vd}

%ví dụ 5
\begin{vd}[VD]%[1K2K5-4]%[Trương Đăng Khoa]
	Xét tính bị chặn của dãy số $\left(u_n\right)$ với $u_n=\sin n+ \cos n$.
	\dapso{dãy số bị chặn}
	\loigiai{
		Ta có $\begin{aligned}[t]
			&\ \sin n+\cos n \\
			=&\ \sqrt{2}\left(\dfrac{1}{\sqrt{2}}\sin n+\dfrac{1}{\sqrt{2}}\cos n\right)\\
			=&\ \sqrt{2}\left(\sin n\cdot\cos \dfrac{\pi}{4}+\cos n\cdot\sin \dfrac{\pi}{4}\right)\\
			=&\ \sqrt{2}\sin \left(n+\dfrac{\pi}{4}\right).
		\end{aligned}$\\
		Vì $\begin{aligned}[t]
			&\ -1\leq \sqrt{2}\sin \left(n+\dfrac{\pi}{4}\right) \leq 1\\
			\Rightarrow&\ -\sqrt{2}\leq \sqrt{2} \sin \left(n+\dfrac{\pi}{4}\right)\leq \sqrt{2}\\
			\Rightarrow&\ -\sqrt{2}\leq \sin n+\cos n \leq \sqrt{2},\ \forall n\in\mathbb{N}^\ast\\
			\Rightarrow&\ -\sqrt{2}\leq u_n \leq \sqrt{2},\ \forall n\in\mathbb{N}^\ast.
		\end{aligned}$\\
		Vậy dãy số $\left(u_n\right)$ là dãy số bị chặn.}
\end{vd}
\subsubsection{Bài tập tự luận}
 
%Bài 1 
\begin{bt}[TH]%[1K2B5-4]%[Trương Đăng Khoa]
	Xét tính bị chặn của các dãy số sau 
		\begin{listEX}[3]
			\item $u_n=\dfrac{1}{2n^2-1}$.
			\item 
			$u_n=3\cdot\cos\dfrac{n x}{3}$. 
			\item  $u_n=2n^3+1$.
			\item  $u_n=\dfrac{n^2+2n}{n^2+n+1}$.
			\item  $u_n=n+\dfrac{1}{n}$.
		\end{listEX}
	\loigiai{
		\begin{enumerate}
			\item  $u_n=\dfrac{1}{2n^2-1}$.\\
			Ta có $2n^2-1\ge 1\Rightarrow u_n=\dfrac{1}{2n^2-1}\le 1$, $\forall n\ge 1$.\\
			Vậy dãy số bị chặn trên bởi $1$.\\
			\item $u_n=3\cdot\cos\dfrac{n x}{3}$ có $-1\le\cos\dfrac{n x}{3}\le 1\Rightarrow-3\le 3\cdot\cos\dfrac{n x}{3}\le 3$.\\
			Vậy dãy số bị chặn dưới bởi $-3$ và chặn trên bởi $3$.
			\item  $u_n=2n^3+1$ có $2n^3+1\ge 3$, $\forall n\ge 1$.\\
			Vậy dãy số bị chặn dưới bởi $3$.
			\item $u_n=\dfrac{n^2+2n}{n^2+n+1}$ có $u_n=\dfrac{n^2+2n}{n^2+n+1}=1+\dfrac{n-1}{n^2+n+1}\ge 1$, $\forall n\ge 1$.\\
			Vậy dãy số bị chặn dưới bởi $1$.
			\item  $u_n=n+\dfrac{1}{n}$ có $u_n=n+\dfrac{1}{n}\ge 2\sqrt{n\cdot\dfrac{1}{n}}=2$, $\forall n>0$.\\
			Vậy dãy số bị chặn bởi $2$.
		\end{enumerate}
	}
\end{bt}
%Bài 2
\begin{bt}[VD]%[1K2K5-4]%[Trương Đăng Khoa]
	Xét tính bị chặn của dãy số $(u_n)$ với:
	\begin{listEX}[3]	 
		\item $u_{n}=\dfrac{4}{n}-5$.
		\item $u_{n}=\dfrac{n+4}{n+2}$.
		\item $u_{n}=\dfrac{5}{n^2+1}+\dfrac{n+2}{n+1}+\cos n$.
	\end{listEX}
	\loigiai{
		\begin{enumerate} 
			\item $u_{n}=\dfrac{4}{n}-5$.\\
			Ta có $u_n=\dfrac{4}{n}-5 \le \dfrac{4}{1}-5=-1$, $\forall n \in \mathbb{N}^{*}$ suy ra dãy $(u_n)$ bị chặn trên bởi $-1$.\\
			Mặt khác $u_n=\dfrac{4}{n}-5 \ge -5 \,\, \forall n \in \mathbb{N}^{*}$ suy ra dãy $(u_n)$ bị chặn dưới bởi $-5$.\\
			Vậy dãy $(u_n)$ bị chặn.	
			\item $u_{n}=\dfrac{n+4}{n+2}$.\\
			Ta có $u_n=	\dfrac{n+4}{n+2}=1+\dfrac{2}{n+2}> 1$, $\forall n \in \mathbb{N}^{*}$ suy ra dãy $(u_n)$ bị chặn dưới bởi $1$.\\
			Mặt khác $u_n=\dfrac{n+4}{n+2}=1+\dfrac{2}{n+2} \le 1+\dfrac{2}{1+2}=\dfrac{3}{5}$, $\forall n \in \mathbb{N}^{*}$ suy ra dãy $(u_n)$ bị chặn trên bởi $\dfrac{3}{5}$.\\
			Vậy dãy $(u_n)$ bị chặn.
			\item $u_{n}=\dfrac{5}{n^2+1}+\dfrac{n+2}{n+1}+\cos n$.\\
			Ta có $u_n=	\dfrac{5}{n^2+1}+\dfrac{n+2}{n+1}+\cos n=\dfrac{5}{n^2+1}+1+\dfrac{1}{n+1}+\cos n<5$, $\forall n \in \mathbb{N}^{*}$.\\
			Suy ra dãy $(u_n)$ bị chặn trên bởi $5$.\\
			Mặt khác $u_n=	\dfrac{5}{n^2+1}+\dfrac{n+2}{n+1}+\cos n=\dfrac{5}{n^2+1}+1+\dfrac{1}{n+1}+\cos n>0$, $\forall n \in \mathbb{N}^{*}$.\\
			Suy ra dãy $(u_n)$ bị chặn trên bởi $0$.\\
			Vậy dãy $(u_n)$ bị chặn.
		\end{enumerate}			
	}		
\end{bt}
%Bài 3
\begin{bt}[VDC]%[1K2G5-4]%[Trương Đăng Khoa]
	Xét tính bị chặn của dãy số $u_n=\left(1+\dfrac{1}{n}\right)^n$, $n\in N^\ast$.
	\loigiai{
		Ta có $u_n=\left(1+\dfrac{1}{n}\right)^n>0$, $\forall n\in N^\ast$ nên $(u_n)$ bị chặn dưới $(1)$.\\
		Lại có $\begin{aligned}[t]
			u_n&\ =\left(1+\dfrac{1}{n}\right)^n=\displaystyle\sum\limits_{k=0}^n C_n^k\left(\dfrac{1}{n}\right)^k\\
			&\ =\displaystyle\sum\limits_{k=0}^n\left[\dfrac{n!}{k!\cdot(n-k)!\cdot n^k}\right]\\
			&\ =\displaystyle\sum\limits_{k=0}^n\left[\dfrac{1}{k!}\cdot\dfrac{(n-k+1)}{n}\cdot\dfrac{(n-k+2)}{n}\ldots\dfrac{(n-k+k)}{n}\right]\le\displaystyle\sum\limits_{k=0}^n\dfrac{1}{k!},\, n\in \mathbb{N}^\ast
		\end{aligned}$\\
		Mà $\begin{aligned}[t]
			\displaystyle\sum\limits_{k=0}^n\dfrac{1}{k!}&\ \le 1+1+\dfrac{1}{1\cdot2}+\dfrac{1}{2\cdot3}+\dfrac{1}{3\cdot4}+\ldots+\dfrac{1}{(n-1)\cdot n}\\
			&\ =2+\left(1-\dfrac{1}{2}\right)+\left(\dfrac{1}{2}-\dfrac{1}{3}\right)+\ldots+\left(\dfrac{1}{n-1}-\dfrac{1}{n}\right)\\
			&\ 
			=3-\dfrac{1}{n}<3,\, \forall n\in \mathbb{N}^\ast.
		\end{aligned}$\\
		Suy ra $u_n<3$, $\forall n\in \mathbb{N}^\ast$ nên dãy số $(u_n)$ bị chặn trên $(2)$.\\
		Từ $(1)$ và $(2)$  suy ra dãy số $(u_n)$ bị chặn.}
\end{bt}
%Bài 4
\begin{bt}[VD]%[1K2K5-4]%[Trương Đăng Khoa]
	Cho dãy số $(u_n)$ xác định bởi $u_1=0$ và $u_{n+1}=\dfrac{1}{2}u_n+4$, $ \forall n\geq 1$.
	\begin{enumerate}
		\item Chứng minh dãy $(u_n)$ bị chặn trên bởi số $8$.
		\item Chứng minh dãy $(u_n)$ tăng, từ đó suy ra dãy $(u_n)$ bị chặn.
	\end{enumerate}
	\loigiai{
		\begin{enumerate}
			\item Ta chứng minh $u_n\leq 8$ với mọi $n\geq 1$.
			\begin{itemize}
				\item Khi $n=1$, ta có $u_1=0 <8$.
				\item Giả sử $u_n\leq 8$ với $n=k\geq 1$, tức là $u_k\leq 8$.\\
				Ta cần chứng minh $u_{k+1}\leq 8$.\\
				Thật vậy, $u_{k+1}=\dfrac{1}{2}u_k+4\leq \dfrac{1}{2}\cdot 8+4\leq 8$.
			\end{itemize}
			Vậy $u_n\leq 8$ với mọi $n\geq 1$, hay $(u_n)$ bị chặn trên bởi $8$.
			\item Với mọi $n\geq 1$, ta có $u_{n+1}-u_n=4-\dfrac{1}{2}u_n$. Mà $u_n\leq 8$ nên $u_{n+1}-u_n\geq 0$.\\
			Suy ra $u_n$ là dãy số tăng. Do đó $(u_n)$ bị chặn dưới bởi $u_1=0$.\\
			Kết hợp với câu a, ta được dãy số $(u_n)$ bị chặn.
		\end{enumerate}
	}
\end{bt}
%Bài 5
\begin{bt}[VD]%[1K2K5-4]%[Trương Đăng Khoa]
	Trong các dãy số $(u_n)$ sau, dãy số nào bị chặn trên, bị chặn dưới và bị chặn?
	\begin{listEX}[3]
		\item $u_n=n^2+5$.
		\item $u_n=\dfrac{3n+1}{2n+5}$.
		\item $u_n=(-1)^n\cos \dfrac{\pi}{2n}$.
		\item $u_n=\dfrac{n^2+2n}{n^2+n+1}$.
		\item $u_n=\dfrac{n}{\sqrt{n^2+2n}+n}$.
	\end{listEX}
	\loigiai{
		\begin{enumerate}
			\item Dãy số bị chặn dưới bởi $6$, không bị chặn trên.
			\item Dãy $(u_n)$ bị chặn dưới bởi $0$. Vì $u_n<\dfrac{3n+1}{2n}=\dfrac{3}{2}+\dfrac{1}{2n}<\dfrac{3}{2}+1=\dfrac{5}{2}$ nên dãy số bị chặn trên bởi $\dfrac{5}{2}$. Vậy dãy số bị chặn.
			\item Ta có $|u_n|\leq 1$ nên dãy số bị chặn trên bởi 1, bị chặn dưới bởi $-1$.
			\item Dãy số bị chặn dưới bởi $0$. Vì $u_n<\dfrac{n^2+2n}{n^2}=1+\dfrac{2}{n}\leq 3$ nên dãy số bị chặn trên. Vậy dãy số bị chặn.
			\item Ta có $0<u_n\leq 1$ vậy dãy số bị chặn.
		\end{enumerate}
	}
\end{bt}

\subsubsection{Câu hỏi trắc nghiệm}

\Opensolutionfile{ans}[ans/ans-1K2-1-Dang4]
%Câu 1
\begin{ex}%[1K2K5-4]%[Trương Đăng Khoa]
	Cho dãy số $(u_n)$ xác định bởi $u_1=3$ và $u_{n+1}=\dfrac{u_n+1}{2}$, $\forall n\geq 1$. Mệnh đề nào sau đây là đúng?
	\choice
	{\True Dãy số bị chặn}
	{Dãy số bị chặn trên}
	{Dãy số bị chặn dưới}
	{Dãy số không bị chặn}
	\loigiai{Ta chứng minh $u_n>1, \forall n\geq 1$ bằng phương pháp quy nạp.\\
		Suy ra dãy số bị chặn dưới bởi $1$.\\
		Ta có
		$u_{n+1}-u_n=\dfrac{1-u_n}{2}<0$, $\forall n\geq 1$.\\
		Do đó dãy số này là dãy số giảm nên nó bị chặn trên bởi $u_1=3$.\\
		Vậy dãy số đã cho là dãy số bị chặn.		
	}
\end{ex}
%Câu 2
\begin{ex}%[1K2K5-4]%[Trương Đăng Khoa]
	Cho dãy số $(u_n)$ xác định bởi $u_1=\sqrt{2}$ và $u_{n+1}=\sqrt{2+u_n}$, $\forall n\geq 1$. Mệnh đề nào sau đây là đúng?
	\choice
	{Dãy số bị chặn trên}
	{Dãy số bị chặn dưới}
	{\True Dãy số bị chặn}
	{Dãy số không bị chặn}
	\loigiai{Vì $u_n\geq 0$, $\forall n\geq 1$ nên dãy số bị chặn dưới bởi $0$.\\
		Ta chứng minh $u_n\geq 2, \forall n\geq 1$. Suy ra dãy số bị chặn trên bởi $2$.\\
		Vậy dãy số đã cho là dãy số bị chặn.		
	}
\end{ex}
%Câu 3
\begin{ex}%[1K2K5-4]%[Trương Đăng Khoa]
	Xét tính bị chặn của  dãy số $(u_n)$ với $u_n=\dfrac{1}{1\cdot2}+\dfrac{1}{2\cdot3}+\ldots+\dfrac{1}{n\cdot(n+1)}$.
	\choice{Không bị chặn}{Bị chặn trên}{Bị chặn dưới}{\True Bị chặn}
	\loigiai{
		Ta có $u_n=1-\dfrac{1}{2}+\dfrac{1}{2}-\dfrac{1}{3}+\ldots+\dfrac{1}{n}-\dfrac{1}{n+1}=1-\dfrac{1}{n+1}$.\\
		Do đó $0\leq u_n \leq 1$, $\forall n\geq 1$.\\
		Vậy dãy số đã cho bị chặn.
	}
\end{ex}
%Câu 4
\begin{ex}%[1K2G5-4]%[Trương Đăng Khoa]
	Cho dãy số $(u_n)$ với $u_n=\dfrac{1}{1\cdot4}+\dfrac{1}{2\cdot5}+\ldots+\dfrac{1}{n\cdot(n+3)}$. Dãy số $\left(u_n\right)$ bị chặn dưới và chặn trên lần lượt bởi các số $m$ và $M$ nào dưới đây?
	\choice
	{$m=0$, $M=1$}
	{$m=1$, $M=\dfrac{1}{2}$}
	{$m=1$, $M=\dfrac{10}{19}$}
	{\True $m=0$, $M=\dfrac{11}{18}$}
	\loigiai{
		Rõ ràng $u_n>0$, $\forall n\in\mathbb{N}^\ast$ nên $(u_n)$ bị chặn dưới.\\
		Mặt khác $\dfrac{1}{k(k+3)}=\dfrac{1}{3}\left(\dfrac{1}{k}-\dfrac{1}{k+3}\right)$.\\
		Suy ra $\begin{aligned}[t]
			u_n&\  =\dfrac{1}{3}\bigg[\left(1-\dfrac{1}{4}\right)+\left(\dfrac{1}{2}-\dfrac{1}{5}\right)+\left(\dfrac{1}{3}-\dfrac{1}{6}\right)+\left(\dfrac{1}{4}-\dfrac{1}{7}\right)+\\
			&\  \ldots+\left(\dfrac{1}{n-3}-\dfrac{1}{n}\right)+\left(\dfrac{1}{n-2}-\dfrac{1}{n+1}\right)+\left(\dfrac{1}{n-1}-\dfrac{1}{n+2}\right)+\left(\dfrac{1}{n}-\dfrac{1}{n+3}\right)\bigg]\\
			&\ = \dfrac{1}{3}\left(1+\dfrac{1}{2}+\dfrac{1}{3}-\dfrac{1}{n+1}-\dfrac{1}{n+2}-\dfrac{1}{n+3}\right)<\dfrac{11}{18}, \, \forall n\in\mathbb{N}^\ast.
		\end{aligned}$\\
		Do đó $(u_n)$ bị chặn trên.\\
		Vậy $m=0$, $M=\dfrac{11}{18}$.
	}
\end{ex}
%Câu 5
\begin{ex}%[1K2G5-4]%[Trương Đăng Khoa]
	Cho dãy số $(u_n)$ biết $u_n=\dfrac{1\cdot 3\cdot 5\ldots(2n-1)}{2\cdot 4\cdot 6\cdot 2n}$. Dãy số $\left(u_n\right)$ bị chặn dưới và chặn trên lần lượt bởi các số $m$ và $M$. Tính giá trị biểu thức $m+M$?
	\choice{$\dfrac{1}{\sqrt{2}}$}{\True $\dfrac{1}{\sqrt{3}}$}{$\dfrac{1}{\sqrt{5}}$}{$\dfrac{1}{\sqrt{7}}$}
	\loigiai{
		Xét $ \dfrac{2 k-1}{2 k}<\dfrac{2 k-1}{\sqrt{4 k^2-1}}
		=\dfrac{\sqrt{(2 k-1)^2}}{\sqrt{(2 k-1)(2 k+1)}} =\dfrac{\sqrt{2 k-1}}{\sqrt{2 k+1}}$,  $\forall k \ge 1$.\\
		$\Rightarrow u_n<\dfrac{\sqrt{1}}{\sqrt{3}} \cdot\dfrac{\sqrt{3}}{\sqrt{5}} \cdot\dfrac{\sqrt{5}}{\sqrt{7}}\cdot \ldots \cdot \dfrac{\sqrt{2 n-1}}{\sqrt{2 n+1}}=\dfrac{1}{\sqrt{2 n+1}} \le\dfrac{1}{\sqrt{3}}$,  $\forall n \in\mathbb{N}^\ast$.\\
		$\Rightarrow 0<u_n<\dfrac{1}{\sqrt{3}}$, $\forall n \in\mathbb{N}^\ast$.\\
		Vậy $m+M=0+\dfrac{1}{\sqrt{3}}$.
	}
\end{ex}
%Câu 6
\begin{ex}%[1K2G5-4]%[Trương Đăng Khoa]
	Cho dãy số $(u_n)$, với $u_n=\dfrac{1}{2^2}+\dfrac{1}{3^2}+\ldots+\dfrac{1}{n^2}$, $\forall n=2;3;4;\ldots$. Khẳng định nào sau đây là đúng?
	\choice
	{\True Dãy số bị chặn}
	{Dãy số bị chặn trên}
	{Dãy số bị chặn dưới}
	{Dãy số không bị chặn}
	\loigiai{
		Ta có $u_n>0\Rightarrow(u_n)$ bị chặn dưới bởi $0$.\\
		Mặt khác $\dfrac{1}{k^2}<\dfrac{1}{(k-1) k}=\dfrac{1}{k-1}-\dfrac{1}{k}$, ($k\in\mathbb{N}^\ast$, $k\ge 2$) nên suy ra
		\begin{eqnarray*}
			u_n&<&\dfrac{1}{1 \cdot 2}+\dfrac{1}{2 \cdot 3}+\dfrac{1}{3 \cdot 4}+\cdots+\dfrac{1}{n(n+1)}\\
			&=&1-\dfrac{1}{2}+\dfrac{1}{2}-\dfrac{1}{3}+\dfrac{1}{2}-\dfrac{1}{4}+\cdots+\dfrac{1}{n}-\dfrac{1}{n+1}=1-\dfrac{1}{n+1}<1.
		\end{eqnarray*}
		Nên dãy $(u_n)$ bị chặn trên, do đó dãy $(u_n)$ bị chặn.
	}
\end{ex}
%Câu 7
\begin{ex}%[1K2G5-4]%[Trương Đăng Khoa]
	Cho dãy số $\left(u_n\right)$ và đặt $u_n= \displaystyle \sum_{k=1}^{n} a_k$ với $a_k=\dfrac{1}{4k^2-1}$. Mệnh đề nào sau đây là đúng?
	\choice{$0<u_n <1$}
	{$0\leq u_n\leq \dfrac{1}{2}$}
	{\True $0<u_n<\dfrac{1}{2}$}
	{$0\leq u_n\leq 1$}
	\loigiai{
		\begin{itemize}
			\item
			Ta có $a_k=\dfrac{1}{4k^2-1}=\dfrac{1}{(2k+1)(2k-1)}=\dfrac{1}{2}\cdot\dfrac{(2k+1)-(2k-1)}{(2k+1)(2k-1)}=\dfrac{1}{2}\cdot\left(\dfrac{1}{2k-1}-\dfrac{1}{2k+1}\right)$.\\
			\item Mặt khác $u_n=\displaystyle \sum_{k=1}^{n} a_k$.
			Do đó
			\begin{eqnarray*}
				&u_n&=\dfrac{1}{2}\cdot\left(\dfrac{1}{1}-\dfrac{1}{3}\right)+\dfrac{1}{2}\cdot\left(\dfrac{1}{3}-\dfrac{1}{5}\right)+\ldots + \dfrac{1}{2}\cdot \left(\dfrac{1}{2n-1}-\dfrac{1}{2n+1}\right)\\
				&&=\dfrac{1}{2}\left(\dfrac{1}{1}-\dfrac{1}{2n+1}\right)\\
				&&=\dfrac{1}{2}\cdot\dfrac{2n}{2n+1}=\dfrac{n}{2n+1}.
			\end{eqnarray*}
			\item 
			
			Với mọi $n \in \mathbb{N}^\ast$ thì $u_n>0$ nên dãy số $\left(u_n\right)$ bị chặn dưới.\\
			Ta lại có $u_n=\dfrac{1}{2}\cdot\left(1-\dfrac{1}{2n+1}\right)<\dfrac{1}{2}$.\\
			Vậy dãy số bị chặn.
		\end{itemize}
	}
\end{ex}
%Câu 8
\begin{ex}%[1K2G5-4]%[Trương Đăng Khoa]
	Cho dãy số $\left(u_n\right)$ và đặt $u_n= \displaystyle \sum_{k=1}^{n} a_k$ với $a_k=\dfrac{1}{k(k+4)}$.  Dãy số $\left(u_n\right)$ bị chặn dưới và chặn trên lần lượt bởi các số $m$ và $M$ nào sau đây?
	\choice
	{\True $m=0$, $M=\dfrac{25}{48}$}
	{$m=0$, $M=\dfrac{25}{12}$}
	{$m=1$, $M=\dfrac{1}{4}$}
	{$m=1$, $M=\dfrac{1}{2}$}
	\loigiai{
		Ta có $a_k=\dfrac{1}{k(k+4)}=\dfrac{1}{4}\cdot\dfrac{4}{k(k+4)}=\dfrac{1}{4}\cdot\dfrac{k+4-k}{k(k+4)}=\dfrac{1}{4}\cdot\left(\dfrac{1}{k}-\dfrac{1}{k+4}\right)$.\\
		Mặt khác $u_n=\displaystyle \sum_{k=1}^{n} a_k$.
		Do đó
		\begin{eqnarray*}
			&u_n&=\dfrac{1}{4}\cdot\left(\dfrac{1}{1}-\dfrac{1}{5}\right)+\dfrac{1}{4}.\left(\dfrac{1}{2}-\dfrac{1}{6}\right)+\ldots + \dfrac{1}{4}\cdot\left(\dfrac{1}{n}-\dfrac{1}{n+4}\right)\\
			&&=\dfrac{1}{4}\left(\dfrac{1}{1}+\dfrac{1}{2}+\dfrac{1}{3}+\dfrac{1}{4}-\dfrac{1}{n+1}-\dfrac{1}{n+2}-\dfrac{1}{n+3}-\dfrac{1}{n+4}\right)\\
			&&=\dfrac{1}{4}\left(\dfrac{25}{12}-\dfrac{1}{n+1}-\dfrac{1}{n+2}-\dfrac{1}{n+3}-\dfrac{1}{n+4}\right).
		\end{eqnarray*}
		Với mọi $n \in \mathbb{N}^\ast$ thì $u_n>0$ nên dãy số $\left(u_n\right)$ bị chặn dưới.\\
		Ta lại có $u_n=\dfrac{1}{4}\cdot\left(\dfrac{25}{12}-\dfrac{1}{n+1}-\dfrac{1}{n+2}-\dfrac{1}{n+3}-\dfrac{1}{n+4}\right)<\dfrac{1}{4}\cdot\dfrac{25}{12}=\dfrac{25}{48}$.\\
		Vậy $m=0$, $M=\dfrac{25}{48}$.
	}
\end{ex}
%Câu 9
\begin{ex}%[1K2G5-4]%[Trương Đăng Khoa]
	Xét tính bị chặn của dãy số $\left(u_n\right)$ và đặt $u_n=\displaystyle \sum_{k=1}^{n} a_k$ với $a_k=\dfrac{1}{k(k+1)}$.
	\choice{\True Bị chặn}{Bị chặn dưới}{Bị chặn trên}{Không bị chặn.}
	\loigiai{
		Ta có $a_k=\dfrac{1}{k(k+1)}=\dfrac{1}{k}-\dfrac{1}{k+1}$. Do đó\\
		$u_n=\displaystyle \sum_{k=1}^{n}a_k=\left(1-\dfrac{1}{2}\right)+\left(\dfrac{1}{2}-\dfrac{1}{3}\right)+\ldots+\left(\dfrac{1}{n-1}-\dfrac{1}{n}\right)+\left(\dfrac{1}{n}-\dfrac{1}{n+1}\right)=1-\dfrac{1}{n+1}=\dfrac{n}{n+1}$.\\
		Với mọi $n \in \mathbb{N}^*$ thì $u_n>0$ nên dãy số $\left(u_n\right)$ bị chặn dưới.\\
		Ta lại có $u_n=1-\dfrac{n}{n+1}<1$, $\forall n \in \mathbb{N}^\ast$ nên dãy số $\left(u_n\right)$ bị chặn trên.\\
		Vậy dãy số bị chặn.
	}
\end{ex}
%Câu 10
\begin{ex}%[1K2G5-4]%[Trương Đăng Khoa]
	Cho dãy số $(u_n)$, xác định bởi $\heva{&u_1=6\\&u_{n+1}=\sqrt{6+u_n},\, \forall n\in\mathbb{N}^\ast}$. Mệnh đề nào sau đây là đúng?
	\choice{
		$\sqrt{6}<u_n<2\sqrt{3}$	
	}
	{\True $\sqrt{6}\leq u_n\leq 2\sqrt{3}$}
	{$\sqrt{6}<u_n\leq 2\sqrt{3}$	}
	{$\sqrt{6}\geq u_n<2\sqrt{3}$	}
	\loigiai{
		Ta có 
		$\heva{&u_1=6\\
			&u_{n+1}=\sqrt{6+u_n}} \Rightarrow
		\heva{
			&u_1=6\\
			&u_{n+1} \ge 0 } \Rightarrow u_n \ge 0 \Rightarrow\heva{&u_1=6\\
			&u_{n+1}=\sqrt{6+u_n}\ge\sqrt{6}}
		\Rightarrow u_n \ge\sqrt{6}$.\\
		Ta chứng minh quy nạp $\heva{&u_n\le 2\sqrt{3}\\ &u_1\le 2\sqrt{3}\\ &u_k\le 2\sqrt{3}.}$\\
		$\Rightarrow u_{k+1}=\sqrt{6+u_{k+1}} \le\sqrt{6+2 \sqrt{3}}<\sqrt{6+6}=2 \sqrt{3}$.\\
		Vậy $\sqrt{6} \leq  u_n \leq 2\sqrt{3}$.
	}
\end{ex}
\Closesolutionfile{ans}
\begin{indapan}{10}
	{ans/ans-1K2-1-Dang3}
\end{indapan}

\begin{dang}{Toán thực tế về dãy số}
\end{dang}
\subsubsection{Ví dụ minh hoạ}
% \begin{vd}%[1T2B1-5]%[Trương Đăng Khoa]%Ví dụ 1
% 	Một chồng cột gỗ được xếp thành các lớp, hai lớp liên tiếp hơn kém nhau một cột gỗ.
% 	\begin{center}
% 		\begin{tikzpicture}[font=\footnotesize, line join=round, line cap=round, >=stealth,scale=0.8]
% 			\def\r{0.2}
% 			\def\n{25}
% 			\def\g{110}
% 			\fill[teal!50!green](-6*\r,-3*\r)rectangle(3.5*\n*\r,0.5*\n*\r);
% 			\fill[teal!50!green,opacity=0.25](3.5*\n*\r,3*\r)rectangle(-6*\r,2*\n*\r);
% 			\foreach \j in {0,...,12}{
% 				\pgfmathsetmacro{\m}{\n-\j}
% 				\foreach \i in{0,...,\m}{
% 					\fill[left color=orange, right color=teal!30,draw=brown](2*\i*\r,0)++(60:2*\j*\r)++(\g:\r)--++(\g-90:6)arc(\g:-60:\r)--++(\g-270:6)--cycle;
% 					\fill[orange!20!brown!40,draw=teal](2*\i*\r,0)++(60:2*\j*\r)circle(\r);
% 				}
% 			}
% 		\end{tikzpicture}
% 	\end{center}
% 	\begin{enumerate}
% 		\item  Gọi $u_1=25$ là số cột gỗ có ở hàng dưới cùng của chồng cột gỗ, $u_n$ là số cột gỗ có ở hàng thứ $n$ tính từ dưới lên trên. Xét tính tăng, giảm của dãy số này.
% 		\item  Gọi $v_1=14$ là số cột gỗ có ở hàng trên cùng của chồng cột gỗ, $v_n$ là số cột gỗ có ở hàng thứ $n$ tính từ trên xuống dưới. Xét tinh tăng, giảm của dãy số này.
% 	\end{enumerate}
% 	\loigiai{
% 		\begin{enumerate}
% 			\item Ta có $u_n=26-n>u_{n+1}=26-n-1=25-n$.\\
% 			Vậy dãy số $(u_n)$ là dãy số giảm.
% 			\item Ta có $v_n=13+n<v_{n+1}=13+n+1=14+n$.\\
% 			Vậy dãy số $(u_n)$ là dãy số tăng
% 		\end{enumerate}
% 	}
% \end{vd}

\begin{vd}%[1T2B1-5]%[Trương Đăng Khoa]%Ví dụ 2
	Trên lưới ô vuông, mỗi ô cạnh $1$ đơn vị, người ta vẽ $8$ hình vuông và tô màu khác nhau như hình vẽ. Tìm dãy số biểu diễn độ dài cạnh của $8$ hình vuông đó từ nhỏ đến lớn. Có nhận xét gì về dãy số trên?
	\begin{center}
		\begin{tikzpicture}[scale=0.8]
			\def\r{21}
			\def\hv(#1){
				\ifnum #1= 1\else
				\pgfmathsetmacro{\R}{250*rnd}
				\pgfmathsetmacro{\G}{250*rnd}
				\pgfmathsetmacro{\B}{250*rnd}
				\definecolor{mau}{RGB}{\R,\G,\B}
				\fill[mau!30](0,0)rectangle(\r,\r);
				\draw[red,line width=1pt] (0,0) arc(180:90:\r)(0,0)rectangle(\r,\r);
				\pgfmathtruncatemacro{\k}{#1-1}
				\begin{scope}[shift={(45:\r*sqrt(2))},rotate=-90,scale={(sqrt(5)-1)/2}]
					\hv(\k)
					\pgfmathsetmacro{\n}{int((1/(sqrt(5))*(((1+sqrt(5))/2)^(\k)-(1-(sqrt(5))/2)^(\k)+1)}
					\ifnum \k>1
					\path (\r/2,\r/2)node[scale=1.75]{\color{red}$\n$};
					\else
					\fi
				\end{scope}
				\fi
			}
			\begin{scope}[scale=0.35]
				\hv(9)
				\draw[teal](0,0)grid(34,21);
				\path(21/2,21/2)node[scale=2]{21};
			\end{scope}
		\end{tikzpicture}
	\end{center}
	\loigiai{
		\begin{multicols}{4}
			\begin{itemize}
				\item $u_1=1$.
				\item $u_2=1$.
				\item $u_3=2$.
				\item $u_4=3$.
				\item $u_5=5$.
				\item $u_6=8$.
				\item $u_7=13$.
				\item $u_8=21$.
			\end{itemize}
		\end{multicols}
		Ta có dãy số $\left(u_n\right)\colon\heva{& u_1=1\\ &u_2=1\\ &u_n=u_{n-1}-u_{n-2}.}$
	}
\end{vd}

\begin{vd}%[1C2K1-5]%[Trương Đăng Khoa]% Ví dụ 3
	Chị Mai gửi tiền tiết kiệm vào ngân hàng theo thể thức lãi kép như sau. Lần đầu chị gửi $100$ triệu đồng. Sau đó, cứ hết $1$ tháng chị lại gửi thêm vào ngân hàng $6$ triệu đồng. Biết lãi suất của ngân hàng là $0{,}5\%$ một tháng. Gọi $P_n$ (triệu đồng) là số tiền chị có trong ngân hàng sau $n$ tháng.
	\begin{enumerate}
		\item Tính số tiền chị có trong ngân hàng sau $1$ tháng.
		\item Tính số tiền chị có trong ngân hàng sau $3$ tháng.
		\item Dự đoán công thức của $P_n$ tính theo $n$.
	\end{enumerate}
	\loigiai{
		\begin{enumerate}
			\item Số tiền chị có trong ngân hàng sau $1$ tháng là $P_1=+100+100\cdot0{,}5\%+6=100{,}5+6$ (triệu đồng).
			\item Số tiền chị có trong ngân hàng sau 2 tháng là 
			\begin{eqnarray*}
				P_2&=&100{,}5+6+(100{,}5+6)\cdot 0{,}5\%+6\\
				&=&(100{,}5+6)(1+0{,}5\%)+6\\
				&=& 100{,}5(1+0{,}5\%)+6\cdot(1+0{,}5\%)+6\, (\text{triệu đồng}).
			\end{eqnarray*}
			Số tiền chị có trong ngân hàng sau $3$ tháng là
			\begin{eqnarray*}
				P_3&=&(100{,}5+6)(1+0{,}5 \%)+6+[(100{,}5+6)(1+0{,}5 \%)+6] \cdot 0{,}5 \%+6\\
				&=& 100{,}5 \cdot(1+0{,}5 \%)^2+6(1+0{,}5 \%)^2+6 \cdot(1+0{,}5 \%)+6 \text{(triệu đồng)}.
			\end{eqnarray*}
			\item Số tiền chị có trong ngân hàng sau $4$ tháng là
			\begin{eqnarray*}
				P_4&=&(100{,}5+6)(1+0{,}5 \%)^2+6 \cdot(1+0{,}5 \%)+6+\left[(100{,}5+6)(1+0{,}5 \%)^2+6 \cdot(1+0{,}5 \%)+6\right]\cdot 0{,}5 \%+6\\ 
				&=&100{,}5 \cdot(1+0{,}5 \%)^3+6 \cdot(1+0{,}5 \%)^3+6\cdot(1+0{,}5 \%)^2+6 \cdot(1+0{,}5 \%)+6\, (\text{triệu đồng}).
			\end{eqnarray*}
			Số tiền chị có trong ngân hàng sau $n$ tháng là
			$$P_n=100{,}5 \cdot(1+0{,}5 \%)^{n-1}+6\cdot(1+0{,}5 \%)^{n-1}+6\cdot(1+0{,}5 \%)^{n-2}+6 \cdot(1+0{,}5 \%)^{n-3}+\ldots+6$$ với mọi $n \in\mathbb{N}^\ast$.
		\end{enumerate}
	}
\end{vd}

\begin{vd}%[1K2K5-5]%[Trương Đăng Khoa]% Ví dụ 4
	Anh Thanh vừa được tuyển dụng vào một công ty công nghệ, được cam kết lương năm đầu sẽ là $200$ triệu đồng và lương mỗi năm tiếp theo sẽ được tăng thêm $25$ triệu đồng. Gọi $s_n$ (triệu đồng) là lương vào năm thứ $n$ mà anh Thanh làm việc cho công ty đó. Khi đó ta có
	$$s_1=200,\, s_n=s_{n-1}+25\, \text{với}\, n \ge 2.$$
	\begin{enumerate}
		\item Tính lương của anh Thanh vào năm thứ $5$ làm việc cho công ty.
		\item Chứng minh $(s_n)$ là dãy số tăng. Giải thích ý nghĩa thực tế của kết quả này. 
	\end{enumerate}
	\loigiai{
		\begin{enumerate}
			\item Ta có \begin{eqnarray*}
				s_2&=&s_1+25=200+25=225\\
				s_3&=&s_2+25=225+25=250\\
				s_4&=&s_3+25=250+25=275\\
				s_5&=&s_4+25=275+25=300. 
			\end{eqnarray*}
			Vậy lương của anh Thanh vào năm thứ $5$ làm việc cho công ty là $300$ triệu đồng.
			\item  Ta có $s_n=s_{n-1}+25\Leftrightarrow s_n-s_{n-1}=25>0$ với mọi $n\ge 2$, $n\in\mathbb{N}^\ast$.\\
			Tức là $s_n>s_{n-1}$ với mọi $n\ge 2$, $n\in\mathbb{N}^\ast$.\\
			Vậy $(s_n)$ là dãy số tăng.\\
			Điều này có nghĩa là mức lương hàng năm của anh Thanh tăng dần theo thời gian làm việc.
		\end{enumerate}
	}
\end{vd}

\begin{vd}%[1K2K5-5]%[Trương Đăng Khoa]%Ví dụ 5
	Ông An gửi tiết kiệm $100$ triệu đồng kì hạn $1$ tháng với lãi suất $6\%$ một năm theo hình thức tính lãi kép. Số tiền (triệu đồng) của ông An thu được sau $n$ tháng được cho bởi công thứC 
	$$A_n=100\left(1+\dfrac{0{,}06}{12}\right)^n.$$
	\begin{enumerate}
		\item Tìm số tiền ông An nhận được sau tháng thứ nhất, sau tháng thứ hai.
		\item Tìm số tiền ông An nhận được sau $1$ năm.
	\end{enumerate}
	\loigiai{
		\begin{enumerate}
			\item Số tiền ông An nhận được sau tháng thứ nhất là 
			$$A_1=100\left(1+\dfrac{0{,}06}{12}\right)^1=100{,}5\, \text{(triệu đồng)}.$$
			Số tiền ông An nhận được sau tháng thứ hai là 
			$$A_2=100\left(1+\dfrac{0{,}06}{12}\right)^2=101{,}0025\, \text{(triệu đồng)}.$$
			\item  Số tiền ông An nhận được sau $1$ năm ($12$ tháng) là 
			$$A_{12}=100\left(1+\dfrac{0{,}06}{12}\right)^{12} \approx 106{,}17\, \text{(triệu đồng)}.$$
		\end{enumerate}
	}
\end{vd}

\begin{vd}%[1K2G5-5]%[Trương Đăng Khoa]%Ví dụ 6
	Chị Hương vay trả góp một khoản tiền $100$ triệu đồng và đồng ý trả dần $2$ triệu đồng mỗi tháng với lãi suất $0{,}8\%$ số tiền còn lại của mỗi tháng.
	Gọi $A_n$, ($n\in\mathbb{N}$) là số tiền còn nợ (triệu đồng) của chị Hương sau $n$ tháng.
	\begin{enumerate}
		\item Tìm lần lượt $A_0$, $A_1$, $A_2$, $A_3$, $A_4$, $A_5$, $A_6$ đễ tính số tiền còn nợ của chị Hương sau $6$ tháng.
		\item  Dự đoán hệ thức truy hồi đối với dãy số $(A_n)$.
	\end{enumerate}
	\loigiai{
		\begin{enumerate}
			\item  Ta có $A_0=100$ (triệu đồng).
			\begin{itemize}
				\item Tiền lãi chị Hương phải trả sau $1$ tháng là $100\cdot 0{,}8\%=0{,}8$ (triệu đồng).\\
				Do đó, số tiền gốc chị Hương trả được sau $1$ tháng là $2-0{,}8=1{,}2$ (triệu đồng).\\
				Khi đó, số tiền còn nợ của chị Hương sau $1$ tháng là 
				$A_1=100-1{,}2=98{,}8$ (triệu đồng).
				\item  Tiền lãi chị Hương phải trả sau $2$ tháng là $98{,}8\cdot 0{,}8\%=0{,}7904$ (triệu đồng).\\
				Do đó, số tiền gốc chị Hương trả được sau $2$ tháng là $2-0{,}7904=1{,}2096$ (triệu đồng).\\
				Khi đó, số tiền còn nợ của chị Hương sau $2$ tháng là 
				$A_2=98{,}8-1{,}2096=97{,}5904$ (triệu đồng).
				\item Tiền lãi chị Hương phải trả sau $3$ tháng là $97{,}5904\cdot 0{,}8\%=0{,}7807232$ (triệu đồng).\\
				Do đó, số tiền gốc chị Hương trả được sau $3$ tháng là $2-0{,}7807232=1{,}2192768$ (triệu đồng).\\
				Khi đó, số tiền còn nợ của chị Hương sau $3$ tháng là 
				$A_3=97{,}5904-1{,}2192768=96{,}3711232$ (triệu đồng).
				\item Tiền lãi chị Hương phải trả sau $4$ tháng là $96{,}3711232\cdot 0{,}8\%\approx 0{,}77097$ (triệu đồng).\\
				Do đó, số tiền gốc chị Hương trả được sau $4$ tháng là $2-0{,}77097=1{,}22903$ (triệu đồng).\\
				Khi đó, số tiền còn nợ của chị Hương sau $4$ tháng là 
				$A_4=96{,}3711232-1{,}22903=95{,}1420932$ (triệu đồng).
				\item Tiền lãi chị Hương phải trả sau $5$ tháng là $95{,}1420932\cdot 0{,}8\%\approx 0{,}76114$ (triệu đồng).\\
				Do đó, số tiền gốc chị Hương trả được sau $5$ tháng là $2-0{,}76114=1{,}23886$ (triệu đồng).\\
				Khi đó, số tiền còn nợ của chị Hương sau $5$ tháng là $A_5=95{,}1420932-1{,}23886=93{,}9032332$ (triệu đồng).
				\item Tiền lãi chị Hương phải trả sau $6$ tháng là $93{,}9032332\cdot 0{,}8\%\approx 0{,}75123$ (triệu đồng).\\
				Do đó, số tiền gốc chị Hương trả được sau $6$ tháng là $2-0{,}75123=1{,}24877$ (triệu đồng).\\
				Khi đó, số tiền còn nợ của chị Hương sau $6$ tháng là $A_6=93{,}9032332-1{,}24877=92{,}6544632$ (triệu đồng).
			\end{itemize}
			\item Dự đoán hệ thức truy hồi đối với dãy số $(A_n)$ là 
			\[A_0=100,\, A_n=A_{n-1}-\left(2-A_{n-1} \cdot 0{,}8 \%\right)=1{,}008 A_{n-1}-2\]
		\end{enumerate}
	}
\end{vd}

% %%Bài 6. CSC
\def\tenchude{CẤP SỐ CỘNG}
\setcounter{section}{5}
\setcounter{dang}{0}
\setcounter{ex}{0}
\setcounter{bt}{0}
\setcounter{vd}{0}
\section{Cấp số cộng}
\subsection{Tóm tắt lý thuyết}
\begin{tomtat}
	\subsubsection{Định nghĩa}
	Dãy số là cấp số cộng nếu mỗi một số hạng (kể từ số hạng thứ hai) đều bằng tổng của số hạng đứng ngay trước nó với một số không đổi $ d $.\\
	Dãy số $ (u_n) $ là cấp số cộng $ \Leftrightarrow u_{n+1}=u_n+d $, $ \forall n \in \mathbb{N}^* $.\\
	$ d $ là số không đổi, gọi là \textbf{\textit{công sai}} của cấp số cộng.
	\subsubsection{Tính chất}
	Nếu $ (u_n) $ là cấp số cộng thì kể từ số hạng thứ hai (trừ số hạng cuối nếu là cấp số cộng hữu hạn) đều là trung bình cộng của hai số hạng đứng kề nó trong dãy. Tức là $$u_k=\dfrac{u_{k-1}+u_{k+1}}{2}, (\forall k\ge 2, k \in \mathbb{N}^*).$$
	\subsubsection{Số hạng tổng quát}
	Nếu cấp số cộng $ (u_n) $ có số hạng đầu $ u_1 $ và công sai $ d $ thì số hạng tổng quát $ u_n $ được xác định bởi công thức $$u_n=u_1+(n-1)d \text{ với $n\ge 2$}.$$
	\subsubsection{Tổng $ n $ số hạng đầu tiên}
	Cho cấp số cộng $ (u_n) $. Tổng $ n $ số hạng đầu tiên của cấp số cộng kí hiệu là $ S_n=u_1+u_2+\ldots+u_n $.\\
	Khi đó $ S_n $ được tính theo công thức $$ S_n=\dfrac{n(u_1+u_n)}{2}=\dfrac{n}{2}\left[ 2u_1+(n-1)d\right]. $$
\end{tomtat}
\subsection{Các dạng toán thường gặp}
\begin{dang}{Nhận diện cấp số cộng, công sai $ d $, số hạng tổng quát $ u_n $}
	% Dựa theo định nghĩa của cấp số cộng, để nhận diện $ (u_n) $ là cấp số cộng $ \Leftrightarrow u_{n+1}=u_n+d $.\\
	% Khi đó công sai $ d=u_{n+1}-u_{n} $, $ \forall n \in \mathbb{N}^* $.
\end{dang}
\subsubsection{Ví dụ minh hoạ}
\begin{vd}%[NB]%[DCHT Toán 11 - KNTT -Lê Hải Phụng] %[1K2Y6-1]
	Dãy số hữu hạn nào là một cấp số cộng? Vì sao?
	\begin{listEX}[2]
		\item  $-2$, $1$, $4$, $7$, $10$, $13$, $16$.
		\item  $ 1 $, $ -2 $, $ -4 $, $ -6 $, $ -8 $.
	\end{listEX}
	\dapso{ Dãy số 1 là một cấp số cộng, dãy số 2 không là một cấp số cộng.}
	\loigiai{
		\begin{enumerate}
			\item Ta thấy $ u_2=u_1+3 $  do $ 1=(-2)+3 $.\\
			Vì $ u_k=u_{k-1}+d,\ \forall k\geq2$ $\left(\ 1=\left(-2\right)+3;4=1+3;7=4+3;10=7+3;13=10+3;16=13+3\right) $ nên dãy số đã cho là cấp số cộng. 
			\item Ta thấy $ u_2=u_1+(-3) $  do $-2=1+(-3)$.\\
			Vì $ {u_3\neq u}_2+(-3) $ bởi $ \left(\ -4\neq-2+\left(-3\right)\right)\ $ nên dãy số đã cho không là cấp số cộng.
		\end{enumerate}
	}
\end{vd}
\begin{vd}%[TH]%[DCHT Toán 11 - KNTT -Lê Hải Phụng] %[1K2B6-1]
	Trong các dãy số dưới đây, dãy số nào là cấp số cộng? 
	\begin{listEX}[2]
		\item  Dãy số $\left({a_n}\right)$ với ${a_n}=4n-3$;
		\item  Dãy số $\left({c_n}\right)$ với ${c_n}={2018^n}$.
	\end{listEX}
	\dapso{Dãy số 1 là một cấp số cộng, dãy số 2 không là một cấp số cộng.}
	\loigiai{	
		\begin{enumerate}
			\item Ta có $a_{n+1}=4(n+1)-3=4n+1$ nên $a_{n+1}-a_n=(4n+1)-(4n-3)=4$,$\forall n\ge 1.$.\\
			Do đó $(a_n)$ là cấp số cộng.
			\item Ta có $c_{n+1}=2018^{n+1}$ nên $c_{n+1}-c_n=2018^{n+1}-2018^n=2017\cdot 2018^n$ (phụ thuộc vào giá trị của $n$).\\ 
			Suy ra $(c_n)$ không phải là một cấp số cộng.
		\end{enumerate}	
	}
\end{vd}
\begin{vd}%[NB]%[DCHT Toán 11 - KNTT -Lê Hải Phụng] %[1K2Y6-1]
	Cho cấp số cộng $(u_n)$  có công thức số hạng tổng quát $u_n=3n+1$, $n\in\mathbb{N}^\ast$ . Tìm số hạng đầu $u_1$ và công sai $d$?
	\dapso{$u_1=4 $, $d=3$.}
	\loigiai{
		Từ công thức số hạng tổng quát, ta có $ u_1=4 $, $u_2=7$ suy ra $d=u_2-u_1=3$.
	}
\end{vd}

\begin{vd}%[TH]%[DCHT Toán 11 - KNTT -Lê Hải Phụng] %[1K2B6-1]
	Cho cấp số cộng $(u_n)$ với $u_1=3$, $u_2=9$. Công sai của cấp số cộng đã cho bằng bao nhiêu?
	\dapso{$ d=6 $}
	\loigiai{
		Cấp số cộng $(u_n)$ có số hạng tổng quát là $u_n=u_1+(n-1)d$ với $n \ge 2$.\\
		Suy ra $u_2=u_1+d \Leftrightarrow 9=3+d \Leftrightarrow d=6$.\\
		Vậy công sai của cấp số cộng đã cho là $6$.
	}
\end{vd}
\begin{vd}%[VD]%[DCHT Toán 11 - KNTT -Lê Hải Phụng] %[1K2K6-1]
	Tính số hạng đầu $u_1$ và công sai $d$ của một cấp số cộng biết $u_4=10$ và $u_7=19$.
	\dapso{$ u_1=1 $, $ d=3 $.}
	\loigiai{Ta có $ \heva{& u_4=10 \\ & u_7=19} \Leftrightarrow \heva{& u_1+3d=10 \\ & u_1+6d=19} \Leftrightarrow \heva{& u_1=1 \\ & d=3.}$}
\end{vd}

\begin{vd}%[TH]%[Dự án DCHT-11-KNTT]%[Dao-V- Thuy]%[1K2B5-1]
	Xác định số hạng tổng quát của cấp số cộng $(u_n),$ biết $\heva{&u_7=8\\ &d=2.}$
	\dapso{$u_n=2n-6$}
	\loigiai{
		Ta có
		\begin{equation*}
			\heva{&u_7=8\\ &d=2} \Leftrightarrow \heva{&u_1+6d=8\\&d=2} \Leftrightarrow \heva{&u_1=-4\\ &d=2.}
		\end{equation*}
		Vậy công thức tổng quát của cấp số cộng
		\begin{center}
			$u_n=-4+(n-1)2 \Leftrightarrow u_n=2n-6 $ với $n \geq 2.$
		\end{center}	
	}
\end{vd}

% \begin{vd}%[TH]%[Dự án DCHT-11-KNTT]%[Dao-V- Thuy]%[1K2B5-1]
% 	Tìm số hạng đầu và công sai của cấp số cộng $(u_n)$, biết $\heva{&u_1+u_5-u_3=10\\ &u_1+u_6=17.}$
% 	\dapso{$u_1=16$, $d=-3$}
% 	\loigiai{
% 		Ta có
% 		\begin{align*}
% 			\heva{&u_1+u_5-u_3=10\\ &u_1+u_6=17} &\Leftrightarrow \heva{&u_1+u_1+4d-(u_1+2d)=10\\ &u_1+u_1+5d=17}\\ & \Leftrightarrow\heva{&u_1+2d=10 \\ &2u_1+5d=17} \Leftrightarrow \heva{&u_1=16 \\ &d=-3.}
% 		\end{align*}
% 		Vậy $u_1=16$, $d=-3$.
% 	}
% \end{vd}

\begin{vd}%[TH]%[Dự án DCHT-11-KNTT]%[Dao-V- Thuy]%[1K2B5-1]
	Cho cấp số cộng $(u_n)$ với $\heva{&u_1=-9\\ &u_{n-1}=u_n-5}$. Tìm số hạng tổng quát của cấp số cộng $(u_n)$.
	\dapso{$u_n= 5n-14$}
	\loigiai{
		Từ công thức $u_{n-1}=u_n-5 \Leftrightarrow u_n= u_{n-1}+5$, suy ra $d=5$.\\
		Vậy công thức tổng quát của cấp số cộng $(u_n)$ là $u_n=-9 + 5(n-1)= 5n-14$.
	}
\end{vd}

\begin{vd}%[TH]%[Dự án DCHT-11-KNTT]%[Dao-V- Thuy]%[1K2B5-1]
	Cho cấp số cộng $(u_n)$ có $u_{20}=-52$ và $u_{51}=-145$. Hãy tìm số hạng tổng quát của cấp số cộng đó.
	\dapso{$u_n= -3n+8$}
	\loigiai{
		Ta có
		\begin{eqnarray*}
			\heva{&u_{20}=-52 \\&u_{51}=-145} &\Leftrightarrow& \heva{&u_1+19d=-52\\ &u_1+50d=-145} \Leftrightarrow \heva{&u_1=5 \\ &d= -3.}
		\end{eqnarray*}
		Vậy số hạng tổng quát cần tìm là $u_n= u_1+ (n-1)d= 5+(n-1) \cdot (-3)= -3n+8$.
	}
\end{vd}

\begin{vd}%[VD]%[Dự án DCHT-11-KNTT]%[Dao-V- Thuy]%[1K2B5-1]
	Tìm số hạng đầu và công sai của cấp số cộng $(u_n)$, biết
		\begin{listEX}[2]
			\item $\heva{&u_9=5u_2\\ &u_{13}=2u_6+5.}$
			\item $\heva{&u_1-u_3+u_5=10\\ &u_1+u_6=7.}$
		\end{listEX}
	\dapso{$u_1=3$, $d=4$; $u_1=36$, $d=-13$}
	\loigiai
	{
		\begin{enumerate}
			\item Ta có
			\begin{eqnarray*}
				\heva{&u_9=5u_2\\ &u_{13}=2u_6+5} &\Leftrightarrow& \heva{&u_1+8d= 5 \left( u_1+d \right) \\ &u_1+12d = 2 \left( u_1+5d\right) + 5}\\
				&\Leftrightarrow& \heva{&-4u_1+3d=0\\ &-u_1+2d=5} \Leftrightarrow \heva{&u_1=3 \\ &d=4.}
			\end{eqnarray*}
			Vậy $u_1=3$, $d=4$.
			\item Ta có 
			\begin{eqnarray*}
				\heva{&u_1-u_3+u_5=10\\ &u_1+u_6=7} &\Leftrightarrow& \heva{&u_1-\left( u_1+2d\right) + \left( u_1+4d\right) = 10\\ &u_1+ \left( u_1+5d\right) = 7}\\
				&\Leftrightarrow& \heva{&u_1+2d=10\\ &2u_1+5d=7} \Leftrightarrow \heva{&u_1=36 \\ &d=-13.}
			\end{eqnarray*}
			Vậy $u_1=36$, $d=-13$.
		\end{enumerate}
	}
\end{vd}

\begin{vd}%[VD]%[Dự án DCHT-11-KNTT]%[Dao-V- Thuy]%[1K2K5-1]
	Tìm số hạng đầu và công sai của cấp số cộng $(u_n)$, biết
		\begin{listEX}[2]
			\item $\heva{&-u_3+u_7=8\\ &u_2u_7=75.}$
			\item $\heva{&u_5=4u_3\\ &u_2u_6=-11.}$
		\end{listEX}
	\dapso{$\heva{&u_1=3\\ &d=2} \text{ hoặc } \heva{&u_1=-17\\ &d=2}$; $\heva{&u_1=-4\\ &d=3}$ hoặc $\heva{&u_1=4\\ &d=-3}$}
	\loigiai{
		\begin{enumerate}
			\item Ta có
			\begin{eqnarray*}
				\heva{&-u_3+u_7=8\\ &u_2u_7=75} &\Leftrightarrow& \heva{&-\left( u_1+2d\right) + \left( u_1+6d\right) = 8\\ &\left( u_1+d\right) \left( u_1+6d\right) = 75}\\
				&\Leftrightarrow& \heva{&4d=8\\ &u_1^2+7u_1d+6d^2=75}\\
				&\Leftrightarrow& \heva{&d=2\\ &u_1^2+14u_1-51=0}\\
				&\Leftrightarrow& \heva{&u_1=3\\ &d=2} \text{ hoặc } \heva{&u_1=-17\\ &d=2.}
			\end{eqnarray*}
			Vậy $\heva{&u_1=3\\ &d=2} \text{ hoặc } \heva{&u_1=-17\\ &d=2.}$
			\item Ta có 
			\begin{eqnarray*}
				\heva{&u_5=4u_3\\ &u_2u_6=-11} &\Leftrightarrow& \heva{&u_1+4d=4 \left( u_1+2d\right)\\ &\left( u_1+d\right) \left( u_1+5d\right)= -11}\\
				&\Leftrightarrow& \heva{&3u_1+4d=0 &(1)\\ &u_1^2+6du_1+5d^2=-11 &(2)}
			\end{eqnarray*}
			Từ $(1)$ suy ra $3u_1=-4d$. Thay vào $(2)$ ta được
			\begin{eqnarray*}
				9u_1^2+54du_1+45d^2=-99 &\Leftrightarrow& 16d^2 -72d^2+45d^2=-99\\
				&\Leftrightarrow& -11d^2=-99 \Leftrightarrow \hoac{&d=3\\ &d=-3.}
			\end{eqnarray*}
			Với $d=3$, ta có $u_1=-4$.\\
			Với $d=-3$, ta có $u_1=4$.\\
			Vậy $\heva{&u_1=-4\\ &d=3}$ hoặc $\heva{&u_1=4\\ &d=-3.}$
		\end{enumerate}
	}
\end{vd}

\subsubsection{Bài tập tự luận}
 

\begin{bt}%[NB]%[DCHT Toán 11 - KNTT -Lê Hải Phụng] %[1K2Y6-1]
	Trong các dãy số sau, dãy số nào là một cấp số cộng?
	\begin{listEX}[1]
		\item $1$, $-3$, $-7$, $-11$, $-15$, $\ldots$;
		\item $1$, $-2$, $-4$, $-6$, $-8,$ $\ldots$.
		\item $ \dfrac{1}{2} $, $0$, $-\dfrac{1}{2}$, $-1$, $-\dfrac{3}{2}$, $\ldots$
	\end{listEX}
	\dapso{1) và 3) là cấp số cộng.}
	\loigiai{Ta lần lượt đi kiểm tra: $ u_2-u_1=u_3-u_2=u_4-u_3=\ldots $?\\
		Xét từng dãy số thì ta thấy 1) và 3) là cấp số cộng. 
	}
\end{bt}

\begin{bt}%[NB]%[DCHT Toán 11 - KNTT -Lê Hải Phụng] %[1K2Y6-1]
	Trong các dãy số sau, dãy nào là cấp số cộng. Tìm số hạng đầu và công sai của cấp số cộng đó.
	\begin{listEX}[2]
		\item Dãy số $ (u_n) $ với $ u_n=19n-5 $;
		\item Dãy số $ (u_n) $ với $ u_n=n^2+n+1 $. 
	\end{listEX}
	\dapso{Dãy số 1) $ (u_n) $ là một cấp số cộng với số hạng đầu là $ u_1=19\cdot1-5=14 $ và công sai $ d=19 $. Dãy số 2) không là một cấp số cộng.}
	\loigiai{
		\begin{enumerate}
			\item Dãy số $ (u_n) $ với $ u_n=19n-5 $.\\
			Ta có $ u_{n+1}-u_n=19(n+1)-5-(19n-5)=19 $. Vậy $ (u_n) $ là một cấp số cộng với số hạng đầu là $ u_1=19\cdot1-5=14 $ và công sai $ d=19 $.
			\item Dãy số $ (u_n) $ với $ u_n=n^2+n+1 $.\\
			Ta có $ u_{n+1}-u_n=(n+1)^2+(n+1)+1-(n^2+n+1)=2n+2$ phụ thuộc vào $ n $. Vậy $ (u_n) $ không là một cấp số cộng.
	\end{enumerate}}
\end{bt}

\begin{bt}%[TH]%[DCHT Toán 11 - KNTT -Lê Hải Phụng] %[1K2B6-1]
	Cho cấp số cộng $\left(u_n\right)$ với $u_1=3$, $u_2=9$. Công sai của cấp số cộng đã cho bằng bao nhiêu?
	\dapso{Công sai của cấp số cộng đã cho là 6.}
	\loigiai{Cấp số cộng $(u_n)$ có số hạng tổng quát là
		$u_n=u_1+\left(n-1\right)d$ với $n \ge 2$
		(số hạng đầu $u_1$ và công sai $d$)\\
		Suy ra $ u_2=u_1+d\Leftrightarrow9=3+d\Leftrightarrow d=6 $.\\
		Vậy công sai của cấp số cộng đã cho là 6.
	}
\end{bt}


\begin{bt}%[TH]%[Dự án DCHT-11-KNTT]%[Dao-V- Thuy]%[1K2B5-1]
	Xác định công thức tổng quát của cấp số cộng $(u_n)$, biết $\heva{&u_{11}=5\\ &d=-6.}$
	\loigiai{
		Ta có
		\begin{equation*}
			\heva{&u_{11}=5\\ &d=-6} \Leftrightarrow \heva{&u_1+10d=5\\ &d=-6} \Leftrightarrow \heva{&u_1=65\\ &d=-6.}
		\end{equation*}
		Vậy công thức tổng quát của cấp số cộng:
		\begin{center}
			$u_n=65+(n-1).(-6) \Leftrightarrow u_n=-6n+71$  với $n \geq 2.$
		\end{center}	
	}
\end{bt}

\begin{bt}%[TH]%[Dự án DCHT-11-KNTT]%[Dao-V- Thuy]%[1K2B5-1]
	Tìm số hạng đầu và công sai của cấp số cộng $(u_n),$ biết $\heva{&u_2+u_5-u_3=10\\ &u_4+u_6=26.}$
	\loigiai{
		Ta có
		\begin{align*}
			\heva{&u_2+u_5-u_3=10\\ &u_4+u_6=26} 
			&\Leftrightarrow \heva{&u_1+d+u_1+4d-(u_1+2d)=10\\ &u_1+3d+u_1+5d=26}\\ 
			& \Leftrightarrow\heva{&u_1+3d=10 \\ &2u_1+8d=26}
			\Leftrightarrow \heva{&u_1=1 \\&d=3.}
		\end{align*}
		Vậy $u_1=1$, $d=3$.
	}
\end{bt}

\begin{bt}%[TH]%[Dự án DCHT-11-KNTT]%[Dao-V- Thuy]%[1K2B5-1]
	Tìm số hạng đầu và công sai của cấp số cộng, biết
		\begin{listEX}[3]
			\item $\heva{&u_7 = 27\\&u_{15} = 59.}$
			\item $\heva{&u_9 = 5u_2\\&u_{13} = 2u_6 + 5.}$
			\item $\heva{&u_2 + u_4 - u_6 = -7\\&u_8 - u_7 = 2u_4.}$
			\item $\heva{&u_3 - u_7 = -8\\&u_2 \cdot u_7 = 75.}$
			\item $\heva{&u_6 + u_7 = 60\\&u_4^2 + u_{12}^2 = 1170.}$
		\end{listEX}
	\loigiai{
		\begin{enumerate}
			\item Ta có $\heva{&u_7 = 27\\&u_{15} = 59} \Leftrightarrow \heva{&u_1 + 6d = 27\\&u_1 + 14d = 59} \Leftrightarrow \heva{&u_1 = 3\\&d = 4.}$\\
			Vậy số hạng đầu của cấp số cộng là $u_1 = 3$, công sai là $d = 4$.
			\item Ta có $\heva{&u_9 = 5u_2\\&u_{13} = 2u_6 + 5} \Leftrightarrow \heva{&u_1 + 8d = 5u_1 + 5d\\&u_1 + 12d = 2u_1 + 10d + 5} \Leftrightarrow \heva{&4u_1 - 3d = 0\\&-u_1 + 2d = 5} \Leftrightarrow \heva{&u_1 = 3\\&d = 4.}$\\
			Vậy số hạng đầu của cấp số cộng là $u_1 = 3$, công sai là $d = 4$.
			\item Ta có $\heva{&u_2 + u_4 - u_6 = -7\\&u_8 - u_7 = 2u_4} \Leftrightarrow \heva{&u_1 + d + u_1 + 3d - u_1 - 5d = -7\\&u_1 + 7d - u_1 - 6d = 2u_1 + 6d} \Leftrightarrow \heva{&u_1 - d = -7\\&2u_1 + 5d = 0} \Leftrightarrow \heva{&u_1 = -5\\&d = 2.}$\\
			Vậy số hạng đầu của cấp số cộng là $u_1 = -5$, công sai là $d = 2$.
			\item Ta có $\heva{&u_3 - u_7 = -8\\&u_2 \cdot u_7 = 75} \Leftrightarrow \heva{&u_1 + 2d -u_1 - 6d = -8\\&(u_1 + d)(u_1 + 6d) = 75} \Leftrightarrow \heva{&d = 2\\&u_1^2 + 14u_1 - 51 = 0} \Leftrightarrow \heva{&d = 2\\&\hoac{&u_1 = 3\\&u_1 = -17.}}$\\
			Vậy số hạng đầu của cấp số cộng là $u_1 = 3$, công sai là $d = 2$ hoặc $u_1 = -17$, $d = 2$.
			\item Ta có $\heva{&u_6 + u_7 = 60\\&u_4^2 + u_{12}^2 = 1170} \Leftrightarrow \heva{&2u_6 + d = 60&(1)\\&(u_6 - 2d)^2 + (u_6 + 6d)^2 = 1170.&(2)}$\\
			Từ (1), suy ra $d = 60 - 2u_6$, thay vào (2), ta có
			$$(5u_6 - 120)^2 + (360 - 11u_6)^2 = 1170 \Leftrightarrow 146u_6^2 - 9120u_6 + 142830 = 0 \,\, (\text{vô nghiệm}).$$ 
			Vậy không tồn tại cấp số cộng thỏa yêu cầu bài toán.
		\end{enumerate}
	}
\end{bt}
% \begin{bt}%[TH]%[DCHT Toán 11 - KNTT -Lê Hải Phụng] %[1K2B6-1]
% 	Tìm số hạng đầu tiên, công sai của cấp số cộng sau $ \heva{&u_5=19\\&u_9=35.}$
	
% 	\dapso{Số hạng đầu tiên $ u_1=3 $, công sai $ d=4 $.}
% 	\loigiai{Áp dụng công thức $ u_n=u_1+(n-1)d $ ta có $\heva{&u_5=19\\&u_9=35} \Leftrightarrow \heva{&u_1+4d=19\\&u_1+8d=35} \Leftrightarrow \heva{&u_1=3\\&d=4.}$\\
% 	Vậy số hạng đầu tiên $ u_1=3 $, công sai $ d=4 $.}
% \end{bt}

\begin{bt}%[VD]%[1K2K6-1]
	Cho cấp số cộng $ (u_n) $ thỏa mãn $ \heva{&u_2+u_4-u_6=-7\\&u_8+u_7=2u_4} $. Xác định số hạng đầu $ u_1 $ và công sai $ d $ cấp số cộng.        
	
	% \dapso{$ u_1=5 $, $ d=2 $.}
	\loigiai{
		Ta có $ \heva{&u_2+u_4-u_6=-7\\&u_8+u_7=2u_4} \Leftrightarrow \heva{& u_1+d+(u_1+3d)-(u_1+5d)=-7 \\ & u_1+7d-(u_1+6d)=2(u_1+3d)} \Leftrightarrow \heva{& u_1-d=-7 \\ & 2u_1+5d=0} \Leftrightarrow \heva{&u_1=-5\\&d=2.}$}
\end{bt}

\begin{bt}%[VD]%[DCHT Toán 11 - KNTT -Lê Hải Phụng] %[1K2K6-1]
Cho cấp số cộng $ (u_n) $ thỏa mãn $ \heva{&u_2-u_3+u_5=10\\&u_4+u_6=26} $. Xác định số hạng đầu $ u_1 $ và công sai $ d $ cấp số cộng.         
\dapso{$ u_1=1 $, $ d=3 $.}
\loigiai{Ta có $ \heva{&u_2-u_3+u_5=10\\&u_4+u_6=26} \Leftrightarrow \heva{& u_1+d-(u_1+2d)+u_1+4d=10 \\ & u_1+3d+u_1+5d=26} \Leftrightarrow \heva{& u_1+3d=10 \\ & u_1+4d=13} \Leftrightarrow \heva{u_1=1\\d=3.}$}
\end{bt}

\begin{bt}%[VDC]%[DCHT Toán 11 - KNTT -Lê Hải Phụng] %[1K2G6-1]
Tính số hạng đầu $ u_1 $ và công sai $d$ của một cấp số cộng biết $ \heva{&u_1+u_2+u_3=27\\&u_1^2+u_2^2+u_3^2=275} $

\dapso{$ u_1=5 $, $ d=4 $ hoặc $ u_1=13 $, $ d=-4 $.}
\loigiai{Ta có $ \heva{&u_1+u_2+u_3=27\\&u_1^2+u_2^2+u_3^2=275} \Leftrightarrow \heva{&u_2-d+u_2+u_2+d=27\\&(u_2-d)^2+u_2^2+(u_2+d)^2=275}\Leftrightarrow \heva{&u_2=9\\&3u_2^2+2d^2=275.}$\\
Thay $ u_2=9 $ vào $ 3u_2^2+2d^2=275 $ ta được $ d=4 $ hay $ d=-4 $.\\
Vậy $ u_1=5 $, $ d=4 $ hoặc $ u_1=13 $, $ d=-4 $.}
\end{bt}
\subsubsection{Câu hỏi trắc nghiệm}
\Opensolutionfile{ans}[ans/ans-1K2-2-Dang1]

\begin{ex}%[DCHT Toán 11 - KNTT -Lê Hải Phụng] %[1K2Y6-1]
Trong các dãy số sau, dãy số nào là một cấp số cộng?
\choice
{\True $ 1 $; $ -3 $; $ -7 $; $ -11 $; $ -15 $; $ \ldots $}
{$ 1 $; $ -3 $; $ -6 $; $ -9 $; $ -12 $; $ \ldots $}
{$ 1 $; $ -2 $; $ -4 $; $ -6 $; $ -8 $; $ \ldots $}
{$ 1 $; $ -3 $; $ -5 $; $ -7 $; $ -9 $; $ \ldots $}
\loigiai
{
	Ta lần lượt tính khoảng cách $ d $ các phần tử, ta thấy dãy số đáp án A có $ d= -4$.
}
\end{ex}
%Cau2
\begin{ex}%[DCHT Toán 11 - KNTT -Lê Hải Phụng] %[1K2Y6-1]
Dãy số nào sau đây \textbf{không} phải là cấp số cộng?
\choice
{$ -\dfrac{2}{3} $; $ -\dfrac{1}{3} $; $ 0 $; $ \dfrac{1}{3} $; $ \dfrac{2}{3} $; $ 1 $; $ \dfrac{4}{3} $}
{$ 15\sqrt{2} $; $ 12\sqrt{2} $; $ 9\sqrt{2} $; $ 6\sqrt{2} $}
{\True $ \dfrac{4}{5} $; $ 1 $; $ \dfrac{7}{5} $; $ \dfrac{9}{5} $; $ \dfrac{11}{5} $}
{$ \dfrac{1}{\sqrt{3}} $; $ \dfrac{2\sqrt{3}}{3} $; $ \sqrt{3} $; $ \dfrac{4\sqrt{3}}{3} $; $ \dfrac{5}{\sqrt{3}} $}
\loigiai
{
	Ta lần lượt tính khoảng cách $ d $ các phần tử, ta thấy dãy số trừ đáp án C có khoảng cách các phần tử không bằng nhau.
}
\end{ex}
%Cau3
\begin{ex}%[DCHT Toán 11 - KNTT -Lê Hải Phụng] %[1K2Y6-1]
Cho cấp số cộng $ (u_n) $ với $ u_1=2 $ và $ u_2=6 $. Công sai của cấp số cộng đã cho là	
\choice
{\True $ 4 $}
{$ -4 $}
{$ 8 $}
{$ 3 $}
\loigiai
{
	Ta có $ u_2=6 \Leftrightarrow 6=u_1+d \Leftrightarrow d=4 $.
}
\end{ex}
%Cau4
\begin{ex}%[DCHT Toán 11 - KNTT -Lê Hải Phụng] %[1K2Y6-1]
Cho cấp số cộng $ (u_n) $ với $ u_1=-3 $ và $ u_6=27 $. Công sai $ d $ của cấp số cộng đã cho là	
\choice
{$ d=7 $}
{$ d=5 $}
{$ d=8 $}
{\True $ d=6 $}
\loigiai
{
	Ta có $ u_6=27 \Leftrightarrow 27=u_1+5d \Leftrightarrow d=6 $.
}
\end{ex}
%Cau5
\begin{ex}%[DCHT Toán 11 - KNTT -Lê Hải Phụng] %[1K2B6-1]
Cho cấp số cộng $ (u_n) $ với $ u_{17}=33 $ và $ u_{33}=65 $. Công sai của cấp số cộng đã cho là	
\choice
{$ 1 $}
{$ 3 $}
{$ -2 $}
{\True $ 2 $}
\loigiai
{
	Gọi $ u_1 $, $ d $ lần lượt là số hạng đầu và công sai của cấp số cộng $ (u_n) $.\\
	Khi đó, ta có $ u_{17}=u_1+16d $, $ u_{33}=u_1+32d $\\
	Suy ra $ u_{33}-u_{17}=65-33 \Leftrightarrow 16d=32 \Leftrightarrow d=2 $\\
	Vậy công sai bằng $ 2 $.
}
\end{ex}
%Cau6
\begin{ex}%[DCHT Toán 11 - KNTT -Tên GV] %[1K2B6-1]
Cho cấp số cộng có $ u_1=-3 $ và $ d=4 $. Chọn khẳng định đúng trong các khẳng định sau.
\choice
{$ u_5=15 $}
{$ u_4=8 $}
{\True $ u_3=5 $}
{$ u_2=2 $}
\loigiai
{
	Ta có $ u_3=u_1+2d=-3+2\cdot4=5 $.
}
\end{ex}
%Cau7
\begin{ex}%[DCHT Toán 11 - KNTT -Tên GV] %[1K2Y6-1]
Cho cấp số cộng có $ u_1=11 $ và công sai $ d=4 $. Hãy tính $ u_{99} $.
\choice
{$ 401 $}
{\True $ 403 $}
{$ 402 $}
{$ 404 $}
\loigiai
{
	Ta có $ u_{99}=u_1+98d=11+98\cdot4=403 $.
}
\end{ex}
%Cau8
\begin{ex}%[DCHT Toán 11 - KNTT -Tên GV] %[1K2B6-1]
Một cấp số cộng $ (u_n) $ có $ u_{13}=8 $ và $ d=-3 $. Tìm số hạng thứ ba của cấp số cộng $ (u_n) $.
\choice
{$ 50 $}
{$ 28 $}
{\True $ 38 $}
{$ 44 $}
\loigiai
{
	Ta có $ u_{13}=u_1+12d \Leftrightarrow 8=u_1+12\cdot(-3)\Rightarrow u_1=44 \Rightarrow u_{3}=u_1+2d=44-6=38$.
}
\end{ex}
%Cau9
\begin{ex}%[DCHT Toán 11 - KNTT -Tên GV] %[1K2Y6-1]
Cho cấp số cộng $(u_n) $ có số hạng đầu $ u_1=2 $ và công sai $ d=4 $. Hãy tính giá trị $ u_{2019} $ bằng
\choice
{\True $ 8074 $}
{$ 4074 $}
{$ 8078 $}
{$ 4078 $}
\loigiai
{
	Ta có $ u_{2019}=u_1+2018d=2+2018\cdot 4=8074 $.
}
\end{ex}
%Cau10
\begin{ex}%[DCHT Toán 11 - KNTT -Tên GV] %[1K2K6-1]
Cho cấp số cộng $ (u_n) $ có số hạng tổng quát là $ u_n=3n-2 $. Tìm công sai $ d $ của cấp số cộng.
\choice
{\True $ d=3 $}
{$ d=2 $}
{$ d=-2 $}
{$ d=-3 $}
\loigiai
{
	Ta có $ u_{n+1}-u_n=3(n+1)-2-3n+2=3 $. Suy ra công sai $ d=3 $.
}
\end{ex}

\begin{ex}%[Dự án DCHT-11-KNTT]%[Dao-V- Thuy]%[1K2Y5-1]
	Cho cấp số cộng $(u_n)$ có số hạng đầu $u_1$ và công sai $d$. Công thức tìm số hạng tổng quát $u_n$ là 
	\choice
	{\True $u_n=u_1+(n-1)d$}
	{$u_n=u_1+nd$}
	{$u_n=u_1+(n+1)d$}
	{$u_n=nu_1+d$}
	\loigiai{
		Ta có $u_n=u_1+(n-1)d$.
	}
\end{ex}

\begin{ex}%[Dự án DCHT-11-KNTT]%[Dao-V- Thuy]%[1K2Y5-1]
	Cho cấp số cộng $(u_n)$ có $u_1=-3$ và $d=\dfrac{1}{2}$. Khẳng định nào sau đây đúng?
	\choice
	{$u_n=-3+\dfrac{1}{2}(n+1 )$}
	{$u_n=-3+\dfrac{1}{2}n-1$}
	{\True $u_n=-3+\dfrac{1}{2}(n-1)$}
	{$u_n=-3+\dfrac{1}{4}(n-1 )$}
	\loigiai {
		Ta có $\heva{
			&u_1=-3 \\
			& d=\dfrac{1}{2} \\
		}\xrightarrow{CTTQ} u_n=u_1+(n-1 )d=-3+\dfrac{1}{2}(n-1 )$.}
\end{ex}

\begin{ex}%[Dự án DCHT-11-KNTT]%[Dao-V- Thuy]%[1K2Y5-1]
	Cho cấp số cộng $\left(u_n\right)$ xác định bởi $u_n=2n+1$. Xác định số hạng đầu $u_1$ và công sai $d$ của cấp số cộng.
	\choice
	{$u_1=3$, $d=1$}
	{$u_1=1$, $d=1$}
	{\True $u_1=3$, $d=2$}
	{$u_1=1$, $d=2$}
	\loigiai{
		Ta có $u_1=2\cdot 1+1=3$ và $u_2=2\cdot 2+1=5$, nên $d=u_2-u_1=2$.
	}
\end{ex}

\begin{ex}%[Dự án DCHT-11-KNTT]%[Dao-V- Thuy]%[1K2B5-1]
	Cho cấp số cộng $\left(u_n\right)$ có $u_4=-12$, $u_{14}=18$. Tìm số hạng đầu $u_1$ và công sai $d$ của cấp số cộng $\left(u_n\right)$. 
	\choice 
	{$u_1=-20$, $d=-3$}
	{$u_1=-22$, $d=3$ }
	{\True $u_1=-21$, $d=3$}
	{$u_1=-21$, $d=-3$}
	\loigiai{
		Ta có $$\heva{&u_4=u_1+(4-1)d\\&u_{14}=u_1+(14-1)d} \Leftrightarrow \heva{&-12=u_1+3d\\&18=u_1+13d}\Leftrightarrow \heva{&u_1=-12\\&d=3.}$$
	}
\end{ex}

\begin{ex}%[Dự án DCHT-11-KNTT]%[Dao-V- Thuy]%[1K2B5-1]
	Tìm số hạng đầu và công sai của cấp số cộng $(u_n)$ thỏa mãn $\heva{&u_1+u_9=12\\&u_4-3u_2=1.}$
	\choice
	{$u_1=\dfrac{1}{2}$; $d=\dfrac{13}{8}$}
	{$u_1=-1$; $d=\dfrac{13}{8}$}
	{\True $u_1=-\dfrac{1}{2}$; $d=\dfrac{13}{8}$}
	{$u_1=-1$; $d=2$}
	\loigiai{Ta có: $\heva{&u_1+u_9=12\\&u_4-3u_2=1}\Leftrightarrow\heva{&u_1+(u_1+8d)=12\\&(u_1+3d)-3(u_1+d)=1}\Leftrightarrow\heva{&2u_1+8d=12\\&-2u_1=1}\Leftrightarrow\heva{&d=\dfrac{13}{8}\\&u_1=-\dfrac{1}{2}}$}
\end{ex}

\begin{ex}%[Dự án DCHT-11-KNTT]%[Dao-V- Thuy]%[1K2B5-1]
	Cho cấp số cộng $(u_n)$ có $u_4=-12$ và $u_{14} =18$. Khi đó, số hạng đầu tiên $u_1$ và công sai $d$ của cấp số cộng $(u_n)$ lần lượt là
	\choice
	{$u_1=-20$, $d=-3$}
	{$u_1=-22$, $d=3$}
	{\True $u_1=-21$, $d=3$}
	{$u_1=-21$, $d=-3$}
	\loigiai{Ta có: $\heva{&u_4=-12\\&u_{14}=18}\Leftrightarrow\heva{&u_1+3d=-12\\&u_1+13d=18}\Leftrightarrow\heva{&u_1=-21\\&d=3.}$}
\end{ex}

\begin{ex}%[Dự án DCHT-11-KNTT]%[Dao-V- Thuy]%[1K2B5-1]
	Cho cấp số cộng $(u_n )$ có các số hạng đầu lần lượt là $5;\,9;\,13;\,17;\ldots $. Tìm số hạng tổng quát $u_n$ của cấp số cộng.
	\choice
	{$u_n=5n+1$}
	{$u_n=5n-1$}
	{\True $u_n=4n+1$}
	{$u_n=4n-1$}
	\loigiai{
		Cấp số cộng đã cho có $u_1=5$, $ d=u_2-u_1=4 $. Suy ra $u_n=u_1+(n-1 )d=5+4(n-1 )=4n+1$.
		}
\end{ex}

\begin{ex}%[Dự án DCHT-11-KNTT]%[Dao-V- Thuy]%[1K2B5-1]
	Cho cấp số cộng $(u_n)$ có $u_3=15$ và $d=-2$. Tìm $u_n$.
	\choice
	{\True $u_n=-2n+21$}
	{$u_n=-\dfrac{3}{2}n+12$}
	{$u_n=-3n-17$}
	{$u_n=\dfrac{3}{2}{{n}^2}-4$}
	\loigiai {
		Ta có $\heva{ & 15=u_3=u_1+2d \\& d=-2}
		\Leftrightarrow \heva{&u_1=19 \\& d=-2}
		\Rightarrow u_n=u_1+(n-1 )d=-2n+21$.
		}
\end{ex}

\begin{ex}%[Dự án DCHT-11-KNTT]%[Dao-V- Thuy]%[1K2B5-1]
	Trong các dãy số được cho dưới đây, dãy số nào {\bf không} phải là cấp số cộng?
	\choice
	{$u_n=-4n+9$}
	{$u_n=-2n+19$}
	{$u_n=-2n-21$}
	{\True $u_n=-2^n+15$}
	\loigiai {
		Dãy số $u_n=-2^n+15$ không có dạng $an+b$ nên có không phải là cấp số cộng.}
\end{ex}

\begin{ex}%[Dự án DCHT-11-KNTT]%[Dao-V- Thuy]%[1K2B5-1]
	Cho cấp số cộng $(u_n)$ có $u_4=-12$ và $u_{14}=18$. Tìm số hạng đầu tiên $u_1$ và công sai $d$ của cấp số cộng đã cho.
	\choice
	{\True $u_1=-21$; $d=3$}
	{$u_1=-20$; $d=-3$}
	{$u_1=-22$; $d=3$}
	{$u_1=-21$; $d=-3$}
	\loigiai {
		Ta có 
		$\heva{&u_4=-12\\ &u_{14}=18} \Leftrightarrow \heva{
			&u_1+3d=-12\\
			&u_1+13d=18 \\
		}\Leftrightarrow \heva{
			&u_1=-21 \\
			& d=3. \\
		}$}
\end{ex}

\begin{ex}%[Dự án DCHT-11-KNTT]%[Dao-V- Thuy]%[1K2K5-1]
	Cho cấp số cộng $(u_n)$ thoả mãn $\heva{&u_2-u_3+u_5=10\\ &u_3+u_4=17}$. Số hạng đầu tiên và công sai của cấp số cộng đó lần lượt là
	\choice
	{\True $1$ và $3$}
	{$-3$ và $4$}
	{$4$ và $-3$}
	{$-4$ và $-3$}
	\loigiai{
		$\heva{&u_2-u_3+u_5=10\\ &u_3+u_4=17}\Leftrightarrow\heva{&(u_1+d)-(u_1+2d)+(u_1+4d)=10\\&(u_1+2d)+(u_1+3d)=17}\Leftrightarrow\heva{&u_1+3d=10\\&2u_1+5d=17}\Leftrightarrow\heva{&u_1=1\\&d=3.}$}
\end{ex}

\begin{ex}%[Dự án DCHT-11-KNTT]%[Dao-V- Thuy]%[1K2K5-1]
	Cho cấp số cộng $(u_n)$ có công sai $d<0$, $u_{31}+u_{34}=11$ và $(u_{31})^2 + (u_{34})^2=101$. Số hạng tổng quát của $(u_n)$ là
	\choice
	{$u_{n}=86-3n$}
	{$u_{n}=92-3n$}
	{$u_{n}=95-3n$}
	{\True $u_{n}=103-3n$}
	\loigiai{Gọi cấp số cộng $(u_n)$ có công sai $d$.\\
		$(u_{31})^2 + (u_{34})^2=101 \Leftrightarrow \left( {u_{31}+u_{34}}\right)^2-2u_{31}.u_{34}=101$ $\Rightarrow u_{31}.u_{34}=10$.\\
		Do đó, ta có $\heva{&u_{31}+u_{34}=11\\ &u_{31}.u_{34}=10}$ $\Rightarrow \heva{&u_{31}=10 \\ &u_{34}=1}$(vì $d<0$)\\
		$u_{31}+u_{34}=11 \Rightarrow 2u_{31}+3d =11 \Rightarrow d=-3 \,\,\text{và}\,\, u_{1}=100$.\\
		Do đó: $u_{n}=103-3n$.}
\end{ex}
\Closesolutionfile{ans}
% \begin{indapan}{10}
% 	{ans/ans-1K2-2-Dang2}
% \end{indapan}
\begin{dang}{Tổng của $n$ số hạng đầu tiên của một cấp số cộng. Tính chất của cấp số cộng}
	Tổng của $n$ số hạng đầu tiên:	Đặt ${{S}_{n}}={{u}_{1}}+{{u}_{2}}+{{u}_{3}}+\cdots+{{u}_{n}}.$ Khi đó
	\begin{itemize}
		\item [$\bullet$] ${{S}_{n}}=\dfrac{n\left( {{u}_{1}}+{{u}_{n}} \right)}{2}=\dfrac{n\left( {{u}_{2}}+{{u}_{n-1}} \right)}{2}=\dfrac{n\left( {{u}_{3}}+{{u}_{n-2}} \right)}{2}=\cdots$
		\item [$\bullet$] Vì ${{u}_{n}}={{u}_{1}}+\left( n-1 \right)d$ nên công thức trên có thể viết lại là \fbox{${{S}_{n}}=\dfrac{n}{2}\left[2u_1 + \left(n-1\right)d \right]  .$}
	\end{itemize}
	Tính chất của cấp số cộng:
	\begin{itemize}
		\item [\ding{172}] Nếu $a$; $b$; $c$ theo thứ tự lập thành cấp số cộng thì $a+c=2b$.
		\item [\ding{173}] Lưu ý:
		\begin{itemize}
			\item [$\bullet$] Nếu cho ba số liên tiếp của một cấp số cộng, ta có thể xem ba số đó là $$a-d;\quad a; \quad a+d$$
			\item [$\bullet$] Nếu cho bốn số liên tiếp của một cấp số cộng, ta có thể xem ba số đó là $$a-3d;\quad a-d; \quad a+d; \quad a+3d.$$
		\end{itemize}
	\end{itemize}
\end{dang}
\viduminhhoa
\begin{vd}
	Cho một cấp số cộng $(u_n)$ có $u_3 + u_{28} = 100$. Hãy tính tổng của $30$ số hạng đầu tiên của cấp số cộng đó.\dapso{$1500$}
	\loigiai{Ta có $S_{30} = \dfrac{30(u_1 + u_{30})}{2} = \dfrac{30(u_1 + 2d + u_{30} - 2d)}{2} = \dfrac{30(u_3 + u_{28})}{2} = \dfrac{30 \cdot 100}{2} = 1500$.}
\end{vd}\dongcham{7}

\begin{vd}
	Cho một cấp số cộng $(u_n)$ có $S_6 = 18$ và $S_{10} = 110$. Tính $S_{20}$.	\dapso{$ 620 $.}
	\loigiai{
		Giả sử cấp số cộng $(u_n)$ có số hạng đầu là $u_1$ và công sai là $d$.\\
		Ta có $S_6 = 6u_1 + \dfrac{6 \cdot 5}{2}d \Leftrightarrow 6u_1 + 15d = 18$. \quad (1)\\
		$S_{10} = 10u_1 + \dfrac{10 \cdot 9}{2}d \Leftrightarrow 10u_1 + 45d = 110$. \quad (2)\\
		Từ (1) và (2), ta có hệ phương trình $\heva{&6u_1 + 15d = 18\\&10u_1 + 45d = 110} \Leftrightarrow \heva{&u_1 = -7\\&d = 4.}$\\
		Khi đó $S_{20} = 20u_1 + \dfrac{20 \cdot 19}{2}d = 20 \cdot (-7) + 190 \cdot 4 = 620$.
	}
\end{vd}\dongcham{8}


\begin{vd}
	Tìm số hạng đầu và công sai của cấp số cộng, biết
	\begin{tasks}(2)
		\task $\heva{&u_1^2 + u_2^2 + u_3^2 = 155\\&S_3 = 21.}$	\dapso{$u_1 = 9$, $d = -2$ hoặc $u_1 = 5$, $d = 2$.}
		\task $\heva{&S_3 = 12\\&S_5 = 35.}$	\dapso{$u_1 = 1$, $d = 3$.}
	\end{tasks}
	\loigiai{
		\begin{listEX}
			\item $\heva{&u_1^2 + u_2^2 + u_3^2 = 155\\&S_3 = 21} \Leftrightarrow \heva{&u_1^2 + (u_1 + d)^2 + (u_1 + 2d)^2 = 155 &(1)\\&3u_1 + 3d = 21.&(2)}$\\
			Từ (2), ta có $3u_1 + 3d = 21 \Rightarrow d = 7 - u_1$, thay vào (1)
			$$u_1^2 + 7^2 + (14 - u_1)^2 = 155 \Leftrightarrow 2u_1^2 - 28u_1 + 90 = 0 \Leftrightarrow \hoac{&u_1 = 9\\&u_1 = 5.}$$
			Với $u_1 = 9$ thì $d = -2$. Với $u_1 = 5$ thì $d = 2$.\\
			Vậy số hạng đầu của cấp số cộng là $u_1 = 9$, công sai là $d = -2$ hoặc $u_1 = 5$, $d = 2$.
			\item $\heva{&S_3 = 12\\&S_5 = 35} \Leftrightarrow \heva{&3u_1 + 3d = 12\\&5u_1 + 10d = 35} \Leftrightarrow \heva{&u_1 = 1\\&d = 3.}$\\
			Vậy số hạng đầu của cấp số cộng là $u_1 = 1$, công sai là $d = 3$.
	\end{listEX}}
\end{vd}\dongcham{12}

\begin{vd}
	Tìm số hạng tổng quát của cấp số cộng, biết 
	$\heva{&S_4 = 20\\&\dfrac{1}{u_1} + \dfrac{1}{u_2} + \dfrac{1}{u_3} + \dfrac{1}{u_4} = \dfrac{25}{24}}$ và cấp số cộng có công sai là một số nguyên âm.	\dapso{$ u_n=10-2n $.}
	\loigiai{
		$\heva{&S_4 = 20 &(1)\\&\dfrac{1}{u_1} + \dfrac{1}{u_2} + \dfrac{1}{u_3} + \dfrac{1}{u_4} = \dfrac{25}{24}&(2).}$\\
		Từ (1), suy ra $u_1 + u_4 = u_2 + u_3 = 10$ và $u_1 = 5 - \dfrac{3}{2}d$.\\
		Từ (2), ta có 
		\begin{eqnarray*}
			& &\dfrac{u_1 + u_4}{u_1 \cdot u_4} + \dfrac{u_2 + u_3}{u_2 \cdot u_3} = \dfrac{25}{24} \Leftrightarrow \dfrac{10}{u_1(u_1 + 3d)} + \dfrac{10}{(u_1 + d)(u_1 + 2d)} = \dfrac{25}{24}\\
			&\Leftrightarrow & \dfrac{10}{\left(5 - \dfrac{3}{2}d\right)\left(5 + \dfrac{3}{2}d\right)} + \dfrac{10}{\left(5 - \dfrac{1}{2}d\right)\left(5 + \dfrac{1}{2}d\right)} = \dfrac{25}{24} \Leftrightarrow \dfrac{10}{25 - \dfrac{9}{4}d^2} + \dfrac{10}{25 - \dfrac{1}{4}d^2} = \dfrac{25}{24}\\
			&\Leftrightarrow & 10\left(25 - \dfrac{9}{4}d^2 + 25 - \dfrac{1}{4}d^2\right) = \dfrac{25}{24}\left(25 - \dfrac{9}{4}d^2\right)\left(25 - \dfrac{1}{4}d^2\right)\\
			&\Leftrightarrow & \dfrac{75}{128}d^4 - \dfrac{1925}{48}d^2 + \dfrac{3625}{24} = 0 \Leftrightarrow \hoac{&d^2 = \dfrac{580}{9}\\&d^2 = 4} \Leftrightarrow \hoac{&d = \pm \dfrac{2\sqrt{145}}{3}\\&d = \pm 2.}
		\end{eqnarray*}
		Với $d = -2$ thì $u_1 = 8$. Suy ra $u_n=u_1+(n-1)d=10-2n$}
\end{vd}\dongcham{18}

\begin{vd}
	Tính các tổng sau
	\begin{tasks}(2)
		\task $S = 1 + 3 + 5 + \cdots + (2n - 1) + (2n + 1)$.\dapso{$S = (n + 1)^2$}
		\task $S = 100^2 - 99^2 + 98^2 - 97^2 + \cdots + 2^2 - 1^2$.\dapso{$S = 5050$}
	\end{tasks}
	\loigiai{
		\begin{enumEX}{1}
			\item $S = 1 + 3 + 5 + \cdots + (2n - 1) + (2n + 1)$.\\
			Xét cấp số cộng $(u_k)$, $k \in \mathbb{N}^*$ với số hạng đầu là $u_1 = 1$ và công sai là $d = 2$.\\
			Ta có $u_k = u_1 + (k - 1)d \Leftrightarrow 2n + 1 = 1 + 2(k - 1) \Leftrightarrow k = n + 1$.\\
			Vậy $S = \dfrac{k(u_1 + u_k)}{2} = \dfrac{(n + 1)(1 + 2n + 1)}{2} = (n + 1)^2$.
			\item $S = 100^2 - 99^2 + 98^2 - 97^2 + \cdots + 2^2 - 1^2 = 199 + 195 + \cdots + 3$.\\
			Xét cấp số cộng $(u_n)$ có số hạng đầu $u_1 = 199$ và công sai $d = u_2 - u_1 = 195 - 199 = -4$.\\
			Ta có $u_n = u_1 + (n - 1)d \Leftrightarrow 3 = 199 - 4(n - 1) \Leftrightarrow n = 50$.\\
			Khi đó $S = \dfrac{n(u_1 + u_{50})}{2} = \dfrac{50(199 + 3)}{2} = 5050$.
			
		\end{enumEX}
	}
\end{vd}\dongcham{18}

\begin{vd}
	Tìm ba số hạng liên tiếp của một cấp số cộng biết tổng của chúng bằng $27$ và tổng các bình phương của chúng là $293$.\dapso{$4$, $9$, $14$}
	\loigiai{
		Gọi ba số hạng liên tiếp của cấp số cộng là $x - d$, $x$, $x + d$ trong đó $d$ là công sai của cấp số cộng.\\
		Khi đó ta có $x - d + x + x + d = 27 \Leftrightarrow 3x = 27 \Leftrightarrow x = 9$.\\
		Mà $(x - d)^2 + x^2 + (x + d)^2 = 293 \Leftrightarrow (9 - d)^2 + 81 + (9 + d)^2 = 293 \Leftrightarrow 2d^2 -50 = 0 \Leftrightarrow \hoac{&d = 5\\&d = -5.}$\\	
		Với $d = 5$ thì ba số hạng của cấp số cộng là $4$, $9$, $14$.\\
		Với $d = -5$ thì ba số hạng của cấp số cộng là $14$, $9$, $4$.\\
		Vậy ba số hạng liên tiếp của cấp số cộng là $4$, $9$, $14$.
	}
\end{vd}\dongcham{14}

\begin{vd}
	Tìm bốn số hạng liên tiếp của một cấp số cộng, biết tổng của chúng bằng $10$ và tổng bình phương của chúng bằng $30$.\dapso{$1$, $2$, $3$, $4$}
	\loigiai{
		Gọi bốn số hạng liên tiếp của cấp số cộng là $x - 3d$, $x - d$, $x 
		+ d$, $x + 3d$ với $2d$ là công sai của cấp số cộng.\\
		Khi đó ta có $x - 3d + x - d + x + d + x + 3d = 10 \Leftrightarrow 4x = 10 \Leftrightarrow x = \dfrac{5}{2}$.\\
		Mặt khác $$(x - 3d)^2 + (x - d)^2 + (x + d)^2 + (x + 3d)^2 = 30 \Leftrightarrow 4x^2 + 20d^2 = 30 \Leftrightarrow d^2 = \dfrac{1}{4} \Leftrightarrow \hoac{&d = \dfrac{1}{2}\\&d = -\dfrac{1}{2}.}$$
		Với $x = \dfrac{5}{2}$ thì $d = \dfrac{1}{2}$, khi đó bốn số hạng liên tiếp của cấp số cộng là $1$, $2$, $3$, $4$.\\
		Với $x = \dfrac{5}{2}$ thì $d = -\dfrac{1}{2}$, khi đó bốn số hạng liên tiếp của cấp số cộng là $4$, $3$, $2$, $1$.\\
		Vậy bốn số hạng liên tiếp của cấp số cộng là $1$, $2$, $3$, $4$.
	}
\end{vd}\dongcham{14}

\begin{vd}
	Ba góc của một tam giác vuông lập thành một cấp số cộng. Tìm ba góc đó.
	\loigiai{Gọi ba góc của tam giác lần lượt là $A$, $B$, $C$.
		Khi đó ta có $A + B + C = 180^\circ$.\\
		Do ba góc $A$, $B$, $C$ của tam giác theo thứ tự lập thành một cấp số cộng nên $B-A=C-A \Leftrightarrow A + C = 2B$.\\
		Do đó $2B + B = 180^\circ \Rightarrow 3B = 180^\circ \Rightarrow B = 60^\circ$.\\
		Do tam giác $ABC$ vuông nên giả sử $C = 90^\circ$ khi đó công sai $d$ của cấp số cộng là $d = C - B = 30^\circ$.\\
		Vậy góc $A$ của tam giác là $A = 30^\circ$.}
\end{vd}\dongcham{10}

% \begin{vd}
% 	Cho $a$, $b$, $c$ là ba số hạng liên tiếp của một cấp số cộng. Chứng minh rằng
% 	\begin{tasks}(1)
% 		\task $a^2 + 2bc = c^2 + 2ab$.
% 		\task $2(a+b+c)^3 = 9 \left[ a^2(b+c) + b^2(a+c) + c^2(a+b) \right]$.
% 		\task  $b^2 + bc +c^2$, $a^2 + ac + c^2$, $a^2 + ab + b^2$ cũng là một cấp số cộng.
% 	\end{tasks}
% 	\loigiai{
% 		\begin{enumerate}[a)]
% 			\item Vì $a$, $b$, $c$ là ba số liên tiếp của một cấp số cộng nên $a + c = 2b \Rightarrow a = 2b -c$.\\
% 			Do đó
% 			$$a^2 +2bc = (2b-c)^2 + 2bc = 4b^2 - 2bc + c^2 = 2b(2b -c) + c^2 = 2ba + c^2 = c^2 + 2ab.$$
% 			Vậy $a^2 + 2bc = c^2 + 2ab$ (đpcm).
			
% 			\item Vì $a$, $b$, $c$ là ba số liên tiếp của một cấp số cộng nên $a + c = 2b \Rightarrow a = 2b -c$.\\
% 			Do đó
% 			\allowdisplaybreaks
% 			\begin{eqnarray*}
% 				& \mbox{VT}  & = 2(a+b+c)^3 = 2(3b)^3 = 54b^3\\
% 				& \mbox{VP}  & = 9\left[ a^2(b+c) + b^2(a+c) + c^2(a+b) \right] \\
% 				& & = 9\left[ (2b-c)^2(b+c) + b^2(2b-c+c) + c^2(2b-c+b) \right] \\ 
% 				& & = 9\left[ (4b^2 - 4bc+c^2)(b+c) + b^2(2b) + c^2(3b-c) \right] \\
% 				& & = 9\left[ 4b^3 - 4b^2c +bc^2 + 4b^2c - 4bc^2 + c^3 + 2b^3 + 3bc^2 - c^3 \right] \\
% 				& & = 9\cdot (6b^3) = 54b^3 = \mbox{ VT }.
% 			\end{eqnarray*}
% 			Vậy $2(a+b+c)^3 = 9 \left[ a^2(b+c) + b^2(a+c) + c^2(a+b) \right]$ (đpcm).
			
% 			\item Vì ba số $a$, $b$, $c$ theo thứ tự lập thành một cấp số cộng thì $a + c = 2b \Rightarrow a = 2b - c$.\\
% 			Xét
% 			\allowdisplaybreaks
% 			\begin{eqnarray*}
% 				& 2(a^2 + ac + c^2) - (a^2 + ab + b^2) & = a^2 + a(2c-b) + 2c^2 - b^2 \\
% 				& & = (2b-c)^2 + (2b-c)(2c-b) + 2c^2 - b^2 \\
% 				& & = b^2 + bc  +c^2\\
% 				&\Rightarrow (b^2 + bc  +c^2) + (a^2 + ab + b^2) &= 2(a^2 + ac + c^2).
% 			\end{eqnarray*}
% 			Vậy ba số: $b^2 + bc +c^2$, $a^2 + ac + c^2$, $a^2 + ab + b^2$ cũng là một cấp số cộng.
% 		\end{enumerate}
% 	}
% \end{vd}\dongcham{25}

\begin{vd}%[TH]%[Dự án DCHT-11-KNTT]%[Dao-V- Thuy]%[1K2B5-2]
	Xác định $4$ góc của một tứ giác lồi, biết rằng $4$ góc hợp thành cấp số cộng và góc lớn nhất bằng $5$ lần góc nhỏ nhất.
	\dapso{$36^\circ; \, 72^\circ; \, 108^\circ; \, 144^\circ$}
	\loigiai
	{
		Gọi số đo bốn góc cần tìm là $u_1$, $u_2$, $u_3$, $u_4$. Ta có
		\begin{eqnarray*}
			\heva{&u_1+u_2+u_3+u_4=360\\ &u_5=5u_1} \Leftrightarrow \heva{&4u_1+6d=360\\ &4d=4u_1} \Leftrightarrow \heva{&u_1=36\\ &d=36.}
		\end{eqnarray*}
		Vậy số đo bốn góc cần tìm là
		\[
		36^\circ; \, 72^\circ; \, 108^\circ; \, 144^\circ.
		\]
	}
\end{vd}

\subsubsection{Bài tập tự luận}
 

\begin{bt}%[TH]%[Dự án DCHT-11-KNTT]%[Dao-V- Thuy]%[1K2B5-2]
	Giữa các số $10$ và $64$ hãy đặt thêm $17$ số nữa để được một cấp số cộng.
	\dapso{$13; 16; 19; 22; 25; 28; 31; 34; 37; 40; 43; 46; 49; 52; 55; 58; 61$}
	\loigiai{
		Ta có
		\begin{equation*}
			\heva{&u_1=10\\ &u_{19}=64} \Leftrightarrow \heva{&u_1=10\\ &u_1+18d=64} \Leftrightarrow \heva{&u_1=10\\ &d=3.}
		\end{equation*}
		Vậy $17$ số đặt thêm giữa các số $10$ và $64$ để được một cấp số cộng là
		\begin{center}
			13; 16; 19; 22; 25; 28; 31; 34; 37; 40; 43; 46; 49; 52; 55; 58; 61.
		\end{center} 
	}
\end{bt}

\begin{bt}%[TH]%[Dự án DCHT-11-KNTT]%[Dao-V- Thuy]%[1K2B5-2]
	Tổng ba số hạng liên tiếp của một cấp số cộng bằng $2$ và tổng các bình phương của ba số đó bằng $\dfrac{14}{9}$. Xác định ba số đó và tính công sai của cấp số cộng.
	\dapso{$1;\dfrac{2}{3};\dfrac{1}{3}$ ứng với $d=-\dfrac{1}{3}$ hoặc $\dfrac{1}{3};\dfrac{2}{3};1$ ứng với $d=\dfrac{1}{3}$}
	\loigiai{
		Ta có hệ
		\begin{align*}
			&\quad \heva{&u_k+u_{k+1}+u_{k+2}=2\\ &u^2_k+u^2_{k+1}+u^2_{k+2}=\dfrac{14}{9}}
			\Leftrightarrow \heva{&u_k+u_k+d+u_k+2d=2\\ &u^2_k+\left(u_k+d\right)^2 +\left(u_k+2d\right)^2=\dfrac{14}{9}} \\
			&\Leftrightarrow \heva{&3u_k+3d=2\\ &3u^2_k+6u_kd+5d^2=\dfrac{14}{9}}
			\Leftrightarrow \heva{&u_k=1\\ &d=-\dfrac{1}{3}} \text{ hoặc } \heva{&u_k=\dfrac{1}{3}\\ &d=\dfrac{1}{3}.}
		\end{align*}
		Vậy ba số hạng liên tiếp của cấp số cộng thỏa yêu cầu bài toán $1;\dfrac{2}{3};\dfrac{1}{3}$ ứng với $d=-\dfrac{1}{3}$ hoặc $\dfrac{1}{3};\dfrac{2}{3};1$ ứng với $d=\dfrac{1}{3}.$
	}
\end{bt}

\begin{bt}%[TH]%[Dự án DCHT-11-KNTT]%[Dao-V- Thuy]%[1K2B5-2]
	Một cấp số cộng có $7$ số hạng với công sai $d$ dương và số hạng thứ tư bằng $11$. Hãy tìm các số hạng còn lại của cấp số cộng đó, biết hiệu của số hạng thứ ba và số hạng thứ năm bằng $6$.
	\dapso{$u_1=2$; $ u_2=5$; $u_4=11$; $u_6=17$; $u_7=20$}
	\loigiai{
		Gọi số hạng đầu của cấp số cộng là $u_1$, công sai $d$.
		Vì số hạng thứ tư của cấp số cộng bằng $11$ nên ta có $u_4=11$.\\
		Do $d$ dương nên $ u_5>u_3$.\\
		Vì hiệu của số hạng thứ ba và số hạng thứ năm bằng $6$ nên ta có $ u_5-u_3=6$.\\
		Ta có \begin{align*}
			\heva{&u_4=11\\&u_5-u_3=6 }
			\Leftrightarrow \heva{&u_1+3d=11\\&(u_1+4d)-(u_1+2d)=6 }
			\Leftrightarrow \heva{&u_1+3\cdot 3=11\\&d=3 }
			\Leftrightarrow \heva{&u_1=2\\&d=3.}
		\end{align*}
		Vậy các số  hạng còn lại của cấp số cộng là $u_1=2$; $ u_2=5$; $u_4=11$; $u_6=17$; $u_7=20$.
	}
\end{bt}

\begin{bt}%[VD]%[Dự án DCHT-11-KNTT]%[Dao-V- Thuy]%[1K2K5-2]
	Tìm bốn số hạng liên tiếp của một cấp số cộng, biết rằng:
	\begin{enumerate}
		\item Tổng của chúng bằng $10$ và tổng bình phương bằng $70$.
		\item Tổng của chúng bằng $22$ và tổng bình phương bằng $66$.
		\item  Tổng của chúng bằng $36$ và tổng bình phương bằng $504$.
		\item  Chúng có tổng bằng $20$ và tích của chúng bằng $384$.
		\item  Tổng của chúng bằng $ 20$, tổng nghịch đảo của chúng bằng $ \dfrac{25}{24}$ và các số này là những số nguyên.
		\item  Nó là số đo của một tứ giác lồi và góc lớn nhất gấp $5$ lần góc nhỏ nhất.
	\end{enumerate}
	\dapso{$-2$; $ 1$; $ 4$; $7$.} 
	\dapso{không tồn tại bốn số hạng liên tiếp của cấp số cộng thỏa mãn yêu cầu đề bài. $0$; $ 6$; $ 12$; $18$.}
	\dapso{$2$; $ 4$; $ 6$; $8$ hoặc $5-\sqrt{241}$; $ \dfrac{15-\sqrt{241}}{3}$; $ \dfrac{15+\sqrt{241}}{3}$; $5+\sqrt{241}$.} 
	\dapso{$30^\circ$; $70^\circ$; $ 110^\circ$; $150^\circ$.}
	\loigiai{
		\begin{enumerate}
			\item Gọi bốn số hạng liên tiếp của cấp số cộng là $x-3d$; $x-d$; $x+d$, $x+3d$ trong đó $2d$ là công sai.\\
			Theo đề bài ta có 
			\begin{align*}
				&\quad \heva{& (x-3d)+(x-d)+(x+d)+(x+3d)=10\\& (x-3d)^2+(x-d)^2+(x+d)^2+(x+3d)^2=70}
				\Leftrightarrow \heva{& 4x=10 \\& 4x^2+20d^2=70}\\
				&\Leftrightarrow  \heva{&x=\dfrac{5}{2}\\& 4\cdot \left( \dfrac{5}{2}\right) ^2+20d^2=70}
				\Leftrightarrow  \heva{&x=\dfrac{5}{2}\\& d^2=\dfrac{9}{4}}
				\Leftrightarrow  \heva{&x=\dfrac{5}{2}\\& d=\pm \dfrac{3}{2}.}
			\end{align*}
			Vậy bốn số hạng liên tiếp của cấp số cộng là $-2$; $ 1$; $ 4$; $7$.
			\item Gọi bốn số hạng liên tiếp của cấp số cộng là $x-3d$; $x-d$; $x+d$; $x+3d$ trong đó $2d$ là công sai.\\
			Theo đề bài ta có 
			\begin{align*}
				&\quad \heva{& (x-3d)+(x-d)+(x+d)+(x+3d)=22\\& (x-3d)^2+(x-d)^2+(x+d)^2+(x+3d)^2=66}
				\Leftrightarrow \heva{& 4x=22 \\& 4x^2+20d^2=66}\\
				&\Leftrightarrow  \heva{&x=\dfrac{11}{2}\\& 4\cdot \left( \dfrac{11}{2}\right) ^2+20d^2=66}
				\Leftrightarrow  \heva{&x=\dfrac{11}{2}\\& d^2=\dfrac{-11}{4} 
					\ (\text{loại}).}\\
			\end{align*}
			Vậy không tồn tại bốn số hạng liên tiếp của cấp số cộng thỏa mãn yêu cầu đề bài.
			\item Gọi bốn số hạng liên tiếp của cấp số cộng là $x-3d$; $x-d$; $x+d$; $x+3d$ trong đó $2d$ là công sai.\\
			Theo đề bài ta có 
			\begin{align*}
				&\quad \heva{& (x-3d)+(x-d)+(x+d)+(x+3d)=36\\& (x-3d)^2+(x-d)^2+(x+d)^2+(x+3d)^2=504}
				\Leftrightarrow \heva{& 4x=36 \\& 4x^2+20d^2=504}\\
				&\Leftrightarrow  \heva{&x=9\\& 4\cdot 9^2+20d^2=504}
				\Leftrightarrow  \heva{&x=9\\& d^2=9}
				\Leftrightarrow  \heva{&x=9\\& d=\pm 3.}
			\end{align*}
			Vậy  bốn  số hạng liên tiếp của cấp số cộng là $0$; $ 6$; $ 12$; $18$.
			\item Gọi bốn số hạng liên tiếp của cấp số cộng là $x-3d$; $x-d$; $x+d$; $x+3d$ trong đó $2d$ là công sai.\\
			Theo đề bài ta có 
			\begin{align*}
				&\quad \heva{& (x-3d)+(x-d)+(x+d)+(x+3d)=20\\& (x-3d)(x-d)(x+d)(x+3d)=384}
				\Leftrightarrow  \heva{&x=5\\& (x^2-d^2)(x^2-9d^2)=384}\\
				&\Leftrightarrow  \heva{&x=5\\& (25-d^2)(25-9d^2)=384}
				\Leftrightarrow  \heva{&x=5\\& 9d^4-250d^2+241=0}
				\Leftrightarrow  \heva{&x=5\\& \hoac{&d^2=1\\& d^2=\dfrac{241}{9}}}
				\Leftrightarrow  \heva{&x=5\\& \hoac{&d=\pm 1\\& d=\pm \dfrac{\sqrt{241}}{3}.}}
			\end{align*}
			Vậy bốn số hạng liên tiếp của cấp số cộng là $2$; $ 4$; $ 6$; $8$ hoặc $5-\sqrt{241}$; $ \dfrac{15-\sqrt{241}}{3}$; $ \dfrac{15+\sqrt{241}}{3}$; $5+\sqrt{241}$.
			\item Gọi bốn số hạng liên tiếp của cấp số cộng là $x-3d$; $x-d$; $x+d$; $x+3d$ trong đó $2d$ là công sai trong đó $ 2d \in \mathbb{Z}$.\\
			Theo đề bài ta có 
			\begin{align*}
				&\quad \heva{& (x-3d)+(x-d)+(x+d)+(x+3d)=20\\& \dfrac{1}{x-3d}+\dfrac{1}{x-d}+\dfrac{1}{x+d}+\dfrac{1}{x+3d}=\dfrac{25}{24}}
				\Leftrightarrow \heva{& 4x=20 \\& \dfrac{1}{5-3d}+\dfrac{1}{5-d}+\dfrac{1}{5+d}+\dfrac{1}{5+3d}=\dfrac{25}{24}}\\
				&\Leftrightarrow  \heva{&x=5\\& \dfrac{10}{25-9d^2}+\dfrac{10}{25-d^2}=\dfrac{25}{24}}
				\Leftrightarrow  \heva{&x=5\\& 9d^4-250d^2+241=0}\\
				&\Leftrightarrow  \heva{&x=5\\& \hoac{&d^2=1 \\& d^2=\dfrac{241}{9} }}
				\Leftrightarrow  \heva{&x=5\\& \hoac{&d= \pm 1 \ (\text{thỏa mãn})\\& d= \pm \dfrac{\sqrt{241}}{3} \,\,(\text{loại vì} \,  2d \in \mathbb{Z}).}}
			\end{align*}
			Vậy bốn  số hạng  nguyên liên tiếp của cấp số cộng là $2$; $ 4$; $ 6$; $8$.
			\item Gọi bốn số hạng liên tiếp của cấp số cộng xếp theo thứ tự tăng dần  là $x-3d$; $x-d$; $x+d$; $x+3d$ trong đó $2d>0$ là công sai.\\
			Theo đề bài ta có 
			\begin{align*}
				&\quad \heva{& (x-3d)+(x-d)+(x+d)+(x+3d)=360^\circ\\& x+3d=5(x-3d)}\\
				&\Leftrightarrow \heva{& 4x=360^\circ\\& 4x=18d}
				\Leftrightarrow  \heva{&x=90^\circ\\& 4 \cdot 90^\circ=18d}
				\Leftrightarrow  \heva{&x=90^\circ\\& d=20^\circ.}
			\end{align*}
			Vậy bốn  góc của tứ giác lồi lần lượt là  $30^\circ$; $70^\circ$; $ 110^\circ$; $150^\circ$.
		\end{enumerate}
	}
\end{bt}
% \subsubsection{Câu hỏi trắc nghiệm}
% \Opensolutionfile{ans}[ans/ans-1K2-2-Dang3]
% \begin{ex}%[Dự án DCHT-11-KNTT]%[Dao-V- Thuy]%[1K2Y5-2]
% 	Cấp số cộng $(u_n)$ có số hạng đầu $u_1=-5$ và công sai $d=3$. Tính $u_{15}$.
% 	\choice
% 	{$u_{15}=27$}
% 	{\True $u_{15}=37$}
% 	{$u_{15}=47$}
% 	{$u_{15}=57$}
% 	\loigiai{$u_{15}=u_1+14d=-5+14\times 3=37$.}
% \end{ex}

% \begin{ex}%[Dự án DCHT-11-KNTT]%[Dao-V- Thuy]%[1K2Y5-2]
% 	Cho cấp số cộng có các số hạng ban đầu là $1$; $5$; $9$; $13$; $\cdots$. Số hạng thứ $ 6 $ của cấp số cộng này là bao nhiêu?
% 	\choice{\True $ 21$}
% 	{$19 $}
% 	{$ 22$}
% 	{$ 20$}
% 	\loigiai{Ta có $u_1=1$, $d=5-1=4$ nên $u_6=1+5d=1+20=21$.
% 	}
% \end{ex}

% \begin{ex}%[Dự án DCHT-11-KNTT]%[Dao-V- Thuy]%[1K2Y5-2]
% 	Cho cấp số cộng $\left( u_n \right)$ có các số hạng lần lượt là $-4;\,1;\,6;\,x$. Tìm giá trị của $x$.
% 	\choice
% 	{$x=7$}
% 	{$x=10$}
% 	{\True $x=11$}
% 	{$x=12$}
% 	\loigiai{
% 		Dễ thấy $u_1=-4$, $d=5$ nên $u_4=-4+3\cdot 5=11$.
% 	}
% \end{ex}

% \begin{ex}%[Dự án DCHT-11-KNTT]%[Dao-V- Thuy]%[1K2B5-2]
% 	Cho cấp số cộng $(u_n)$ có $u_1=-5$ và $d=3$. Mệnh đề nào sau đây đúng?
% 	\choice
% 	{$u_{15}=34$}
% 	{$u_{15}=45$}
% 	{\True $u_{13}=31$}
% 	{$u_{10}=35$}
% 	\loigiai {
% 		$\heva{
% 			& u_1=-5 \\
% 			& d=3 \\
% 		}\Rightarrow u_n=3n-8\Rightarrow \heva{
% 			& u_{15}=37 \\
% 			& u_{13}=31 \\
% 			& u_{10}=22. \\
% 		}$}
% \end{ex}

% \begin{ex}%[Dự án DCHT-11-KNTT]%[Dao-V- Thuy]%[1K2B5-2]
% 	Cho cấp số cộng có số hạng đầu là $u_1=-\dfrac{1}{2}$, công sai $d=\dfrac{1}{2}$. Trong mỗi bộ gồm năm số hạng dưới đây, bộ năm số nào là các số hạng liên tiếp của dãy này?
% 	\choice
% 	{$-\dfrac{1}{2};\,0;\,1;\,\dfrac{1}{2};\,1$}
% 	{$-\dfrac{1}{2};\,0;\,\dfrac{1}{2};\,0;\,\dfrac{1}{2}$}
% 	{$\dfrac{1}{2};\,1;\,2;\,\dfrac{5}{2};\,\dfrac{7}{2}$}
% 	{\True $1;\,\dfrac{3}{2};\,2;\,\dfrac{5}{2};\,3$}
% 	\loigiai{
% 		Ta có $u_1=-\dfrac{1}{2}$; $u_2=0$; $u_3=\dfrac{1}{2}$, $u_4=1$; $u_5=\dfrac{3}{2}$; $u_6=2$; $u_7=\dfrac{5}{2}$; $u_8=3$.
% 	}
% \end{ex}

% \begin{ex}%[Dự án DCHT-11-KNTT]%[Dao-V- Thuy]%[1K2B5-2]
% 	Cho cấp số cộng $(u_n)$ có $u_7=\dfrac{19}{5}$ và công sai $d=\dfrac{2}{5}$. Tính $u_{10}$.
% 	\choice
% 	{$\dfrac{2}{5}$}
% 	{$\dfrac{19}{5}$}
% 	{\True $5$}
% 	{$\dfrac{27}{5}$}
% 	\loigiai{Ta có: $u_7=u_1+6d\Rightarrow u_1=u_7-6d=\dfrac{19}{5}-6\cdot \dfrac{2}{5}=\dfrac{7}{5}$.\\
% 		Suy ra $u_{10}=u_1+9d=\dfrac{7}{5}+9\cdot \dfrac{2}{5}=5$.}
% \end{ex}

% \begin{ex}%[Dự án DCHT-11-KNTT]%[Dao-V- Thuy]%[1K2B5-2]
% 	Cho cấp số cộng $(u_n)$ có số hạng đầu $u_1=-1$ và công sai $d=-3$. Số hạng thứ $20$ của cấp số cộng này là
% 	\choice
% 	{\True $u_{20}=-58$}
% 	{$u_{20}=60$}
% 	{$u_{20}=-72$}
% 	{$u_{20}=-61$}
% 	\loigiai{Số hạng thứ $20$ là: $u_{20}=u_1+19d=-1+19\cdot (-3)=-58$.}
% \end{ex}

% \begin{ex}%[Dự án DCHT-11-KNTT]%[Dao-V- Thuy]%[1K2B5-2]
% 	Cho cấp số cộng $(u_n)$ có $u_1=-5$ và $d=3$. Số $100$ là số hạng thứ mấy của cấp số cộng?
% 	\choice
% 	{Thứ $15$}
% 	{Thứ $20$}
% 	{Thứ $35$}
% 	{\True Thứ $36$}
% 	\loigiai {
% 		Ta có $\heva{
% 			&u_1=-5 \\
% 			& d=3 \\
% 		}$. Vì $u_n=100 \Rightarrow 100=u_n=u_1+(n-1 )d=3n-8\Leftrightarrow n=36$.
% 	}
% \end{ex}

% \begin{ex}%[Dự án DCHT-11-KNTT]%[Dao-V- Thuy]%[1K2B5-2]
% 	Cho cấp số cộng $(u_n)$ có $u_2=2001$ và $u_5=1995$. Khi đó $u_{1001}$ bằng
% 	\choice
% 	{$u_{1001}=4005$}
% 	{$u_{1001}=4003$}
% 	{\True $u_{1001}=3$}
% 	{$u_{1001}=1$}
% 	\loigiai {
% 		$\heva{
% 			& 2001=u_2=u_1+d \\
% 			& 1995=u_5=u_1+4d \\
% 		}\Leftrightarrow \heva{
% 			&u_1=2003 \\
% 			& d=-2 \\
% 		}\Rightarrow u_{1001}=u_1+1000d=3$.}
% \end{ex}

% \begin{ex}%[Dự án DCHT-11-KNTT]%[Dao-V- Thuy]%[1K2B5-2]
% 	Cho cấp số cộng $(u_n)$ biết $\heva{&u_1+u_3=7\\&u_2+u_4=12}$. Tính $u_{21}$.
% 	\choice
% 	{$u_{21}=1$}
% 	{\True $u_{21}=51$}
% 	{$u_{21}=31$}
% 	{$u_{21}=21$}
% 	\loigiai{
% 		Ta có $\heva{&u_1+u_3=7\\&u_2+u_4=12}\Leftrightarrow\heva{&u_1+u_1+2d=7\\&u_1+d+u_1+3d=12}\Leftrightarrow\heva{&2u_1+2d=7\\&2u_1+4d=12}\Leftrightarrow\heva{&u_1=1\\&d=\dfrac{5}{2}.}$\\
% 		Suy ra $u_{21}=u_1+20d=1+20\cdot\dfrac{5}{2}=1+50=51$.}
% \end{ex}

% \begin{ex}%[Dự án DCHT-11-KNTT]%[Dao-V- Thuy]%[1K2B5-2]
% 	Một cấp số cộng có $7$ số hạng. Biết rằng tổng của số hạng đầu và số hạng cuối bằng $30$, tổng của số hạng thứ ba và số hạng thứ sáu bằng $35$. Tìm số hạng thứ bảy của cấp số cộng đã cho.
% 	\choice
% 	{$u_7=25$}
% 	{\True $u_7=30$}
% 	{$u_7=35$}
% 	{$u_7=40$}
% 	\loigiai{
% 		Theo đề ta có: $\heva{&u_1+u_7=30\\&u_3+u_6=35}\Leftrightarrow \heva{&u_1+(u_1+6d)=30\\&(u_1+2d)+(u_1+5d)=35}\Leftrightarrow \heva{&2u_1+6d=30\\&2u_1+7d=35}\Leftrightarrow \heva{&u_1=0\\&d=5.}$\\
% 		Do đó $u_7=u_1+6d=0+6\cdot 5=30$.}
% \end{ex}

% \begin{ex}%[Dự án DCHT-11-KNTT]%[Dao-V- Thuy]%[1K2G5-2]
% 	Cho dãy số $(u_n)$ có xác định bởi $\heva{&u_1=-2,\\&u_{n+1}=\dfrac{u_n}{1-u_n}} \; (\text{với } n\in\mathbb{N}^*)$ và dãy số $(v_n)$ được xác định bởi $v_n=\dfrac{u_n+1}{u_n}$. Số hạng thứ $2023$ của dãy $(v_n)$ là
% 	\choice
% 	{$ -\dfrac{2023}{3}$}
% 	{$ -\dfrac{4046}{3}$}
% 	{\True$ -\dfrac{4043}{2}$}
% 	{$ -2023$}
% 	\loigiai{
% 		Ta có $v_{n+1}-v_n=\dfrac{u_{n+1}+1}{u_{n+1}}-\dfrac{u_n+1}{u_n}=\dfrac{\dfrac{u_n}{1-u_n}+1}{\dfrac{u_n}{1-u_n}}-\dfrac{u_n+1}{u_n}=\dfrac{1}{u_n}-\dfrac{u_n+1}{u_n}=-1$. Vậy $(v_n)$	là một CSC có công sai $d=-1$.
% 		\\Mặt khác, ta có $v_1=\dfrac{u_1+1}{u_1}=\dfrac{1}{2}$, do đó số hạng tổng quát $v_n=\dfrac{1}{2}+(n-1)(-1)=-n+\dfrac{3}{2}$. \\
% 		Do đó $v_{2023}=-2023+\dfrac{3}{2}=-\dfrac{4043}{2}$.
% 	}
% \end{ex}
% \Closesolutionfile{ans}
% \begin{indapan}{10}
% 	{ans/ans-1K2-2-Dang3}
% \end{indapan}

\begin{dang}{Các bài toán thực tế}
	Các bài toán thực tế về cấp số cộng có thể được giải bằng cách sử dụng công thức của cấp số cộng. Công thức của cấp số cộng là: $ u_n = u_1 + (n-1)d $. Trong đó:
	\begin{itemize}
		\item $ u_n $ là số hạng thứ $ n $ của cấp số cộng.
		\item $ u_1 $ là số hạng đầu tiên của cấp số cộng.
		\item $ d $ là công sai của cấp số cộng.
		\item Một số công thức thường gặp:
		\begin{enumEX}[\faCheckCircleO]{1}
			\item $u_n=\dfrac{u_{n-1}+u_{n+1}}{2}=u_1+(n-1)d$.
			\item $S_n=\dfrac{(u_1+u_n)\cdot n}{2}=\dfrac{2u_1+(n-1)d}{2}\cdot n$.
		\end{enumEX}			
	\end{itemize}
\end{dang}
\subsubsection{Ví dụ minh hoạ}
\begin{vd}%[NB]%[DCHT Toán 11 - KNTT - Nguyễn Hữu Đức] %[1K2B6-6]
	Một người có một khoản tiền gửi ngân hàng với lãi suất 10\% /năm theo hình thức lãi đơn. Nếu sau $ 5 $ năm người đó nhận được tổng số tiền là $ 550 $ triệu đồng thì số tiền gửi ban đầu của người đó là bao nhiêu?
	\dapso{$366{,}67$ triệu đồng.}
	\loigiai{
		Gọi $x$ là số tiền gửi ban đầu của người đó $ (x>0) $.\\
		Sau 5 năm, số tiền nhận được bằng số tiền gốc cộng với lãi suất:
		$$
		x + 0{,}1x \times 5 = 1{,}5x.
		$$
		Theo đề bài, tổng số tiền nhận được sau 5 năm là $550$ triệu đồng, do đó ta có phương trình:
		$$
		1{,}5x = 550.
		$$
		Giải phương trình ta có:
		$$
		x = \frac{550}{1{,}5} \approx 366{,}67.
		$$
		Vậy số tiền gửi ban đầu của người đó là $366{,}67$ triệu đồng.
	}
\end{vd}
\begin{vd}%[DCHT Toán 11 - KNTT - Nguyễn Hữu Đức] %[1K2B6-6]
	Bạn An muốn mua một món quà tặng mẹ nhân ngày mùng $8/3$. Bạn quyết định tiết kiệm từ ngày $1/2/2017$ đến hết ngày $6/3/2017$. Ngày đầu An có $5\,000$ đồng, kể từ ngày thứ hai số tiền An tiết kiệm được ngày sau cao hơn ngày trước mỗi ngày $1\,000$ đồng. Tính số tiền An tiết kiệm được để mua quà tặng mẹ.
	\dapso{$731\,000$ đồng.}
	\loigiai{
		Tính số ngày mà An tiết kiệm được từ ngày $1/2/2017$ đến hết ngày $6/3/2017$:\\
		Số ngày từ ngày $1/2/2017$ đến hết ngày $28/2/2017$ là $28$ ngày.\\
		Số ngày từ ngày $1/3/2017$ đến hết ngày $6/3/2017$ là $6$ ngày.\\
		Vậy An tiết kiệm được $28+6=34$ ngày.\\
		Gọi $u_n$ là số tiền An tiết kiệm được vào ngày thứ $n$ kể từ ngày $1/2/2017$.\\
		Theo đề ta có $u_1=5\,000$ đồng.\\
		Vì ngày sau An tiết kiệm được nhiều hơn ngày trước mỗi ngày $1\,000$ đồng nên $u_n=u_{n-1}+1\,000$, với $n\ge 2$.\\
		Vậy $(u_n)$ là một cấp số cộng với $u_1=5\,000$ và công sai $d=1\,000$.\\
		Tổng số tiền An tiết kiệm được trong $34$ ngày là:
		$$S_{34}=\dfrac{n}{2} \left(2u_1+33d\right)= \dfrac{34}{2} \left(2\cdot 5\,000+33\cdot 1\,000\right)=731\,000.$$
		Vậy số tiền An tiết kiệm được để mua quà tặng mẹ là $731\,000$ đồng.
	}
\end{vd}

\begin{vd}%[TH]%[DCHT Toán 11 - KNTT - Nguyễn Hữu Đức] %[1K2B6-6]
	[Cấp số nhân] Một hội đồng quản trị quyết định tăng lương cho nhân viên hàng năm theo tỷ lệ cố định. Ví dụ, lương của một nhân viên được tăng thêm $ 5 $\% so với năm trước. Hỏi nếu lương của một nhân viên là $ 10 $ triệu đồng/năm vào năm nay, thì lương của nhân viên đó sẽ là bao nhiêu vào năm thứ $ 5 $?
	\dapso{$12{,}1550625 $ triệu đồng/năm.}
	\loigiai{		
		Theo giả thiết, lương của nhân viên được tăng thêm $ 5 $ \% so với năm trước đó.
		\begin{itemize}
			\item Vậy lương của nhân viên vào năm thứ $ 2 $ sẽ là $ 10\cdot(1+0{,}05)=10{,}5 $ triệu đồng/năm.
			\item Tương tự, lương của nhân viên vào năm thứ $ 3 $ sẽ là $ 10{,}5 \cdot(1+0{,}05)=11{,}025 $ triệu đồng/năm.
			\item Lương của nhân viên vào năm thứ $ 4 $ sẽ là $ 11{,}025\cdot (1+0{,}05)=11{,}57625 $ triệu đồng/năm.
			\item Cuối cùng, lương của nhân viên vào năm thứ $ 5 $ sẽ là $ 11{,}57625\cdot(1+0{,}05)=12{,}1550625 $ triệu đồng/năm.
		\end{itemize}
		Vậy lương của nhân viên đó vào năm thứ $ 5 $ sẽ là $ 12{,}1550625 $ triệu đồng/năm.\\
		Chú ý: Lương của nhân viên đó vào năm thứ $ 5 $ sẽ là $ u_5=u_1+4d=10+4\cdot 10\cdot 0{,}05=12 $ triệu đồng chỉ đúng trong trường hợp lương của một nhân viên được tăng thêm $ 5 $\% so với năm đầu tiên.
	}
\end{vd}


\begin{vd}%[TH]%%[DCHT Toán 11 - KNTT - Nguyễn Hữu Đức] %[1K2B6-6]
	Hùng đang tiết kiệm để mua một cây guitar. Trong tuần đầu tiên, anh ta để dành  $ 42 $ đô la, và trong mỗi tuần tiết theo, anh ta đã thêm $ 8 $ đô la vào tài khoản tiết kiệm của mình. Cây guitar Hùng cần mua có giá $ 400 $ đô la. Hỏi vào tuần thứ bao nhiêu thì anh ấy có đủ tiền để mua cây guitar đó?
	\dapso{$ n=46 $.}
	\loigiai{
		Gọi $ n $ là số tuần anh ta đã thêm $ 8 $ đô la vào tài khoản tiết kiệm của mình.\\
		Số tiền anh ta tiết kiệm được sau $ n $ tuần đó là $ S=42+8n $. \\
		Theo bài ra $ S=42+8n\ge 400\Leftrightarrow n\ge 44,75\Rightarrow n=45 $.\\
		Vậy kể cả tuần đầu thì tuần thứ $ 46 $ anh ta có đủ tiền để mua cây guitar đó.
	}
\end{vd}

\begin{vd}%[DCHT Toán 11 - KNTT - Nguyễn Hữu Đức]%[1K2Y6-6]
	[Cấp số nhân] Hàng tháng ông An gửi vào ngân hàng một số tiền như nhau là $5\,000\,000$ đồng (vào ngày đầu mỗi tháng) với lãi suất $0{,}5\%$ một tháng, biết tiền lãi của tháng trước được nhập vào tiền gốc của tháng sau. Hỏi sau $36$ tháng ông An nhận được số tiền vốn và lãi là bao nhiêu? (làm tròn đến hàng đơn vị).
	\dapso{ $ 197\,663\,927 $ đồng.}
	\loigiai{
		Gọi $a$ là số tiền ông An gửi vào hàng tháng, $r$ là lãi suất trên một tháng và $P_n$ là số tiền vốn và lãi ông An nhận được sau $n$ tháng.
		\begin{itemize}
			\item Sau một tháng, ông An có số tiền là $P_1=a+ar=a(1+r)$.
			\item Đầu tháng thứ hai, ông An có số tiền là $P_1+a=a(1+r)+a$.
			\item Sau hai tháng, ông An có số tiền là $P_2=a(1+r)+a+\left[a(1+r)+a\right]r=a\left[(1+r)^2+(1+r)\right]$.
			\item Cuối tháng thứ $36$, ông An có số tiền là
			\begin{align*}
				P_{36}&=a\left[(1+r)^{36}+(1+r)^{35}+\ldots+(1+r)\right]\\
				&=a(1+r)\dfrac{(1+r)^{36}-1}{r}\\
				&=5000000\cdot (1+0{,}005)\cdot\dfrac{(1+0{,}005)^{36}-1}{0{,}005}\\
				&\approx 197\,663\,927 \quad \text{(đồng)}.
			\end{align*}
		\end{itemize}
	}
\end{vd}

\begin{vd}[VDT]%[DCHT Toán 11 - KNTT - Nguyễn Hữu Đức] %[1K2Y6-6]
	Một xưởng có đăng tuyển công nhân với đãi ngộ về lương như sau: Trong quý đầu tiên thì xưởng trả là $ 6 $ triệu đồng/quý và kể từ quý thứ $ 2 $ sẽ tăng lên $ 0{,}5 $ triệu cho $ 1 $ quý. Hỏi với đãi ngộ trên thì sau $ 5 $ năm làm việc tại xưởng, tổng số lương của công nhân đó là bao nhiêu?
	\dapso{ $ 215 $ triệu đồng.}
	\loigiai{
		Gọi $u_n$ (triệu đồng) là số lương của công nhân trong quý thứ $n$.\\
		Theo đề:\\
		Quý đầu: $ u_1 = 6 $ triệu.\\
		Các quý tiếp theo: $ u_{n+1} = u_{n} + 0,5 $ với $\forall n \ge 1$.\\
		Mức lương của công nhân mỗi quý là $ 1 $ số hạng của dãy số $ u_n $. Mặt khác, lương của quý sau hơn lương quý trước là $ 0,5 $ triệu nên dãy số $ u_n $ là một cấp số cộng với công sai $ d = 0{,}5 $.\\
		Ta biết $ 1 $ năm sẽ có $ 4 $ quý nên $ 5 $ năm sẽ có $ 5\cdot 4 = 20 $ quý. Theo yêu cầu của đề bài ta cần tính tổng của $ 20 $ số hạng đầu tiên của cấp số cộng ($ u_n $).\\
		Lương tháng quý $ 20 $ của công nhân: $ u_{20} = 6 + (20 - 1)\cdot 0{,}5 = 15{,}5 $ triệu đồng.\\
		Tổng số lương của công nhân nhận được sau $ 5 $ năm làm việc tại xưởng: $ S_{12}=20\cdot (6+15,5)2=215 $ (triệu đồng).
	}
\end{vd}

% \subsubsection{Bài tập tự luận}
 
%[DCHT Toán 11 - KNTT - Nguyễn Hữu Đức] %[1K2B6-6]
% \begin{bt}%[NB]%[DCHT Toán 11 - KNTT - Nguyễn Hữu Đức] %[1K2B6-6]
% 	Sinh nhật bạn của An vào ngày $ 01 $ tháng năm. An muốn mua một món quà sinh nhật cho bạn nên quyết định bỏ ống heo $ 100 $ đồng vào ngày $ 01 $ tháng $ 01 $ năm $ 2016 $, sau đó cứ liên tục ngày sau hơn ngày trước $ 100 $ đồng. Hỏi đến ngày sinh nhật của bạn, An đã tích lũy được bao nhiêu tiền? (thời gian bỏ ống heo tính từ ngày $ 01 $ tháng $ 01 $ năm $ 2016 $ đến ngày $ 30 $ tháng $ 04 $  năm $ 2016 $).\dapso{$ 738\,100 $ đồng.}
% 	\loigiai{
% 		Từ ngày $1$ tháng $1$ năm $2016$ đến ngày $30$ tháng $4$ năm $2016$ có tổng cộng $31+29+31+30=121$ ngày.\\
% 		Gọi $S$ là số tiền An tích lũy được vào ngày sinh nhật của bạn.\\
% 		Do An bỏ được $100$ đồng vào ngày đầu tiên nên số tiền An tích lũy được vào ngày thứ $n$ là 
% 		$$S= 100 + 100(n-1).$$
% 		Vậy tổng số tiền An tích lũy được là:
% 		$$
% 		S=100 + 200 + \cdots + 12\,100 = \frac{121(100 + 12\,100)}{2} = 738\,100
% 		.$$
% 		Vậy An đã tích lũy được $738\,100$ đồng vào ngày sinh nhật của bạn.}
% \end{bt}

% \begin{bt}%[TH]%[DCHT Toán 11 - KNTT - Nguyễn Hữu Đức] %[1K2Y6-6]
% 	Người ta trồng $ 3\,003 $ cây theo dạng một hình tam giác như sau: hàng thứ nhất trồng $ 1 $ cây, hàng thứ hai trồng $ 2 $ cây, hàng thứ ba trồng $ 3 $ cây... cứ tiếp tục trồng như thế cho đến khi hết số cây. Số hàng cây được trồng là bao nhiêu?
% 	\dapso{$ 77 $ hàng.}
% 	\loigiai{
% 		Tổng số cây trồng được là $1 + 2 + 3 + \cdots + n$, nghĩa là tổng của $n$ số tự nhiên đầu tiên. Ta cần tìm số $n$ để tổng này bằng $3003$.\\
% 		Ta có công thức tổng của $n$ số tự nhiên đầu tiên là:
% 		$$
% 		1 + 2 + 3 + \cdots + n = \frac{n(n+1)}{2}.
% 		$$
% 		Giải phương trình:
% 		$$
% 		\frac{n(n+1)}{2} = 3\,003.
% 		$$
% 		Ta có:
% 		$$
% 		n(n+1) = 6\,006
% 		\Rightarrow n=77.$$
% 		Vậy số hàng cây được trồng là $77$.
% 	}
% \end{bt}
% \begin{bt}%[TH]%[DCHT Toán 11 - KNTT - Nguyễn Hữu Đức] %[1K2B6-6]
% 	Một công ty định mức sản phẩm hàng tháng theo cấp số cộng. Ví dụ, sản lượng hàng tháng của một công ty được tăng thêm $10$ sản phẩm so với tháng trước. Nếu công ty sản xuất được $ 100 $ sản phẩm trong tháng này, hỏi công ty sẽ sản xuất được bao nhiêu sản phẩm trong tháng thứ $ 12 $?
% 	\dapso{$ 210 $ sản phẩm.}
% 	\loigiai{
% 		Công thức cấp số cộng được sử dụng để tính sản lượng hàng tháng của công ty. Nếu công ty sản xuất được $100$ sản phẩm trong tháng này và sản lượng hàng tháng được tăng thêm $10$ sản phẩm so với tháng trước, ta có thể sử dụng công thức sau để tính sản lượng hàng tháng của công ty trong tháng thứ $12$:
% 		$$
% 		a_n = a_1 + (n-1)d.
% 		$$
% 		Trong đó $a_1$ là sản lượng hàng tháng ban đầu, $d$ là công sai và $n$ là số tháng.\\
% 		Với bài toán này, ta có: $a_1 = 100$, $d = 10$, $n = 12$.\\
% 		Sản lượng hàng tháng của công ty trong tháng thứ $12$ là:
% 		$$
% 		a_{12} = a_1 + (n-1)d = 100 + (12-1) \times 10 = 210.
% 		$$
% 		Vậy công ty sẽ sản xuất được $210$ sản phẩm trong tháng thứ $12$.
% 	}
% \end{bt}

\subsubsection{Câu hỏi trắc nghiệm}
\Opensolutionfile{ans}[ans/ans-1K2-2-dang4]
\begin{ex}%[TH]%[DCHT Toán 11 - KNTT - Nguyễn Hữu Đức]%[1K2B6-6]
	Một công ty đang cần tuyển dụng thêm nhân viên. Công ty quyết định tăng số lượng nhân viên hàng tháng theo cấp số cộng. Nếu công ty đã có $ 20 $ nhân viên và quyết định tăng thêm $ 2 $ nhân viên hàng tháng, hỏi sau bao nhiêu tháng công ty sẽ có 50 nhân viên?
	\choice
	{$19$ tháng}
	{\True $16$ tháng}
	{$36$ tháng}
	{$26$ tháng}
	\loigiai{
		Để giải bài toán này, ta có thể sử dụng công thức cấp số cộng:
		$$
		a_n = a_1 + (n-1) \times d.
		$$
		Trong đó $a_1$ là số lượng nhân viên ban đầu, $d$ là số lượng nhân viên tăng hàng tháng và $n$ là số tháng.\\
		Ta cần tìm số tháng $n$ để công ty có được $50$ nhân viên. Thay các giá trị vào công thức cấp số cộng ta có:
		$$
		50 = 20 + (n-1) \times 2.
		$$
		Suy ra:
		$$
		n = \frac{50 - 20}{2} + 1 = 16
		.$$
		Vậy sau $16$ tháng kể từ khi công ty quyết định tăng số lượng nhân viên hàng tháng theo cấp số cộng, công ty sẽ có được $50$ nhân viên.
	}
\end{ex}
\begin{ex}%[VD]%[DCHT Toán 11 - KNTT - Nguyễn Hữu Đức] %[1K2B6-6]
	Một người đang tăng cường luyện tập thể thao hàng ngày. Anh ta quyết định tăng mức độ luyện tập theo cấp số cộng hàng tuần. Nếu anh ta bắt đầu với mức luyện tập $ 30 $ phút mỗi ngày và tăng thêm $ 5 $ phút mỗi ngày, hỏi anh ta sẽ luyện tập được bao lâu để đạt được mức luyện tập $ 60 $ phút mỗi ngày?
	\choice
	{$16$ ngày}
	{\True $6$ ngày}
	{$9$ ngày}
	{$7$ ngày}
	\loigiai{
		Gọi $n$ là số ngày liên tiếp mà người đó tăng mức độ luyện tập. Theo đó, mức độ luyện tập của người đó sau $n$ ngày là:
		$$
		30 + 5n\, \text{(phút)}.
		$$
		Vì để đạt được mức luyện tập $ 60 $ phút mỗi ngày nên:
		$$
		30 + 5n = 60.
		$$
		Từ đó suy ra:
		$$
		n = \frac{60-30}{5} = 6.
		$$
		Vậy người đó cần luyện tập liên tiếp trong $6$ ngày để đạt được mức luyện tập $60$ phút mỗi ngày.	
	}
\end{ex}
\begin{ex}%[VD]%[DCHT Toán 11 - KNTT - Nguyễn Hữu Đức] %[1K2Y6-6]
	Nếu một công ty công nghệ mới thành lập có số lượng người dùng ban đầu là $ 10\,000 $ và mỗi tháng tăng thêm cố định $ 5\,000 $ lượng người dùng, thì sau bao lâu có số lượng người dùng là $ 1 $ triệu. 
	\choice
	{\True $198$ tháng}
	{$197$ tháng}
	{$18$ tháng}
	{$98$ tháng}
	
	\loigiai{
		Ta cần tính số tháng $n$ theo công thức sau:
		$$10\,000 + 5\,000n = 1\,000\,000.$$
		$$\Rightarrow n = \frac{1\,000\,000 - 10\,000}{5\,000} = 198.$$
		Vậy sau khoảng $198$ tháng (khoảng $16$ năm và $6$ tháng), công ty sẽ đạt được $1$ triệu người dùng.
	}
\end{ex}
\begin{ex}%[VDC]%[DCHT Toán 11 - KNTT - Nguyễn Hữu Đức] %[1K2Y6-6]
	Một nhà đầu tư đang đầu tư vào một quỹ đầu tư với mức lợi nhuận cố định hàng năm. Nếu nhà đầu tư đầu tư vào quỹ đầu tư với số tiền ban đầu là $ 20 $ triệu đồng và mức lợi nhuận hàng năm là $ 10 $\%, hỏi số tiền nhà đầu tư sẽ nhận được sau $ 7 $ năm?
	\choice
	{\True $34$ triệu đồng}
	{$14$ triệu đồng}
	{$30$ triệu đồng}
	{$39$ triệu đồng}
	\loigiai{
		Với số tiền ban đầu là $ 20 $ triệu đồng và mức lợi nhuận hàng năm là $ 10 $\%, ta có thể tính được số tiền nhà đầu tư sẽ nhận được sau $ 1 $ năm, sau đó sử dụng cấp số cộng để tính số tiền nhà đầu tư sẽ nhận được sau $ 7 $ năm.
		
		Số tiền nhà đầu tư sẽ nhận được sau $ 1 $ năm là:
		
		$ 20 $ triệu đồng $\times$ $ 10 $\% = $ 2 $ triệu đồng
		
		Số tiền nhà đầu tư sẽ nhận được sau $ 7 $ năm là:
		
		$ 2 $ triệu đồng $\times$ $ 7 $ năm + $ 20 $ triệu đồng = $ 34 $ triệu đồng
		
		Vậy sau $ 7 $ năm, nhà đầu tư sẽ nhận được tổng cộng $ 34 $ triệu đồng.
	}
\end{ex}
\begin{ex}%[VDC]%[DCHT Toán 11 - KNTT - Nguyễn Hữu Đức] %[1K2Y6-6]
	Một công ty sản xuất bánh kẹo tăng sản lượng sản phẩm của mình lên mỗi tháng. Nếu sản lượng ban đầu là $ 1\,000 $ sản phẩm, một sản phẩm lợi nhuận $ 1 $ USD và tăng thêm $ 200 $ sản phẩm mỗi tháng, thì sau bao nhiêu tháng lợi nhuận công ty $ 1 $ triệu đô.
	\choice
	{$8\,000$ tháng}
	{$7\,000$ tháng}
	{$9\,000$ tháng}
	{\True $5\,000$ tháng}
	\loigiai{
		Để tính thời gian công ty đạt được lợi nhuận 1 triệu đô, chúng ta cần biết lợi nhuận của công ty đạt được bao nhiêu sau mỗi tháng.\\
		Giả sử sản lượng ban đầu là $1\,000$ sản phẩm một sản phẩm lợi nhuận $1$ USD và tăng thêm $200$ sản phẩm mỗi tháng. Ta có thể tính được lợi nhuận của công ty sau mỗi tháng như sau:
		\begin{itemize} 
			\item Tháng 1: $1\,000 \times 1 = 1000$ USD.
			\item Tháng 2: $(1\,000 + 200) \times 1 = 1200$ USD.
			\item Tháng 3: $(1\,000 + 2 \times 200) \times 1 = 1\,400$ USD.
			\item Tháng 4: $(1\,000 + 3 \times 200) \times 1 = 1\,600$ USD.
			\item Tháng $n$: $(1\,000 + (n - 1) \times 200) \times 1 = (n - 1) \times 200 + 1\,000$ USD.
		\end{itemize}
		Để tính thời gian để công ty đạt được lợi nhuận $1$ triệu đô, ta giải phương trình sau:
		
		$(n - 1) \times 200 + 1\,000 = 10^6$
		
		$\Rightarrow (n - 1) \times 200 = (10^6 - 1000)$
		
		$\Rightarrow n - 1 = \dfrac{10^6 - 1\,000}{200}$
		
		$\Rightarrow n = \dfrac{10^6 - 1\,000}{200} + 1$
		
		$\Rightarrow n = 5\,001$
		
		Vậy sau $5\,000$ tháng, công ty sẽ đạt được lợi nhuận $1$ triệu đô.
	}
\end{ex}
\begin{ex}%[VDC]%[DCHT Toán 11 - KNTT - Nguyễn Hữu Đức] %[1K2Y6-6]
	Một công ty tăng lương cho nhân viên hàng năm bằng cách thêm một số tiền cố định vào lương của họ. Ví dụ: Nếu lương ban đầu của một nhân viên là $ 10 $ triệu đồng và công ty tăng lương $ 2 $ triệu đồng mỗi năm, thì lương của nhân viên sẽ là bao nhiêu nếu làm cho công ty $ 19 $ năm?
	\choice
	{$ 16 $ triệu đồng}
	{$ 26 $ triệu đồng}
	{$ 28 $ triệu đồng}
	{\True $ 46 $ triệu đồng}
	\loigiai{
		Do tăng lương cho nhân viên hàng năm bằng cách thêm một số tiền cố định nên ta có thể sử dụng công thức tính số hạng thứ $ n $ của cấp số cộng
		$ a_n = a_1 + (n - 1)d $.\\
		Ở bài toán này, ta có:\\
		$ a_1 = 10 $ (triệu đồng) là lương ban đầu của nhân viên.\\
		$ d = 2 $ (triệu đồng) là công sai của cấp số cộng.\\
		$ n = 19  $ là số thứ tự của số hạng.\\
		Ta thay các giá trị này vào công thức trên để tính lương của nhân viên sau $ 19 $ năm:\\
		$ a_{19} = 10 + (19 - 1)2 \Rightarrow$
		$ a_{19} = 46 $ (triệu đồng).\\
		Vậy lương của nhân viên sau $ 19 $ năm làm việc cho công ty là $ 46 $ triệu đồng.
	}
\end{ex}

\begin{ex}%[VDC]%[DCHT Toán 11 - KNTT - Nguyễn Hữu Đức] %[1K2Y6-6]
	Tài sản thường bị khấu hao khiến chúng có tuổi thọ hữu ích giới hạn. Ví dụ, nếu một công ty mua một chiếc xe tải với giá $ 35\,000 $ đô la và nó bị khấu hao với tốc độ không đổi là $ 700 $ đô la mỗi tháng, thì sau bao lâu giá trị của nó còn $ 5\,000 $ đô la.
	\choice
	{$ x = 23 $ tháng}
	{\True  $ x = 43 $ tháng}
	{$ x = 41 $ tháng}
	{$ x = 40 $ tháng}
	\loigiai{
		\textit{Cách 1:} Thời gian để giá trị của chiếc xe tải trên được khấu hao xuống còn $5.000 $ đô la có thể được tính bằng cách sử dụng công thức sau:\\
		Giá trị khởi đầu của chiếc xe tải là $35\,000$
		Giá trị cuối cùng của chiếc xe tải là $5\,000$
		Tốc độ khấu hao tương ứng $700$/tháng\\
		Để tìm ra thời gian cần thiết để giá trị của chiếc xe tải giảm xuống còn $5.000$, ta cần tìm số tháng được khấu hao.\\
		Giả sử số tháng cần khấu hao là $ x $ tháng.\\
		Giá trị của chiếc xe tải sau $ x $ tháng khấu hao được tính bằng:\\ 
		$35\,000 - 700x = 5\,000$.\\
		Giải phương trình trên ta có: $ x \approx 43 $ tháng\\
		Vì vậy, sau $ 43 $ tháng, giá trị của chiếc xe tải sẽ giảm xuống còn $5\,000$.
		Ngoài ra ta có thể giải theo cấp số cộng như sau:\\
		\textit{Cách 2:} Ta có thể sử dụng cộng thức tính số hạng thứ $ n $ của cấp số cộng
		$ a_n = a_{1} + (n - 1)d $
		\begin{itemize}
			\item $ u_1 = 35\,000 $ (đô la) là giá trị ban đầu của xe tải.
			\item $ d = -700 $ (đô la) là công sai của cấp số cộng (âm vì giá trị xe tải giảm).
			\item $ a_n = 5\,000 $ (đô la) là giá trị cuối cùng của xe tải.
		\end{itemize}
		Ta thay các giá trị này vào công thức trên để tính số tháng mà xe tải bị khấu hao đến $ 5\,000 $ đô la:
		$$ 5\,000 = 35\,000 + (n - 1)(-700)\Rightarrow n = 43{,}857.$$
		Vậy sau khoảng $ 43{,}857 $ tháng, tức là khoảng $ 3 $ năm và $ 7 $ tháng, giá trị của xe tải sẽ còn khoảng $ 5\,000 $ đô la.
	}
\end{ex}
\begin{ex}%[VDC]%[DCHT Toán 11 - KNTT - Nguyễn Hữu Đức] %[1K2Y6-6]
	Các thiết bị điện tử như máy tính, điện thoại, hoặc máy ảnh thường bị khấu hao nhanh chóng do sự phát triển của công nghệ mới. Ví dụ, nếu một người mua một máy tính Macbook với giá $ 2\,000 $ đô la và nó bị khấu hao với tốc độ không đổi là $ 100 $ đô la mỗi tháng, thì giá trị của Macbook còn lại $ 1\,000 $ đô la sau bao nhiêu tháng?
	%\dapso{$ 11 $ tháng.}
	\choice
	{$ x = 12 $ tháng}
	{$ x = 43 $ tháng}
	{\True  $ x = 11 $ tháng}
	{$ x = 10 $ tháng}
	\loigiai{
		Để giải bài toán này, ta có thể sử dụng công thức tính số hạng thứ $ n $ của cấp số cộng $ a_n = a + (n - 1)d. $\\
		Ở bài toán này, ta có:\\
		$ a = 2\,000 $ (đô la) là giá trị ban đầu của máy tính Macbook.\\
		$ d = -100 $ (đô la) là công sai của cấp số cộng (âm vì giá trị máy tính giảm).\\
		$ a_n = 1\,000 $ (đô la) là giá trị cuối cùng của máy tính Macbook.\\
		Ta thay các giá trị này vào công thức trên để tính số tháng mà máy tính bị khấu hao đến $ 1\,000 $ đô la:
		$$ 1\,000 = 2\,000 + (n - 1)(-100)\Rightarrow n=11. $$
		Vậy sau $ 11 $ tháng, giá trị của máy tính Macbook sẽ còn $ 1\,000 $ đô la.
	}
\end{ex}
\begin{ex}%[VDC]%[DCHT Toán 11 - KNTT - Nguyễn Hữu Đức] %[1K2Y6-6]
	Ban đầu có 1m$^2$ bèo sinh sôi trên mặt hồ biết tốc độ sinh sôi ngày sau hơn ngày trước $ 0{,}5 $m$^2 $. Biết diện tích mặt hồ nước là $ 120 $m$^2 $ hỏi sau bao lâu bèo phủ đầy mặt hồ?
	%\dapso{$ 238 $ ngày}
	\choice
	{$ x = 120 $ tháng}
	{$ x = 143 $ tháng}
	{\True  $ x = 238 $ tháng}
	{$ x = 130 $ tháng}
	\loigiai{
		Giả sử sau $ x $ ngày, diện tích của bèo phủ đầy mặt hồ là $ S m^2 $.
		
		Theo đề bài, ta biết được rằng:
		\begin{itemize}
			\item Tốc độ sinh sôi của bèo là $ 0{,}5 $m$^2 $/ngày.
			\item Ban đầu, diện tích của bèo là  1 m$^2 $.
			\item Diện tích mặt hồ là  $ 120 $m$^2 $.
		\end{itemize}
		Vậy ta có phương trình sau đây:	$ S = 1 + 0{,}5x. $\\
		Điều kiện để bèo phủ đầy mặt hồ là $ S = 120 $.\\
		$ 1 + 0{,}5x = 120 $ hay	$ 0{,}5x = 119 $ $\Rightarrow x = 238 $ ngày.\\
		Vậy sau $ 238 $ ngày, bèo sẽ phủ đầy mặt hồ.
	}
\end{ex}
\begin{ex}%[VDC]%[DCHT Toán 11 - KNTT - Nguyễn Hữu Đức] %[1K2Y6-6]
	Nhà hát lớn Dạ Cỗ Vĩ Lan ở An Cư có hàng ghế đầu kí hiệu dãy A là $50$ chỗ hàng ghế, sau dãy B là $48$ chỗ và như thế hàng sau ít hơn hàng trước $ 2 $ ghế, biết hàng cuối cùng có $ 10 $ ghế. Tính tổng số dãy ghế và tổng số chỗ ngồi?
	%\dapso{ $21$ dãy và $ 630 $ chỗ.}
	\choice
	{\True  $21$ dãy và $ 630 $ chỗ}
	{$20$ dãy và $ 630 $ chỗ}
	{$11$ dãy và $ 630 $ chỗ}
	{$21$ dãy và $ 930 $ chỗ}
	\loigiai{
		Gọi $n$ là số dãy ghế. Theo đề bài, ta có:
		$$
		\begin{cases}
			S=50 + 48 + \cdots + 10 = \dfrac{50+10}{2}n \\
			S=\dfrac{2.50+(n-1)\cdot (-2)}{2}n
		\end{cases}
		$$
		Từ phương trình đầu tiên, ta có:
		$$
		S = 50 + 48 + \cdots + 10 = \frac{50+10}{2}n = 30n.
		$$
		Từ phương trình thứ hai, ta có:
		$$
		S = \frac{2\cdot 50+(n-1)\cdot(-2)}{2}n = (50 - n + 1)n = (51 - n)n.
		$$
		Do đó, ta có:
		$$
		30n = (51 - n)n
		\Rightarrow n=21.$$
		Vậy $n = 21$ dãy ghế và $ 30\cdot 21=630 $ ghế.
		
	}
\end{ex}
\begin{ex}%[VDC]%[DCHT Toán 11 - KNTT - Nguyễn Hữu Đức] %[1K2Y6-6]
	Người ta trồng  cây theo dạng một hình tam giác như sau: hàng thứ nhất trồng $ 1 $ cây, hàng thứ hai trồng $ 3 $ cây, hàng thứ ba trồng $ 5 $ cây,... cứ tiếp tục trồng như thế cho đến khi hết số cây là $ 6\,561 $. Số hàng cây được trồng là bao nhiêu?
	%\dapso{ $ 81 $ hàng.}
	\choice
	{\True  $ 81 $ hàng}
	{$ 16 $ hàng}
	{$ 100 $ hàng}
	{$ 89 $ hàng}
	\loigiai{
		Để giải bài toán này, ta cần tìm số hàng cây được trồng cho đến khi tổng số cây là $ 2023 $. 
		\begin{itemize}
			\item Hàng thứ nhất trồng $ 1 $ cây. 
			\item Hàng thứ hai trồng $ 3 $ cây ($ 1 $ cây $ + 2 $ cây).
			\item Hàng thứ ba trồng $ 5 $ cây ($ 1 $ cây $ + 2 $ cây $ + 2 $ cây).
			\item ...
		\end{itemize}
		Vậy ta thấy rằng số cây trồng trong hàng thứ $n$ là $(n-1)\cdot 2+1$. \\
		Số cây được trồng trong $n$ hàng đầu tiên là: 
		$$1 + 3 + 5 + ... + (2n-1) = n^2.$$ 
		Để tìm số hàng cây được trồng cho đến khi tổng số cây là $ 6561 $, ta giải phương trình sau:\\ 
		$n^2 = 6\,561.$ 
		Vậy số hàng cây được trồng là $ 81 $ hàng.
	}
\end{ex}
\begin{ex}%[VDC]%[DCHT Toán 11 - KNTT - Nguyễn Hữu Đức] %[1K2Y6-6]
	Người ta thả một $ 1 $ m$^2$ lá bèo vào một hồ nước. Kinh nghiệm cho thấy sau $ x $ giờ, bèo sẽ sinh sôi kín cả mặt hồ $ 500 $ m$^2 $. Biết rằng sau mỗi giờ, lượng lá bèo tăng thêm $ 0{,}5 $ m$^2 $ và tốc độ tăng không đổi tìm $ x $?
	%\dapso{ $999$ giờ.}
	\choice
	{$888$ giờ}
	{$777$ giờ}
	{\True  $999$ giờ}
	{$700$ giờ}
	\loigiai{
		Bài toán này có thể giải bằng cách sử dụng công thức tăng trưởng của bèo. Giả sử lượng lá bèo ban đầu là $ 1 $ m$^2$, sau mỗi giờ lượng lá bèo tăng thêm $ 0{,}5 $ m$^2$. Sau $x$ giờ, lượng lá bèo đã phủ kín mặt hồ $ 500 $ m$^2$. Ta có thể viết phương trình sau:
		$$1 + 0{,}5x = 500.$$
		Giải phương trình ta được:
		$$x = \frac{500-1}{0{,}5} \approx 999.$$
		Vậy sau khoảng $999$ giờ (khoảng 41 ngày), lượng lá bèo sẽ phủ kín mặt hồ $ 500 $ m$^2$.
	}
\end{ex}
\Closesolutionfile{ans}
% \begin{indapan}{10}
% 	{ans/ans-1K2-2-dang6}
% \end{indapan}
% %%Bài 7. CSN
\def\tenchude{CẤP SỐ NHÂN}
\setcounter{section}{6}
\setcounter{dang}{0}
\setcounter{ex}{0}
\setcounter{bt}{0}
\setcounter{vd}{0}
\section{Cấp số nhân}
\subsection{Tóm tắt lý thuyết}
\begin{tomtat}
	\subsubsection{Định nghĩa} 
	Cấp số nhân là một dãy số (hữu hạn hoặc vô hạn) mà trong đó, kể từ số hạng thứ hai, mỗi số hạng đều bằng tích một số đứng ngay trước nó với một số $ q $ không đổi, nghĩa là:
	$$ u_{n}=u_{n-1}\cdot q\,\,\text{với}\,\forall n\in \mathbf{N}{,}\,n\ge 2 $$
	Số $ q $ được gọi là công bội của cấp số nhân
	\subsubsection{Số hạng tổng quát của cấp số nhân}
	Nếu cấp số nhân $ (u_n) $ có số hạng đầu là $ u_1 $ và công bội $ q $ thì số hạng tổng quát $ u_n $ của nó được xác định bởi công thức:
	$$u_n = u_1 \cdot q^{n-1}\,,n\ge 2$$
	\subsubsection{Tổng của $ n $ số hạng đầu tiên của cấp số nhân}
	Giả sử $ (u_n) $ là cấp số nhân có công bội $ q\ne 1 $. Đặt $ S_n=u_1+u_2+\cdots +u_n, $ khi đó
	$$S_n = u_1\cdot\frac{1-q^n}{1-q}.$$
	\begin{note}
		Khi $ q=1 $ thì $ S_n=n\cdot u_1 $.
	\end{note}	
	\begin{itemize}
		\item Công bội của cấp số nhân: $q = \sqrt[n-1]{\frac{u_n}{u_1}}$.
		\item Số hạng đầu tiên của cấp số nhân: $u_1 = \frac{u_n}{q^{n-1}}$.
		\item $ a,b,c $ là ba số hạng liên tiếp cấp số nhân thì $ a\cdot c=b^2 $. 
	\end{itemize}	
\end{tomtat}
\subsection{Các dạng toán thường gặp}
\begin{dang}{Nhận diện cấp số nhân, công bội $ q $}
	Để nhận diện (chứng minh) mỗi dãy số là cấp số nhân, ta làm như sau:\\
	Chứng minh $ u_{n+1}=u_nq $, $ \forall n\in\mathbb{N}^* $ và $ q $ là một số không đổi.\\
	Nếu $ u_n\ne 0 $, $ \forall n\in\mathrm{N}^* $ thì ta lập tỉ số $ \dfrac{u_{n+1}}{u_n}=k $.
	\begin{itemize}
		\item Nếu $ k $ là hằng số thì $ (u_n) $ là cấp số nhân với công bội $ q=k $.
		\item Nếu $ k $ phụ thuộc vào $ n $ thì $ (u_n) $ không phải là cấp số nhân.
	\end{itemize}
	Để chứng minh dãy $ (u_n) $ không phải là một cấp số nhân. Khi đó, ta chỉ cần chỉ ra ba số hạng liên tiếp không tạo thành một cấp số nhân, chẳng hạn $ \dfrac{u_3}{u_2}\ne \dfrac{u_2}{u_1} $.\\
	Để chứng minh ba số $ a,b,c $ theo thứ tự đó lập được một cấp số nhân, thì ta chứng minh $ ac=b^2 $ hoặc $ |b|=\sqrt{ac} $.	
\end{dang}
\subsubsection{Ví dụ minh hoạ}
\begin{vd}%[NB]%[DCHT Toán 11 - KNTT -Tên Huỳnh Thanh Chí]%[1K2Y7-1]
	Dãy số $ 1;1;1;1;\ldots $ có phải là một cấp số nhân hay không?
	\dapso{Dãy số $ 1;1;1;1;\ldots $ là một cấp số nhân.}
	\loigiai{
	Dễ thấy $ \dfrac{u_2}{u_1}=\dfrac{u_3}{u_2}=\ldots=1 $ là một số không đổi.\\
	Do đó dãy số $ 1;1;1;1;\ldots $ là một cấp số nhân.
	}
\end{vd}
\begin{vd}%[TH]%[DCHT Toán 11 - KNTT -Tên Huỳnh Thanh Chí] %[ID6 chương trình mới]
	Dãy số $ u_n=3^n $ có phải là một cấp số nhân không? Nếu có, hãy tìm công bội của cấp số nhân đó.
	\dapso{$ (u_n) $ là cấp số nhân với công bội $ q=3 $.}
	\loigiai{
	Ta có $ \dfrac{u_{n+1}}{u_n}=\dfrac{3^{n+1}}{3^n}=\dfrac{3^n\cdot 3}{3^n}=3 $ là số không đổi nên $ (u_n) $ là cấp số nhân với công bội $ q=3 $.
	}
\end{vd}
\begin{vd}%[TH]%[DCHT Toán 11 - KNTT -Tên Huỳnh Thanh Chí]%[1K2B7-1]
	Dãy số $ \heva{& u_1=3\\ & u_{n+1}=\dfrac{9}{u_n}} $ có phải là một cấp số nhân không? Nếu có, hãy tìm công bội của cấp số nhân đó.
	\dapso{$ (u_n) $ là một cấp số nhân với công bội $ q=1 $.}
	\loigiai{
	Xét dãy số $ \heva{& u_1=3\\ & u_{n+1}=\dfrac{9}{u_n}} $ có
	$ \dfrac{u_{n+1}}{u_n}=\dfrac{9}{u_n}:\dfrac{9}{u_{n-1}}=\dfrac{u_{n-1}}{u_n}\Rightarrow u_{n+1}=u_{n-1},\forall n\ge 2 $.\\
	Do đó ta có $ \heva{& u_1=u_3=u_5=\ldots=u_{2n+1}=\ldots \quad (1)\\ & u_2=u_4=u_6=\ldots=u_{2n}=\ldots \quad (2).} $\\
	Theo đề bài ta có $ u_1=3 \Rightarrow u_2=\dfrac{9}{u_1}=3 $ (3).\\
	Từ $ (1), (2) $ và $ (3) $ suy ra $ u_1=u_2=u_3=u_4=\ldots=u_{2n}=u_{2n+1}=\ldots $.\\
	Do đó $ (u_n) $ là một cấp số nhân với công bội $ q=1 $.
	}
\end{vd}
\begin{vd}%[TH]%[DCHT Toán 11 - KNTT -Tên Huỳnh Thanh Chí]%[1K2B7-1]
	Cho $ (u_n) $ là cấp số nhân có công bội $ q\ne 0,u_1\ne 0 $. Chứng minh rằng dãy số $ (v_n) $ với $ v_n=u_nu_{2n} $ cũng là một cấp số nhân.
	\dapso{$ (v_n) $ là một cấp số nhân với công bội là $ q^3 $.}
	\loigiai{
		Ta có $ \dfrac{v_n}{v_{n-1}}=\dfrac{u_nu_{2n}}{u_{n-1}u_{2(n-1)}}=\dfrac{u_1q^{n-1}\cdot u_1q^{2n-1}}{u_1q^{n-2}\cdot u_1q^{2n-3}}=q^3 $. Do đó $ (v_n) $ là một cấp số nhân với công bội là $ q^3 $.
	}
\end{vd}
\begin{vd}[VDT]%[DCHT Toán 11 - KNTT -Tên Huỳnh Thanh Chí]%[1K2K7-1]
	Cho dãy số $ (u_n) $ được xác định bởi $ \heva{& u_1=2\\ & u_{n+1}=4u_n+9},\forall n\in\mathbb{N}^* $. Chứng minh rằng dãy số $ (v_n) $ xác định bởi $ v_n=u_n+3,\forall n\in\mathbb{N}^* $ là một cấp số nhân. Hãy xác định số hạng đầu và công bội của cấp số nhân đó.
	\dapso{$ (v_n) $ là cấp số nhân với công bội $ q=4 $ và số hạng đầu $ v_1=5 $.}
	\loigiai{
	Ta có $ v_n=u_n+3 $ (1) và $ v_{n+1}=u_{n+1}+3 $ (2).\\
	Theo đề ta có $ u_{n+1}=4u_n+9 \Rightarrow u_{n+1}+3=4(u_n+3) $ (3).\\
	Thay (1) và (2)	vào (3) ta được $ v_{n+1}=4v_n \Rightarrow \dfrac{v_{n+1}}{v_n}=4,\forall n\in\mathbb{N}^* $.\\
	Suy ra $ (v_n) $ là cấp số nhân với công bội $ q=4 $ và số hàng đầu $ v_1=u_1+3=2+3=5 $.
	}
\end{vd}
\subsubsection{Bài tập tự luận}
 
\begin{bt}%[NB]%[DCHT Toán 11 - KNTT -Tên Huỳnh Thanh Chí]%[1K2Y7-1]
	Dãy số $25$; $5$; $1$; $\dfrac{1}{5}$; $\ldots$ có phải là một cấp số nhân không? Nếu có hãy tìm công bội của cấp số nhân đó.
	\dapso{Dãy số $25$; $5$; $1$; $\dfrac{1}{5}$; $\ldots$ là một cấp số nhân với công bội $ q=\dfrac{1}{5} $.}
	\loigiai{
	Ta có $ \dfrac{u_2}{u_1}=\dfrac{u_3}{u_2}=\ldots=\dfrac{1}{5} $ là một số không đổi.\\
	Do đó dãy số $25$; $5$; $1$; $\dfrac{1}{5}$; $\ldots$ là một cấp số nhân với công bội $ q=\dfrac{1}{5} $.
	}
\end{bt}
\begin{bt}%[NB]%[DCHT Toán 11 - KNTT -Tên Huỳnh Thanh Chí]%[1K2B7-1]
	Dãy số $1$; $n$; $n^2$; $n^3$; $ n^4 $; $\ldots$ (với $ n>1 $) có phải là một cấp số nhân không? Nếu có hãy tìm công bội của cấp số nhân đó.
	\dapso{Dãy số $1$; $n$; $n^2$; $n^3$; $ n^4 $; $\ldots$ (với $ n>1 $) là một cấp số nhân với công bội $ q=n $.}
	\loigiai{
		Ta có $ \dfrac{u_2}{u_1}=\dfrac{u_3}{u_2}=\ldots=n $ (với $ n>1 $) là một số không đổi.\\
		Do đó dãy số $1$; $n$; $n^2$; $n^3$; $ n^4 $; $\ldots$ (với $ n>1 $) là một cấp số nhân với công bội $ q=n $.
	}
\end{bt}
\begin{bt}%[TH]%[DCHT Toán 11 - KNTT -Tên Huỳnh Thanh Chí]%[1K2B7-1]
	Cho dãy số $ (u_n) $ được xác định bởi $ \heva{& u_1=2\\ & u_{n+1}=u_n^2} $. Hỏi dãy số $ (u_n) $ có là một cấp số nhân hay không?
	\dapso{Dãy số $ (u_n) $ không là một cấp số nhân.}
	\loigiai{
	Ta có $ u_2=u_1^2=4,u_3=u_2^2=16,u_4=u_3^2=256 $.\\
	Suy ra $ \dfrac{u_2}{u_1}=2 $; $ \dfrac{u_3}{u_2}=4 $ và $ \dfrac{u_4}{u_3}=16 $. Vì $ \dfrac{u_2}{u_1}\ne \dfrac{u_3}{u_2}\ne \dfrac{u_4}{u_3} $ nên $ (u_n) $ không là một cấp số nhân	
	}
\end{bt}
\begin{bt}%[TH]%[DCHT Toán 11 - KNTT -Tên Huỳnh Thanh Chí]%[1K2B7-1]
	Cho dãy số $ (u_n) $, biết $ u_1=2 $ và $ u_{n+1}=\dfrac{1}{3}u_n $. Chứng minh $ (u_n) $ là một cấp số nhân và tìm số hạng $ u_3 $.
	\dapso{}
	\loigiai{
	Ta có $ u_{n+1}=\dfrac{1}{3}u_n\Rightarrow \dfrac{u_{n+1}}{u_n}=\dfrac{1}{3} $ là một số không đổi nên $ (u_n) $ là một cấp số nhân với công bội là $ q=\dfrac{1}{3} $.\\
	Do đó $ u_3=u_2\cdot q=u_1\cdot q^2=2\cdot \dfrac{1}{3^2}=\dfrac{2}{9} $.
	}
\end{bt}
\begin{bt}%[TH]%[DCHT Toán 11 - KNTT -Tên Huỳnh Thanh Chí]%[1K2B7-1]
	Cho $ (u_n) $ là cấp số nhân có công bội $ q\ne 0,u_1\ne 0 $. Chứng minh rằng dãy số $ (v_n) $ với $ v_n=\dfrac{u_nu_{2n+1}}{4} $ cũng là một cấp số nhân.
	\dapso{$ (v_n) $ là một cấp số nhân với công bội là $ q^3 $.}
	\loigiai{
		Ta có $ \dfrac{v_n}{v_{n-1}}=\dfrac{\dfrac{u_nu_{2n+1}}{4}}{\dfrac{u_{n-1}u_{2(n-1)+1}}{4}}=\dfrac{u_1q^{n-1}\cdot u_1q^{2n}}{u_1q^{n-2}\cdot u_1q^{2n-2}}=q^3 $. Do đó $ (v_n) $ là một cấp số nhân với công bội là $ q^3 $.}
\end{bt}
\begin{bt}%[VD]%[DCHT Toán 11 - KNTT -Tên Huỳnh Thanh Chí]%[1K2K7-1]
	Cho dãy số $ (u_n) $ được xác định bởi $ \heva{& u_1=3\\ & u_{n+1}=2u_n-2},\forall n\in\mathbb{N}^* $. Chứng minh rằng dãy số $ (v_n) $ xác định bởi $ v_n=2u_n-4,\forall n\in\mathbb{N}^* $ là một cấp số nhân. Hãy xác định số hạng đầu và công bội của cấp số nhân đó.
	\dapso{$ (v_n) $ là cấp số nhân với công bội $ q=2 $ và số hạng đầu $ v_1=2 $.}
	\loigiai{
	Ta có $ v_n=2u_n-4 $ (1) và $ v_{n+1}=2u_{n+1}-4 $ (2).\\
	Theo đề ta có $ u_{n+1}=2u_n-2 \Rightarrow 2u_{n+1}-4=2(2u_n-4) $ (3).\\
	Thay (1) và (2)	vào (3) ta được $ v_{n+1}=2v_n \Rightarrow \dfrac{v_{n+1}}{v_n}=2,\forall n\in\mathbb{N}^* $.\\
	Suy ra $ (v_n) $ là cấp số nhân với công bội $ q=2 $ và số hàng đầu $ v_1=2u_1-4=2\cdot 3-4=2 $.
	}
\end{bt}
\subsubsection{Câu hỏi trắc nghiệm}
\Opensolutionfile{ans}[ans/ans-1K2-3-dang1]
\begin{ex}%[DCHT Toán 11 - KNTT -Tên Huỳnh Thanh Chí]%[1K2Y7-1]
	Trong các dãy số sau, dãy số nào là một cấp số nhân?
	\choice
	{\True $128$; $-64$; $32$; $-16$; $8$; $\ldots$} 
	{$\sqrt{2}$; $2$; $4$; $4\sqrt{2}$; $\ldots$}
	{$5$; $6$; $7$; $8$; $\ldots$}
	{$15$; $5$; $1$; $\dfrac{1}{5}$; $\ldots$}
	\loigiai{
	Xét phương án $128$; $-64$; $32$; $-16$; $8$; $\ldots$. \\
	Có $ \dfrac{u_2}{u_1}=\dfrac{u_3}{u_2}=\ldots=-\dfrac{1}{2} $ là một số không đổi nên dãy số $128$; $-64$; $32$; $-16$; $8$; $\ldots$ là một cấp số nhân.
	}
\end{ex}
\begin{ex}%[DCHT Toán 11 - KNTT -Tên Huỳnh Thanh Chí]%[1K2Y7-1]
	Dãy số nào sau đây \textbf{không phải} là cấp số nhân?
	\choice
	{$1$; $-1$; $1$; $-1$; $\ldots$}
	{$3$; $3^2$; $3^3$; $3^4$; $\ldots$}
	{$a$; $a^3$; $a^5$; $a^7$; $\ldots$  $(a\not =0)$}
	{\True $\dfrac{1}{\pi}$; $\dfrac{1}{{\pi}^2}$; $\dfrac{1}{{\pi}^4}$; $\dfrac{1}{{\pi}^6}$; $\ldots$}
	\loigiai{
	Xét dãy $\dfrac{1}{\pi}$; $\dfrac{1}{{\pi}^2}$; $\dfrac{1}{{\pi}^4}$; $\dfrac{1}{{\pi}^6}$; $\ldots$ có $ \dfrac{u_2}{u_1}\ne \dfrac{u_3}{u_2} \left(\dfrac{1}{\pi}\ne\dfrac{1}{\pi^2} \right) $.\\
	Do đó dãy $\dfrac{1}{\pi}$; $\dfrac{1}{{\pi}^2}$; $\dfrac{1}{{\pi}^4}$; $\dfrac{1}{{\pi}^6}$; $\ldots$ không là một cấp số nhân.
	}
\end{ex}
\begin{ex}%[DCHT Toán 11 - KNTT -Tên Huỳnh Thanh Chí]%[1K2Y7-1]
	Dãy số $1$; $2$; $4$; $8$; $16$; $32$; $\ldots$ là một cấp số nhân với 
	\choice
	{Công bội là $1$ và số hạng đầu tiên là $2$}
	{\True Công bội là $2$ và số hạng đầu tiên là $1$}
	{Công bội là $2$ và số hạng đầu tiên là $2$}
	{Công bội là $1$ và số hạng đầu tiên là $1$}
	\loigiai{
	Ta có $ q=\dfrac{u_2}{u_1}=\dfrac{u_3}{u_2}=\ldots=2 $. \\
	Vậy dãy số đã cho là một cấp số nhân với công bội là $ q=2 $ và số hạng đầu tiên là $ u_1=1 $.
	}
\end{ex}
\begin{ex}%[DCHT Toán 11 - KNTT -Tên Huỳnh Thanh Chí]%[1K2Y7-1]
	Cho cấp số nhân $(u_n)$ với $u_1=-2$ và công bội $q=-5$. Viết bốn số hạng đầu tiên của cấp số nhân.
	\choice
	{$-2$; $10$; $50$; $-250$}
	{\True $-2$; $10$; $-50$; $250$}
	{$-2$; $-10$; $-50$; $-250$}
	{$-2$; $10$; $50$; $250$}
	\loigiai{
	Vì $ (u_n) $ là một cấp số nhân nên ta có $ u_{n+1}=u_nq $. \\
	Do đó $ u_2=u_1q=(-2)\cdot (-5)=10 $, $ u_3=u_2q=10\cdot (-5)=-50 $, $ u_4=u_3q=(-50)\cdot (-5)=250 $.\\
	Vậy bốn số hạng đầu tiên của cấp số nhân đó là $ -2; 10; -50; 250 $.
	}
\end{ex}
\begin{ex}%[DCHT Toán 11 - KNTT -Tên Huỳnh Thanh Chí]%[1K2B7-1]
	Một cấp số nhân có hai số hạng liên tiếp là $3$ và $12$. Số hạng tiếp theo của cấp số nhân là
	\choice
	{$15$}
	{$21$}
	{$36$}
	{\True $48$}
	\loigiai{
		Một cấp số nhân có hai số hạng liên tiếp là $3$ và $12$, do đó ta có $ q=\dfrac{u_{n+1}}{u_n}=\dfrac{12}{3}=4 $.\\
		Vậy số hạng tiếp theo của cấp số nhân đó là $ u_{n+2}=u_{n+1}q=12\cdot 4=48 $.
	}
\end{ex}
\begin{ex}%[DCHT Toán 11 - KNTT -Tên Huỳnh Thanh Chí]%[1K2B7-1]
	Cho cấp số nhân $(u_n)$ có số hạng tổng quát là $u_n=\dfrac{3}{2}\cdot 5^n$. Khi đó số hạng đầu $u_1$ và công bội $q$ là
	\choice
	{$u_1=\dfrac{3}{2}, q=\dfrac{1}{5}$} 
	{$u_1=\dfrac{3}{2}, q=5$}
	{$u_1=\dfrac{15}{2}, q=\dfrac{1}{5}$}
	{\True $u_1=\dfrac{15}{2}, q=5$}
	\loigiai{	
	Ta có $ u_1=\dfrac{3}{2}\cdot5^1=\dfrac{15}{2}$ và $ u_2=\dfrac{3}{2}\cdot 5^2=\dfrac{75}{2} $.\\
	Vì $ (u_n) $ là một cấp số nhân nên $ q=\dfrac{u_2}{u_1}=\dfrac{75}{2}:\dfrac{15}{2}=5 $.
	}
\end{ex}
\begin{ex}%[DCHT Toán 11 - KNTT -Tên Huỳnh Thanh Chí]%[1K2B7-1]
	Trong các dãy số $(u_n)$ cho bởi số hạng tổng quát $u_n$ sau, dãy số nào là một cấp số nhân?
	\choice
	{\True $u_n=\dfrac{1}{3^{n-2}}$}
	{$u_n=\dfrac{n}{3^n}$}
	{$u_n=(n+2)\cdot 3^n$}
	{$u_n=n^2$}
	\loigiai{
		\begin{itemize}
			\item Với $u_n=\dfrac{1}{3^{n-2}}$, ta có $ q=\dfrac{u_{n+1}}{u_n}=\dfrac{1}{3^{n-3}}:\dfrac{1}{3^{n-2}}=3 $ là một số không đổi.\\
			Vậy dãy số $ (u_n) $ có số hạng tổng quát $u_n=\dfrac{1}{3^{n-2}}$ là một cấp số nhân.
			\item Với $u_n=\dfrac{n}{3^n}$, ta có $ q=\dfrac{u_{n+1}}{u_n}=\dfrac{n+1}{3^{n+1}}:\dfrac{n}{3^n}=\dfrac{n+1}{3n} $ không phải là một số không đổi.\\
			Vậy dãy số $ (u_n) $ có số hạng tổng quát $u_n=\dfrac{n}{3^n}$ không là một cấp số nhân.
			\item Với $u_n=(n+2)\cdot 3^n$, ta có $ q=\dfrac{u_{n+1}}{u_n}=\dfrac{(n+3)\cdot 3^{n+1}}{(n+2)\cdot 3^n}=\dfrac{3(n+3)}{n+2} $ không phải là một số không đổi.\\
			Vậy dãy số $ (u_n) $ có số hạng tổng quát $u_n=(n+2)\cdot 3^n$ không là một cấp số nhân.
			\item Với $u_n=n^2$, ta có $ q=\dfrac{u_{n+1}}{u_n}=\dfrac{(n+1)^2}{n^2}=\left(1+\dfrac{1}{n}\right)^2 $ không là một số không đổi.\\
			Vậy dãy số $ (u_n) $ có số hạng tổng quát $u_n=n^2$ không là một cấp số nhân.
		\end{itemize}
	}
\end{ex}
\begin{ex}%[DCHT Toán 11 - KNTT -Tên Huỳnh Thanh Chí]%[1K2B7-1]
	Trong các dãy số $(u_n)$ cho bởi số hạng tổng quát $u_n$ sau, dãy số nào là một cấp số nhân?
	\choice
	{$u_n=7-3n$}
	{$u_n=7-3^n$}
	{$u_n=\dfrac{7}{3n}$}
	{\True $u_n=7\cdot 3^n$}
	\loigiai{
	\begin{itemize}
		\item Với $u_n=7-3n$, ta có $ q=\dfrac{u_{n+1}}{u_n}=\dfrac{7-3(n+1)}{7-3n}=\dfrac{4-3n}{7-3n} $ không phải là một số không đổi.\\
		Vậy dãy số $ (u_n) $ có số hạng tổng quát $u_n=7-3n$ không là một cấp số nhân.
		\item Với $u_n=7-3^n$, ta có $ q=\dfrac{u_{n+1}}{u_n}=\dfrac{7-3^{n+1}}{7-3^n}=\dfrac{7-3\cdot 3^n}{7-3^n}=1-\dfrac{2\cdot 3^n}{7-3^n} $ không phải là một số không đổi.\\
		Vậy dãy số $ (u_n) $ có số hạng tổng quát $u_n=7-3^n$ không là một cấp số nhân.
		\item Với $u_n=\dfrac{7}{3n}$, ta có $ q=\dfrac{u_{n+1}}{u_n}=\dfrac{7}{3(n+1)}:\dfrac{7}{3n}=\dfrac{n}{n+1} $ không phải là một số không đổi.\\
		Vậy dãy số $ (u_n) $ có số hạng tổng quát $u_n=\dfrac{7}{3n}$ không là một cấp số nhân.
		\item Với $u_n=7\cdot 3^n$, ta có $ q=\dfrac{u_{n+1}}{u_n}=\dfrac{7\cdot 3^{n+1}}{7\cdot 3^n}=3 $ là một số không đổi.\\
		Vậy dãy số $ (u_n) $ có số hạng tổng quát $u_n=7\cdot 3^n$ là một cấp số nhân.
	\end{itemize}
	}
\end{ex}
\begin{ex}%[DCHT Toán 11 - KNTT -Tên Huỳnh Thanh Chí]%[1K2B7-1]
	Mệnh đề nào sau đây \textbf{sai}?
	\choice
	{Dãy số có tất cả các số hạng bằng nhau là một cấp số nhân}
	{Dãy số có tất cả các số hạng bằng nhau là một cấp số cộng}
	{Một cấp số cộng có công sai dương là một dãy số tăng}
	{\True Một cấp số nhân có công bội $q>1$ là một dãy tăng}
	\loigiai{
	\begin{itemize}
		\item Dãy số có tất cả các số hạng bằng nhau là một cấp số nhân là mệnh đề đúng. \\
		Vì xét dãy số $ (u_n) $ là một cấp số nhân.
		Khi đó $ u_{n+1}=u_n\cdot q $ với $ u_n\ne 0,q=1 $ thì $ u_{n+1}=u_n $.
		\item Dãy số có tất cả các số hạng bằng nhau là một cấp số cộng là mệnh đề đúng.\\
		Vì $ u_{n+1}=u_n+d $, với $ d=0 $ thì $ u_{n+1}=u_n $.
		\item Một cấp số cộng có công sai dương là một dãy số tăng là mệnh đề đúng.\\
		Ta xét dãy $ (u_n) $ là một cấp số cộng có công sai $ d>0 $.\\
		Vì $ u_{n+1}=u_n+d \Rightarrow u_{n+1}-u_n=d>0 $. \\
		Do đó dãy $ (u_n) $ là dãy số tăng.
		\item Một cấp số nhân có công bội $q>1$ là một dãy tăng là mệnh đề \textbf{sai}.\\
		Ta xét dãy số $ (u_n) $ là một cấp số nhân có công bội $ q>1 $.\\
		Vì $ u_{n+1}=u_nq $ với $ u_n\ne 0,q>1 $. Khi đó $ u_{n+1}-u_n=u_nq-u_n=u_n(q-1) $.\\
		Nếu $ u_n<0 $ thì $ u_{n+1}-u_n=u_nq-u_n=u_n(q-1) <0 $. \\
		Do đó dãy $ (u_n) $ là dãy số giảm.
	\end{itemize}	
	}
\end{ex}
\begin{ex}%[DCHT Toán 11 - KNTT -Tên Huỳnh Thanh Chí]%[1K2K7-1]
	Cho dãy số $(u_n)$ được xác định bởi $ u_1=2,u_n=2u_{n-1}+3n-1 $. Công thức số hạng tổng quát của dãy số đã cho là biểu thức có dạng $ a2^n+bn+c $, với $ a,b,c\in\mathbb{Z},n\ge 2, n\in\mathbb{N} $. Khi đó tổng $ a+b+c $ có giá trị bằng
	\choice
	{$ -4 $}
	{$ 4 $}
	{\True $ -3 $}
	{$ 3 $}
	\loigiai{
	Ta có $ u_n=2u_{n-1}+3n-1 \Leftrightarrow u_n+3n+5=2\left[u_{n-1}+3(n-1)+5\right] $ với $ n\ge 2, n\in\mathbb{N} $.\\
	Đặt $ v_n=u_n+3n+5 $, ta có $ v_n=2v_{n-1} $ với $ n\ge 2, n\in\mathbb{N} $.\\
	Như vậy $ (v_n) $ là cấp số nhân với công bội $ q=2 $ và $ v_1=10 $.\\
	Do đó $ v_n=10\cdot 2^{n-1}=5\cdot 2^n $.\\
	Suy ra $ u_n+3n+5=5\cdot 2^n $ hay $ u_n=5\cdot 2^n-3n-5 $ với $ n\ge 2, n\in\mathbb{N} $.\\
	Vậy $ a=5,b=-3,c=-5 $, suy ra $ a+b+c=-3 $.
	}
\end{ex}
\Closesolutionfile{ans}
% \begin{indapan}{10}
% 	{ans/ans-1K2-3-dang1}
% \end{indapan}
\begin{dang}{Số hạng tổng quát của cấp số nhân}
	Dựa vào giả thuyết, ta lập một hệ phương trình chứa công bội $ q $ và số hạng đầu $ u_n $. Giải hệ phương trình này tìm được $ u_1 $ và $ q $.\\
	Nếu cấp số nhân $ (u_n) $ có số hạng đầu $ u_1 $ và công bội $ q $ thì số hạng tổng quát $ u_n $ được xác định bởi công thức $$ u_n=u_1\cdot q^{n-1} \text{ với } n\ge 2. $$
\end{dang}
\subsubsection{Ví dụ minh hoạ}
\begin{vd}%[NB]%[DCHT Toán 11 - KNTT -Tên Huỳnh Thanh Chí]%[1K2Y7-2]
	Tìm số hạng tổng quát của dãy số $ 2;4;8;16;32;\ldots $, biết dãy $ (u_n) $ là một cấp số nhân.  
	\dapso{$ u_n=2\cdot 2^{n-1} $.}
	\loigiai{
		Vì dãy số $ (u_n) $ là một cấp số nhân nên $ q=\dfrac{u_2}{u_1}=\dfrac{u_3}{u_2}=\ldots=2 $ và số hạng đầu $ u_1=2 $.\\
		Do đó dãy số $ 2;4;8;16;32;\ldots $ là một cấp số nhân có số hạng tổng quát là $ u_n=u_1q^{n-1}=2\cdot 2^{n-1} $.
	}
\end{vd}
\begin{vd}%[TH]%[DCHT Toán 11 - KNTT -Tên Huỳnh Thanh Chí]%[1K2B7-2]
	Tìm số hạng đầu, công bội và số hạng tổng quát của cấp số nhân, biết $ \heva{& u_1+u_5=51\\ & u_2+u_6=102.} $
	\dapso{$ u_1=3 $, $ q=2 $ và $ u_n=3\cdot 2^{n-1} $.}
	\loigiai{
	Vì $ (u_n) $ là một cấp số nhân nên $ u_n=u_1\cdot q^{n-1} $.\\
	Ta có $ \heva{& u_1+u_5=51\\ & u_2+u_6=102}\Leftrightarrow\heva{& u_1+u_1q^4=51\\ & u_1q+u_1q^5=102}\Leftrightarrow\heva{& u_1(1+q^4)=51 \qquad (1)\\ & u_1q(1+q^4)=102 \qquad (2).} $\\
	Chia từng vế của $ (2) $ cho $ (1) $ ta được $ \dfrac{u_1q(1+q^4)}{u_1(1+q^4)}=\dfrac{102}{51} \Leftrightarrow q=2 $.\\
	Suy ra $ u_1=\dfrac{51}{1+q^4}=\dfrac{51}{17}=3 $, $ u_n=u_1\cdot q^{n-1}=3\cdot 2^{n-1} $.\\
	Vậy $ u_1=3 $, $ q=2 $ và $ u_n=3\cdot 2^{n-1} $.
	}
\end{vd}
\begin{vd}%[TH]%[DCHT Toán 11 - KNTT -Tên Huỳnh Thanh Chí]%[1K2B7-2]
	Tìm số hạng đầu, công bội và số hạng tổng quát của cấp số nhân, biết $ \heva{& u_1+u_6=30\\ & u_2+u_7=120.} $
	\dapso{$ u_1=\dfrac{6}{205} $, $ q=4 $ và $ u_n=\dfrac{6}{205}\cdot 4^{n-1} $.}
	\loigiai{
		Vì $ (u_n) $ là một cấp số nhân nên $ u_n=u_1\cdot q^{n-1} $.\\
		Ta có $ \heva{& u_1+u_6=30\\ & u_2+u_7=120}\Leftrightarrow\heva{& u_1+u_1q^5=30\\ & u_1q+u_1q^6=102}\Leftrightarrow\heva{& u_1(1+q^5)=30 \qquad (1)\\ & u_1q(1+q^5)=120 \qquad (2).} $\\
		Chia từng vế của $ (2) $ cho $ (1) $ ta được $ \dfrac{u_1q(1+q^5)}{u_1(1+q^5)}=\dfrac{120}{30} \Leftrightarrow q=4 $.\\
		Suy ra $ u_1=\dfrac{30}{1+q^5}=\dfrac{30}{1+4^5}=\dfrac{6}{205} $, $ u_n=u_1\cdot q^{n-1}=\dfrac{6}{205}\cdot 4^{n-1} $.\\
		Vậy $ u_1=\dfrac{6}{205} $, $ q=4 $ và $ u_n=\dfrac{6}{205}\cdot 4^{n-1} $.
	}
\end{vd}
\begin{vd}%[TH]%[DCHT Toán 11 - KNTT -Tên Huỳnh Thanh Chí]%[1K2B7-2]
	Tìm số hạng đầu, công bội và số hạng tổng quát của cấp số nhân, biết $ \heva{& u_3=40\\ & u_6=160.} $
	\dapso{$ u_1=\dfrac{40}{9} $, $ q=3 $ và $ u_n=40\cdot 3^{n-3} $.}
	\loigiai{
	Vì $ (u_n) $ là một cấp số nhân nên $ u_n=u_1\cdot q^{n-1} $.\\	
	Ta có $ \heva{& u_3=40\\ & u_6=1080}\Leftrightarrow \heva{& u_1q^2=40 \qquad (1)\\ & u_1q^5=1080\qquad (2).} $\\
	Chia từng vế của $ (2) $ cho $ (1) $ ta được $ \dfrac{u_1q^5}{u_1q^2}=\dfrac{1080}{40} \Leftrightarrow q^3=27 \Leftrightarrow q=3 $.\\
	Suy ra $ u_1=\dfrac{40}{q^2}=\dfrac{40}{3^2}=\dfrac{40}{9} $, $ u_n=u_1\cdot q^{n-1}=\dfrac{40}{9}\cdot 3^{n-1}=40\cdot 3^{n-3} $.\\
	Vậy $ u_1=\dfrac{40}{9} $, $ q=3 $ và $ u_n=40\cdot 3^{n-3} $.
	}
\end{vd}
\begin{vd}[VDT]%[DCHT Toán 11 - KNTT -Tên Huỳnh Thanh Chí]%[1K2K7-2]
	Tìm số hạng đầu, công bội và số hạng tổng quát của cấp số nhân có công bội $ q\in \mathbb{Z},q\ne 0 $, biết $ \heva{& u_2+u_4=10\\ & u_1+u_3+u_5=-21.} $
	\dapso{$ u_1=-1 $, $ q=-2 $ và $ u_n=\dfrac{(-2)^n}{2} $.}
	\loigiai{
		Vì $ (u_n) $ là một cấp số nhân nên $ u_n=u_1\cdot q^{n-1} $ với $ q\in \mathbb{Z},q\ne 0 $.\\
		Ta có $ \heva{& u_2+u_4=10\\ & u_1+u_3+u_5=-21}\Leftrightarrow\heva{& u_1q+u_1q^3=10\\ & u_1q+u_1q^2+u_1q^4=-21}\Leftrightarrow\heva{& u_1(q+q^3)=10 \qquad (1)\\ & u_1(1+q^2+q^4)=-21 \qquad (2).} $\\
		Chia từng vế của $ (2) $ cho $ (1) $ ta được 
		\allowdisplaybreaks
		\begin{eqnarray*}
			 \dfrac{u_1(1+q^2+q^4)}{u_1(q+q^3)}=\dfrac{-21}{10} 
			 &\Leftrightarrow& 10q^4+21q^3+10q^2+21q+10=0 \\
			 &\Leftrightarrow& (q+2)(2q+1)(5q^2-2q+5)=0 \\
			 &\Leftrightarrow& \hoac{& q=-2 \text{ (thỏa mãn)}\\ & q=-\dfrac{1}{2} \text{ (loại)}.}
		\end{eqnarray*}
		Suy ra $ u_1=\dfrac{10}{q+q^3}=-1 $, $ u_n=u_1\cdot q^{n-1}=(-1)\cdot (-2)^{n-1}=-(-2)^{n-1}=-\dfrac{(-2)^n}{-2}=\dfrac{(-2)^n}{2} $.\\
		Vậy $ u_1=-1 $, $ q=-2 $ và $ u_n=\dfrac{(-2)^n}{2} $.
	}
\end{vd}
\subsubsection{Bài tập tự luận}
 
% \begin{bt}%[NB]%[DCHT Toán 11 - KNTT -Tên Huỳnh Thanh Chí]%[1K2Y7-2]
% 	Tìm số hạng thứ $ 100 $ của cấp số nhân $ 8;-4;2;-1;\ldots $
% 	\dapso{$ u_{100}=-\dfrac{1}{2^{96}} $.}
% 	\loigiai{
% 	Cấp số nhân này có số hạng đầu $ u_1=8 $ và công bội $ q=\dfrac{-4}{8}=-\dfrac{1}{2} $.\\
% 	Do đó số hạng tổng quát $ u_n=8\cdot \left(-\dfrac{1}{2}\right)^{n-1} $.\\
% 	Vậy $ u_{100}=8\cdot \left(-\dfrac{1}{2}\right)^{100-1}=8\cdot \left(-\dfrac{1}{2}\right)^{99}=-\dfrac{1}{2^{96}} $.
% 	}
% \end{bt}
\begin{bt}%[NB]%[DCHT Toán 11 - KNTT -Tên Huỳnh Thanh Chí]%[1K2B7-2]
	Tìm số hạng tổng quát của dãy số $ 3;12;48;192;\ldots $, biết dãy $ (u_n) $ là một cấp số nhân.  
	\dapso{$ u_n=3\cdot 4^{n-1} $.}
	\loigiai{
		Vì dãy số $ (u_n) $ là một cấp số nhân nên $ q=\dfrac{u_2}{u_1}=\dfrac{12}{3}=4 $ và số hạng đầu $ u_1=3 $.\\
		Do đó dãy số $ 3;12;48;192;\ldots $ là một cấp số nhân có số hạng tổng quát là $ u_n=u_1q^{n-1}=3\cdot 4^{n-1} $.}
\end{bt}
\begin{bt}%[TH]%[DCHT Toán 11 - KNTT -Tên Huỳnh Thanh Chí]%[1K2B7-2]
	Tìm số hạng tổng quát của cấp số nhân, biết $ \heva{& u_1+u_3=51\\ & u_2+u_4=153.} $
	\dapso{$ u_n=\dfrac{51}{10}\cdot 3^{n-1} $.}
	\loigiai{
		Vì $ (u_n) $ là một cấp số nhân nên $ u_n=u_1\cdot q^{n-1} $.\\
		Ta có $ \heva{& u_1+u_3=51\\ & u_2+u_4=153}\Leftrightarrow\heva{& u_1+u_1q^2=51\\ & u_1q+u_1q^3=153}\Leftrightarrow\heva{& u_1(1+q^2)=51 \qquad (1)\\ & u_1q(1+q^2)=153 \qquad (2).} $\\
		Chia từng vế của $ (2) $ cho $ (1) $ ta được $ \dfrac{u_1q(1+q^2)}{u_1(1+q^2)}=\dfrac{153}{51} \Leftrightarrow q=3 $.\\
		Suy ra $ u_1=\dfrac{51}{1+q^2}=\dfrac{51}{10} $, $ u_n=u_1\cdot q^{n-1}=\dfrac{51}{10}\cdot 3^{n-1} $.\\
		Vậy số hạng tổng quát $ u_n=\dfrac{51}{10}\cdot 3^{n-1} $.
	}
\end{bt}
\begin{bt}%[TH]%[DCHT Toán 11 - KNTT -Tên Huỳnh Thanh Chí]%[1K2B7-2]
	Tìm số hạng đầu, công bội và số hạng tổng quát của cấp số nhân, biết $ \heva{& u_3=15\\ & u_6=120.} $
	\dapso{$ u_1=\dfrac{15}{4} $, $ q=2 $ và $ u_n=15\cdot 3^{n-3} $.}
	\loigiai{
		Vì $ (u_n) $ là một cấp số nhân nên $ u_n=u_1\cdot q^{n-1} $.\\	
		Ta có $ \heva{& u_3=15\\ & u_6=120}\Leftrightarrow \heva{& u_1q^2=15 \qquad (1)\\ & u_1q^5=120\qquad (2).} $\\
		Chia từng vế của $ (2) $ cho $ (1) $ ta được $ \dfrac{u_1q^5}{u_1q^2}=\dfrac{120}{15} \Leftrightarrow q^3=8 \Leftrightarrow q=2 $.\\
		Suy ra $ u_1=\dfrac{15}{q^2}=\dfrac{15}{2^2}=\dfrac{15}{4} $, $ u_n=u_1\cdot q^{n-1}=\dfrac{15}{4}\cdot 2^{n-1}=15\cdot 2^{n-3} $.\\
		Vậy $ u_1=\dfrac{15}{4} $, $ q=2 $ và $ u_n=15\cdot 3^{n-3} $.
	}
\end{bt}
\begin{bt}%[TH]%[DCHT Toán 11 - KNTT -Tên Huỳnh Thanh Chí]%[1K2B7-2]
	Tìm số hạng tổng quát của cấp số nhân, biết $ \heva{& u_4=35\\ & u_8=560.} $
	\dapso{$ u_n=35\cdot 2^{n-4} $ với $ q=2 $ hoặc $ u_n=35\cdot (-2)^{n-4} $ với $ q=-2 $.}
	\loigiai{
		Vì $ (u_n) $ là một cấp số nhân nên $ u_n=u_1\cdot q^{n-1} $.\\	
		Ta có $ \heva{& u_4=35\\ & u_8=560}\Leftrightarrow \heva{& u_1q^3=35 \qquad (1)\\ & u_1q^7=560\qquad (2).} $\\
		Chia từng vế của $ (2) $ cho $ (1) $ ta được $ \dfrac{u_1q^7}{u_1q^3}=\dfrac{560}{35} \Leftrightarrow q^4=16 \Leftrightarrow \hoac{& q=2\\ & q=-2.} $\\
		Với $ q=2 $. Suy ra $ u_1=\dfrac{35}{q^3}=\dfrac{35}{8} $, $ u_n=u_1\cdot q^{n-1}=\dfrac{35}{8}\cdot 2^{n-1}=35\cdot 2^{n-4} $.\\
		Với $ q=-2 $. Suy ra $ u_1=\dfrac{35}{q^3}=-\dfrac{35}{8} $, $ u_n=u_1\cdot q^{n-1}=-\dfrac{35}{8}\cdot (-2)^{n-1}=35\cdot (-2)^{n-4} $.\\
		Vậy $ u_n=35\cdot 2^{n-4} $ với $ q=2 $ hoặc $ u_n=35\cdot (-2)^{n-4} $ với $ q=-2 $.
	}
\end{bt}

\subsubsection{Câu hỏi trắc nghiệm}
\Opensolutionfile{ans}[ans/ans-1K2-3-dang2]
\begin{ex}%[DCHT Toán 11 - KNTT -Tên Huỳnh Thanh Chí]%[1K2Y7-2]
	Cho cấp số nhân $(u_n)$ có số hạng đầu là $u_1\ne 0$ và công bội $q\ne 0$. Số hạng tổng quát của cấp số nhân bằng
	\choice
	{$ u_{n}=u_1+(n-1)q $}
	{\True $ u_{n}=u_1\cdot q^{n-1} $}
	{$ u_{n}=u_1\cdot q^n $}
	{$ u_{n}=u_1\cdot q^{n+1} $}
	\loigiai{
	Số hạng tổng quát của cấp số nhân là $ u_{n}=u_1\cdot q^{n-1} $.
	}
\end{ex}
\begin{ex}%[DCHT Toán 11 - KNTT -Tên Huỳnh Thanh Chí]%[1K2Y7-2]
	Cấp số nhân  $\left(u_{n}\right)$ có $u_{n}=\dfrac{3}{5}\cdot 2^{n}$. Số hạng đầu tiên và công bội $ q $ là
	\choice
	{$u_1=\dfrac{6}{5},q=3$}
	{$u_1=\dfrac{6}{5},q=-2$}
	{\True $u_1=\dfrac{6}{5},q=2$}
	{$u_1=\dfrac{6}{5},q=5$}
	\loigiai{
	Ta có $u_{n}=\dfrac{3}{5}\cdot 2^{n}=\dfrac{6}{5}\cdot 2^{n-1}$, suy ra $ u_1=\dfrac{6}{5} $ và $ q=2 $.
	}
\end{ex}
\begin{ex}%[DCHT Toán 11 - KNTT -Tên Huỳnh Thanh Chí]%[1K2Y7-2]
	Cho cấp số nhân $(u_n)$ có $u_1=-3$ và công bội $q=\dfrac{2}{3}$. Chọn mệnh đề đúng?
	\choice
	{$u_5=-\dfrac{27}{16}$}
	{$u_5=-\dfrac{16}{27}$}
	{\True $u_5=\dfrac{16}{27}$}
	{$u_5=\dfrac{27}{16}$}
	\loigiai{
	Số hạng tổng quát của cấp số nhân là $ u_{n}=u_1\cdot q^{n-1}=3\cdot\left(\dfrac{2}{3}\right)^{n-1} $.\\
	Vậy $ u_5=3\cdot \left(\dfrac{2}{3}\right)^{5-1}=\dfrac{16}{27} $.
	}
\end{ex}
\begin{ex}%[DCHT Toán 11 - KNTT -Tên Huỳnh Thanh Chí]%[1K2Y7-2]
	Dãy số có số hạng tổng quát $u_{n}=\dfrac{1}{\sqrt{3}}^{2n}$ là một cấp số nhân có công bội $ q $ bằng
	\choice
	{$ \dfrac{1}{\sqrt{3}} $}
	{$ \sqrt{3} $}
	{$ \dfrac{1}{9} $}
	{\True $ \dfrac{1}{3} $}
	\loigiai{
		Ta có $u_{n}=\dfrac{1}{\sqrt{3}}^{2n}=\left[\left(\dfrac{1}{\sqrt{3}}\right)^2\right]^n=\left(\dfrac{1}{3}\right)^n=\dfrac{1}{3}\cdot\left(\dfrac{1}{3}\right)^{n-1} $.\\
		Suy ra công bội của cấp số nhân $ q=\dfrac{1}{3} $.
	}
\end{ex}
\begin{ex}%[DCHT Toán 11 - KNTT -Tên Huỳnh Thanh Chí]%[1K2Y7-2]
	Cho cấp số nhân $(u_n)$ có $u_1=1, u_2=-2$. Mệnh đề nào sau đây đúng?
	\choice
	{\True $u_{2024}=-2^{2023}$}
	{$u_{2024}=2^{2023}$}
	{$u_{2024}=-2^{2024}$}
	{$u_{2024}=2^{2024}$}
	\loigiai{
	Số hạng tổng quát của cấp số nhân là $ u_{n}=u_1\cdot q^{n-1}=\left(-2\right)^{n-1} $.\\
	Vậy $ u_{2024}=\left(-2\right)^{2024-1}=(-2)^{2023}=-2^{2023} $.
	}
\end{ex}
\begin{ex}%[DCHT Toán 11 - KNTT -Tên Huỳnh Thanh Chí]%[1K2B7-2]
	Cho cấp số nhân có $\heva{& u_4-u_2=54\\ & u_5-u_3=108}$. Số hạng đầu tiên $u_1$ và công bội $ q $ của cấp số nhân là
	\choice
	{\True $ u_1=9 $ và $ q=2 $ }
	{$ u_1=9 $ và $ q=-2 $}
	{$ u_1=-9 $ và $ q=2 $}
	{$ u_1=-9 $ và $ q=-2 $}
	\loigiai{
	Ta có $\heva{& u_4-u_2=54\\ & u_5-u_3=108}\Leftrightarrow \heva{& u_1q^3-u_1q=54\\ & u_1q^4-u_1q^2=108}\Leftrightarrow\heva{& u_1q(q^2-1)-54 \quad (1)\\ & u_1q^2(q^2-1)=108 \quad (2).} $\\
	Chia từng vế của $ (2) $ cho $ (1) $ ta được $ \dfrac{u_1q^2(q^2-1)}{u_1q(q^2-1)}=\dfrac{108}{54} \Leftrightarrow q=2 $.\\
	Suy ra $ u_1=\dfrac{54}{q^3-q}=\dfrac{54}{2^3-2}=9 $.
	}
\end{ex}
\begin{ex}%[DCHT Toán 11 - KNTT -Tên Huỳnh Thanh Chí]%[1K2B7-2]
	Cho cấp số nhân $\left( u_n\right)$ biết $\heva{& u_1+ u_2+ u_3=31\\& u_1+ u_3=26 }$. Giá trị $u_1$ và $ q $ là
	\choice
	{$ u_1=2; q=5 $ hoặc $u_1=25; q=\dfrac{1}{5}$}
	{$ u_1=5; q=1 $ hoặc $u_1=25; q=\dfrac{1}{5}$}
	{$ u_1=25; q=5 $ hoặc $u_1=1; q=\dfrac{1}{5}$}
	{\True$ u_1=1; q=5 $ hoặc $u_1=25; q=\dfrac{1}{5}$}
	\loigiai{
	Vì $ (u_n) $ là một cấp số nhân nên $ u_n=u_1\cdot q^{n-1} $.\\	
	Ta có $ \heva{& u_1+ u_2+ u_3=31\\& u_1+ u_3=26 }\Leftrightarrow\heva{& u_2=5\\ & u_1+u_3=26}\Leftrightarrow\heva{& u_1q=5 \quad (1)\\ & u_1(1+q^2)=26\quad (2).} $\\
	Chia từng vế của $ (2) $ cho $ (1) $ ta được $ \dfrac{q^2+1}{q}=\dfrac{26}{5} \Leftrightarrow 5q^2-26q+5=0 \Leftrightarrow \hoac{& q=5\\ & q=\dfrac{1}{5}.} $\\
	Với $ q=5 $. Suy ra $ u_1=\dfrac{5}{q}=\dfrac{5}{5}=1 $.\\
	Với $ q=\dfrac{1}{5} $. Suy ra $ u_1=\dfrac{5}{q}=5:\dfrac{1}{5}=25 $.\\
	Vậy $ u_1=1 $ với $ q=5 $ hoặc $ u_1=25 $ với $ q=\dfrac{1}{5} $.
	}
\end{ex}
\begin{ex}%[DCHT Toán 11 - KNTT -Tên Huỳnh Thanh Chí]%[1K2B7-2]
	Số hạng đầu tiên và công bội của cấp số nhân thỏa mãn $\heva{& u_5+u_2=36\\ & u_6-u_4=48}$ (với $ q>0 $) là
	\choice
	{$ u_1=4,q=4 $}
	{$ u_1=2,q=4 $}
	{\True$ u_1=2,q=2 $}
	{$ u_1=4,q=2 $}
	\loigiai{
	Ta có $\heva{& u_5+u_2=36\\ & u_6-u_4=48}\Leftrightarrow\heva{& u_1q^4+u_1q=36\\ & u_1q^5-u_1q^3=48}\Leftrightarrow\heva{& u_1q(q^3+1)=36\quad (1)\\ & u_1q(q^4-q^2)=48\quad (2).} $\\
	 Chia từng vế của $ (2) $ cho $ (1) $ ta được $$ \dfrac{u_1q(q^4-q^2)}{u_1q(q^3+1)}=\dfrac{48}{36}\Leftrightarrow \dfrac{q^4-q^2}{q^3+1}=\dfrac{4}{3}\Leftrightarrow 3q^4-4q^3-3q^2-4=0\hoac{& q=2\\ &q=-1.} $$
	 Từ điều kiện $ q>0 $ suy ra công bội của cấp số nhân là $ q=2 $, do đó $ u_1=\dfrac{36}{q^4+q}=2 $.\\
	 Vậy $ u_1=2 $ và $ q=2 $.
	}
\end{ex}
\begin{ex}%[DCHT Toán 11 - KNTT -Tên Huỳnh Thanh Chí]%[1K2B7-2]
	Cho cấp số nhân $u_2=\dfrac{1}{4},u_5=16$. Công bội và số hạng đầu tiên của cấp số nhân là
	\choice
	{$q=\dfrac{1}{2};u_1=\dfrac{1}{2}$}
	{$q=\dfrac{-1}{2};u_1=\dfrac{-1}{2}$}
	{\True $q=4;u_1=\dfrac{1}{16}$}
	{$q=-4;u_1=\dfrac{-1}{16}$}
	\loigiai{
	Ta có $ u_2=u_1q=\dfrac{1}{4} $ (1) và $ u_5=u_1q^4=16 $ (2).\\
	Lấy $ (2) $ chia cho $ (1) $ vế theo vế ta được $ \dfrac{u_1q^4}{u_1q}=\dfrac{16}{\tfrac{1}{4}} \Leftrightarrow q^3=64 \Leftrightarrow q=4 $.\\
	Suy ra $ u_1=\dfrac{1}{4}:q=\dfrac{1}{4}:4=\dfrac{1}{16} $.\\
	Vậy $ u_1=\dfrac{1}{16},q=4 $.
	}
\end{ex}
\begin{ex}%[DCHT Toán 11 - KNTT -Tên Huỳnh Thanh Chí]%[1K2K7-2]
	Người ta thiết kế một cái tháp gồm $ 11 $ tầng. Diện tích mặt trên của mỗi tầng bằng nửa diện tích mặt trên của tầng ngay bên dưới và diện tích mặt trên của tầng 1 bằng nửa diện tích của đế tháp (có diện tích là $12\ 288$ m$ ^2 $). Diện tích mặt trên cùng (của tầng thứ $ 11 $) có giá trị nào sau đây?
	\choice
	{\True$ 6 $ m$ ^2 $}
	{$ 8 $ m$ ^2 $}
	{$ 10 $ m$ ^2 $}
	{$ 12 $ m$ ^2 $}
	\loigiai{
	Vì diện tích của mặt trên của mỗi tầng bằng nửa diện tích mặt trên của tầng ngay bên dưới và diện tích mặt trên của tầng 1 bằng nửa diện tích của đế tháp.\\
	Do đó diện tích của mỗi tầng tạo nên dãy số và dãy số đó là một cấp số nhân có công bội $ q=\dfrac{1}{2} $.\\
	Vậy số hạng tổng quát của cấp số nhân đó là $ u_n=12\ 288\cdot \left(\dfrac{1}{2}\right)^{n-1} $.\\
	Vì từ đế tháp đến tầng thứ 11 của tháp sẽ có 12 mặt nền, do đó diện tích của mặt của tầng thứ 11 là $ u_{12}=12\ 288\cdot\left(\dfrac{1}{2}\right)^{12-1}=6 $ m$ ^2 $.
	}
\end{ex}
\Closesolutionfile{ans}
% \begin{indapan}{10}
% 	{ans/ans-1K2-3-dang2}
% \end{indapan}
\begin{dang}{Tìm số hạng cụ thể của CSN}
	Ta chuyển các số hạng của CSN về số hạng đầu $u_1$ và công bội $q$. Sử dụng công thức $u_n=u_1\cdot q^{n-1}$. \\
	Chia hai phương trình vế theo vế ta thu được phương trình theo $q$. \\
	Giải tìm $q$ và $u_1$. Từ đó tìm được số hạng cần tìm thỏa ycbt.
\end{dang}
\subsubsection{Ví dụ minh hoạ}
\begin{vd}%[NB]%[1K2Y3-3]
	Cho $u_n$ là CSN thỏa $u_1=2$; $u_4=16$. Tìm số hạng thứ $5$ của CSN.
	\loigiai{
		Do $u_n$ là CSN nên ta có $u_4=u_1\cdot q^3 \Rightarrow q^3=\dfrac{u_4}{u_1}=8 \Rightarrow q=2$. \\
		Vậy $u_5=u_1\cdot q^4=2\cdot 2^4=32$.
	}
\end{vd}
\begin{vd}%[TH]%[1K2B3-3]
	Cho cấp số nhân $(u_n)$ có $\heva{&u_4+u_6=-540 \\ &u_3+u_5=180}$. Tính số hạng đầu $u_1$ và công bội $q$ của cấp Số nhân.
	\loigiai{
		Ta có $\heva{&u_4 + u_6=-540 \\ &u_3+u_5=180}
		\Leftrightarrow \heva{&u_1q^3(1+q^2)=-540 \\ &u_1q^2(1+q^2)=180}
		\Leftrightarrow \heva{&u_1=2 \\ &q=-3.}$ \\
		Vậy $\heva{&u_1=2 \\ &q=-3}$ là số hạng cần tìm.
	}
\end{vd}
\begin{vd}%[TH]%[1K2B3-3]
	Cho cấp số nhân có $u_1=-3$, $q=\dfrac{2}{3}$. Số $\dfrac{-96}{243}$ là số hạng thứ mấy của cấp số nhân?
	\loigiai{
		Giả sử số $\dfrac{-96}{243}$ là số hạng thứ $n$ của cấp số nhân.\\
		Ta có: $u_1\cdot q^{n-1}=\dfrac{-96}{243}\Leftrightarrow(-3)\left(\dfrac{2}{3}\right)^{n-1}=\dfrac{-96}{243}\Leftrightarrow n=6$.\\
		Vậy số $\dfrac{-96}{243}$ là số hạng thứ $6$ của cấp số nhân.}
\end{vd}
\begin{vd}%[TH]%[1K2B3-3]
	Cấp số nhân $\left(u_{n}\right)$ có số hạng tổng quát là $u_n=\dfrac{3}{5} \cdot 2^{n-1}, n \in \mathbb{N}^*$. Số hạng đầu tiên và công bội của cấp số nhân đó là
	\loigiai{
		Ta có $u_{1}=\dfrac{3}{5} \cdot 2^{1-1}=\dfrac{3}{5}$ và $u_{2}=\dfrac{3}{5} \cdot 2^{2-1}=\dfrac{6}{5} \Rightarrow q=\dfrac{u_{2}}{u_{1}}=2$.\\
		Vậy $u_{1}=\dfrac{3}{5}$ và $q=2$.
	}
\end{vd}

\subsubsection{Bài tập tự luận}
 
\begin{bt}%[TH]%[1K2B3-3]
	Cho cấp số nhân $(u_n)$ biết $\heva{&u_4-u_2=25 \\ &u_3-u_1=50.}$
	\begin{enumEX}{1}
		\item Tìm số hạng đầu và công bội của cấp số nhân $(u_n)$.
		\item Tìm số hạng thứ $8$ của cấp số nhân $(u_n)$.
	\end{enumEX}
	\dapso{$\heva{&q=\dfrac{1}{2} \\ &u_1=-200}$, $u_8=-\dfrac{25}{16}$}
	\loigiai{
		\begin{enumerate}
			\item Ta có $\heva{&u_4-u_2=25 \\ &u_3-u_1=50} 
			\Leftrightarrow \heva{&u_1(q^3-q)=25 \\ &u_1(q^2-q)=50} 
			\Rightarrow \heva{&q=\dfrac{1}{2} \\ &u_1=-200.}$
			\item Ta có $u_8=u_1\cdot q^7=-200\cdot \dfrac{1}{2^7}=-\dfrac{25}{16}$.
		\end{enumerate}
	}
\end{bt}
\begin{bt}%[TH]%[1K2B3-3]
	Tìm số hạng thứ $10$ của cấp số nhân $(u_n)$ biết $\heva{&u_4-u_2=72 \\ &u_5-u_3=144.}$	
	\dapso{$u_{10}=6144$}
	\loigiai{
		Ta có $\heva{&u_4-u_2=72 \\ &u_5-u_3=144}\Leftrightarrow \heva{&u_4-u_2=72 \\ &q(u_4-u_2)=144}
		\Rightarrow \heva{&q=2 \\ &u_1(q^3-q)=72} \Leftrightarrow \heva{&q=2 \\ &u_1=12.}$ \\
		Khi đó $u_{10}=u_1\cdot q^9=6144$.
	}
\end{bt}
\begin{bt}%[TH]%[1K2B3-3]
	Cho một cấp số nhân có $5$ số hạng biết $2$ số hạng đầu là số dương, tích số hạng đầu và số hạng thứ $3$ là $1$, tích số hạng thứ $3$ và số hạng cuối là $\dfrac{1}{16}$. Tìm cấp số nhân này.	
	\dapso{$2; 1; \dfrac{1}{2}; \dfrac{1}{4}; \dfrac{1}{8}$}
	\loigiai{
		Gọi $5$ số hạng cần tìm có dạng $\dfrac{x}{q^2}$; $\dfrac{x}{q}$; $x$; $xq$; $xq^2$.\\
		Theo đề ra ta có $\heva{&\dfrac{x}{q^2}\cdot x=1 \\ &x\cdot xq^2=\dfrac{1}{16}} 
		\Leftrightarrow \heva{&x=\dfrac{1}{2} \\ &q=\dfrac{1}{2}}$ (do hai số hạng đầu dương nên $q>0$). \\
		Vậy $5$ số hạng cần tìm là $2; 1; \dfrac{1}{2}; \dfrac{1}{4}; \dfrac{1}{8}$.
	}
\end{bt}
\begin{bt}%[TH]%[1K2B3-3]
	Tìm số hạng đầu và công bội của cấp số nhân $(u_n)$ biết $\heva{&u_2+u_5-u_4=10 \\ &u_3+u_6-u_5=20.}$
	\dapso{$\heva{&q=2 \\ &u_1=1}$}
	\loigiai{
		Ta có $\heva{&u_2+u_5-u_4=10 \\ &u_3+u_6-u_5=20} 
		\Leftrightarrow \heva{&u_1(q+q^4-q^3)=10 \\ &u_1(q^2+q^5-q^4)=20}
		\Leftrightarrow \heva{&q=2 \\ &u_1=1.}$
	}
\end{bt}
\begin{bt}%[TH]%[1K2B3-4]
	Tìm $5$ số lập thành một cấp số nhân có công bội bằng $\dfrac{1}{4}$ số thứ nhất và tổng $2$ số đầu là $\dfrac{5}{4}$.
	\dapso{$1; \dfrac{1}{4}; \dfrac{1}{16}; \dfrac{1}{64}; \dfrac{1}{128}$ hoặc $-5; -\dfrac{5}{4}; -\dfrac{5}{16}; -\dfrac{5}{64}; -\dfrac{1}{128}$}
	\loigiai{
		Theo đề, ta có $\heva{&q=\dfrac{1}{4}u_1 \\ &u_1+u_2=\dfrac{5}{4}}
		\Leftrightarrow \heva{&q=\dfrac{1}{4}u_1 \\ &u_1+u_1\cdot q=\dfrac{5}{4}}
		\Leftrightarrow \heva{&q=\dfrac{1}{4}u_1 \\ &u_1^2+4u_1-5=0}
		\Leftrightarrow \heva{&q=\dfrac{1}{4} \\ &u_1=1}$ hoặc $\heva{&q=-\dfrac{5}{4} \\ &u_1=-5.}$\\
		Vậy có hai CSN là $1; \dfrac{1}{4}; \dfrac{1}{16}; \dfrac{1}{64}; \dfrac{1}{128}$ và $-5; -\dfrac{5}{4}; -\dfrac{5}{16}; -\dfrac{5}{64}; -\dfrac{1}{128}$.
	}
\end{bt}
\begin{bt}%[TH]%[1K2B3-4]
	Tìm $3$ số lập thành một cấp số nhân có tổng là $63$ và tích là $1728$.
	\dapso{$3; 12; 48$}
	\loigiai{
		Gọi ba số cần tìm là $\dfrac{x}{q}; x; xq$. Theo đề ra, ta có $x^3=1728\Rightarrow x=12$. \\
		Mặt khác $\dfrac{x}{q}+x+xq=63\Leftrightarrow 12q+12+\dfrac{12}{q}=63
		\Leftrightarrow 12q^2-51q+12=0 \Leftrightarrow \hoac{&q=4 \\ &q=\dfrac{1}{4}\cdot}$ \\
		Vậy CSN cần tìm là $3; 12; 48$.
	}
\end{bt}
\subsubsection{Câu hỏi trắc nghiệm}
\Opensolutionfile{ans}[ans/ans-1K2-3-dang3]
\begin{ex}%[1K2B3-3]
	Cho cấp số nhân $(u_n)$ có $u_{20}=8u_{17}$. Công bội của cấp số nhân là
	\choice
	{\True $q=2$}
	{$q=-2$}
	{$q=4$}
	{$q=-4$}
	\loigiai{
		Ta có $u_{20}=8u_{17}\Rightarrow u_1\cdot q^{19}=8\cdot u_1\cdot q^{16}\Rightarrow  q=2$.
	}
\end{ex}
\begin{ex}%[1K2B3-3]
	Cho cấp số nhân $\left(u_n\right)$ có $10$ số hạng với công bội $q\neq 0$ và $u_1\neq 0$. Đẳng thức nào sau đây là đúng?
	\choice
	{$u_7=u_4\cdot q^6$}
	{\True $u_7=u_4\cdot q^3$}
	{$u_7=u_4\cdot q^4$}
	{$u_7=u_4\cdot q^5$}
	\loigiai
	{Ta có $u_7=u_1\cdot q^6=\left(u_1\cdot q^3\right)\cdot q^3=u_4\cdot q^3$.
	}
\end{ex}
\begin{ex}%[1K2B3-3]
	Cho cấp số nhân $(u_n)$ có số hạng đầu $u_1=2$ và công bội $q=3$. Giá trị $u_{2019}$ bằng
	\choice
	{$3\cdot2^{2019}$}
	{$2\cdot3^{2019}$}
	{$3\cdot2^{2018}$}
	{\True $2\cdot3^{2018}$}
	\loigiai{
		Áp dụng công thức của số hạng tổng quát $u_n=u_1\cdot q^{n-1}=2\cdot 3^{2018}$.}
\end{ex}
\begin{ex}%[1K2B3-3]
	Cho cấp số nhân $(u_n)$ với công bội $q < 0$ và $u_2=4$, $u_4=9$. Tìm $u_1$.
	\choice
	{$u_1=6$}
	{\True $u_1=-\dfrac{8}{3}$}
	{$u_1=-6$}
	{$u_1=\dfrac{8}{3}$}
	\loigiai{
		Vì $q<0$, $u_2>0$ nên $u_3<0$. Do đó $u_3=-\sqrt{u_2\cdot u_4}=-\sqrt{4\cdot 9}=-6$.\\
		Ta có $u_2^2=u_1\cdot u_3\Rightarrow u_1=\dfrac{u_2^2}{u_3}=\dfrac{4^2}{-6}=-\dfrac{8}{3}$.
	}
\end{ex}
\begin{ex}%[1K2B3-3]
	Cho cấp số nhân $\left(u_{n}\right)$ có $u_{2}=-6, u_{3}=3$. Công bội $q$ của cấp số nhân đã cho bằng
	\choice
	{$2 $}
	{$\dfrac{1}{2}$}
	{\True $-\dfrac{1}{2}$}
	{$-2$}
	\loigiai{
		Công bội của cấp số nhân đã cho là $$q=\dfrac{u_3}{u_2}=-\dfrac{1}{2}.$$}
\end{ex}
\begin{ex}%[1K2Y3-3]
	Cho cấp số nhân có $u_1=-3$, $q=\dfrac{2}{3}$. Tính $u_5$?
	\choice
	{$u_5=\dfrac{27}{16}$}
	{\True $u_5=\dfrac{-16}{27}$}
	{$u_5=\dfrac{-27}{16}$}
	{$u_5=\dfrac{16}{27}$}
	\loigiai{
		Ta có: $u_5=u_1\cdot q^4=(-3)\left(\dfrac{2}{3}\right)^4=-\dfrac{16}{27}$.}
\end{ex}
\begin{ex}%[1K2Y3-3]
	Cho cấp số nhân $(u_n)$ có $u_2=\dfrac{1}{4}$; $u_5=-16$. Tìm $q$ và số hạng đầu tiên của cấp số nhân?
	\choice
	{$q=\dfrac{1}{2};u_1=\dfrac{1}{2}$}
	{$q=-\dfrac{1}{2},u_1=-\dfrac{1}{2}$}
	{\True $q=-4,u_1=\dfrac{1}{16}$}
	{$q=-4,u_1=-\dfrac{1}{16}$}
	\loigiai{
		Ta có $\heva{&u_2=\dfrac{1}{4}\\&u_5=16}\Rightarrow \heva{&u_1\cdot q=\dfrac{1}{4}\\&u_1\cdot q^4=-16}\Rightarrow q^3=-64\Rightarrow q=-4 \Rightarrow u_1=\dfrac{1}{16}$.
	}
\end{ex}
\begin{ex}%[1K2B3-3]
	Cho cấp số nhân $(u_n)$, biết: $u_n=81,u_{n+1}=9$. Lựa chọn đáp án đúng.
	\choice
	{$q=-\dfrac{1}{9}$}
	{\True $q=\dfrac{1}{9}$}
	{$q=9$}
	{$q=-9$}
	\loigiai{
		Ta có  $q=\dfrac{u_{n+1}}{u_n}=\dfrac{9}{81}=\dfrac{1}{9}$.
	}
\end{ex}
\begin{ex}%[1K2Y3-3]
	Cho cấp số nhân $\left( {u_n} \right)$ với $u_1=2$ và công bội $q=3$. Số hạng $u_2$ bằng
	\choice
	{$8$}
	{\True $6$}
	{$12$}
	{$18$}
	\loigiai{
		Ta có $u_2=u_1\cdot q=2\cdot 3=6$.}
\end{ex}
\begin{ex}%[1K2Y3-3]
	Cho cấp số nhân $(u_n)$ với $u_1=2$ và $u_3=8$. Số hạng thứ hai của cấp số nhân đã cho bằng
	\choice
	{$u_2=4$}
	{$u_2=6$}
	{\True $u_2=\pm 4$}
	{$u_2=-4$}
	\loigiai{
		Ta có $u_1 \cdot u_3=u_2^2 \Leftrightarrow u^2_2=16 \Leftrightarrow \hoac{&u_2=4\\&u_2=-4.}$
	}
\end{ex}
\begin{ex}%[1K2B3-3]
	Cho cấp số nhân $(u_n)$ có $u_1=-1; q=\dfrac{-1}{10}$. Số $\dfrac{1}{10^{103}}$ là số hạng thứ bao nhiêu?
	\choice
	{số hạng thứ $103$}
	{số hạng thứ $105$}
	{\True số hạng thứ $104$}
	{Đáp án khác}
	\loigiai{
		Ta có $u_n=u_1\cdot q^{n-1}\Leftrightarrow	\dfrac{1}{10^{103}}=-1\cdot \left(\dfrac{-1}{10}\right)^{n-1} \Leftrightarrow \left(\dfrac{-1}{10}\right)^{n-1}=\left(\dfrac{-1}{10}\right)^{103}\Rightarrow n=104$.
	}
\end{ex}
\begin{ex}%[1K2Y3-3]
	Cho cấp số nhân $\left(u_n\right)$ có các số hạng lần lượt là $3$, $9$, $27$, $81$,\ldots Khi đó $u_n$ bằng
	\choice
	{$3+3^n$}
	{$3^{n-1}$}
	{$3^{n+1}$}
	{\True $3^n$}
	\loigiai
	{Cấp số nhân đã cho có $u_1=3$ và công bội $q=3$ nên $u_n=u_1\cdot q^{n-1}=3\cdot 3^{n-1}=3^n$.
	}
\end{ex}
\begin{ex}%[1K2K3-3]
	Cho cấp số nhân $(u_n)$ có $u_1=3$ và $15u_1-4u_2+u_3$ đạt giá trị nhỏ nhất. Tìm số hạng thứ $13$ của cấp số nhân đã cho.
	\choice
	{\True $u_{13}=12288$}
	{$u_{13}=3072$}
	{$u_{13}=24567$}
	{$u_{13}=49152$}
	\loigiai{
		Gọi $q$ là công bội của cấp số nhân $(u_n)$.\\
		Ta có $15u_1-4u_2+u_3=45-12q+3q^2=3(q-2)^2+33\geq 33$ $\forall q \in \mathbb{R}$.\\
		Suy ra $15u_1-4u_2+u_3$ đạt giá trị nhỏ nhất khi $q=2$.\\
		Khi đó $u_{13}=u_1q^{12}=12288$.}
\end{ex}
\begin{ex}%[1K2B3-3]
	Cho cấp số nhân $(u_n)$ biết $u_1+u_5=51$ và $u_2+u_6=102$. Hỏi số $12288$ là số hạng thứ mấy của cấp số nhân $(u_n)$?
	\choice
	{\True Số hạng thứ $13$}
	{Số hạng thứ $10$}
	{Số hạng thứ $11$}
	{Số hạng thứ $12$}
	\loigiai{
		Gọi $q$ là công bội của cấp số nhân đã cho. Theo đề bài, ta có\\
		\centerline{$\heva{&u_1+u_5=51\\&u_2+u_6=102}\Leftrightarrow\heva{&u_1\left(1+q^4\right)=51\\&u_1q\left(1+q^4\right)=102}\Rightarrow q=2\Rightarrow u_1=3\Rightarrow u_n=3\cdot 2^{n-1.}$}
		Mặt khác $u_n=12288\Leftrightarrow 3\cdot 2^{n-1}=12288\Leftrightarrow 2^{n-1}=2^{12}\Leftrightarrow n=13$.
	}
\end{ex}
% \begin{ex}%[1K2K3-3]
% 	Một tứ giác lồi có số đo các góc lập thành một cấp số nhân. Biết rằng số đo của góc nhỏ nhất bằng $\dfrac{1}{9}$ số đo của góc nhỏ thứ ba. Hãy tính số đo của các góc trong tứ giác đó.
% 	\choice
% 	{$5^{\circ}$, $15^{\circ}$, $45^{\circ}$, $225^{\circ}$}
% 	{\True $9^{\circ}$, $27^{\circ}$, $81^{\circ}$, $243^{\circ}$}
% 	{$7^{\circ}$, $21^{\circ}$, $63^{\circ}$, $269^{\circ}$}
% 	{$8^{\circ}$, $32^{\circ}$, $72^{\circ}$, $248^{\circ}$}
% 	\loigiai{
% 		Gọi các góc của tứ giác là $a$, $aq$, $aq^2$, $aq^3,$ trong đó $q>1$.\\
% 		Theo giả thiết, ta có $a=\dfrac{1}{9}aq^2$ nên $q=3$.\\
% 		Suy ra các góc của tứ giác là $a$, $3a$, $9a$, $27a$.\\
% 		Vì tổng các góc trong tứ giác bằng $360^{\circ}$ nên ta có $a+3a+9a+27a=360^{\circ}\Leftrightarrow a=9^{\circ}$.\\
% 		Vậy số đo các góc trong tứ giác lần lượt là $9^{\circ}$, $27^{\circ}$, $81^{\circ}$, $243^{\circ}$. }
% \end{ex}
\Closesolutionfile{ans}
% \begin{indapan}{10}
% 	{ans/ans-1K2-3-dang3}
% \end{indapan}
\begin{dang}{Tìm điều kiện để một dãy số lập thành CSN}
	Dãy số $a, b, c$ lập thành CSN khi $b^2=a\cdot c$. \\
	Dãy số $a, b, c, d$ lập thành CSN khi $\heva{&b^2=a\cdot c \\ &c^2=b\cdot d.}$
\end{dang}
\subsubsection{Ví dụ minh hoạ}
\begin{vd}%[NB]%[1K2B3-4]
	Cho dãy $3,x,12,y$. Tìm $x,y$ để dãy là CSN.
	\loigiai{
		Dãy là CSN khi $\heva{&x^2=3\cdot 12 \\ &12^2=x\cdot y}\Leftrightarrow 
		\heva{&x=6 \\ &y=24}$ hoặc $\heva{&x=-6 \\ &y=-24.}$
	}
\end{vd}
\begin{vd}%[TH]%[1K2B3-4]
	Cho dãy  $x-1, 2x, 4x+3$. Tìm $x$ để dãy là CSN. 
	\loigiai{
		Dãy là CSN khi $(2x)^2=(x-1)(4x+3) \Leftrightarrow x=-3$.
	} 
\end{vd}
\begin{vd}%[VD]%[1K2K3-4]
	Các số $x+6y$, $5x+2y$, $8x+y$ theo thứ tự đó lập thành một cấp số cộng, đồng thời, các số $x+\dfrac{5}{3}$, $y-1$, $2x-3y$ theo thứ tự đó lập thành một cấp số nhân. Hãy tìm $x$ và $y$.
	\loigiai{
		\begin{itemize}
			\item Ba số $x+6y$, $5x+2y$, $8x+y$ lập thành cấp số cộng nên $(x+6y)+(8x+y)=2(5x+2y)\Leftrightarrow x=3y$.
			\item Ba số $x+\dfrac{5}{3}$, $y-1$, $2x-3y$ lập thành cấp số nhân nên $\left(x+\dfrac{5}{3}\right)(2x-3y)=\left(y-1\right)^2$.
		\end{itemize}
		Thay $x=3y$ vào ta được $8y^2+7y-1=0\Leftrightarrow y=-1$ hoặc $y=\dfrac{1}{8}$.\\
		Với $y=-1$ thì $x=-3$; với $y=\dfrac{1}{8}$ thì $x=\dfrac{3}{8}$.}
\end{vd}
\begin{vd}%[VD]%[1K2K3-4]
	Tìm tất cả các giá trị của tham số $m$ để phương trình sau có ba nghiệm phân biệt lập thành một cấp số nhân $x^3-7x^2+2\left(m^2+6m\right)x-8=0$.
	\loigiai{
		+ \textbf{Điều kiện cần:} \\
		Giả sử phương trình đã cho có ba nghiệm phân biệt $x_1$,$x_2$,$x_3$ lập thành một cấp số nhân.\\
		Theo định lý Vi-ét, ta có $x_1x_2x_3=8$.\\
		Theo tính chất của cấp số nhân, ta có $x_1x_3=x_2^2$. Suy ra  $x_2^3=8\Leftrightarrow x_2=2$.\\
		Với nghiệm $x=2$, ta có $m^2+6m-7=0\Leftrightarrow\hoac{&m=1\\&m=-7.}$ \\
		+ \textbf{Điều kiện đủ:}\\
		Với $m=1$ hoặc $m=-7$ thì $m^2+6m=7$.\\
		Khi đó phương trình ban đầu trở thành $x^3-7x^2+14x-8=0$.\\
		Giải phương trình này, ta được các nghiệm là $1$,$2$,$4$. Hiển nhiên ba nghiệm này lập thành một cấp số nhân với công bội $q=2$.\\
		Vậy $m=1$ và $m=-7$ là các giá trị cần tìm.}
\end{vd}
\begin{vd}%[VD]%[1K2K3-4]
	Các số $x+6y$, $5x+2y$, $8x+y$ theo thứ tự đó lập thành một cấp số cộng; đồng thời các số $x-1$, $y+2$, $x-3y$ theo thứ tự đó lập thành một cấp số nhân. Tính $x^2+y^2$.
	\loigiai{
		Theo giả thiết ta có
		\[
		\heva{&(x+6y)+(8x+y)=2(5x+2y)\\&(x-1)(x-3y)=\left(y+2\right)^2}\Leftrightarrow\heva{&x=3y\\&(3y-1)(3y-3y)=\left(y+2\right)^2}\Leftrightarrow \heva{&x=3y\\&0=\left(y+2\right)^2}\Leftrightarrow \heva{&x=-6\\&y=-2.}
		\]
		Vậy $x^2+y^2=40$.}
\end{vd}
\subsubsection{Bài tập tự luận}
 
\begin{bt}%[TH]%[1K2B3-4]
	Xác định $x$ dương để $2x-3$; $x$; $2x+3$ lập thành cấp số nhân.
	\dapso{$x=\sqrt{3}$}
	\loigiai{
		Ba số $2x-3$; $x$; $2x+3$ lập thành cấp số nhân khi $x^2=(2x-3)(2x+3) \Leftrightarrow x=\pm \sqrt{3}$.\\
		Do $x>0$ nên chọn $x=\sqrt{3}$.
	}
\end{bt}
\begin{bt}%[TH]%[1K2B3-4]
	Cho cấp số nhân $x, 12, y, 192$. Tìm $x$ và $y$.
	\dapso{$\heva{&x=-3 \\ &y=-48}$}	
	\loigiai{
		Bốn số $x, 12, y, 192$ lập thành CSN khi $\heva{&xy=12^2 \\ &y^2=12\cdot 192}
		\Leftrightarrow \heva{&x=3 \\ &y=48}$ hoặc $\heva{&x=-3 \\ &y=-48.}$
	}
\end{bt}
\begin{bt}%[TH]%[1K2B3-4]
	Tìm $x$ để dãy số $1$, $x^2$, $6-x^2$ lập thành cấp số nhân.
	\dapso{$x=\pm \sqrt{2}$}
	\loigiai{
		Ta có $1, x^2, 6-x^2$ lập thành cấp số nhân $\Leftrightarrow x^4=6-x^2 \Leftrightarrow x= \pm \sqrt{2}$.
	}
\end{bt}
\begin{bt}%[TH]%[1K2B3-4]
	Viết $6$ số xen giữa hai số $-2$ và $256$ để được một cấp số nhân có $8$ số hạng. Tìm cấp số nhân này.
	\dapso{$-2; 4; -8; 16; -32; 64; -128; 256$}
	\loigiai{
		Theo đề ra, ta có $\heva{&u_1=-2 \\ &u_8=256} \Leftrightarrow \heva{&u_1=-2 \\ &u_1\cdot q^7=256}
		\Leftrightarrow \heva{&u_1=-2 \\ &q=-2.}$ \\
		Cấp số nhân cần tìm là $-2; 4; -8; 16; -32; 64; -128; 256$.
	}
\end{bt}
\begin{bt}%[VD]%%[1K2K3-4]
	Bốn góc của một tứ giác lồi lập thành một cấp số nhân, góc lớn nhất gấp $8$ lần góc nhỏ nhất. Tìm $4$ góc đó.
	\dapso{$24^\circ; 48^\circ; 96^\circ; 192^\circ$}	
	\loigiai{
		Giả sử $4$ góc của tứ giác là $A\leq B\leq C\leq D$. Suy ra $A+B+C+D=360^\circ$.\\
		Theo đề, ta có $D=8A \Leftrightarrow Aq^3=8A \Leftrightarrow q=2$. Khi đó, ta được
		$$A(1+q+q^2+q^3)=360^\circ \Rightarrow A=24^\circ.$$
		Vậy $4$ góc của tứ giác lần lượt là $24^\circ; 48^\circ; 96^\circ; 192^\circ$.
	}
\end{bt}
\begin{bt}%[VD]%%[1K2K3-4]
	Tìm tất cả các giá trị của tham số $m$ để phương trình sau có ba nghiệm phân biệt lập thành một cấp số nhân $x^3-7mx^2+2(m^2+6 m)x-64=0$.
	\dapso{$m-8$}
	\loigiai{
		+ Điều kiện cần: \\
		Giả sử phương trình đã cho có ba nghiệm phân biệt $x_1; x_2; x_3$ lập thành một cấp số nhân. \\
		Theo định lý Vi-ét, ta có $x_1 \cdot x_2 \cdot x_3=64$. \\
		Theo tính chất của cấp số nhân, ta có $x_1\cdot x_3=x_2^2$. Suy ra ta có $x_2^3=64 \Leftrightarrow x_2=4$. \\
		Thay $x=4$ vào phương trình đã cho ta được 
		$$4^3-7m\cdot 4^2+2(m^2+6m) \cdot 4-64=0
		\Leftrightarrow m^2-8m=0
		\Leftrightarrow \hoac{&m=0 \\ &m=8.}$$
		+ Điều kiện đủ: \\
		Với $m=0$ thay vào phương trình đã cho ta được: $x^3-64=0$ hay $x=4$
		(nghiệm kép-loại). \\
		Với $m=8$ thay vào phương trình đã cho nên ta có phương trình $x^3-56x^2+224x-64=0$. \\
		Phương trình này có $3$ nghiệm phân biệt lập thành cấp số nhân. \\
		Vậy $m=8$ là giá trị cần tìm.
	}
\end{bt}
\subsubsection{Câu hỏi trắc nghiệm}
\Opensolutionfile{ans}[ans/ans-1K2-3-dang4]
% \begin{ex}%[1K2K3-4]
% 	Bốn góc của một tứ giác tạo thành cấp số nhân và góc lớn nhất gấp $27$ lần góc nhỏ nhất. Tổng của góc lớn nhất và góc bé nhất bằng
% 	\choice
% 	{$56^{\circ}$}
% 	{$102^{\circ}$}
% 	{$168^{\circ}$}
% 	{\True $252^{\circ}$}
% 	\loigiai{
% 		Giả sử 4 góc $A$, $B$, $C$, $D$ (với $A<B<C<D$) theo thứ tự đó lập thành cấp số nhân thỏa yêu cầu với công bội $q$.\\
% 		Theo giả thiết ta có
% 		\[\heva{&A+B+C+D=360\\&D=27A}\Leftrightarrow\heva{&A\left(1+q+q^2+q^3\right)=360\\&Aq^3=27A}\Leftrightarrow\heva{&q=3\\&A=9.}\]
% 		Suy ra $D=A\cdot q^3=9\cdot 3^3=243$.\\
% 		Vậy tổng số đo góc lớn nhất và góc bé nhất là $A+D=252^\circ$.
% 	}
% \end{ex}
\begin{ex}%[1K2B3-4]
	Xác định $x$ để $3$ số $2x-1$; $x$; $2x+1$ theo thứ tự lập thành một cấp số nhân:
	\choice
	{$x=\pm\sqrt{3}$}
	{$x=\pm\dfrac{1}{3}$}
	{\True $x=\pm\dfrac{1}{\sqrt{3}}$}
	{Không có giá trị nào của $x$}
	\loigiai{
		Ba số $2x-1$; $x$; $2x+1$ theo thứ tự lập thành cấp số nhân\\
		$\Leftrightarrow (2x-1)(2x+1)=x^2 \Leftrightarrow 3x^2=1 \Leftrightarrow x=\pm\dfrac{1}{\sqrt{3}}$.
	}
\end{ex}
\begin{ex}%[1K2K3-4]
	Cho $4$ số nguyên dương, trong đó $3$ số đầu lập thành cấp số cộng, $3$ số cuối lập thành cấp số nhân. Biết tổng số đầu và cuối là $37$, tổng $2$ số hạng giữa là $36$. Hỏi số lớn nhất thuộc khoảng nào sau đây?
	\choice
	{$\left(26;29\right)$}
	{\True $\left(24;26\right)$}
	{$\left(30;33\right)$}
	{$\left(22;25\right)$}
	\loigiai{
		Giả sử $4$ số đó là $a$, $b$, $c$, $d$ $\left(a,b,c,d\in\mathbb{N}^{\ast}\right)$. \\
		Do $a$, $b$, $c$ lập thành cấp số cộng nên ta có $a+c=2b$ $(1)$.\\
		Do $b$, $c$, $d$ lập thành cấp số nhân nên ta có $b \cdot d=c^2$ $(\ast)$. \\
		Theo giả thiết ta có $\heva{ & a+d=37 & & (2) \\ & b+c=36. & & (3)}$ \\
		Từ $(1)$, $(2)$, $(3)$ ta có $\heva{ & a=-d+37 \\ & b=\dfrac{-d+73}{3} \\ & c=\dfrac{d+35}{3}.}$ \\
		Thay vào $(\ast)$ ta có $\dfrac{-d+73}{3}\cdot d=\left(\dfrac{d+35}{3}\right)^2 \Leftrightarrow 4d^2-149d+1225=0 \Leftrightarrow \hoac{ & d=25 \\ & d=\dfrac{49}{4} & & \text{(loại)}.}$\\
		Với $d=25$, ta có $a=12$, $b=16$, $c=20$. \\
		Vậy số lớn nhất là $25\in\left(24;26\right)$.
	}
\end{ex}
\begin{ex}%[1K2K3-4]
	Ba số $x$, $y$, $z$ theo thứ tự lập thành một cấp số nhân với công bội $q$ khác $1$ đồng thời các số $x$, $2y$, $3z$ theo thứ tự lập thành một cấp số cộng với công sai khác $0$. Tìm giá trị của $q$.
	\choice
	{$q=-\dfrac{1}{3}$}
	{$q=\dfrac{1}{9}$}
	{$q=-3$}
	{\True $q=\dfrac{1}{3}$}
	\loigiai{
		Theo giả thiết ta có
		\[\heva{&y=xq \\ & z=xq^2\\&x+3z=2(2y)}\Rightarrow x+3xq^2=4xq \Rightarrow x\left(3q^2-4q+1\right)=0\Leftrightarrow \hoac{&x=0\\&3q^2-4q+1=0.}\]
		Nếu $x=0\Rightarrow y=z=0\Rightarrow$ công sai của cấp số cộng $x$, $2y$, $3z$ bằng 0 (vô lí).\\
		Nếu $3q^2-4q+1=0\Leftrightarrow\hoac{&q=1\\&q=\dfrac{1}{3}}\Leftrightarrow q=\dfrac{1}{3}$ vì $(q\not=1)$.}
\end{ex}
\begin{ex}%[1D3Y4-4]
	Trong các dãy số $(u_n)$ cho bởi số hạng tổng quát $u_n$ sau, dãy số nào là một cấp số nhân?
	\choice
	{$u_n=\dfrac{1}{3^n}-1$}
	{$u_n=n+\dfrac{1}{3}$}
	{$u_n=n^2-\dfrac{1}{3}$}
	{\True $u_n=\dfrac{1}{3^{n-2}}$}
	\loigiai{
		Từ các đáp án trên, với dãy $(u_n)$ cho bởi $u_n=\dfrac{1}{3^{n-2}}$ là một cấp số nhân, vì
		$$T=\dfrac{u_{n+1}}{u_n}=\dfrac{3^{n-2}}{3^{n-1}}=\dfrac{1}{3} \text{ (không đổi).}$$
	}
\end{ex}
\begin{ex}%[1K2B3-4]
	Trong các mệnh đề dưới đây, mệnh đề nào là \textbf{sai}?
	\choice
	{Dãy số $\left(a_n\right)$, với $a_1=3$ và $a_{n+1}=\sqrt{a_n+6}$, $\forall n\geq 1,$ vừa là cấp số cộng vừa là cấp số nhân}
	{\True Dãy số $\left(d_n\right)$, với $d_1=-3$ và $d_{n+1}=2d_n^2-15$, $\forall n\geq 1,$ vừa là cấp số cộng vừa là cấp số nhân}
	{Dãy số $\left(b_n\right)$, với $b_1=1$ và $b_{n+1}\left(2b_n^2+1\right)=3$, $\forall n\geq 1,$ vừa là cấp số cộng vừa là cấp số nhân}
	{Dãy số $\left(c_n\right)$, với $c_1=2$ và $c_{n+1}=3c_n^2-10$, $\forall n\geq 1,$ vừa là cấp số cộng vừa là cấp số nhân}
	\loigiai{
		Kiểm tra từng phương án ta có
		\begin{itemize}
			\item Ta có $a_2=3$, $a_2=3$, \ldots Bằng phương pháp quy nạp toán học chúng ra chứng minh được rằng $a_n=3$, $\forall n\geq 1$. Do đó $\left(a_n\right)$ là dãy số không đổi. Suy ra nó vừa là cấp số cộng (công sai bằng $0$) vừa là cấp số nhân (công bội bằng $1$).
			\item Tương tự như phương án trên, chúng ta chỉ ra được $b_n=1$, $\forall n\geq 1$. Do đó $\left(b_n\right)$ là dãy số không đổi. Suy ra nó vừa là cấp số cộng (công sai bằng $0$) vừa là cấp số nhân (công bội bằng $1$).
			\item Tương tự như phương án trên, chúng ta chỉ ra được $c_n=2$, $\forall n\geq 1$. Do đó $\left(c_n\right)$ là dãy số không đổi. Suy ra nó vừa là cấp số cộng (công sai bằng $0$) vừa là cấp số nhân (công bội bằng $1$).
			\item Ta có $d_1=-3$, $d_2=3$, $d_3=3$. Ba số hạng này không lập thành cấp số cộng cũng không lập thành cấp số nhân nên dãy số $\left(d_n\right)$ không phải là cấp số cộng và cũng không là cấp số nhân.
	\end{itemize}}
\end{ex}
\begin{ex}%[1K2K3-4]
	Biết rằng tồn tại hai giá trị $m_1$ và $m_2$ để phương trình \[2x^3+2\left(m^2+2m-1\right)x^2-7\left(m^2+2m-2\right)x-54=0\] có ba nghiệm phân biệt lập thành một cấp số nhân. Tính giá trị của biểu thức $P=m_1^3+m_2^3$.
	\choice
	{$P=56$}
	{$P=8$}
	{$P=-8$}
	{\True $P=-56$}
	\loigiai{
		Theo định lý Vi-ét, ta có $x_1\cdot x_2\cdot x_3 = -\dfrac{d}{a}=-\dfrac{-54}{2}=27 \Leftrightarrow x_2^3=27 \Leftrightarrow x_2=3$.\\
		Điều kiện cần để phương trình đã cho có ba nghiệm phân biệt lập thành một cấp số nhân là $x=3$ phải là nghiệm của phương trình đã cho. Suy ra
		\[m^2+2m-8=0\Leftrightarrow \hoac{&m=2 \\&m=-4.}\]
		Vì giả thiết cho biết tồn tại đúng hai giá trị của tham số $m$ nên $m=2$ và $m=-4$ là các giá trị thỏa mãn.\\
		Vậy $P=2^3+\left(-4\right)^3=-56$.
	}
\end{ex}
\begin{ex}%[1K2K3-4]
	Cho bốn số $a$, $b$, $c$, $d$ biết rằng $a$, $b$, $c$ theo thứ tự đó lập thành một cấp số nhân với công bội $q>1$; còn $b$, $c$, $d$ theo thứ tự đó lập thành cấp số cộng. Tìm $q$, biết rằng $a+d=14$ và $b+c=12$.
	\choice
	{$q=\dfrac{20+\sqrt{73}}{24}$}
	{\True $q=\dfrac{19+\sqrt{73}}{24}$}
	{$q=\dfrac{21+\sqrt{73}}{24}$}
	{$q=\dfrac{18+\sqrt{73}}{24}$}
	\loigiai{
		Giả sử $a$, $b$, $c$ lập thành cấp số cộng công bội $q$. Khi đó theo giả thiết ta có
		\[\heva{&b=aq,\;c=aq^2\\&b+d=2c\\&a+d=14 \\&b+c=12}\Rightarrow\heva{&aq+d=2aq^2&\quad(1)\\&a+d=14&(2)\\& a\left(q+q^2\right)=12.&(3)}\]
		\begin{itemize}
			\item Nếu $q=0\Rightarrow b=c=d=0$. (Vô lí!)
			\item Nếu $q=-1\Rightarrow b=-a$; $c=a\Rightarrow b+c=0$. (Vô lí!)
		\end{itemize}
		Vậy $q\not=0$, $q\not=-1$, từ $(2)$ và $(3)$ ta có $d=14-a$ và $a=\dfrac{12}{q+q^2}$. Thay vào $(1)$ ta được
		\[\begin{aligned}\dfrac{12q}{q+q^2}+\dfrac{14q^2+14q-12}{q+q^2}=\dfrac{24q^3}{q+q^2}
			&\Leftrightarrow 12q^3-7q^2-13q+6=0\\
			&\Leftrightarrow(q+1)\left(12q^2-19q+6\right)=0\\
			&\Leftrightarrow \hoac{ & q=-1 \quad \text{(loại)} \\ & q=\dfrac{19+\sqrt{73}}{24} \\ & q=\dfrac{19-\sqrt{73}}{24}.}
		\end{aligned}\]
		Vì $q>1$ nên $q=\dfrac{19+\sqrt{73}}{24}$.
	}
\end{ex}
\begin{ex}%[1K2K3-4]
	Cho dãy số tăng $a$, $b$, $c$ $\left(c\in\mathbb{Z}\right)$ theo thứ tự lập thành cấp số nhân; đồng thời $a$, $b+8$, $c$ theo thứ tự lập thành cấp số cộng và $a$, $b+8$, $c+64$ theo thứ tự lập thành cấp số nhân. Tính giá trị biểu thức $P=a-b+2c$.
	\choice
	{$P=32$}
	{$P=\dfrac{92}{9}$}
	{\True $P=64$}
	{$P=\dfrac{184}{9}$}
	\loigiai{
		Theo giả thiết, ta có hệ phương trình \[\heva{&ac=b^2\\&a+c=2(b+8)\\&a(c+64)=(b+8)^2}\Leftrightarrow\heva{&ac=b^2&\quad(1)\\&a-2b=16-c&(2)\\&ac+64a=\left(b+8\right)^2.&\quad(3)}\]
		Thay $(1)$ vào $(3)$ ta được $b^2+64a=b^2+16b+64\Leftrightarrow 4a-b=4.\qquad(4)$\\
		Kết hợp $(2)$ với $(4)$ ta được \[\heva{&a-2b=16-c\\&4a-b=4}\Leftrightarrow\heva{&a=\dfrac{c-8}{7}\\&b=\dfrac{4c-60}{7}.} \qquad (5)\]
		Thay $(5)$ vào $(1)$ ta được
		\[7(c-8)c=(4c-60)^2\Leftrightarrow 9c^2-424c+3600=0\Leftrightarrow\hoac{&c=36\\&c=\dfrac{100}{9}}\Leftrightarrow c=36.\qquad\left(\text{Vì}\;c\in\mathbb{Z}\right)\]
		Với $c=36$ $\Rightarrow a=4$, $b=12\Rightarrow P=4-12+72=64$.}
\end{ex}
\begin{ex}%[1K2K3-4]
	Cho $3$ số $a$, $b$, $c$ theo thứ tự lập thành cấp số nhân với công bội khác $1$ . Biết cũng theo thứ tự đó chúng lần lượt là số thứ nhất, thứ tư và thứ tám của một cấp số cộng công sai là $d$, $(d \neq 0)$. Tính $\dfrac{a}{d}$.
	\choice
	{$\dfrac{4}{3}$}
	{\True $9$}
	{$\dfrac{4}{9}$}
	{$3$}
	\loigiai{
		Do $a, b, c$ theo thứ tự lần lượt là số thứ nhất, thứ tư và thứ tám của một cấp số cộng công sai là $d,(d \neq 0)$ nên $\left\{\begin{array}{l}b=a+3 d \\ c=a+7 d\end{array}\right.$.\\
		Hơn nữa $a, b, c$ theo thứ tự lập thành cấp số nhân với công bội khác 1 nên $a c=b^{2}$.\\
		Khi đó \begin{eqnarray*}
			a(a+7 d)=(a+3 d)^{2} &\Leftrightarrow& a^{2}+7 a d=a^{2}+6 a d+9 d^{2}\\
			&\Leftrightarrow& 9 d^{2}-a d=0 \Leftrightarrow 9 d=a \Leftrightarrow \dfrac{a}{d}=9.
		\end{eqnarray*}
		Vậy $\dfrac{a}{d}=9$.
	}
\end{ex}
\begin{ex}%[1K2B3-4]
	Cho dãy số $(u_n)$ là một cấp số nhân với $u_n\neq 0$, $n\in\mathbb{N}^*$. Dãy số nào sau đây không phải là cấp số nhân?
	\choice
	{\True $u_1+2$; $u_2+2$; $u_3+2$; $\ldots$}
	{$3u_1$; $3u_2$; $3u_3$; $\ldots$}
	{$\dfrac{1}{u_1}$; $\dfrac{1}{u_2}$; $\dfrac{1}{u_3}$; $\ldots$}
	{$u_1$; $u_3$; $u_5$; $\ldots$}
	\loigiai{
		Giả sử $(u_n)$ là một cấp số nhân với công bội $q$.\\
		Ta có $u_2=u_1 q$, $u_3=u_1 q^2$.\\
		Dễ thấy $\dfrac{u_2+2}{u_1+2}=\dfrac{u_1 q+2}{u_1+2}$ và $\dfrac{u_3+2}{u_2+2}=\dfrac{u_1 q^2+2}{u_1 q+2}$.\\
		Do $\dfrac{u_2+2}{u_1+2}\neq\dfrac{u_3+2}{u_2+2}$ $\Rightarrow$ dãy số $u_1+2$; $u_2+2$; $u_3+2$; $\ldots$ không phải là cấp số nhân.
	}
\end{ex}
\begin{ex}%[1K2B3-4]
	Xác định $x$ để $3$ số $x-2$; $x+1$; $3-x$ theo thứ tự lập thành một cấp số nhân
	\choice
	{$x=\pm 1$}
	{\True Không có giá trị nào của $x$}
	{$x=-3$}
	{$x=2$}
	\loigiai{
		Ba số $x-2$; $x+1$; $3-x$ theo thứ tự lập thành một cấp số nhân\\
		$\Leftrightarrow(x-2)(3-x)=(x+1)^2 \Leftrightarrow 2x^2-3x+7=0$ (Phương trình vô nghiệm).
	}
\end{ex}
\begin{ex}%[1D3Y4-4]
	Trong các dãy số $(u_n)$ cho bởi số hạng tổng quát $u_n$ sau, dãy số nào là một cấp số nhân?
	\choice
	{\True $u_n=7\cdot 3^n$}
	{$u_n=\dfrac{7}{3n}$}
	{$u_n=7-3^n$}
	{$u_n=7-3n$}
	\loigiai{
		Từ các đáp án trên, với dãy $(u_n)$ cho bởi $u_n=7\cdot 3^n$ là một cấp số nhân, vì
		$$T=\dfrac{u_{n+1}}{u_n}=\dfrac{7\cdot 3^{n+1}}{7\cdot 3^n}=3 \text{ (không đổi).}$$
	}
\end{ex}
\begin{ex}%[1K2K3-4]
	Số hạng thứ hai, số hạng đầu và số hạng thứ ba của một cấp số cộng với công sai khác $0$ theo thứ tự đó lập thành một cấp số nhân với công bội $q$. Tìm $q$.
	\choice
	{\True $q=-2$}
	{$q=-\dfrac{3}{2}$}
	{$q=\dfrac{3}{2}$}
	{$q=2$}
	\loigiai{
		Giả sử ba số hạng $a$; $b$; $c$ lập thành cấp số cộng thỏa mãn yêu cầu, khi đó $b$; $a$; $c$ theo thứ tự đó lập thành cấp số nhân công bội $q\neq 1$. Ta có\\
		\[\heva{&a+c=2b\\&a=bq; c=bq^2}\Rightarrow bq+bq^2=2b\Leftrightarrow\hoac{&b=0\\&q^2+q-2=0.}\]
		\begin{itemize}
			\item Nếu $b=0\Rightarrow a=b=c=0$ nên $a$; $b$; $c$ là cấp số cộng công sai $d=0$. (Vô lí!)
			\item Nếu $q^2+q-2=0$ thì $ q=1$ hoặc $q=-2$. Dễ thấy trường hợp $q = 1$ là không thỏa mãn, vì khi đó $a = b = c$. Do đó $q = -2$.
		\end{itemize}
	}
\end{ex}
\begin{ex}%[1K2K3-4]
	Ba số $x$, $y$, $z$ lập thành một cấp số cộng và có tổng bằng $21$. Nếu lần lượt thêm các số $2$, $3$, $9$ vào ba số đó (theo thứ tự của cấp số cộng) thì được ba số lập thành một cấp số nhân. Tính $F=x^2+y^2+z^2$.
	\choice
	{$F=389$ hoặc $F=395$}
	{$F=395$ hoặc $F=179$}
	{$F=441$ hoặc $F=357$}
	{\True $F=389$ hoặc $F=179$}
	\loigiai{
		Theo tính chất của cấp số cộng, ta có $x+z=2y$.\\
		Kết hợp với giả thiết $x+y+z=21$, ta suy ra $3y=21\Leftrightarrow y=7$.\\
		Gọi $d$ là công sai của cấp số cộng thì $x=y-d=7-d$ và $z=y+d=7+d$.\\
		Sau khi thêm các số $2$, $3$, $9$ vào ba số $x$, $y$, $z$ ta được ba số là $x+2$, $y+3$, $z+9$ hay $9-d$, $10$, $16+d$.\\
		Theo tính chất của cấp số nhân, ta có $(9-d)(16+d)=10^2\Leftrightarrow d^2+7d-44=0$.\\
		Giải phương trình ta được $d=-11$ hoặc $d=4$.\\
		Với $d=-11$, cấp số cộng $18$, $7$, $-4$. Lúc này $F=389$.\\
		Với $d=4$, cấp số cộng $3$, $7$, $11$. Lúc này $F=179$.}
\end{ex}
\Closesolutionfile{ans}
% \begin{indapan}{10}
% 	{ans/ans-1K2-3-dang4}
% \end{indapan}
\begin{dang}{Tính tổng của cấp số nhân}
	Phương pháp
	\begin{itemize}
		\item Xác định số hạng đầu $u_1$, công bội $q$.
		\item Áp dụng công thức tính tổng các số hạng của cấp số nhân. 
	\end{itemize}
\end{dang}
\subsubsection{Ví dụ minh hoạ}
\begin{vd}%[NB]%[DCHT Toán 11 - KNTT -Đỗ Chí Tâm] %[1K2Y7-5]
	Tính tổng 10 số hạng đầu tiên của cấp số nhân $(u_n)$, biết $u_1=-3$ và công bội $q=-2$. \dapso{$S_{10}=1023$}
	\loigiai{
		Ta có: $S_{10}=\dfrac{u_1\left(1-q^{10}\right)}{1-q}=1023$.}
\end{vd}

\begin{vd}%[TH]%[DCHT Toán 11 - KNTT -Đỗ Chí Tâm] %[1K2B7-5]
	Tính tổng $8$ số hạng đầu tiên của cấp số nhân $(u_n),$ biết $u_1=3$ và $u_2=6$. \dapso{$S_8=765$}
	\loigiai{
		Ta có: $u_2=u_1.q \Leftrightarrow 6=3.q \Leftrightarrow q=2$\\
		$\hspace*{1.2cm}  S_8=u_1\dfrac{1-q^8}{1-q}=3.\dfrac{1-2^8}{1-2}=765. $
	}
\end{vd}

\begin{vd}[thuộc chương giới hạn]%[TH]%[DCHT Toán 11 - KNTT -Đỗ Chí Tâm] %[1K2B7-5]
	Tính tổng vô hạn $S=1+\dfrac{1}{2}+\dfrac{1}{2^2}+...+\dfrac{1}{2^n}+...$ \dapso{$S=2$}
	\loigiai{
		Đây là tổng của cấp số nhân lùi vô hạn, với $u_1=1, q=\dfrac{1}{2}$. Khi đó
		$$S=\dfrac{u_1}{1-q}=\dfrac{1}{1-\dfrac{1}{2}}=2.$$
	}
\end{vd}

\begin{vd}%[VD]%[DCHT Toán 11 - KNTT -Đỗ Chí Tâm] %[1K2K7-5]
	Tính tổng $200$ số hạng đầu tiên của dãy số $(u_n)$ biết $\heva{&u_1=1\\&u_{n+1}=3u_n}$.
	\loigiai{
		Dễ thấy dãy đã cho là một cấp số nhân với công bội $q=3; u_1=1$.\\
		Từ đó $S_{200}=u_1\dfrac{q^{200}-1}{q-1}$ $=\dfrac{3^{200}-1}{2}$.
	}
\end{vd}

\begin{vd}%[VD]%[DCHT Toán 11 - KNTT -Đỗ Chí Tâm]%[1K2K7-5]
	Một cấp số nhân có số hạng đầu $u_1=3$, công bội $q=2$. Biết $S_n=765$, tìm $n$.
	% \dapso{$n=8$}
	\loigiai{ 
		Áp dụng công thức tính tổng của cấp số nhân ta có
		$S_n=765 \Leftrightarrow \dfrac{u_1( 1-q^n )}{1-q}=765 \Leftrightarrow \dfrac{3.( 1-2^n )}{1-2}=765 \Leftrightarrow 2^n=256=2^8 \Leftrightarrow n=8$.} 
\end{vd}

\subsubsection{Bài tập tự luận}
 
\begin{bt}%[DCHT Toán 11 - KNTT -Đỗ Chí Tâm] %[1K2Y7-5]
	Một cấp số nhân có số hạng đầu $u_1=3$ và công bội $q=2 $. Tính tổng $8$ số hạng đầu của cấp số nhân.
	\dapso{$765$}		
	\loigiai{
		Ta có $S_8=\dfrac{u_1\left(1-q^8\right)}{1-q}=\dfrac{3 \left(1-2^8\right)}{1-2}=765$.	
	}	
\end{bt}

% \begin{bt}%[DCHT Toán 11 - KNTT -Đỗ Chí Tâm]%[1K2Y7-5]
% 	Một cấp số nhân có số hạng đầu $u_1=1$ và công bội $q=3$. Tính $S_{10}.$
% 	\dapso{$29524$}	
% 	\loigiai{
% 		Ta có $S_{10}=\dfrac{u_1\left(1-q^{10}\right)}{1-q}=1.\dfrac{1-3^{10}}{1-3}=29524$.	
% 	}	
% \end{bt}

% \begin{bt}%[DCHT Toán 11 - KNTT -Đỗ Chí Tâm] %[1K2Y7-5]
% 	Một cấp số nhân $(u_n)$ có $u_1=4$ và công bội $q=2$. Tính $S_{20}$.
% 	\dapso{$4194300$}	
% 	\loigiai{
% 		Ta có $S_{20}=\dfrac{u_1\left(1-q^{20}\right)}{1-q}=4194300$.}
% \end{bt}

\begin{bt}[thuộc chương giới hạn]%[DCHT Toán 11 - KNTT -Đỗ Chí Tâm] %[1K2B7-5]	
	Tính tổng $S=1+\dfrac{1}{3}+\dfrac{1}{3^2}+\cdots+\dfrac{1}{3^n}+\cdots$.
	\dapso{$\dfrac{3}{2}$}			
	\loigiai{
		Đây là tổng của một cấp số nhân lùi vô hạn với $u_1=1, q=\dfrac{1}{3}$\\
		Suy ra   $S=\dfrac{u_1}{1-q}=\dfrac{1}{1-\dfrac{1}{3}}=\dfrac{3}{2}.$
	}
\end{bt}

\begin{bt}%[DCHT Toán 11 - KNTT -Đỗ Chí Tâm] %[1K2B7-5]	
	Cho cấp số nhân có $q=-3, S_6=730$. Tính $u_1$.
	\dapso{$4$}		
	\loigiai{
		$S_6=u_1.\dfrac{1-q^6}{1-q}\Rightarrow u_1=S_6\cdot\dfrac{1-q}{1-q^6}=730\cdot \dfrac{1-(-3)}{1-(-3)^6}=4$.
	}
\end{bt}

% \begin{bt}%[DCHT Toán 11 - KNTT -Đỗ Chí Tâm] %[1K2B7-5]
% 	Một cấp số nhân $(u_n)$ có $u_1=-5$, $u_2=10$.Tính tổng của $15$ số hạng đầu của cấp số nhân đó.
% 	\dapso{$-54615$}	
% 	\loigiai{
% 		Công bội của cấp số nhân đã cho là: $q=\dfrac{u_2}{u_1}=\dfrac{10}{-5}=-2$.\\
% 		Tổng của $15$ số hạng đầu của cấp số nhân đó là $S_{15}=-5\cdot \dfrac{1-(-2)^{15}}{1-(-2)}=-54615$.
% 	}
% \end{bt}

% \begin{bt}%[DCHT Toán 11 - KNTT -Đỗ Chí Tâm] %[1K2B7-5]
% 	Một cấp số nhân $(u_n)$ có $u_1=2$, $u_2=-2$.Tính tổng của $9$ số hạng đầu của cấp số nhân đó.
% 	\dapso{$2$}	
% 	\loigiai{
% 		Công bội của cấp số nhân đã cho là: $q=\dfrac{u_2}{u_1}=\dfrac{-2}{2}=-1$.\\
% 		Tổng của $9$ số hạng đầu của cấp số nhân đó là: $S_9=2\cdot \dfrac{1-(-1)^9}{1-(-1)}=2$.
% 	}
% \end{bt}

\begin{bt}%[DCHT Toán 11 - KNTT -Đỗ Chí Tâm] %[1K2K7-5]
	Một cấp số nhân $(u_n)$ có $u_3=8$, $u_5=32$ và công bội $q>0$. Tính tổng của $10$ số hạng đầu tiên của cấp số nhân.
	\dapso{$2046$}	
	\loigiai{
		$\heva{u_3&=8\\u_5&=32}\Leftrightarrow \heva{u_1.q^2&=8\\u_1.q^4&=32}\Rightarrow q^2=\dfrac{32}{8}=4\Rightarrow q=2, u_1=2$\\
		\hspace*{5.2cm}$\Rightarrow S_{10}=u_1.\dfrac{1-q^{10}}{1-q}=2.\dfrac{2-2^{10}}{1-2}=2046$.
	}
\end{bt}

\begin{bt}%[DCHT Toán 11 - KNTT -Đỗ Chí Tâm] %[1K2K7-5]	
	Tính tổng $S=2+6+18+...+13122$.
	\dapso{$19682$}
	\loigiai{
		Xét cấp số nhân có $u_1=2, q=3$. Khi đó $13122=u_1.q^{n-1}\Leftrightarrow 13122=2.3^{n-1}\Leftrightarrow n=9$\\
		Vậy $S=S_9=u_1\dfrac{1-q^9}{1-q}=2.\dfrac{1-3^9}{1-3}=19682$
	}
\end{bt}

\begin{bt}%[DCHT Toán 11 - KNTT -Đỗ Chí Tâm] %[1K2K7-5]	
	Tính tổng $S=1+2+4+8+\cdots+1024$.	
	\dapso{$2047$}
	\loigiai{
		Xét cấp số nhân có $u_1=1, q=2$. Khi đó $1024=u_1.q^{n-1}\Leftrightarrow 1024=1.2^{n-1}\Leftrightarrow n=11$.\\
		Vậy $S=S_{11}=u_1\dfrac{1-q^{11}}{1-q}=1.\dfrac{1-2^{11}}{1-2}=2047$.
	}
\end{bt}

\begin{bt}%[DCHT Toán 11 - KNTT -Đỗ Chí Tâm] %[1K2K7-5]	
	Một cấp số nhân có $u_1=1, q=3$, biết $S_n=3280$. Tìm $n$.	
	\dapso{$8$}
	\loigiai{
		$S_n=u_1\dfrac{1-q^n}{1-q}=1.\dfrac{1-3^n}{1-3}=3280\Rightarrow n=8$.
	}
\end{bt}

% \begin{bt}%[DCHT Toán 11 - KNTT -Đỗ Chí Tâm] %[1K2K7-5]	
% 	Một cấp số nhân $(u_n)$ có $u_4+u_6=-540, u_3+u_5=180$. Tính $S_5$.
% 	\dapso{$122$}
% 	\loigiai{
% 		$\heva{u_4+u_6&=-540\\u_3+u_5&=180}\Leftrightarrow \heva{u_1.q^3+u_1.q^5&=-540\\u_1.q^2+u_1.q^4&=180} \Leftrightarrow \heva{u_1q^3(1+q^2)&=-540\\u_1q^2(1+q^2)&=180}\Leftrightarrow \heva{u_1&=2\\ q&=-3}$.\\
% 		Vậy $S_5=u_1\dfrac{1-q^5}{1-q}=2.\dfrac{1-(-3)^5}{1+3}=122$.
% 	}
% \end{bt}

\begin{bt}%[DCHT Toán 11 - KNTT -Đỗ Chí Tâm] %[1K2K7-5]
	Bốn số hạng liên tiếp của một cấp số nhân, trong đó số hạng thứ hai nhỏ hơn số hạng thứ nhất $35$, còn số hạng thứ ba lớn hơn số hạng thứ tư $560$. Tìm tổng của bốn số hạng trên, biết công bội mang giá trị dương.
	\dapso{$-\dfrac{2975}{3}$}
	\loigiai{
		Theo đề ta có $\heva{u_1-u_2&=35\\u_3-u_4&=560}\Leftrightarrow \heva{&u_1-u_1q=35\\&u_1q^2-u_1q^3=560}\Leftrightarrow \heva{&u_1(1-q)=35\,\,\,\,\,\,\, (1)\\&u_1q^2(1-q)=560\,\,\,\, (2)}$\\
		Thay $(1)$ vào $(2)$ ta được $q^2=16\Leftrightarrow q=\pm 4$.\\
		Với $q=4$ thay vào $(1)$ ta được $u_1=-\dfrac{35}{3}$.	\\
		$S_4=u_1.\dfrac{1-q^4}{1-q}=-\dfrac{2975}{3}$.
	}	
\end{bt}

\begin{bt}[thuộc chương giới hạn]%[DCHT Toán 11 - KNTT -Đỗ Chí Tâm] %[1K2G7-5]
	Tổng của một cấp số nhân lùi vô hạn bằng $ \dfrac{1}{4} $, tổng ba số hạng đầu tiên của cấp số nhân đó bằng $ \dfrac{7}{27} $. Tổng của số hạng đầu và công bội của cấp số nhân đó bằng
	\dapso{$S=0$}	
	\loigiai{Gọi $ u_1 $ và $ q $ với ($ |q|<1 $) lần lượt là số hạng đầu và cộng bội của cấp số nhân lùi vô hạn. Theo giả thiết, ta có $$\begin{cases}
			\dfrac{u_1}{q-1}=\dfrac{1}{4}\\
			u_1+u_1q+u_1q^2=\dfrac{7}{27}
		\end{cases} \Leftrightarrow \begin{cases}
			\dfrac{u_1}{q-1}=\dfrac{1}{4}\\
			u_1 (1-q^3)=\dfrac{7}{27}(1-q)
		\end{cases} \Leftrightarrow \begin{cases}\dfrac{u_1}{1-q}=\dfrac{1}{4}\\ q^3=-\dfrac{1}{27} \end{cases}\Leftrightarrow
		\heva{& u_1=\dfrac{1}{3}\\& q=-\dfrac{1}{3}.}$$ 
		Vậy $ u_1+q=0 $.
	}
\end{bt}

\begin{bt}%[DCHT Toán 11 - KNTT -Đỗ Chí Tâm] %[1K2G7-5]
	Một du khách vào trường đua ngựa đặt cược, lần đầu đặt $20.000$ đồng, mỗi lần sau tiền đặt gấp đôi số tiền lần đặt trước. Người đó thua $10$ lần liên tiếp và thắng ở lần thứ $11$. Hỏi du khách trên thắng hay thua bao nhiêu tiền?
	\dapso{$20000$ đồng}
	\loigiai{
		Số tiền du khách đặt trong mỗi lần (kể từ lần đầu) là một cấp số nhân có $u_1=20.000$ và công bội $q=2$.\\
		Du khách thua trong 10 lần liên tiếp đầu tiên nên tổng số tiền thua là
		$$S_{10}=\dfrac{u_1(1-q^{10})}{1-q}=\dfrac{20000(1-2^{10})}{1-2}=20000(2^{10}-1) \text{(đồng).}$$ 
		Số tiền du khách thắng trong lần thứ 11 là $u_{11}=u_1q^{10}=20000.2^{10}$ (đồng).\\
		Ta có $u_{11}-S_{10}=20000>0$. Vậy du khách thắng $20000$ đồng.
	}
\end{bt}
\subsubsection{Câu hỏi trắc nghiệm}
\Opensolutionfile{ans}[ans/ans-1K2-3-Dang5]
\begin{ex}%[1K2Y7-5]
	Cho cấp số nhân $u_1,u_2,u_3,\ldots,u_n$ với công bội $q$ ($q\neq 0$, $q\neq 1$). Đặt \[S_n=u_1+u_2+u_3+\cdots +u_n.\] Khẳng định nào sau đây là đúng?
	\choice
	{\True $S_n = \dfrac{u_1\left(q^n-1\right)}{q-1}$}
	{$S_n = \dfrac{u_1\left(q^n+1\right)}{q+1}$}
	{$S_n = \dfrac{u_1\left(q^{n-1}-1\right)}{q+1}$}
	{$S_n = \dfrac{u_1\left(q^{n-1}-1\right)}{q-1}$}
	\loigiai{
		Ta có $S_n=u_1+u_2+u_3+\cdots +u_n = u_1\cdot \dfrac{1-q^n}{1-q} = \dfrac{u_1\left(q^n-1\right)}{q-1}$.
	}
\end{ex}
\begin{ex}%[1K2Y7-5]
	Cho cấp số nhân $(u_n)$ có số hạng đầu $u_1=12$ và công sai $q=\dfrac{3}{2}$. Tổng $5$ số hạng đầu của cấp số nhân bằng
	\choice
	{$\dfrac{93}{4}$}
	{$\dfrac{633}{2}$}
	{\True $\dfrac{633}{4}$}
	{$\dfrac{93}{2}$}
	\loigiai{
		Gọi $S_5$ là tổng $5$ số hạng đầu của cấp số nhân đã cho. Khi đó ta có $$S_5=u_1\cdot \dfrac{1-q^5}{1-q}=12\cdot \dfrac{1-\left(\dfrac{3}{2}\right)^5}{1-\dfrac{3}{2}}=\dfrac{633}{4}.$$
	}
\end{ex}
\begin{ex}%[1K2Y7-5]
	Cho cấp số nhân $(u_n)$ có số hạng đầu $u_1=3$, công bội $q=-2$. Tính tổng $10$ số hạng đầu tiên của cấp số nhân $(u_n)$.
	\choice
	{\True $-1023$}
	{$1023$}
	{$513$}
	{$-513$}
	\loigiai{
		Tổng của $10$ số hạng đầu bằng
		$$S_{10}=u_1\cdot\dfrac{q^{10}-1}{q-1}=3\cdot\dfrac{(-2)^{10}-1}{-2-1}=-1023.$$
	}
\end{ex}
\begin{ex}%[1K2Y7-5]
	Cho cấp số nhân $(u_n)$ có $u_2=-2$ và $u_5=54$. Tính tổng $1000$ số hạng đầu tiên của cấp số nhân đã cho.
	\choice
	{$S_{1000}=\dfrac{3^{1000}-1}{2}$}
	{\True $S_{1000}=\dfrac{1-3^{1000}}{6}$}
	{$S_{1000}=\dfrac{3^{1000}-1}{6}$}
	{$S_{1000}=\dfrac{1-3^{1000}}{4}$}
	\loigiai{
		Ta có $u_5=u_2\cdot q^3\Leftrightarrow q^3=\dfrac{u_5}{u_2}=\dfrac{54}{-2}=-27=(-3)^3\Rightarrow q=-3$ và $u_1=\dfrac{u_2}{q}=\dfrac{2}{3}$.\\
		Suy ra $S_{1000}=u_1\cdot\dfrac{1-q^n}{1-q}=\dfrac{2}{3}\cdot\dfrac{1-(-3)^{1000}}{1+3}=\dfrac{1-3^{1000}}{6}$.
	}
\end{ex}
\begin{ex}%[1K2Y7-5]
	Tính tổng tất cả các số hạng của một cấp số nhân, biết số hạng đầu bằng $18$, số hạng thứ hai bằng $54$ và số hạng cuối bằng $39366$.
	\choice
	{$19674$}
	{\True $59040$}
	{$177138$}
	{$~6552$}
	\loigiai{
		$u_1=18,u_2=54 \Rightarrow q=3$.\\
		$u_n=39366 \Leftrightarrow u_1 \cdot q^{n-1}=39366 \Leftrightarrow 18 \cdot 3^{n-1}=39366 \Leftrightarrow 3^{n-1}=3^7 \Leftrightarrow n=8$.\\
		Vậy $S_8=18 \cdot \dfrac{1-3^8}{1-3}=59040$.}
\end{ex}
\begin{ex}%[1K2B7-5]
	Dãy số $\left(u_n\right)$ xác định bởi $\heva{&	u_1=1\\&u_{n + 1}= \dfrac{1}{2}u_n}$ với $n \ge 1$. Tính tổng $S = u_1 + u_2 +\cdots + u_{10}$.
	\choice
	{$S = \dfrac {1023} {2048}$}
	{$S = \dfrac{5}{2}$}
	{\True $\dfrac {1023} {512}$}
	{$S = 2$}
	\loigiai{
		Ta có các số hạng của dãy số $\left(u_n\right)$ là  $1,\dfrac{1}{2},\dfrac{1}{4},\dfrac{1}{8},\dfrac {1}{16},\dfrac{1}{32},\ldots ,\dfrac{1} {2^n}$. Khi đó $\left(u_n\right)$ lập thành một cấp số nhân có $u_1 = 1$ và công bội $q = \dfrac{1}{2}$. \\
		Suy ra $S = u_1 + u_2 +\cdots + u_{10} 
		=1+\dfrac{1}{2} + \dfrac{1}{4} +\cdots + \dfrac{1}{2^9}
		=\dfrac{1\cdot\left[1-\left(\dfrac{1}{2}\right)^{10}\right]}{1 - \dfrac{1}{2}}
		=\dfrac {1023} {512}$.}
\end{ex}
\begin{ex}%[1K2B7-5]
	Cho cấp số nhân $({{u}_{n}} )$ có ${{u}_{1}}=-6$ và $q=-2$. Tổng $n$ số hạng đầu tiên của cấp số nhân đã cho bằng $2046$. Tìm $n$.
	\choice
	{$n=9$}
	{$n=12$}
	{$n=11$}
	{\True $n=10$}
	\loigiai {
		Ta có
		$2046={{S}_{n}}={{u}_{1}}\cdot \dfrac{1-{{q}^{n}}}{1-q}=-6\cdot \dfrac{1-{{(-2 )}^{n}}}{1-(-2 )}=2({{(-2 )}^{n}}-1 )\Rightarrow {{(-2 )}^{n}}=1024\Leftrightarrow n=10$.}
\end{ex}
\begin{ex}%[1K2B7-5]
	Tổng $100$ số hạng đầu của dãy số $\left(u_n\right)$ với $u_n=2 n-1$ là
	\choice
	{$199$}
	{$2^{100}-1$}
	{\True $10000$}
	{$9999$}
	\loigiai{
		Ta có $(u_n)$ là cấp số cộng công sai $d=2$ và $u_1=1$.\\
		Do đó $S_{n}=n\cdot u_1+\dfrac{n(n-1)}{2}\cdot d=100 \cdot 1 +\dfrac{100 \cdot 99 }{2} \cdot 2 = 10000$.
	}
\end{ex}
\begin{ex}%[1K2B7-5]
	Cho dãy số $(u_n)$ với $u_n = \left(\dfrac{1}{2}\right)^n+1, \forall n \in \mathbb{N}^*$. Tính $S_{2019}=u_1+u_2+u_3+ \cdots + u_{2019}$.
	\choice
	{$S_{2019}=2019+\dfrac{1}{2^{2019}}$}
	{$S_{2019}=\dfrac{4039}{2}$}
	{$S_{2019}=\dfrac{6057}{2}$}
	{\True $S_{2019}=2020-\dfrac{1}{2^{2019}}$}
	\loigiai{
		Ta có
		\allowdisplaybreaks
		\begin{eqnarray*}
			S_{2019} &=& u_1+u_2+u_3+ \cdots + u_{2019}\\
			&=& \left(\dfrac{1}{2}+1\right) + \left[\left(\dfrac{1}{2}\right)^2+1\right] + \left[\left(\dfrac{1}{2}\right)^3+1\right] + \cdots + \left[\left(\dfrac{1}{2}\right)^{2019}+1\right]\\
			&=& 2019 + \dfrac{1}{2} + \left(\dfrac{1}{2}\right)^2 + \left(\dfrac{1}{2}\right)^3 + \cdots + \left(\dfrac{1}{2}\right)^{2019}\\
			&=& 2019 + \dfrac{1}{2} \cdot \dfrac{1-\left(\dfrac{1}{2}\right)^{2019}}{1-\dfrac{1}{2}} = 2019 + 1 - \dfrac{1}{2^{2019}}\\
			&=& 2020 - \dfrac{1}{2^{2019}}.
		\end{eqnarray*}
	}
\end{ex}
\begin{ex}%[1K2K7-5]
	Cho $S=11+101+1001+\cdots +\underbrace{1000\ldots 01}_{(n-1)\text{ chữ số 0}}$. Khẳng định nào sau đây là đúng?
	\choice
	{$S=10\left(\dfrac{10^n-1}{9}\right)$}
	{$S=10\left(\dfrac{10^n-1}{9}\right)-n$}
	{\True $S=10\left(\dfrac{10^n-1}{9}\right)+n$}
	{$S=\left(\dfrac{10^n-1}{9}\right)+n$}
	\loigiai{
		Ta có
		\begin{align*}
			S&=(10+1)+(10^2+1)+(10^3+1)+\cdots +(10^n+1)\\
			&=\left(10+10^2+10^3+\cdots +10^n\right)+\underbrace{1+1+1+\cdots+1}_{n\text{ số } 1}\\
			&=10\left(\dfrac{10^n-1}{9}\right)+n.
		\end{align*}
	}
\end{ex}
\begin{ex}%[1K2K7-5]
	Gọi $S=1+11+111+\cdots+\underbrace{111\ldots1}_{(n\text{ số }1)}$ thì $S$ nhận giá trị nào sau đây?
	\choice
	{\True $S=\dfrac{1}{9}\left[10\left(\dfrac{10^n-1}{9}\right)-n \right]$}
	{$S=\dfrac{10^n-1}{81}$}
	{$S=10\left(\dfrac{10^n-1}{81}\right)-n$}
	{$S=10\left(\dfrac{10^n-1}{81} \right)$}
	\loigiai {
		Ta có
		$S=\dfrac{1}{9}(9+99+999+\cdots+\underbrace{99\ldots9}_{\text{n số }9} )=\dfrac{1}{9}\cdot \left[ 10\cdot \dfrac{1-{{10}^{n}}}{1-10}-n \right]$.}
\end{ex}
\begin{ex}%[1K2K7-5]
	Cho dãy số $(u_n)$ thỏa mãn $\heva{&u_1=1\\&u_n=2u_{n-1}+1, n\geq2}$. Tổng $S=u_1+u_2+ \cdots +u_{20}$ là
	\choice
	{$2^{21}-20$}
	{\True $2^{21}-22$}
	{$2^{20}$}
	{$2^{20}-20$}
	\loigiai{
		Dự đoán công thức số hạng tổng quát $u_n=2^n-1$ (Chứng minh bằng phương pháp quy nạp TH).\\
		$S=2^1+2^2+\cdots+2^{20}-20=2\cdot\dfrac{1-2^{20}}{1-2}-20=2^{21}-22$.
	}
\end{ex}
\begin{ex}%[1K2K7-5]
	Biết rằng $S=1+2\cdot 3+{{3\cdot 3}^2}+\cdots+{{11\cdot 3}^{10}}=a+\dfrac{{{21\cdot 3}^{b}}}{4}$. Tính $P=a+\dfrac{b}{4}$.
	\choice
	{\True $P=3$}
	{$P=4$}
	{$P=1$}
	{$P=2$}
	\loigiai {
		Từ giả thiết suy ra $3S=3+{{2\cdot 3}^2}+{{3\cdot 3}^3}+\cdots+{{11\cdot 3}^{11}}$.\\
		Do đó
		{\allowdisplaybreaks
			\begin{eqnarray*}
				-2S&=&S-3S=1+3+{{3}^2}+\cdots+{{3}^{10}}-{{10\cdot 3}^{11}}\\
				&=&\dfrac{1-{{3}^{11}}}{1-3}-{{11\cdot 3}^{11}}=-\dfrac{1}{2}-\dfrac{{{21\cdot 3}^{11}}}{2}\Rightarrow S=\dfrac{1}{4}+\dfrac{21}{4}\cdot{{3}^{11}}.
			\end{eqnarray*}
		}
		Vì $S=\dfrac{1}{4}+\dfrac{{{21\cdot 3}^{11}}}{4}=a+\dfrac{{{21\cdot 3}^{b}}}{4}\Rightarrow a=\dfrac{1}{4},\,\,b=11\Rightarrow P=\dfrac{1}{4}+\dfrac{11}{4}=3$.}
\end{ex}
\Closesolutionfile{ans}
% \begin{indapan}{10}
% 	{ans/ans-1K2-3-Dang5}
% \end{indapan}

\begin{dang}{Kết hợp cấp số cộng và cấp số nhân}
	Nhắc lại tính chất CSC, CSN
	\begin{itemize}
		\item $3$ số $a,b,c$ theo thứ tự lập thành CSC thì $a+c=2b$.
		\item $3$ số $a,b,c$ theo thứ tự lập thành CSN thì $a.c=b^2$.
	\end{itemize}
\end{dang}
\subsubsection{Ví dụ minh hoạ}
\begin{vd}%[TH]%[DCHT Toán 11 - KNTT -Đỗ Chí Tâm] %[1K2K7-6]
	Ba số $x, y, z$ theo thứ tự đó lập thành một CSN với công bội $q (q\ne 1)$, đồng thời các số $x, 2y, 3z$ theo thứ tự đó lập thành một CSC với công sai $d$ . Hãy tìm $q$?
	\loigiai{
		Ta có $x+3z=2.2y \Leftrightarrow x+3xq^2=2.2xq\Leftrightarrow 1+3q^2=4q \Leftrightarrow \hoac{q&=\dfrac{1}{3}\\q&=1 (L)}$
	}
\end{vd}

\begin{vd}%[TH]%[DCHT Toán 11 - KNTT -Đỗ Chí Tâm] %[1K2K7-6]
	Biết rằng $a, b, c$ là ba số hạng liên tiếp của một CSC và $a, c, b$ là ba số hạng liên tiếp của một CSN, đồng thời $a+b+c=30$. Tìm $a,b,c$.
	\loigiai{
		Theo đề ta có  $\heva{&a+c=2b\,\,\,\hspace*{0.8cm}(1)\\&ab=c^2\,\,\,\hspace*{1.2cm}(2)\\&a+b+c=30\,\,\, (3)}$\\
		Từ $(1)$ và $(3)$ ta được $3b=30\Leftrightarrow b=10$\\
		Thay $b=10$ vào $(1), (2)$ ta được $\heva{a+c=20\\10a=c^2}\Leftrightarrow \hoac{&c=10, a=10\,\, (L)\\ &c=-20, a=40 (N)}$\\
		Vậy $a=40, b=10, c=-20$
	}
\end{vd}

\begin{vd}%[VD]%[DCHT Toán 11 - KNTT -Đỗ Chí Tâm] %[1K2K7-6]
	Ba số $x, y, z$ theo thứ tự đó lập thành một CSN. Ba số $x, y-4 , z$ theo thứ tự đó lập thành CSN. Đồng thời các số $x, y-4 , z-9$ theo thứ tự đó lập thành CSC. Tìm $x,y,z$.
	
	\loigiai{
		Theo đề ta có  $\heva{&xz=y^2\,\,\,\hspace*{2.2cm}(1)\\&xz=(y-4)^2\,\,\,\hspace*{1.4cm}(2)\\&x+(z-9)=2(y-4)\,\,\, (3)}$\\
		Từ $(1)$ và $(2)$ ta có $y^2=(y-4)^2\Leftrightarrow y=2$\\
		Thế $y=2$ vào $(1)$ và $(3)$ ta được $\heva{xz=4\\x+z=5}\Rightarrow x=4, z=1$ hoặc $x=1, z=4$\\
		Vậy có 2 bộ $(x,y,z)$ thỏa yêu cầu bài toán là $(1,2,4)$ và $(4,2,1)$
	}
\end{vd}

\begin{vd}%[VD]%[DCHT Toán 11 - KNTT -Đỗ Chí Tâm] %[1K2K7-6]
	Cho $a,b,c$ là ba số hạng liên tiếp của một CSN và $a,b,c-4$ là ba số hạng liên tiếp của một CSC, đồng thời $a,b-1,c-5$ là ba số hạng liên tiếp của một CSN. Tìm $a,b,c$ biết $a,b,c$ là các số nguyên. 
	
	\loigiai{
		Theo đề ta có  $\heva{&ac=b^2\,\,\,\hspace*{1.9cm}(1)\\&a+c-4=2b\,\,\,\hspace*{0.8cm}(2)\\&a(c-5)=(b-1)^2\,\,\,\,(3)}$\\
		Thay $(1)$ vào $(3)$: $b^2-5a=b^2-2b+1\Leftrightarrow b=\dfrac{5a+1}{2}$\\
		Thay vào $(2)$ ta được $a+c-4=5a+1\Leftrightarrow c=4a+5$\\
		Thế $b, c$ theo $a$ vào $(1)$ ta được $9a^2-10a+1=0\Leftrightarrow a=1 \vee a=\dfrac{1}{9} (L)$
		Vậy $a=1, b=3, c=9$
	}
\end{vd}

\begin{vd}%[VDC]%[DCHT Toán 11 - KNTT -Đỗ Chí Tâm]%[1K2G7-6]
	Cho $4$ số nguyên dương, trong đó $3$ số đầu lập thành một CSC, $3$ số hạng sau thành lập CSN.
	Biết rằng tổng của số hạng đầu và số hạng cuối là $37$, tổng của hai số hạng giữa là $36$. Tìm tổng $4$ số đó
	\loigiai{
		Gọi 4 số cần tìm lần lượt là $a, b, c, d$\\
		$a, b, c$ là $3$ số hạng liên tiếp của CSC. Ta có $a+c=2b\,\,\, (1)$\\
		$b,c,d$ là $3$ số hạng liên tiếp của CSN. Ta có $bd=c^2\,\,\, (2)$\\
		Theo giả thuyết ta có $\heva{a+d&=37\,\,\, (3)\\b+c&=36\,\,\, (4)}$\\
		Từ $(4)\Rightarrow b=36-c$ thay vào $(1)$ ta được $a=72-3c$, thay $a$ vào $(3)$ ta được $d=-35+3c$\\
		Thế $b,d$ vào $(2)$ ta được $(36-c)(-35+3c)=c^2\Rightarrow c=20 \vee c=\dfrac{63}{4} (L)$\\
		Vậy $c=20, a=12, b=16, d=95\Rightarrow S=a+b+c+d=143$
	}
\end{vd}

\subsubsection{Bài tập tự luận}
 
\begin{bt}%[DCHT Toán 11 - KNTT -Đỗ Chí Tâm]%[1K2K7-6]
	Biết $x, y, x+4$ theo thứ tự lập thành cấp số cộng và $x+1, y+1, 2y+2$ theo thứ tự lập thành cấp số nhân với $x, y$ là số thực dương. Tính $x+y$.
	\dapso{$4$}
	\loigiai{
		Theo giả thiết ta có:\\
		$\heva{&x+(x+4)=2y\\&(x+1)(2y+2)=(y+1)^2}\Leftrightarrow \heva{&y=x+2\\&(x+1)(2x+6)=(x+3)^2}\Leftrightarrow \heva{&x=1\Rightarrow y=3\\&x=-3\Rightarrow y=-1}$\\
		Do đó $x+y=4$.
	}	
\end{bt}

\begin{bt}%[DCHT Toán 11 - KNTT -Đỗ Chí Tâm] %[1K2K7-6]
	Cho $3$ số $a, b, c$ theo thứ tự tạo thành một cấp số nhân với công bội khác $1$. Biết cũng theo thứ tự đó chúng lần lượt là số hạng thứ nhất, thứ tư và thứ tám của một cấp số cộng với công sai $d\ne 0$. Tính $\dfrac{a}{d}$.
	\dapso{$9$}
	\loigiai{
		$a, b, c$ lần lượt là số hạng thứ nhất, thứ tư, thứ tám của một CSC với công sai $d$\\
		ta có $\heva{b &=a+3d\\c&=a+7d}$.\\
		Mặt khác $a, b, c$ là $3$ số hạng liên tiếp của CSN nên\\ $a.c=b^2\Leftrightarrow a(a+7d)=(a+3d)^2\Leftrightarrow a^2+7ad=a^2+6ad+9d^2\Leftrightarrow 9d^2=ad\Leftrightarrow \dfrac{a}{d}=9$.
	}	
\end{bt}

\begin{bt}%[DCHT Toán 11 - KNTT -Đỗ Chí Tâm] %[1K2K7-6]
	Tìm tích các số dương $a$ và $b$ sao cho $a, a + 2b, 2a + b$ lập thành một cấp số cộng và $(b + 1)^2, ab + 5,(a + 1)^2$ lập thành một cấp số nhân.
	\dapso{$3$}
	\loigiai{
		Theo tính chất CSC ta có $a+(2a+b)=2(a+2b)\,\,\,\, (1)$\\
		Theo tính chất CSN ta có $(b+1)^2.(a+1)^2=(ab+5)^2\,\,\,\, (2)$\\
		Từ $(1)$ ta được $a=3b$, thay vào $(2)$ ta được $(b+1)^2(3b+1)^2=(3b^2+5)^2$\\
		$\Leftrightarrow \hoac{&(b+1)(3b+1)=(3b^2+5)\\& (b+1)(3b+1)=-(3b^2+5) \,\, (\text{Vô nghiêm})}\Leftrightarrow b=1, a=3\Rightarrow ab=3.$	
	}	
\end{bt}

\begin{bt}%[DCHT Toán 11 - KNTT -Đỗ Chí Tâm] %[1K2K7-6]
	$a,b,c\,(a\ne b\ne c)$ là ba số hạng liên tiếp của một cấp số cộng và $b,c,a$ là ba số hạng liên tiếp của một cấp số nhân, đồng thời $a.b.c=125$. Tìm $a,b,c$.	
	\dapso{$5$}
	\loigiai{
		$a,b, c$ là ba số hạng liên tiếp của cấp số cộng, nên có $a+c=2b$.\\
		$b,c,a$ là ba số hạng liên tiếp của một cấp số nhân, nên có $b.a=c^2$.\\
		Ta có hệ $\heva{&a+c=2b\,\,\,\, (1)\\&b.a=c^2\,\,\,\,\,\,\,\,(2) \\&a.b.c=125\,\,\,\, (3)}$\\
		Thay $(2)$ vào $(3)$ ta được $c^3=125\Rightarrow c=5$\\
		Thay $c=5$ vào $(1), (2)$ ta được hệ $\heva{a+5&=2b\\ab&=25}\Leftrightarrow \heva{&a=2b-5\\&2b^2-5b-25=0}\Leftrightarrow \hoac{&b=5\Rightarrow a=5\\&b=-\dfrac{5}{2}\Rightarrow a=-10}$\\
		Vậy $a=-10, b=-\dfrac{5}{2}, c=5$.	
	}	
\end{bt}

\begin{bt}%[DCHT Toán 11 - KNTT -Đỗ Chí Tâm] %[1K2K7-6]
	Một cấp số cộng và một cấp số nhân đều là các dãy tăng các số hạng thứ nhất của hai dãy số đều bằng $3$, các số hạng thứ hai bằng nhau. Tỉ số giữa các số hạng thứ ba của CSN và CSC là $\dfrac{9}{5}$. Tìm tích ba số hạng của cấp số cộng thỏa mãn tính chất trên.
	\dapso{$405$}
	\loigiai{
		Gọi $u_1, u_2, u_3$ là $3$ số hạng liên tiếp của CSC.\\
		Gọi $a_1, a_2, a_3$ là $3$ số hạng liên tiếp của CSN.\\
		Theo đề ta có hệ $\heva{&u_1=a_1=3\\&u_2=a_2\\&a_3=\dfrac{9}{5}u_3}\Leftrightarrow \heva{&u_1=a_1=3\\&3+d=3q\\&5(3q^2)=9(3+2d)}\Rightarrow q=3 \vee q=\dfrac{3}{5}$\\
		Chọn $q=3$ vì dãy tăng, khi đó $d=6$. \\
		Vậy $3$ số hạng của cấp số cộng là $3; 9; 15\Rightarrow 3\cdot9\cdot15=405$	
	}	
\end{bt}

\begin{bt}%[DCHT Toán 11 - KNTT -Đỗ Chí Tâm] %[1K2K7-6]
	Một CSC và CSN đều có số hạng đầu tiên là bằng 5, số hạng thứ hai của CSC lớn hơn số hạng thứ hai của CSN là 10, còn các số hạng thứ 3 của hai cấp số thì bằng nhau. Tìm tổng các số hạng của cấp số cộng biết công bội của cấp số nhân không âm.
	\dapso{$75$}
	\loigiai{
		Gọi $u_1, u_2, u_3$ là $3$ số hạng liên tiếp của CSC với công sai $d$.\\
		Gọi $a_1, a_2, a_3$ là $3$ số hạng liên tiếp của CSN với công bội $q$.\\
		Theo đề bài ta có: $\heva{&u_1=a_1=5\\&u_2-a_2=10\\&u_3=a_3}\Leftrightarrow \heva{&u_1=a_1=5\\&u_1+d-a_1q=10\\&u_1+2d=a_1q^2}\Leftrightarrow \heva{&u_1=a_1=5\\&d=5+5q\\&5+2d=5q^2}\Rightarrow q=3\vee q=-1 (L)$\\
		Với $q=3\Rightarrow d=20$. Vậy CSC là $5;25;45 \Rightarrow S=5+25+45=75$
	}	
\end{bt}

\begin{bt}%[DCHT Toán 11 - KNTT -Đỗ Chí Tâm] %[1K2G7-6]	
	Ba số khác nhau có tổng bằng $114$ có thể coi là ba số hạng liên tiếp của một CSN, hoặc coi là số hạng thứ nhất, thứ tư và thứ hai mươi lăm của một CSC. Tìm ba số đó.
	\dapso{$2; 14; 98$}
	\loigiai{
		Gọi $u_1, u_2, u_3$ là $3$ số hạng liên tiếp của CSN với công bội $q$.\\
		Theo đề $u_1=a_1, u_2=a_4, u_3=a_{25}$ với $a_1,a_4, a_{25}$ là 3 số hạng của CSC với công sai $d$.\\
		Ta có $\heva{&a_4=a_1+3d\\&a_{25}=a_1+24d}\Rightarrow 8a_4-a_{25}=7a_1\Leftrightarrow 8u_2-u_3=7u_1\Leftrightarrow 8u_1q-u_1q^2=7u_1$\\
		\hspace*{4.2cm}$\Leftrightarrow q^2-8q+7=0 \Leftrightarrow q=1 (L) \vee q=7 (N)$\\
		Theo đề ta cũng có $u_1+u_2+u_3=114\Leftrightarrow u_1+u_1q+u_1q^2=114\Rightarrow u_1=2$\\
		Vậy $3$ số cần tìm là $2; 14; 98$.
	}	
\end{bt}

\begin{bt}%[DCHT Toán 11 - KNTT -Đỗ Chí Tâm] %[1K2G7-6]
	Ba số khác nhau có tổng là $217$ có thể coi là các số hạng liên tiếp của một CSN hoặc là các số hạng thứ $2$ thứ $9$ và thứ $44$ của một CSC. Tìm 3 số đó. 
	\dapso{$7; 35; 175$}
	\loigiai{
		Gọi $u_1, u_2, u_3$ là $3$ số hạng liên tiếp của CSN với công bội $q$.\\
		Theo đề $u_1=a_2, u_2=a_9, u_3=a_{44}$ với $a_2,a_9, a_{44}$ là 3 số hạng của CSC với công sai $d$.\\
		Ta có $\heva{&a_9=a_2+7d\\&a_{44}=a_2+42d}\Rightarrow 6a_9-a_{44}=5a_2\Leftrightarrow 6u_2-u_3=5u_1\Leftrightarrow 6u_1q-u_1q^2=5u_1$\\
		\hspace*{4.2cm}$\Leftrightarrow q^2-6q+5=0 \Leftrightarrow q=1 (L) \vee q=5 (N)$\\
		Theo đề ta cũng có $u_1+u_2+u_3=217\Leftrightarrow u_1+u_1q+u_1q^2=217\Rightarrow u_1=7$\\
		Vậy $3$ số cần tìm là $7; 35; 175$.
	}	
\end{bt}

% \begin{bt}%[DCHT Toán 11 - KNTT -Đỗ Chí Tâm] %[1K2K7-6]
% 	$a,b,c$ là ba số hạng liên tiếp của một CSN và $a,b+2,c+9$ là ba số hạng liên tiếp của một CSC, đồng thời $a,b+2,c$ là ba số hạng liên tiếp của một CSN khác. Tìm $a$.
% 	\dapso{$m=\dfrac{-7\pm3\sqrt{5}}{2}$}
% 	\loigiai{
% 		Vì $a,b,c$ là ba số hạng liên tiếp của CSN, ta có $ac=b^2\,\,\,\, (1)$\\
% 		Vì $a,b+2,c+9$ là ba số hạng liên tiếp của CSC, ta có $a+(c+9)=2(b+2)\,\,\,\, (2)$\\
% 		Vì $a,b+2,c$ là ba số hạng liên tiếp của CSN, ta có $a.c=(b+2)^2\,\,\,\, (3)$\\
% 		Thế $(1)$ vào $(3)$, ta được $b^2=(b+2)^2\Leftrightarrow b=-1$\\
% 		Thay $b=-1$ vào $(1), (2)$, ta được $\heva{&ac=1\\&a+c=-7}\Leftrightarrow a=\dfrac{-7+3\sqrt{5}}{2}\vee a=\dfrac{-7-3\sqrt{5}}{2}$
% 	}	
% \end{bt}

% \begin{bt}%[DCHT Toán 11 - KNTT -Đỗ Chí Tâm] %[1K2G7-6]
% 	Một CSC và CSN có cùng các số hạng thứ $m+1$, thứ $n+1$, thứ $p+1$ và $3$ số hạng này là $3$ số dương $a, b, c$. Tính $T=a^{b-c}.b^{c-a}.c^{a-b}$.
% 	\dapso{$m=1$}
% 	\loigiai{
% 		$a=u_1+md=q^m.v_1$\\
% 		$b=u_1+nd=q^n.v_1$\\
% 		$c=u_1+pd=q^p.v_1$\\
% 		Suy ra $T=a^{b-c}.b^{c-a}.c^{a-b}=\left(q^mv_1 \right)^{(n-p)d}. \left(q^nv_1 \right)^{(p-m)d}. \left(q^pv_1 \right)^{(m-n)d}=1$
% 	}	
% \end{bt}

% \begin{bt}%[DCHT Toán 11 - KNTT -Đỗ Chí Tâm] %[1K2G7-6]
% 	Tìm $m$ dương để phương trình $x^3+(5-m)x^2+(6-5m)x-6m=0 \,\,\,(*)$ có $3$ nghiệm phân biệt lập thành cấp số nhân.
% 	\dapso{$m=\sqrt{6}$}
% 	\loigiai{
% 		$(*)\Leftrightarrow (x+2) \left(x^2+(3-m)x-3m \right)=0\Leftrightarrow x=-2 \vee x=-3 \vee x=m$.\\
% 		Để $(*)$ có 3 nghiệm phân biệt thì $m\ne -3$ và $m\ne -2$.\\
% 		Do $3$ nghiệm này lập thành cấp số nhân, ta sắp xếp các nghiệm này theo thứ tự tăng dần được các dãy số sau
% 		\begin{itemize}
% 			\item $-3;-2; m$ lập thành CSN $\Leftrightarrow$ $-3m=(-2)^2\Leftrightarrow m=-\dfrac{4}{3}$
% 			\item $-3; m; -2$ lập thành CSN $\Leftrightarrow -3(-2)=m^2\Leftrightarrow m=\pm \sqrt{6}$
% 			\item $m; -3; -2$ lập thành CSN $\Leftrightarrow m(-2)=(-3)^2\Leftrightarrow m=-\dfrac{9}{2}$
% 		\end{itemize}	
% 		So với điều kiện thì $m=\sqrt{6}$ thỏa yêu cầu bài toán.
% 	}	
% \end{bt}
% \begin{bt}%[DCHT Toán 11 - KNTT -Đỗ Chí Tâm] %[1K2G7-6]
% 	Tìm tham số $m$ để phương trình $x^3-(2m+1)x^2+2mx=0 \,\, (*)$ có $3$ nghiệm phân biệt lập thành một cấp số cộng, biết $m<0$.
% 	\dapso{$m=-\dfrac{1}{2}$}
% 	\loigiai{
% 		$(*)\Leftrightarrow x.\left(x^2-(2m+1)x+2m \right)=0\Leftrightarrow x=0 \vee x=1 \vee x=2m$.\\
% 		Để $(*)$ có 3 nghiệm phân biệt thì $m\ne 0$ và $m\ne \dfrac{1}{2}$.\\
% 		Do $3$ nghiệm này lập thành cấp số cộng, ta sắp xếp các nghiệm này theo thứ tự tăng dần được các dãy số sau
% 		\begin{itemize}
% 			\item $2m;0; 1$ lập thành CSC $\Leftrightarrow$ $2m+1=2.0\Leftrightarrow m=-\dfrac{1}{2}$
% 			\item $0; 2m; 1$ lập thành CSC $\Leftrightarrow 0+1=4m\Leftrightarrow m=\dfrac{1}{4}$
% 			\item $0; 1; 2m$ lập thành CSC $\Leftrightarrow 0+2m=2.1\Leftrightarrow m=1$
% 		\end{itemize}	
% 		Vậy $m=-\dfrac{1}{2}$ là giá trị cần tìm.
% 	}	
% \end{bt}
% \subsubsection{Câu hỏi trắc nghiệm}
% \Opensolutionfile{ans}[ans/ans-1K2-3-Dang6]
% \begin{ex}%[1K2K7-6]
% 	Các số $x+6y$, $5x+2y$, $8x+y$ theo thứ tự đó lập thành một cấp số cộng, đồng thời các số $x-1$, $y+2$, $x-3y$ theo thứ tự đó lập thành một cấp số nhân. Tính $x^2+y^2$.
% 	\choice
% 	{$x^2+y^2=25$}
% 	{\True $x^2+y^2=40$}
% 	{$x^2+y^2=100$}
% 	{$x^2+y^2=10$}
% 	\loigiai{
% 		Theo bài ra, ta có
% 		\[ \heva{& (x+6 y)+(8 x+y)=2(5 x+2 y) \\ & (y+2)^2=(x-1)(x-3y)} \Rightarrow \heva{& x=3y \\ & (y+2)^2=0}\Rightarrow \heva{& x=-6 \\ & y=-2}\Rightarrow x^2+y^2=40. \]
% 	}
% \end{ex}

% \begin{ex}%[1K2K7-6]
% 	Cho hai số dương $ a $ và $ b $ không vượt quá $ 10 $ sao cho $ a-b $; $ 2 $; $ b $ theo thứ tự tạo thành một cấp số cộng và $ a+b $; $ 3a-2b $; $ 5a $ theo thứ tự lập thành một cấp số nhân. Tính giá trị của $ S=a+b $.
% 	\choice
% 	{$ S=8 $}
% 	{$ S=20 $}
% 	{$ S=7 $}
% 	{\True $ S=5 $}
% 	\loigiai{
% 		Vì $ a-b $; $ 2 $; $ b $ theo thứ tự tạo thành một cấp số cộng nên $ 2-(a-b)=b-2 \Leftrightarrow a=4 $.\\
% 		Vì $ a+b $; $ 3a-2b $; $ 5a $ theo thứ tự lập thành một cấp số nhân nên $$ \dfrac{3a-2b}{a+b}=\dfrac{5a}{3a-2b}\Leftrightarrow \dfrac{12-2b}{4+b}=\dfrac{20}{12-2b}\Leftrightarrow \hoac{&b=1\\&b=16 \text{ (loại).}} $$
% 		Vậy $ S=a+b=4+1=5. $
% 	}
% \end{ex}
% \begin{ex}%[1K2K7-6]
% 	Số hạng thứ hai, số hạng đầu và số hạng thứ ba của một cấp số cộng với công sai khác 0 theo thứ tự đó lập thành một cấp số nhân với công bội $q$. Tìm $q$.
% 	\choice
% 	{$q=2$}
% 	{\True $q=-2$}
% 	{$q=\dfrac{3}{2}$}
% 	{$q=-\dfrac{3}{2}$}
% 	\loigiai {
% 		Giả sử ba số hạng $a;b;c$ lập thành cấp số cộng thỏa yêu cầu, khi đó $b;a;c$ theo thứ tự đó lập thành cấp số nhân công bội $q$. Ta có\\
% 		$\heva{
% 			& a+c=2b \\
% 			& a=bq;\,c=b{{q}^2} \\
% 		}\Rightarrow bq+b{{q}^2}=2b\Leftrightarrow \hoac{
% 			& b=0 \\
% 			& {{q}^2}+q-2=0. \\
% 		}$ \\
% 		Nếu $b=0\Rightarrow a=b=c=0$ nên $a;b;c$ là cấp số cộng công sai $d=0$ (vô lí).\\
% 		Nếu ${{q}^2}+q-2=0\Leftrightarrow q=1$ hoặc $q=-2$ Nếu $q=1\Rightarrow a=b=c$ (vô lí), do đó $q=-2$.}
% \end{ex}

% \begin{ex}%[1K2K7-6]
% 	Cho ba số $a$, $b$, $c$ theo thứ tự tạo thành cấp số nhân với công bội khác $1$. Biết cũng theo thứ tự đó chúng lần lượt là số hạng thứ nhất, thứ tư và thứ tám của một cấp số cộng công sai là $s\neq 0$. Tính $\dfrac{a}{s}$.
% 	\choice
% 	{$3$}
% 	{$\dfrac{4}{9}$}
% 	{\True $9$}
% 	{$\dfrac{4}{3}$}
% 	\loigiai{
% 		Rõ ràng $a\neq 0$. Vì $s$ là công sai cấp số cộng nên $a$, $a+3s$, $a+7s$ lập thành cấp số nhân, do đó
% 		$$a(a+7s)=(a+3s)^2 \Leftrightarrow 9s^2-as=0\Leftrightarrow \hoac{& s=0\quad\text{(loại)}\\ & a=9s}\Leftrightarrow \dfrac{a}{s}=9.$$
% 	}
% \end{ex}

% \begin{ex}%[1K2K7-6]
% 	Xét các số thực dương $a, b$ sao cho $-25$, $2a$, $3b$ là cấp số cộng và $2$, $a+2$, $b-3$ là cấp số nhân. Khi đó $a^2 + b^2-3ab$ bằng
% 	\choice
% 	{$76$}
% 	{$89$}
% 	{$31$}
% 	{\True $59$}
% 	\loigiai{
% 		$-25,\,2a,\,3b$ là cấp số cộng $ \Leftrightarrow 2.2a=-25+3b \Leftrightarrow b=\dfrac{1}{3}(4a+25)$.
% 		\begin{eqnarray*}
% 			& 2,a+2,b-3 \text{ là cấp số nhân}& \Leftrightarrow (a+2)^2=2(b-3)\\
% 			& & \Leftrightarrow (a+2)^2=2\left[\dfrac{1}{3}(4a+25)-3\right]\\
% 			& & \Leftrightarrow 3a^2+4a-20=0 \\
% 			& & \Leftrightarrow \hoac{&a=2 \\&a=-\dfrac{10}{3}\,(\text{loại}).}
% 		\end{eqnarray*}
% 		Suy ra $b=11 \Rightarrow a^2+b^2-3ab=59$.
% 	}
% \end{ex}
% \begin{ex}%[1K2K7-6]
% 	Cho ba số $x$; $5$; $2y$ theo thứ tự lập thành cấp số cộng và ba số $x$; $4$; $2y$ theo thứ tự lập thành cấp số nhân thì $|x-2y|$ bằng
% 	\choice
% 	{$10$}
% 	{$8$}
% 	{$9$}
% 	{\True $6$}
% 	\loigiai{
% 		Theo giả thiết ta có
% 		\begin{itemize}
% 			\item Do $x;5;2y$ theo thứ tự lập thành một cấp số cộng nên ta có $5=\dfrac{x+2y}{2}\quad\quad (1)$.
% 			\item Do $x;4;2y$ theo thứ tự lập thành một cấp số nhân nên ta có $x\cdot 2y=4^2\quad\quad (2)$.
% 		\end{itemize}
% 		Từ $(1)$ và $(2)$ ta có $\heva{&x+2y=10\\&x\cdot 2y =16}\Rightarrow \hoac{& x=2,2y=8\\&x=8,2y=2}\Rightarrow \left|x-2y\right|=6$.
% 	}
% \end{ex}

% \begin{ex}%[1K2K7-6]
% 	Cho dãy số tăng $a, b, c\,\,(c\in \mathbb{Z} )$ theo thứ tự lập thành cấp số nhân; đồng thời $a,b+8,c$ theo thứ tự lập thành cấp số cộng và $a, b+8, c+64$ theo thứ tự lập thành cấp số nhân. Tính giá trị biểu thức $P=a-b+2c$.
% 	\choice
% 	{$P=\dfrac{184}{9}$}
% 	{\True $P=64$}
% 	{$P=\dfrac{92}{9}$}
% 	{$P=32$}
% 	\loigiai{
% 		Ta có $\heva{
% 			& ac=b^2 \\
% 			& a+c=2(b+8 ) \\
% 			& a(c+64 )={{(b+8 )}^2} \\
% 		}\Leftrightarrow \heva{
% 			& ac=b^2\quad \quad(1 ) \\
% 			& a-2b=16-c\quad(2 ) \\
% 			& ac+64a=(b+8 )^2\quad(3 )\\
% 		}$.
% 		Thay $(1)$ vào $(3)$ ta được: $$b^2+64a=b^2+16b+64\Leftrightarrow 4a-b=4. \quad
% 		(4 )$$
% 		Kết hợp $(2)$ với $(4)$ ta được: $\heva{
% 			& a-2b=16-c \\
% 			& 4a-b=4 \\
% 		}\Leftrightarrow \heva{
% 			& a=\dfrac{c-8}{7} \\
% 			& b=\dfrac{4c-60}{7}. \\
% 		}\,\,\,\,\,(5 )$ \\
% 		Thay $(5)$ vào $(1)$ ta được:\\
% 		$7(c-8 )c={{(4c-60 )}^2}\Leftrightarrow 9{{c}^2}-424c+3600=0\Leftrightarrow \hoac{
% 			& c=36 \\
% 			& c=\dfrac{100}{9} \\
% 		}\Leftrightarrow c=36\,\,(c\in \mathbb{Z} )$. \\
% 		Với $c=36\Rightarrow a=4,\,\,b=12\Rightarrow P=4-12+72=64$.}
% \end{ex}

% \begin{ex}%[1K2K7-6]
% 	Cho bốn số $a$, $b$, $c$, $d$ theo thứ tự đó tạo thành cấp số nhân với công bội khác $1$. Biết tổng ba số hạng đầu bằng $\dfrac{148}{9}$, đồng thời theo thứ tự đó $a$, $b$, $c$ lần lượt là số hạng thứ nhất, thứ tư và thứ tám của một cấp số cộng. Tính giá trị của biểu thức $T=a-b+c-d$.
% 	\choice
% 	{$T=-\dfrac{101}{27}$}
% 	{$T=\dfrac{100}{27}$}
% 	{\True $T=-\dfrac{100}{27}$}
% 	{$T=\dfrac{101}{27}$}
% 	\loigiai{
% 		Gọi $s$ ($s\neq 0$) là công sai của cấp số cộng. Vì $a$, $b$, $c$ theo thứ tự lần lượt là số hạng thứ nhất, thứ tư và thứ tám của cấp số cộng nên $b=a+3s$ và $c=a+7s$.\\
% 		Mặt khác, $a$, $b$, $c$ theo thứ tự tạo thành cấp số nhân với công bội khác $1$ nên
% 		\[ac=b^2 \Leftrightarrow a(a+7s)=(a+3s)^2 \Leftrightarrow as=9s^2 \Leftrightarrow a=9s \,\,(\text{vì } s\neq 0).\]
% 		Suy ra $b=12s$, $c=16s$.\\
% 		Theo giả thiết
% 		\[a+b+c=\dfrac{148}{9} \Leftrightarrow 9s+12s+16s=\dfrac{148}{9} \Leftrightarrow 37s=\dfrac{148}{9} \Leftrightarrow s=\dfrac{4}{9}.\]
% 		Suy ra $a=4$, $b=\dfrac{16}{3}$, $c=\dfrac{64}{9}$. Từ đó ta tính được $d=4\cdot \left(\dfrac{4}{3}\right)^3 = \dfrac{256}{27}$.\\
% 		Vậy $T=a-b+c-d = 4-\dfrac{16}{3}+\dfrac{64}{9}-\dfrac{256}{27} = -\dfrac{100}{27}$.
% 	}
% \end{ex}
% \begin{ex}%[1K2K7-6]
% 	Cho $x$ và $y$ là các số nguyên thỏa mãn các số $x+6y$ ,$5x+2y$, $8x+y$ theo thứ tự lập thành cấp cộng và các số $x-\dfrac{5}{3}y$, $y-1$, $2x-3y$ theo thứ tự lập thành cấp số nhân. Tính tổng $S=2x+3y$.
% 	\choice
% 	{$9$}
% 	{$6$}
% 	{$-6$}
% 	{\True $-9$}
% 	\loigiai{
% 		Vì các số $x+6y$ ,$5x+2y$, $8x+y$ theo thứ tự lập thành cấp cộng nên ta có
% 		$$ (x+6y)+(8x+y)=2(5x+2y)\Leftrightarrow x=3y. $$
% 		Vì các số $x-\dfrac{5}{3}y$, $y-1$, $2x-3y$ theo thứ tự lập thành cấp số nhân nên ta có
% 		$$ \left(x-\dfrac{5}{3}y\right) (2x-3y)=(y-1)^2.  $$
% 		Thay $x=3y$ vào phương trình trên, ta được
% 		\begin{eqnarray*}
% 			& & \left(3y-\dfrac{5}{3}y\right) (6y-3y)=(y-1)^2\\
% 			&\Leftrightarrow & 4y^2=y^2-2y+1\\
% 			&\Leftrightarrow & \hoac{& y=-1\\& y=\dfrac{1}{3}}.
% 		\end{eqnarray*}
% 		Ta loại trường hợp $y=\dfrac{1}{3}$ vì $y$ là số nguyên. Suy ra $x=3y=3(-1)=-3$. Vậy $$S=2x+3y=2(-3)+3(-1)=-9.$$
% 	}
% \end{ex}
\Closesolutionfile{ans}
% \begin{indapan}{10}
% 	{ans/ans-1K2-3-Dang6}
% \end{indapan}
\begin{dang}{Bài toán thực tế}
	\textit{Bài toán lãi kép:} Một người gửi tiết kiệm vào ngân hàng một số tiền $A$ với lãi suất $r\%$ mỗi kì hạn. Số tiền lãi sẽ được nhập vào vốn ban đầu để tính lãi cho kì hạn tiếp theo. Hỏi sau $n$ kì hạn thì người đó có tất cả bao nhiêu tiền?\\
	\textit{Lời giải:} Gọi $u_n$ là số tiền người đó có sau $n$ kì hạn. Ta có:
	\begin{itemize}
		\item Số tiền người đó có sau kì hạn thứ nhất là: $u_1=A+A\cdot r\%=A\left(1+r\%\right)$.
		\item Số tiền người đó có sau $n$ kì hạn là: $u_n=u_{n-1}+u_{n-1}\cdot r\%=u_n\left(1+r\%\right)$.
	\end{itemize}
	Suy ra dãy số $(u_n)$ là một cấp số nhân với số hạng đầu $u_1=A\left(1+r\%\right)$ và công bội $q=1+r\%$.\\
	Vậy số tiền người đó có sau $n$ kì hạn là: \fbox{$u_n=A\left(1+r\%\right)^n$}.
\end{dang}
\subsubsection{Ví dụ minh hoạ}
\begin{vd}%[VD]%[DCHT Toán 11 - KNTT -Tên Huỳnh Thanh Chí]%[1K2K7-2]
	Trong một lọ nuôi cấy vi khuẩn, ban đầu có $ 5\ 000 $ con vi khuẩn và số lượng vi khuẩn tăng lên thêm $ 8\% $ mỗi giờ. Hỏi sau $ 5 $ giờ thì số lượng vi khuẩn là bao nhiêu?
	\dapso{}
	\loigiai{
	Ta có $ A=5\ 000 $ là số lượng vi khuẩn ban đầu, $ r=8\%=0{,}08 $ là tỉ lệ gia tăng vi khuẩn sau một giờ.
	\begin{itemize}
	\item Tại thời điểm sau $ 1 $ giờ: $ u_1=5000+5000\cdot 0{,}08= 5\ 000\cdot(1{,}08)$.
	\item Tại thời điểm sau $ n $ giờ: $ u_n=u_{n-1}+u_{n-1}\cdot0{,}08=u_{n-1}\cdot (1{,}08)$.
	\end{itemize}
	Do đó ta có thể nhận thấy rằng, số lượng vi khuẩn ở thời gian $ n $ giờ là một cấp số nhân có số hạng đầu $ u_1=5000.1,08 $ và công bội $ q=1{,}08 $.\\
	% Suy ra số hạng tổng quát $ u_n=5\ 000\cdot (1{,}08)^{n} $.\\
	Vậy số lượng vi khuẩn sau $ 5 $ giờ là $ u_5=5\ 000\cdot (1{,}08)^{5}\approx 7346 $ (vi khuẩn).
	}
\end{vd}
\begin{vd}%[VD]%[1K2K3-3]
	Người ta thiết kế một cái tháp gồm $10$ tầng theo cách: Diện tích bề mặt trên của mỗi tầng bằng nửa diện tích bề mặt trên của tầng ngay bên dưới và diện tích bề mặt của tầng 1 bằng nửa diện tích bề mặt đế tháp. Biết diện tích bề mặt đế tháp là $12\, 288$ m$^2$, tính diện tích bề mặt trên cùng của tháp.
	\loigiai{
		Gọi $S$ là diện tích mặt đế và $T_1, T_2, \ldots, T_{10}$ là diện tích bề mặt của tầng 1, tầng 2, \ldots, tầng 10.\\
		Khi đó, ta có
		\allowdisplaybreaks
		\begin{eqnarray*}
			&T_1&=\dfrac{1}{2}\cdot S;\\
			&T_n&=\dfrac{1}{2}\cdot T_{n-1}
		\end{eqnarray*}
		Suy ra $\left(T_n\right)$ là cấp số nhân có số hạng đầu $T_1=\dfrac{1}{2}\cdot 12288=6144$ m$^2$ và công bội $q=\dfrac{1}{2}$.\\
		Vậy diện tích bề mặt trên cùng của tháp là $T_{10}=\dfrac{1}{2^{10}}\cdot 12288=12$ m$^2$.
	}
\end{vd}
\begin{vd}%[TH] %[DCHT Toán 11 - KNTT - Dung Phuong] %[1K2B7-7]
	Dân số trung bình của Việt Nam năm $2020$ là $97{,}6$ triệu người, tỉ lệ tăng dân số là $1{,}14 \% /$năm.
	\begin{flushright}
		\textit{(Nguồn: Niên giám thống kê của Việt Nam năm 2020, NXB Thống kê, 2021)}
	\end{flushright}
	Giả sử tỉ lệ tăng dân số không đổi qua các năm.
	\begin{enumerate}
		\item Sau 1 năm, dân số của Việt Nam sẽ là bao nhiêu triệu người (làm tròn kết quả đến hàng phần mười)?
		\item Viết công thức tính dân số Việt Nam sau $n$ năm kể từ năm $2020$.
	\end{enumerate}
	\dapso{$\approx 98{,}7$ triệu người}
	\loigiai{
		\begin{enumerate}
			\item Sau 1 năm, dân số của Việt Nam sẽ là
			\allowdisplaybreaks
			\begin{eqnarray*}
				u_1&=&97{,}6+97{,}6 \cdot 0{,}0114=97{,}6 \cdot(1+0{,}0114)\\
				&=&97{,}6 \cdot 1{,}0114 \approx 98{,}7 (\text{triệu người}).
			\end{eqnarray*} 
			\item Gọi $u_n$ là dân số của Việt Nam sau $n$ năm.\\
			Do tỉ lệ tăng dân số hàng năm là $1{,}14 \%$ nên ta có
			\allowdisplaybreaks
			\begin{eqnarray*}
				u_n &=&u_{n-1}+u_{n-1} \cdot 0{,}0114=u_{n-1} \cdot(1+0{,}0114) \\
				&=&u_{n-1} \cdot 1{,}0114\ \text{với}\ n \geq 2.
			\end{eqnarray*}
			Do đó, $\left(u_n\right)$ là cấp số nhân có số hạng đầu $u_1=97{,}6\cdot 1{,}0114$, công bội $q=1{,}0114$.\\
			Vậy dân số của Việt Nam sau $n$ năm kể từ năm $2020$ là
			\[u_n=97{,}6 \cdot 1{,}0114 \cdot 1{,}0114^{n-1}=97{,}6 \cdot 1{,}0114^n\ (\text{triệu người}). \]
		\end{enumerate}
	}
\end{vd}

\begin{vd}%[TH] %[DCHT Toán 11 - KNTT - Dung Phuong] %[1K2B7-7]
	Bác Linh gửi vào ngân hàng $100$ triệu đồng tiền tiết kiệm với hình thức lãi kép, kì hạn 1 năm với lãi suất $6 \% /$năm. Viết công thức tính số tiền (cả gốc và lãi) mà bác Linh có được sau $n$ năm (giả sử lãi suất không thay đổi qua các năm).
	\dapso{$100 \cdot 1{,}06^{n-1}$ triệu đồng}
	\loigiai{
		Gọi $u_n$ là số tiền (cả gốc lẫn lãi) mà bác Linh có được sau $n$ năm.\\
		Do lãi suất 1 năm là $6\%$ nên ta có
		\allowdisplaybreaks
		\begin{eqnarray*}
			u_n &=&u_{n-1}+u_{n-1} \cdot 0{,}06=u_{n-1} \cdot(1+0{,}06) \\
			&=&u_{n-1} \cdot 1{,}06\ \text{với}\ n \geq 2.
		\end{eqnarray*}
		Do đó, $\left(u_n\right)$ là cấp số nhân có số hạng đầu $u_1=100\cdot 1{,}06$ (triệu đồng), công bội $q=1{,}06$.\\
		Vậy số tiền mà bác Linh có được sau $n$ năm là
		\[u_n=100 \cdot 1{,}06^{n}\ (\text{triệu đồng}). \]
	}
\end{vd}
\begin{vd}%[VD] %[DCHT Toán 11 - KNTT - Dung Phuong] %[1K2K7-7]
	Một hình vuông có cạnh $1$ đơn vị dài được chia thành chín hình vuông nhỏ hơn và hình vuông ở chính giữa được tô màu xanh như hình. Mỗi hình vuông nhỏ hơn lại được chia thành chín hình vuông con, và mỗi hình vuông con ở chính giữa lại được tô màu xanh. Nếu quá trình này được tiếp tục lặp lại năm lần, thì tổng diện tích các hình vuông được tô màu xanh là bao nhiêu?
	\dapso{$\dfrac{26281}{39366}$}
	\begin{center}
		\begin{tikzpicture}[>=stealth,thick,scale=0.7]
			\def\n{1}
			\def\a{3}
			\pgfmathsetmacro{\m}{int(3^(\n))}
			\def\hv#1{
				\ifnum#1>0
				\fill[blue!50] (-\a/3,-\a/3) rectangle (\a/3,\a/3);
				\pgfmathtruncatemacro{\k}{#1-1}
				\foreach \i in {0,...,3}{\begin{scope}[shift={(90*\i:2)},scale=1/3]\hv{\k}\end{scope}}
				\foreach \i in {0,...,3}{\begin{scope}[shift={(45+90*\i:{4/sqrt(2)})},scale=1/3]\hv{\k}\end{scope}}
				\fi
			}
			\draw(-\a,-\a) rectangle (\a,\a);
			\hv{\n}
			\foreach \i in {0,1,...,\m}{
				\draw[blue!50] 
				({-\a+2*\i *\a/\m},\a)--++(270:2*\a)
				(\a,{-\a+2*\i *\a/\m})--++(180:2*\a)
				;
			}
		\end{tikzpicture}
		\begin{tikzpicture}[>=stealth,thick,scale=0.7]
			\def\n{2}
			\def\a{3}
			\pgfmathsetmacro{\m}{int(3^(\n))}
			\def\hv#1{
				\ifnum#1>0
				\fill[blue!50] (-\a/3,-\a/3) rectangle (\a/3,\a/3);
				\pgfmathtruncatemacro{\k}{#1-1}
				\foreach \i in {0,...,3}{\begin{scope}[shift={(90*\i:2)},scale=1/3]\hv{\k}\end{scope}}
				\foreach \i in {0,...,3}{\begin{scope}[shift={(45+90*\i:{4/sqrt(2)})},scale=1/3]\hv{\k}\end{scope}}
				\fi
			}
			\draw(-\a,-\a) rectangle (\a,\a);
			\hv{\n}
			\foreach \i in {0,1,...,\m}{
				\draw[blue!50] 
				({-\a+2*\i *\a/\m},\a)--++(270:2*\a)
				(\a,{-\a+2*\i *\a/\m})--++(180:2*\a)
				;
			}
		\end{tikzpicture}
		\begin{tikzpicture}[>=stealth,thick,scale=0.7]
			\def\n{4}
			\def\a{3}
			\def\hv#1{
				\ifnum#1>0
				\fill[blue!50] (-\a/3,-\a/3) rectangle (\a/3,\a/3);
				\pgfmathtruncatemacro{\k}{#1-1}
				\foreach \i in {0,...,3}{\begin{scope}[shift={(90*\i:2)},scale=1/3]\hv{\k}\end{scope}}
				\foreach \i in {0,...,3}{\begin{scope}[shift={(45+90*\i:{4/sqrt(2)})},scale=1/3]\hv{\k}\end{scope}}
				\fi
			}
			\draw (-\a,-\a) rectangle (\a,\a);
			\hv{\n}
		\end{tikzpicture}
	\end{center}
	\loigiai{
		Lần phân chia thứ nhất, $1$ hình vuông thành $9$ hình vuông con, diện tích hình vuông tô màu xanh là $u_1=\dfrac{1}{9}$.\\
		Lần phân chia thứ hai, $8$ hình vuông thành $9$ hình vuông con, diện tích hình vuông tô màu xanh tăng thêm là $u_2=\dfrac{1}{9}\left(\dfrac{8}{9}\right)$.\\
		Lần phân chia thứ ba, $8^2$ hình vuông thành $9$ hình vuông con, diện tích hình vuông tô màu xanh tăng thêm là $u_3=\dfrac{1}{9}\left(\dfrac{8}{9}\right)^2$.\\
		Lần phân chia thứ tư, $8^3$ hình vuông thành $9$ hình vuông con, diện tích hình vuông tô màu xanh tăng thêm là $u_4=\dfrac{1}{9}\left(\dfrac{8}{9}\right)^3$.\\
		Lần phân chia thứ năm, $8^4$ hình vuông thành $9$ hình vuông con, diện tích hình vuông tô màu xanh tăng thêm là $u_5=\dfrac{1}{9}\left(\dfrac{8}{9}\right)^4$.\\
		Như vậy diện tích các hình vuông tăng thêm sau mỗi lần chia  tạo thành cấp số nhân có công bội là $q=\dfrac{8}{9}$, số hạng đầu là $u_1=\dfrac{1}{9}$.\\
		Do đó, tổng diện tích hình vuông tô màu xanh sau $5$ lần chia là\\
		\[u_1+u_2+u_3+u_4+u_5=\dfrac{1-q^5}{1-q}\cdot u_1=\dfrac{1-\left(\dfrac{8}{9}\right)^5}{1-\dfrac{8}{9}}\cdot \dfrac{1}{9}=\dfrac{26281}{39366}.\]	
		
	}
\end{vd}

\begin{vd}%[TH] %[DCHT Toán 11 - KNTT - Dung Phuong] %[1K2B7-7]
	Một khay nước có nhiệt độ $23^\circ$ được đặt vào ngăn đá của tủ lạnh. Biết sau mỗi giờ, nhiệt độ của nước giảm $20\%$. Tính nhiệt độ của khay nước đó sau $6$ giờ theo đơn vị độ $C$.
	\dapso{$ \approx 7,5^\circ $ gam}
	\loigiai{
		Nhiệt độ sau mỗi giờ của khay nước theo thứ tự lập thành cấp số nhân với $u_1=23.(1-20\%)$ và $q=(1-20\%)$.\\
		Ta có $u_6=u_1.q^5=23.(1-20\%)^6 \approx 7,5$.\\
		Nhiệt độ của khay nước sau $6$ giờ là $ \approx 6,0^\circ $.
	}
	
\end{vd}
\begin{vd}%[TH] %[DCHT Toán 11 - KNTT - Dung Phuong] %[1K2B7-7]
	Chu kì bán rã của nguyên tố phóng xạ poloni $210$ là $138$ ngày, nghĩa là sau $138$ ngày, khối lượng của nguyên tố đó chi còn một nửa (theo: https://vi.wikipedia.org/wiki/Poloni-210). Tính khối lượng còn lại của $20$ gam poloni $210$ sau:
	\begin{listEX}[2]
		\item [a)]  $690$ ngày;
		\item [b)] $7314$ ngày (khoảng $20$ năm).
	\end{listEX}
	\dapso{$\dfrac{20}{2^{53}}$ gam}
	\loigiai{
		\begin{listEX}[1]
			\item [a)] Ta có $\dfrac{690}{138}=5$ suy ra khối lượng còn lại sau 690 này là $\dfrac{20}{2^5}=0{,}625$ gam;
			\item [b)] Ta có $\dfrac{7314}{138}=53$ suy ra khối lượng còn lại sau 7314 này là $\dfrac{20}{2^{53}}$ gam.
		\end{listEX}
	}
\end{vd}	
\begin{vd}%[TH] %[DCHT Toán 11 - KNTT - Dung Phuong] %[1K2K7-7]
	Tế bào E.Coli trong điều kiện nuôi cấy thích hợp cứ $20$ phút lại phân đôi một lần. Hỏi sau $24$ giờ, tế bào ban đầu sẽ phân chia thành bao nhiếu tế bào?
	\dapso{$2^{72}$}
	\loigiai{
		Lần phân chia thứ nhất, $1$ tế bào thành $2$ tế bào, số tế bào lần $1$ phân chia là $u_1 = 2$.\\
		Lần phân chia thứ hai $2$, số tế bào lần $2$ phân chia là  $u_2=2\cdot 2 = u_1 \cdot 2$.\\
		Lần phân chia thứ $3$ có  $4$ tế bào phân chia, số tế bào lần $3$ phân chia là $u_3=2\cdot u_2$.\\
		Như vậy một tế bào phân đôi sẽ tạo thành cấp số nhân có công bội là $2$, số hạng đầu là $u_1=2$.\\
		Sau $n$ lần phân chia từ một tế bào phân được thành $u_n=2^{n-1}u_1$.\\
		Đổi $24$ giờ $=24 \cdot 60 =  72 \cdot 20$ (phút)  $\Rightarrow 24$ giờ gấp $72$ lần $20$ phút. \\
		Do đó, sau $24$ giờ số tế bào nhận được là $u_{72}=2^{71}\cdot 2 = 2^{72}$ (tế bào).
	}
\end{vd}

\subsubsection{Bài tập tự luận}
 


\begin{bt}%[TH] %[DCHT Toán 11 - KNTT - Dung Phuong] %[1K2B7-7]
	Một quốc gia có dân số năm 2011 là $P$ triệu người. Trong $10$ năm tiếp theo, mỗi năm dân số tăng $a \%$. Chứng minh rằng dân số các năm từ năm 2011 đến năm 2021 của quốc gia đó tạo thành cấp số nhân. Tìm công bội của cấp số nhân này.
	\loigiai{
		Coi ngày điều tra dân số năm 2011 và năm 2021 trùng nhau thì từ năm 2011 đến năm 2021 là 10 năm. Vậy dân số nước ta tính đến năm 2021 là 
		\[u_{10} = P\cdot \left(1+a\%\right)^{10}.\]
		Ta có \[u_{1} = P\cdot \left(1+a\%\right)^{1}.\]
		\[u_{2} = P\cdot \left(1+a\%\right)^{2}.\]
		Và công bội của cáp số nhân này là $\, \dfrac{u_2}{u_1} = q = \dfrac{P\cdot \left(1+a\%\right)^{2}}{P\cdot \left(1+a\%\right)^{1}} = 1+a\%.$
	}
\end{bt}
\begin{bt}%[TH] %[DCHT Toán 11 - KNTT - Dung Phuong] %[1K2B7-7]
	Vào năm 2020, dân số của một quốc gia là khoảng $97$ triệu người và tốc độ tăng trưởng dân số là $0{,}91 \%$. Nếu tốc độ tăng trưởng dân số này được giữ nguyên hằng năm, hãy ước tính dân số của quốc gia đó vào năm 2030.
	\dapso{$106{,}1973784$}
	\loigiai{
		Dân số năm 2021 tăng lên so với năm 2020 là $97 \cdot 0{,}91 \% $ triệu người.\\
		Dân số năm 2021 là 
		\begin{center}
			$97 + 97 \cdot 0{,}91 \% = 97\cdot (1+0{,}91 \%)$ triệu người.
		\end{center}
		Dân số năm 2022 tăng lên so với năm 2021 là $97\cdot (1+0{,}91 \%)\cdot 0{,}91 \% $ triệu người.\\
		Dân số năm 2022 là 
		\begin{center}
			$97\cdot (1+0{,}91 \%) + 97\cdot (1+0{,}91 \%) \cdot 0{,}91 \% = 97\cdot (1+0{,}91 \%)^2$ triệu người.
		\end{center}
		Dân số năm 2023 tăng lên so với năm 2021 là $97\cdot (1+0{,}91 \%)^2\cdot 0{,}91 \% $ triệu người.\\
		Dân số năm 2023 là
		\begin{center}
			$97\cdot (1+0{,}91 \%)^2 + 97\cdot (1+0{,}91 \%)^2\cdot 0{,}91 \% = 97\cdot (1+0{,}91 \%)^3$ triệu người.
		\end{center}
		Tương tự vậy ta có dân số năm 2030 là $97\cdot (1+0{,}91 \%)^{10} = 106{,}1973784$ triệu người.
	}
\end{bt}
\begin{bt}%[TH] %[DCHT Toán 11 - KNTT - Dung Phuong] %[1K2B7-7]
	Một tỉnh có $2$ triệu dân vào năm 2020 với tỉ lệ tăng dân số là $1$ \%/năm. Gọi $u_n$ là số dân của tỉnh đó sau $n$ năm. Giả sử tỉ lệ tăng dân số là không đổi.
	\begin{enumEX}[a)]{1} 
		\item Viết công thức tính số dân của tỉnh đó sau $n$ năm kể từ năm 2020.
		\item Tính số dân của tỉnh đó sau $10$ năm kể từ năm 2020.  
	\end{enumEX}
	\loigiai{
		\begin{enumEX}[a)]{1} 
			\item Với $u_n$ là số dân của tỉnh đó sau $n$ năm. \\
			Ta có $u_1=2 \cdot 1,01$ (triệu dân).\\
			$u_{n+1}=u_n+u_n\cdot 0{,}01 = 1{,}01u_n$. \\
			Do đó, $(u_n)$ là cấp số nhân với số hạng đầu $u_1=2 \cdot 1,01$ và công bội $q=1{,}01$. \\
			Vậy công thức tính số dân của tỉnh đó sau $n$ năm là $u_n=u_1q^{n-1}\Rightarrow u_n=2\cdot 1{,}01^{n}$.
			\item Số dân của tỉnh đó sau $10$ năm kể từ năm 2020 là $u_{10}=2\cdot 1{,}01^10 = 2{,}209$ (triệu dân).
		\end{enumEX}
	}
\end{bt}
\begin{bt}%[TH] %[DCHT Toán 11 - KNTT - Dung Phuong] %[1K2B7-7]
	Giả sử một thành phố có dân số năm 2022 là khoảng $2{,}1$ triệu người và tốc độ gia tăng dân số trung bình mỗi năm là $0{,}75 \%$.
	\begin{listEX}[1]
		\item [a)]  Dự đoán dân số của thành phố đó vào năm $2032$; \dapso{$\approx2262924$ (người)}
		\item [b)]  Nếu tốc độ gia tăng dân số vẫn giữ nguyên như trên thì uớc tính vào năm nào dân số của thành phố đó sẽ tăng gấp đôi so với năm $2022$?  \dapso{$2116$}	\end{listEX}
	
	\loigiai{
		\begin{listEX}
			\item [a)] 	Giả sử dân số năm $2022$ là $u_1=2{,}1\cdot 10^6$ thì dân số năm 
			$2023$ 
			là 
			$u_2=u_1+ 0{,}0075u_1=1{,}0075u_1$.\\
			Tương tự dân số năm $2024$ là $u_3=1{,}0075u_2$.\\
			Do đó dân số của thành phố qua các năm lập thành một cấp số nhân với 
			$u_1=2{,}1\cdot10^6$; $q=1{,}0075$.\\
			Vậy dân số năm $2032$ tương ứng với $u_{11}=u_1\cdot q^{10}=2,1\cdot 
			10^6\cdot1{,}0075^{10}\approx2262924$ (người).
			\item [b)] Giả sử đến năm thứ $n$ thì dân số gấp đôi năm $2022$. \\
			Suy ra 
			$u_n=2u_1 \Leftrightarrow q^{n-1}=2\Leftrightarrow  1{,}0075^{n-1}=2 
			\Leftrightarrow n \approx 93{,}7.$\\
			Vậy $94$ năm sau tức là năm $2116$ thì dân số thành phố sẽ gấp đôi năm $2022$.
	\end{listEX}}
\end{bt}
\begin{bt}%[TH] %[DCHT Toán 11 - KNTT - Dung Phuong] %[1K2B7-7]
	Giả sử anh Tuấn kí hợp đồng lao động trong $10$ năm với điều khoản về tiền lương như sau: Năm thứ nhất, tiền lương của anh Tuấn là $60$ triệu. Kể từ năm thứ hai trở đi, mỗi năm tiền lương của anh Tuấn được tăng lên $8 \%$. Tính tổng số tiền lương anh Tuấn lĩnh được trong $10$ năm đi làm (đơn vị: triệu đồng, làm tròn đến hàng phần nghìn).
	\dapso{$\approx 869{,}194$ triệu người}
	\loigiai{
		Gọi $u_n$ là số tiền lương (triệu đồng) anh Tuấn được lĩnh ở năm làm việc thứ $n$. Ta có: $u_1=60$;
		\[u_n=u_{n-1}+u_{n-1} \cdot 0{,}08=u_{n-1} \cdot(1+0{,}08)=u_{n-1} \cdot 1{,}08. \]
		Do đó, $\left(u_n\right)$ là cấp số nhân có số hạng đầu $u_1=60$, công bội $q=1{,}08$. Áp dụng công thức tính tổng $S_n$, ta có tổng số tiền lương anh Tuấn lĩnh được trong $10$ năm đi làm là
		\[S_{10}=\dfrac{60\cdot\left(1-1{,}08^{10}\right)}{1-1{,}08} \approx 869{,}194\ (\text{triệu người}). \]
	}
\end{bt}
\begin{bt}%[TH] %[DCHT Toán 11 - KNTT - Dung Phuong] %[1K2B7-7]
	Một công ty xây dựng mua một chiếc máy ủi với giá $3$ tỉ đồng. Cứ sau mỗi năm sử dụng, giá trị của chiếc máy ủi này lại giảm $20 \%$ so với giá trị của nó trong năm liền trước đó. Tìm giá trị còn lại của chiếc máy ủi đó sau $5$ năm sử dụng.
	\dapso{$983$ triệu đồng}
	\loigiai{
		Gọi $u_n$ là Giá trị của máy ủi sau $n$ sử dụng. \\
		Dãy số ($u_n$) là một cấp số nhân có $u_1=3.0,8$, $q=0{,}8$.\\
		Số hạng tổng quát của cấp số nhân này là $u_n=3\cdot 0{,}2^{n}$.\\
		Ta có $u_5=3\cdot 0{,}8^5=0{,}98304$.\\
		Tương ứng giá trị của chiếc máy ủi sau $5$ năm xấp xỉ $983$ triệu đồng.
	}
\end{bt}
\begin{bt}%[TH] %[DCHT Toán 11 - KNTT - Dung Phuong] %[1K2B7-7]
	Một gia đình mua một chiếc ô tô giá $800$ triệu đồng. Trung bình sau mỗi năm sử dụng, giá trị còn lại của ô tô giảm đi $4 \%$ (so với năm trước đó).
	\begin{enumEX}[a)]{1} 
		\item Viết công thức tính giá trị của ô tô sau $1$ năm, $2$ năm sử dụng.
		\item Viết công thức tính giá trị của ô tô sau $n$ năm sử dụng.
		\item Sau $10$ năm, giá trị của ô tô ước tính còn bao nhiêu triệu đồng? 
	\end{enumEX} 
	\dapso{$\approx 531{,}87$ triệu đồng}
	\loigiai{
		Gọi $u_n$ là giá trị còn lại của ô tô sau $n$ năm sử dụng. 
		\begin{enumEX}[a)]{1} 
			\item Giá trị của ô tô sau $1$ năm sử dụng là $u_1=800-800\cdot0{,}04=800\cdot0{,}96=768$ triệu đồng.\\
			Giá trị của ô tô sau $2$ năm sử dụng là $u_2=u_1-u_1\cdot0{,}04=u_1\cdot0{,}96=737{,}28$ triệu đồng.
			\item Ta có $u_n=u_{n-1}-u_{n-1}\cdot0{,}04=u_{n-1}\cdot0{,}96$. \\
			Do đó, $(u_n)$ là cấp số nhân với số hạng đầu $u_1=768$ và công bội $q=0{,}96$. \\
			Vậy sau $n$ năm sử dụng, giá trị còn lại của chiếc ô tô là $u_n=u_1q^{n-1}\Rightarrow u_n=768\cdot0{,}96^{n-1}$.
			\item Sau $10$ năm, ước tính giá trị của ô tô còn lại là $u_{10}=768\cdot0{,}96^9\approx 531{,}87$ triệu đồng.
		\end{enumEX} 
	}
\end{bt}
% \begin{bt} [VD] %[DCHT Toán 11 - KNTT - Dung Phuong] %[1K2K7-7]
% 	Ông An vay ngân hàng $1$ tỉ đồng với lãi suất $12\%/$năm. Ông đã trả nợ theo cách: Bắt đầu từ tháng thứ nhất sau khi vay, cuối tháng ông trả ngân hàng số tiền là $a$ (đồng) và đã trả hết nợ sau đúng $2$ năm kể từ ngày vay. Hỏi số tiền mỗi tháng mà ông An phải trả là bao nhiêu đồng (làm tròn kết quả đến hàng nghìn)?
% 	\dapso{$47073500$}
% 	\loigiai{
% 		Do lãi suất là $12\%$/năm tương đương với lãi là $1\%$/tháng.\\
% 		Sau $1$ tháng, ông An còn nợ là: $10^9.(1+1\%)-a=10^9.(1,01)-S_1$.\\
% 		Sau $2$ tháng, ông An còn nợ là: $10^9.(1.01)^2-a.(1.01)-a=10^9(1,01)^2-S_2$.\\
% 		Sau $3$ tháng, ông An còn nợ là: $10^9.(1.01)^3-a(1.01)^2-a(1.01)-a=10^9.(1.01)^3-S_3$.\\
% 		Sau $24$ tháng, ông An còn nợ là: $10^9.(1.01)^{24}-S_{24}=0$.\\
% 		Do đó $S_{24}=10^9.(1.01)^{24}  \Leftrightarrow a.\dfrac{1-(1.01)^{24}}{1-(1.01)}=10^9.(1.01)^{24} \Leftrightarrow a =\dfrac{10^9.(1.01)^{24}.0.01}{(1.01)^{24}-1}\approx 47073472,22$.\\
% 		Vậy mỗi tháng ông An phải trả $47073500$.
% 	}
% \end{bt}
\begin{bt}%[VD] %[DCHT Toán 11 - KNTT - Dung Phuong] %[1K2K7-7]
	\immini
	{
		Một người nhảy bungee (một trò chơi mạo hiểm mà người chơi nhảy từ một nơi có địa thế cao xuống với dây đai an toàn buộc xung quanh người) từ một cây cầu và căng một sợi dây dài $100$ m. Sau mỗi lần rơi xuống, nhờ sự đàn hồi của dây, người nhảy được kéo lên một quãng đường có độ dài bằng $75$\% so với lần rơi trước đó và lại bị rơi xuống đúng bằng quãng đường vừa được kéo lên. Tính tổng quãng đường người đó đi được sau $10$ lần kéo lên và lại rơi xuống. 
	}
	{
		\begin{tikzpicture}[xscale=.5, font=\small, line join=round, line cap=round, >=stealth,yscale=1]
			\def\a{-0.12} % Hệ số a phải khác 0
			\def\b{0.86}
			\def\c{0}
			\def\m{-0.12} % Hệ số a phải khác 0
			\def\n{0.86}
			\def\p{-0.3}
			\clip (-2,-2)rectangle(9,2);
			\fill[red!40] (-2,1.8)--(9,1.8)--(9,1.55)--(-2,1.55)--cycle;%Mặt phẳng của cây cầu
			\fill[green!50] (-2,1.55)--(0,0)--(1.5,-1)--(-1,-2)--(-2,-2)--cycle;
			\fill[green!50] (9,1.55)--(7,0)--(5.5,-1)--(8,-2)--(9,-2)--cycle;
			\fill[blue!40] (-1,-2)--(8,-2)--(5.5,-1)--(1.5,-1)--cycle;
			\draw[color=blue!50,line width=2pt,<->] (2,-.5)--(2,1.5)node[left,midway]{$100$ m};	
			\draw[color=blue!50,line width=2pt,<->] (5,-.5)--(5,1)node[midway,right]{$0{,}75\cdot100$ m};
			\draw(3.5,1.55)--(3.5,-.25)node[rotate=200]{\faChild};
			\draw[color=white,<->] (3.5,1.8)--(3.5,1.55); 
		\end{tikzpicture}
	}
	\dapso{$\approx666{,}2 \text{ m}$}
	\loigiai{
		Gọi $u_n$ là quãng đường người đó được kéo lên ở lần thứ $n$ được kéo lên và lại rơi xuống (đơn vị tính: mét). \\
		Ta có $u_1=0{,}75\cdot100=100\cdot1{,}5=75$ m và $u_n=0{,}75\cdot u_{n-1}$. \\
		Vậy $(u_n)$ là cấp số nhân với số hạng đầu $u_1=75$ và công bội $q=0{,}75$. \\
		Tổng quãng đường người đó đi được sau $10$ lần kéo lên và lại rơi xuống là 
		$$\begin{aligned}
			S&=100+2u_1+2u_2+\cdots+2u_{10}\\
			&=100+2S_{10}
			=100+2\cdot\dfrac{75\left(1-0{,}75^{10}\right)}{1-0{,}75}\\
			&\approx666{,}2 \text{ m}.
		\end{aligned}$$ 
	}
\end{bt}
\begin{bt} [TH] %[DCHT Toán 11 - KNTT - Dung Phuong] %[1K2B7-7]
	Một cái tháp có $11$ tầng. Diện tích của mặt sàn tầng $2$ bằng nửa diện tích của mặt đáy tháp và diện tích của mặt sàn mỗi tầng bằng nửa diện tích của mặt sàn mỗi tầng ngay bên dưới. Biết mặt đáy tháp có diện tích là $12 288m^2$. Tính diện tích của mặt sàn tầng trên cùng của tháp theo đơn vị mét vuông.
	\dapso{$12m^2$}
	\loigiai{ (Lưu ý: Một số nơi xem tầng 1 là tầng trệt. Nên bài toán này giống bài toán tháp 10 tầng ở phần trên)
		Do diện tích của mặt sàn tính từ tầng một lập thành một cấp số nhân với $u_2=\dfrac{1}{2}.12288=6144$ và $q=\dfrac{1}{2}$.\\
		Ta có $\heva{u_2&=6144 \\ q&=\dfrac{1}{2}}  \Leftrightarrow \heva{u_1&=12288 \\ q&=\dfrac{1}{2}}$.\\
		Ta có $u_{11}=u_1.q^{10}=12288.\dfrac{1}{2^{10}}=12m^2$.
		Vậy diện tích của mặt sàn tầng trên cùng là	$12m^2$.
	}
\end{bt}

\begin{bt}%[TH]%[DCHT Toán 11 - KNTT - Dung Phuong]%[1K2B7-7]
	\immini{Cho hình vuông $C_1$ có cạnh bằng $4$. Người ta chia mỗi cạnh hình vuông thành bốn phần bằng nhau và nối các điểm chia một cách thích hợp để có hình vuông $C_2$ . Từ hình vuông $C_2$ lại làm tiếp tục như trên để có hình vuông $C_3$. Cứ tiếp tục quá trình như trên, ta nhận được dãy các hình vuông $C_1, C_2, C_3, \ldots , C_n, \ldots$ Gọi $a_n$ là độ dài cạnh hình vuông $C_n$. Chứng minh rằng dãy số $\left(a_n\right)$ là cấp số nhân.}{
		\begin{tikzpicture}[scale=.8]
			\def\a{2}  %cạnh hình vuông
			\def\t{.7}  % tỷ lệ điểm cho vòng lặp tiếp
			\path 
			(-\a,-\a) coordinate (A1)
			(-\a,\a) coordinate (B1)
			(\a,\a) coordinate (C1)
			(\a,-\a) coordinate (D1);
			\draw (A1)--(B1)--(C1)--(D1)--cycle;
			\foreach \i[count=\j from 2] in {1,...,10}
			\draw
			(barycentric cs:A\i=\t,B\i=1-\t) coordinate (A\j)--
			(barycentric cs:B\i=\t,C\i=1-\t) coordinate (B\j)--
			(barycentric cs:C\i=\t,D\i=1-\t) coordinate (C\j)--
			(barycentric cs:D\i=\t,A\i=1-\t) coordinate (D\j)--cycle
			;	
			% \node at (0,-2.2) [below]{\textit{Hình 4}};
		\end{tikzpicture}	
	}
	\loigiai{
		\immini{Gọi cạnh một hình vuông thứ $n$, $n+1$ lần lượt là $a_n, a_{n+1}$.\\
			Do $MN=\sqrt{MB^2+BN^2}=\sqrt{\left(\dfrac{AB}{4}\right)^2+\left(\dfrac{3AB}{4}\right)^2 }=AB\cdot\dfrac{\sqrt{10}}{4}$.\\
			Nên ta có cạnh hình vuông thứ $n+1$ là:\\ $a_{n+1}=a_n.\dfrac{\sqrt{10}}{4}$.\\
			Vậy dãy số $\left(a_n\right)$ là cấp số nhân.	
		}{
			\begin{tikzpicture}[scale=0.8,>=stealth, font=\footnotesize, line join=round, line cap=round]
				\path
				(0,0) coordinate (A)
				(4,0) coordinate (B)
				(4,4) coordinate (C)
				(0,4) coordinate (D)
				($(A)!0.75!(B)$) coordinate (M)
				($(B)!0.75!(C)$) coordinate (N)
				($(C)!0.75!(D)$) coordinate (P)
				($(D)!0.75!(A)$) coordinate (Q)
				;
				\draw (A)--(B)--(C)--(D)--cycle (M)--(N)--(P)--(Q)--cycle;
				\node at ($(A)!0.5!(B)$)[below]{$a_n$};
				\node at ($(M)!0.5!(N)$)[left]{$a_{n+1}$};
				\foreach \p/\q in {A/180,B/0,C/0,D/180,M/-90,N/0,P/90,Q/180}
				\fill[black] (\p) circle (1.0pt) ($(\p)+(\q:2.5mm)$) node{$\p$};
		\end{tikzpicture}
	}
	}
\end{bt}

\begin{bt}%[TH] %[DCHT Toán 11 - KNTT - Dung Phuong] %[1K2B7-7]
	Một cây đàn organ có tần số âm thanh các phím liên tiếp tạo thành một cấp số nhân. Cho biết tần số phím La trung là $400$ Hz và tần số của phím La cao cao hơn $12$ phím là $800$ Hz (nguồn: https://vi.wikipedia.org/wikiOrgan). Tìm công bội của cấp số nhân nói trên (làm tròn kết quả đến hàng phần nghìn).
	\dapso{$q = \pm \sqrt[12]{2}$}
	\loigiai{
		Theo đề ta có $\heva{&u_1=400\\&u_{13}=800} \Leftrightarrow \heva{&u_1=400\\&u_1q^{12}=800} \Rightarrow q^{12} = 2 \Rightarrow q = \pm \sqrt[12]{2}$.
	}
\end{bt}


\begin{bt}%[VD]%[DCHT Toán 11 - KNTT - Dung Phuong] %[1K2K7-7] 
	Một loại thuốc được dùng mỗi ngày một lần. Lúc đầu nồng độ thuốc trong máu của bệnh nhân tăng nhanh, nhưng mỗi liều kế tiếp có tác dụng ít hơn liều trước đó. Lượng thuốc trong máu ở ngày thứ nhất là $50 \,\mathrm{mg}$, và mỗi ngày sau đó giảm chỉ còn một nửa so với ngày kề trước đó. Tính tổng lượng thuốc (tính bằng $\mathrm{mg}$) trong máu của bệnh nhân sau khi dùng thuốc $10$ ngày liên tiếp.
	\dapso{$99{,}902$ mg.}
	\loigiai{
		Gọi $u_n$ là giá trị của lượng thuốc trong máu của bệnh nhân trong ngày thứ $n$. \\
		Dãy số này là một cấp số nhân có $u_1=50$, $q=\dfrac{1}{2}$.\\
		Tổng của $n$ số hạng đầu tiên của cấp số nhân là $S_n=u_1\dfrac{1-q^n}{1-q}$.\\
		Theo bài toán, ta có $S_{10}=50 \cdot\dfrac{1-\left(\dfrac{1}{2}\right)^{10}}{1-\dfrac{1}{2}} \approx 99{,}902$.\\
		Vậy tổng lượng thuốc trong máu của bệnh nhân sau khi dùng thuốc $10$ ngày liên tiếp là $99{,}902$ mg.	
	}
\end{bt}
\subsubsection{Câu hỏi trắc nghiệm}
\Opensolutionfile{ans}[ans/ans-1K2-2-Dang7]
\begin{ex}%[1K2K7-7]
	\immini{
		Cho hình vuông có cạnh là $1$. Nối các trung điểm của hình vuông trên ta được một hình vuông có diện tích $S_1$, tiếp tục quá trình trên với các hình vuông với diện tích là $S_2$; $S_3$; $\ldots ;S_n;\ldots$. Tính tổng vô hạn $S_1+ S_2+ S_3+\cdots+S_n+\cdots$.
		\choice
		{$2$}
		{$\dfrac{1}{2}$}
		{\True $1$}
		{$\dfrac{3}{2}$}
	}
	{\hspace*{1 cm}
		\begin{tikzpicture}[scale=0.8,line cap=round,line join=round]
			\path
			(0,0) coordinate (A)
			(4,0) coordinate (B)
			(0,4) coordinate (D);			
			\coordinate (C) at ($(B)-(A)+(D)$);
			\coordinate (H) at ($(A)!0.5!(B)$);
			\coordinate (I) at ($(A)!0.5!(D)$);
			\coordinate (J) at ($(D)!0.5!(C)$);
			\coordinate (K) at ($(B)!0.5!(C)$);
			\coordinate (E) at ($(I)!0.5!(H)$);
			\coordinate (F) at ($(H)!0.5!(K)$);
			\coordinate (G) at ($(K)!0.5!(J)$);
			\coordinate (O) at ($(J)!0.5!(I)$);
			\coordinate (M) at ($(E)!0.5!(F)$);
			\coordinate (N) at ($(F)!0.5!(G)$);
			\coordinate (P) at ($(G)!0.5!(O)$);
			\coordinate (Q) at ($(O)!0.5!(E)$);
			\draw (A)--(B)--(C)--(D)--cycle (I)--(H)--(K)--(J)--cycle
			(E)--(F)--(G)--(O)--cycle (M)--(N)--(P)--(Q)--cycle;
			\foreach \p in {A,B,C,D,E,F,G,H,I,J,K,M,N,P,Q,O}
			\fill[black] (\p) circle (1.0pt);			
		\end{tikzpicture}
	}
	\loigiai{
		Ta có $S_1=\dfrac{1}{2}$, $S_2=\dfrac{1}{4}$, $S_3=\dfrac{1}{8},\cdots  S_n=\dfrac{1}{2^n},\ldots$ tạo thành $1$ cấp số nhân với công bội $q=\dfrac{1}{2}<1$. \\
		Vậy $S_1+ S_2+ S_3+\cdots+S_n+\cdots=\dfrac{\dfrac{1}{2}}{1-\dfrac{1}{2}}=1$.
	}
\end{ex}
\begin{ex}%[1K2K7-7]
	Cho $n$ là số nguyên dương và $n$ tam giác $A_1B_1C_1,A_2B_2C_2,\ldots,A_nB_nC_n$, trong đó các điểm lần ${A}_{i+1},{B}_{i+1},{C}_{i+1}$ lượt nằm trên các cạnh $B_iC_i,A_iC_i,A_iB_i(i=1,2,\ldots,n-1)$ sao cho ${A}_{i+1}C_i=3{A}_{i+1}B_i,{B}_{i+1}A_i=3{B}_{i+1}C_i,{C}_{i+1}B_i=3{C}_{i+1}A_i$. Gọi $S$ là tổng tất cả các diện tích của tam giác $A_1B_1C_1,A_2B_2C_2,\ldots,A_nB_nC_n$ biết rằng tam giác $A_1B_1C_1$ có diện tích bằng $\dfrac{9}{16}$. Tìm số nguyên dương sao cho $S=\dfrac{{16}^{29}-7^{29}}{{16}^{29}}$.
	\choice
	{$n=28$}
	{$n=2018$}
	{$n=30$}
	{\True $n=29$}
	\loigiai{
		Gọi $S_i(i=1,2,3,...,n)$ là diện tích của $\Delta A_iB_iC_i$. Ta có $\dfrac{S_{A_1B_2C_2}}{S_{A_1B_1C_1}}=\dfrac{A_1B_2}{A_1C_1}\cdot \dfrac{A_1C_2}{A_1B_1}=\dfrac{1}{4}\cdot \dfrac{3}{4}=\dfrac{3}{16}$. Tương tự, ta có $\dfrac{S_{A_2B_1C_2}}{S_{A_1B_1C_1}}=\dfrac{S_{A_2B_2C_1}}{S_{A_1B_1C_1}}=\dfrac{3}{16}$. Do đó $\dfrac{S_{A_2B_2C_2}}{S_{A_1B_1C_1}}=1-3\cdot \dfrac{3}{16}=\dfrac{7}{16}\Rightarrow S_2=\dfrac{7}{16}S_1$.\\
		Tương tự, ta có ${S}_{i+1}=\dfrac{7}{16}S_i,i=1,2,\ldots,n$.
		Khi đó $$S=S_1\left[1+\dfrac{7}{16}+\cdots+{\left(\dfrac{7}{16}\right)}^{n-1}\right]=\dfrac{9}{16}\cdot \dfrac{1-{\left(\dfrac{7}{16}\right)}^n}{1-\dfrac{7}{16}}=1-{\left(\dfrac{7}{16}\right)}^n.$$
		Theo giả thiết ta có $1-{\left(\dfrac{7}{16}\right)}^n=1-{\left(\dfrac{7}{16}\right)}^{29}\Leftrightarrow n=29$.}
\end{ex}
\begin{ex}%[1K2K7-7]
	Người ta thiết kế một cái tháp gồm $11$ tầng. Diện tích bề mặt trên của mỗi tầng bằng nửa diện của mặt trên tầng ngay bên dưới và diện tích tầng $1$ bằng nửa diện tích của đế tháp. Biết đế tháp có diện tích là $12288\, \mathrm{m}^2$. Tính diện tích mặt trên cùng.
	\choice
	{$12\, \mathrm{m}^2$}
	{\True $6\, \mathrm{m}^2$}
	{$10\, \mathrm{m}^2$}
	{$8\, \mathrm{m}^2$}
	\loigiai{
		Gọi $S_{i}$ là diện tích của tầng thứ $i$ với $i = 1,2,\ldots,11$.\\
		Do giả thiết suy ra $S_{i + 1} = \dfrac{1}{2}S_{i}$ với $i = 1,2,\ldots,10$.\\
		Do đó $\left\{S_{i}\right\}$ là một cấp số nhân với công bội $q = \dfrac{1}{2}$. Do đó  $S_{11} = \dfrac{1}{2^{10}}S_{1} = \dfrac{1}{2^{11}}\cdot 12288 = 6\left(\mathrm{m}^2\right)$.
	}
\end{ex}

\begin{ex}%[1K2K7-7]
	Cho tứ giác $ABCD$ có bốn góc tạo thành cấp số nhân có công bội $ q=2 $. Góc có số đo nhỏ nhất trong bốn góc đó là
	\choice
	{\True $ 24^\circ $}
	{$ 1^\circ $}
	{$ 12^\circ $}
	{$ 30^\circ $}
	\loigiai{
		Gọi số đo bốn góc của tứ giác $ ABCD $ là $ x $, $ 2x $, $ 4x $, $ 8x $.
		\\ Có $ x+2x+4x+8x=360 \Leftrightarrow 15x=360 \Leftrightarrow x=24 $.}
\end{ex}

\begin{ex}%[1K2K7-7]
	Một du khách vào chuồng đua ngựa đặt cược, lần đầu tiên đặt $20000$ đồng, mỗi lần sau tiền đặt gấp đôi lần tiền đặt cược trước. Người đó thua lần $9$ liên tiếp và thắng ở lần thứ $10$. Hỏi du khách đó thắng hay thua bao nhiêu tiền?
	\choice
	{\True Thắng $20000$ đồng}
	{Thua $40000$ đồng}
	{Hòa vốn}
	{Thua $20000$ đồng}
	\loigiai{
		Số tiền đặt cược lần thứ $n$ là $u_n=u_1\cdot 2^{n-1}$ với $u_1=20000$. \\
		Ta có: $u_{10}-\displaystyle\sum_{n=1}^9 u_1\cdot 2^{n-1}=20000\cdot 2^9-\displaystyle\sum_{n=1}^9 20000\cdot 2^{n-1}=20000$. \\
		Vậy du khách thắng $20000$ đồng.
	}
\end{ex}

% \begin{ex}%[1K2K7-7]
% 	Một người gửi tiết kiệm vào ngân hàng với lãi suất $7{,}5$ \%/năm. Biết rằng nếu không rút tiền ra khỏi ngân hàng thì cứ sau mỗi năm số tiền lãi sẽ được nhập vào vốn để tính lãi cho năm tiếp theo. Hỏi sau ít nhất bao nhiêu năm người đó thu được (cả số tiền gửi ban đầu và lãi) gấp đôi số tiền đã gửi, giả định trong khoảng thời gian này lãi suất không thay đổi và người đó không rút tiền ra?
% 	\choice
% 	{$12$ năm}
% 	{$11$ năm}
% 	{\True $10$ năm}
% 	{$9$ năm}
% 	\loigiai{
% 		Áp dụng công thức: $S_n=A(1+r)^n \Rightarrow n=\log_{(1+r)}\left(\dfrac{S_n}{A}\right) \Rightarrow n=\log_{\left(1+7{,}5\%\right)}(2)\approx 9{,}6$.}
% \end{ex}

\begin{ex}%[1K2K7-7]
	Cho tam giác $ ABC $ cân tại $ A $ có cạnh đáy $ BC $,  đường cao $ AH $ và cạnh bên $ AB $ theo thứ tự đó lập thành cấp số nhân công bội $ q $. Giá trị của $ q $ là
	\choice
	{$ q=\dfrac{1}{2}\sqrt{\sqrt{2}+1} $ }
	{$ q=\sqrt{2}+1 $ }
	{$ q=\sqrt{2(\sqrt{2}+1)} $}
	{\True $ q=\dfrac{1}{2}\sqrt{2(\sqrt{2}+1)} $ }
	\loigiai{
		Giả sử $ BC=u_1 $, $ AH=u_1\cdot q $ và $ AB=u_1\cdot q^2 $ với $ u_1> 0, q> 0 $.\\
		Do $ \triangle ABC $ cân tại $ A $ suy ra
		\begin{align*}
			AB^2=AH^2+\dfrac{BC^2}{4}\Leftrightarrow
			& u_1^2\cdot q^4=\dfrac{u_1^2}{4}+u_1^2\cdot q^2\\
			\Leftrightarrow & 4q^4-4q^2-1=0\\
			\Leftrightarrow & q^2=\dfrac{1\pm \sqrt{2}}{2}.
		\end{align*}
		Kết hợp với điều kiện bài toán ta có $ q=\sqrt{\dfrac{1+ \sqrt{2}}{2}}=\dfrac{1}{2}\sqrt{2(\sqrt{2}+1)} $.
	}
\end{ex}
\begin{ex}%[1K2K7-7]
	Giả sử một người đi làm được lĩnh lương khởi điểm là $2.000.000$ đồng/tháng. Cứ $3$ năm người ấy lại được tăng lương một lần với mức tăng bằng $7\%$ của tháng trước đó. Hỏi sau $36$ năm làm việc người ấy lĩnh được tất cả bao nhiêu tiền?
	\choice
	{\True $ 1.287.968.492 $ đồng}
	{$ 10.721.769.110 $ đồng}
	{$ 7{,}068289036\cdot 10^8 $ đồng}
	{$ 429.322.830{,}5 $ đồng}
	\loigiai{
		Ta có $36$ năm tương ứng với $12$ kỳ lương; mỗi kỳ lương có $36$ tháng và kỳ sau tăng $7\%$ so với kỳ trước. Do đó tổng số tiền mỗi kỳ lương là một cấp số nhân với $u_1=36\times 2=72$ (triệu đồng) và công bội $q=1{,}07$.\\
		Vậy tổng số tiền sau $36$ năm là $T=\dfrac{72\cdot \left[(1{,}07)^{12}-1\right]}{1{,}07-1}=1287{,}968492$ (triệu đồng).
	}
\end{ex}

\begin{ex}%[1K2K7-7]
	Từ độ cao $55{,}8$ (mét) của tháp nghiên Pisa nước Italia người ta thả một quả bóng cao su chạm xuống đất. Giả sử mỗi lần chạm đất bóng lại nảy lên độ cao bằng $\dfrac{1}{10}$ độ cao mà bóng đạt trước đó. Tổng độ dài hành trình (mét) của bóng được thả từ lúc ban đầu cho đến khi nó nằm yên trên mặt đất thuộc khoảng nào trong các khoảng sau đây?
	\choice
	{$(69;72)$}
	{$(60;63)$}
	{\True $(67;69)$}
	{$(64;66)$}
	\loigiai{
		Đặt $u_1=55{,}8$ (mét) là quãng đường bóng rơi khi thả xuống, $u_{n+1}=\dfrac{1}{10^{n}} u_1, n\ge 1$ là quãng đường bóng rơi sau lần nảy lên thứ $n$. \\
		Ta có $(u_n)$ là dãy cấp số nhân với $u_1=55{,}8$ và công bội $q=\dfrac{1}{10}$.\\
		Suy ra tổng quãng đường quả bóng rơi xuống là $\displaystyle \lim \limits_{n \rightarrow +\infty} u_1 \cdot \dfrac{1-q^n}{1-q}=\displaystyle \lim \limits_{n \rightarrow +\infty}55{,}8\cdot\dfrac{1-\left( \dfrac{1}{10}\right)^n}{1-\dfrac{1}{10}}=62 $.\\
		Ngoài ra ta còn phải tính tổng quãng đường mà bóng nảy lên. Ta có tổng quãng đường bóng nảy lên bằng tổng quãng đường rơi của bóng trừ đi quãng đường thả rơi xuống.\\
		Vậy tổng quãng đường hành trình của quả bóng là $62+62-55{,}8=68{,}2$ (mét).
	}
\end{ex}

\begin{ex}%[1K2K7-7]
	Một gia đình lập kế hoạch tiết kiệm như sau: Họ lập một sổ tiết kiệm tại một ngân hàng và cứ đầu mỗi tháng họ gửi
	vào sổ tiết kiệm đó $15$ triệu đồng. Giả sử lãi suất tiền gửi không đổi là $0{,}6$ \%/tháng và tiền gửi được tính lãi theo hình thức lãi
	kép. Hỏi sau $3$ năm gia đình đó tiết kiệm được số tiền gần nhất với con số nào dười đây?
	\choice
	{$543240000$ đồng}
	{$589269000$ đồng}
	{$669763000$ đồng}
	{\True $604359000$ đồng}
	\loigiai{
		Gọi $S_0$ triệu đồng là số tiền gia đình đó định kỳ gửi tiết kiệm vào đầu hằng tháng, $r$ là lãi suất tiền gửi hằng tháng. Ta có $S_0=15$ triệu đồng, $r=0{,}6$
		\%/tháng.\\
		Gọi $S_i$, $i=\overline{1,n}$ là số tiền trong sổ tiết kiệm cuối tháng thứ $i$.\\
		Ta có \begin{itemize}
			\item $S_1=S_0+S_0\cdot r=S_0(1+r)$,
			\item  $S_2=\left[ S_0+S_0(1+r)\right]+\left[ S_0+S_0(1+r)\right]r=S_0 (1+r)+S_0(1+r)^2$,
			\item  $\begin{aligned}[t]
				S_3=&\ \left[S_0+S_0(1+r)+S_0(1+r)^2 \right] +\left[S_0+S_0(1+r)+S_0(1+r)^2 \right]r\\
				=&\ S_0(1+r)+S_0(1+r)^2+S_0(1+r)^3,\end{aligned}$,
			\item \ldots
			\item$\begin{aligned}[t]S_n=&\ S_0(1+r)+S_0(1+r)^2+S_0(1+r)^3+\cdots +S_0(1+r)^n\\=&\ S_0\left[ (1+r)+(1+r)^2+(1+r)^3+\cdots+(1+r)^n\right]\\
				=&\  S_0(1+r)\cdot \dfrac{(1+r)^{n}-1}{(1+r)-1}=S_0(1+r)\cdot \dfrac{(1+r)^{n}-1}{r}.
			\end{aligned}$
		\end{itemize}
		Vậy sau $3$ năm, tức cuối tháng thứ $36$ thì gia đình tiết kiệm được số tiền là
		\[S_{36}=15\cdot 10^6(1+0{,}6\cdot 10^{-2})\cdot \dfrac{(1+0{,}6\cdot 10^{-2})^{36}-1}{0{,}6\cdot 10^{-2}}=604358538{,}2 \ \text{đồng}.\]
	}
\end{ex}
\Closesolutionfile{ans}
% \begin{indapan}{10}
% 	{ans/ans-1K2-2-Dang7}
% \end{indapan}

% %%Ôn tập chương II
\setcounter{dang}{0}
\setcounter{ex}{0}
\setcounter{bt}{0}
\setcounter{vd}{0}
\section*{Ôn tập chương 2}
\Opensolutionfile{ans}[ans/ans-1K2-Ontapchuong2]
\begin{ex}%[1K2Y5-2]
	Cho dãy số $\left(u_n\right)$, biết $u_n=\left(-1\right)^n.2n$. Mệnh đề nào sau đây sai?
	\choice
	{$u_1=-2$}
	{$u_2=4$}
	{$u_3=-6$}
	{\True $u_4=-8$}
	\loigiai{
		Thay trực tiếp vào kiểm tra, ta có
		\begin{eqnarray*}
			u_1&=&-2.1=-2\\
			u_2&=&(-1)^2.2.2=4\\
			u_3&=&(-1)^3.2.3=-6\\
			u_4&=&(-1)^4.2.4=8.
		\end{eqnarray*}
	}
\end{ex}
\begin{ex}%[1K2Y5-2]
	Cho dãy số $\left(u_n\right)$, biết $u_n=\left(-1\right)^n.\dfrac{2^n}{n}$. Tìm số hạng $u_3$.
	\choice
	{$u_3=\dfrac{8}{3}$}
	{$u_3=2$}
	{$u_3=-2$}
	{\True $u_3=-\dfrac{8}{3}$}
	\loigiai{
		Thay trực tiếp vào kiểm tra, ta có
		\begin{center}
			$u_3=(-1)^3.\dfrac{2^3}{3}=-\dfrac{8}{3}$.
		\end{center}
	}
\end{ex}
\begin{ex}%[1K2Y5-2]
	Cho dãy số $\left(u_n\right)$, biết $u_n=\dfrac{2n+5}{5n-4}$. Số $\dfrac{7}{12}$ là số hạng thứ mấy của dãy số?
	\choice
	{\True $8$}
	{$6$}
	{$9$}
	{$10$}
\end{ex}
\loigiai{
	Ta có
	\allowdisplaybreaks
	\begin{eqnarray*}
		&&u_n=\dfrac{2n+5}{5n-4}\\
		&\Leftrightarrow&\dfrac{7}{12}=	\dfrac{2n+5}{5n-4}\\
		&\Leftrightarrow&24n+60=35n-28\\
		&\Leftrightarrow&11n=88\\
		&\Leftrightarrow&n=8.
	\end{eqnarray*}
	Vậy số $\dfrac{7}{12}$ là số hạng thứ 8.
}
\begin{ex}%[1K2Y5-2]
	Cho dãy số $\left(u_n\right)$, biết $u_n=2^n$. Tìm số hạng $u_{n+1}$.
	\choice
	{\True $u_{n+1}=2^n.2$}
	{$u_{n+1}=2^n+1$}
	{$u_{n+1}=2\left(n+1\right)$}
	{$u_{n+1}=2^n+2$}
	\loigiai{
		Ta có
	}
	\loigiai{
		Thay $n$ bằng $n+1$ trong công thức $u_n$ ta được
		\allowdisplaybreaks
		\begin{eqnarray*}
			u_{n+1}&=&2^{n+1}\\
			& =&2.2^n.
		\end{eqnarray*}
	}
\end{ex}
\begin{ex}%[1K2B5-2]
	Cho dãy số $\left(u_n\right)$, biết $u_n=5^{n+1}$. Tìm số hạng $u_{n-1}$.
	\choice
	{$u_{n-1}=5^{n-1}$}
	{\True $u_{n-1}=5^{n}$}
	{$u_{n-1}=5.5^{n+1}$}
	{$u_{n-1}=5.5^{n-1}$}
	\loigiai{
		Thay $n$ bằng $n-1$ trong công thức $u_n$ ta được
		\allowdisplaybreaks
		\begin{eqnarray*}
			u_{n-1}& = &5^{n-1+1}\\
			& = &5^n.
		\end{eqnarray*}
	}
\end{ex}
\begin{ex}%[1K2Y5-1]
	Cho dãy số có các số hạng đầu là $-2;0;2;4;6;...$. Số hạng tổng quát của dãy số này là công thức nào dưới đây?
	\choice
	{$u_n=-2n$}
	{$u_n=n-2$}
	{$u_n=-2\left(n+1\right)$}
	{\True $u_n=2n-4$}
	\loigiai{
		Kiểm tra $u_1=-2$ ta loại các đáp án B và C. Tương tự kiểm tra $u_2=0$ ta loại đáp án A.
	}
\end{ex}
\begin{ex}%[1K2B5-1]
	Cho dãy số $\left(u_n\right)$, được xác định $\heva{&u_1=\dfrac{1}{2}\\&u_{n+1}=u_n-2}$. Số hạng tổng quát $u_n$ của dãy số là số hạng nào dưới đây?
	\choice
	{$u_n=\dfrac{1}{2}+2\left(n-1\right)$}
	{\True $u_n=\dfrac{1}{2}-2\left(n-1\right)$}
	{$u_n=\dfrac{1}{2}-2n$}
	{$u_n=\dfrac{1}{2}+2n$}
	\loigiai{
		Ta có
		\begin{center}
			$\heva{&u_1=\dfrac{1}{2}\\&u_{n+1}=u_n-2}\Rightarrow\heva{&u_1=\dfrac{1}{2}\\&u_2=-\dfrac{3}{2}\\&u_3=-\dfrac{7}{2}}$
		\end{center} 
		Ta thấy chỉ có đáp án B đều thoả mãn.
	}
\end{ex}
\begin{ex}%[1K2B5-1]
	Cho dãy số $\left(u_n\right)$, được xác định $\heva{&u_1=-2\\&u_{n+1}=-2-\dfrac{1}{u_n}}$. Số hạng tổng quát $u_n$ của dãy số là số hạng nào dưới đây?
	\choice
	{$u_n=\dfrac{-n+1}{n}$}
	{$u_n=\dfrac{n+1}{n}$}
	{\True $u_n=-\dfrac{n+1}{n}$}
	{$u_n=-\dfrac{n}{n+1}$}
	\loigiai{
		Ta có
		\begin{center}
			$\heva{&u_1=-2\\&u_{n+1}=-2-\dfrac{1}{u_n}}\Rightarrow\heva{&u_1=-2\\&u_2=-\dfrac{3}{2}}$
		\end{center}
		Ta thấy chỉ có đáp án C thoả mãn.
	}
\end{ex}
\begin{ex}%[1K2Y5-2]
	Cho cấp số cộng có số hạng đầu $u_1=-\dfrac{1}{2}$, công sai $d=\dfrac{1}{2}$. Năm số hạng liên tiếp đầu tiên của cấp số này là.
	\choice
	{$-\dfrac{1}{2};0;1;\dfrac{1}{2};1$}
	{$-\dfrac{1}{2};0;\dfrac{1}{2};0;\dfrac{1}{2}$}
	{$\dfrac{1}{2};1;\dfrac{3}{2};2;\dfrac{5}{2}$}
	{\True $-\dfrac{1}{2};0;\dfrac{1}{2};1;\dfrac{3}{2}$}
	\loigiai{
		Ta dùng công thức tổng quát $u_n=u_1+(n-1)d=-\dfrac{1}{2}+(n-1)\dfrac{1}{2}=-1+\dfrac{n}{2}$ để tính các số hạng của một cấp số cộng. Ta có
		\begin{center}
			$u_1=-\dfrac{1}{2},u_2=0,u_3=\dfrac{1}{2},u_4=1,u_5=\dfrac{3}{2}$.
		\end{center}
	}
\end{ex}
\begin{ex}%[1K2B6-3]
	Viết ba số hạng xen giữa các số $2$ và $22$ để được một cấp số cộng có năm số hạng.
	\choice
	{\True 
		$7;12;17$}
	{$6;10;14$}
	{$8;13;18$}
	{$6;12;18$}
	\loigiai{
		Giữa $2$ và $22$ có thêm ba số hạng nữa lập thành cấp số cộng, xem như ta có một cấp số cộng có năm số hạng với $u_1=22;u_5=22$, ta cần tìm $u_2,u_3,u_4$. Ta có
		\begin{eqnarray*}
			&&u_5=u_1+4d\\
			&\Leftrightarrow&d=\dfrac{u_5-u_1}{4}\\&\Leftrightarrow&d=5\\
			&\Rightarrow&\heva{&u_2=7\\&u_3=12\\&u_4=17}.
		\end{eqnarray*}
	}
\end{ex}
\begin{ex}%[1K2K6-3]
	Biết các số $C_n^1;C_n^2;C_n^3$ theo thứ tự lập thành một cấp số cộng với $n>3$. Tìm $n$.
	\choice
	{$n=5$}
	{\True $n=7$}
	{$n=9$}
	{$n=11$}
	\loigiai{
		Ba số $C_n^1;C_n^2;C_n^3$ theo thứ tự $u_1;u_2;u_3$ lập thành một cấp số cộng nên
		\begin{eqnarray*}
			&&u_1+u_3=2u_2\\
			&\Leftrightarrow&C_n^1+C_n^3=2C_n^2\\
			&\Leftrightarrow&n+\dfrac{(n-2)(n-1)n}{6}=2.\dfrac{(n-1)n}{2}\\
			&\Leftrightarrow&1+\dfrac{n^2-3n+2}{6}=n-1\\
			&\Leftrightarrow&n^2-9n+14\\
			&\Leftrightarrow&\hoac{&n=2\\&n=7}	
		\end{eqnarray*}
		Kết hợp với điều kiện $n>3$, do đó $n=7$ thoả mãn yêu cầu bài toán.
	}
\end{ex}
\begin{ex}%[1K2B6-2]
	Cho cấp số cộng $\left(u_n\right)$ có các số hạng đầu lần lượt là $5; 9; 13; 17;...$. Tìm số hạng tổng quát $u_n$ của cấp số cộng.
	\choice
	{$u_n=5n+1$}
	{$u_n=5n-1$}
	{\True $u_n=4n+1$}
	{$u_n=4n-1$}
	\loigiai{
		Các số $5; 9; 13; 17 th;...$ theo thứ tự đó lập thành cấp số cộng $\left(u_n\right)$ nên
		\begin{center}
			$\heva{&u_1=5\\&d=u_2-u_1=4}\Rightarrow u_n=u_1+(n-1)d=5+4(n-1)=4n+1$.
		\end{center}
	}
\end{ex}
\begin{ex}%[1K2K6-1]
	Cho cấp số cộng $\left(u_n\right)$ có $u_1=3$ và $d=\dfrac{1}{2}$. Khẳng định nào sau đây đúng?
	\choice
	{$u_n=-3+\dfrac{1}{2}(n+1)$}
	{$u_n=-3+\dfrac{1}{2}n-1$}
	{\True $u_n=-3+\dfrac{1}{2}(n-1)$}
	{$u_n=-3+\dfrac{1}{4}(n-1)$}
	\loigiai{
		Ta có
		\begin{center}
			$\heva{&u_1=-3\\&d=\dfrac{1}{2}}\Rightarrow u_n=u_1+(n-1)d=-3+\dfrac{1}{2}(n-1)$.
		\end{center}
	}
\end{ex}
\begin{ex}%[1K2K6-1]
	Trong các dãy số được cho dưới đây, dãy số nào là cấp số cộng?
	\choice
	{\True $u_7=7-3n$}
	{$u_7=7-3^n$}
	{$u_7=\dfrac{7}{3n}$}
	{$u_7=7.3^n$}
	\loigiai{
		Dãy $\left(u_n\right)$ là cấp số cộng khi và chỉ khi $u_n=an+b$ với $a,b$ là hằng số.
	}
\end{ex}
\begin{ex}%[1K2B6-3]
	Cho cấp số cộng $\left(u_n\right)$ có $u_1=-5$ và $d=3$. Mệnh đề nào sau đây đúng?
	\choice
	{$u_{15}=34$}
	{$u_{15}=45$}
	{\True $u_{13}=31$}
	{$u_{10}=35$}
	\loigiai{
		Ta có
		\begin{center}
			$\heva{&u_1=-5\\&d=3}\Rightarrow u_n=3n-8\Rightarrow\heva{&u_{15}=37\\&u_{13}=31\\&u_{10}=22}$.
		\end{center}
	}
\end{ex}
\begin{ex}%[1K2B6-3]
	Cho cấp số cộng $\left(u_n\right)$ có $d=-2$ và $S_8=72$. Tìm số hạng đầu tiên $u_1$.
	\choice
	{\True $u_1=16$}
	{$u_1=-16$}
	{$u_1=\dfrac{1}{16}$}
	{$u_1=-\dfrac{1}{16}$}
	\loigiai{
		Ta có $\heva{&d=-2\\&S_8=72}\Leftrightarrow\heva{&d=-2\\&8u_1+\dfrac{8.7}{2}d=72}\Rightarrow 8u_1+28.(-2)=72\Leftrightarrow u_1=16$.
	}
\end{ex}
\begin{ex}%[1K2K6-3]
	Một cấp số cộng có số hạng đầu là $1$, công sai là $4$, tổng của n số hạng đầu là $561$. Khi đó số
	hạng thứ $n$ của cấp số cộng đó là $u_n$ có giá trị là bao nhiêu?
	\choice
	{$u_n=57$}
	{$u_n=61$}
	{\True $u_n=65$}
	{$u_n=69$}
	\loigiai{
		Ta có $\heva{&u_1=1,d=4\\&S_n=561}\Leftrightarrow\heva{&u_=1,d=4\\&nu_1+\dfrac{n(n-1)}{2}d=561}\Rightarrow n+\dfrac{n^2-n}{2}.4=561\Leftrightarrow 2n^2-n-561=0\Leftrightarrow n=17$.\\
		Từ đây suy ra $u_{17}=u_1+16d=1+16.4=65$.
	}
\end{ex}
\begin{ex}%[1K2K6-5]
	Tổng $n$ số hạng đầu tiên của một cấp số cộng là $S_n=\dfrac{3n^2-19n}{4}$ với $n\in\mathbb{N}^*$. Tìm số hạng đầu
	tiên $u_1$ và công sai $d$ của cấp số cộng đã cho.
	\choice
	{$u_1=2,d=-\dfrac{1}{2}$}
	{\True $u_1=-4,d=\dfrac{3}{2}$}
	{$u_1=-\dfrac{3}{2},d=-2$}
	{$u_1=\dfrac{5}{2},d=\dfrac{1}{2}$}
	\loigiai{
		Ta có $\dfrac{3n^2-19n}{4}=\dfrac{3}{4}n^2-\dfrac{19n}{4}=S_n=nu_1+\dfrac{n^2-n}{2}d=\dfrac{d}{2}n^2+\left(u_1-\dfrac{d}{2}\right)n$.\\
		Đồng nhất hai vế của phương trình, ta có $\heva{&\dfrac{d}{2}=\dfrac{3}{4}\\&u_1-\dfrac{d}{2}=-\dfrac{19}{4}}\Leftrightarrow\heva{&u_1=-4\\&d=\dfrac{3}{2}}$.
	}
\end{ex}
\begin{ex}%[1K2K6-3]
	Cho cấp số cộng $\left(u_n\right)$ có $u_2=2001$ và $u_5=1995$. Khi đó $u_{1001}$ bằng.
	\choice
	{$u_{1001}=4005$}
	{$u_{1001}=4003$}
	{\True $u_{1001}=3$}
	{$u_{1001}=1$}
	\loigiai{
		Ta có $\heva{&u_2=2001\\&u_5=1995}\Leftrightarrow\heva{&u_1+d=2001\\&u_1+4d=1995}\Leftrightarrow \heva{&u_1=2003\\&d=-2}\Rightarrow u_{1001}=u_1+1000d=3$.
	}
\end{ex}
\begin{ex}%[1K2B6-1]
	Cho cấp số cộng $\left(u_n\right)$ biết $u_n=-1,u_{n+1}=8$. Tính công sai $d$ của cấp số cộng đó.
	\choice
	{$d=-9$}
	{$d=7$}
	{$d=-7$}
	{\True $d=9$}
	\loigiai{
		Ta có $d=u_{n+1}-u_n=8-(-1)=9$.
	}
\end{ex}
\begin{ex}%[1K2K6-5]
	Cho cấp số cộng $\left(u_n\right)$ thỏa mãn $u_2+u_{23}=60$. Tính tổng $S_24$ của $24$ số hạng đầu tiên của
	cấp số cộng đã cho.
	\choice
	{$S_{24}=60$}
	{$S_{24}=120$}
	{\True $S_{24}=720$}
	{$S_{24}=1440$}
	\loigiai{
		Ta có $u_2+u_{23}=60\Leftrightarrow u_1+d+u_1+22d=60\Leftrightarrow 2u_1+23d=60$.\\
		Khi đó $S_{24}=\dfrac{24}{2}\left(u_1+u_{24}\right)=12\left(u_1+u_1+23d\right)=12.60=720$.
	}
\end{ex}
\begin{ex}%[1K2K6-1]
	Một cấp số cộng có $6$ số hạng. Biết rằng tổng của số hạng đầu và số hạng cuối bằng $17$, tổng
	của số hạng thứ hai và số hạng thứ tư bằng $14$. Tìm công sai $d$ của câp số cộng đã cho.
	\choice
	{$d=2$}
	{\True $d=-3$}
	{$d=4$}
	{$d=5$}
	\loigiai{
		Ta có $\heva{&u_1+u_6=17\\&u_2+u_4=14}\Leftrightarrow\heva{&2u_1+5d=17\\&2u_1+6d=14}\Leftrightarrow\heva{&u_1=16\\&d=-3}$.
	}
\end{ex}
\begin{ex}%[1K2K6-1]
	Cho cấp số cộng $\left(u_n\right)$ thỏa mãn $\heva{&u_7-u_3=8\\&u_2u_7=75}$. Tìm công sai $d$ của cấp số cộng đã cho.
	\choice
	{$d=\dfrac{1}{2}$}
	{$d=\dfrac{1}{3}$}
	{\True $d=2$}
	{$d=3$}
	\loigiai{
		Ta có $\heva{&u_7-u_3=8\\&u_2u_7=75}\Leftrightarrow\heva{&u_1+6d-u_1-2d=8\\&(u_1+d)(u_1+6d)=75}\Leftrightarrow\heva{&d=2\\&(u_1+2)(u_1+12)=75}$.
	}
\end{ex}
\begin{ex}%[1K2K6-3]
	Ba góc của một tam giác vuông tạo thành cấp số cộng. Hai góc nhọn của tam giác có số đo
	(độ) là
	\choice
	{$20^\circ$ và $70^\circ$}
	{$45^\circ$ và $45^\circ$}
	{$20^\circ$ và $45^\circ$}
	{\True $30^\circ$ và $60^\circ$}
	\loigiai{
		Ba góc $A,B,C$ của một tam giác vuông theo thứ tự đó $(A<B<C)$ lập thánh cấp số cộng nên $C=90,C+A=2B$.\\
		Ta có $\heva{&A+B+C=180\\&A+C=2B\\&C=90}\Leftrightarrow\heva{&A=30\\&B=60\\&C+90}$.
	}
\end{ex}
\begin{ex}%[1K2K6-3]
	Một tam giác vuông có chu vi bằng $3$ và độ dài các cạnh lập thành một cấp số cộng. Độ dài các
	cạnh của tam giác đó là
	\choice
	{$\dfrac{1}{2};1;\dfrac{3}{2}$}
	{$\dfrac{1}{3};1;\dfrac{5}{3}$}
	{\True $\dfrac{3}{4};1;\dfrac{5}{4}$}
	{$\dfrac{1}{4};1;\dfrac{7}{4}$}
	\loigiai{
		Ba cạnh $a,b,c,(a<b<c)$ của một tam giác theo thứ tự đó lập thành một cấp số cộng.\\
		Ta có $\heva{&a^2+b^2=c^2\\&a+b+c=3\\&a+c=2b}\Leftrightarrow\heva{&a^2+b^2=c^2\\&3b=3\\&a+c=2b}\Leftrightarrow\heva{&a^2+b^2=c^2\\&b=1\\&a=2-c}$.\\
		Từ đây suy ra $a^2+b^2=c^2\Rightarrow (2-c)^2+1=c^2\Leftrightarrow c=\dfrac{5}{4}\Leftrightarrow\heva{&a=\dfrac{3}{4}\\&b=1\\&c=\dfrac{5}{4}}$.
	}
\end{ex}
\begin{ex}%[1K2K6-6]
	Một rạp hát có $30$ dãy ghế, dãy đầu tiên có $25$ ghế. Mỗi dãy sau có hơn dãy trước $3$ ghế. Hỏi rạp
	hát có tất cả bao nhiêu ghế?
	\choice
	{$1635$}
	{$1792$}
	{\True $2055$}
	{$3125$}
	\loigiai{
		Số ghế của mỗi dãy (bắt đầu từ dãy đầu tiên) theo thứ tự đó lập thành một cấp số cộng có $30$ số hạng có công sai $d=3$ và $u_1=25$.\\
		Tổng số ghế là $S_{30}=30u_1+\dfrac{30.29}{2}d=2055$.
	}
\end{ex}
\begin{ex}%[1K2K6-6]
	Người ta trồng $3003$ cây theo một hình tam giác như sau: hàng thứ nhất trồng $1$ cây, hàng thứ hai trồng $2$ cây, hàng thứ ba trồng $3$ cây,... .Hỏi có tất cả bao nhiêu hàng cây?
	\choice
	{$73$}
	{$75$}
	{\True $77$}
	{$79$}
	\loigiai{
		Số cây mỗi hàng (bắt đầu từ hàng thứ nhất) lập thành một cấp số cộng $(u_n)$ có $u_1=1,d=1$. Giả sử có $n$ hàng cây thì $u_1+u_2+...+u_n=S_n$.\\
		Ta có $S_n=1.n+\dfrac{n(n-1)}{2}.1=3003\Leftrightarrow n=77$.
	}
\end{ex}
\begin{ex}%[1K2G6-6]
	Một chiếc đồng hồ đánh chuông, kể từ thời điểm $0$ (giờ) thì sau mỗi giờ thì số tiếng chuông được đánh đúng bằng số giờ mà đồng hồ chỉ tại thời điểm đánh chuông. Hỏi một ngày đồng hồ đó đánh bao nhiêu tiếng chuông?
	\choice
	{$78$}
	{$156$}
	{\True $300$}
	{$48$}
	\loigiai{
		Kể từ lúc $1$ (giờ) đến $24$ (giờ) số tiếng chuông được đánh lập thành cấp số cộng có $24$ số hạng với $u_1=1$, công sai $d=1$. Vậy số tiếng chuông được đánh trong $1$ ngày là $S_{24}=1.24+\dfrac{24.23}{2}.1=300$.
	}
\end{ex}
\begin{ex}%[1K2G6-6]
	Trên một bàn cờ có nhiều ô vuông, người ta đặt $7$ hạt dẻ vào ô đầu tiên, sau đó đặt tiếp vào ô thứ
	hai số hạt nhiều hơn ô thứ nhất là $5$, tiếp tục đặt vào ô thứ ba số hạt nhiều hơn ô thứ hai là $5$,...
	và cứ thế tiếp tục đến ô thứ $n$. Biết rằng đặt hết số ô trên bàn cờ người ta phải sử dụng $25450$
	hạt. Hỏi bàn cờ đó có bao nhiêu ô vuông?
	\choice
	{$98$}
	{\True $100$}
	{$102$}
	{$104$}
	\loigiai{
		Số hạt dẻ trên mỗi ô (bắt đầu từ ô thứ nhất) theo thứ tự đó lập thành cấp số cộng $(u_n)$ có $u_1=7,d=5$. Gọi $n$ là số ô trên bàn cờ thì $u_1+u_2+...+u_n=S_n$.\\
		Ta có $S_n=25450\Leftrightarrow 7n+\dfrac{n(n-1)}{2}.7=25450\Leftrightarrow n=100$.
	}
\end{ex}
\begin{ex}%[1K2G6-6]
	Một gia đình cần khoan một cái giếng để lấy nước. Họ thuê một đội khoan giếng nước đến để khoan giếng nước. Biết giá của mét khoan đầu tiên là $80.000$ đồng, kể từ mét khoan thứ $2$ giá của mỗi mét khoan tăng thêm $5000$ đồng so với giá của mét khoan trước đó. Biết cần phải khoan sâu xuống $50$ mét mới có nước. Vậy hỏi phải trả bao nhiêu tiền để khoan cái giếng đó?
	\choice
	{$5.250.000$ đồng}
	{\True $10.125.000$ đồng}
	{$4.00.000$ đồng}
	{$4.245.000$ đồng}
	\loigiai{
		Giá tiền khoang mỗi mét (bắt đầu từ mét đầu tiên) lập thành cấp số cộng $(u_n)$ có $u_1=80000,d=5000$. Do cần khoang $50$ mét nên tổng số tiền cần trả là $S_{50}=80000.50+\dfrac{50.49}{2}.5000=10125000$.
	}
\end{ex}
\begin{ex}%[1K2Y7-3]
	Một cấp số nhân có hai số hạng liên tiếp là $16$ và $36$. Số hạng tiếp theo là
	\choice
	{$720$}
	{\True$81$}
	{$64$}
	{$56$}
	\loigiai{
		Ta có cấp số nhân $(u_n)$ có $\heva{&u_n=36\\&u_{n+1}=36}\Rightarrow q=\dfrac{u_{n+1}}{u_n}=\dfrac{9}{4}$. Từ đây suy ra $u_{n+2}=u_{n+1}.q=36.\dfrac{9}{4}=81$.
	}
\end{ex}
\begin{ex}%[1K2B7-3]
	Tìm x để các số $2;8;x;128$ theo thứ tự đó lập thành một cấp số nhân.
	\choice
	{$x=14$}
	{\True 
		$x=32$}
	{$x=64$}
	{$x=68$}
	\loigiai{
		Cấp số nhân$ 2;8;x;128$ theo thứ tự đó sẽ là $u_1,u_2,u_3,u_4$.\\
		Ta có $\heva{&\dfrac{u_2}{u_1}=\dfrac{u_3}{u_2}\\&\dfrac{u_3}{u_2}=\dfrac{u_4}{u_3}}\Leftrightarrow\heva{&\dfrac{8}{2}=\dfrac{x}{8}\\&\dfrac{128}{x}=\dfrac{x}{8}}\Leftrightarrow\heva{&x=32\\&x^2=1024}\Rightarrow x=32$.
	}
\end{ex}
\begin{ex}%[1K2K7-3]
	Tìm tất cả giá trị của $x$ để ba số $2x-11;x;2x+1$ theo thứ tự đó lập thành một cấp số nhân.
	\choice
	{\True $x=\pm\dfrac{1}{\sqrt{3}}$}
	{$x=\pm\dfrac{1}{3}$}
	{$x=\pm\sqrt{3}$}
	{$x=\pm3$}
	\loigiai{
		Cấp số nhân $2x-1;x;2x+1$, suy ra $(2x-1)(2x+1)=x^2\Leftrightarrow x=\pm \dfrac{1}{\sqrt{3}}$.
	}
\end{ex}
\begin{ex}%[1K2K7-3]
	Với giá trị $x,y$ nào dưới đây thì các số hạng lần lượt là $-2;x;-18;y$ theo thứ tự đó lập thành cấp số nhân?
	\choice
	{$\heva{&x=6\\&y=-54}$}
	{$\heva{&x=-10\\&y=-26}$}
	{\True $\heva{&x=-6\\&y=-54}$}
	{$\heva{&x=-6\\&y=54}$}
	\loigiai{
		Cấp số nhân $-2;x;-18;y$, suy ra $\heva{&\dfrac{x}{-2}=\dfrac{-18}{x}\\&\dfrac{-18}{x}=\dfrac{y}{-18}}\Leftrightarrow\heva{&x=\pm6\\& y=\pm 54}$. Vậy $(x,y)=(6;54)$ hoặc $(x;y)=(-6;-54)$.
	}
\end{ex}
\begin{ex}%[1K2K7-3]
	Hai số hạng đầu của của một cấp số nhân là $2x+1$ và $4x^2-1$. Số hạng thứ ba của cấp số nhân là.
	\choice
	{$2x-1$}
	{$2x+1$}
	{\True $8x^3-4x^2-2x+1$}
	{$8x^3+4x^2-2x-1$}
	\loigiai{
		Công bội của cấp số nhân là $q=\dfrac{4x^2-1}{2x+1}=2x-1$. Vậy số hạng thứ ba của cấp số nhân là $(4x^2-1)(2x-1)=8x^3-4x^2-2x+1$.
	}
\end{ex}
\begin{ex}%[1K2B7-1]
	Trong các dãy số $(u_n)$ cho bởi số hạng tổng quát nu sau, dãy số nào là một cấp số nhân
	\choice
	{\True 
		$u_n=\dfrac{1}{3^{n-2}}$}
	{$u_n=\dfrac{1}{3^{n}}-1$}
	{$u_n=n+\dfrac{1}{3}$}
	{$u_n=n^2-\dfrac{1}{3}$}
	\loigiai{
		Dãy $u_n=\dfrac{1}{3^{n-2}}=3\left(\dfrac{1}{3}\right)^{n-1}$ là cấp số nhân có $u_1=3,q=\dfrac{1}{3}$.
	}
\end{ex}
\begin{ex}%[1K2B7-1]
	Một cấp số nhân có $6$ số hạng, số hạng đầu bằng $2$ và số hạng thứ sáu bằng $486$. Tìm công bội $q$ của cấp số nhân đã cho.
	\choice
	{\True $q=3$}
	{$q=-3$}
	{$q=2$}
	{$q=-2$}
	\loigiai{
		Ta có $\heva{&u_1=2\\&u_6=486}\Rightarrow u_6=u_1q^5\Leftrightarrow 486=2.q^5\Leftrightarrow q=3$.
	}
\end{ex}
\begin{ex}%[1K2B7-1]
	Cho cấp số nhân $\left(u_n\right)$ có $u_1=-3$ và $q=\dfrac{2}{3}$ Mệnh đề nào sau đây đúng.
	\choice
	{$u_5=-\dfrac{27}{16}$}
	{\True $u_5=-\dfrac{16}{27}$}
	{$u_5=\dfrac{16}{27}$}
	{$u_5=\dfrac{27}{16}$}
	\loigiai{
		Ta có $\heva{&u_1=-3\\&q=\dfrac{2}{3}}\Rightarrow u_5=u_1.q^4=-3.\left(\dfrac{2}{3}\right)^4=-\dfrac{16}{27}$.
	}
\end{ex}
\begin{ex}%[1K2K7-3]
	Cho cấp số nhân $\left(u_n\right)$ có $u_1=3$ và $q=-2$. Số $192$ là số hạng thứ mấy của cấp số nhân đã cho.
	\choice
	{$5$}
	{$6$}
	{\True $7$}
	{Không là số hạng của cấp số đã cho}
	\loigiai{
		Ta có $u_n=u_1.q^{n-1}\Leftrightarrow 192=3.(-2)^{n-1}\Leftrightarrow n=7$.
	}
\end{ex}
\begin{ex}%[1K2K7-3]
	Một cấp số nhân có công bội bằng $3$ và số hạng đầu bằng $5$. Biết số hạng chính giữa là $32805$. Hỏi cấp số nhân đã cho có bao nhiêu số hạng?
	\choice
	{$18$}
	{\True $17$}
	{$16$}
	{$9$}
	\loigiai{
		Ta có $u_n=u_1.q^{n-1}\Leftrightarrow 32805=3.5^{n-1}\Leftrightarrow n=9$. Vậy $u_9$ là số hạng chính giữa của cấp số nhân, nên cấp số nhân đã cho có $17$ số hạng.
	}
\end{ex}
\begin{ex}%[1K2K7-5]
	Cho cấp số nhân $\left(u_n\right)$ có $u_1=-3$ và $q=-2$. Tính tổng $10$ số hạng đầu tiên của cấp số nhân đã cho.
	\choice
	{$S_{10}=-511$}
	{$S_{10}=-1025$}
	{$S_{10}=1025$}
	{\True $S_{10}=1023$}
	\loigiai{
		Ta có $\heva{&u_1=-3\\&q=-2}\Rightarrow S_{10}=u_1.\dfrac{q^{n}-1}{q-1}=(-3).\dfrac{(-2)^{10}-1}{-2-1}=1023$.
	}
\end{ex}
\begin{ex}%[1K2G7-5]
	Cho cấp số nhân có các số hạng lần lượt là $1;4;16;64;...$. Gọi $S_n$ là tổng của $n$ số hạng đầu tiên của cấp số nhân đó. Mệnh đề nào sau đây đúng?
	\choice
	{$S_n=4^{n-1}$}
	{$S_n=\dfrac{n\left(1+4^{n-1}\right)}{2}$}
	{\True $S_n=\dfrac{4^n-1}{3}$}
	{$S_n=\dfrac{4\left(4^n-1\right)}{3}$}
	\loigiai{
		Ta có $\heva{&u_1=-3\\&q=4}\Rightarrow S_n=u_1.\dfrac{q^{n}-1}{q-1}=\dfrac{4^n-1}{3}$.
	}
\end{ex}
\begin{ex}%[1K2G7-1]
	Số hạng thứ hai, số hạng đầu và số hạng thứ ba của một cấp số cộng với công sai khác $0$ theo thứ tự đó lập thành một cấp số nhân với công bội $q$. Tìm $q$.
	\choice
	{$q=2$}
	{\True $q=-2$}
	{$q=-\dfrac{3}{2}$}
	{$q=\dfrac{3}{2}$}
	\loigiai{
		Giả sử ba số hạng $a;b;c$ lập thành cấp số cộng thỏa yêu cầu, khi đó $b;a;c$ theo thứ tự đó lập thành cấp số nhân công bội $q$. Ta có $\heva{&a+c=2b\\&a=bq\\&c=bq^2}\Rightarrow bq+bq^2=2b\Leftrightarrow\heva{&b=0\\&q^2+q-2=0}$.\\
		Nếu $b=0\Rightarrow a=b=c=0$ nên $a;b;c$ là cấp số cộng công sai $d=0$ (vô lí).\\
		Nếu $q^2+q-2=0\Leftrightarrow\hoac{&q=1\\&q=-2}$. Nếu $q=1\Rightarrow a=b=c$ (vô lí), do đó $q=-2$.
	}
\end{ex}
\begin{ex}%[1K2G7-1]
	Cho bố số $a,b,c,d$ biết rằng $a,b,c$ theo thứ tự đó lập thành một cấp số nhân công bội $q>1$,
	còn $b,c,d$ theo thứ tự đó lập thành cấp số cộng. Tìm $q$ biết rằng $a+d=14$ và $b+c=12$.
	\choice
	{$q=\dfrac{18+\sqrt{73}}{24}$}
	{\True $q=\dfrac{19+\sqrt{73}}{24}$}
	{$q=\dfrac{20+\sqrt{73}}{24}$}
	{$q=\dfrac{21+\sqrt{73}}{24}$}
	\loigiai{
		Giả sử $a,b,c$ lập thành cấp số cộng công bội $q$. Khi đó theo giả thiết ta có\\
		$\heva{&b=aq\\&c=aq^2\\&b+d=2c\\&a+d=14\\&c+d=12}\Rightarrow\heva{&aq+d=aq^2,&(1)\\&a+d=14,&(2)\\&a\left(q+q^2\right)=12,&(3)}$.\\
		Nếu $q=0\Rightarrow b=c=d=0$ (vô lý).\\
		Nếu $q=-1\Rightarrow b=-a=-c\Rightarrow b+c=0$ (vô lý).\\
		Vậy $q\ne 0,q\ne -1$, từ $(2)$ và $(3)$, ta có $d=14-a$ và $a=\dfrac{12}{q+q^2}$, thay vào $(1)$, ta được\\
		$\dfrac{12q}{q+q^2}+\dfrac{14q^2+14q-12}{q+q^2}=\dfrac{24q^3}{q+q^2}\Leftrightarrow 12q^3-7q^2-13q+6=0\Leftrightarrow q=\dfrac{19\pm \sqrt{73}}{24}$.\\
		Mà $q>1$ nên $q=\dfrac{19+\sqrt{73}}{24}$.
	}
\end{ex}
\begin{ex}%[1K2G7-5]
	Gọi $S=1+11+111+\cdots+111\ldots1$ ($n$ số $1$) thì $S$ nhận giá trị nào sau đây?
	\choice
	{$S=\dfrac{10^n-1}{81}$}
	{$S=10\cdot\dfrac{10^n-1}{81}$}
	{$S=10\cdot\dfrac{10^n-1}{81}-1$}
	{\True $S=\dfrac{1}{9}\left[10\cdot\dfrac{10^n-1}{9}-1\right]$}
	\loigiai{
		Ta có $S=\dfrac{1}{9}\left(9+99+999+\cdots+999\ldots9\right)=\dfrac{1}{9}\left(10+100+1000+\cdots+100\ldots0-n\right)=\dfrac{1}{9}\left[10\cdot\dfrac{10^n-1}{9}-1\right]$.
	}
\end{ex}
\begin{ex}%[1K2G7-7]
	Biết rằng $S=1+2\cdot3+3\cdot3^2+\cdots+11.3^{10}=a+\dfrac{21\cdot3^b}{4}$. Tính $P=a+\dfrac{b}{4}$.
	\choice
	{$P=1$}
	{$P=2$}
	{\True $P=3$}
	{$P=4$}
	\loigiai{
		Từ giả thiết suy ra $3S=3+2\cdot3^2+3\cdot3^3+\cdots+11\cdot3^{11}$.\\
		Do đó $-2S=S-3S=1+3+3^2+3^3+\cdots+3^{10}-10.3^{11}=\dfrac{1-3^{11}}{1-3}-11\cdot3^{11}\Rightarrow S=\dfrac{1}{4}+\dfrac{21}{4}\cdot3^{11}$.\\
		Vậy $a=\dfrac{1}{4},b=11$, suy ra $P=3$.
	}
\end{ex}
\begin{ex}%[1K2K7-1]
	Một cấp số nhân có ba số hạng là $a,b,c$ (theo thứ tự đó) trong đó các số hạng đều khác $0$ và công bội $q\ne 0$. Mệnh đề nào sau đây là đúng.
	\choice
	{$\dfrac{1}{a^2}=\dfrac{1}{bc}$}
	{\True $\dfrac{1}{b^2}=\dfrac{1}{ac}$}
	{$\dfrac{1}{c^2}=\dfrac{1}{ba}$}
	{$\dfrac{1}{a}+\dfrac{1}{b}=\dfrac{2}{c}$}
	\loigiai{
		Ta có $ac=b^2\Rightarrow \dfrac{1}{b^2}=\dfrac{1}{ac}$
	}
\end{ex}
\begin{ex}%[1K2K7-3]
	Bốn góc của một tứ giác tạo thành cấp số nhân và góc lớn nhất gấp $27$ lần góc nhỏ nhất. Tổng của góc lớn nhất và góc bé nhất bằng.
	\choice
	{$56^\circ$}
	{$102^\circ$}
	{\True $252^\circ$}
	{$168^\circ$}
	\loigiai{
		Giả sử $4$ góc $A, B, C, D$ (với $A<B<C<D$) theo thứ tự đó lập thành cấp số nhân thỏa yêu cầu với công bội $q$.\\
		Ta có $\heva{&A+B+C+D=360\\&D=27A}\Leftrightarrow\heva{&A\left(1+q+q^2+q^3\right)=360\\&Aq^3=27A}\Leftrightarrow\heva{&q=3\\&A=9\\&D=243}\Rightarrow A+D=252$.
	}
\end{ex}
\begin{ex}%[1K2G7-7]
	Người ta thiết kế một cái tháp gồm $11$ tầng. Diện tích bề mặt trên của mỗi tầng bằng nữa diện tích của mặt trên của tầng ngay bên dưới và diện tích mặt trên của tầng $1$ bằng nửa diện tích của đế tháp (có diện tích là $12288m^2$). Tính diện tích mặt trên cùng.
	\choice
	{\True $6m^2$}
	{$8m^2$}
	{$10m^2$}
	{$12m^2$}
	\loigiai{
		Diện tích bề mặt của mỗi tầng (kể từ $1$) lập thành một cấp số nhân có công bội $q=\dfrac{1}{2}$ và
		$u_1=\dfrac{12288}{2}=6144$. Khi đó diện tích mặt trên cùng là $u_{11}=u_1\cdot q^{10}=6144\cdot\left(\dfrac{1}{2}\right)^{10}=6$.
	}
\end{ex}
\begin{ex}%[1K2G7-7]
	Một du khách vào chuồng đua ngựa đặt cược, lần đầu đặt $20000$ đồng, mỗi lần sau tiền đặt gấp
	đôi lần tiền đặt cọc trước. Người đó thua $9$ lần liên tiếp và thắng ở lần thứ $10$. Hỏi du khác trên thắng hay thua bao nhiêu?
	\choice
	{Hoà vốn}
	{Thua $20000$ đồng}
	{\True Thắng $20000$ đồng}
	{Thua $40000$ đồng}
	\loigiai{
		Số tiền du khác đặt trong mỗi lần (kể từ lần đầu) là một cấp số nhân có $u_1=20000$ và công bội $q=2$. Du khách thua trong $9$ lần đầu tiên nên tổng số tiền thua là $S_9=u_1.\dfrac{q^9-1}{q-1}=20000\cdot\dfrac{2^9-1}{2-1}=10220000$.\\
		Số tiền mà du khách thắng trong lần thứ $10$ là $u_{10}=u_1\cdot q^9=20000\cdot2^9=10240000$.\\
		Ta có $u_{10}-S_9=20000>0$ nên du khách thắng $20000$.
	}
\end{ex}
\Closesolutionfile{ans}
% \begin{indapan}{10}
% 	{ans/ans-1K2-Ontapchuong2}
% \end{indapan}

%Chương III
% \setcounter{chapter}{2}
\setcounter{subsubsection}{0}
\setcounter{ex}{0}
\setcounter{bt}{0}
\chap{Một số yếu tố thống kê và xác suất}
% \section{Các số đặc trưng đo xu thế trung tâm cho mẫu số liệu ghép nhóm}

\subsection{Tóm tắt lý thuyết}
\begin{tomtat}
	\subsubsection{Mẫu số liệu ghép nhóm}
		\begin{enumerate}
			\item \textbf{\textit{Mẫu số liệu ghép nhóm}} là mẫu số liệu cho dưới dạng bảng tần số ghép nhóm.
			\item Mỗi số liệu gồm một số giá trị của mẫu số liệu được ghép nhóm theo một tiêu chí xác định có dạng $\left[a;b\right)$, trong đó $a$ là \textit{đầu mút trái}, $b$ là \textit{đầu mút phải}. Độ dài nhóm là $b-a$.
			\item \textbf{\textit{Tần số tích luỹ}} của một nhóm là số số liệu trong mẫu số liệu có giá trị nhỏ hơn giá trị đầu mút phải của nhóm đó. Tần số tích luỹ của nhóm $1$, nhóm $2$, $\ldots$, nhóm $m$ kí hiệu lần lượt là $cf_1$, $cf_2$, $\ldots$, $cf_m$.
		\end{enumerate}
	\subsubsection{Số trung bình cộng (số trung bình)}
		\begin{enumerate}
			\item Trung điểm $x_i$ của nửa khoảng (tính bằng trung bình cộng của hai đầu mút) ứng với nhóm $i$ là \textit{giá trị đại diện} của nhóm đó.
			\item \textit{Số trung bình cộng} của mẫu số liệu ghép nhóm, kí hiệu $\overline{x}$, được tính theo công thức 
				$$\overline{x} = \dfrac{n_1x_1 + n_2x_2 + \ldots + n_mx_m}{n}.$$
		\end{enumerate}
	\subsubsection{Trung vị}
	Để tính trung vị của mẫu số liệu ghép nhóm, ta làm như sau:
\begin{itemize}
    \item \textbf{Bước 1:} Xác định nhóm chứa trung vị. Giả sử đó là nhóm thứ $p : [a_p; a_{p + 1})$.
    \item \textbf{Bước 2:} Trung vị là 
    $$M_e = a_p + \dfrac{\dfrac{n}{2} - (m_1 + \cdots + m_{p-1})}{m_p} \cdot ( a_{p + 1} - a_p)$$
    trong đó $n$ là cỡ mẫu, $m_p$ là tần số nhóm $p$. Với $p=1$ ta quy ước $m_1 + \cdots + m_{p-1} = 0$.
\end{itemize}
		\begin{note}
			Nhóm chứa trung vị là nhóm đầu tiên có tần số tích luỹ $cf_p=m_1 + \cdots + m_{p}$ lớn hơn hoặc bằng $\dfrac{n}{2}$
		\end{note}
	\subsubsection{Tứ phân vị}Để tính tứ phân vị thứ nhất $Q_1$ của mẫu số liệu ghép nhóm, trước hết ta xác định nhóm chứa $Q_1$, giả sử đó là nhóm thứ  $p:\left[a_p;a_{p+1} \right)$. Khi đó 
	$$Q_1=a_p+\dfrac{\dfrac{n}{4}-\left(m_1+\cdots+m_{p-1}\right)}{m_p}\cdot \left(a_{p+1}-a_p\right).$$
	trong đó, $n$ là cỡ mẫu, $m_p$ là tần số nhóm $p$. Với $p=1$, ta quy ước $m_1+\cdots+m_{p-1}=0$.\\
	Để tính tứ phân vị thứ ba $Q_3$ của mẫu số liệu ghép nhóm, trước hết ta xác định nhóm chứa $Q_3$, giả sử đó là nhóm thứ  $p:\left[a_p;a_{p+1} \right)$. Khi đó 
	$$Q_3=a_p+\dfrac{\dfrac{3n}{4}-\left(m_1+\cdots+m_{p-1}\right)}{m_p}\cdot \left(a_{p+1}-a_p\right).$$
	Trong đó $n$ là cỡ mẫu, $m_p$ là tần số nhóm $p$. Với $p=1$, ta quy ước $m_1+\cdots+m_{p-1}=0$.\\
	Tứ phân vị thứ hai $Q_2$ chính là trung vị $M_e$.\\
\subsubsection{Mốt của mẫu số liệu ghép nhóm}
Để tìm mốt của mẫu số liệu ghép nhóm, ta thực hiện theo các bước sau:
\begin{enumerate}
	\item [Bước 1.] Xác định nhóm có tần số lớn nhất (gọi là nhóm chứa mốt), giả sử là nhóm $j:\left[a_j;a_{j+1} \right)$.
	\item [Bước 2.] Mốt được xác định là $M_o=a_j+\dfrac{m_i-m_{j-1}}{\left(m_i-m_{j-1}\right)+\left(m_i-m_{j+1}\right)}\cdot h$.\\
\end{enumerate}
trong đó, $m_j$ là tần số nhóm $j$ (quy ước $m_0=m_{k+1}=0$) và $h$ là độ dài của nhóm.	

\end{tomtat}
%=================================================
\setcounter{subsubsection}{0}
\setcounter{ex}{0}
\setcounter{bt}{0}
\subsection{Các dạng toán thường gặp}
\begin{dang}{Mẫu số liệu ghép nhóm}
\end{dang}
\subsubsection{Ví dụ minh hoạ}
\begin{vd}%[Cánh Diều]%[1C5Y1-1]
	\immini{
		\textbf{Bảng bên} biểu diễn mẫu số liệu ghép nhóm được cho dưới dạng bảng tần số ghép nhóm. Hãy cho biết 
		\begin{enumerate}
			\item Mẫu số liệu có bao nhiêu số liệu; bao nhiêu nhóm?
			\item Tần số của mỗi nhóm.
		\end{enumerate}
	}{
		\begin{tabular}{|c|c|}
			\hline
			\textbf{Nhóm} & \textbf{Tần số}\\ 
			\hline
			$\left[0;5\right)$ & $11$\\
			\hline
			$\left[5;10\right)$ & $31$\\
			\hline
			$\left[10;15\right)$ & $45$\\
			\hline
			$\left[15;20\right)$ & $21$\\
			\hline
			$\left[20;26\right)$ & $12$\\
			\hline
			& $n = 120$ \\
			\hline
		\end{tabular}
	}
	\loigiai{
		\begin{enumerate}
			\item Mẫu số liệu gồm $120$ số liệu và $5$ nhóm.
			\item Tần số lần lượt của các nhóm $1$, $2$, $3$, $4$, $5$ lần lượt là $11$, $31$, $45$, $21$, $12$.
		\end{enumerate}	
	}
\end{vd}
\begin{vd}%[CTST]%[1T5B1-1]
	Một cửa hàng đã thống kê số ba lô bán được mỗi ngày trong tháng 9 với kết quả cho như sau: \begin{center}
		\begin{tabular}{lllllllllllllll}
			$12$ & $29$ & $12$ & $19$ & $15$ & $21$ & $19$ & $29$ & $28$ & $12$ & $15$ & $25$ & $16$ & $20$ & $29$\\
			$21$ & $12$ & $24$ & $14$ & $10$ & $12$ & $10$ & $23$ & $27$ & $28$ & $18$ & $16$ & $10$ & $20$ & $21$
		\end{tabular}
	\end{center}
	Hãy chia mẫu số liệu trên thành 5 nhóm, lập bảng tần số ghép nhóm, hiệu chỉnh bảng tần số ghép nhóm và xác định giá trị đại diện cho mỗi nhóm.
	\loigiai{
		Khoảng biến thiên của mẫu số liệu trên là $R=29-10=19$.\\
		Độ dài mỗi nhóm $L>\dfrac{R}{k}=\dfrac{19}{5}=3{,}8$.\\
		Ta chọn $L=4$ và chia dữ liệu thành các nhóm $[10; 14)$, $[14; 18)$, $[18; 22)$, $[22; 26)$, $[26; 30)$.\\
		Khi đó ta có bảng tần số ghép nhóm sau
		\begin{center}
			\begin{tabular}{|c|c|c|c|c|c|}
				\hline \textbf{Cân nặng} &{$[10; 14)$} &{$[14; 18)$} &{$[18; 22)$} &{$[22; 26)$} &{$[26; 30)$} \\
				\hline \textbf{Giá trị đại diện} & $12$ & $16$ & $20$ & $24$ & $28$ \\
				\hline \textbf{Số ba lô bán được} & $8$ & $5$ & $8$ & $3$ & $6$ \\
				\hline
			\end{tabular}
		\end{center}
	}
\end{vd}
\begin{vd}%[KNTT]%[Ngọc Hiếu]%[1K3B8-1]
	Bảng thống kê sau cho biết thời gian chạy (phút) của $30$ vận động viên (VĐV) trong một giải chạy Marathon.
	\begin{center}
		\begin{tabular}{|c|c|c|c|c|c|c|c|c|c|c|c|c|}
			\hline
			Thời gian&$129$&$130$&$133$&$134$&$135$&$136$&$138$&$141$&$142$&$143$&$144$&$145$\\
			\hline
			Số VĐV&$1$&$2$&$1$&$1$&$1$&$2$&$3$&$3$&$4$&$5$&$2$&$5$\\
			\hline
		\end{tabular}
	\end{center}
	Hãy chuyển mẫu số liệu trên sang mẫu số liệu ghép nhóm gồm sáu nhóm có độ dài bằng nhau và bằng $3$.
	\loigiai{
		Giá trị nhỏ nhất là $129$, giá trị lớn nhất là $145$ nên khoảng biến thiên là $145-129=16$. Tổng độ dài của sáu nhóm là $18$. Để cho đối xứng, ta chọn đầu mút trái của nhóm đầu tiên là $127{,}5$ và đầu mút phải của nhóm cuối cùng là $145{,}5$ ta được các nhóm là $[127{,}5;130{,}5),\; [130{,5};133{,5}],\ldots , [142{,}5;145{,}5]$. Đếm số giá trị thuộc mỗi nhóm, ta có mẫu số liệu ghép nhóm như sau
		\begin{center}
			\fontsize{9}{1pt}
			{\begin{tabular}{|c|c|c|c|c|c|c|}
					\hline
					Thời gian&$[125{,}5;130{,}5)$&$[130{,}5;133{,}5)$&$[133{,}5;136{,}5)$&$[136{,}5;139{,}5)$&$[139{,}5;142{,}5)$&$[142{,}5;145{,}5)$\\
					\hline
					Số VĐV&$3$&$1$&$4$&$3$&$7$&$12$\\
					\hline
			\end{tabular}}
		\end{center}
	}
\end{vd}
\begin{vd}%[Cánh Diều]%[1C5B1-1]
	Một trường trung học phổ thông chọn $36$ học sinh nam của khối $11$, do chiều cao của các bạn học sinh đó và thu được mẫu số liệu sau (đơn vị: centimét):
	$$
	\begin{array}{llllllllllll}
		160 & 161 & 161 & 162 & 162 & 162 & 163 & 163 & 163 & 164 & 164 & 164 \\
		164 & 165 & 165 & 165 & 165 & 165 & 166 & 166 & 166 & 166 & 167 & 167 \\
		168 & 168 & 168 & 168 & 169 & 169 & 170 & 171 & 171 & 172 & 172 & 174
	\end{array}
	$$
	Lập bảng tần số ghép nhóm bao gồm cả tần số tích luỹ cho mẫu số liệu trên có $5$ nhóm ứng với $5$ nửa khoảng:
	$$
	\left[160;163 \right),\ \left[163;169 \right),\ \left[166;169 \right),\ \left[169;172 \right),\ \left[172;175 \right).
	$$
	\loigiai{
		Bảng tần số ghép nhóm bao gồm cả tần số tích luỹ như sau:
		\begin{center}
			\begin{tabular}{|c|c|c|}
				\hline
				\textbf{Nhóm} & \textbf{Tần số} & \textbf{Tần số tích luỹ}\\ 
				\hline
				$\left[169;163\right)$ & $6$ & $6$\\
				\hline
				$\left[163;166\right)$ & $12$ & $18$\\
				\hline
				$\left[166;169\right)$ & $10$ & $28$\\
				\hline
				$\left[169;172\right)$ & $5$ & $33$\\
				\hline
				$\left[172;175\right)$ & $3$ & $36$\\
				\hline
				& $n = 36$ &\\
				\hline
			\end{tabular}
		\end{center}
	}
\end{vd}
\subsubsection{Bài tập rèn luyện}
\begin{bt}%[KNTT]%[1K3B8-1]
	Trong các mẫu số liệu sau, mẫu nào là mẫu số liệu ghép nhóm? Đọc và giải thích mẫu số liệu ghép nhóm đó.
	\begin{enumerate}
		\item Số tiền mà sinh viên chi cho thanh toán cước điện thoại trong tháng.
		\begin{center}
			\begin{tabular}{|c|c|c|c|c|c|}
				\hline
				Số tiền (nghìn đồng)&$[0;50)$&$[50;100)$&$[100;150)$&$[150;200)$&$[200;250)$\\
				\hline
				Số sinh viên&$5$&$12$&$23$&$17$&$3$\\
				\hline
			\end{tabular}
		\end{center}
		\item Thống kê nhiệt độ tại một điểm trong $40$ ngày, ta có bảng số liệu sau
		\begin{center}
			\begin{tabular}{|c|c|c|c|c|}
				\hline
				Nhiệt độ $(^\circ$ C)&$[19;22)$&$[22;25)$&$[25;28)$&$[28;31)$\\
				\hline
				Số ngày&$7$&$15$&$12$&$6$\\
				\hline
			\end{tabular}
		\end{center}
	\end{enumerate}
	\loigiai{
		Cả hai mẫu số liệu trên đều là mẫu số lớp ghép nhóm.
		\begin{enumerate}
			\item Có năm nhóm là
			\begin{itemize}
				\item Dưới $50$ nghìn đồng có $5$ sinh viên.
				\item Từ $50$ đến dưới $100$ nghìn đồng có $12$ sinh viên.
				\item Từ $100$ đến dưới $150$ nghìn đồng có $23$ sinh viên.
				\item Từ $150$ đến dưới $200$ nghìn đồng có $17$ sinh viên.
				\item Từ $200$ đến dưới $250$ nghìn đồng có $3$ sinh viên.
			\end{itemize}
			\item Có bốn nhóm là
			\begin{itemize}
				\item Từ $19^\circ$ C đến dưới $22^\circ$ C có $7$ ngày.
				\item Từ $22^\circ$ C đến dưới $25^\circ$ C có $15$ ngày.
				\item Từ $25^\circ$ C đến dưới $28^\circ$ C có $12$ ngày.
				\item Từ $128^\circ$ C đến dưới $31^\circ$ C có $6$ ngày.
			\end{itemize}
		\end{enumerate}
	}
\end{bt}	
\begin{bt}%[KNTT]%[1K3B8-1]
	Số sản phẩm một công nhân làm được trong một ngày được cho như sau:
	\begin{center}
		\begin{tabular}{c c c c c c c c c c c c c}
			$18$&$25$&$39$&$12$&$54$&$27$&$46$&$25$&$19$&$8$&$36$&$22$&\\
			$20$&$19$&$17$&$44$&$5$&$18$&$23$&$28$&$25$&$34$&$46$&$27$&$16$
		\end{tabular}
	\end{center}
	Hãy chuyển mẫu số liệu sang dạng ghép nhóm với sáu nhóm có độ dài bằng nhau.
	\loigiai{
		Khoảng biến thiên là $54-5=49$.\\
		Ta chia thành các nhóm sau $[4{,}5;13); [13;21{,}5);[21{,}5;30);\ldots ;[47;55{,}5)$.\\
		Đếm số giá trị của mỗi nhóm, ta có bảng ghép nhóm sau:
		\begin{center}
			\begin{tabular}{|c|c|c|c|c|c|c|}
				\hline
				Số sản phẩm &$[4{,}5;13)$&$[13;21{,}5)$&$[21{,}5;30)$&$[30;38{,}5)$&$[38{,}5;47)$&$[47;55{,}5)$\\
				\hline
				Số công nhân&$3$&$7$&$8$&$2$&$4$&$1$\\
				\hline
			\end{tabular}
		\end{center}
	}
\end{bt}
\begin{bt}%[KNTT]%[1K3B8-1]
	Thời gian ra sân (giờ) của một số cựu cầu thủ ở giải ngoại hạng Anh qua các thời kì được cho như sau:
	\begin{center}
		\begin{tabular}{c c c c c c c c}
			$653$ & $632$ & $609$ & $572$ & $565$ & $535$ & $516$ & $514$ \\
			$508$ & $505$ & $504$ & $504$ & $503$ & $499$ & $496$ & $492$ 
		\end{tabular}
	\end{center}
	Hãy chuyển mẫu số liệu trên sang dạng ghép nhóm với bảy nhóm có độ dài bằng nhau.
	\loigiai{
		Khoảng biến thiên là $653-492=161$.\\
		Ta chia thành các nhóm sau $[492;515); [515;538);[538;561);\ldots; [630;653]$.\\
		Đếm số giá trị của mỗi nhóm, ta có bảng ghép nhóm sau:
		\begin{center}
			\begin{tabular}{|c|c|c|c|c|c|c|c|}
				\hline
				Thời gian &$[492;515)$&$[515;538)$&$[538;561)$&$[561;584)$&$[584;607)$&$[607;630)$&$[630;653]$\\
				\hline
				Số cầu thủ &$9$&$2$&$0$&$2$&$0$&$1$&$2$\\
				\hline
			\end{tabular}
		\end{center}
	}
\end{bt}

%===================================
\setcounter{subsubsection}{0}
\setcounter{ex}{0}
\setcounter{bt}{0}
\begin{dang}{Số trung bình cộng (số trung bình)}
\end{dang}
\subsubsection{Ví dụ minh hoạ}
\begin{vd}%[Cánh Diều]%[1C5Y1-2]
	\immini
	{
		Một nhà thực vật học đo chiều dài của $74$ lá cây (đơn vị: milimét) và thu được bảng tần số như bảng bên. Tính chiều dài trung bình của $74$ lá cây trên theo đơn vị milimét (làm tròn kết quả đến hàng phần trăm).
	}
	{
		\begin{tabular}{|c|c|c|}
			\hline
			\textbf{Nhóm} & \textbf{Giá trị đại diện} & \textbf{Tần số}\\ 
			\hline
			$\left[5{,}45;5{,}85\right)$ & $5{,}65$ & $5$\\
			$\left[5{,}85;6{,}25\right)$ & $6{,}05$ & $9$\\
			$\left[6{,}25;6{,}65\right)$ & $6{,}45$ & $15$\\
			$\left[6{,}65;7{,}05\right)$ & $6{,}85$ & $19$\\
			$\left[7{,}05;7{,}45\right)$ & $7{,}25$ & $16$\\
			$\left[7{,}45;7{,}85\right)$ & $7{,}65$ & $8$\\
			$\left[7{,}85;8{,}25\right)$ & $8{,}05$ & $2$\\
			\hline
			&  & $n = 74$\\
			\hline
		\end{tabular}
	}
	\loigiai{
		Chiều dài trung bình của $74$ lá cây mà nhà thực vật học đo xấp xỉ là 
		\[
		\overline{x} = \dfrac{5\cdot 5{,}65 + 9 \cdot 6{,}05 + 15\cdot 6{,}45 + 19\cdot 6{,}85 + 16 \cdot 7{,}25 + 8\cdot 7{,}65 + 2\cdot 8{,}05}{74} \approx 6{,}80\ (\text{mm}).
		\]
	}
\end{vd}
\begin{vd}%[CTST]%[1T5B1-2]
	Kết quả khảo sát cân nặng của $25$ quả cam ở mỗi lô hàng $A$ và $B$ được cho ở bảng sau:
	\begin{center}
		\begin{tabular}{|c|c|c|c|c|c|}
			\hline \multicolumn{1}{|c|}{Cân nặng $(\mathrm{g})$} &{$[150; 155)$} &{$[155; 160)$} &{$[160; 165)$} &{$[165; 170)$} &{$[170; 175)$} \\
			\hline Số quả cam ở lô hàng $A$ & 2 & 6 & 12 & 4 & 1 \\
			\hline Số quả cam ở lô hàng $B$ & 1 & 3 & 7 & 10 & 4 \\
			\hline
		\end{tabular}
	\end{center}
	\begin{enumerate}
		\item Hãy ước lượng cân nặng trung bình của mỗi quả cam ở lô hàng $A$ và lô hàng $B$.
		\item Nếu so sánh theo số trung bình thì cam ở lô hàng nào nặng hơn?
	\end{enumerate}
	\loigiai{
		Ta có bảng thống kê số lượng cam theo giá trị đại diện:
		\begin{center}
			\begin{tabular}{|c|c|c|c|c|c|}
				\hline \multicolumn{1}{|c|}{Cân nặng $(\mathrm{g})$} &{$152{,}5$} &{$157{,}5$} &{$162{,}5$} &{$167{,}5$} &$172{,}5$\\
				\hline Số quả cam ở lô hàng $A$ & 2 & 6 & 12 & 4 & 1 \\
				\hline Số quả cam ở lô hàng $B$ & 1 & 3 & 7 & 10 & 4 \\
				\hline
			\end{tabular}
		\end{center}
		\begin{enumerate}
			\item Cân nặng trung bình của mỗi quả cam ở lô hàng $A$ xấp xỉ bằng
			\[(2\cdot 152{,}5+6\cdot 157{,}5+12\cdot 162{,}5+4\cdot 167{,}5+1\cdot 172{,}5): 25=161{,}7\ (\mathrm{g}). \]
			Cân nặng trung bình của mỗi quả cam ở lô hàng $B$ xấp xỉ bằng
			\[(1\cdot 152{,}5+3\cdot 157{,}5+7\cdot 162{,}5+10\cdot 167{,}5+4\cdot 172{,}5): 25=165{,}1\ (\mathrm{g}). \]
			\item Nếu so sánh theo số trung bình thì cam ở lô hàng $B$ nặng hơn cam ở lô hàng $A$.
		\end{enumerate}
	}
\end{vd}
\begin{vd}%[KNTT]%[1K3B9-1]
	Tìm cân nặng trung bình của học sinh lớp $11D$ cho trong Bảng $3.5$.
	\begin{center}
		\begin{tabular}{|c|c|c|c|c|c|c|}
			\hline
			Cân nặng	& $\left[40{,}5;45{,}5 \right)$ & $\left[45{,}5;50{,}5 \right)$ & $\left[50{,}5;55{,}5 \right)$ & $\left[55{,}5;60{,}5 \right)$ & $\left[60{,}5;65{,}5 \right)$ & $\left[65{,}5;70{,}5 \right)$ \\
			\hline
			Số học sinh&$10$	& $7$ & $16$ &$4$  & $2$ & $3$ \\
			\hline
		\end{tabular}

		Bảng $3.5$. Cân nặng của học sinh lớp $11D$.	
	\end{center}	
	\loigiai{
		Trong mỗi khoảng cân nặng, giá trị đại diện là trung bình cộng của hai giá trị đầu mút nên ta có bảng sau
		\begin{center}
			\begin{tabular}{|c|c|c|c|c|c|c|}
				\hline
				Cân nặng (kg)	& $43$ & $48$ & $53$ & $58$ & $63$ & $68$ \\
				\hline
				Số học sinh &$10$ & $7$ & $16$ &$4$  & $2$ & $3$ \\
				\hline
			\end{tabular}
		\end{center}	
		Tổng số học sinh là $n=42$. Cân nặng trung bình của học sinh lớp $11D$ là $$\overline{x}=\dfrac{10\cdot 43+7\cdot 48+16\cdot 53+4\cdot 58+2\cdot 63+3\cdot 68}{42}\approx51{,}81\,\mathrm{(kg)}.$$
	}
\end{vd}
\subsubsection{Bài tập rèn luyện}
\begin{bt}%[Cánh diều]%[1C5B1-5]
	Mẫu số liệu dưới đây ghi lại tốc độ của $40$ ô tô khi đi qua một trạm đo tốc độ (đơn vị: km/h)
	\[
	\begin{array}{cccccccccc}
		48{,}5 & 43 & 50 & 55 & 45 & 60 & 53 & 55,5 & 44 & 65 \\
		51 & 62,5 & 41 & 44,5 & 57 & 57 & 68 & 49 & 46{,}5 & 53{,}5 \\
		61 & 49{,}5 & 54 & 62 & 59 & 56 & 47 & 50 & 60 & 61 \\
		49{,}5 & 52{,}5 & 57 & 47 & 60 & 55 & 45 & 47,5 & 48 & 61{,}5
	\end{array}
	\]
	\begin{enumerate}
		\item Lập bảng tần số ghép nhóm cho mẫu số liệu trên có sáu nhóm ứng với sáu nửa khoảng:
		\[
		[40 ; 45),[45 ; 50),[50 ; 55),[55 ; 60),[60 ; 65),[65 ; 70).
		\]
		\item Xác định số trung bình cộng của mẫu số liệu ghép nhóm trên.
	\end{enumerate}
	\loigiai{
		\begin{enumerate}
			\item Ta có bảng tần số ghép nhóm của mẫu số liệu trên như sau:
			\begin{center}
				\begin{tabular}{|c|c|c|c|}
					\hline
					\textbf{Nhóm} & \textbf{Giá trị đại diện} & \textbf{Tần số} & \textbf{Tần số tích luỹ}\\ 
					\hline
					$\left[40;45\right)$ & $42{,}5$ & $4$ & $4$\\
					$\left[45;50\right)$ & $47{,}5$ & $11$ & $15$\\
					$\left[50;55\right)$ & $52{,}5$ & $7$ & $22$\\
					$\left[55;60\right)$ & $57{,}5$ & $8$ & $30$\\
					$\left[60;65\right)$ & $62{,}5$ & $8$ & $38$\\
					$\left[65;70\right)$ & $67{,}5$ & $2$ & $40$\\
					\hline
					&  & $n = 40$ &\\
					\hline
				\end{tabular}
			\end{center}
			\item Trung bình cộng của mẫu số liệu trên là
			\[
			\overline{x} = \dfrac{42{,}5 \cdot 4 + 47{,}5 \cdot 11 + 52{,}5 \cdot 7+ 57{,}5 \cdot 8+ 62{,}5 \cdot 8 + 67{,}5 \cdot 2}{40} = 53{,}875\text{ (km/h)}.
			\]
			\item Ta thấy: Nhóm $2$ ứng với nửa khoảng $\left[45;50\right)$ là nhóm có tần số lớn nhất với $u=45$, $g=5$, $n_2 = 11$. Nhóm $1$ có tần số $n_1 = 4$, nhóm $3$ có tần số $n_3 = 7$.
		\end{enumerate}
	}
\end{bt}
\begin{bt}%[KNTT]%[1K3B9-4]
	Tuổi thọ (năm) của 50 bình ắc quy ô tô được cho như sau:
	\begin{center}
		\begin{tabular}{|c|c|c|c|c|c|c|}
			\hline
			Tuổi thọ (năm)	& $\left[2;2{,}5 \right)$ & $\left[2{,}5;3 \right)$ & $\left[3;3{,}5 \right)$&$\left[3{,}5;4 \right)$&$\left[4;4{,}5 \right)$&$\left[4{,}5;5 \right)$  \\
			\hline
			Tần số &$4$	& $9$ & $14$ &$11$  & $7$&$5$ \\
			\hline
		\end{tabular}
	\end{center}
	Tính tuổi thọ trung bình của $50$ bình ắc quy ô tô này.
	\loigiai{
		Ta có bảng sau
		\begin{center}
			\begin{tabular}{|c|c|c|c|c|c|c|}
				\hline
				Tuổi thọ (năm)	& $2{,}25$ & $2{,}75$ & $3{,}25$&$3{,}75$&$4{,}25$&$4{,}75$  \\
				\hline
				Tần số &$4$	& $9$ & $14$ &$11$  & $7$&$5$\\
				\hline
			\end{tabular}	
		\end{center}
		Tuổi thọ trung bình của 50 bình ắc quy ô tô này là
		$$\overline{x}=\dfrac{2{,}25\cdot 4+2{,}75\cdot 9+3{,}25\cdot 14+3{,}75\cdot 11+4{,}25\cdot 7+4{,}75\cdot 5}{50}=3{,}48 \, \text{(năm)}.$$
	}
\end{bt}
\begin{bt}%[KNTT]%[1K3B9-4]
	\immini{
		Phỏng vấn một số học sinh lớp $11$ về thời gian (giờ) ngủ của một buổi tối, thu được bảng số liệu ở bên. So sánh thời gian ngủ trung bình của các bạn học sinh nam và nữ.
	}
	{
		\begin{tabular}{|c|c|c|}
			\hline
			Thời gian	& Số học sinh nam & Số học sinh nữ\\
			\hline
			$\left[4;5 \right)$	& $6$ & $4$ \\
			\hline
			$\left[5;6 \right)$	& $10$ & $8$ \\
			\hline
			$\left[6;7 \right)$	& $13$ & $10$ \\
			\hline
			$\left[7;8 \right)$	& $9$ & $11$ \\
			\hline
			$\left[8;9 \right)$	& $7$ & $8$ \\
			\hline
		\end{tabular}
	}
	\loigiai{
		Trong mỗi khoảng thời gian, giá trị đại diện là trung bình cộng của giá trị hai đầu mút nên ta có bảng sau:
		\begin{center}
			\begin{tabular}{|c|c|c|}
				\hline
				Thời gian	& Số học sinh nam & Số học sinh nữ\\
				\hline
				$4{,}5$	& $6$ & $4$ \\
				\hline
				$5{,}5$	& $10$ & $8$ \\
				\hline
				$6{,}5$	& $13$ & $10$ \\
				\hline
				$7{,}5$	& $9$ & $11$ \\
				\hline
				$8{,}5$	& $7$ & $8$ \\
				\hline
			\end{tabular}	
		\end{center}
		Tổng số học sinh nam là $n_1=6+10+13+9+7=45$.\\ Thời gian ngủ trung bình của học sinh nam là:
		$$\overline{x_1}=\dfrac{4{,}5\cdot 6+5{,}5\cdot10+6{,}5\cdot13+7{,}5\cdot9+8{,}5\cdot7}{45}=\dfrac{587}{90}\approx 6{,}52\,\, \text{(giờ)}.$$
		Tổng số học sinh nữ là $n_2=4+8+10+11+8=41$. Thời gian ngủ trung bình của học sinh nữ là:
		$$\overline{x_2}=\dfrac{4,5\cdot4+5,5\cdot8+6,5\cdot10+7,5\cdot11+8,5\cdot8}{41}=\dfrac{555}{82}\approx 6{,}77 \,\,\text{(giờ)}.$$
		Vì $\overline{x_2}>\overline{x_1}$ nên thời gian ngủ trung bình của các bạn học sinh nữ lớn hơn thời gian ngủ trung bình của các bạn nam.
	}
\end{bt}
\begin{bt}%[KNTT]%[1K3B9-4]
	Quãng đường (km) từ nhà đến nơi làm việc của 40 công nhân một nhà máy được ghi lại như sau:
	\begin{center}
		\begin{tabular}{cccccccccccccccccccc}
			$5$	& $3$ &$10$ & $20$ & $25$ & $11$ & $13$ & $7$ & $12$ & $31$\\
			$19$ &$10$  &$12$  & $17$ & $18$ & $11$ & $32$ & $17$ &$16$  &$2$ \\
			$7$	& $9$ &$7$ & $8$ & $3$ & $5$ & $12$ & $15$ & $18$ & $3$\\
			$12$ &$14$  &$2$  & $9$ & $6$ & $15$ & $15$ & $7$ &$6$  &$12$
		\end{tabular}
	\end{center}
	\begin{enumerate}
		\item [a)] Ghép nhóm dãy số liệu trên thành các khoảng có độ rộng bằng nhau, khoảng đầu tiên là $\left[0;5\right)$. Tìm giá trị đại diện cho mỗi nhóm.
		\item [b)] Tính số trung bình của mẫu số liệu không ghép nhóm và mẫu số liệu ghép nhóm. Giá trị nào chính xác hơn?
	\end{enumerate}
	\loigiai{
		\begin{enumerate}
			\item [a)] Giá trị nhỏ nhất của mẫu số liệu là $2$, giá trị lớn nhất là $32$, khoảng đầu tiên của mẫu số liệu ghép nhóm là $\left[0;5\right)$ nên ta ghép nhóm mẫu số liệu như sau
			\begin{center}
				\begin{tabular}{|c|c|c|c|c|c|c|c|}
					\hline
					Quãng đường		 & $\left[0;5\right)$ & $\left[5;10\right)$ & $\left[10;15\right)$ & $\left[15;20\right)$ & $\left[20;25\right)$& $\left[25;30\right)$& $\left[30;35\right)$\\
					\hline
					Số công nhân		& $5$ & $11$ & $11$ & $9$ & $1$ & $1$ & $2$ \\
					\hline
				\end{tabular}
			\end{center}
			Trong mỗi khoảng, giá trị đại điện là trung bình cộng của hai giá trị đầu mút nên ta có bảng sau
			\begin{center}
				\begin{tabular}{|c|c|c|c|c|c|c|c|}
					\hline
					Quãng đường		 & $2{,}5$ & $7{,}5$ & $12{,}5$ & $17{,}5$ & $22{,}5$& $27{,}5$& $32{,}5$\\
					\hline
					Số công nhân		& $5$ & $11$ & $11$ & $9$ & $1$ & $1$ & $2$ \\
					\hline
				\end{tabular}
			\end{center}
			\item [b)] Số trung bình của mẫu số liệu không ghép nhóm là
			$$\overline{x}=\dfrac{5+3+10+\cdots +12}{40}=11{,}9.$$
			Số trung bình của mẫu số liệu ghép nhóm là
			$$\overline{x}=\dfrac{5\cdot 2{,}5+11\cdot 7{,}5+11\cdot 12{,}5+9\cdot 17{,}5+1\cdot 22{,}5+1\cdot 27{,}5+2\cdot 32{,}5}{40}=12{,}625.$$
			Số trung bình của mẫu số liệu không ghép nhóm sẽ chính xác hơn số trung bình của mẫu số liệu ghép nhóm vì số trung bình của dữ liệu không ghép nhóm sử dụng chính xác các số liệu, còn số trung bình của dữ liệu ghép nhóm sử dụng giá trị đại diện của mỗi khoảng ghép nhóm.
		\end{enumerate}
	}
\end{bt}
\begin{bt}%[CTST]%[1T5B1-2]
	Anh Văn ghi lại cự li 30 lần ném lao của mình ở bảng sau (đơn vị: mét):
	\begin{center}
		\begin{tabular}{|c|c|c|c|c|c|c|c|c|c|}
			\hline $72{,}1$ & $72{,}9$ & $70{,}2$ & $70{,}9$ & $72{,}2$ & $71{,}5$ & $72{,}5$ & $69{,}3$ & $72{,}3$ & $69{,}7$ \\
			\hline $72{,}3$ & $71{,}5$ & $71{,}2$ & $69{,}8$ & $72{,}3$ & $71{,}1$ & $69{,}5$ & $72{,}2$ & $71{,}9$ & $73{,}1$ \\
			\hline $71{,}6$ & $71{,}3$ & $72{,}2$ & $71{,}8$ & $70{,}8$ & $72{,}2$ & $72{,}2$ & $72{,}9$ & $72{,}7$ & $70{,}7$ \\
			\hline
		\end{tabular}
	\end{center}
	\begin{enumerate}
		\item Tính cự li trung bình của mỗi lần ném.
		\item Tổng hợp lại kết quả ném của anh Văn vào bảng tần số ghép nhóm theo mẫu sau:
		\begin{center}
			\begin{tabular}{|c|c|c|c|c|c|}
				\hline Cự li $(\mathrm{m})$ &{$[69{,}2; 70)$} &{$[70; 70{,}8)$} &{$[70{,}8; 71{,}6)$} &{$[71{,}6; 72{,}4)$} &{$[72{,}4; 73{,}2)$} \\
				\hline Số lần & $?$ & $?$ & $?$ & $?$ & $?$ \\
				\hline
			\end{tabular}
		\end{center}
		\item Hãy ước lượng cự li trung bình mỗi lần ném từ bảng tần số ghép nhóm trên.
		\item Khả năng anh Văn ném được khoảng bao nhiêu mét là cao nhất?
	\end{enumerate}
	\loigiai{
		\begin{enumerate}
			\item Điểm tổng của mỗi đợt gồm 10 lần ném
			\begin{center}
				\begin{tabular}{|c|c|c|c|c|c|c|c|c|c|c|}
					\hline Điểm &Điểm &Điểm &Điểm &Điểm &Điểm &Điểm &Điểm &Điểm &Điểm &Tổng \\
					\hline $72{,}1$ & $72{,}9$ & $70{,}2$ & $70{,}9$ & $72{,}2$ & $71{,}5$ & $72{,}5$ & $69{,}3$ & $72{,}3$ & $69{,}7$ &$713{,}6$\\
					\hline $72{,}3$ & $71{,}5$ & $71{,}2$ & $69{,}8$ & $72{,}3$ & $71{,}1$ & $69{,}5$ & $72{,}2$ & $71{,}9$ & $73{,}1$ &$714{,}9$\\
					\hline $71{,}6$ & $71{,}3$ & $72{,}2$ & $71{,}8$ & $70{,}8$ & $72{,}2$ & $72{,}2$ & $72{,}9$ & $72{,}7$ & $70{,}7$ &$718{,}4$\\
					\hline
				\end{tabular}
			\end{center}
			Cự li trung bình của mỗi lần ném của anh Văn
			\[\overline{x}=\dfrac{713{,}6+714{,}9+718{,}4}{30}\approx71{,}56\ (\mathrm{m}). \]
			\item Bảng tần số ghép nhóm kết quả ném của anh Văn:
			\begin{center}
				\begin{tabular}{|c|c|c|c|c|c|}
					\hline Cự li $(\mathrm{m})$ &{$[69{,}2; 70)$} &{$[70; 70{,}8)$} &{$[70{,}8; 71{,}6)$} &{$[71{,}6; 72{,}4)$} &{$[72{,}4; 73{,}2)$} \\
					\hline Số lần & $4$ & $2$ & $7$ & $12$ & $5$ \\
					\hline
				\end{tabular}
			\end{center}
			\item Bảng tần số ghép nhóm kết quả ném của anh Văn (theo giá trị đại diện):
			\begin{center}
				\begin{tabular}{|c|c|c|c|c|c|}
					\hline Cự li $(\mathrm{m})$ &{$[69{,}2; 70)$} &{$[70; 70{,}8)$} &{$[70{,}8; 71{,}6)$} &{$[71{,}6; 72{,}4)$} &{$[72{,}4; 73{,}2)$} \\
					\hline Giá trị đại diện &$69{,}6$ &$70{,}4$ &$71{,}2$ &$72{,}0$ &$72{,}8$\\
					\hline Số lần & $4$ & $2$ & $7$ & $12$ & $5$ \\
					\hline
				\end{tabular}
			\end{center}
			Cự li trung bình mỗi lần ném của anh Văn qua bảng tần số ghép nhóm
			\[(69{,}6\cdot 4+70{,}4\cdot 2+71{,}2\cdot 7+72\cdot 12+72{,}8\cdot 5):30=71{,}52\ (\mathrm{m}).  \]
			\item Nhóm chứa mốt của mẫu số liệu trên là nhóm $[71{,}6; 72{,}4)$.\\
			Do đó $u_m=71{,}6$; $n_{m-1}=7$; $n_m=12$; $n_{m+1}=5$; $u_{m+1}-u_m=72{,}4-71{,}6=0{,}8$.\\
			Mốt của mẫu số liệu ghép nhóm là
			\[M_0=71{,}6+\dfrac{12-7}{(12-7)+(12-5)} \cdot 0{,}8=\dfrac{101}{14} \approx 71{,}93. \]
			Dựa vào kết quả trên thì khả năng anh Văn ném được cao nhất là khoảng $71{,}93$ mét.
		\end{enumerate}
	}
\end{bt}
\begin{bt}%[CTST]%[1T5B1-2]
	Người ta đếm số xe ô tô đi qua một trạm thu phí mỗi phút trong khoảng thời gian từ $9$ giờ đến $9$ giờ $30$ phút sáng. Kết quả được ghi lại ở bảng sau:
	\begin{center}
		\begin{tabular}{|c|c|c|c|c|c|c|c|c|c|c|c|c|c|c|}
			\hline $15$ & $16$ & $13$ & $21$ & $17$ & $23$ & $15$ & $21$ & $6$ & $11$ & $12$ & $23$ & $19$ & $25$ & $11$ \\
			\hline $25$ & $7$ & $29$ & $10$ & $28$ & $29$ & $24$ & $6$ & $11$ & $23$ & $11$ & $21$ & $9$ & $27$ & $15$ \\
			\hline
		\end{tabular}
	\end{center}
	\begin{enumerate}
		\item Tính số xe trung bình đi qua trạm thu phí trong mỗi phút.
		\item Tổng hợp lại số liệu trên vào bảng tần số ghép nhóm theo mẫu sau:
		\begin{center}
			\begin{tabular}{|c|c|c|c|c|c|}
				\hline Số xe &{$[6; 10]$} &{$[11; 15]$} &{$[16; 20]$} &{$[21; 25]$} &{$[26; 30]$} \\
				\hline Số lần & $?$ & $?$ & $?$ & $?$ & $?$ \\
				\hline
			\end{tabular}
		\end{center}
		\item Hãy ước lượng trung bình số xe đi qua trạm thu phí trong mỗi phút từ bảng tần số ghép nhóm trên.
	\end{enumerate}
	\loigiai{
		\begin{enumerate}
			\item 
%			Bảng tần số
%			\begin{center}
%				\begin{tabular}{|c|c|c|c|c|c|c|c|c|c|c|c|c|c|c|c|c|c|c|c|}
%					\hline Giá trị &$6$ & $7$ & $9$ & $10$ & $11$ & $12$ & $13$ & $15$ & $16$ & $17$ & $19$ & $21$ & $23$ & $24$ & $25$ & $27$ & $28$ & $29$ &\\
%					\hline Tần số &$2$ & $1$ & $1$ & $1$ & $4$ & $1$ & $1$ & $3$ & $1$ & $1$ & $1$ & $3$ & $3$ & $1$ & $2$ & $1$ & $1$ & $2$ &$N=30$\\
%					\hline
%				\end{tabular}
%			\end{center}
			Số xe trung bình đi qua trạm thu phí trong mỗi phút là
			\allowdisplaybreaks
			\begin{eqnarray*}
				\overline{x}&=&\dfrac{6\cdot 2+7+9+10+11\cdot 4+12+13+15\cdot 3}{30}\\
				&&+\dfrac{16+17+19+21\cdot 3+23\cdot 3+24+25\cdot 2+27+28+29\cdot 2}{30}\\
				&\approx& 17{,}43\ (\text{xe}).
			\end{eqnarray*}	
			\item Bảng tần số ghép nhóm
			\begin{center}
				\begin{tabular}{|c|c|c|c|c|c|}
					\hline Số xe &{$[6; 10]$} &{$[11; 15]$} &{$[16; 20]$} &{$[21; 25]$} &{$[26; 30]$} \\
					\hline Số lần & $5$ & $9$ & $3$ & $9$ & $4$ \\
					\hline
				\end{tabular}
			\end{center}
			\item Bảng tần số ghép nhóm (theo giá trị đại diện) được hiệu chỉnh lại như sau
			\begin{center}
				\begin{tabular}{|c|c|c|c|c|c|}
					\hline Số xe &{$[5{,}5; 10{,}5)$} &{$[10{,}5; 15{,}5)$} &{$[15{,}5; 20{,}5)$} &{$[20{,}5; 25{,}5)$} &{$[25{,}5; 30{,}5)$} \\
					\hline Giá trị đại diện &{$8$} &{$13$} &{$18$} &{$23$} &{$28$} \\
					\hline Số lần & $5$ & $9$ & $3$ & $9$ & $4$ \\
					\hline
				\end{tabular}
			\end{center}
			Số xe trung bình đi qua trạm qua bảng tần số ghép nhóm là
			\[\overline{x}=\dfrac{8\cdot 5+13\cdot 9+18\cdot 3+23\cdot 9+28\cdot 4}{30}\approx 17{,}67\ (\text{xe}). \]
		\end{enumerate}
	}
\end{bt}
\begin{bt}%[CTST]%[1T5B1-2]
	Một thư viện thống kê số lượng sách được mượn mỗi ngày trong ba tháng ở bảng sau:
	\begin{center}
		\begin{tabular}{|c|c|c|c|c|c|c|c|}
			\hline Số sách &{$[16; 20]$} &{$[21; 25]$} &{$[26; 30]$} &{$[31; 35]$} &{$[36; 40]$} &{$[41; 45]$} &{$[46; 50]$} \\
			\hline Số ngày & 3 & 6 & 15 & 27 & 22 & 14 & 5 \\
			\hline
		\end{tabular}
	\end{center}
	Hãy ước lượng số trung bình của mẫu số liệu ghép nhóm trên.
	\loigiai{
		Vì số lượng sách được mượn là số nguyên nên ta hiệu chỉnh bảng tần số ghép nhóm (theo giá trị đại diện) như sau
		\begin{center}
			{\footnotesize \begin{tabular}{|c|c|c|c|c|c|c|c|}
					\hline Số sách &{$[15{,}5; 20{,}5)$} &{$[20{,}5; 25{,}5)$} &{$[25{,}5; 30{,}5)$} &{$[30{,}5; 35{,}5)$} &{$[35{,}5; 40{,}5]$} &{$[40{,}5; 45{,}5)$} &{$[45{,}5; 50{,}5)$} \\
					\hline Giá trị đại diện &{$18$} &{$23$} &{$28$} &{$33$} &{$38$} &{$43$} &{$48$} \\
					\hline Số ngày & 3 & 6 & 15 & 27 & 22 & 14 & 5 \\
					\hline
			\end{tabular}}
		\end{center}
		Trung bình số lượng sách được mượn mỗi ngày trong 3 tháng của thư viện là
		\[\overline{x}=\dfrac{18\cdot 3+23\cdot 6+28\cdot 15+33\cdot 27+38\cdot 22+43\cdot 14+48\cdot 5}{92}\approx 34{,}58. \]
	}
\end{bt}
\begin{bt}%[CTST]%[1T5B1-2]
	Kết quả đo chiều cao của $200$ cây keo $3$ năm tuổi ở một nông trường được biểu diễn ở biểu đồ dưới đây.
	\begin{center}
		\begin{tikzpicture}[scale=1,font=\scriptsize]
			\def\hoanh{11.5};
			\def\tung{6.5};
			\def\mau{cyan};
			\foreach \x/\n in{1/20,3/35,5/60,7/55,9/30}{\draw[line width=16mm,\mau] (\x,0)--++(0,{\n/10});
				\draw[dashed] (\x,{\n/10})node[above]{$\n$}--(0,{\n/10}) node[left]{$\n$};}
			\foreach \x/\p in {1/[8{,}5;8{,}8),3/[8{,}8;9{,}1),5/[9{,}1;9{,}4),7/[9{,}4;9{,}7),9/[9{,}7;10{,}0)}{\node[below] at (\x,0){\scriptsize $\p$};}
			\draw[->] (0,0)--(\hoanh,0) node[below]{($m$)};
			\draw[->] (0,0)node[below left]{$O$}--(0,\tung) node[left]{(Số cây)};
			\path (current bounding box.north) node[above]		{\textbf{Chiều cao 200 cây keo 3 năm tuổi}};
		\end{tikzpicture}
	\end{center}
	Hãy ước lượng số trung bình của mẫu số liệu ghép nhóm trên.
	\loigiai{
		Bảng tần số ghép nhóm (theo giá trị đại diện)
		\begin{center}
			\begin{tabular}{|c|c|c|c|c|c|}
				\hline Chiều cao &$[8{,}5; 8{,}8)$ &{$[8{,}8; 9{,}1)$} &{$[9{,}1; 9{,}4)$} &{$[9{,}4; 9{,}7)$} &{$[9{,}7; 10{,}0)$} \\
				\hline Giá trị đại diện &$8{,}65$ &$8{,}95$ &$9{,}25$ &$9{,}55$ &$9{,}85$ \\
				\hline Số cây & $20$ & $35$ & $60$ & $55$ & $30$\\
				\hline
			\end{tabular}
		\end{center}
		Chiều cao trung bình của $200$ cây keo 3 năm tuổi là
		\[\overline{x}=\dfrac{8{,}65\cdot 20+8{,}95\cdot 35+9{,}25\cdot 60+9{,}55\cdot 55+9{,}85\cdot 30}{200}\approx 9{,}31. \]
	}
\end{bt}
\begin{bt}%[CTST]%[1T5K2-2]
	Kiểm tra điện lượng của một số viên pin tiểu do một hãng sản xuất thu được kết quả như sau:
	\begin{center}
		\begin{tabular}{|c|c|c|c|c|c|}
			\hline 
			\begin{tabular}{c}
				\textbf{Điện lượng} \\	\textbf{(nghìn mAh)}
			\end{tabular} 
			& $ \left[ 0{,}9 ; 0{,}95\right)  $ & $ \left[ 0{,}95 ; 1{,}0\right)  $ & $ \left[ 1{,}0 ; 1{,}05\right)  $ &$ \left[ 1{,}05 ; 1{,}1\right)  $  &  $ \left[ 1{,}1 ; 1{,}15\right)  $\\ 
			\hline 
			\textbf{Số viên pin}& $ 10 $ & $ 20 $ & $ 35 $ & $ 15 $ & $ 5 $ \\ 
			\hline 
		\end{tabular} 
	\end{center}
	Hãy ước lượng số trung bình của mẫu số liệu ghép nhóm trên.
	\loigiai{
		Tìm số trung bình của mẫu số liệu ghép nhóm.\\
		Ta có bảng thống kê điện lượng của pin theo giá trị đại diện là:
		\begin{center}
			\begin{tabular}{|c|c|c|c|c|c|}
				\hline 
				\begin{tabular}{c}
					\textbf{Điện lượng} \\	\textbf{(nghìn mAh)}
				\end{tabular} 
				& $ \left[ 0{,}9 ; 0{,}95\right)  $ & $ \left[ 0{,}95 ; 1{,}0\right)  $ & $ \left[ 1{,}0 ; 1{,}05\right)  $ &$ \left[ 1{,}05 ; 1{,}1\right)  $  &  $ \left[ 1{,}1 ; 1{,}15\right)  $\\ 
				\hline 
				\textbf{Giá trị đại diện}& $ 0{,}925 $ & $ 0{,}975 $ & $ 1{,}025 $ & $ 1{,}075 $ & $ 1{,}125 $ \\ 
				\hline
				\textbf{Số viên pin}& $ 10 $ & $ 20 $ & $ 35 $ & $ 15 $ & $ 5 $ \\ 
				\hline 
			\end{tabular} 
		\end{center}
		Số trung bình của mẫu số liệu ghép nhóm theo dõi điện lượng của một số viên pin xấp xỉ bằng $$\dfrac{0{,}925\cdot 10 + 0{,}975\cdot 20 +1{,}025 \cdot 35 +1{,}075 \cdot 15+1{,}125 \cdot 5}{10+20+35+15+5}\approx 1{,}016.$$
	}
\end{bt}

%===================================
\setcounter{subsubsection}{0}
\setcounter{ex}{0}
\setcounter{bt}{0}
\begin{dang}{Trung vị}
\end{dang}
\subsubsection{Ví dụ minh hoạ}
\begin{vd}%[Cánh Diều]%[1C5B1-3]
	\immini
	{
		Sau khi kiểm tra về số học sinh trong $100$ lớp học, người ta chia mẫu số liệu đó thành năm nhóm căn cứ vào số lượng học sinh của mỗi lớp (đơn vị: học sinh) và lập bảng tần số ghép nhóm bao gồm tần số tích luỹ như bảng bên. Tìm trung vị của mẫu số liệu đó.
	}
	{
		\begin{tabular}{|c|c|c|}
			\hline
			\textbf{Nhóm} & \textbf{Tần số} & \textbf{Tần số tích luỹ}\\ 
			\hline
			$\left[36;38\right)$ & $9$ & $9$\\
			$\left[38;40\right)$ & $15$ & $24$\\
			$\left[40;42\right)$ & $25$ & $49$\\
			$\left[42;44\right)$ & $30$ & $79$\\
			$\left[44;46\right)$ & $21$ & $100$\\
			\hline
			& $n = 100$ &\\
			\hline
		\end{tabular}
	}
	\loigiai{
		Số phần tử của mẫu là $n=100$. Ta có $\dfrac{n}{2} = \dfrac{100}{2} = 50$.\\
		Do $cf_3 = 49 < 50 < cf_4 = 79$ nên nhóm $4$ là nhóm đầu tiên có tần số tích luỹ lớn hơn hoặc bằng $50$.\\
		Xét nhóm $4$ là nhóm $\left[42;44\right)$ có $r=42$; $d=2$ và $n_4=30$ và nhóm $3$ là nhóm $\left[40;42\right)$ có $cf_3 = 49$.\\
		Khi đó trung vị của mẫu số liệu là 
		\[
		M_e = 42 + \dfrac{50 - 49}{30} \cdot 2 \approx 42\text{ (học sinh)}.
		\]
	}
\end{vd}
\begin{vd}%[KNTT]%[1K3B9-2]
	Thời gian (phút) truy cập internet mỗi buổi tối của một số học sinh được cho trong bảng sau:
	\begin{center}
		\begin{tabular}{|c|c|c|c|c|c|c|}
			\hline
			Thời gian (phút)	& $\left[9{,}5;12{,}5 \right)$ & $\left[12{,}5;15{,}5 \right)$ & $\left[15{,}5;18{,}5 \right)$ & $\left[18{,}5;21{,}5 \right)$ & $\left[21{,}5;24{,}5 \right)$ \\
			\hline
			Số học sinh&$3$	& $12$ & $15$ &$24$  & $2$  \\
			\hline
		\end{tabular}	
	\end{center}
	Tính trung vị của mẫu số liệu ghép nhóm này.
	\loigiai{
		Cỡ mẫu là $n=3+12+15+24+2=56$.\\
		Gọi $x_1,\,\ldots,\,x_{56}$ là thời gian vào internet của $56$ học sinh và giả sử dãy này đã được sắp xếp theo thứ tự tăng dần. Khi đó, trung vị là $\dfrac{x_{28}+x_{29}}{2}$. Do $2$ giá trị $x_{28},\,x_{29}$ thuộc nhóm $\left[15{,}5;18{,}5 \right)$ nên nhóm này chứa trung vị. Do đó, $p=3$; $a_3=15{,}5$; $m_3=15$; $m_1+m_2=3+12=15$; $a_4-a_3=3$ và ta có $$M_e=15{,}5+\dfrac{\dfrac{56}{2}-15}{15}\cdot 3=18{,}1.$$
	}
\end{vd}
\begin{vd}%[CTST]%[1T5B2-1]
	Kết quả khảo sát cân nặng của $ 25 $ quả bơ ở một lô hàng cho trong bảng sau:
	\begin{center}
		\begin{tabular}{|c|c|c|c|c|c|}
			\hline 
			\textbf{Cân nặng}\textbf{ (g)}	& $ \left[150 ; 155 \right) $ & $ \left[ 155 ; 160\right)  $ & $ \left[160 ; 165\right)  $ & $ \left[ 165 ; 170\right)  $ & $ \left[170 ; 175 \right)  $ \\ 
			\hline 
			\textbf{Số quả bơ}	& $ 1 $ & $ 7 $ & $ 12 $ & $ 3 $ & $ 2 $ \\ 
			\hline 
		\end{tabular} 
	\end{center}
	Hãy tìm trung vị của mẫu số liệu ghép nhóm trên.
	\loigiai{
		Gọi $ x_1; x_2; \ldots ; x_{25} $ là cân nặng của $ 25$ quả bơ xếp theo thứ tự không giảm.\\
		Do $ x_1\in \left[150 ; 155 \right) $; $ x_2, \ldots, x_8 \in \left[ 155 ; 160\right) $; $ x_9, \ldots, x_{20} \in \left[ 160 ; 165\right) $ nên trung vị của mẫu số liệu $ x_1; x_2; \ldots; x_{25} $ là $ x_{13}\in \left[ 160 ; 165\right)$.\\
		Ta xác định được $ n=25 $, $ n_m=12 $, $ C=1+7=8 $, $ u_m=160 $, $ u_{m+1}=165 $.\\
		Vậy trung vị của mẫu số liệu ghép nhóm là $$ M_e=160+\dfrac{\dfrac{25}{2}-8}{12}\cdot(165-160) =161{,}875.$$
	}
\end{vd}
\begin{vd}%[CTST]%[1T5K2-1]
	Trong tuần lễ bảo vệ môi trường, các học sinh khối $ 11 $ tiến hành thu nhặt vỏ chai nhựa để tái chế. Nhà trường thống kê kết quả thu nhặt vỏ chai của học sinh khối $ 11 $ ở bảng sau
	\begin{center}
		\begin{tabular}{|c|c|c|c|c|c|}
			\hline 
			\textbf{Số vỏ chai nhựa}	& $ \left[ 11 ; 15\right]  $ & $ \left[ 16 ; 29\right]  $ & $ \left[21 ; 25 \right]  $ & $ \left[ 26 ; 30\right]  $ & $ \left[31 ; 35 \right]  $ \\ 
			\hline 
			\textbf{Số học sinh}	& $ 53 $ & $ 82 $ & $ 48 $ & $ 39 $ & $ 18 $ \\ 
			\hline 
		\end{tabular} 
	\end{center}
	Hãy tìm trung vị của mẫu số liệu ghép nhóm trên.
	\loigiai{
		Do số vỏ chai là số nguyên nên ta hiệu chỉnh lại như sau:
		\begin{center}
			\begin{tabular}{|c|c|c|c|c|c|}
				\hline 
				\textbf{Số vỏ chai nhựa}	& $ \left[ 10{,}5 ; 15{,}5\right) $ & $ \left[ 15{,}5 ; 20{,}5\right) $ & $ \left[ 20{,}5 ; 25{,}5\right) $ & $ \left[ 25{,}5 ; 30{,}5\right) $ & $ \left[30{,}5 ; 35{,}5 \right)  $ \\ 
				\hline 
				\textbf{Số học sinh}& $ 53 $ & $ 82 $ & $ 48 $ & $ 39 $ & $ 18 $ \\ 
				\hline 
			\end{tabular} 
		\end{center}
		Số học sinh tham gia thu nhặt vỏ chai nhựa là $$ n=53+82+48+39+18=240.$$
		Gọi $ x_1; x_2; \ldots ; x_{240} $ lần lượt là số vỏ chai $ 240 $ học sinh khối $ 11 $ thu nhặt được xếp theo thứ tự không giảm.\\
		Do $ x_1, \ldots, x_{53}\in \left[10{,}5 ; 15{,}5 \right) $; $ x_{54}, \ldots, x_{135}\in \left[ 15{,}5 ; 20{,}5\right)$ nên trung vị của mẫu số liệu $ x_1; x_2; \ldots;x_{240} $ là $$ \dfrac{1}{2}\left( x_{120}+x_{121}\right)\in \left[ 15{,}5 ; 20{,}5\right).$$
		Ta xác định được $ n=240$; $ n_m=82 $; $ C=53 $; $ u_m=15{,}5 $; $ u_{m+1}=20{,}5 $.\\
		Trung vị của mẫu số liệu ghép nhóm là $$ M_e=15{,}5+\dfrac{\dfrac{240}{2}-53}{82}\cdot \left( 20{,}5-15{,}5\right)=\dfrac{803}{41}\approx 19{,}59. $$
	}
\end{vd}
\begin{vd}%[CTST]%[1T5K2-1]
	Trong một hội thao, thời gian chạy $200$m của một nhóm các vận động viên được ghi lại ở bảng sau
	\begin{center}
		\begin{tabular}{|c|c|c|c|c|c|}
			\hline 
			\textbf{Thời gian} \textbf{(giây)}& $ \left[21 ; 21{,}5 \right)  $ & $ \left[ 21{,}5 ; 22\right)  $ & $ \left[ 22 ; 22{,}5\right)  $ & $ \left[ 22{,}5 ; 23\right)  $ & $ \left[ 23 ; 23{,}5\right)  $ \\ 
			\hline 
			\textbf{Số vận động viên} & $ 5 $ & $ 12 $ & $ 32 $ & $ 45 $ & $ 30 $ \\ 
			\hline 
		\end{tabular} 
	\end{center}
	Dựa vào bảng số liệu trên, ban tổ chức muốn chọn ra khoảng $ 50 \% $ số vận động viên chạy nhanh nhất để tiếp tục thi vòng $ 2 $. Ban tổ chức nên chọn các vận động viên có thời gian chạy không quá bao nhiêu giây?
	\loigiai{
		Số vận động viên tham gia là $$n=5+12+32+45+30=124.$$
		Gọi $ x_1; x_2; \ldots ; x_{124} $ lần lượt là thời gian chạy $ 200 $ m của $ 124 $ vận động viên được xếp theo thứ tự không giảm.\\
		Do $ x_1, \ldots, x_5 \in \left[ 21 ; 21{,}5 \right)$, $ x_6, \ldots, x_{17} \in \left[ 21{,}5 ; 22\right) $, $ x_{18}, \ldots, x_{49} \in \left[22 ; 22{,}5\right) $, $ x_{50},\ldots, x_{94} \in \left[ 22{,}5 ; 23\right) $ nên trung vị của mẫu số liệu $ x_1; x_2; \ldots ;x_{124} $ là
		$$\dfrac{1}{2}\cdot \left( x_{62}+x_{63}\right) \in  \left[ 22{,}5 ; 23\right).$$
		Ta xác định được $ n=124 $; $ n_m=45$; $ C=5+12+32=49 $; $ u_m= 22{,}5$; $ u_{m+1}=23$.\\
		Trung vị của mẫu số liệu ghép nhóm là $$M_e=22{,}5 +\dfrac{\dfrac{124}{2}-49}{45}\cdot \left( 23-22{,}5\right)= \dfrac{1019}{45}\approx 22{,}64.$$
		Vậy ban tổ chức nên chọn các vận động viên  có thời gian chạy không quá $ 22{,}64$ (giây) được tiếp tục thi vòng hai.
	}
\end{vd}
\subsubsection{Bài tập rèn luyện}
\begin{bt}%[Cánh diều]%[1C5B1-5]
	\immini
	{
		Bảng bên cho ta bảng tần số ghép nhóm số liệu thống kê chiều cao của $40$ mẫu cây ở một vườn thực vật (đơn vị: centimét). Xác định trung vị của mẫu số liệu ghép nhóm trên.
	}
	{
		\begin{tabular}{|c|c|c|}
			\hline
			\textbf{Nhóm} & \textbf{Tần số} & \textbf{Tần số tích luỹ}\\ 
			\hline
			$\left[30;40\right)$ & $4$ & $4$\\
			$\left[40;50\right)$ & $10$ & $14$\\
			$\left[50;60\right)$ & $14$ & $28$\\
			$\left[60;70\right)$ & $6$ & $34$\\
			$\left[70;80\right)$ & $4$ & $38$\\
			$\left[80;90\right)$ & $2$ & $40$\\
			\hline
			& $n = 40$ &\\
			\hline
		\end{tabular}
	}
	\loigiai{
		Ta có $\dfrac{n}{2} = 20$, mà $14<20<28$ nên nhóm $3$ là nhóm đầu tiên có tần số tích luỹ lớn hơn hoặc bằng $20$. \\
		Xét nhóm $3$ là nhóm $\left[50;60\right)$ có $r=50$, $d=10$, $n_3=14$ và nhóm $2$ có $cf_2 = 14$.\\
		Khi đó, tứ phân vị thứ hai (cũng là trung vị) là
		\[
		M_e = 50 + \dfrac{20 - 14}{14} \cdot 10 = 54{,}3\text{ (cm)}.
		\]
	}
\end{bt}
\begin{bt}%[Cánh diều]%[1C5B1-5]
	Mẫu số liệu sau ghi lại cân nặng của $30$ bạn học sinh (đơn vị: kilôgam)
	\[
	\begin{array}{cccccccccc}
		17 & 40 & 39 & 40{,}5 & 42 & 51 & 41{,}5 & 39 & 41 & 30\\
		40 & 42 & 40{,}5 & 39{,}5 & 41 & 40{,}5 & 37 & 39{,}5 & 40 & 41\\
		38{,}5 & 39{,}5 & 40 & 41 & 39 & 40{,}5 & 40 & 38{,}5 & 39{,}5 & 41{,}5
	\end{array}
	\]
	\begin{enumerate}
		\item Lập bảng tần số ghép nhóm cho mẫu số liệu trên có tám nhóm ứng với tám nửa khoảng:
		\[
		[15 ; 20),[20 ; 25),[25 ; 30),[30 ; 35),[35 ; 40),[40 ; 45),[45 ; 50),[50 ; 55).
		\]
		\item Xác định trung vị của mẫu số liệu ghép nhóm trên.
	\end{enumerate}
	\loigiai{
		\begin{enumerate}
			\item Ta có bảng tần số ghép nhóm của mẫu số liệu trên như sau:
			\begin{center}
				\begin{tabular}{|c|c|c|c|}
					\hline
					\textbf{Nhóm} & \textbf{Giá trị đại diện} & \textbf{Tần số} & \textbf{Tần số tích luỹ}\\ 
					\hline
					$\left[15;20\right)$ & $17{,}5$ & $1$ & $1$\\
					$\left[20;25\right)$ & $22{,}5$ & $0$ & $1$\\
					$\left[25;30\right)$ & $27{,}5$ & $0$ & $1$\\
					$\left[30;35\right)$ & $32{,}5$ & $1$ & $2$\\
					$\left[35;40\right)$ & $37{,}5$ & $10$ & $12$\\
					$\left[40;45\right)$ & $42{,}5$ & $17$ & $29$\\
					$\left[45;50\right)$ & $47{,}5$ & $0$ & $29$\\
					$\left[50;55\right)$ & $52{,}5$ & $1$ & $30$\\
					\hline
					&  & $n = 30$ &\\
					\hline
				\end{tabular}
			\end{center}
			\item Ta có $\dfrac{n}{2} = 15$, mà $12<15<29$ nên nhóm $6$ là nhóm đầu tiên có tần số tích luỹ lớn hơn hoặc bằng $15$. \\
			Xét nhóm $6$ là nhóm $\left[40;45\right)$ có $r=40$, $d=5$, $n_6=17$ và nhóm $5$ có $cf_5 = 12$.\\
			Khi đó, trung vị là
			\[
			M_e = 40 + \dfrac{15-12}{17} \cdot 5 = 40{,}9\text{ (kg)}.
			\]
		\end{enumerate}
	}
\end{bt}
\begin{bt}%[CTST]%[1T5G2-2]
	Cân nặng của một số lợn con mới sinh thuộc hai giống $ A $ và $ B $ được cho ở biểu đồ dưới đây (đơn vị: kg).
	\begin{center}
		\begin{tikzpicture}[>=stealth,line join=round,line cap=round,font=\footnotesize,scale=0.85,line width=1pt]
			\draw[->] (0,0)--(0,5)node[left]{(\text{Số con})};
			\foreach \y in {1,2,3,4}
			\draw[shift={(0,\y)}] (0,0)--(-2pt,0) node[left]{\scriptsize ${\y}0$};
			%	\path (4.5,6) node {\normalsize{\textbf{Cân nặng của một số lợn con mới sinh}}};
			\path (4.5,5.5) node {
				$\begin{array}{c}
					\normalsize{\textbf{Cân nặng của một số}}\\
					\normalsize{\textbf{lợn con mới sinh}}
				\end{array}$
			};
			%% nhãn
			\path (2.5,-1.5) node[rectangle,fill=cyan,draw=none]{};
			\path (3.6,-1.5) node {\text{Giống $ A $}};
			\path (5,-1.5) node[rectangle,fill=orange,draw=none]{};
			\path (6.1,-1.5) node {\text{Giống $ B $}};
			% đường gióng
			\foreach \y in {1,2,3,4}{
				\draw[line width=0.2pt] (0,\y)--(8.4,\y);
			}
			%% cột
			\draw[fill=cyan,draw=none] (0,0)--(0,0.8)--(1,0.8)node[midway,above]{$ 8 $}--(1,0)--cycle;
			\draw[fill=orange,draw=none] (1,0)--(1,1.3)--(2,1.3)node[midway,above]{$ 13 $}--(2,0)--cycle;
			\draw[fill=cyan,draw=none] (2,0)--(2,2.8)--(3,2.8)node[midway,above]{$ 28 $}--(3,0)--cycle;
			\draw[fill=orange,draw=none] (3,0)--(3,1.4)--(4,1.4)node[midway,above]{$ 14 $}--(4,0)--cycle;
			\draw[fill=cyan,draw=none] (4,0)--(4,3.2)--(5,3.2)node[midway,above]{$ 32 $}--(5,0)--cycle;
			\draw[fill=orange,draw=none] (5,0)--(5,2.4)--(6,2.4)node[midway,above]{$ 24 $}--(6,0)--cycle;
			\draw[fill=cyan,draw=none] (6,0)--(6,1.7)--(7,1.7)node[midway,above]{$ 17 $}--(7,0)--cycle;
			\draw[fill=orange,draw=none] (7,0)--(7,1.4)--(8,1.4)node[midway,above]{$ 14 $}--(8,0)--cycle;
			%% miền
			\node [below] at (1,0){$ \left[1{,}0 ; 1{,}1 \right)$};
			\node [below] at (3,0){$ \left[1{,}1 ; 1{,}2 \right)$};
			\node [below] at (5,0){$ \left[1{,}2 ; 1{,}3 \right)$};
			\node [below] at (7,0){$ \left[1{,}3 ; 1{,}4 \right)$};
			\draw[->] (0,0)node [below left=-2pt]{$ O $}--(9,0)node[below]{(\text{kg})};
		\end{tikzpicture}
	\end{center}
	Hãy so sánh cân nặng của lợn con mới sinh giống $ A $ và giống $ B $ theo số trung bình và trung vị.
	\loigiai{
		Bảng tần số ghép nhóm thống kê cân nặng của lợn con mới sinh giống $ A $ và giống $ B $ như sau:
		\begin{center}
			\begin{tabular}{|c|c|c|c|c|}
				\hline 
				\textbf{Cân nặng (kg)}	& $ \left[1{,}0 ; 1{,}1 \right)$  &$ \left[1{,}1 ; 1{,}2 \right)$  &$ \left[1{,}2 ; 1{,}3 \right)$  & $ \left[1{,}3 ; 1{,}4 \right)$ \\ 
				\hline 
				\textbf{Giá trị đại diện (kg)}	& $1{,}05 $ & $ 1{,}15 $ & $ 1{,}25 $ & $ 1{,}35 $ \\ 
				\hline 
				\begin{tabular}{c}
					\textbf{Giống A}
					\\ 
					\textbf{(đơn vị: con)}
				\end{tabular} 	& $ 8 $ & $ 28 $ & $ 32 $ & $ 17 $ \\ 
				\hline 
				\begin{tabular}{c}
					\textbf{Giống B}
					\\ 
					\textbf{(đơn vị: con)}
				\end{tabular} 	& $ 13 $ & $ 14 $ & $ 24 $ & $ 14 $ \\ 
				\hline 
			\end{tabular} 
		\end{center}
		Cân nặng trung bình của lợn con mới sinh giống $ A $ là $$ \dfrac{1{,}05\cdot 8 + 1{,}15 \cdot 28 + 1{,}25 \cdot 32 + 1{,}35 \cdot 17}{8+28+32+17}=\dfrac{2071}{1700}\approx 1{,}218.$$
		Cân nặng trung bình của lợn con mới sinh giống $ B $ là $$ \dfrac{1{,}05\cdot 13 + 1{,}15 \cdot 14 + 1{,}25 \cdot 24 + 1{,}35 \cdot 14}{13+14+24+14}=\dfrac{121}{100} \approx 1{,}21.$$
		Suy ra cân nặng trung bình của lợn con mới sinh giống $ A $  lớn hơn cân nặng trung bình của lợn con mới sinh giống $ B $. \\
		Trung vị của mẫu số liệu ghép nhóm cân nặng của lợn con  giống $ A $ là $$M_e=1{,}2+\dfrac{\dfrac{85}{2}-(8+28)}{32}\cdot (1{,}3-1{,}2)=\dfrac{781}{640}\approx 1{,}22.$$
		Trung vị của mẫu số liệu ghép nhóm cân nặng của lợn con  giống  $ B $ là $$M_e=1{,}2+\dfrac{\dfrac{65}{2}-(13+14)}{24}\cdot (1{,}3-1{,}2)=\dfrac{587}{480}\approx 1{,}22.$$
		Suy ra trung vị của mẫu số liệu ghép nhóm cân nặng của của lợn con giống $ A $ bằng trung vị của mẫu số liệu ghép nhóm cân nặng của của lợn con giống $ B $.
	}
\end{bt}

\setcounter{subsubsection}{0}
\setcounter{ex}{0}
\setcounter{bt}{0}
\begin{dang}{Tứ phân vị}
	
\end{dang}
\subsubsection{Ví dụ minh hoạ}
\begin{vd}
	\immini
	{
		Bảng bên cho biết tần số ghép nhóm số liệu thống kê cân nặng của $40$ học sinh lớp $11A$ trong một trường trung học phổ thông (đơn vị: kilôgam). Xác định tứ phân vị của mẫu số liệu ghép nhóm.
	}
	{
		\begin{tabular}{|c|c|c|}
			\hline
			\textbf{Nhóm} & \textbf{Tần số} & \textbf{Tần số tích luỹ}\\ 
			\hline
			$\left[30;40\right)$ & $2$ & $2$\\
			$\left[40;50\right)$ & $10$ & $12$\\
			$\left[50;60\right)$ & $16$ & $28$\\
			$\left[60;70\right)$ & $8$ & $36$\\
			$\left[70;80\right)$ & $2$ & $38$\\
			$\left[80;90\right)$ & $2$ & $40$\\
			\hline
			& $n = 40$ &\\
			\hline
		\end{tabular}
	}
	\loigiai{
		Số phần tử của mẫu là $n=40$.
		\begin{itemize}
			\item Ta có $\dfrac{n}{4} = 10$, mà $2<10<12$ nên nhóm $2$ là nhóm đầu tiên có tần số tích luỹ lớn hơn hoặc bằng $10$. \\
			Xét nhóm $2$ là nhóm $\left[40;50\right)$ có $s=40$, $h=10$, $n_2=10$ và nhóm $1$ có $cf_1 = 2$.\\
			Khi đó, tứ phân vị thứ nhất là
			\[
			Q_1 = 40 + \dfrac{10-2}{10} \cdot 10 = 48\text{ (kg)}.
			\]
			\item Ta có $\dfrac{n}{2} = 20$, mà $12<20<28$ nên nhóm $3$ là nhóm đầu tiên có tần số tích luỹ lớn hơn hoặc bằng $20$. \\
			Xét nhóm $3$ là nhóm $\left[50;60\right)$ có $r=50$, $d=10$, $n_3=16$ và nhóm $2$ có $cf_2 = 12$.\\
			Khi đó, tứ phân vị thứ hai là
			\[
			Q_2 = 50 + \dfrac{20-12}{16} \cdot 10 = 55\text{ (kg)}.
			\]
			\item Ta có $\dfrac{3n}{4} = 30$, mà $28<30<36$ nên nhóm $4$ là nhóm đầu tiên có tần số tích luỹ lớn hơn hoặc bằng $30$. \\
			Xét nhóm $4$ là nhóm $\left[60;70\right)$ có $t=50$, $l=10$, $n_4=8$ và nhóm $3$ có $cf_3 = 28$.\\
			Khi đó, tứ phân vị thứ ba là
			\[
			Q_3 = 60 + \dfrac{30-28}{8} \cdot 10 = 62{,}5\text{ (kg)}.
			\]
		\end{itemize}
		Vậy tứ phân vị của mẫu số liệu trên là $48$, $55$ và $62{,}5$.
	}
\end{vd}
\subsubsection{Bài tập rèn luyện}
\begin{bt}%[1K3B9-3]
	Thời gian (phút) truy cập internet mỗi buổi tối của một số học sinh được cho trong bảng sau:
	\begin{center}
		\begin{tabular}{|c|c|c|c|c|c|c|}
			\hline
			Thời gian (phút)	& $\left[9{,}5;12{,}5 \right)$ & $\left[12{,}5;15{,}5 \right)$ & $\left[15{,}5;18{,}5 \right)$ & $\left[18{,}5;21{,}5 \right)$ & $\left[21{,}5;24{,}5 \right)$ \\
			\hline
			Số học sinh&$3$	& $12$ & $15$ &$24$  & $2$  \\
			\hline
		\end{tabular}	
	\end{center}
	Tìm tứ phân vị thứ nhất $Q_1$ và tứ phân vị thứ ba $Q_3$ của mẫu số liệu ghép nhóm.
	\loigiai{
		Cỡ mẫu là $n=3+12+15+24+2=56$.\\
		Tứ phân vị thứ nhất $Q_1$ là $\dfrac{x_{14}+x_{15}}{2}$. Do $2$ giá trị $x_{28},\,x_{29}$ thuộc nhóm $\left[12{,}5;15{,}5 \right)$ nên nhóm này chứa $Q_1$. Do đó, $p=2$; $a_2=12{,}5$; $m_2=12$; $m_1=3$; $a_3-a_2=3$ và ta có $$Q_1=12{,}5+\dfrac{\dfrac{56}{4}-3}{12}\cdot 3=15{,}25.$$
		Với tứ phân vị thứ ba $Q_3$ là $\dfrac{x_{42}+x_{43}}{2}$. Do $2$ giá trị $x_{42},\,x_{43}$ thuộc nhóm $\left[18{,}5;21{,}5 \right)$ nên nhóm này chứa $Q_3$. Do đó, $p=4$; $a_4=18{,}5$; $m_4=24$; $m_1+m_2+m_3=3+12+15=30$; $a_5-a_4=3$ và ta có $$Q_3=18{,}5+\dfrac{\dfrac{3\cdot 56}{4}-30}{24}\cdot 3=20.$$
	}
\end{bt}
\begin{bt}%[1K3B9-4]
	Điểm thi môn Toán (thang điểm 100, điểm được làm tròn đến 1) của 60 thí sinh được cho trong bảng sau:
	\begin{center}
		\begin{tabular}{|c|c|c|c|c|c|}
			\hline
			Điểm		& $0-9$ & $10-19$ & $20-29$ & $30-39$ & $40-49$ \\
			\hline
			Số thí sinh	& $1$ & $2$ & $4$ & $6$ & $15$ \\
			\hline
			Điểm	& $50-59$ & $60-69$ & $70-79$ & $80-89$ & $90-99$  \\
			\hline
			Số thí sinh	& $12$ & $10$ & $6$ & $3$ & $1$  \\
			\hline
		\end{tabular}
	\end{center}
	\begin{enumerate}
		\item [a)] Hiệu chỉnh để thu được mẫu số liệu ghép nhóm dạng Bảng $3.2$.
		\item [b)] Tìm các tứ phân vị và giải thích ý nghĩa của chúng.
	\end{enumerate}
	\loigiai{
		\begin{enumerate}
			\item [a)] Bảng số liệu ghép nhóm về điểm thi môn Toán của 60 thí sinh
			\begin{center}
				\begin{tabular}{|c|c|c|c|c|c|}
					\hline
					Điểm		& $\left[0;20\right)$ & $\left[20;40\right)$ & $\left[40;60\right)$ & $\left[60;80\right)$ & $\left[80;100\right)$ \\
					\hline
					Số thí sinh	& $3$ & $10$ & $27$ & $16$ & $4$ \\
					\hline
				\end{tabular}
			\end{center}
			
			\item [b)] Cỡ mẫu $n=60$. Gọi $x_1$, $x_2$,$\ldots$, $x_{60}$ là điểm thi môn Toán của 60 học sinh và giả sử dãy này đã được sắp xếp theo thứ tự tăng dần. Khi đó, trung vị là $\dfrac{x_{30}+x_{31}}{2}$.\\
			Do hai giá trị $x_{30}$, $x_{31}$ thuộc nhóm $\left[40;60\right)$ nên nhóm này chứa trung vị. Do đó, $p=3;a_3=40;m_3=27;m_1+m_2=13;a_4-a_3=20$ và ta có
			$$Q_2=M_e=40+\dfrac{\dfrac{60}{2}-13}{27}\cdot 20\approx 52{,}6.$$
			Tứ phân vị thứ nhất $Q_1=\dfrac{x_{15}+x_{16}}{2}$. Do hai giá trị $x_{15}$, $x_{16}$ thuộc nhóm $\left[40;60\right)$ nên nhóm này chứa $Q_1$. Do đó, $p=3;\,a_3=40;\,m_3=27;\,m_1+m_2=13;\,a_4-a_3=20$ và ta có
			$$Q_1=40+\dfrac{\dfrac{60}{4}-13}{27}\cdot 20\approx 41{,}5.$$
			Tứ phân vị thứ ba $Q_3=\dfrac{x_{45}+x_{46}}{2}$. Do hai giá trị $x_{45}$, $x_{46}$ thuộc nhóm $\left[60;80\right)$ nên nhóm này chứa $Q_3$. Do đó, $p=4;\,a_4=60;\,m_4=16;\,m_1+m_2+m_3=40;\,a_5-a_4=20$ và ta có
			$$Q_3=60+\dfrac{\dfrac{3\cdot 60}{4}-40}{16}\cdot 20\approx 66{,}3.$$
			Khoảng cách từ $Q_1$ đến $Q_2$ là $11{,}1$ còn khoảng cách từ $Q_2$ và $Q_3$ là $13{,}7$. Điều này cho thấy mẫu số liệu tập trung với mật độ cao hơn ở bên trái $Q_2$ và mật độ thấp hơn ở bên phải $Q_2$.
		\end{enumerate}	
	}
\end{bt}
\begin{bt}%[1K3B9-4]
	\immini{
		Phỏng vấn một số học sinh lớp $11$ về thời gian (giờ) ngủ của một buổi tối, thu được bảng số liệu ở bên.
		\begin{enumerate}
			\item [a)] So sánh thời gian ngủ trung bình của các bạn học sinh nam và nữ.
			\item [b)] Hãy cho biết $75\%$ học sinh khối $11$ ngủ ít nhất bao nhiêu giờ?
		\end{enumerate}
	}
	{
		\begin{tabular}{|c|c|c|}
			\hline
			Thời gian	& Số học sinh nam & Số học sinh nữ\\
			\hline
			$\left[4;5 \right)$	& $6$ & $4$ \\
			\hline
			$\left[5;6 \right)$	& $10$ & $8$ \\
			\hline
			$\left[6;7 \right)$	& $13$ & $10$ \\
			\hline
			$\left[7;8 \right)$	& $9$ & $11$ \\
			\hline
			$\left[8;9 \right)$	& $7$ & $8$ \\
			\hline
		\end{tabular}
	}
	\loigiai{
		\begin{enumerate}
			\item [a)] Trong mỗi khoảng thời gian, giá trị đại diện là trung bình cộng của giá trị hai đầu mút nên ta có bảng sau:
			\begin{center}
				\begin{tabular}{|c|c|c|}
					\hline
					Thời gian	& Số học sinh nam & Số học sinh nữ\\
					\hline
					$4{,}5$	& $6$ & $4$ \\
					\hline
					$5{,}5$	& $10$ & $8$ \\
					\hline
					$6{,}5$	& $13$ & $10$ \\
					\hline
					$7{,}5$	& $9$ & $11$ \\
					\hline
					$8{,}5$	& $7$ & $8$ \\
					\hline
				\end{tabular}	
			\end{center}
			Tổng số học sinh nam là $n_1=6+10+13+9+7=45$.\\ Thời gian ngủ trung bình của học sinh nam là:
			$$\overline{x_1}=\dfrac{4{,}5\cdot 6+5{,}5\cdot10+6{,}5\cdot13+7{,}5\cdot9+8{,}5\cdot7}{45}=\dfrac{587}{90}\approx 6{,}52\,\, \text{(giờ)}.$$
			Tổng số học sinh nữ là $n_2=4+8+10+11+8=41$. Thời gian ngủ trung bình của học sinh nữ là:
			$$\overline{x_2}=\dfrac{4,5\cdot4+5,5\cdot8+6,5\cdot10+7,5\cdot11+8,5\cdot8}{41}=\dfrac{555}{82}\approx 6{,}77 \,\,\text{(giờ)}.$$
			Vì $\overline{x_2}>\overline{x_1}$ nên thời gian ngủ trung bình của các bạn học sinh nữ lớn hơn thời gian ngủ trung bình của các bạn nam.
			\item [b)] Tổng số học sinh được điều tra là $n=n_1+n_2=45+41=86$.\\
			Giả sử $x_1;x_2;x_3;\cdot \cdot;x_{86}$ là dãy giá trị được sắp xếp theo thứ tự không giảm.\\
			Ta có bảng sau:
			\begin{center}
				\begin{tabular}{|c|c|c|}
					\hline
					Thời gian	& Số học sinh \\
					\hline
					$\left[4;5 \right)$	& $10$  \\
					\hline
					$\left[5;6 \right)$	& $18$  \\
					\hline
					$\left[6;7 \right)$	& $23$  \\
					\hline
					$\left[7;8 \right)$	& $20$  \\
					\hline
					$\left[8;9 \right)$	& $15$  \\
					\hline
				\end{tabular}
			\end{center}
			Tứ phân vị thứ nhất $Q_1$ là $x_{22}$. Do $x_{22}$ thuộc nhóm $\left[5;6\right)$ nên nhóm này chứa $Q_1$.\\ Do đó, $p=2;\,a_2=5;\,m_2=18;\,m_1=10;\,a_3-a_2=1$ và ta có
			$$Q_1=5+\dfrac{\dfrac{86}{4}-10}{18}\cdot 1=\dfrac{203}{36}\approx 5{,}64 \text{(giờ)}.$$
			Nghĩa là có $25\%$ học sinh khối $11$ ngủ ít hơn $5{,}64$ giờ.\\ Vậy $75\%$ học sinh khối $11$ ngủ ít nhất $5{,}64$ giờ.
		\end{enumerate}
	}
\end{bt}
\setcounter{subsubsection}{0}
\setcounter{ex}{0}
\setcounter{bt}{0}
\begin{dang}{Mốt}
	
\end{dang}
\subsubsection{Ví dụ minh hoạ}
\begin{vd}%[Tex hóa SGK CD, TVN-223]%[1C5B1-5]
	Kết quả kiểm tra môn Toán của lớp $11D$ như sau
	\[
	\begin{array}{cccccccccccccccccccc}
	5 & 6 & 7 & 5 & 6 & 9 & 10 & 8 & 5 & 5 & 4 & 5 & 4 & 5 & 7 & 4 & 5 & 8 & 9 & 10 \\
	5 & 3 & 5 & 6 & 5 & 7 & 5 & 8 & 4 & 9 & 5 & 6 & 5 & 6 & 8 & 8 & 7 & 9 & 7 & 9
	\end{array}
	\]
	\begin{enumerate}
		\item Lập bảng tần số ghép nhóm của mẫu số liệu trên có bốn nhóm ứng với bốn nửa khoảng $\left[3;5\right)$, $\left[5;7\right)$, $\left[7;9\right)$, $\left[9;11\right)$.
		\item Mốt của bảng số liệu ghép nhóm trên là bao nhiêu (làm tròn kết quả đến hàng phần mười)?
	\end{enumerate}
	\loigiai{
		\immini
		{
			\begin{enumerate}
				\item Bảng bên là bảng tần số ghép nhóm cho kết quả kiểm tra môn Toán của lớp $11D$.
				\item Ta thấy: Nhóm $2$ ứng với nửa khoảng $\left[5;7\right)$ là nhóm có tần số lớn nhất với $u=5$, $g=2$, $n_2 = 18$. Nhóm $1$ có tần số $n_1 = 5$, nhóm $3$ có tần số $n_3=10$.\\
				Khi đó, mốt của mẫu số liệu là 
				\[
				M_o = 5 + \left( \dfrac{18- 5}{2\cdot 18 - 5 - 10} \right) \cdot 2 \approx 6{,}2.
				\]
			\end{enumerate}
		}
		{
			\begin{tabular}{|c|c|}
				\hline
				\textbf{Nhóm} & \textbf{Tần số}\\ 
				\hline
				$\left[3;5\right)$ & $5$\\
				\hline
				$\left[5;7\right)$ & $18$\\
				\hline
				$\left[7;9\right)$ & $10$\\
				\hline
				$\left[9;11\right)$ & $7$\\
				\hline
				& $n = 40$ \\
				\hline
			\end{tabular}
		}
	}
\end{vd}
\subsubsection{Bài tập rèn luyện}
\begin{bt}%[Tex hóa SGK CD, TVN-223]%[1C5B1-5]
	Mẫu số liệu dưới đây ghi lại tốc độ của $40$ ô tô khi đi qua một trạm đo tốc độ (đơn vị: km/h):
	\[
	\begin{array}{cccccccccc}
	48{,}5 & 43 & 50 & 55 & 45 & 60 & 53 & 55,5 & 44 & 65 \\
	51 & 62,5 & 41 & 44,5 & 57 & 57 & 68 & 49 & 46{,}5 & 53{,}5 \\
	61 & 49{,}5 & 54 & 62 & 59 & 56 & 47 & 50 & 60 & 61 \\
	49{,}5 & 52{,}5 & 57 & 47 & 60 & 55 & 45 & 47,5 & 48 & 61{,}5
	\end{array}
	\]
	\begin{enumerate}
		\item Lập bảng tần số ghép nhóm cho mẫu số liệu trên có sáu nhóm ứng với sáu nửa khoảng:
		\[
		[40 ; 45),[45 ; 50),[50 ; 55),[55 ; 60),[60 ; 65),[65 ; 70).
		\]
		\item Mốt của mẫu số liệu ghép nhóm trên là bao nhiêu?
	\end{enumerate}
	\loigiai{
		\begin{enumerate}
			\item Ta có bảng tần số ghép nhóm của mẫu số liệu trên như sau:
			\begin{center}
				\begin{tabular}{|c|c|c|c|}
					\hline
					\textbf{Nhóm} & \textbf{Giá trị đại diện} & \textbf{Tần số} & \textbf{Tần số tích luỹ}\\ 
					\hline
					$\left[40;45\right)$ & $42{,}5$ & $4$ & $4$\\
					$\left[45;50\right)$ & $47{,}5$ & $11$ & $15$\\
					$\left[50;55\right)$ & $52{,}5$ & $7$ & $22$\\
					$\left[55;60\right)$ & $57{,}5$ & $8$ & $30$\\
					$\left[60;65\right)$ & $62{,}5$ & $8$ & $38$\\
					$\left[65;70\right)$ & $67{,}5$ & $2$ & $40$\\
					\hline
					&  & $n = 40$ &\\
					\hline
				\end{tabular}
			\end{center}
			
			\item Ta thấy: Nhóm $2$ ứng với nửa khoảng $\left[45;50\right)$ là nhóm có tần số lớn nhất với $u=45$, $g=5$, $n_2 = 11$. Nhóm $1$ có tần số $n_1 = 4$, nhóm $3$ có tần số $n_3 = 7$.\\
			Khi đó, mốt của mẫu số liệu là 
			\[
			M_o = 45 + \left( \dfrac{11 - 4}{2\cdot 11 - 4 - 7} \right) \cdot 5 \approx 48{,}2\text{ (km/h)}.
			\]
		\end{enumerate}
	}
\end{bt}
\begin{bt}%[Tex hóa SGK CD, TVN-223]%[1C5B1-5]
	Mẫu số liệu sau ghi lại cân nặng của $30$ bạn học sinh (đơn vị: kilôgam):
	\[
	\begin{array}{cccccccccc}
	17 & 40 & 39 & 40{,}5 & 42 & 51 & 41{,}5 & 39 & 41 & 30\\
	40 & 42 & 40{,}5 & 39{,}5 & 41 & 40{,}5 & 37 & 39{,}5 & 40 & 41\\
	38{,}5 & 39{,}5 & 40 & 41 & 39 & 40{,}5 & 40 & 38{,}5 & 39{,}5 & 41{,}5
	\end{array}
	\]
	\begin{enumerate}
		\item Lập bảng tần số ghép nhóm cho mẫu số liệu trên có tám nhóm ứng với tám nửa khoảng:
		\[
		[15 ; 20),[20 ; 25),[25 ; 30),[30 ; 35),[35 ; 40),[40 ; 45),[45 ; 50),[50 ; 55).
		\]
		
		\item Mốt của mẫu số liệu ghép nhóm trên là bao nhiêu?
	\end{enumerate}
	\loigiai{
		\begin{enumerate}
			\item Ta có bảng tần số ghép nhóm của mẫu số liệu trên như sau:
			\begin{center}
				\begin{tabular}{|c|c|c|c|}
					\hline
					\textbf{Nhóm} & \textbf{Giá trị đại diện} & \textbf{Tần số} & \textbf{Tần số tích luỹ}\\ 
					\hline
					$\left[15;20\right)$ & $17{,}5$ & $1$ & $1$\\
					$\left[20;25\right)$ & $22{,}5$ & $0$ & $1$\\
					$\left[25;30\right)$ & $27{,}5$ & $0$ & $1$\\
					$\left[30;35\right)$ & $32{,}5$ & $1$ & $2$\\
					$\left[35;40\right)$ & $37{,}5$ & $10$ & $12$\\
					$\left[40;45\right)$ & $42{,}5$ & $17$ & $29$\\
					$\left[45;50\right)$ & $47{,}5$ & $0$ & $29$\\
					$\left[50;55\right)$ & $52{,}5$ & $1$ & $30$\\
					\hline
					&  & $n = 30$ &\\
					\hline
				\end{tabular}
			\end{center}
		
			\item Ta thấy: Nhóm $6$ ứng với nửa khoảng $\left[40;45\right)$ là nhóm có tần số lớn nhất với $u=40$, $g=5$, $n_6 = 17$. Nhóm $5$ có tần số $n_5 = 10$, nhóm $7$ có tần số $n_7 = 0$.\\
			Khi đó, mốt của mẫu số liệu là 
			\[
			M_o = 40 + \left( \dfrac{17 - 10}{2\cdot 17 - 10 - 0} \right) \cdot 5 \approx 41{,}5\text{ (kg)}.
			\]
		\end{enumerate}
	}
\end{bt}
\begin{bt}%[Tex hóa SGK CD, TVN-223]%[1C5B1-5]
	\immini
	{
		Bảng bên cho ta bảng tần số ghép nhóm số liệu thống kê chiều cao của $40$ mẫu cây ở một vườn thực vật (đơn vị: centimét).
		
		
			 Mốt của mẫu số liệu ghép nhóm trên là bao nhiêu?
		
	}
	{
		\begin{tabular}{|c|c|c|}
			\hline
			\textbf{Nhóm} & \textbf{Tần số} & \textbf{Tần số tích luỹ}\\ 
			\hline
			$\left[30;40\right)$ & $4$ & $4$\\
			$\left[40;50\right)$ & $10$ & $14$\\
			$\left[50;60\right)$ & $14$ & $28$\\
			$\left[60;70\right)$ & $6$ & $34$\\
			$\left[70;80\right)$ & $4$ & $38$\\
			$\left[80;90\right)$ & $2$ & $40$\\
			\hline
			& $n = 40$ &\\
			\hline
		\end{tabular}
	}
	\loigiai{
	Ta thấy: Nhóm $3$ ứng với nửa khoảng $\left[50;60\right)$ là nhóm có tần số lớn nhất với $u=50$, $g=10$, $n_3 = 14$. Nhóm $2$ có tần số $n_2 = 10$, nhóm $4$ có tần số $n_4 = 6$.\\
			Khi đó, mốt của mẫu số liệu là 
			\[
			M_o = 50 + \left( \dfrac{14 - 10}{2\cdot 14 - 10 - 6} \right) \cdot 10 \approx 53{,}3\text{ (cm)}.
			\]
	
	}
\end{bt}
\subsection{Bài tập trắc nghiệm}
\Opensolutionfile{ans}[ans/ansOC3]
% \begin{ex}%[1C5Y1-1]
% 	Một cuộc khảo sát đã tiến hành xác định tuổi (theo năm) của $120$ chiếc ô-tô. Kết quả điều tra được cho trong bảng sau
% 	\begin{center}
% 		\begin{tabular}{ |c|c|c|c|c|c|c| }
% 			\hline
% 			Nhóm & $[0;4)$ & $[4;8)$ & $[8;12)$ & $[12;16)$ & $[16;20)$ &  \\
% 			\hline
% 			Tần số & $23$ & $25$ & $27$ & $26$ & $19$ & $n=120$ \\
% 			\hline
% 		\end{tabular}
% 	\end{center}
% 	Mẫu số liệu trên có bao nhiêu nhóm?
% 	\choice
% 	{$10$}
% 	{$11$}
% 	{\True $5$}
% 	{$7$}
% 	\loigiai{
% 		Từ bảng, ta thấy mẫu số liệu trên có $5$ nhóm.
% 	}
% \end{ex}
% \begin{ex}%[1C5Y1-1]
% 	Một cuộc khảo sát đã tiến hành xác định tuổi (theo năm) của $120$ chiếc ô-tô. Kết quả điều tra được cho trong bảng sau
% 	\begin{center}
% 		\begin{tabular}{ |c|c|c|c|c|c|c| }
% 			\hline
% 			Nhóm & $[0;4)$ & $[4;8)$ & $[8;12)$ & $[12;16)$ & $[16;20)$ &  \\
% 			\hline
% 			Tần số & $23$ & $25$ & $27$ & $26$ & $19$ & $n=120$ \\
% 			\hline
% 		\end{tabular}
% 	\end{center}
% 	Nhóm có tần số bằng $19$ là
% 	\choice
% 	{$[0;4)$}
% 	{$[8;12)$}
% 	{$[12;16)$}
% 	{\True $[16;20)$}
% 	\loigiai{
% 		Từ bảng, ta thấy nhóm có tần số bằng $19$ là $[16;20)$.
% 	}
% \end{ex}
\begin{ex}%[1C5Y1-1]
	Một cuộc khảo sát đã tiến hành xác định tuổi (theo năm) của $120$ chiếc ô-tô. Kết quả điều tra được cho trong bảng sau
	\begin{center}
		\begin{tabular}{ |c|c|c|c|c|c|c| }
			\hline
			Nhóm & $[0;4)$ & $[4;8)$ & $[8;12)$ & $[12;16)$ & $[16;20)$ &  \\
			\hline
			Tần số & $23$ & $25$ & $27$ & $26$ & $19$ & $n=120$ \\
			\hline
		\end{tabular}
	\end{center}
	Số ô-tô có độ tuổi dưới $12$ là
	\choice
	{\True $75$}
	{$27$}
	{$48$}
	{$26$}
	\loigiai{
		Từ bảng, ta thấy số ô-tô có độ tuổi dưới $12$ là $23+25+27=75$.
	}
\end{ex}
% \begin{ex}%[1C5Y1-1]
% 	Cho mẫu số liệu ghép nhóm sau
% 	\begin{center}
% 		\begin{tabular}{ |c|c|c|c|c|c|c|c| }
% 			\hline
% 			Thời gian & $[15;20)$ & $[20;25)$ & $[25;30)$ & $[30;35)$ & $[35;40)$ & $[40;45)$ & $[45;50)$ \\
% 			\hline
% 			Số nhân viên & $6$ & $14$ & $25$ & $37$ & $21$ & $13$ & $9$ \\
% 			\hline
% 		\end{tabular}
% 	\end{center}
% 	Tần số của nhóm $[15;20)$ là bao nhiêu?
% 	\choice
% 	{\True $6$}
% 	{$7$}
% 	{$14$}
% 	{$25$}
% 	\loigiai{
% 		Ta thấy tần số của nhóm $[15;20)$ là $6$.
% 	}
% \end{ex}
\begin{ex}%[1C5Y1-1]
	Khảo sát thời gian tập thể dục trong ngày của một số học sinh khối $11$ thu được mẫu số liệu ghép nhóm sau
	\begin{center}
		\begin{tabular}{ |c|c|c|c|c|c| }
			\hline
			Thời gian (phút)& $[0;20)$ & $[20;40)$ & $[40;60)$ & $[60;80)$ & $[80;100)$ \\
			\hline
			Số học sinh & $5$ & $9$ & $12$ & $10$ & $6$ \\
			\hline
		\end{tabular}
	\end{center}
	Giá trị đại diện của nhóm $[20;40)$ là
	\choice
	{$10$}
	{\True $30$}
	{$20$}
	{$40$}
	\loigiai{
		Giá trị đại diện của nhóm $[20;40)$ là $\dfrac{20+40}{2}=30$.
	}
\end{ex}
\begin{ex}%[1C5Y1-1]
	Doanh thu bán hàng trong $20$ ngày được lựa chọn ngẫu nhiên của một cửa hàng được ghi lại ở bảng sau (đơn vị: triệu đồng):
	\begin{center}
		\begin{tabular}{ |c|c|c|c|c|c| }
			\hline
			Doanh thu & $[5;7)$ & $[7;9)$ & $[9;11)$ & $[11;13)$ & $[13;15)$ \\
			\hline
			Số ngày & $2$ & $7$ & $7$ & $3$ & $1$ \\
			\hline
		\end{tabular}
	\end{center}
	Doanh thu bán hàng của cửa hàng trong ngày $A$ là $7$ triệu đồng thì được xếp vào nhóm nào?
	\choice
	{$[5;7)$}
	{\True $[7;9)$}
	{$[9;11)$}
	{$[13;15)$}
	\loigiai{
		Doanh thu bán hàng là $7$ triệu đồng thì được xếp vào nhóm $[7;9)$.
	}
\end{ex}
\begin{ex}%[1C5Y1-1]
	Doanh thu bán hàng trong $20$ ngày được lựa chọn ngẫu nhiên của một cửa hàng được ghi lại ở bảng sau (đơn vị: triệu đồng):
	\begin{center}
		\begin{tabular}{ |c|c|c|c|c|c| }
			\hline
			Doanh thu & $[5;7)$ & $[7;9)$ & $[9;11)$ & $[11;13)$ & $[13;15)$ \\
			\hline
			Số ngày & $2$ & $7$ & $7$ & $3$ & $1$ \\
			\hline
		\end{tabular}
	\end{center}
	Các nhóm có độ dài bằng
	\choice
	{\True $2$}
	{$3$}
	{$4$}
	{$5$}
	\loigiai{
		Các nhóm có độ dài bằng nhau, và bằng $2$.
	}
\end{ex}
\begin{ex}%[1C5B1-1]
	Cho bảng số liệu về khối lượng của $30$ củ khoai tây thu hoạch từ một thửa ruộng như hình bên dưới. Tần suất của lớp $[100;110)$ là bao nhiêu?
	\begin{center}
		\begin{tabular}{ |c|c|c|c|c|c| }
			\hline
			Lớp khối lượng (gam) & $[70;80)$ & $[80;90)$ & $[90;100)$ & $[100;110)$ & $[110;120]$ \\
			\hline
			Tần số & $3$ & $6$ & $12$ & $6$ & $3$ \\
			\hline
		\end{tabular}
	\end{center}
	\choice
	{\True $20\%$}
	{$10\%$}
	{$40\%$}
	{$90\%$}
	\loigiai{
		Tần suất ghép lớp $[100;110)$ là $\dfrac{6}{30}\cdot 100\%=20\%$.
	}
\end{ex}
\begin{ex}%[1C5B1-1]
	Cân nặng của $28$ học sinh nam lớp $11$ được cho ở bảng sau
	\begin{center}
		\begin{tabular}{ |c|c|c|c|c|c|c| }
			\hline
			Cân nặng & $[45;49)$ & $[49;53)$ & $[53;57)$ & $[57;61)$ & $[61;65)$\\
			\hline
			Số học sinh & $4$ & $5$ & $7$ & $7$ & $5$ \\
			\hline
		\end{tabular}
	\end{center}
	Tấn số tích lũy của nhóm $[49;53)$ là bao nhiêu?
	\choice
	{$5$}
	{$4$}
	{\True $9$}
	{$20$}
	\loigiai{
		Tần số tích lũy của nhóm $[49;53)$ là $4+5=9$.
	}
\end{ex}
% \begin{ex}%[1C5K1-1]
% 	Một thư viện thống kê sô người đến đọc sách vào buổi tối trong $30$ ngày của tháng vừa qua như sau
% 	\begin{center}
% 		\begin{tabular}{ cccccccccc }
% 			$26$ & $35$ & $68$ & $84$ & $33$ & $84$ & $62$ & $45$ & $57$ & $46$ \\
% 			$35$ & $29$ & $28$ & $50$ & $26$ & $34$ & $75$ & $74$ & $43$ & $49$ \\
% 			$54$ & $55$ & $83$ & $82$ & $81$ & $54$ & $27$ & $36$ & $41$ & $52$ \\
% 		\end{tabular}
% 	\end{center}
% 	Bạn An lập bảng tần số mẫu số liệu trên như sau
% 	\begin{center}
% 		\begin{tabular}{ |c|c|c|c|c|c|c|c| }
% 			\hline
% 			Nhóm & $[25;35)$ & $[35;45)$ & $[45;55)$ & $[55;65)$ & $[65;75]$ & $[75;85)$ & \\
% 			\hline
% 			Tần số & $7$ & $4$ & $7$ & $3$ & $3$ & $6$ & $n=30$ \\
% 			\hline
% 		\end{tabular}
% 	\end{center}
% 	Bạn Khuê lập bảng tần số mẫu số liệu trên như sau
% 	\begin{center}
% 		\begin{tabular}{ |c|c|c|c|c|c|c|c|c| }
% 			\hline
% 			Nhóm & $[23;31)$ & $[31;39)$ & $[39;47)$ & $[47;55)$ & $[55;63]$ & $[71;79)$ & $[79;87)$ & \\
% 			\hline
% 			Tần số & $5$ & $5$ & $4$ & $5$ & $3$ & $1$ & $2$ & $n=30$ \\
% 			\hline
% 		\end{tabular}
% 	\end{center}
% 	Hỏi bảng tần số của bạn nào đúng?
% 	\choice
% 	{Bảng tần số của bạn An}
% 	{\True Bảng tần số của bạn Khuê}
% 	{Cả hai bạn đều đúng}
% 	{Cả hai bạn đều sai}
% 	\loigiai{
% 		Từ các bảng tần số mẫu số liệu trên, ta thấy bảng tần số của bạn An sai ở hai nhóm $[35;45)$ (tần số đúng bằng $5$) và nhóm $[65;75)$ (tần số đúng bằng $2$).
% 	}
% \end{ex}
% \begin{ex}%[0D5Y3-1]
% 	Cho dãy số liệu thống kê: $21$, $23$, $24$, $25$, $22$, $20$. Số trung bình cộng của các số liệu thống kê đã cho là
% 	\choice
% 	{$23{,}5$}
% 	{$22$}
% 	{\True $22{,}5$}
% 	{$14$}
% 	\loigiai{
% 		Ta có $\overline{x}=\dfrac{21+23+24+25+22+20}{6}=22{,}5$.
% 	}
% \end{ex}
% \begin{ex}%[0D5Y3-1]
% 	Điều tra về số con của $30$ gia đình ở khu vực, kết quả thu được như sau
% 	\begin{center}
% 		\begin{tabular}{|c|c|c|c|c|c|c|}
% 			\hline 
% 			Giá trị (số con) & $0$ &$1$ & $2$ &$3$ & $4$ & Tổng \\ 
% 			\hline 
% 			Tần số & $1$ & $7$ & $15$ & $5$ & $2$ & $N=30$ \\ 
% 			\hline 
% 		\end{tabular}
% 	\end{center}
% 	Tìm số trung bình $\overline{x}$ của mẫu số liệu trên.
% 	\choice
% 	{\True $\overline{x}=2$}
% 	{$\overline{x}=1$}
% 	{$\overline{x}=1{,}5$}
% 	{$\overline{x}=3$}
% 	\loigiai{
% 		Ta có $\overline{x}=\dfrac{0\cdot 1+1\cdot 7+2\cdot 15+3\cdot 5+4\cdot 2}{30}=2$.
% 	}
% \end{ex}
% \begin{ex}%[0D5Y3-1]
% 	Điểm môn Toán của lớp $11A$ được cho trong bảng sau
% 	\begin{center}
% 		\begin{tabular}{|l|c|c|c|c|c|c|c|c|c|c|}
% 			\hline
% 			Điểm&$1$&$2$&$3$&$4$&$5$&$6$&$7$&$8$&$9$&$10$\\
% 			\hline
% 			Tần số&$2$&$1$&$4$&$3$&$9$&$7$&$5$&$5$&$3$&$1$\\
% 			\hline
% 		\end{tabular}
% 	\end{center}
% 	Điểm trung bình của các học sinh lớp $10A$ là bao nhiêu?
% 	\choice
% 	{$5$}
% 	{$5{,}5$}
% 	{$5{,}6$}
% 	{\True $5{,}7$}
% 	\loigiai{
% 		Điểm trung bình lớp $11A$ là
% 		$$\overline{x}= \dfrac{1 \cdot 2 + 2 \cdot 1 + \cdots + 10 \cdot 1}{40}= \dfrac{227}{40} \approx 5{,}7.$$
% 	}
% \end{ex}
% \begin{ex}%[0D5Y3-1]
% 	Kết quả điểm kiểm tra môn Toán của $40$ học sinh lớp $10A$ được trình bày ở bảng sau:
% 	\begin{center}
% 		\begin{tabular}{|c|c|c|c|c|c|c|c|c|}
% 			\hline 
% 			Điểm&$4$  &$5$  &$6$  &$7$  &$8$  &$9$  &$10$  &Cộng  \\ 
% 			\hline 
% 			Tần số&$2$  &$8$  &$7$  &$10$  &$8$  &$3$  &$2$  &$40$  \\ 
% 			\hline 
% 		\end{tabular} 
% 	\end{center}
% 	Tính số trung bình cộng của bảng trên. (làm tròn kết quả đến một chữ số thập phân).
% 	\choice
% 	{\True $6{,}8$}
% 	{$6{,}4$}
% 	{$7{,}0$}
% 	{$6{,}7$}
% 	\loigiai{
% 		Ta có $\overline{x}=\dfrac{4\cdot 2+ 5\cdot 8+ 6\cdot 7+ 7\cdot 10+ 8\cdot 8+ 9\cdot 3+ 10\cdot 2}{40}\approx 6{,}8$.
% 	}
% \end{ex}
\begin{ex}%[0D5B3-1]
	Điểm môn Toán của lớp $10$A được cho như bảng sau
	\begin{center}
		\begin{tabular}{|c|c|c|c|c|c|}
			\hline
			Điểm &$[0;2)$& $[2;4)$& $[4;6)$& $[6;8)$& $[8;10)$\\\hline
			Tần số& $3$& $5$& $12$& $12$& 8\\ \hline
		\end{tabular}
	\end{center}
	Điểm trung bình của các học sinh lớp $10$A là bao nhiêu?
	\choice
	{$5$}
	{\True $5{,}85$}
	{$5{,}65$}
	{$5{,}45$}	
	\loigiai{
		Điểm trung bình $\overline{x}=\dfrac{1\cdot 3+3\cdot 5+5\cdot 12+7\cdot 12+9\cdot 8}{40}=5{,}85$.		
	}
\end{ex}
\begin{ex}%[0D5B3-1]
	Cho bảng phân bố tần số ghép lớp
	\begin{center}
		\begin{tabular}{|l|c|c|c|c|}
			\hline
			{Lớp các giá trị $x$}&{[8; 10)}&{[10; 12)}&{[12; 14]}&{Cộng}\\
			\hline
			{Tần số $n_i$}&{15}&{30}&{55}&{100}\\
			\hline
		\end{tabular}	
	\end{center}
	Số trung bình của các giá trị trong bảng trên là
	\choice
	{$9$}
	{$13$}
	{$11$}
	{\True $11{,}8$}	
	\loigiai{
		Giá trị đại diện của lớp $\left[8; 10\right)$: $c_1=\dfrac{8 + 10}{2}=9$.\\ 
		Giá trị đại diện của lớp $\left[10; 12\right)$: $c_2=\dfrac{10 + 12}{2}=11$.\\ 
		Giá trị đại diện của lớp $\left[12; 14\right)$: $c_3=\dfrac{12 + 14}{2}=13$.\\ 
		Vậy số trung bình cộng $\overline{x}=\dfrac{9\cdot 15 + 11\cdot 30 + 13\cdot 55}{15 + 30 + 55}=\dfrac{59}{5}$.
	}
\end{ex}
\begin{ex}%[0D5B3-1]
	Kết quả khảo sát cân nặng của $25$ quả cam ở lô hàng $A$ được cho như sau
	\begin{center}
		\begin{tabular}{|l|c|c|c|c|c|}
			\hline
			Cân nặng (g) & $[150;155)$ & $[155;160)$ & $[160;165)$ & $[165;170)$ & $[170;175)$ \\
			\hline
			Số quả cam & $2$ & $6$ & $12$ & $4$ & $1$ \\
			\hline
		\end{tabular}	
	\end{center}
	Tính cân nặng trung bình của mỗi quả cam ở lô hàng $A$.
	\choice
	{\True $161{,}7$ (g)}
	{$161{,}7$ (kg)}
	{$155$ (g)}
	{$160$ (kg)}
	\loigiai{
		Ta có giá trị đại diện của các nhóm lần lượt là $152{,}5$; $157{,}5$; $162{,}5$; $167{,}5$; $172{,}5$.\\
		Vậy cân nặng trung bình của mỗi quả cam là
		$$\overline{x}=\dfrac{152{,}5\cdot 2+ 157{,}5\cdot 6+162{,}5\cdot 12+167{,}5\cdot 4+172{,}5\cdot 1}{25}=161{,}7 \text{(g).}$$
	}
\end{ex}
% \begin{ex}%[0D5K3-1]
% 	Ba nhóm học sinh gồm $10$ người, $15$ người, $25$ người. Khối lượng trung bình của mỗi nhóm lần lượt là $50$ kg, $38$ kg, $40$ kg. Khối lượng trung bình của cả ba nhóm học sinh là
% 	\choice
% 	{\True $41{,}4$ kg}
% 	{$42{,}4$ kg}
% 	{$26$ kg}
% 	{$37$ kg}
% 	\loigiai{
% 		Tổng khối lượng nhóm thứ nhất là $50\cdot 10=500$ (kg).\\
% 		Tổng khối lượng nhóm thứ hai là $38\cdot 15=570$ (kg).\\
% 		Tổng khối lượng nhóm thứ ba là $40\cdot 25=1000$ (kg).\\
% 		Tổng khối lượng cả ba nhóm là $500+570+1000=2070$ (kg).\\
% 		Tổng số người cả ba nhóm là $10+15+25=50$ (người).\\
% 		Khối lượng trung bình của cả ba nhóm học sinh là $\dfrac{2070}{50}=41{,}4$ (kg).
% 	}
% \end{ex}
\begin{ex}%[0D5K3-1]
	Sau một kì thi học sinh giỏi Toán, người ta thống kê kết quả (thang điểm $20$) và thu được bảng tần số sau
	\begin{center}
		\begin{tabular}{|l|c|c|c|c|}
			\hline
			Lớp điểm & $[6;10]$ & $[11;15]$ & $[16;20]$ & Cộng\\
			\hline
			Tần số & $22$ & $12$ & $6$ & $40$ \\
			\hline
		\end{tabular}
	\end{center}
	Nếu những học sinh chỉ cần đạt điểm trung bình của bảng điểm trên đều được nhận Giấy Khen của ban tổ chức, thì số học sinh được nhận Giấy Khen là bao nhiêu?
	\choice
	{$11$}
	{\True $18$}
	{$12$}
	{$6$}
	\loigiai{
		Ta lập lại bảng với thêm dòng giá trị đại diện
		\begin{center}
			\begin{tabular}{|l|c|c|c|c|}
				\hline
				Lớp điểm & $[6;10]$ & $[11;15]$ & $[16;20]$ & Cộng\\
				\hline
				Giá trị đại diện & $8$ & $13$ & $18$ & \\
				\hline
				Tần số & $22$ & $12$ & $6$ & $40$ \\
				\hline
			\end{tabular}
		\end{center}
		Điểm trung bình là $\overline{x}=\dfrac{22\cdot 8+12\cdot 13+6\cdot 18}{40}=11$.\\
		Vậy số học sinh được nhận thưởng là $12+6=18$ (học sinh).
	}
\end{ex}
% \begin{ex}%[0D5G3-1]
% 	Cho biết tình hình thu hoạch lúa vụ mùa năm $1980$ của ba hợp tác xã ở địa phương $V$ như sau
% 	\begin{center}
% 		\begin{tabular}{|l|c|c|c|}
% 			\hline
% 			Hợp tác xã & $A$ & $B$ & $C$ \\
% 			\hline
% 			Năng suất lúa (tạ/ha) & $40$ & $38$ & $36$ \\
% 			\hline
% 			Diện tích trồng lúa (ha) & $150$ & $130$ & $120$ \\
% 			\hline
% 		\end{tabular}	
% 	\end{center}
% 	Hãy tính năng suất lúa trung bình của vụ mùa năm $1980$ trong toàn bộ ba hợp tác xã kể trên.
% 	\choice
% 	{\True $38{,}15$ tạ/ha}
% 	{$38{,}05$ tạ/ha}
% 	{$38{,}10$ tạ/ha}
% 	{$38{,}20$ tạ/ha}
% 	\loigiai{
% 		Sản lượng lúa của hợp tác xã $A$ là $40\cdot 150=6000$ (tạ).\\
% 		Sản lượng lúa của hợp tác xã $B$ là $38\cdot 130=4940$ (tạ).\\
% 		Sản lượng lúa của hợp tác xã $C$ là $36\cdot 120=4320$ (tạ).\\
% 		Tổng sản lượng lúa của cả ba hợp tác xã là $6000+4940+4320=15260$ (tạ).\\
% 		Tổng diện tích trồng lúa của cả ba hợp tác xã là $150+130+120=400$ (ha).\\
% 		Vậy năng suất lúa trung bình của cả ba hợp tác xã là $\dfrac{15260}{400}=38{,}15$ tạ/ha.\\
% 		\textbf{\underline{Lưu ý}:} Không thể tính năng suất trung bình bằng cách $\dfrac{40+38+36}{3}=38$ (tạ/ha), vì khi chênh lệch diện tích càng lớn thì số trung bình càng không chính xác.
% 	}
% \end{ex}
% \begin{ex}%[0D5Y3-2]
% 	Tiền lương hàng tháng của $7$ nhân viên trong một công ty du lịch là $650$, $840$, $690$, $2500$, $720$, $670$, $3000$ (đơn vị: nghìn đồng). Tìm số trung vị của các số liệu thống kê đã cho.
% 	\choice
% 	{$690$}
% 	{$2500$}
% 	{\True $720$}
% 	{$670$}
% 	\loigiai{ 
% 		Sắp xếp thứ tự các số liệu thống kê, ta thu được dãy tăng các số liệu như sau: $650$, $670$, $690$, $720$, $840$, $2500$, $3000$ (nghìn đồng).\\
% 		Ta có số các số liệu thống kê là $n=7=2\cdot 3+1$ nên số trung vị là $M_e=x_4=720$.
% 	}
% \end{ex}
% \begin{ex}%[0D5Y3-2]
% 	Điểm học kì một của một học sinh được cho bởi bảng số liệu sau (Đơn vị: điểm)
% 	\begin{center}
% 		\begin{tabular}{|c|c|c|c|c|c|c|c|c|}
% 			\hline
% 			5& 6 &6&7& 7 &8 &8& 8,5&9\\
% 			\hline
% 		\end{tabular}
% 	\end{center}
% 	Số trung vị của bảng nói trên là
% 	\choice
% 	{$6$}
% 	{$9$}
% 	{\True $7$}
% 	{$8$}
% 	\loigiai{ Ta có $N=9$ là số lẻ. Số liệu thứ $\dfrac{N+1}{2} = 5$ là số trung vị.\\
% 		Do đó số trung vị là $M_e = 7$ (điểm).
% 	}
% \end{ex}
% \begin{ex}%[0D5Y3-2]
% 	Điều tra số học sinh giỏi khối $10$ của $15$ trường cấp ba trên địa bản tỉnh $A$, ta được bảng số liệu như sau
% 	\begin{center}
% 		\begin{tabular}{|c|c|c|c|c|c|c|c|c|c|c|c|c|c|c|}
% 			\hline
% 			22& 29 &29&29& 30 &31 &32& 32&33 &34 &34 &35 &35 &35 &36  \\
% 			\hline	
% 		\end{tabular}
% 	\end{center}
% 	Số trung vị của bảng nói trên là
% 	\choice
% 	{\True $8$}
% 	{$9$}
% 	{$6$}
% 	{$7$}
% 	\loigiai{ Ta có $N=15$ là số lẻ $\Rightarrow$ số liệu thứ $\dfrac{15+1}{2}=8$ là số trung vị.\\
% 		Vậy số trung vị là $M_e = 8$.
% 	}
% \end{ex}
% \begin{ex}%[0D5Y3-2]
% 	Cho mẫu số liệu thống kê $\{6;4;4;1;9;10;7\}$. Số liệu trung vị của mẫu số liệu thống kê trên là
% 	\choice
% 	{$1$}
% 	{\True $6$}
% 	{$4$}
% 	{$10$}
% 	\loigiai{
% 		Sắp xếp thành dãy không giảm: $1$, $4$, $4$, $6$, $7$, $9$, $10$.\\
% 		Từ dãy trên ta có số trung vị là số $6$ trong dãy trên.
% 	}
% \end{ex}
% \begin{ex}%[0D5B3-2]
% 	Điểm kiểm tra môn Toán của $10$ học sinh được cho như sau: $6$; $7$; $7$; $6$; $7$; $8$; $8$; $7$; $9$; $9$. Số trung vị của mẫu số liệu trên là	
% 	\choice
% 	{$6$}
% 	{\True $7$}
% 	{$8$}
% 	{$9$}
% 	\loigiai{
% 		Ta sắp xếp số liệu theo thứ tự không giảm như sau: $6$; $6$; $7$; $7$; $7$; $7$; $8$; $8$; $9$; $9$.\\
% 		Dãy số trên có tất cả $10$ giá trị, và $2$ giá trị chính giữa bằng $7$.\\
% 		Vậy số trung vị của mẫu số liệu trên là $\dfrac{7+7}{2}=7$.
% 	}
% \end{ex}
% \begin{ex}%[0D5B3-2]
% 	Một cửa hàng dép da đã thống kê cỡ dép của một số khách hàng nam cho kết quả như sau: $39$; $38$; $39$; $40$; $41$; $41$; $43$; $37$; $38$; $40$; $43$; $41$; $42$; $41$; $42$. Tìm trung vị của mẫu số liệu trên.
% 	\choice
% 	{$37$}
% 	{$39$}
% 	{\True $41$}
% 	{$43$}
% 	\loigiai{
% 		Ta sắp xếp số liệu theo thứ tự không giảm: $37$; $38$; $38$; $39$; $39$; $40$; $40$; $41$; $41$; $41$; $41$; $42$; $42$; $42$; $43$.\\
% 		Vì $n=15$ là số lẻ nên số trung vị là số chính giữa của dãy số liệu.\\
% 		Vậy trung vị là $M_e=41$.
% 	}
% \end{ex}
% \begin{ex}%[0D5B3-2]
% 	Điều tra số học sinh của $30$ lớp học, ta được bảng số liệu như sau
% 	\begin{center}
% 		\begin{tabular}{|c|c|c|c|c|c|c|c|c|c|c|c|c|c|c|}
% 			\hline
% 			35& 39 &39&40& 40 &41 &41& 41&41 &44 &44 &45 &45 &45 &46  \\
% 			\hline
% 			48 &48 &48&48& 49 &49 &49&49 &49 &49 &50 &50 &50 &50 &51  \\
% 			\hline	
% 		\end{tabular}
% 	\end{center}
% 	Số trung vị của bảng nói trên là
% 	\choice
% 	{$46$}
% 	{$49$}
% 	{\True $47$}
% 	{$48$}
% 	\loigiai{ Ta có $N=30$ là số chẵn. Số liệu thứ $15$ và $16$ lần lượt là $46$, $48$ là số trung vị.\\
% 		Vậy số trung vị là $M_e = \dfrac{46+48}{2}=47$ (học sinh).
% 	}
% \end{ex}
% \begin{ex}%[0D5B3-2]
% 	Cho bảng phân bố tần số
% 	\begin{center}
% 		\begin{tabular}{|l|c|c|c|c|c|c|}
% 			\hline
% 			Tuổi & $18$ & $19$ & $20$ & $21$ & $22$ & Cộng \\
% 			\hline
% 			Tần số & $10$ & $50$ & $70$ & $29$ & $10$ & $169$ \\
% 			\hline	
% 		\end{tabular}
% 	\end{center}
% 	Số trung vị của bảng phân bố tần số đã cho là
% 	\choice
% 	{$18$ tuổi}
% 	{\True $20$ tuổi}
% 	{$19$ tuổi}
% 	{$21$ tuổi}
% 	\loigiai{
% 		Sau khi sắp xếp các tuổi trên thành dãy không giảm, do có $169$ số nên số trung vị là số thứ $85$ trong dãy trên.\\
% 		Mà số thứ $85$ trong dãy là $20$. Vậy $M_e=20$.
% 	}
% \end{ex}
% \begin{ex}%[0D5B3-2]
% 	Để khảo sát kết quả thi tuyển sinh môn Toán trong kì thi tuyển sinh Đại học năm vừa qua của trường $A$, người điều tra chọn một mẫu gồm $100$ học sinh tham gia kì thi tuyển sinh đó. Điểm môn Toán (thang điểm $10$) của các học sinh này được cho ở bảng phân bố tần số sau đây
% 	\begin{center}
% 		\begin{tabular}{|l|c|c|c|c|c|c|c|c|c|c|c|c|c|}
% 			\hline
% 			Điểm & $0$ & $1$ & $2$ & $3$ & $4$ & $5$ & $6$ & $7$ & $8$ & $9$ & $10$ &  \\
% 			\hline
% 			Tần số & $1$ & $1$ & $3$ & $5$ & $8$ & $13$ & $19$ & $24$ & $14$ & $10$ & $2$ & $n=100$ \\
% 			\hline	
% 		\end{tabular}
% 	\end{center}
% 	Số trung vị của mẫu số liệu trên.
% 	\choice
% 	{\True $M_e=6{,}5$}
% 	{$M_e=7{,}5$}
% 	{$M_e=5{,}5$}
% 	{$M_e=6$}
% 	\loigiai{
% 		Do kích thước mẫu $n=100$ là một số chẵn nên số trung vị là trung bình cộng của hai giá trị đứng thứ $\dfrac{n}{20}=50$ và $\dfrac{n}{2}+1=51$.\\
% 		Do đó $M_e=\dfrac{6+7}{2}=6{,}5$.
% 	}
% \end{ex}
% \begin{ex}%[0D5B3-2]
% 	Số áo bán được trong một quý ở một cửa hàng bán áo sơ-mi nam được cho trong bảng sau
% 	\begin{center}
% 		\begin{tabular}{|l|c|c|c|c|c|c|c|c|}
% 			\hline
% 			Cỡ số & $36$ & $37$ & $38$ & $39$ & $40$ & $41$ & $42$ & Cộng \\
% 			\hline
% 			Số áo bán được & $13$ & $45$ & $126$ & $110$ & $126$ & $40$ & $5$ & $465$ \\
% 			\hline	
% 		\end{tabular}
% 	\end{center}
% 	Hãy tìm số trung vị của các số liệu thống kê trên.
% 	\choice
% 	{$37$}
% 	{$38$}
% 	{\True $39$}
% 	{$40$}
% 	\loigiai{
% 		Ta sắp xếp dãy số áo bán được theo dãy không giảm
% 		$$36, 36, 36, \ldots, 36, 37, 37, \ldots, 37, 38, 38, \ldots, 38, \ldots, 42, 42.$$
% 		Dãy trên gồm $465$ số nên số trung vị là số thứ $233$.\\
% 		Mà số thứ $233$ là số $39$. Vậy $M_e=233$.
% 	}
% \end{ex}
\begin{ex}
	Cho bảng tần số về cân nặng của 180 người dân trong một xã như sau: (đơn vị: kg)
	\begin{center}
	\begin{tabular}{|c|c|c|}
	\hline
	\textbf{Nhóm} & \textbf{Tần số} & \textbf{Tần số tích luỹ}\\ 
	\hline
	$\left[0;10\right)$ & $6$ & $6$\\
	$\left[10;20\right)$ & $15$ & $21$\\
	$\left[20;30\right)$ & $37$ & $58$\\
	$\left[30;40\right)$ & $48$ & $106$\\
	$\left[40;50\right)$ & $22$ & $128$\\
	$\left[50;60\right)$ & $29$ & $157$\\
	$\left[60;70\right)$ & $23$ & $180$\\

	\hline
	& $n = 180$ &\\
	\hline
\end{tabular}	
	\end{center}
	Tứ phân vị thứ nhất của mẫu số liệu trên là
	\choice
	{$56{,}486$ kg}
	{\True $26{,}486$ kg}
	{$25{,}496$ kg}
	{$36{,}486$ kg}
	\loigiai{
	Số phần tử của mẫu là $n=180$  và $\dfrac{n}{4}=\dfrac{\cdot 180}{4}=45$.\\ Ta có  $21<135<58$ nên nhóm $3$ là nhóm  đầu tiên có   tần số tích luỹ  lớn hơn hoặc bằng $45$.\\
	Xét nhóm $3$ là  nhóm $\left[20;30\right)$ có $s=20$, $h=10$, $n_3=37$, nhóm $2$ là nhóm có $cf_2=21$.\\
	Vậy $Q_1=20+\dfrac{45-21}{37}\cdot 10\approx 26{,}49$ (kg).
}
	\end{ex}

\begin{ex}
	Cho bảng tần số chiều cao của 46 học sinh nam của khối lớp $11$ như sau
	\begin{center}
		\begin{tabular}{|c|c|}
			\hline
			\textbf{Nhóm} & \textbf{Tần số} \\
			\hline
			$\left[155;160\right)$ & $3$ \\
			$\left[160;165\right)$ & $18$ \\
			$\left[165;170\right)$ & $10$ \\
			$\left[170;175\right)$ & $15$ \\
			
			\hline
			& $n = 46$ \\
			\hline
		\end{tabular}	
	\end{center}
	Xác định tứ phân vị thứ nhất của mẫu số liệu trên
\choice
{$161{,}36$}
{$161{,}63$}
{\True $162{,}36$}
{$162{,}63$}
\loigiai{
	Số phần tử của mẫu là $n=46$; $\dfrac{n}{4}=11{,}5$.\\
		Ta có $cf_1=3$, $cf_2=3+18=21$ và $6<11{,}5<21$ nên nhóm $2$ là nhóm đầu tiên có   tần số tích luỹ  lớn hơn hoặc bằng $11{,}5$.\\
		Xét nhóm $2$ là nhóm $\left[160;165\right)$, ta có $s=160$, $h=5$, $n_6=21$; nhóm $1$ có tần số tích luỹ bằng $6$.\\
		Vậy $Q_1=160+\dfrac{11{,}5-3}{18}\cdot 5=162{,}36$.
}
\end{ex}
\begin{ex}
	Cho bảng tần số chiều cao của 46 học sinh nam của khối lớp $11$ như sau
	\begin{center}
		\begin{tabular}{|c|c|}
			\hline
			\textbf{Nhóm} & \textbf{Tần số} \\
			\hline
			$\left[155;160\right)$ & $3$ \\
			$\left[160;165\right)$ & $18$ \\
			$\left[165;170\right)$ & $10$ \\
			$\left[170;175\right)$ & $15$ \\
			
			\hline
			& $n = 46$ \\
			\hline
		\end{tabular}	
	\end{center}
	Xác định tứ phân vị thứ ba của mẫu số liệu trên
	\choice
	{$162{,}36$}
	{$166{,}5$}
	{ $166$}
	{\True $171{,}16$}
	\loigiai{
		Số phần tử của mẫu là $n=46$; $\dfrac{3n}{4}=34{,}5$.\\
		Ta có $cf_3=3+18+10=31$, $cf_4=31+15=46$ và $31<34{,}5<46$ nên nhóm $4$ là nhóm đầu tiên có   tần số tích luỹ  lớn hơn hoặc bằng $34{,}5$.\\
		Xét nhóm $4$ là nhóm $\left[170;175\right)$, ta có $t=170$, $l=5$, $n_4=15$; nhóm $3$ có tần số tích luỹ bằng $31$.\\
		Vậy $Q_3=170+\dfrac{34{,}5-31}{15}\cdot 5=171{,}16$.
	}
\end{ex}
\begin{ex}
	\immini{Cho bảng tần số ghép nhóm số liệu thống kê  chiều cao  của $40$ mẫu cây ở  một vườn thực vật 	(đơn vị: centimét).\\
		Xác định tứ phân vị thứ hai của  số liệu ghép nhóm trên
\choice
{\True $56{,}43$}
{$56{,}34$}
{$46{,}43$}
{$36{,}43$}		

}{\begin{tabular}{|c|c|c|}
	\hline
	\textbf{Nhóm} & \textbf{Tần số} & \textbf{Tần số tích luỹ}\\ 
	\hline
	$\left[30;40\right)$ & $4$ & $4$\\
	$\left[40;50\right)$ & $10$ & $14$\\
	$\left[50;60\right)$ & $14$ & $28$\\
	$\left[60;70\right)$ & $6$ & $34$\\
	$\left[70;80\right)$ & $4$ & $38$\\
	$\left[80; 	90\right)$ & $2$ & $40$\\

	
	\hline
	& $n = 40$ &\\
	\hline
\end{tabular}}
\loigiai{
	Số phần tử của mẫu là $n=46$; $\dfrac{n}{2}=23$.\\
	Ta có $cf_2=14<23<cf_3=28$ nên nhóm $3$ là nhóm đầu tiên có tần số tích lũy lớn hơn hoặc bằng $23$.\\
	Xét nhóm $3$ là nhóm $[50;60)$ có $r=50$, $d=10$, $n_3=14$ và nhóm $2$ là nhóm $\left[40;50\right)$ có $cf_2=14$.\\
	Do đó $Q_2=50+\dfrac{23-14}{14}\cdot 10=56{,}43$.
}
\end{ex}
\begin{ex}
Một bảng xếp hạng đã tính điềm chuẩn hoá cho chỉ số nghiên cứu của một số trường đại học ở
Việt Nam và thu được kết quả sau
\begin{center}
\begin{tabular}{|c|c|c|c|c|c|c|}
	\hline
	\textbf{Điểm} & Dưới $20$  &  $[20;30)$&  $[30;40)$ &  $[40;60)$ &  $[60;80)$ &  $[80;100)$\\
	\hline
	Số điểm & $4$ & $19$&$6$&$2$&$3$&$1$\\

	
	
	\hline

\end{tabular}	
\end{center}
Xác định điểm ngưỡng đề đưa ra danh sách $25$\% trường đại học có chỉ số nghiên cứu tốt nhất Việt Nam.	
\choice
{$25{,}26$}
{\True $35{,}42$}
{$45{,}35$}
{$45{,}42$}	
\loigiai{
Điểm ngưỡng để đưa ra danh sách $25$\% trường đại học có chỉ số nghiên cứu tốt nhất Việt Nam là tứ phân
vị thứ ba.\\
Ta có  $n=35$ và $\dfrac{3n}{4}=26{,}25$.\\
Do $cf_2=4+19=23<26{,}25<cf_3=23+6=29$ nên nhóm $[30;40])$ là nhóm đầu tiên có tần số tích lũy lớn hơn hoặc bằng $23{,}25$.\\
Nhóm $[30;40])$ có $r=30$, $d=10$, $n_3=6$; nhóm $2$ có $cf_2=23$. Do đó
$$Q_3=30+\dfrac{26{,}25-23}{6}\cdot 10\approx 35{,}41.$$ 
Vậy để đưa ra danh sách $25$\% trường đại học có chỉ số nghiên cứu tốt nhất Việt Nam ta lấy các trường có
điểm chuẩn hóa trên $35{,}42$.
}
\end{ex}
% \begin{ex}%1%[1K3B9-3]
% 	Điểm thi môn Toán (thang điểm 100, điểm được làm tròn đến 1) của 60 thí sinh được cho trong bảng sau
% 	\begin{center}
% 		\begin{tabular}{|l|c|c|c|c|c|}
% 			\hline Điểm & $[0-9,5)$ & $[9,5-19,5)$ & $[19,5-29,5)$ & $[29,5-39,5)$ & $[39,5-49,5)$ \\
% 			\hline Số thí sinh & $1 $& $2$ & $4$ & $6$ & $15$ \\
% 			\hline Điểm & $[49,5-59,5)$ & $[59,5-69,5)$ & $[69,5-79,5)$ & $[79,5-89,5)$ & $[89,5-99,5)$ \\
% 			\hline Số thí & $12$ & $10$ & $6$ & $3$ & $1$ \\
% 			\hline
% 		\end{tabular}    
% 	\end{center}
% 	Tìm  tứ phân vị thứ hai của mẫu số liệu.
% 	\choice
% 	{\True $Q_2\approx 51,17$}
% 	{$Q_2\approx 51,67$}
% 	{$Q_2\approx 49,5$}
% 	{$Q_2\approx 41,3$}
% 	\loigiai{Cỡ mẫu là $n=60$.\\
% 		Tứ phân vị thứ nhất $Q_2$ là $\dfrac{x_{30}+x_{31}}{2}$. Do $x_{30}$, $x_{31}$ đều thuộc nhóm $[49,5 ; 59,5)$ nên nhóm này chứa $Q_2$. \\Do đó, $p=6 ; \;a_6=49,5 ;\; m_6=12 ; \;m_1+\ldots+m_5=28, \;a_7-a_6=10$ và ta có
% 		$$
% 		Q_2=49,5+\dfrac{\frac{60}{2}-28}{12}\cdot 10\approx51,17.
% 		$$
% 	}
% \end{ex}
% %2
% \begin{ex}%[1K3B9-3]
% 	Điểm thi môn Toán (thang điểm 100, điểm được làm tròn đến 1) của $60$ thí sinh được cho trong bảng sau
% 	\begin{center}
% 		\begin{tabular}{|l|c|c|c|c|c|}
% 			\hline Điểm & $[0-9,5)$ & $[9,5-19,5)$ & $[19,5-29,5)$ & $[29,5-39,5)$ & $[39,5-49,5)$ \\
% 			\hline Số thí sinh & $1 $& $2$ & $4$ & $6$ & $15$ \\
% 			\hline Điểm & $[49,5-59,5)$ & $[59,5-69,5)$ & $[69,5-79,5)$ & $[79,5-89,5)$ & $[89,5-99,5)$ \\
% 			\hline Số thí & $12$ & $10$ & $6$ & $3$ & $1$ \\
% 			\hline
% 		\end{tabular}    
% 	\end{center}
% 	Tìm  tứ phân vị thứ nhất của mẫu số liệu.
% 	\choice
% 	{\True $Q_1\approx 41,3$}
% 	{$Q_1\approx 51,67$}
% 	{$Q_1\approx 40,83$}
% 	{$Q_1\approx 51,17$}
% 	\loigiai{Cỡ mẫu là $n=60$.\\
% 		Tứ phân vị thứ nhất $Q_1$ là $\dfrac{x_{15}+x_{16}}{2}$. Do $x_{15}$, $x_{16}$ đều thuộc nhóm $[39,5-49,5)$ nên nhóm này chứa $Q_1$. \\Do đó, $p=5 ; \;a_5=39,5 ;\; m_5=15 ; \;m_1+\ldots+m_4=13, \;a_6-a_5=10$ và ta có
% 		$$
% 		Q_1=39,5+\dfrac{\frac{60}{4}-13}{15}\cdot 10\approx 40,83.
% 		$$
% 	}
% \end{ex}
%3
\begin{ex}%[1K3B9-3]
	Phỏng vấn một số học sinh khối 11 vể thời gian (giờ) ngủ của một buổi tối, thu được bảng số liệu như sau.
	\begin{center}
		\begin{tabular}{|l|c|c|c|c|c|}
			\hline Thời gian  (giờ)  &{$[4 ; 5)$}&{$[5 ; 6)$}&{$[6 ; 7)$}&{$[7 ; 8)$}&{$[8 ; 9)$}\\
			\hline Số học sinh & $6$ & $10$ & $13$ & $9$ & $7$ \\
			\hline
		\end{tabular}     
	\end{center}
	Hãy cho biết $75 \%$ học sinh khối 11 ngủ ít nhất bao nhiêu giờ?
	\choice
	{$7,675$}
	{\True $7,53$}
	{$8$}
	{ $7,9$}
	\loigiai{
		Cỡ mẫu là $n=45$.\\
		Gọi $x_1, \ldots, x_{45}$ là mẫu số liệu được sắp xếp theo thứ tự không giảm. Khi đó, trung vị là $x_{23}$. Do đó, tứ phân vị thứ ba $Q_3$ là $x_{34}$. Do $x_{34}$ đều thuộc nhóm $[7;8)$ nên nhóm này chứa $Q_3$. Do đó, $p=4 ; \;a_4=7 ;\; m_4=9 ; \;m_1+m_2+m_3=29 ; \;a_5-a_4=1$ và ta có
		$$
		Q_3=7+\dfrac{\frac{3 \cdot 45}{4}-29}{9}\cdot 1\approx7,53.
		$$ 
		Vậy $75\%$ học sinh khối 11 ngủ ít nhất $7,53$ giờ.
	}
\end{ex}
%4
% \begin{ex}%[1K3B9-3]
% 	Điểm thi môn Toán (thang điểm 100, điểm được làm tròn đến 1) của 60 thí sinh được cho trong bảng sau
% 	\begin{center}
% 		\begin{tabular}{|l|c|c|c|c|c|}
% 			\hline Điểm & $[0-9,5)$ & $[9,5-19,5)$ & $[19,5-29,5)$ & $[29,5-39,5)$ & $[39,5-49,5)$ \\
% 			\hline Số thí sinh & $1 $& $2$ & $4$ & $6$ & $15$ \\
% 			\hline Điểm & $[49,5-59,5)$ & $[59,5-69,5)$ & $[69,5-79,5)$ & $[79,5-89,5)$ & $[89,5-99,5)$ \\
% 			\hline Số thí & $12$ & $10$ & $6$ & $3$ & $1$ \\
% 			\hline
% 		\end{tabular}    
% 	\end{center}
% 	Tìm  tứ phân vị thứ ba của mẫu số liệu.
% 	\choice
% 	{$Q_3=41,3$}
% 	{$Q_3=51,67$}
% 	{$Q_3=45$}
% 	{\True $Q_3=65$}
% 	\loigiai{Cỡ mẫu là $n=60$.\\
% 		Với tứ phân vị thứ ba $Q_3$ là $\dfrac{x_{45}+x_{46}}{2}$. Do $x_{45}$, $x_{46}$ đều thuộc nhóm $[60 ; 70)$ nên nhóm này chứa $Q_3$. Do đó, $p=7 ; \;a_7=60 ;\; m_7=10 ; \;m_1+\ldots+m_6=40 ; \;a_8-a_7=10$ và ta có
% 		$$
% 		Q_3=59,5+\dfrac{\frac{3 \cdot 60}{4}-40}{10}\cdot 10=64,5.
% 		$$
% 	}
% \end{ex}
%5
\begin{ex}%[1K3B9-3]
	Một hãng xe ô tô thống kê lại số lần gặp sự cố về động cơ về động cơ của $100$ chiếc xe cùng loại sau 2 năm sử dụng đầu tiên ở dảng sau
	\begin{center}
		\begin{tabular}{|l|c|c|c|c|c|}
			\hline Số lần gặp sự cố  &{$[0,5;2,5)$}&{$[2,5;4,5)$}&{$[4,5;6,5)$}&{$[6,5 ; 8,5)$}&{$[8,5;10,5)$}\\
			\hline Số xe & $17$ & $33$ & $25$ & $20$ & $5$ \\
			\hline
		\end{tabular}     
	\end{center}
	Tìm tứ phân vị thứ nhất của mẫu số liệu.
	\choice
	{\True $Q_1\approx 4$}
	{$Q_1\approx 2,98$}
	{$Q_1\approx 2,5$}
	{$Q_1\approx 3,5$}
	\loigiai{
		Cỡ mẫu là $n=100$.\\
		Gọi $x_1, \ldots, x_{100}$ là mẫu số liệu được sắp xếp theo thứ tự không giảm. Khi đó, trung vị là $\dfrac{x_{50}+x_{51}}{2}$. 
		Do đó, tứ phân vị thứ nhất $Q_1$ là $\dfrac{x_{25}+x_{26}}{2}$. Do $x_{25}$, $x_{26}$ đều thuộc nhóm $[2,5;4,5)$ nên nhóm này chứa $Q_1$. \\Do đó, $p=2 ; \;a_2=2,5;\; m_2=33 ; \;m_1=17, \;a_3-a_2=2$ và ta có
		$$
		Q_1=2,5+\dfrac{\frac{100}{4}-17}{33}\cdot 2\approx 2,98.
		$$
	}    
\end{ex}
%6
% \begin{ex}%[1K3B9-3]
% 	Một hãng xe ô tô thống kê lại số lần gặp sự cố về động cơ về động cơ của $100$ chiếc xe cùng loại sau 2 năm sử dụng đầu tiên ở dảng sau
% 	\begin{center}
% 		\begin{tabular}{|l|c|c|c|c|c|}
% 			\hline Số lần gặp sự cố  &{$[0,5;2,5)$}&{$[2,5;4,5)$}&{$[4,5;6,5)$}&{$[6,5 ; 8,5)$}&{$[8,5;10,5)$}\\
% 			\hline Số xe & $17$ & $33$ & $25$ & $20$ & $5$ \\
% 			\hline
% 		\end{tabular}     
% 	\end{center}
% 	Tìm   tứ phân vị thứ hai của mẫu số liệu.
% 	\choice
% 	{\True $Q_2=4,5$}
% 	{$Q_2\approx 5,12$}
% 	{$Q_2\approx 4,89$}
% 	{$Q_2\approx 5,2$}
% 	\loigiai{
% 		Cỡ mẫu là $n=100$.\\
% 		Gọi $x_1, \ldots, x_{100}$ là mẫu số liệu được sắp xếp theo thứ tự không giảm. Khi đó, trung vị là $\dfrac{x_{50}+x_{51}}{2}$. Do $x_{50} \in [2,5;4,5)$, $x_{51} \in [4,5;6,5)$  nên tứ phân vị thứ hai của mẫu số liệu ghép nhóm là  $Q_2=4,5$. 
% 	}
% \end{ex}
%7
\begin{ex}%[1K3B9-3]
	Một hãng xe ô tô thống kê lại số lần gặp sự cố về động cơ về động cơ của $100$ chiếc xe cùng loại sau 2 năm sử dụng đầu tiên ở dảng sau
	\begin{center}
		\begin{tabular}{|l|c|c|c|c|c|}
			\hline Số lần gặp sự cố  &{$[0,5;2,5)$}&{$[2,5;4,5)$}&{$[4,5;6,5)$}&{$[6,5 ; 8,5)$}&{$[8,5;10,5)$}\\
			\hline Số xe & $17$ & $33$ & $25$ & $20$ & $5$ \\
			\hline
		\end{tabular}     
	\end{center}
	Tìm   tứ phân vị thứ ba của mẫu số liệu.  
	\choice
	{$Q_3=6,3$}
	{$Q_3=6,8$}
	{$Q_3=7,2$}
	{\True $Q_3=6,5$}
	\loigiai{ Cỡ mẫu là $n=100$.\\
		Với tứ phân vị thứ ba $Q_3$ là $\dfrac{x_{75}+x_{76}}{2}$. Do $x_{75} \in [4,5;6,5)$, $x_{76} \in [6,5 ; 8,5)$  nên tứ phân vị thứ ba của mẫu số liệu ghép nhóm là $Q_3=6,5$. 
		
	}
\end{ex}
%8
% \begin{ex}%[1K3B9-3]
% 	Lương tháng của một số nhân viên văn phòng được ghi lại như sau (đơn vị: triệu đồng)
% 	\begin{center}
% 		\begin{tabular}{|l|c|c|c|c|c|}
% 			\hline Lương tháng (triệu đồng)  &{$[6;8)$}&{$[8;10)$}&{$[10;12)$}&{$[12;14)$}\\
% 			\hline Số nhân viên & $3$ & $6$ & $8$ & $7$  \\
% 			\hline
% 		\end{tabular}     
% 	\end{center}  
% 	Tìm tứ phân vị thứ nhất của mẫu số liệu.
% 	\choice
% 	{\True $Q_1= 9$}
% 	{$Q_1= 8,5$}
% 	{$Q_1= 9,5$}
% 	{$Q_1= 8,2$}
% 	\loigiai{
% 		Cỡ mẫu là $n=24$.\\
% 		Gọi $x_1, \ldots, x_{24}$ là mẫu số liệu được sắp xếp theo thứ tự không giảm. Khi đó, trung vị là $\dfrac{x_{12}+x_{13}}{2}$. 
% 		Do đó, tứ phân vị thứ nhất $Q_1$ là $\dfrac{x_{6}+x_{7}}{2}$. Do $x_{6}$, $x_{7}$ đều thuộc nhóm $[8;10)$ nên nhóm này chứa $Q_1$. \\Do đó, $p=2 ; \;a_2=8;\; m_2=6 ; \;m_1=3, \;a_3-a_2=2$ và ta có
% 		$$
% 		Q_1=8+\dfrac{\frac{24}{4}-3}{6}\cdot 2=9.
% 		$$
% 	}    
% \end{ex}
% %9
% \begin{ex}%[1K3B9-3]
% 	Lương tháng của một số nhân viên văn phòng được ghi lại như sau (đơn vị: triệu đồng)
% 	\begin{center}
% 		\begin{tabular}{|l|c|c|c|c|c|}
% 			\hline Lương tháng (triệu đồng)  &{$[6;8)$}&{$[8;10)$}&{$[10;12)$}&{$[12;14)$}\\
% 			\hline Số nhân viên & $3$ & $6$ & $8$ & $7$  \\
% 			\hline
% 		\end{tabular}     
% 	\end{center}   
% 	Tìm   tứ phân vị thứ hai của mẫu số liệu.
% 	\choice
% 	{\True $Q_2=10,75$}
% 	{$Q_2= 10,5$}
% 	{$Q_2= 11$}
% 	{$Q_2=11,5$}
% 	\loigiai{
% 		Cỡ mẫu là $n=24$.\\
% 		Gọi $x_1, \ldots, x_{24}$ là mẫu số liệu được sắp xếp theo thứ tự không giảm. Khi đó, trung vị là $\dfrac{x_{12}+x_{13}}{2}$. 
% 		Do đó,  tứ phân vị thứ hai $Q_2$ là $\dfrac{x_{12}+x_{13}}{2}$. Do $x_{12}$, $x_{13}$ đều thuộc nhóm $[10;12)$ nên nhóm này chứa $Q_2$. \\Do đó, $p=3 ; \;a_3=10;\; m_3=8 ; \;m_1+m_2=9, \;a_4-a_3=2$ và ta có
% 		$$
% 		Q_2=10+\dfrac{\frac{24}{2}-9}{8}\cdot 2=10,75.
% 		$$
% 	}
% \end{ex}
% %10
% \begin{ex}%[1K3B9-3]
% 	Lương tháng của một số nhân viên văn phòng được ghi lại như sau (đơn vị: triệu đồng)
% 	\begin{center}
% 		\begin{tabular}{|l|c|c|c|c|c|}
% 			\hline Lương tháng (triệu đồng)  &{$[6;8)$}&{$[8;10)$}&{$[10;12)$}&{$[12;14)$}\\
% 			\hline Số nhân viên & $3$ & $6$ & $8$ & $7$  \\
% 			\hline
% 		\end{tabular}     
% 	\end{center}  
% 	Tìm   tứ phân vị thứ ba của mẫu số liệu.
% 	\choice 
% 	{$Q_3\approx 12,5$}
% 	{$Q_3\approx 13,2$}
% 	{$Q_3\approx 13,5$}
% 	{\True $Q_3\approx 12,3$}
% 	\loigiai{
% 		Cỡ mẫu là $n=24$.\\
% 		Gọi $x_1, \ldots, x_{24}$ là mẫu số liệu được sắp xếp theo thứ tự không giảm. Khi đó, trung vị là $\dfrac{x_{12}+x_{13}}{2}$. 
% 		Do đó,  tứ phân vị thứ ba $Q_3$ là $\dfrac{x_{18}+x_{19}}{2}$. Do $x_{18}$, $x_{19}$ đều thuộc nhóm $[12;14)$ nên nhóm này chứa $Q_3$. \\Do đó, $p=4 ; \;a_4=12;\; m_4=7 ; \;m_1+m_2+m_3=17, \;a_4-a_3=2$ và ta có
% 		$$
% 		Q_2=12+\dfrac{\frac{24\cdot 3}{4}-17}{7}\cdot 2\approx 12,3.
% 		$$
% 	}
% \end{ex}
%11
% \begin{ex}%[1K3B9-3]
% 	Số điểm một cầu thủ bóng rổ ghi được trong 20 trận đấu được cho ở bảng sau
% 	\begin{center}
% 		\begin{tabular}{|l|c|c|c|c|c|}
% 			\hline Điểm số  &{$[5,5;10,5)$}&{$[10,5;15,5)$}&{$[15,5;20,5)$}&{$[20,5;25,5)$}\\
% 			\hline Số trận & $3$ & $9$ & $2$ & $6$  \\
% 			\hline
% 		\end{tabular}     
% 	\end{center}  
% 	Tìm   tứ phân vị thứ ba của mẫu số liệu.
% 	\choice 
% 	{$Q_3\approx 23,5$}
% 	{$Q_3\approx 22,2$}
% 	{$Q_3\approx 21,6$}
% 	{\True $Q_3\approx 21,3$}
% 	\loigiai{
% 		Cỡ mẫu là $n=20$.\\
% 		Gọi $x_1, \ldots, x_{20}$ là mẫu số liệu được sắp xếp theo thứ tự không giảm. Khi đó, trung vị là $\dfrac{x_{10}+x_{11}}{2}$. 
% 		Do đó,  tứ phân vị thứ ba $Q_3$ là $\dfrac{x_{15}+x_{16}}{2}$. Do $x_{15}$, $x_{16}$ đều thuộc nhóm $[20,5;25,5)$ nên nhóm này chứa $Q_3$. \\Do đó, $p=4 ; \;a_4=20,5;\; m_4=6 ; \;m_1+m_2+m_3=14, \;a_5-a_4=25,5-20,5=5$ và ta có
% 		$$
% 		Q_2=20,5+\dfrac{\frac{20\cdot 3}{4}-14}{6}\cdot 5 \approx 21,3.
% 		$$
% 	}
% \end{ex}
% %12
% \begin{ex}%[1K3B9-3]
% 	Số điểm một cầu thủ bóng rổ ghi được trong 20 trận đấu được cho ở bảng sau
% 	\begin{center}
% 		\begin{tabular}{|l|c|c|c|c|c|}
% 			\hline Điểm số  &{$[5,5;10,5)$}&{$[10,5;15,5)$}&{$[15,5;20,5)$}&{$[20,5;25,5)$}\\
% 			\hline Số trận & $3$ & $9$ & $2$ & $6$  \\
% 			\hline
% 		\end{tabular}     
% 	\end{center} 
% 	Tìm tứ phân vị thứ nhất của mẫu số liệu.
% 	\choice
% 	{\True $Q_1\approx 11,6$}
% 	{$Q_1\approx 11,3$}
% 	{$Q_1\approx 21,6$}
% 	{$Q_1\approx 21,3$}
% 	\loigiai{
% 		Cỡ mẫu là $n=20$.\\
% 		Gọi $x_1, \ldots, x_{20}$ là mẫu số liệu được sắp xếp theo thứ tự không giảm. Khi đó, trung vị là $\dfrac{x_{10}+x_{11}}{2}$. 
% 		Do đó, tứ phân vị thứ nhất $Q_1$ là $\dfrac{x_{5}+x_{6}}{2}$. Do $x_{5}$, $x_{6}$ đều thuộc nhóm $[10,5;15,5)$ nên nhóm này chứa $Q_1$. \\Do đó, $p=2 ; \;a_2=10,5;\; m_2=9 ; \;m_1=3, \;a_3-a_2=5$ và ta có
% 		$$
% 		Q_1=10,5+\dfrac{\frac{20}{4}-3}{9}\cdot 5\approx 11,6.
% 		$$
% 	}
% \end{ex}
%13
\begin{ex}%[1K3B9-3]
	Số điểm một cầu thủ bóng rổ ghi được trong 20 trận đấu được cho ở bảng sau
	\begin{center}
		\begin{tabular}{|l|c|c|c|c|c|}
			\hline Điểm số  &{$[5,5;10,5)$}&{$[10,5;15,5)$}&{$[15,5;20,5)$}&{$[20,5;25,5)$}\\
			\hline Số trận & $3$ & $9$ & $2$ & $6$  \\
			\hline
		\end{tabular}     
	\end{center} 
	Tìm tứ phân vị thứ nhất của mẫu số liệu.
	\choice
	{\True $Q_1\approx 11,6$}
	{$Q_1\approx 14,4$}
	{$Q_1\approx 15,6$}
	{$Q_1\approx 21,3$}
	\loigiai{
		Cỡ mẫu là $n=20$.\\
		Gọi $x_1, \ldots, x_{20}$ là mẫu số liệu được sắp xếp theo thứ tự không giảm. Khi đó, trung vị là $\dfrac{x_{10}+x_{11}}{2}$. 
		Do đó, tứ phân vị thứ nhất $Q_2$ là $\dfrac{x_{10}+x_{11}}{2}$. Do $x_{10}$, $x_{11}$ đều thuộc nhóm $[10,5;15,5)$ nên nhóm này chứa $Q_2$. \\Do đó, $p=2 ; \;a_2=10,5;\; m_2=9 ; \;m_1=3, \;a_3-a_2=5$ và ta có
		$$
		Q_1=10,5+\dfrac{\frac{20}{2}-3}{9}\cdot 5\approx 14,4.
		$$   
	}
\end{ex}
%14
% \begin{ex}%[1K3B9-3]
% 	Một người thống kê lại thời gian thực hiện các cuộc gọi điện thoại của người đó trong một tuần cho trong bảng sau
% 	\begin{center}
% 		\begin{tabular}{|l|c|c|c|c|c|}
% 			\hline Số bệnh nhân &{$[0;60)$}&{$[60;120)$}&{$[120;180)$}&{$[180;240)$}&{$[240;300)$}\\
% 			\hline Số ngày & $8$ & $10$ & $7$ & $5$ & $2$ \\
% 			\hline
% 		\end{tabular}     
% 	\end{center}
% 	Tìm   tứ phân vị thứ ba của mẫu số liệu.  
% 	\choice
% 	{$Q_3\approx 175,28$}
% 	{$Q_3\approx 150,32 $}
% 	{$Q_3=175$}
% 	{\True $Q_3\approx 171,43$}
% 	\loigiai{ Cỡ mẫu là $n=32$.\\
% 		Gọi $x_1, \ldots, x_{32}$ là mẫu số liệu được sắp xếp theo thứ tự không giảm. Khi đó, trung vị là $\dfrac{x_{16}+x_{17}}{2}$.\\
% 		Do đó, tứ phân vị thứ ba $Q_3$ là $\dfrac{x_{24}+x_{25}}{2}$. Do $x_{24} $, $x_{25} \in [120;180)$   nên nhóm này chứa $Q_3$. \\Do đó, $p= 3; \;a_3=120 ;\; m_3=7 ; \;m_1+m_2=18 ; \;a_4-a_3=60$ và ta có
% 		$$
% 		Q_3=120+\dfrac{\frac{3 \cdot 32}{4}-18}{7}\cdot 60\approx 171,43.
% 		$$
% 	}
% \end{ex}
% %15
% \begin{ex}%[1K3B9-3]
% 	Một người thống kê lại thời gian thực hiện các cuộc gọi điện thoại của người đó trong một tuần cho trong bảng sau
% 	\begin{center}
% 		\begin{tabular}{|l|c|c|c|c|c|}
% 			\hline Số bệnh nhân &{$[0;60)$}&{$[60;120)$}&{$[120;180)$}&{$[180;240)$}&{$[240;300)$}\\
% 			\hline Số ngày & $8$ & $10$ & $7$ & $5$ & $2$ \\
% 			\hline
% 		\end{tabular}     
% 	\end{center}
% 	Tìm   tứ phân vị thứ hai của mẫu số liệu.  
% 	\choice
% 	{$Q_2\approx 80,25$}
% 	{$Q_2\approx 100,32$}
% 	{$Q_2=115$}
% 	{\True $Q_2=108$}
% 	\loigiai{ Cỡ mẫu là $n=32$.\\
% 		Gọi $x_1, \ldots, x_{32}$ là mẫu số liệu được sắp xếp theo thứ tự không giảm. Khi đó, trung vị là $\dfrac{x_{16}+x_{17}}{2}$.\\
% 		Do đó, tứ phân vị thứ hai $Q_2$ là $\dfrac{x_{16}+x_{17}}{2}$. Do $x_{16} $, $x_{17} \in [60;120)$   nên nhóm này chứa $Q_2$. \\Do đó, $p= 2; \;a_2=60 ;\; m_2=10 ; \;m_1=8 ; \;a_3-a_2=60$ và ta có
% 		$$
% 		Q_3=60+\dfrac{\frac{ 32}{2}-8}{10}\cdot 60=108.
% 		$$
% 	}    
% \end{ex}
% \begin{ex}
% 	Một công ty xây dựng khảo sát khách hàng xem họ có nhu cầu mua nhà ở mức giá nào. Kết quả khảo sát được ghi lại ở bảng sau
% 	\begin{center}
% 		\begin{tabular}{|c|c|c|c|c|c|}
% 			\hline \begin{tabular}{c}
% 				\textbf{Mức giá} \\
% 				\textbf{(triệu đồng/$\mathrm{m}^2)$}
% 			\end{tabular} &{$[10; 14)$} &{$[14; 18)$} &{$[18; 22)$} &{$[22; 26)$} &{$[26; 30)$} \\
% 			\hline \textbf{Số khách hàng} & 54 & 78 & 120 & 45 & 12 \\
% 			\hline
% 		\end{tabular}
% 	\end{center}
	
% 	 Công ty nên xây nhà ở mức giá nào để nhiều người có nhu cầu mua nhất?
% 	\choice
% 	{\True $19{,}4$ triệu đồng$/ \mathrm{m}^2$ }
% 	{$20{,}4$ triệu đồng$/ \mathrm{m}^2$ }
% 	{$19{,}6$ triệu đồng$/ \mathrm{m}^2$ }
% 	{$20{,}6$ triệu đồng$/ \mathrm{m}^2$ }	
% 	\loigiai{
% 		\ Nhóm chứa mốt của mẫu số liệu trên là nhóm $[18; 22)$.\\ Do đó $u_m=18$, $n_{m-1}=78$, $n_m=120$, $n_{m+1}=45$, $u_{m+1}-u_m=22-18=4$.\\
% 			Mốt của mẫu số liệu ghép nhóm là
% 			\[M_0=18+\dfrac{120-78}{(120-78)+(120-45)} \cdot 4=\dfrac{758}{39} \approx 19{,}4. \]
% 		 Dựa vào kết quả trên ta có thể dự đoán rằng nếu công ty xây nhà ở mức giá $19{,}4$ triệu đồng$/ \mathrm{m}^2$ thì sẽ có nhiều người có nhu cầu mua nhất.
		
% 	}
% \end{ex}
\begin{ex}%[1T5B1-3]
	Số cuộc gọi điện thoại một nguời thực hiện mỗi ngày trong $30$ ngày được lựa chọn ngẫu nhiên được thống kê trong bảng sau:
\begin{center}
	\begin{tabular}{|c|c|c|c|c|c|}
		\hline Số cuộc gọi &{$[3; 5]$} &{$[6; 8]$} &{$[9; 11]$} &{$[12; 14]$} &{$[15; 17]$} \\
		\hline Số ngày & 5 & 13 & 7 & 3 & 2 \\
		\hline
	\end{tabular}
\end{center}
 Tìm mốt của mẫu số liệu ghép nhóm trên.
 Hãy dự đoán xem khả năng người đó thực hiện bao nhiêu cuộc gọi mỗi ngày là cao nhất.
\choice
{$4$}
{$6$}
{$5$}
{\True $7$}	
\loigiai{
	Do số cuộc gọi là số nguyên nên ta hiệu chỉnh lại như sau:
	\begin{center}
		\begin{tabular}{|c|c|c|c|c|c|}
			\hline Số cuộc gọi &{$[2{,}5; 5{,}5)$} &{$[5{,}5; 8{,}5)$} &{$[8{,}5; 11{,}5)$} &{$[11{,}5; 14{,}5)$} &{$[14{,}5; 17{,}5)$} \\
			\hline Số ngày & 5 & 13 & 7 & 3 & 2 \\
			\hline
		\end{tabular}
	\end{center}	
		 Nhóm chứa mốt của mẫu số liệu trên là nhóm $[5{,}5; 8{,}5)$.\\
		Do đó $u_m=5{,}5$; $n_{m-1}=5$; $n_m=13$; $n_{m+1}=7$; $u_{m+1}-u_m=8{,}5-5{,}5=3$.\\
		Mốt của mẫu số liệu ghép nhóm là
		\[M_0=5{,}5+\dfrac{13-5}{(13-5)+(13-7)} \cdot 3=\dfrac{101}{14} \approx 7{,}2. \]
	 Dựa vào kết quả trên ta có thể dự đoán rằng khả năng người đó thực hiện $7$ cuộc gọi mỗi ngày là cao nhất.
	
}
\end{ex}
\begin{ex}
	Một thư viện thống kê số lượng sách được mượn mỗi ngày trong ba tháng ở bảng sau:
	\begin{center}
		\begin{tabular}{|c|c|c|c|c|c|c|c|}
			\hline Số sách &{$[16; 20]$} &{$[21; 25]$} &{$[26; 30]$} &{$[31; 35]$} &{$[36; 40]$} &{$[41; 45]$} &{$[46; 50]$} \\
			\hline Số ngày & 3 & 6 & 15 & 27 & 22 & 14 & 5 \\
			\hline
		\end{tabular}
	\end{center}
	Hãy ước lượng  mốt của mẫu số liệu ghép nhóm trên.
	\choice
	{$34{,}33$}
	{\True $34{,}03$}
	{$35{,}63$}
	{$34{,}13$}	
	\loigiai{
		Vì số lượng sách được mượn là số nguyên nên ta hiệu chỉnh bảng tần số ghép nhóm (theo giá trị đại diện) như sau
		\begin{center}
			{\footnotesize \begin{tabular}{|c|c|c|c|c|c|c|c|}
					\hline Số sách &{$[15{,}5; 20{,}5)$} &{$[20{,}5; 25{,}5)$} &{$[25{,}5; 30{,}5)$} &{$[30{,}5; 35{,}5)$} &{$[35{,}5; 40{,}5]$} &{$[40{,}5; 45{,}5)$} &{$[45{,}5; 50{,}5)$} \\
					\hline Giá trị đại diện &{$18$} &{$23$} &{$28$} &{$33$} &{$38$} &{$43$} &{$48$} \\
					\hline Số ngày & 3 & 6 & 15 & 27 & 22 & 14 & 5 \\
					\hline
			\end{tabular}}
		\end{center}
		Trung bình số lượng sách được mượn mỗi ngày trong 3 tháng của thư viện là
		\[\overline{x}=\dfrac{18\cdot 3+23\cdot 6+28\cdot 15+33\cdot 27+38\cdot 22+43\cdot 14+48\cdot 5}{92}\approx 34{,}58. \]
		Nhóm chứa mốt của mẫu số liệu trên là nhóm $[30{,}5; 35{,}5)$.\\
		Do đó $u_m=30{,}5$; $n_{m-1}=15$; $n_m=27$; $n_{m+1}=22$; $u_{m+1}-u_m=35{,}5-30{,}5=5$.\\
		Mốt của mẫu số liệu ghép nhóm là
		\[M_0=30{,}5+\dfrac{27-15}{(27-15)+(27-22)} \cdot 5\approx 34{,}03. \]
	}
\end{ex}
\begin{ex}
	Kết quả đo chiều cao của $200$ cây keo $3$ năm tuổi ở một nông trường được biểu diễn ở biểu đồ dưới đây.
	\begin{center}
		\begin{tikzpicture}[scale=1,font=\scriptsize]
		\def\hoanh{11.5};
		\def\tung{6.5};
		\def\mau{cyan};
		\foreach \x/\n in{1/20,3/35,5/60,7/55,9/30}{\draw[line width=16mm,\mau] (\x,0)--++(0,{\n/10});
			\draw[dashed] (\x,{\n/10})node[above]{$\n$}--(0,{\n/10}) node[left]{$\n$};}
		\foreach \x/\p in {1/[8{,}5;8{,}8),3/[8{,}8;9{,}1),5/[9{,}1;9{,}4),7/[9{,}4;9{,}7),9/[9{,}7;10{,}0)}{\node[below] at (\x,0){\scriptsize $\p$};}
		\draw[->] (0,0)--(\hoanh,0) node[below]{($m$)};
		\draw[->] (0,0)node[below left]{$O$}--(0,\tung) node[left]{(Số cây)};
		\path (current bounding box.north) node[above]		{\textbf{Chiều cao 200 cây keo 3 năm tuổi}};
		\end{tikzpicture}
	\end{center}
	Mốt của mẫu số liệu ghép nhóm trên là
	\choice
	{$9{,}35$}
	{$10{,}53$}
	{$10{,}35$}
	{$9{,}53$}	
	\loigiai{
		Bảng tần số ghép nhóm (theo giá trị đại diện)
		\begin{center}
			\begin{tabular}{|c|c|c|c|c|c|}
				\hline Chiều cao &$[8{,}5; 8{,}8)$ &{$[8{,}8; 9{,}1)$} &{$[9{,}1; 9{,}4)$} &{$[9{,}4; 9{,}7)$} &{$[9{,}7; 10{,}0)$} \\
				\hline Giá trị đại diện &$8{,}65$ &$8{,}95$ &$9{,}25$ &$9{,}55$ &$9{,}85$ \\
				\hline Số cây & $20$ & $35$ & $60$ & $55$ & $30$\\
				\hline
			\end{tabular}
		\end{center}
		Chiều cao trung bình của $200$ cây keo 3 năm tuổi là
		\[\overline{x}=\dfrac{8{,}65\cdot 20+8{,}95\cdot 35+9{,}25\cdot 60+9{,}55\cdot 55+9{,}85\cdot 30}{200}\approx 9{,}31. \]
		Nhóm chứa mốt của mẫu số liệu trên là nhóm $[9{,}1; 9{,}4)$.\\
		Do đó $u_m=9{,}1$; $n_{m-1}=35$; $n_m=60$; $n_{m+1}=55$; $u_{m+1}-u_m=9{,}4-9{,}1=0{,}3$.\\
		Mốt của mẫu số liệu ghép nhóm là
		\[M_0=9{,}1+\dfrac{60-35}{(60-35)+(60-55)} \cdot 0{,}3= 9{,}35. \]
	}
\end{ex}
\begin{ex}%[1K3B9-4]
	Bảng số liệu ghép nhóm sau cho biết chiều cao (cm) của $50$ học sinh lớp $11A$.
	\begin{center}
		\begin{tabular}{|c|c|c|c|c|c|c|}
			\hline
			Khoảng chiều cao (cm)	& $\left[145;150 \right)$ & $\left[150;155 \right)$ & $\left[155;160 \right)$ & $\left[160;165 \right)$&$\left[165;170 \right)$  \\
			\hline
			Số học sinh&$7$	& $14$ & $10$ &$10$  & $9$ \\
			\hline
		\end{tabular}
	\end{center}
Mốt của mẫu số liệu ghép nhóm là
	\choice
	{$154{,}20$}
	{\True $153{,}18$}
	{$155{,}12$}
	{$158{,}36$}	
	\loigiai{
		Tần số lớn nhất là $14$ nên nhóm chứa mốt là nhóm $\left[150;155 \right)$. Ta có $j=2$, $a_2=150$, $m_2=14$, $m_1=7$, $m_3=10$, $h=5$. Do đó $$M_o=150+\dfrac{14-7}{\left(14-7\right)+\left(14-10\right)\cdot 5}\approx 153{,}18.$$	
		Số học sinh có chiều cao khoảng $153{,}18$ là nhiều nhất.
	}
\end{ex}
\begin{ex}%[1K3Y9-4]
	Chọn khẳng định \textbf{sai}.
	\choice
	{ Mốt của mẫu số liệu không ghép nhóm là giá trị có khả năng xuất hiện cao nhất khi lấy mẫu}
	{Mốt của mẫu số liệu sau khi ghép nhóm xấp xỉ với mốt của mẫu số liệu không ghép nhóm}
	{\True Một mẫu số liệu ghép nhóm chỉ có một mốt}
	{Một mẫu số liệu ghép nhóm có thể có nhiều nhóm chứa mốt và nhiều mốt}
	\loigiai{
		Mốt của mẫu số liệu không ghép nhóm là giá trị có khả năng xuất hiện cao nhất khi lấy mẫu.\\ Mốt của mẫu số liệu sau khi ghép nhóm xấp xỉ với mốt của mẫu số liệu không ghép nhóm. \\
		Một mẫu số liệu ghép nhóm có thể có nhiều nhóm chứa mốt và nhiều mốt.\\
		Do đó khẳng định sai là: Một mẫu số liệu ghép nhóm chỉ có một mốt.
	}    
\end{ex}
% \begin{ex}%[1K3Y9-4]
% 	Người ta ghi lại tuổi thọ của một số con ong cho kết quả như sau:
% 	\begin{center}
% 		\begin{tabular}{|l|c|c|c|c|c|c|}
% 			\hline Tuồi thọ (ngày) &{$[0;20)$}&{$[20;40)$}&{$[40;60)$}&{$[60;80)$}&{$[80;100)$}\\
% 			\hline Số lượng & $5$ & $12$ & $23$ & $31$ & $29$  \\
% 			\hline
% 		\end{tabular}
% 	\end{center}
% 	Nhóm chứa mốt của mẫu số liệu này là
% 	\choice
% 	{ $[20;40)$}
% 	{$[40;60)$}
% 	{\True $[60;80)$}
% 	{$[80;100)$}
% 	\loigiai{
% 		Nhóm chứa mốt của mẫu số liệu này là $[60;80)$.
% 	}    
% \end{ex}
% %6
% \begin{ex}%[1K3B9-4]
% 	Người ta ghi lại tuổi thọ của một số con ong cho kết quả như sau:
% 	\begin{center}
% 		\begin{tabular}{|l|c|c|c|c|c|c|}
% 			\hline Tuồi thọ (ngày) &{$[0;20)$}&{$[20;40)$}&{$[40;60)$}&{$[60;80)$}&{$[80;100)$}\\
% 			\hline Số lượng & $5$ & $12$ & $23$ & $31$ & $29$  \\
% 			\hline
% 		\end{tabular}
% 	\end{center}
% 	Mốt của mẫu số liệu này là
% 	\choice
% 	{ $M_0=70$}
% 	{$M_0=60$}
% 	{\True $M_0=76$}
% 	{$M_0=31$}
% 	\loigiai{
% 		Tần số lớn nhất là $31$ nên nhóm chứa mốt là nhóm $[60;80)$. \\
% 		Ta có, $j=4, a_4=60, m_4=31$, $m_3=23, m_5=29, h=20$. Do đó
% 		$$
% 		M_0=60+\frac{31-23}{(31-29)+(14-11)}\cdot 20 =76.
% 		$$
% 	}   
% \end{ex}
% %7
% \begin{ex}%[1K3Y9-4]
% 	Doanh thu bán hàng 20 ngày được lựa chọn ngẫu nhiên của một cửa hàng được ghi lại ở bảng sau (đơn vị: triệu đồng)
% 	\begin{center}
% 		\begin{tabular}{|l|c|c|c|c|c|c|}
% 			\hline Doanh thu &{$[5;7)$}&{$[7;9)$}&{$[9;11)$}&{$[11;13)$}&{$[13;15)$}\\
% 			\hline Số ngày & $2$ & $9$ & $7$ & $3$ & $1$ \\
% 			\hline
% 		\end{tabular}
% 	\end{center}
% 	Nhóm chứa mốt của mẫu số liệu này là
% 	\choice
% 	{ $[9;11)$}
% 	{$[5;7)$}
% 	{\True $[7;9)$}
% 	{$[11;13)$}
% 	\loigiai{
% 		Tần số lớn nhất là $9$ nên nhóm chứa mốt là nhóm $[7;9)$. \\
% 	}   
% \end{ex}
% %8
% \begin{ex}%[1K3B9-4]
% 	Doanh thu bán hàng 20 ngày được lựa chọn ngẫu nhiên của một cửa hàng được ghi lại ở bảng sau (đơn vị: triệu đồng)
% 	\begin{center}
% 		\begin{tabular}{|l|c|c|c|c|c|c|}
% 			\hline Doanh thu &{$[5;7)$}&{$[7;9)$}&{$[9;11)$}&{$[11;13)$}&{$[13;15)$}\\
% 			\hline Số ngày & $2$ & $9$ & $7$ & $3$ & $1$ \\
% 			\hline
% 		\end{tabular}
% 	\end{center}
% 	Xác định mốt của mẫu số liệu.
% 	\choice
% 	{ $M_0\approx 8$}
% 	{$M_0\approx 8,5$}
% 	{\True $M_0\approx 8,56$}
% 	{$M_0\approx 9$}
% 	\loigiai{
% 		Tần số lớn nhất là $9$ nên nhóm chứa mốt là nhóm $[7;9)$. \\
% 		Ta có, $j=2, a_2=7, m_2=9$, $m_1=2, m_3=7, h=2$. Do đó
% 		$$
% 		M_0=7+\frac{9-2}{(9-2)+(9-7)}\cdot 2\approx 8,56.
% 		$$
% 	}    
% \end{ex}
% %9
% \begin{ex}%[1K3B9-4]
% 	Điểm kiểm tra môn Toán của lớp 12A được cho trong bảng sau
% 	\begin{center}
% 		\begin{tabular}{|l|c|c|c|c|c|c|c|c|}
% 			\hline Khoảng điểm &{$[6,5;7)$}&{$[7;7,5)$}&{$[7,5;8)$}&{$[8;8,5)$}&{$[8,5;9)$}&{$[9;9,5)$}&{$[9,5;10)$}\\
% 			\hline Tần số & $8$ & $10$ & $16$ & $24$& $13$ & $7$ & $4$ \\
% 			\hline
% 		\end{tabular}
% 	\end{center}
% 	Xác định mốt của mẫu số liệu ghép nhóm này.    
% 	\choice
% 	{ $M_0\approx 8,4$}
% 	{$M_0\approx 8,5$}
% 	{\True $M_0\approx 8,21$}
% 	{$M_0\approx 24$}
% 	\loigiai{
% 		Tần số lớn nhất là $24$ nên nhóm chứa mốt là nhóm $[8;8,5)$. \\
% 		Ta có, $j=4, a_4=8, m_4=24$, $m_3=16, m_5=13, h=0,5$. Do đó
% 		$$
% 		M_0=8+\frac{24-16}{(24-16)+(24-13)}\cdot 0,5\approx 8,21.
% 		$$
% 	}
% \end{ex}
% %10
% \begin{ex}%[1K3Y9-4]
% 	Điểm kiểm tra môn Toán của lớp 12A được cho trong bảng sau
% 	\begin{center}
% 		\begin{tabular}{|l|c|c|c|c|c|c|c|c|}
% 			\hline Khoảng điểm &{$[6,5;7)$}&{$[7;7,5)$}&{$[7,5;8)$}&{$[8;8,5)$}&{$[8,5;9)$}&{$[9;9,5)$}&{$[9,5;10)$}\\
% 			\hline Tần số & $8$ & $10$ & $16$ & $24$& $13$ & $7$ & $4$ \\
% 			\hline
% 		\end{tabular}
% 	\end{center}
% 	Nhóm chứa mốt của mẫu số liệu này là
% 	\choice
% 	{ $[7;7,5)$}
% 	{$[7,5;8)$}
% 	{\True $[8;8,5)$}
% 	{$[8,5;9)$}
% 	\loigiai{
% 		Tần số lớn nhất là $24$ nên nhóm chứa mốt là nhóm $[8,5;9)$. \\
% 	}       
% \end{ex}
% %11
% \begin{ex}%[1K3Y9-4]
% 	Để kiểm tra thời gian sử dụng pin của   chiếc điện thoại mới, bạn A thống kê thời gian sử dụng điện thoại của mình từ lúc sạc đầy cho đến khi hết pin ở bảng sau
% 	\begin{center}
% 		\begin{tabular}{|l|c|c|c|c|c|c|c|c|}
% 			\hline Thời gian sử dụng (giờ) &{$[7;9)$}&{$[9;11)$}&{$[11;13)$}&{$[13;15)$}&{$[15;17)$}\\
% 			\hline Số lần & $2$ & $5$ & $7$ & $6$& $3$  \\
% 			\hline
% 		\end{tabular}
% 	\end{center}
% 	Nhóm chứa mốt của mẫu số liệu này là
% 	\choice
% 	{ $[9;11)$}
% 	{\True $[11;13)$}
% 	{ $[13;15)$}
% 	{$[15;17)$}
% 	\loigiai{
% 		Tần số lớn nhất là $7$ nên nhóm chứa mốt là nhóm $[11;13)$. \\
% 	}        
% \end{ex}
% %12
% \begin{ex}%[1K3B9-4]
% 	Để kiểm tra thời gian sử dụng pin của   chiếc điện thoại mới, bạn A thống kê thời gian sử dụng điện thoại của mình từ lúc sạc đầy cho đến khi hết pin ở bảng sau
% 	\begin{center}
% 		\begin{tabular}{|l|c|c|c|c|c|c|c|c|}
% 			\hline Thời gian sử dụng (giờ) &{$[7;9)$}&{$[9;11)$}&{$[11;13)$}&{$[13;15)$}&{$[15;17)$}\\
% 			\hline Số lần & $2$ & $5$ & $7$ & $6$& $3$  \\
% 			\hline
% 		\end{tabular}
% 	\end{center}   
% 	Xác định mốt của mẫu số liệu ghép nhóm này.    
% 	\choice
% 	{ $M_0\approx 11,67$}
% 	{$M_0\approx 12$}
% 	{\True $M_0\approx 12,33$}
% 	{$M_0\approx 7$}
% 	\loigiai{
% 		Tần số lớn nhất là $7$ nên nhóm chứa mốt là nhóm $[11;13)$. \\
% 		Ta có, $j=3, a_3=11, m_3=7$, $m_2=5, m_4=6, h=2$. Do đó
% 		$$
% 		M_0=11+\frac{7-5}{(7-5)+(7-6)}\cdot 2\approx 12,33.
% 		$$
% 	}
% \end{ex}
% %13
% \begin{ex}%[1K3Y9-4]
% 	Tổng lượng mưa trong tháng 8 đo được tại một trạm quan trắc đặt tại Vũng Tàu từ năm 2002 đến năm 2020 được ghi lại như sau (đơn vị: mm)
% 	\begin{center}
% 		\begin{tabular}{|l|c|c|c|c|c|c|c|c|}
% 			\hline Tổng lượng mưa trong tháng 8 (mm) &{$[120;175)$}&{$[175;230)$}&{$[230;285)$}&{$[285;340)$}\\
% 			\hline Số năm & $10$ & $5$ & $3$ & $1$  \\
% 			\hline
% 		\end{tabular}
% 	\end{center}  
% 	Nhóm chứa mốt của mẫu số liệu này là
% 	\choice
% 	{ $[175;230)$}
% 	{ $[230;285)$}
% 	{\True $[120;175)$}
% 	{$[285;340)$}
% 	\loigiai{
% 		Tần số lớn nhất là $10$ nên nhóm chứa mốt là nhóm $[120;175)$. \\
% 	}        
% \end{ex}
% %14
% \begin{ex}%[1K3B9-4]
% 	Tổng lượng mưa trong tháng 8 đo được tại một trạm quan trắc đặt tại Vũng Tàu từ năm 2002 đến năm 2020 được ghi lại như sau (đơn vị: mm)
% 	\begin{center}
% 		\begin{tabular}{|l|c|c|c|c|c|c|c|c|}
% 			\hline Tổng lượng mưa trong tháng 8 (mm) &{$[120;175)$}&{$[175;230)$}&{$[230;285)$}&{$[285;340)$}\\
% 			\hline Số năm & $10$ & $5$ & $3$ & $1$  \\
% 			\hline
% 		\end{tabular}
% 	\end{center}  
% 	Xác định mốt của mẫu số liệu ghép nhóm này.    
% 	\choice
% 	{ $M_0\approx 172,25$}
% 	{$M_0\approx 146,125$}
% 	{\True $M_0\approx 156,67$}
% 	{$M_0\approx 10$}
% 	\loigiai{
% 		Tần số lớn nhất là $10$ nên nhóm chứa mốt là nhóm $[120;175)$. \\
% 		Ta có, $j=1, a_1=120, m_1=10$, $m_2=5, m_0=0, h=55$. Do đó
% 		$$
% 		M_0=120+\frac{10-0}{(10-0)+(10-5)}\cdot 55\approx 156.67.
% 		$$
% 	}
% \end{ex}
% %15
% \begin{ex}%[1K3B9-4]
% 	Một công ty xây dựng khảo sát khách hàng xem họ có nhu cầu mua nhà ở mức giá nào. Kết quả khảo sát được ghi lại ở bảng sau (đơn vị: triệu đồng/$\mathrm{m}^2$
% 	\begin{center}
% 		\begin{tabular}{|l|c|c|c|c|c|c|c|c|}
% 			\hline Mức giá &{$[10;14)$}&{$[14;18)$}&{$[18;22)$}&{$[22;26)$}&{$[26;30)$}\\
% 			\hline Số khách hàng & $54$ & $78$ & $120$ & $45$& $12$  \\
% 			\hline
% 		\end{tabular}
% 	\end{center}   
% 	Xác định mốt của mẫu số liệu ghép nhóm này.    
% 	\choice
% 	{ $M_0\approx 18$}
% 	{\True $M_0\approx 19,4$}
% 	{ $M_0\approx 20$}
% 	{$M_0\approx 120$}
% 	\loigiai{
% 		Tần số lớn nhất là $120$ nên nhóm chứa mốt là nhóm $[18;22)$. \\
% 		Ta có, $j=3, a_3=18, m_3=120$, $m_2=78, m_4=45, h=4$. Do đó
% 		$$
% 		M_0=18+\frac{120-78}{(120-78)+(120-45)}\cdot 4\approx 19,4.
% 		$$
% 	}
% \end{ex}
% \begin{ex}%[1K3K9-4]
% 	Một công ty xây dựng khảo sát khách hàng xem họ có nhu cầu mua nhà ở mức giá nào. Kết quả khảo sát được ghi lại ở bảng sau (đơn vị: triệu đồng/$\mathrm{m}^2$
% 	\begin{center}
% 		\begin{tabular}{|l|c|c|c|c|c|c|c|c|}
% 			\hline Mức giá &{$[10;14)$}&{$[14;18)$}&{$[18;22)$}&{$[22;26)$}&{$[26;30)$}\\
% 			\hline Số khách hàng & $54$ & $78$ & $120$ & $45$& $12$  \\
% 			\hline
% 		\end{tabular}
% 	\end{center}   
% 	Công ty nên xây nhà ở mức giá nào để nhiều người có nhu cầu mua nhất?    
% 	\choice
% 	{ $ 18$ triệu đồng/$\mathrm{m}^2$}
% 	{\True $19,4$ triệu đồng/$\mathrm{m}^2$}
% 	{ $20$ triệu đồng/$\mathrm{m}^2$}
% 	{$21$ triệu đồng/$\mathrm{m}^2$}
% 	\loigiai{
% 		Tần số lớn nhất là $120$ nên nhóm chứa mốt là nhóm $[18;22)$. \\
% 		Ta có, $j=3, a_3=18, m_3=120$, $m_2=78, m_4=45, h=4$. Do đó
% 		$$
% 		M_0=18+\frac{120-78}{(120-78)+(120-45)}\cdot 4\approx 19,4.
% 		$$
% 		Dựa vào kết quả trên ta dự đoán rằng nếu công ty xây nhà ở mức giá $19,4$ triệu đồng/$\mathrm{m}^2$ thì sẽ có nhiều người có nhu cầu mua nhất.
% 	}
% \end{ex}
% \begin{ex}%[1K3Y9-4]
% 	Số cuộc gọi điện thoại một người thực hiện mỗi ngày trong 30 ngày được lựa chọn ngẫu nhiên được thống kê trong bảng sau
% 	\begin{center}
% 		\begin{tabular}{|l|c|c|c|c|c|c|c|c|}
% 			\hline Số cuộc gọi &{$[2,5;5,5)$}&{$[5,5;8,5)$}&{$[8,5;11,5)$}&{$[11,5;14,5)$}&{$[14,5;17,5)$}\\
% 			\hline Số ngày & $5$ & $13$ & $7$ & $3$& $2$  \\
% 			\hline
% 		\end{tabular}
% 	\end{center}   
% 	Nhóm chứa mốt của mẫu số liệu này là
% 	\choice
% 	{ $[2,5;5,5)$}
% 	{\True $[5,5;8,5)$}
% 	{ $[8,5;11,5)$}
% 	{$[11,5;14,5)$}
% 	\loigiai{
% 		Tần số lớn nhất là $13$ nên nhóm chứa mốt là nhóm $[5,5;8,5)$. \\
% 	}
% \end{ex}
% \begin{ex}%[1K3K9-4]
% 	Số cuộc gọi điện thoại một người thực hiện mỗi ngày trong 30 ngày được lựa chọn ngẫu nhiên được thống kê trong bảng sau
% 	\begin{center}
% 		\begin{tabular}{|l|c|c|c|c|c|c|c|c|}
% 			\hline Số cuộc gọi &{$[2,5;5,5)$}&{$[5,5;8,5)$}&{$[8,5;11,5)$}&{$[11,5;14,5)$}&{$[14,5;17,5)$}\\
% 			\hline Số ngày & $5$ & $13$ & $7$ & $3$& $2$  \\
% 			\hline
% 		\end{tabular}
% 	\end{center}   
% 	Hãy dự đoán xem khả năng người đó thực hiện bao nhiêu cuộc gọi mỗi ngày là cao nhất?   
% 	\choice
% 	{ $5$}
% 	{\True $7$}
% 	{ $6$}
% 	{$8$}
% 	\loigiai{
% 		Tần số lớn nhất là $13$ nên nhóm chứa mốt là nhóm $[5,5;8,5)$. \\
% 		Ta có, $j=2, a_2=5,5, m_2=13$, $m_1=5, m_3=7, h=3$. Do đó
% 		$$
% 		M_0=5,5+\frac{13-5}{(13-5)+(13-7)}\cdot 3\approx 7,2.
% 		$$
% 		Do đó ta có thể dự đoán khả năng người đó thực hiện $7$ cuộc gọi mỗi ngày là cao nhất.
% 	}
% \end{ex}

\Closesolutionfile{ans}
% ---------Mục lục chính
\FULLWIDTH
\tableofcontents %lệnh in mục lục chính
\begin{center}
\includegraphics[width=5cm]{QRcode/11D2.png}
\end{center}
\end{document}